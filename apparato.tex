L'apparato strumentale consiste in diverse componenti che, assieme, permettono di effettuare la misura che ci si è prefissati. 
\subsection{Rivelatori}
I rivelatori utilizzati sono 6 scintillatori plastici modello EJ-200 (della \textit{Eljen Technology}), dalle misure che vengono garantite di fabbrica come
600x250x10 mm con una precisione fornita come $\pm 0.5$ mm. Su tali scintillatori sono stati scavati dei canali sui quali è stata  posta una fibra ottica il grado di raccogliere
i fotoni di scintillazione generati dal passaggio di una carica. Per attaccare la fibra ottica è stato utilizzato del cemento ottico EJ-500; la fibra ottica è stata
incollata in modo che la distanza che un fotone percorre prima di entrare nella fibra ottica fosse minima considerando che tale fotone può essere gernerato in un punto qualsiasi
dello scintillatore. Poi gli estremi della fibra ottica, uscenti dallo scintillatore, sono stati levigati usando della carta vetro di diverso spessore.\\

Successivamente si è passati al \textit{wrapping} degli scintillatori: affinché siano utlizzabili gli scintillatori devono essere  avvolti in un materiale riflettente
(così che non si perdano fotoni che escono dagli scintillatori) e poi di un maateriale assorbente (così che non entrino fotoni dentro lo scintillatore). Per fare
ciò si sono usati tre layer differenti di materiali che hanno avvolto ogni singolo scintillatore:
\begin{itemize}
\item Foglio di alluminio: come prima cosa si è avvolto lo scintillatore in alluminio, stando attenti che tale alluminio formasse meno pieghe possibili: infatti eventuali pieghe
posssono diminuire il coefficiente di riflessione dell'alluminio e portare a rottura del layer stesso, provocando perdita di fotoni. Per poter posizionare al meglio
questo layer si è fatta molta attenzione nel tagliare il foglio della misura corretta e nel piegarlo nel miglior  modo attorno allo scintillatore stesso. Particolari
accorgimenti sono stati necessari per gli spigoli, dove si è fatto un doppio layer di alluminio che permetttesse di chiudere nel miglior modo possibile lo
scintillatore. Nell'Immagine \ref{img:wrap_al} si può vedere una foto fatta durante la fase di wrapping con alluminio di uno scintillatore, dove si può anche vedere la fibra
ottica. Il wrapping con la carta alluminio è stato fatto lasciando aperta una finestra della dimensione del circuito di lettura in prossimità del punto in cui la fibra
ottica esce dallo scintillatore
%VA INSERITA L'IMMAGINE DAL NOME wrapp_al.jpg
\item Cartone nero sugli spigoli: per bloccare la carta alluminio attorno allo scintillatore e impedire alla luce esterna di entrare da tali spigoli si è tagliato del cartone
nero spesso in modo che potesse ricoprire le superfici laterali dello scintillatore e parte delle superfici di base. Tale cartone è stato tagliato in modo che si incastrasse
nel miglior modo possibile a chiudere gli spigoli dello scintillatore, poi tale cartoncino è stato piegato utilizzando una punta in ferro (in modo che venisse piegato
e non tagliato) ed è stato fissato alla carta alluminio con del nastro adesivo. Un'immagine dello scintillatore dopo questa fase di sistemazione dei bordi si può vedere
nell'Immagine \ref{img:wrapped_scint}.
%VA INSERITA L'IMMAGINE DAL NOME wrapped_scint.jpg
\item Plastica nera assorbente: Come ultimo layer si sono ritagliati due rettangoli in plastica nera che potessero assorbire i fotoni e sono stati posizionati a coprire
le due superfici di base degli scintillatori. La plastica è stata poi fissata al resto del wrapping utilizzando del nastro isolante nero, in modo da coprire eventuali buchi
nella copertura esterna assorbente dello scintillatore. Nell?immagine \ref{img:end_wrapping} (risalente all'anno scorso, il procedimento di wrapping è statto fatto
allo stesso modo) si può vedere lo scintillatore una volta finito il wrapping.
%VA INSERITA L'IMMAGINE DAL NOME end_wrapping.jpg
\end{itemize}
Il wraapping è stato comunque eseguito nel modo più omogeneo possibile, nel senso che si è provato a creare il meno possibile dei bozzi nei rivelatori che rompono
la simmettria del sistema di acquisizione quando un rivelatore viene poggiato sopra un altro.

\subsection{Elettronica di acquisizione}
Per poter effettivamente acquisire i dati è necessaria dell'elettronica di acquisizone che legga i fotoni in uscita dalle fibre ottiche. Per fare ciò, si è utilizzata per
ogni rivelatore una scheda stampata consistente in una coppia di fotodiodi (uno per ogni uscita della fibra otttica) e un amplificazione che permettesse di
leggere il valore di tensione associato alla quantità di fotoni che hanno raggiunto il fotodiodo. Uno schema dell'elettronica di tale circuito si può vedere nell'Immagine
\ref{img:schema_pcb}, dove per il secondo diodo lo schema è uguale con l'unica differenza che si può cambiare il voltaggio di bias. Questo vuol dire che al variare del
voltaggio di bias cambia il guadagno del circuito e, per esempio, si può andare a discriminare su quanta variazione di tensione è associata alla rilevazione di un singolo
fotone.\\
%VA INSERITA L'IMMAGINE DAL NOME schema_pcb.jpg

Oltre a questa elettronica si sono utilizzati due generatori di tensione (uno fissato a $\pm 5$ V per l'alimentazione degli operazionali e uno collegato ad una piccola
cassettina di derivazione con dei resistori per poter cambaire i voltaggi di bias di ogni singolo fotodiodo), un oscilloscopio Pico modello 5000 con doppio canale di
acquisizione e possibilità di trigger esterno, un generatore di sequenze di trigger programmabile a 16 ingressi per lo studio delle coincidenze e andando avanti
con l'esperimento si utilizzerà anche una scheda per l'introduzzione delle differenze temporali tra i vari segnali.

\subsection{Assorbitore}
L'assorbitore non è ancora stato inserito per effettuare misure, ma sarà costituito da una lastra di rame dalle dimensioni 600x250x25 mm.

\subsection{Solenoide}
Per generare il campo magnetico si avvolgeranno con doppio avvolgimento 20 kg di filo di rame smaltato attorno ad un supporto in acciaio dalle dimensioni circa
di 1000x550x117 mm.