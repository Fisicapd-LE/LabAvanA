Per poter lavorare al meglio con l'apparato strumentale dato e comprendere al meglio il meccanismo di funzionamento si è deciso di scrivere un programma in grado di simulare
il processo che avviene durante le vere e proprie sessioni di misura. In questa sezione si vuole descrivere brevemente il funzionamento di tale programma di simulazione e
l'output che tale programma è in grado di fornire.
\subsection{Generazione dei raggi cosmici}
Come prima cosa è necessario il programma simuli dei raggi cosmici che siano realistici. Per fare questo si è utilizzata la nota distribuzione dei raggi cosmici:
% \begin{equation}
% \ref{eq:distr_cosmici}
% 
% !!!!!!!! inserire distribuzione raggi cosmici !!!!!!!!!!!!
% \end{equation}
per ogni evento un raggio cosmico viene generato in un punto casuale del piano lungo 700 mm lungo $x$ e 350 mm lungo $y$ in un certo punto di $z$ e la direzione di tale
raggio cosmico è data dall'Equazione \ref{eq:distr_cosmici}: in questo modo si ha un raggio cosmico che probabilmente è \textit{interessante} per l'apparato strumentale.
Tali raggi cosmici vengono fatti evolvere nel limite ultrarelativistico: data la loro alta velocità si può considerare che essi seguano una traiettoria rettilinea nonostante
le forze esterne (in particolare rilevante in aria è l'effetto del campo magnetico che tenderebbe a deviare la traiettoria), e si muovano a velocità infinita.

\subsection{Interazione con i rivelatori}
Quando viene fatto evolvere un muone esso potrebbe entrare all'interno dei rivelatori, che possono rivelare tale passaggio. Per modellizzare tale evento, come prima cosa,
si considera che il rivelatore non influenza il raggio cosmico (il muone non può quindi essere assorbito all'interno del rivelatore secondo questa modellizzazione), inoltre
si modellizza l'interazione come un fenomeno esattamente prevedibile e non stocastico come realmente è. Per poter comprendere cosa succede quando un muone attraversa
un rivelatore si utilizzano i dati sperimentali (otttenuti come descritto alla Sezione !!!!!!!INSERIRE RIFERIMENTO A SEZIONE!!!!!!!): data la moda della distribuzione
sperimentale di fotoni, quello si itnerpreta come il numero di fotoni generati dal passaggio di un muone cosmico quando attraversa lo spessore (noto) del rivelatore
in direzione perpendicolare alla faccia del rivelatore stesso. In questo modo si può andare a stimare effettivamente quanti fotoni vengono generati per ogni mm 
di scintillatore attraversato dal muone cosmico (si noti che si stanno trascurando parecchi fattori, come per esempio il diverso assorbimento in diversi punti dell'assorbitore
a diversa distanza dalla fibra ottica all'interno dello scintillatore). Così, usando delle identità trigonometriche, è stato possibile trovare lo spazio percorso
all'interno del rivelatore e, noto quest'ultimo, è stato possibile trovare il numero di fotoni che ci si aspettta arrivino ai canali di acquisizione.

\subsection{Interazione con l'assorbitore}
Per quanto riguarda l'interazione con l'assorbitore, diversamente a quanto fatto per l'interazione con i rivelatori, si considera il processo come stocastico. Un muone
ha una probabilità di interagire con il materiale che dipende dal materiale stesso e dall'energia del muone. Tale relazione è stata linearizzata, e si è considerato che il
muone ha probabilità di decadere uniforme in una regione spaziale ben definita\footnote{tale regione è stata impostata computazionalmente in modo che non si generino troppi
dati inutili, quindi a meno di un coefficiente moltiplicativo da determinare sperimentalmente una volta che l'apparato sarà operativo.}. Si è inoltre modellizzato il fenomeno
in modo tale che il muone possa \textit{solamente} fermarsi all'interno dell'assorbitore o non farlo, si stanno cioè trascurando i casi in cui il muone rallenta
all'interno dell'assorbitore prima di essere assorbito completamente (infatti fisicamente il muone tende a perdere gran parte della sua energia in un urto: se riesce
a fare un secondo urto con il materiale dell'assorbitore prima di decadere esso sarà comunque in una posizione molto vicina a quella del primo urti, quindi si può
considerare che esso abbia fatto effettivamente un solo urto).

\subsection{Implementazione del caampo magnetico}
Il campo magnetico dell'esperimento è generato da un solenoide finito a sezione rettangolare. Si è voluto andare ad utilizzare un campo che non fosse uniforme all'interno
del solenoide, e per farlo si è risolto numericamente tale problema. Si è introdotta la corrente come pareti di corrente uniforme e costante, si è discretizzato lo spazio
e si è utlizzato l'algoritmo di Jacobi per ottenere il potenziale vettore data la densità di corrente introdotta. Poi si è calcolato numericamente il rotore per
andare a trovare effettivamente il campo magnetico. L'algoritmo è stato fattto girare su uno spazio più grande (circa un fattore 4) del solenoide, in modo che siano
fisicamente sensate le condizioni al contorno assorbenti ai bordi del sistema. Da questo calcolo si è trovato come effettivamente il campo magnetico non sia costante
all'interno del solenoide ma abbia una dipendenza dalla posizione, come si può vedere nella Figura \ref{gr:campo_xy} dove si può vedere la proiezione lungo una
sezione del solenoide della componente del campo magnetico parallela all'asse del solenoide stesso. Questo processo di risoluzione numerica dell'equazione differenziale
ha permesso di avere dei valori per il campo magnetico più realistici che dipendano dalla posizione presa in considerazione.
\begin{tikzpicture}
\pgfdeclareplotmark{cross} {
\pgfpathmoveto{\pgfpoint{-0.3\pgfplotmarksize}{\pgfplotmarksize}}
\pgfpathlineto{\pgfpoint{+0.3\pgfplotmarksize}{\pgfplotmarksize}}
\pgfpathlineto{\pgfpoint{+0.3\pgfplotmarksize}{0.3\pgfplotmarksize}}
\pgfpathlineto{\pgfpoint{+1\pgfplotmarksize}{0.3\pgfplotmarksize}}
\pgfpathlineto{\pgfpoint{+1\pgfplotmarksize}{-0.3\pgfplotmarksize}}
\pgfpathlineto{\pgfpoint{+0.3\pgfplotmarksize}{-0.3\pgfplotmarksize}}
\pgfpathlineto{\pgfpoint{+0.3\pgfplotmarksize}{-1.\pgfplotmarksize}}
\pgfpathlineto{\pgfpoint{-0.3\pgfplotmarksize}{-1.\pgfplotmarksize}}
\pgfpathlineto{\pgfpoint{-0.3\pgfplotmarksize}{-0.3\pgfplotmarksize}}
\pgfpathlineto{\pgfpoint{-1.\pgfplotmarksize}{-0.3\pgfplotmarksize}}
\pgfpathlineto{\pgfpoint{-1.\pgfplotmarksize}{0.3\pgfplotmarksize}}
\pgfpathlineto{\pgfpoint{-0.3\pgfplotmarksize}{0.3\pgfplotmarksize}}
\pgfpathclose
\pgfusepathqstroke
}
\pgfdeclareplotmark{cross*} {
\pgfpathmoveto{\pgfpoint{-0.3\pgfplotmarksize}{\pgfplotmarksize}}
\pgfpathlineto{\pgfpoint{+0.3\pgfplotmarksize}{\pgfplotmarksize}}
\pgfpathlineto{\pgfpoint{+0.3\pgfplotmarksize}{0.3\pgfplotmarksize}}
\pgfpathlineto{\pgfpoint{+1\pgfplotmarksize}{0.3\pgfplotmarksize}}
\pgfpathlineto{\pgfpoint{+1\pgfplotmarksize}{-0.3\pgfplotmarksize}}
\pgfpathlineto{\pgfpoint{+0.3\pgfplotmarksize}{-0.3\pgfplotmarksize}}
\pgfpathlineto{\pgfpoint{+0.3\pgfplotmarksize}{-1.\pgfplotmarksize}}
\pgfpathlineto{\pgfpoint{-0.3\pgfplotmarksize}{-1.\pgfplotmarksize}}
\pgfpathlineto{\pgfpoint{-0.3\pgfplotmarksize}{-0.3\pgfplotmarksize}}
\pgfpathlineto{\pgfpoint{-1.\pgfplotmarksize}{-0.3\pgfplotmarksize}}
\pgfpathlineto{\pgfpoint{-1.\pgfplotmarksize}{0.3\pgfplotmarksize}}
\pgfpathlineto{\pgfpoint{-0.3\pgfplotmarksize}{0.3\pgfplotmarksize}}
\pgfpathclose
\pgfusepathqfillstroke
}
\pgfdeclareplotmark{newstar} {
\pgfpathmoveto{\pgfqpoint{0pt}{\pgfplotmarksize}}
\pgfpathlineto{\pgfqpointpolar{44}{0.5\pgfplotmarksize}}
\pgfpathlineto{\pgfqpointpolar{18}{\pgfplotmarksize}}
\pgfpathlineto{\pgfqpointpolar{-20}{0.5\pgfplotmarksize}}
\pgfpathlineto{\pgfqpointpolar{-54}{\pgfplotmarksize}}
\pgfpathlineto{\pgfqpointpolar{-90}{0.5\pgfplotmarksize}}
\pgfpathlineto{\pgfqpointpolar{234}{\pgfplotmarksize}}
\pgfpathlineto{\pgfqpointpolar{198}{0.5\pgfplotmarksize}}
\pgfpathlineto{\pgfqpointpolar{162}{\pgfplotmarksize}}
\pgfpathlineto{\pgfqpointpolar{134}{0.5\pgfplotmarksize}}
\pgfpathclose
\pgfusepathqstroke
}
\pgfdeclareplotmark{newstar*} {
\pgfpathmoveto{\pgfqpoint{0pt}{\pgfplotmarksize}}
\pgfpathlineto{\pgfqpointpolar{44}{0.5\pgfplotmarksize}}
\pgfpathlineto{\pgfqpointpolar{18}{\pgfplotmarksize}}
\pgfpathlineto{\pgfqpointpolar{-20}{0.5\pgfplotmarksize}}
\pgfpathlineto{\pgfqpointpolar{-54}{\pgfplotmarksize}}
\pgfpathlineto{\pgfqpointpolar{-90}{0.5\pgfplotmarksize}}
\pgfpathlineto{\pgfqpointpolar{234}{\pgfplotmarksize}}
\pgfpathlineto{\pgfqpointpolar{198}{0.5\pgfplotmarksize}}
\pgfpathlineto{\pgfqpointpolar{162}{\pgfplotmarksize}}
\pgfpathlineto{\pgfqpointpolar{134}{0.5\pgfplotmarksize}}
\pgfpathclose
\pgfusepathqfillstroke
}
\definecolor{c}{rgb}{1,1,1};
\draw [color=c, fill=c] (0,0) rectangle (20,13.4957);
\draw [color=c, fill=c] (2,1.34957) rectangle (18,12.1461);
\definecolor{c}{rgb}{0,0,0};
\draw [c,line width=0.9] (2,1.34957) -- (2,12.1461) -- (18,12.1461) -- (18,1.34957) -- (2,1.34957);
\definecolor{c}{rgb}{1,1,1};
\draw [color=c, fill=c] (2,1.34957) rectangle (18,12.1461);
\definecolor{c}{rgb}{0,0,0};
\draw [c,line width=0.9] (2,1.34957) -- (2,12.1461) -- (18,12.1461) -- (18,1.34957) -- (2,1.34957);
\definecolor{c}{rgb}{1,0,0};
\draw [color=c, fill=c] (2,1.34957) rectangle (2.0398,1.45542);
\draw [color=c, fill=c] (2.0398,1.34957) rectangle (2.0796,1.45542);
\draw [color=c, fill=c] (2.0796,1.34957) rectangle (2.1194,1.45542);
\draw [color=c, fill=c] (2.1194,1.34957) rectangle (2.1592,1.45542);
\draw [color=c, fill=c] (2.1592,1.34957) rectangle (2.19901,1.45542);
\draw [color=c, fill=c] (2.19901,1.34957) rectangle (2.23881,1.45542);
\draw [color=c, fill=c] (2.23881,1.34957) rectangle (2.27861,1.45542);
\draw [color=c, fill=c] (2.27861,1.34957) rectangle (2.31841,1.45542);
\draw [color=c, fill=c] (2.31841,1.34957) rectangle (2.35821,1.45542);
\draw [color=c, fill=c] (2.35821,1.34957) rectangle (2.39801,1.45542);
\draw [color=c, fill=c] (2.39801,1.34957) rectangle (2.43781,1.45542);
\draw [color=c, fill=c] (2.43781,1.34957) rectangle (2.47761,1.45542);
\draw [color=c, fill=c] (2.47761,1.34957) rectangle (2.51741,1.45542);
\draw [color=c, fill=c] (2.51741,1.34957) rectangle (2.55721,1.45542);
\draw [color=c, fill=c] (2.55721,1.34957) rectangle (2.59702,1.45542);
\draw [color=c, fill=c] (2.59702,1.34957) rectangle (2.63682,1.45542);
\draw [color=c, fill=c] (2.63682,1.34957) rectangle (2.67662,1.45542);
\draw [color=c, fill=c] (2.67662,1.34957) rectangle (2.71642,1.45542);
\draw [color=c, fill=c] (2.71642,1.34957) rectangle (2.75622,1.45542);
\draw [color=c, fill=c] (2.75622,1.34957) rectangle (2.79602,1.45542);
\draw [color=c, fill=c] (2.79602,1.34957) rectangle (2.83582,1.45542);
\draw [color=c, fill=c] (2.83582,1.34957) rectangle (2.87562,1.45542);
\draw [color=c, fill=c] (2.87562,1.34957) rectangle (2.91542,1.45542);
\draw [color=c, fill=c] (2.91542,1.34957) rectangle (2.95522,1.45542);
\draw [color=c, fill=c] (2.95522,1.34957) rectangle (2.99502,1.45542);
\draw [color=c, fill=c] (2.99502,1.34957) rectangle (3.03483,1.45542);
\draw [color=c, fill=c] (3.03483,1.34957) rectangle (3.07463,1.45542);
\draw [color=c, fill=c] (3.07463,1.34957) rectangle (3.11443,1.45542);
\draw [color=c, fill=c] (3.11443,1.34957) rectangle (3.15423,1.45542);
\draw [color=c, fill=c] (3.15423,1.34957) rectangle (3.19403,1.45542);
\draw [color=c, fill=c] (3.19403,1.34957) rectangle (3.23383,1.45542);
\draw [color=c, fill=c] (3.23383,1.34957) rectangle (3.27363,1.45542);
\draw [color=c, fill=c] (3.27363,1.34957) rectangle (3.31343,1.45542);
\draw [color=c, fill=c] (3.31343,1.34957) rectangle (3.35323,1.45542);
\draw [color=c, fill=c] (3.35323,1.34957) rectangle (3.39303,1.45542);
\draw [color=c, fill=c] (3.39303,1.34957) rectangle (3.43284,1.45542);
\draw [color=c, fill=c] (3.43284,1.34957) rectangle (3.47264,1.45542);
\draw [color=c, fill=c] (3.47264,1.34957) rectangle (3.51244,1.45542);
\draw [color=c, fill=c] (3.51244,1.34957) rectangle (3.55224,1.45542);
\draw [color=c, fill=c] (3.55224,1.34957) rectangle (3.59204,1.45542);
\draw [color=c, fill=c] (3.59204,1.34957) rectangle (3.63184,1.45542);
\draw [color=c, fill=c] (3.63184,1.34957) rectangle (3.67164,1.45542);
\draw [color=c, fill=c] (3.67164,1.34957) rectangle (3.71144,1.45542);
\draw [color=c, fill=c] (3.71144,1.34957) rectangle (3.75124,1.45542);
\draw [color=c, fill=c] (3.75124,1.34957) rectangle (3.79104,1.45542);
\draw [color=c, fill=c] (3.79104,1.34957) rectangle (3.83085,1.45542);
\draw [color=c, fill=c] (3.83085,1.34957) rectangle (3.87065,1.45542);
\draw [color=c, fill=c] (3.87065,1.34957) rectangle (3.91045,1.45542);
\draw [color=c, fill=c] (3.91045,1.34957) rectangle (3.95025,1.45542);
\draw [color=c, fill=c] (3.95025,1.34957) rectangle (3.99005,1.45542);
\draw [color=c, fill=c] (3.99005,1.34957) rectangle (4.02985,1.45542);
\draw [color=c, fill=c] (4.02985,1.34957) rectangle (4.06965,1.45542);
\draw [color=c, fill=c] (4.06965,1.34957) rectangle (4.10945,1.45542);
\draw [color=c, fill=c] (4.10945,1.34957) rectangle (4.14925,1.45542);
\draw [color=c, fill=c] (4.14925,1.34957) rectangle (4.18905,1.45542);
\draw [color=c, fill=c] (4.18905,1.34957) rectangle (4.22886,1.45542);
\draw [color=c, fill=c] (4.22886,1.34957) rectangle (4.26866,1.45542);
\draw [color=c, fill=c] (4.26866,1.34957) rectangle (4.30846,1.45542);
\draw [color=c, fill=c] (4.30846,1.34957) rectangle (4.34826,1.45542);
\draw [color=c, fill=c] (4.34826,1.34957) rectangle (4.38806,1.45542);
\draw [color=c, fill=c] (4.38806,1.34957) rectangle (4.42786,1.45542);
\draw [color=c, fill=c] (4.42786,1.34957) rectangle (4.46766,1.45542);
\draw [color=c, fill=c] (4.46766,1.34957) rectangle (4.50746,1.45542);
\draw [color=c, fill=c] (4.50746,1.34957) rectangle (4.54726,1.45542);
\draw [color=c, fill=c] (4.54726,1.34957) rectangle (4.58706,1.45542);
\draw [color=c, fill=c] (4.58706,1.34957) rectangle (4.62687,1.45542);
\draw [color=c, fill=c] (4.62687,1.34957) rectangle (4.66667,1.45542);
\draw [color=c, fill=c] (4.66667,1.34957) rectangle (4.70647,1.45542);
\draw [color=c, fill=c] (4.70647,1.34957) rectangle (4.74627,1.45542);
\draw [color=c, fill=c] (4.74627,1.34957) rectangle (4.78607,1.45542);
\draw [color=c, fill=c] (4.78607,1.34957) rectangle (4.82587,1.45542);
\draw [color=c, fill=c] (4.82587,1.34957) rectangle (4.86567,1.45542);
\draw [color=c, fill=c] (4.86567,1.34957) rectangle (4.90547,1.45542);
\draw [color=c, fill=c] (4.90547,1.34957) rectangle (4.94527,1.45542);
\draw [color=c, fill=c] (4.94527,1.34957) rectangle (4.98507,1.45542);
\draw [color=c, fill=c] (4.98507,1.34957) rectangle (5.02488,1.45542);
\draw [color=c, fill=c] (5.02488,1.34957) rectangle (5.06468,1.45542);
\draw [color=c, fill=c] (5.06468,1.34957) rectangle (5.10448,1.45542);
\draw [color=c, fill=c] (5.10448,1.34957) rectangle (5.14428,1.45542);
\draw [color=c, fill=c] (5.14428,1.34957) rectangle (5.18408,1.45542);
\draw [color=c, fill=c] (5.18408,1.34957) rectangle (5.22388,1.45542);
\draw [color=c, fill=c] (5.22388,1.34957) rectangle (5.26368,1.45542);
\draw [color=c, fill=c] (5.26368,1.34957) rectangle (5.30348,1.45542);
\draw [color=c, fill=c] (5.30348,1.34957) rectangle (5.34328,1.45542);
\draw [color=c, fill=c] (5.34328,1.34957) rectangle (5.38308,1.45542);
\draw [color=c, fill=c] (5.38308,1.34957) rectangle (5.42289,1.45542);
\draw [color=c, fill=c] (5.42289,1.34957) rectangle (5.46269,1.45542);
\draw [color=c, fill=c] (5.46269,1.34957) rectangle (5.50249,1.45542);
\draw [color=c, fill=c] (5.50249,1.34957) rectangle (5.54229,1.45542);
\draw [color=c, fill=c] (5.54229,1.34957) rectangle (5.58209,1.45542);
\draw [color=c, fill=c] (5.58209,1.34957) rectangle (5.62189,1.45542);
\draw [color=c, fill=c] (5.62189,1.34957) rectangle (5.66169,1.45542);
\draw [color=c, fill=c] (5.66169,1.34957) rectangle (5.70149,1.45542);
\draw [color=c, fill=c] (5.70149,1.34957) rectangle (5.74129,1.45542);
\draw [color=c, fill=c] (5.74129,1.34957) rectangle (5.78109,1.45542);
\draw [color=c, fill=c] (5.78109,1.34957) rectangle (5.8209,1.45542);
\draw [color=c, fill=c] (5.8209,1.34957) rectangle (5.8607,1.45542);
\draw [color=c, fill=c] (5.8607,1.34957) rectangle (5.9005,1.45542);
\draw [color=c, fill=c] (5.9005,1.34957) rectangle (5.9403,1.45542);
\draw [color=c, fill=c] (5.9403,1.34957) rectangle (5.9801,1.45542);
\draw [color=c, fill=c] (5.9801,1.34957) rectangle (6.0199,1.45542);
\draw [color=c, fill=c] (6.0199,1.34957) rectangle (6.0597,1.45542);
\draw [color=c, fill=c] (6.0597,1.34957) rectangle (6.0995,1.45542);
\draw [color=c, fill=c] (6.0995,1.34957) rectangle (6.1393,1.45542);
\draw [color=c, fill=c] (6.1393,1.34957) rectangle (6.1791,1.45542);
\draw [color=c, fill=c] (6.1791,1.34957) rectangle (6.21891,1.45542);
\draw [color=c, fill=c] (6.21891,1.34957) rectangle (6.25871,1.45542);
\draw [color=c, fill=c] (6.25871,1.34957) rectangle (6.29851,1.45542);
\draw [color=c, fill=c] (6.29851,1.34957) rectangle (6.33831,1.45542);
\draw [color=c, fill=c] (6.33831,1.34957) rectangle (6.37811,1.45542);
\draw [color=c, fill=c] (6.37811,1.34957) rectangle (6.41791,1.45542);
\draw [color=c, fill=c] (6.41791,1.34957) rectangle (6.45771,1.45542);
\draw [color=c, fill=c] (6.45771,1.34957) rectangle (6.49751,1.45542);
\draw [color=c, fill=c] (6.49751,1.34957) rectangle (6.53731,1.45542);
\draw [color=c, fill=c] (6.53731,1.34957) rectangle (6.57711,1.45542);
\draw [color=c, fill=c] (6.57711,1.34957) rectangle (6.61692,1.45542);
\draw [color=c, fill=c] (6.61692,1.34957) rectangle (6.65672,1.45542);
\draw [color=c, fill=c] (6.65672,1.34957) rectangle (6.69652,1.45542);
\draw [color=c, fill=c] (6.69652,1.34957) rectangle (6.73632,1.45542);
\draw [color=c, fill=c] (6.73632,1.34957) rectangle (6.77612,1.45542);
\draw [color=c, fill=c] (6.77612,1.34957) rectangle (6.81592,1.45542);
\draw [color=c, fill=c] (6.81592,1.34957) rectangle (6.85572,1.45542);
\draw [color=c, fill=c] (6.85572,1.34957) rectangle (6.89552,1.45542);
\draw [color=c, fill=c] (6.89552,1.34957) rectangle (6.93532,1.45542);
\draw [color=c, fill=c] (6.93532,1.34957) rectangle (6.97512,1.45542);
\draw [color=c, fill=c] (6.97512,1.34957) rectangle (7.01493,1.45542);
\draw [color=c, fill=c] (7.01493,1.34957) rectangle (7.05473,1.45542);
\draw [color=c, fill=c] (7.05473,1.34957) rectangle (7.09453,1.45542);
\draw [color=c, fill=c] (7.09453,1.34957) rectangle (7.13433,1.45542);
\draw [color=c, fill=c] (7.13433,1.34957) rectangle (7.17413,1.45542);
\draw [color=c, fill=c] (7.17413,1.34957) rectangle (7.21393,1.45542);
\draw [color=c, fill=c] (7.21393,1.34957) rectangle (7.25373,1.45542);
\draw [color=c, fill=c] (7.25373,1.34957) rectangle (7.29353,1.45542);
\draw [color=c, fill=c] (7.29353,1.34957) rectangle (7.33333,1.45542);
\draw [color=c, fill=c] (7.33333,1.34957) rectangle (7.37313,1.45542);
\draw [color=c, fill=c] (7.37313,1.34957) rectangle (7.41294,1.45542);
\draw [color=c, fill=c] (7.41294,1.34957) rectangle (7.45274,1.45542);
\draw [color=c, fill=c] (7.45274,1.34957) rectangle (7.49254,1.45542);
\draw [color=c, fill=c] (7.49254,1.34957) rectangle (7.53234,1.45542);
\draw [color=c, fill=c] (7.53234,1.34957) rectangle (7.57214,1.45542);
\draw [color=c, fill=c] (7.57214,1.34957) rectangle (7.61194,1.45542);
\draw [color=c, fill=c] (7.61194,1.34957) rectangle (7.65174,1.45542);
\draw [color=c, fill=c] (7.65174,1.34957) rectangle (7.69154,1.45542);
\draw [color=c, fill=c] (7.69154,1.34957) rectangle (7.73134,1.45542);
\draw [color=c, fill=c] (7.73134,1.34957) rectangle (7.77114,1.45542);
\definecolor{c}{rgb}{1,0.186667,0};
\draw [color=c, fill=c] (7.77114,1.34957) rectangle (7.81095,1.45542);
\draw [color=c, fill=c] (7.81095,1.34957) rectangle (7.85075,1.45542);
\draw [color=c, fill=c] (7.85075,1.34957) rectangle (7.89055,1.45542);
\draw [color=c, fill=c] (7.89055,1.34957) rectangle (7.93035,1.45542);
\draw [color=c, fill=c] (7.93035,1.34957) rectangle (7.97015,1.45542);
\draw [color=c, fill=c] (7.97015,1.34957) rectangle (8.00995,1.45542);
\draw [color=c, fill=c] (8.00995,1.34957) rectangle (8.04975,1.45542);
\draw [color=c, fill=c] (8.04975,1.34957) rectangle (8.08955,1.45542);
\draw [color=c, fill=c] (8.08955,1.34957) rectangle (8.12935,1.45542);
\draw [color=c, fill=c] (8.12935,1.34957) rectangle (8.16915,1.45542);
\draw [color=c, fill=c] (8.16915,1.34957) rectangle (8.20895,1.45542);
\draw [color=c, fill=c] (8.20895,1.34957) rectangle (8.24876,1.45542);
\draw [color=c, fill=c] (8.24876,1.34957) rectangle (8.28856,1.45542);
\draw [color=c, fill=c] (8.28856,1.34957) rectangle (8.32836,1.45542);
\draw [color=c, fill=c] (8.32836,1.34957) rectangle (8.36816,1.45542);
\draw [color=c, fill=c] (8.36816,1.34957) rectangle (8.40796,1.45542);
\draw [color=c, fill=c] (8.40796,1.34957) rectangle (8.44776,1.45542);
\draw [color=c, fill=c] (8.44776,1.34957) rectangle (8.48756,1.45542);
\draw [color=c, fill=c] (8.48756,1.34957) rectangle (8.52736,1.45542);
\draw [color=c, fill=c] (8.52736,1.34957) rectangle (8.56716,1.45542);
\draw [color=c, fill=c] (8.56716,1.34957) rectangle (8.60697,1.45542);
\definecolor{c}{rgb}{1,0.466667,0};
\draw [color=c, fill=c] (8.60697,1.34957) rectangle (8.64677,1.45542);
\draw [color=c, fill=c] (8.64677,1.34957) rectangle (8.68657,1.45542);
\draw [color=c, fill=c] (8.68657,1.34957) rectangle (8.72637,1.45542);
\draw [color=c, fill=c] (8.72637,1.34957) rectangle (8.76617,1.45542);
\draw [color=c, fill=c] (8.76617,1.34957) rectangle (8.80597,1.45542);
\draw [color=c, fill=c] (8.80597,1.34957) rectangle (8.84577,1.45542);
\draw [color=c, fill=c] (8.84577,1.34957) rectangle (8.88557,1.45542);
\draw [color=c, fill=c] (8.88557,1.34957) rectangle (8.92537,1.45542);
\draw [color=c, fill=c] (8.92537,1.34957) rectangle (8.96517,1.45542);
\draw [color=c, fill=c] (8.96517,1.34957) rectangle (9.00498,1.45542);
\definecolor{c}{rgb}{1,0.653333,0};
\draw [color=c, fill=c] (9.00498,1.34957) rectangle (9.04478,1.45542);
\draw [color=c, fill=c] (9.04478,1.34957) rectangle (9.08458,1.45542);
\draw [color=c, fill=c] (9.08458,1.34957) rectangle (9.12438,1.45542);
\draw [color=c, fill=c] (9.12438,1.34957) rectangle (9.16418,1.45542);
\draw [color=c, fill=c] (9.16418,1.34957) rectangle (9.20398,1.45542);
\draw [color=c, fill=c] (9.20398,1.34957) rectangle (9.24378,1.45542);
\draw [color=c, fill=c] (9.24378,1.34957) rectangle (9.28358,1.45542);
\draw [color=c, fill=c] (9.28358,1.34957) rectangle (9.32338,1.45542);
\definecolor{c}{rgb}{1,0.933333,0};
\draw [color=c, fill=c] (9.32338,1.34957) rectangle (9.36318,1.45542);
\draw [color=c, fill=c] (9.36318,1.34957) rectangle (9.40298,1.45542);
\draw [color=c, fill=c] (9.40298,1.34957) rectangle (9.44279,1.45542);
\draw [color=c, fill=c] (9.44279,1.34957) rectangle (9.48259,1.45542);
\draw [color=c, fill=c] (9.48259,1.34957) rectangle (9.52239,1.45542);
\definecolor{c}{rgb}{0.88,1,0};
\draw [color=c, fill=c] (9.52239,1.34957) rectangle (9.56219,1.45542);
\draw [color=c, fill=c] (9.56219,1.34957) rectangle (9.60199,1.45542);
\draw [color=c, fill=c] (9.60199,1.34957) rectangle (9.64179,1.45542);
\draw [color=c, fill=c] (9.64179,1.34957) rectangle (9.68159,1.45542);
\draw [color=c, fill=c] (9.68159,1.34957) rectangle (9.72139,1.45542);
\definecolor{c}{rgb}{0.6,1,0};
\draw [color=c, fill=c] (9.72139,1.34957) rectangle (9.76119,1.45542);
\draw [color=c, fill=c] (9.76119,1.34957) rectangle (9.80099,1.45542);
\draw [color=c, fill=c] (9.80099,1.34957) rectangle (9.8408,1.45542);
\draw [color=c, fill=c] (9.8408,1.34957) rectangle (9.8806,1.45542);
\definecolor{c}{rgb}{0.413333,1,0};
\draw [color=c, fill=c] (9.8806,1.34957) rectangle (9.9204,1.45542);
\draw [color=c, fill=c] (9.9204,1.34957) rectangle (9.9602,1.45542);
\draw [color=c, fill=c] (9.9602,1.34957) rectangle (10,1.45542);
\draw [color=c, fill=c] (10,1.34957) rectangle (10.0398,1.45542);
\definecolor{c}{rgb}{0.133333,1,0};
\draw [color=c, fill=c] (10.0398,1.34957) rectangle (10.0796,1.45542);
\draw [color=c, fill=c] (10.0796,1.34957) rectangle (10.1194,1.45542);
\draw [color=c, fill=c] (10.1194,1.34957) rectangle (10.1592,1.45542);
\draw [color=c, fill=c] (10.1592,1.34957) rectangle (10.199,1.45542);
\draw [color=c, fill=c] (10.199,1.34957) rectangle (10.2388,1.45542);
\definecolor{c}{rgb}{0,1,0.0533333};
\draw [color=c, fill=c] (10.2388,1.34957) rectangle (10.2786,1.45542);
\draw [color=c, fill=c] (10.2786,1.34957) rectangle (10.3184,1.45542);
\draw [color=c, fill=c] (10.3184,1.34957) rectangle (10.3582,1.45542);
\draw [color=c, fill=c] (10.3582,1.34957) rectangle (10.398,1.45542);
\definecolor{c}{rgb}{0,1,0.333333};
\draw [color=c, fill=c] (10.398,1.34957) rectangle (10.4378,1.45542);
\draw [color=c, fill=c] (10.4378,1.34957) rectangle (10.4776,1.45542);
\draw [color=c, fill=c] (10.4776,1.34957) rectangle (10.5174,1.45542);
\draw [color=c, fill=c] (10.5174,1.34957) rectangle (10.5572,1.45542);
\draw [color=c, fill=c] (10.5572,1.34957) rectangle (10.597,1.45542);
\draw [color=c, fill=c] (10.597,1.34957) rectangle (10.6368,1.45542);
\definecolor{c}{rgb}{0,1,0.52};
\draw [color=c, fill=c] (10.6368,1.34957) rectangle (10.6766,1.45542);
\draw [color=c, fill=c] (10.6766,1.34957) rectangle (10.7164,1.45542);
\draw [color=c, fill=c] (10.7164,1.34957) rectangle (10.7562,1.45542);
\draw [color=c, fill=c] (10.7562,1.34957) rectangle (10.796,1.45542);
\draw [color=c, fill=c] (10.796,1.34957) rectangle (10.8358,1.45542);
\draw [color=c, fill=c] (10.8358,1.34957) rectangle (10.8756,1.45542);
\draw [color=c, fill=c] (10.8756,1.34957) rectangle (10.9154,1.45542);
\draw [color=c, fill=c] (10.9154,1.34957) rectangle (10.9552,1.45542);
\definecolor{c}{rgb}{0,1,0.8};
\draw [color=c, fill=c] (10.9552,1.34957) rectangle (10.995,1.45542);
\draw [color=c, fill=c] (10.995,1.34957) rectangle (11.0348,1.45542);
\draw [color=c, fill=c] (11.0348,1.34957) rectangle (11.0746,1.45542);
\draw [color=c, fill=c] (11.0746,1.34957) rectangle (11.1144,1.45542);
\draw [color=c, fill=c] (11.1144,1.34957) rectangle (11.1542,1.45542);
\draw [color=c, fill=c] (11.1542,1.34957) rectangle (11.194,1.45542);
\draw [color=c, fill=c] (11.194,1.34957) rectangle (11.2338,1.45542);
\draw [color=c, fill=c] (11.2338,1.34957) rectangle (11.2736,1.45542);
\draw [color=c, fill=c] (11.2736,1.34957) rectangle (11.3134,1.45542);
\draw [color=c, fill=c] (11.3134,1.34957) rectangle (11.3532,1.45542);
\draw [color=c, fill=c] (11.3532,1.34957) rectangle (11.393,1.45542);
\draw [color=c, fill=c] (11.393,1.34957) rectangle (11.4328,1.45542);
\definecolor{c}{rgb}{0,1,0.986667};
\draw [color=c, fill=c] (11.4328,1.34957) rectangle (11.4726,1.45542);
\draw [color=c, fill=c] (11.4726,1.34957) rectangle (11.5124,1.45542);
\draw [color=c, fill=c] (11.5124,1.34957) rectangle (11.5522,1.45542);
\draw [color=c, fill=c] (11.5522,1.34957) rectangle (11.592,1.45542);
\draw [color=c, fill=c] (11.592,1.34957) rectangle (11.6318,1.45542);
\draw [color=c, fill=c] (11.6318,1.34957) rectangle (11.6716,1.45542);
\draw [color=c, fill=c] (11.6716,1.34957) rectangle (11.7114,1.45542);
\draw [color=c, fill=c] (11.7114,1.34957) rectangle (11.7512,1.45542);
\draw [color=c, fill=c] (11.7512,1.34957) rectangle (11.791,1.45542);
\draw [color=c, fill=c] (11.791,1.34957) rectangle (11.8308,1.45542);
\draw [color=c, fill=c] (11.8308,1.34957) rectangle (11.8706,1.45542);
\draw [color=c, fill=c] (11.8706,1.34957) rectangle (11.9104,1.45542);
\draw [color=c, fill=c] (11.9104,1.34957) rectangle (11.9502,1.45542);
\draw [color=c, fill=c] (11.9502,1.34957) rectangle (11.99,1.45542);
\draw [color=c, fill=c] (11.99,1.34957) rectangle (12.0299,1.45542);
\draw [color=c, fill=c] (12.0299,1.34957) rectangle (12.0697,1.45542);
\draw [color=c, fill=c] (12.0697,1.34957) rectangle (12.1095,1.45542);
\draw [color=c, fill=c] (12.1095,1.34957) rectangle (12.1493,1.45542);
\draw [color=c, fill=c] (12.1493,1.34957) rectangle (12.1891,1.45542);
\draw [color=c, fill=c] (12.1891,1.34957) rectangle (12.2289,1.45542);
\draw [color=c, fill=c] (12.2289,1.34957) rectangle (12.2687,1.45542);
\draw [color=c, fill=c] (12.2687,1.34957) rectangle (12.3085,1.45542);
\draw [color=c, fill=c] (12.3085,1.34957) rectangle (12.3483,1.45542);
\definecolor{c}{rgb}{0,0.733333,1};
\draw [color=c, fill=c] (12.3483,1.34957) rectangle (12.3881,1.45542);
\draw [color=c, fill=c] (12.3881,1.34957) rectangle (12.4279,1.45542);
\draw [color=c, fill=c] (12.4279,1.34957) rectangle (12.4677,1.45542);
\draw [color=c, fill=c] (12.4677,1.34957) rectangle (12.5075,1.45542);
\draw [color=c, fill=c] (12.5075,1.34957) rectangle (12.5473,1.45542);
\draw [color=c, fill=c] (12.5473,1.34957) rectangle (12.5871,1.45542);
\draw [color=c, fill=c] (12.5871,1.34957) rectangle (12.6269,1.45542);
\draw [color=c, fill=c] (12.6269,1.34957) rectangle (12.6667,1.45542);
\draw [color=c, fill=c] (12.6667,1.34957) rectangle (12.7065,1.45542);
\draw [color=c, fill=c] (12.7065,1.34957) rectangle (12.7463,1.45542);
\draw [color=c, fill=c] (12.7463,1.34957) rectangle (12.7861,1.45542);
\draw [color=c, fill=c] (12.7861,1.34957) rectangle (12.8259,1.45542);
\draw [color=c, fill=c] (12.8259,1.34957) rectangle (12.8657,1.45542);
\draw [color=c, fill=c] (12.8657,1.34957) rectangle (12.9055,1.45542);
\draw [color=c, fill=c] (12.9055,1.34957) rectangle (12.9453,1.45542);
\draw [color=c, fill=c] (12.9453,1.34957) rectangle (12.9851,1.45542);
\draw [color=c, fill=c] (12.9851,1.34957) rectangle (13.0249,1.45542);
\draw [color=c, fill=c] (13.0249,1.34957) rectangle (13.0647,1.45542);
\draw [color=c, fill=c] (13.0647,1.34957) rectangle (13.1045,1.45542);
\draw [color=c, fill=c] (13.1045,1.34957) rectangle (13.1443,1.45542);
\draw [color=c, fill=c] (13.1443,1.34957) rectangle (13.1841,1.45542);
\draw [color=c, fill=c] (13.1841,1.34957) rectangle (13.2239,1.45542);
\draw [color=c, fill=c] (13.2239,1.34957) rectangle (13.2637,1.45542);
\draw [color=c, fill=c] (13.2637,1.34957) rectangle (13.3035,1.45542);
\draw [color=c, fill=c] (13.3035,1.34957) rectangle (13.3433,1.45542);
\draw [color=c, fill=c] (13.3433,1.34957) rectangle (13.3831,1.45542);
\draw [color=c, fill=c] (13.3831,1.34957) rectangle (13.4229,1.45542);
\draw [color=c, fill=c] (13.4229,1.34957) rectangle (13.4627,1.45542);
\draw [color=c, fill=c] (13.4627,1.34957) rectangle (13.5025,1.45542);
\draw [color=c, fill=c] (13.5025,1.34957) rectangle (13.5423,1.45542);
\draw [color=c, fill=c] (13.5423,1.34957) rectangle (13.5821,1.45542);
\draw [color=c, fill=c] (13.5821,1.34957) rectangle (13.6219,1.45542);
\draw [color=c, fill=c] (13.6219,1.34957) rectangle (13.6617,1.45542);
\draw [color=c, fill=c] (13.6617,1.34957) rectangle (13.7015,1.45542);
\draw [color=c, fill=c] (13.7015,1.34957) rectangle (13.7413,1.45542);
\draw [color=c, fill=c] (13.7413,1.34957) rectangle (13.7811,1.45542);
\draw [color=c, fill=c] (13.7811,1.34957) rectangle (13.8209,1.45542);
\draw [color=c, fill=c] (13.8209,1.34957) rectangle (13.8607,1.45542);
\draw [color=c, fill=c] (13.8607,1.34957) rectangle (13.9005,1.45542);
\draw [color=c, fill=c] (13.9005,1.34957) rectangle (13.9403,1.45542);
\draw [color=c, fill=c] (13.9403,1.34957) rectangle (13.9801,1.45542);
\draw [color=c, fill=c] (13.9801,1.34957) rectangle (14.0199,1.45542);
\draw [color=c, fill=c] (14.0199,1.34957) rectangle (14.0597,1.45542);
\draw [color=c, fill=c] (14.0597,1.34957) rectangle (14.0995,1.45542);
\draw [color=c, fill=c] (14.0995,1.34957) rectangle (14.1393,1.45542);
\draw [color=c, fill=c] (14.1393,1.34957) rectangle (14.1791,1.45542);
\draw [color=c, fill=c] (14.1791,1.34957) rectangle (14.2189,1.45542);
\draw [color=c, fill=c] (14.2189,1.34957) rectangle (14.2587,1.45542);
\draw [color=c, fill=c] (14.2587,1.34957) rectangle (14.2985,1.45542);
\draw [color=c, fill=c] (14.2985,1.34957) rectangle (14.3383,1.45542);
\draw [color=c, fill=c] (14.3383,1.34957) rectangle (14.3781,1.45542);
\draw [color=c, fill=c] (14.3781,1.34957) rectangle (14.4179,1.45542);
\draw [color=c, fill=c] (14.4179,1.34957) rectangle (14.4577,1.45542);
\draw [color=c, fill=c] (14.4577,1.34957) rectangle (14.4975,1.45542);
\draw [color=c, fill=c] (14.4975,1.34957) rectangle (14.5373,1.45542);
\draw [color=c, fill=c] (14.5373,1.34957) rectangle (14.5771,1.45542);
\draw [color=c, fill=c] (14.5771,1.34957) rectangle (14.6169,1.45542);
\draw [color=c, fill=c] (14.6169,1.34957) rectangle (14.6567,1.45542);
\draw [color=c, fill=c] (14.6567,1.34957) rectangle (14.6965,1.45542);
\draw [color=c, fill=c] (14.6965,1.34957) rectangle (14.7363,1.45542);
\draw [color=c, fill=c] (14.7363,1.34957) rectangle (14.7761,1.45542);
\draw [color=c, fill=c] (14.7761,1.34957) rectangle (14.8159,1.45542);
\draw [color=c, fill=c] (14.8159,1.34957) rectangle (14.8557,1.45542);
\draw [color=c, fill=c] (14.8557,1.34957) rectangle (14.8955,1.45542);
\draw [color=c, fill=c] (14.8955,1.34957) rectangle (14.9353,1.45542);
\draw [color=c, fill=c] (14.9353,1.34957) rectangle (14.9751,1.45542);
\draw [color=c, fill=c] (14.9751,1.34957) rectangle (15.0149,1.45542);
\draw [color=c, fill=c] (15.0149,1.34957) rectangle (15.0547,1.45542);
\draw [color=c, fill=c] (15.0547,1.34957) rectangle (15.0945,1.45542);
\draw [color=c, fill=c] (15.0945,1.34957) rectangle (15.1343,1.45542);
\draw [color=c, fill=c] (15.1343,1.34957) rectangle (15.1741,1.45542);
\draw [color=c, fill=c] (15.1741,1.34957) rectangle (15.2139,1.45542);
\draw [color=c, fill=c] (15.2139,1.34957) rectangle (15.2537,1.45542);
\draw [color=c, fill=c] (15.2537,1.34957) rectangle (15.2935,1.45542);
\draw [color=c, fill=c] (15.2935,1.34957) rectangle (15.3333,1.45542);
\draw [color=c, fill=c] (15.3333,1.34957) rectangle (15.3731,1.45542);
\draw [color=c, fill=c] (15.3731,1.34957) rectangle (15.4129,1.45542);
\draw [color=c, fill=c] (15.4129,1.34957) rectangle (15.4527,1.45542);
\draw [color=c, fill=c] (15.4527,1.34957) rectangle (15.4925,1.45542);
\draw [color=c, fill=c] (15.4925,1.34957) rectangle (15.5323,1.45542);
\draw [color=c, fill=c] (15.5323,1.34957) rectangle (15.5721,1.45542);
\draw [color=c, fill=c] (15.5721,1.34957) rectangle (15.6119,1.45542);
\draw [color=c, fill=c] (15.6119,1.34957) rectangle (15.6517,1.45542);
\draw [color=c, fill=c] (15.6517,1.34957) rectangle (15.6915,1.45542);
\draw [color=c, fill=c] (15.6915,1.34957) rectangle (15.7313,1.45542);
\draw [color=c, fill=c] (15.7313,1.34957) rectangle (15.7711,1.45542);
\draw [color=c, fill=c] (15.7711,1.34957) rectangle (15.8109,1.45542);
\draw [color=c, fill=c] (15.8109,1.34957) rectangle (15.8507,1.45542);
\draw [color=c, fill=c] (15.8507,1.34957) rectangle (15.8905,1.45542);
\draw [color=c, fill=c] (15.8905,1.34957) rectangle (15.9303,1.45542);
\draw [color=c, fill=c] (15.9303,1.34957) rectangle (15.9701,1.45542);
\draw [color=c, fill=c] (15.9701,1.34957) rectangle (16.01,1.45542);
\draw [color=c, fill=c] (16.01,1.34957) rectangle (16.0498,1.45542);
\draw [color=c, fill=c] (16.0498,1.34957) rectangle (16.0896,1.45542);
\draw [color=c, fill=c] (16.0896,1.34957) rectangle (16.1294,1.45542);
\draw [color=c, fill=c] (16.1294,1.34957) rectangle (16.1692,1.45542);
\draw [color=c, fill=c] (16.1692,1.34957) rectangle (16.209,1.45542);
\draw [color=c, fill=c] (16.209,1.34957) rectangle (16.2488,1.45542);
\draw [color=c, fill=c] (16.2488,1.34957) rectangle (16.2886,1.45542);
\draw [color=c, fill=c] (16.2886,1.34957) rectangle (16.3284,1.45542);
\draw [color=c, fill=c] (16.3284,1.34957) rectangle (16.3682,1.45542);
\draw [color=c, fill=c] (16.3682,1.34957) rectangle (16.408,1.45542);
\draw [color=c, fill=c] (16.408,1.34957) rectangle (16.4478,1.45542);
\draw [color=c, fill=c] (16.4478,1.34957) rectangle (16.4876,1.45542);
\draw [color=c, fill=c] (16.4876,1.34957) rectangle (16.5274,1.45542);
\draw [color=c, fill=c] (16.5274,1.34957) rectangle (16.5672,1.45542);
\draw [color=c, fill=c] (16.5672,1.34957) rectangle (16.607,1.45542);
\draw [color=c, fill=c] (16.607,1.34957) rectangle (16.6468,1.45542);
\draw [color=c, fill=c] (16.6468,1.34957) rectangle (16.6866,1.45542);
\draw [color=c, fill=c] (16.6866,1.34957) rectangle (16.7264,1.45542);
\draw [color=c, fill=c] (16.7264,1.34957) rectangle (16.7662,1.45542);
\draw [color=c, fill=c] (16.7662,1.34957) rectangle (16.806,1.45542);
\draw [color=c, fill=c] (16.806,1.34957) rectangle (16.8458,1.45542);
\draw [color=c, fill=c] (16.8458,1.34957) rectangle (16.8856,1.45542);
\draw [color=c, fill=c] (16.8856,1.34957) rectangle (16.9254,1.45542);
\draw [color=c, fill=c] (16.9254,1.34957) rectangle (16.9652,1.45542);
\draw [color=c, fill=c] (16.9652,1.34957) rectangle (17.005,1.45542);
\draw [color=c, fill=c] (17.005,1.34957) rectangle (17.0448,1.45542);
\draw [color=c, fill=c] (17.0448,1.34957) rectangle (17.0846,1.45542);
\draw [color=c, fill=c] (17.0846,1.34957) rectangle (17.1244,1.45542);
\draw [color=c, fill=c] (17.1244,1.34957) rectangle (17.1642,1.45542);
\draw [color=c, fill=c] (17.1642,1.34957) rectangle (17.204,1.45542);
\draw [color=c, fill=c] (17.204,1.34957) rectangle (17.2438,1.45542);
\draw [color=c, fill=c] (17.2438,1.34957) rectangle (17.2836,1.45542);
\draw [color=c, fill=c] (17.2836,1.34957) rectangle (17.3234,1.45542);
\draw [color=c, fill=c] (17.3234,1.34957) rectangle (17.3632,1.45542);
\draw [color=c, fill=c] (17.3632,1.34957) rectangle (17.403,1.45542);
\draw [color=c, fill=c] (17.403,1.34957) rectangle (17.4428,1.45542);
\draw [color=c, fill=c] (17.4428,1.34957) rectangle (17.4826,1.45542);
\draw [color=c, fill=c] (17.4826,1.34957) rectangle (17.5224,1.45542);
\draw [color=c, fill=c] (17.5224,1.34957) rectangle (17.5622,1.45542);
\draw [color=c, fill=c] (17.5622,1.34957) rectangle (17.602,1.45542);
\draw [color=c, fill=c] (17.602,1.34957) rectangle (17.6418,1.45542);
\draw [color=c, fill=c] (17.6418,1.34957) rectangle (17.6816,1.45542);
\draw [color=c, fill=c] (17.6816,1.34957) rectangle (17.7214,1.45542);
\draw [color=c, fill=c] (17.7214,1.34957) rectangle (17.7612,1.45542);
\draw [color=c, fill=c] (17.7612,1.34957) rectangle (17.801,1.45542);
\draw [color=c, fill=c] (17.801,1.34957) rectangle (17.8408,1.45542);
\draw [color=c, fill=c] (17.8408,1.34957) rectangle (17.8806,1.45542);
\draw [color=c, fill=c] (17.8806,1.34957) rectangle (17.9204,1.45542);
\draw [color=c, fill=c] (17.9204,1.34957) rectangle (17.9602,1.45542);
\draw [color=c, fill=c] (17.9602,1.34957) rectangle (18,1.45542);
\definecolor{c}{rgb}{1,0,0};
\draw [color=c, fill=c] (2,1.45542) rectangle (2.0398,1.56127);
\draw [color=c, fill=c] (2.0398,1.45542) rectangle (2.0796,1.56127);
\draw [color=c, fill=c] (2.0796,1.45542) rectangle (2.1194,1.56127);
\draw [color=c, fill=c] (2.1194,1.45542) rectangle (2.1592,1.56127);
\draw [color=c, fill=c] (2.1592,1.45542) rectangle (2.19901,1.56127);
\draw [color=c, fill=c] (2.19901,1.45542) rectangle (2.23881,1.56127);
\draw [color=c, fill=c] (2.23881,1.45542) rectangle (2.27861,1.56127);
\draw [color=c, fill=c] (2.27861,1.45542) rectangle (2.31841,1.56127);
\draw [color=c, fill=c] (2.31841,1.45542) rectangle (2.35821,1.56127);
\draw [color=c, fill=c] (2.35821,1.45542) rectangle (2.39801,1.56127);
\draw [color=c, fill=c] (2.39801,1.45542) rectangle (2.43781,1.56127);
\draw [color=c, fill=c] (2.43781,1.45542) rectangle (2.47761,1.56127);
\draw [color=c, fill=c] (2.47761,1.45542) rectangle (2.51741,1.56127);
\draw [color=c, fill=c] (2.51741,1.45542) rectangle (2.55721,1.56127);
\draw [color=c, fill=c] (2.55721,1.45542) rectangle (2.59702,1.56127);
\draw [color=c, fill=c] (2.59702,1.45542) rectangle (2.63682,1.56127);
\draw [color=c, fill=c] (2.63682,1.45542) rectangle (2.67662,1.56127);
\draw [color=c, fill=c] (2.67662,1.45542) rectangle (2.71642,1.56127);
\draw [color=c, fill=c] (2.71642,1.45542) rectangle (2.75622,1.56127);
\draw [color=c, fill=c] (2.75622,1.45542) rectangle (2.79602,1.56127);
\draw [color=c, fill=c] (2.79602,1.45542) rectangle (2.83582,1.56127);
\draw [color=c, fill=c] (2.83582,1.45542) rectangle (2.87562,1.56127);
\draw [color=c, fill=c] (2.87562,1.45542) rectangle (2.91542,1.56127);
\draw [color=c, fill=c] (2.91542,1.45542) rectangle (2.95522,1.56127);
\draw [color=c, fill=c] (2.95522,1.45542) rectangle (2.99502,1.56127);
\draw [color=c, fill=c] (2.99502,1.45542) rectangle (3.03483,1.56127);
\draw [color=c, fill=c] (3.03483,1.45542) rectangle (3.07463,1.56127);
\draw [color=c, fill=c] (3.07463,1.45542) rectangle (3.11443,1.56127);
\draw [color=c, fill=c] (3.11443,1.45542) rectangle (3.15423,1.56127);
\draw [color=c, fill=c] (3.15423,1.45542) rectangle (3.19403,1.56127);
\draw [color=c, fill=c] (3.19403,1.45542) rectangle (3.23383,1.56127);
\draw [color=c, fill=c] (3.23383,1.45542) rectangle (3.27363,1.56127);
\draw [color=c, fill=c] (3.27363,1.45542) rectangle (3.31343,1.56127);
\draw [color=c, fill=c] (3.31343,1.45542) rectangle (3.35323,1.56127);
\draw [color=c, fill=c] (3.35323,1.45542) rectangle (3.39303,1.56127);
\draw [color=c, fill=c] (3.39303,1.45542) rectangle (3.43284,1.56127);
\draw [color=c, fill=c] (3.43284,1.45542) rectangle (3.47264,1.56127);
\draw [color=c, fill=c] (3.47264,1.45542) rectangle (3.51244,1.56127);
\draw [color=c, fill=c] (3.51244,1.45542) rectangle (3.55224,1.56127);
\draw [color=c, fill=c] (3.55224,1.45542) rectangle (3.59204,1.56127);
\draw [color=c, fill=c] (3.59204,1.45542) rectangle (3.63184,1.56127);
\draw [color=c, fill=c] (3.63184,1.45542) rectangle (3.67164,1.56127);
\draw [color=c, fill=c] (3.67164,1.45542) rectangle (3.71144,1.56127);
\draw [color=c, fill=c] (3.71144,1.45542) rectangle (3.75124,1.56127);
\draw [color=c, fill=c] (3.75124,1.45542) rectangle (3.79104,1.56127);
\draw [color=c, fill=c] (3.79104,1.45542) rectangle (3.83085,1.56127);
\draw [color=c, fill=c] (3.83085,1.45542) rectangle (3.87065,1.56127);
\draw [color=c, fill=c] (3.87065,1.45542) rectangle (3.91045,1.56127);
\draw [color=c, fill=c] (3.91045,1.45542) rectangle (3.95025,1.56127);
\draw [color=c, fill=c] (3.95025,1.45542) rectangle (3.99005,1.56127);
\draw [color=c, fill=c] (3.99005,1.45542) rectangle (4.02985,1.56127);
\draw [color=c, fill=c] (4.02985,1.45542) rectangle (4.06965,1.56127);
\draw [color=c, fill=c] (4.06965,1.45542) rectangle (4.10945,1.56127);
\draw [color=c, fill=c] (4.10945,1.45542) rectangle (4.14925,1.56127);
\draw [color=c, fill=c] (4.14925,1.45542) rectangle (4.18905,1.56127);
\draw [color=c, fill=c] (4.18905,1.45542) rectangle (4.22886,1.56127);
\draw [color=c, fill=c] (4.22886,1.45542) rectangle (4.26866,1.56127);
\draw [color=c, fill=c] (4.26866,1.45542) rectangle (4.30846,1.56127);
\draw [color=c, fill=c] (4.30846,1.45542) rectangle (4.34826,1.56127);
\draw [color=c, fill=c] (4.34826,1.45542) rectangle (4.38806,1.56127);
\draw [color=c, fill=c] (4.38806,1.45542) rectangle (4.42786,1.56127);
\draw [color=c, fill=c] (4.42786,1.45542) rectangle (4.46766,1.56127);
\draw [color=c, fill=c] (4.46766,1.45542) rectangle (4.50746,1.56127);
\draw [color=c, fill=c] (4.50746,1.45542) rectangle (4.54726,1.56127);
\draw [color=c, fill=c] (4.54726,1.45542) rectangle (4.58706,1.56127);
\draw [color=c, fill=c] (4.58706,1.45542) rectangle (4.62687,1.56127);
\draw [color=c, fill=c] (4.62687,1.45542) rectangle (4.66667,1.56127);
\draw [color=c, fill=c] (4.66667,1.45542) rectangle (4.70647,1.56127);
\draw [color=c, fill=c] (4.70647,1.45542) rectangle (4.74627,1.56127);
\draw [color=c, fill=c] (4.74627,1.45542) rectangle (4.78607,1.56127);
\draw [color=c, fill=c] (4.78607,1.45542) rectangle (4.82587,1.56127);
\draw [color=c, fill=c] (4.82587,1.45542) rectangle (4.86567,1.56127);
\draw [color=c, fill=c] (4.86567,1.45542) rectangle (4.90547,1.56127);
\draw [color=c, fill=c] (4.90547,1.45542) rectangle (4.94527,1.56127);
\draw [color=c, fill=c] (4.94527,1.45542) rectangle (4.98507,1.56127);
\draw [color=c, fill=c] (4.98507,1.45542) rectangle (5.02488,1.56127);
\draw [color=c, fill=c] (5.02488,1.45542) rectangle (5.06468,1.56127);
\draw [color=c, fill=c] (5.06468,1.45542) rectangle (5.10448,1.56127);
\draw [color=c, fill=c] (5.10448,1.45542) rectangle (5.14428,1.56127);
\draw [color=c, fill=c] (5.14428,1.45542) rectangle (5.18408,1.56127);
\draw [color=c, fill=c] (5.18408,1.45542) rectangle (5.22388,1.56127);
\draw [color=c, fill=c] (5.22388,1.45542) rectangle (5.26368,1.56127);
\draw [color=c, fill=c] (5.26368,1.45542) rectangle (5.30348,1.56127);
\draw [color=c, fill=c] (5.30348,1.45542) rectangle (5.34328,1.56127);
\draw [color=c, fill=c] (5.34328,1.45542) rectangle (5.38308,1.56127);
\draw [color=c, fill=c] (5.38308,1.45542) rectangle (5.42289,1.56127);
\draw [color=c, fill=c] (5.42289,1.45542) rectangle (5.46269,1.56127);
\draw [color=c, fill=c] (5.46269,1.45542) rectangle (5.50249,1.56127);
\draw [color=c, fill=c] (5.50249,1.45542) rectangle (5.54229,1.56127);
\draw [color=c, fill=c] (5.54229,1.45542) rectangle (5.58209,1.56127);
\draw [color=c, fill=c] (5.58209,1.45542) rectangle (5.62189,1.56127);
\draw [color=c, fill=c] (5.62189,1.45542) rectangle (5.66169,1.56127);
\draw [color=c, fill=c] (5.66169,1.45542) rectangle (5.70149,1.56127);
\draw [color=c, fill=c] (5.70149,1.45542) rectangle (5.74129,1.56127);
\draw [color=c, fill=c] (5.74129,1.45542) rectangle (5.78109,1.56127);
\draw [color=c, fill=c] (5.78109,1.45542) rectangle (5.8209,1.56127);
\draw [color=c, fill=c] (5.8209,1.45542) rectangle (5.8607,1.56127);
\draw [color=c, fill=c] (5.8607,1.45542) rectangle (5.9005,1.56127);
\draw [color=c, fill=c] (5.9005,1.45542) rectangle (5.9403,1.56127);
\draw [color=c, fill=c] (5.9403,1.45542) rectangle (5.9801,1.56127);
\draw [color=c, fill=c] (5.9801,1.45542) rectangle (6.0199,1.56127);
\draw [color=c, fill=c] (6.0199,1.45542) rectangle (6.0597,1.56127);
\draw [color=c, fill=c] (6.0597,1.45542) rectangle (6.0995,1.56127);
\draw [color=c, fill=c] (6.0995,1.45542) rectangle (6.1393,1.56127);
\draw [color=c, fill=c] (6.1393,1.45542) rectangle (6.1791,1.56127);
\draw [color=c, fill=c] (6.1791,1.45542) rectangle (6.21891,1.56127);
\draw [color=c, fill=c] (6.21891,1.45542) rectangle (6.25871,1.56127);
\draw [color=c, fill=c] (6.25871,1.45542) rectangle (6.29851,1.56127);
\draw [color=c, fill=c] (6.29851,1.45542) rectangle (6.33831,1.56127);
\draw [color=c, fill=c] (6.33831,1.45542) rectangle (6.37811,1.56127);
\draw [color=c, fill=c] (6.37811,1.45542) rectangle (6.41791,1.56127);
\draw [color=c, fill=c] (6.41791,1.45542) rectangle (6.45771,1.56127);
\draw [color=c, fill=c] (6.45771,1.45542) rectangle (6.49751,1.56127);
\draw [color=c, fill=c] (6.49751,1.45542) rectangle (6.53731,1.56127);
\draw [color=c, fill=c] (6.53731,1.45542) rectangle (6.57711,1.56127);
\draw [color=c, fill=c] (6.57711,1.45542) rectangle (6.61692,1.56127);
\draw [color=c, fill=c] (6.61692,1.45542) rectangle (6.65672,1.56127);
\draw [color=c, fill=c] (6.65672,1.45542) rectangle (6.69652,1.56127);
\draw [color=c, fill=c] (6.69652,1.45542) rectangle (6.73632,1.56127);
\draw [color=c, fill=c] (6.73632,1.45542) rectangle (6.77612,1.56127);
\draw [color=c, fill=c] (6.77612,1.45542) rectangle (6.81592,1.56127);
\draw [color=c, fill=c] (6.81592,1.45542) rectangle (6.85572,1.56127);
\draw [color=c, fill=c] (6.85572,1.45542) rectangle (6.89552,1.56127);
\draw [color=c, fill=c] (6.89552,1.45542) rectangle (6.93532,1.56127);
\draw [color=c, fill=c] (6.93532,1.45542) rectangle (6.97512,1.56127);
\draw [color=c, fill=c] (6.97512,1.45542) rectangle (7.01493,1.56127);
\draw [color=c, fill=c] (7.01493,1.45542) rectangle (7.05473,1.56127);
\draw [color=c, fill=c] (7.05473,1.45542) rectangle (7.09453,1.56127);
\draw [color=c, fill=c] (7.09453,1.45542) rectangle (7.13433,1.56127);
\draw [color=c, fill=c] (7.13433,1.45542) rectangle (7.17413,1.56127);
\draw [color=c, fill=c] (7.17413,1.45542) rectangle (7.21393,1.56127);
\draw [color=c, fill=c] (7.21393,1.45542) rectangle (7.25373,1.56127);
\draw [color=c, fill=c] (7.25373,1.45542) rectangle (7.29353,1.56127);
\draw [color=c, fill=c] (7.29353,1.45542) rectangle (7.33333,1.56127);
\draw [color=c, fill=c] (7.33333,1.45542) rectangle (7.37313,1.56127);
\draw [color=c, fill=c] (7.37313,1.45542) rectangle (7.41294,1.56127);
\draw [color=c, fill=c] (7.41294,1.45542) rectangle (7.45274,1.56127);
\draw [color=c, fill=c] (7.45274,1.45542) rectangle (7.49254,1.56127);
\draw [color=c, fill=c] (7.49254,1.45542) rectangle (7.53234,1.56127);
\draw [color=c, fill=c] (7.53234,1.45542) rectangle (7.57214,1.56127);
\draw [color=c, fill=c] (7.57214,1.45542) rectangle (7.61194,1.56127);
\draw [color=c, fill=c] (7.61194,1.45542) rectangle (7.65174,1.56127);
\draw [color=c, fill=c] (7.65174,1.45542) rectangle (7.69154,1.56127);
\draw [color=c, fill=c] (7.69154,1.45542) rectangle (7.73134,1.56127);
\draw [color=c, fill=c] (7.73134,1.45542) rectangle (7.77114,1.56127);
\definecolor{c}{rgb}{1,0.186667,0};
\draw [color=c, fill=c] (7.77114,1.45542) rectangle (7.81095,1.56127);
\draw [color=c, fill=c] (7.81095,1.45542) rectangle (7.85075,1.56127);
\draw [color=c, fill=c] (7.85075,1.45542) rectangle (7.89055,1.56127);
\draw [color=c, fill=c] (7.89055,1.45542) rectangle (7.93035,1.56127);
\draw [color=c, fill=c] (7.93035,1.45542) rectangle (7.97015,1.56127);
\draw [color=c, fill=c] (7.97015,1.45542) rectangle (8.00995,1.56127);
\draw [color=c, fill=c] (8.00995,1.45542) rectangle (8.04975,1.56127);
\draw [color=c, fill=c] (8.04975,1.45542) rectangle (8.08955,1.56127);
\draw [color=c, fill=c] (8.08955,1.45542) rectangle (8.12935,1.56127);
\draw [color=c, fill=c] (8.12935,1.45542) rectangle (8.16915,1.56127);
\draw [color=c, fill=c] (8.16915,1.45542) rectangle (8.20895,1.56127);
\draw [color=c, fill=c] (8.20895,1.45542) rectangle (8.24876,1.56127);
\draw [color=c, fill=c] (8.24876,1.45542) rectangle (8.28856,1.56127);
\draw [color=c, fill=c] (8.28856,1.45542) rectangle (8.32836,1.56127);
\draw [color=c, fill=c] (8.32836,1.45542) rectangle (8.36816,1.56127);
\draw [color=c, fill=c] (8.36816,1.45542) rectangle (8.40796,1.56127);
\draw [color=c, fill=c] (8.40796,1.45542) rectangle (8.44776,1.56127);
\draw [color=c, fill=c] (8.44776,1.45542) rectangle (8.48756,1.56127);
\draw [color=c, fill=c] (8.48756,1.45542) rectangle (8.52736,1.56127);
\draw [color=c, fill=c] (8.52736,1.45542) rectangle (8.56716,1.56127);
\draw [color=c, fill=c] (8.56716,1.45542) rectangle (8.60697,1.56127);
\definecolor{c}{rgb}{1,0.466667,0};
\draw [color=c, fill=c] (8.60697,1.45542) rectangle (8.64677,1.56127);
\draw [color=c, fill=c] (8.64677,1.45542) rectangle (8.68657,1.56127);
\draw [color=c, fill=c] (8.68657,1.45542) rectangle (8.72637,1.56127);
\draw [color=c, fill=c] (8.72637,1.45542) rectangle (8.76617,1.56127);
\draw [color=c, fill=c] (8.76617,1.45542) rectangle (8.80597,1.56127);
\draw [color=c, fill=c] (8.80597,1.45542) rectangle (8.84577,1.56127);
\draw [color=c, fill=c] (8.84577,1.45542) rectangle (8.88557,1.56127);
\draw [color=c, fill=c] (8.88557,1.45542) rectangle (8.92537,1.56127);
\draw [color=c, fill=c] (8.92537,1.45542) rectangle (8.96517,1.56127);
\draw [color=c, fill=c] (8.96517,1.45542) rectangle (9.00498,1.56127);
\draw [color=c, fill=c] (9.00498,1.45542) rectangle (9.04478,1.56127);
\definecolor{c}{rgb}{1,0.653333,0};
\draw [color=c, fill=c] (9.04478,1.45542) rectangle (9.08458,1.56127);
\draw [color=c, fill=c] (9.08458,1.45542) rectangle (9.12438,1.56127);
\draw [color=c, fill=c] (9.12438,1.45542) rectangle (9.16418,1.56127);
\draw [color=c, fill=c] (9.16418,1.45542) rectangle (9.20398,1.56127);
\draw [color=c, fill=c] (9.20398,1.45542) rectangle (9.24378,1.56127);
\draw [color=c, fill=c] (9.24378,1.45542) rectangle (9.28358,1.56127);
\draw [color=c, fill=c] (9.28358,1.45542) rectangle (9.32338,1.56127);
\definecolor{c}{rgb}{1,0.933333,0};
\draw [color=c, fill=c] (9.32338,1.45542) rectangle (9.36318,1.56127);
\draw [color=c, fill=c] (9.36318,1.45542) rectangle (9.40298,1.56127);
\draw [color=c, fill=c] (9.40298,1.45542) rectangle (9.44279,1.56127);
\draw [color=c, fill=c] (9.44279,1.45542) rectangle (9.48259,1.56127);
\draw [color=c, fill=c] (9.48259,1.45542) rectangle (9.52239,1.56127);
\definecolor{c}{rgb}{0.88,1,0};
\draw [color=c, fill=c] (9.52239,1.45542) rectangle (9.56219,1.56127);
\draw [color=c, fill=c] (9.56219,1.45542) rectangle (9.60199,1.56127);
\draw [color=c, fill=c] (9.60199,1.45542) rectangle (9.64179,1.56127);
\draw [color=c, fill=c] (9.64179,1.45542) rectangle (9.68159,1.56127);
\draw [color=c, fill=c] (9.68159,1.45542) rectangle (9.72139,1.56127);
\definecolor{c}{rgb}{0.6,1,0};
\draw [color=c, fill=c] (9.72139,1.45542) rectangle (9.76119,1.56127);
\draw [color=c, fill=c] (9.76119,1.45542) rectangle (9.80099,1.56127);
\draw [color=c, fill=c] (9.80099,1.45542) rectangle (9.8408,1.56127);
\draw [color=c, fill=c] (9.8408,1.45542) rectangle (9.8806,1.56127);
\definecolor{c}{rgb}{0.413333,1,0};
\draw [color=c, fill=c] (9.8806,1.45542) rectangle (9.9204,1.56127);
\draw [color=c, fill=c] (9.9204,1.45542) rectangle (9.9602,1.56127);
\draw [color=c, fill=c] (9.9602,1.45542) rectangle (10,1.56127);
\draw [color=c, fill=c] (10,1.45542) rectangle (10.0398,1.56127);
\definecolor{c}{rgb}{0.133333,1,0};
\draw [color=c, fill=c] (10.0398,1.45542) rectangle (10.0796,1.56127);
\draw [color=c, fill=c] (10.0796,1.45542) rectangle (10.1194,1.56127);
\draw [color=c, fill=c] (10.1194,1.45542) rectangle (10.1592,1.56127);
\draw [color=c, fill=c] (10.1592,1.45542) rectangle (10.199,1.56127);
\draw [color=c, fill=c] (10.199,1.45542) rectangle (10.2388,1.56127);
\definecolor{c}{rgb}{0,1,0.0533333};
\draw [color=c, fill=c] (10.2388,1.45542) rectangle (10.2786,1.56127);
\draw [color=c, fill=c] (10.2786,1.45542) rectangle (10.3184,1.56127);
\draw [color=c, fill=c] (10.3184,1.45542) rectangle (10.3582,1.56127);
\draw [color=c, fill=c] (10.3582,1.45542) rectangle (10.398,1.56127);
\definecolor{c}{rgb}{0,1,0.333333};
\draw [color=c, fill=c] (10.398,1.45542) rectangle (10.4378,1.56127);
\draw [color=c, fill=c] (10.4378,1.45542) rectangle (10.4776,1.56127);
\draw [color=c, fill=c] (10.4776,1.45542) rectangle (10.5174,1.56127);
\draw [color=c, fill=c] (10.5174,1.45542) rectangle (10.5572,1.56127);
\draw [color=c, fill=c] (10.5572,1.45542) rectangle (10.597,1.56127);
\draw [color=c, fill=c] (10.597,1.45542) rectangle (10.6368,1.56127);
\definecolor{c}{rgb}{0,1,0.52};
\draw [color=c, fill=c] (10.6368,1.45542) rectangle (10.6766,1.56127);
\draw [color=c, fill=c] (10.6766,1.45542) rectangle (10.7164,1.56127);
\draw [color=c, fill=c] (10.7164,1.45542) rectangle (10.7562,1.56127);
\draw [color=c, fill=c] (10.7562,1.45542) rectangle (10.796,1.56127);
\draw [color=c, fill=c] (10.796,1.45542) rectangle (10.8358,1.56127);
\draw [color=c, fill=c] (10.8358,1.45542) rectangle (10.8756,1.56127);
\draw [color=c, fill=c] (10.8756,1.45542) rectangle (10.9154,1.56127);
\draw [color=c, fill=c] (10.9154,1.45542) rectangle (10.9552,1.56127);
\definecolor{c}{rgb}{0,1,0.8};
\draw [color=c, fill=c] (10.9552,1.45542) rectangle (10.995,1.56127);
\draw [color=c, fill=c] (10.995,1.45542) rectangle (11.0348,1.56127);
\draw [color=c, fill=c] (11.0348,1.45542) rectangle (11.0746,1.56127);
\draw [color=c, fill=c] (11.0746,1.45542) rectangle (11.1144,1.56127);
\draw [color=c, fill=c] (11.1144,1.45542) rectangle (11.1542,1.56127);
\draw [color=c, fill=c] (11.1542,1.45542) rectangle (11.194,1.56127);
\draw [color=c, fill=c] (11.194,1.45542) rectangle (11.2338,1.56127);
\draw [color=c, fill=c] (11.2338,1.45542) rectangle (11.2736,1.56127);
\draw [color=c, fill=c] (11.2736,1.45542) rectangle (11.3134,1.56127);
\draw [color=c, fill=c] (11.3134,1.45542) rectangle (11.3532,1.56127);
\draw [color=c, fill=c] (11.3532,1.45542) rectangle (11.393,1.56127);
\draw [color=c, fill=c] (11.393,1.45542) rectangle (11.4328,1.56127);
\definecolor{c}{rgb}{0,1,0.986667};
\draw [color=c, fill=c] (11.4328,1.45542) rectangle (11.4726,1.56127);
\draw [color=c, fill=c] (11.4726,1.45542) rectangle (11.5124,1.56127);
\draw [color=c, fill=c] (11.5124,1.45542) rectangle (11.5522,1.56127);
\draw [color=c, fill=c] (11.5522,1.45542) rectangle (11.592,1.56127);
\draw [color=c, fill=c] (11.592,1.45542) rectangle (11.6318,1.56127);
\draw [color=c, fill=c] (11.6318,1.45542) rectangle (11.6716,1.56127);
\draw [color=c, fill=c] (11.6716,1.45542) rectangle (11.7114,1.56127);
\draw [color=c, fill=c] (11.7114,1.45542) rectangle (11.7512,1.56127);
\draw [color=c, fill=c] (11.7512,1.45542) rectangle (11.791,1.56127);
\draw [color=c, fill=c] (11.791,1.45542) rectangle (11.8308,1.56127);
\draw [color=c, fill=c] (11.8308,1.45542) rectangle (11.8706,1.56127);
\draw [color=c, fill=c] (11.8706,1.45542) rectangle (11.9104,1.56127);
\draw [color=c, fill=c] (11.9104,1.45542) rectangle (11.9502,1.56127);
\draw [color=c, fill=c] (11.9502,1.45542) rectangle (11.99,1.56127);
\draw [color=c, fill=c] (11.99,1.45542) rectangle (12.0299,1.56127);
\draw [color=c, fill=c] (12.0299,1.45542) rectangle (12.0697,1.56127);
\draw [color=c, fill=c] (12.0697,1.45542) rectangle (12.1095,1.56127);
\draw [color=c, fill=c] (12.1095,1.45542) rectangle (12.1493,1.56127);
\draw [color=c, fill=c] (12.1493,1.45542) rectangle (12.1891,1.56127);
\draw [color=c, fill=c] (12.1891,1.45542) rectangle (12.2289,1.56127);
\draw [color=c, fill=c] (12.2289,1.45542) rectangle (12.2687,1.56127);
\draw [color=c, fill=c] (12.2687,1.45542) rectangle (12.3085,1.56127);
\draw [color=c, fill=c] (12.3085,1.45542) rectangle (12.3483,1.56127);
\definecolor{c}{rgb}{0,0.733333,1};
\draw [color=c, fill=c] (12.3483,1.45542) rectangle (12.3881,1.56127);
\draw [color=c, fill=c] (12.3881,1.45542) rectangle (12.4279,1.56127);
\draw [color=c, fill=c] (12.4279,1.45542) rectangle (12.4677,1.56127);
\draw [color=c, fill=c] (12.4677,1.45542) rectangle (12.5075,1.56127);
\draw [color=c, fill=c] (12.5075,1.45542) rectangle (12.5473,1.56127);
\draw [color=c, fill=c] (12.5473,1.45542) rectangle (12.5871,1.56127);
\draw [color=c, fill=c] (12.5871,1.45542) rectangle (12.6269,1.56127);
\draw [color=c, fill=c] (12.6269,1.45542) rectangle (12.6667,1.56127);
\draw [color=c, fill=c] (12.6667,1.45542) rectangle (12.7065,1.56127);
\draw [color=c, fill=c] (12.7065,1.45542) rectangle (12.7463,1.56127);
\draw [color=c, fill=c] (12.7463,1.45542) rectangle (12.7861,1.56127);
\draw [color=c, fill=c] (12.7861,1.45542) rectangle (12.8259,1.56127);
\draw [color=c, fill=c] (12.8259,1.45542) rectangle (12.8657,1.56127);
\draw [color=c, fill=c] (12.8657,1.45542) rectangle (12.9055,1.56127);
\draw [color=c, fill=c] (12.9055,1.45542) rectangle (12.9453,1.56127);
\draw [color=c, fill=c] (12.9453,1.45542) rectangle (12.9851,1.56127);
\draw [color=c, fill=c] (12.9851,1.45542) rectangle (13.0249,1.56127);
\draw [color=c, fill=c] (13.0249,1.45542) rectangle (13.0647,1.56127);
\draw [color=c, fill=c] (13.0647,1.45542) rectangle (13.1045,1.56127);
\draw [color=c, fill=c] (13.1045,1.45542) rectangle (13.1443,1.56127);
\draw [color=c, fill=c] (13.1443,1.45542) rectangle (13.1841,1.56127);
\draw [color=c, fill=c] (13.1841,1.45542) rectangle (13.2239,1.56127);
\draw [color=c, fill=c] (13.2239,1.45542) rectangle (13.2637,1.56127);
\draw [color=c, fill=c] (13.2637,1.45542) rectangle (13.3035,1.56127);
\draw [color=c, fill=c] (13.3035,1.45542) rectangle (13.3433,1.56127);
\draw [color=c, fill=c] (13.3433,1.45542) rectangle (13.3831,1.56127);
\draw [color=c, fill=c] (13.3831,1.45542) rectangle (13.4229,1.56127);
\draw [color=c, fill=c] (13.4229,1.45542) rectangle (13.4627,1.56127);
\draw [color=c, fill=c] (13.4627,1.45542) rectangle (13.5025,1.56127);
\draw [color=c, fill=c] (13.5025,1.45542) rectangle (13.5423,1.56127);
\draw [color=c, fill=c] (13.5423,1.45542) rectangle (13.5821,1.56127);
\draw [color=c, fill=c] (13.5821,1.45542) rectangle (13.6219,1.56127);
\draw [color=c, fill=c] (13.6219,1.45542) rectangle (13.6617,1.56127);
\draw [color=c, fill=c] (13.6617,1.45542) rectangle (13.7015,1.56127);
\draw [color=c, fill=c] (13.7015,1.45542) rectangle (13.7413,1.56127);
\draw [color=c, fill=c] (13.7413,1.45542) rectangle (13.7811,1.56127);
\draw [color=c, fill=c] (13.7811,1.45542) rectangle (13.8209,1.56127);
\draw [color=c, fill=c] (13.8209,1.45542) rectangle (13.8607,1.56127);
\draw [color=c, fill=c] (13.8607,1.45542) rectangle (13.9005,1.56127);
\draw [color=c, fill=c] (13.9005,1.45542) rectangle (13.9403,1.56127);
\draw [color=c, fill=c] (13.9403,1.45542) rectangle (13.9801,1.56127);
\draw [color=c, fill=c] (13.9801,1.45542) rectangle (14.0199,1.56127);
\draw [color=c, fill=c] (14.0199,1.45542) rectangle (14.0597,1.56127);
\draw [color=c, fill=c] (14.0597,1.45542) rectangle (14.0995,1.56127);
\draw [color=c, fill=c] (14.0995,1.45542) rectangle (14.1393,1.56127);
\draw [color=c, fill=c] (14.1393,1.45542) rectangle (14.1791,1.56127);
\draw [color=c, fill=c] (14.1791,1.45542) rectangle (14.2189,1.56127);
\draw [color=c, fill=c] (14.2189,1.45542) rectangle (14.2587,1.56127);
\draw [color=c, fill=c] (14.2587,1.45542) rectangle (14.2985,1.56127);
\draw [color=c, fill=c] (14.2985,1.45542) rectangle (14.3383,1.56127);
\draw [color=c, fill=c] (14.3383,1.45542) rectangle (14.3781,1.56127);
\draw [color=c, fill=c] (14.3781,1.45542) rectangle (14.4179,1.56127);
\draw [color=c, fill=c] (14.4179,1.45542) rectangle (14.4577,1.56127);
\draw [color=c, fill=c] (14.4577,1.45542) rectangle (14.4975,1.56127);
\draw [color=c, fill=c] (14.4975,1.45542) rectangle (14.5373,1.56127);
\draw [color=c, fill=c] (14.5373,1.45542) rectangle (14.5771,1.56127);
\draw [color=c, fill=c] (14.5771,1.45542) rectangle (14.6169,1.56127);
\draw [color=c, fill=c] (14.6169,1.45542) rectangle (14.6567,1.56127);
\draw [color=c, fill=c] (14.6567,1.45542) rectangle (14.6965,1.56127);
\draw [color=c, fill=c] (14.6965,1.45542) rectangle (14.7363,1.56127);
\draw [color=c, fill=c] (14.7363,1.45542) rectangle (14.7761,1.56127);
\draw [color=c, fill=c] (14.7761,1.45542) rectangle (14.8159,1.56127);
\draw [color=c, fill=c] (14.8159,1.45542) rectangle (14.8557,1.56127);
\draw [color=c, fill=c] (14.8557,1.45542) rectangle (14.8955,1.56127);
\draw [color=c, fill=c] (14.8955,1.45542) rectangle (14.9353,1.56127);
\draw [color=c, fill=c] (14.9353,1.45542) rectangle (14.9751,1.56127);
\draw [color=c, fill=c] (14.9751,1.45542) rectangle (15.0149,1.56127);
\draw [color=c, fill=c] (15.0149,1.45542) rectangle (15.0547,1.56127);
\draw [color=c, fill=c] (15.0547,1.45542) rectangle (15.0945,1.56127);
\draw [color=c, fill=c] (15.0945,1.45542) rectangle (15.1343,1.56127);
\draw [color=c, fill=c] (15.1343,1.45542) rectangle (15.1741,1.56127);
\draw [color=c, fill=c] (15.1741,1.45542) rectangle (15.2139,1.56127);
\draw [color=c, fill=c] (15.2139,1.45542) rectangle (15.2537,1.56127);
\draw [color=c, fill=c] (15.2537,1.45542) rectangle (15.2935,1.56127);
\draw [color=c, fill=c] (15.2935,1.45542) rectangle (15.3333,1.56127);
\draw [color=c, fill=c] (15.3333,1.45542) rectangle (15.3731,1.56127);
\draw [color=c, fill=c] (15.3731,1.45542) rectangle (15.4129,1.56127);
\draw [color=c, fill=c] (15.4129,1.45542) rectangle (15.4527,1.56127);
\draw [color=c, fill=c] (15.4527,1.45542) rectangle (15.4925,1.56127);
\draw [color=c, fill=c] (15.4925,1.45542) rectangle (15.5323,1.56127);
\draw [color=c, fill=c] (15.5323,1.45542) rectangle (15.5721,1.56127);
\draw [color=c, fill=c] (15.5721,1.45542) rectangle (15.6119,1.56127);
\draw [color=c, fill=c] (15.6119,1.45542) rectangle (15.6517,1.56127);
\draw [color=c, fill=c] (15.6517,1.45542) rectangle (15.6915,1.56127);
\draw [color=c, fill=c] (15.6915,1.45542) rectangle (15.7313,1.56127);
\draw [color=c, fill=c] (15.7313,1.45542) rectangle (15.7711,1.56127);
\draw [color=c, fill=c] (15.7711,1.45542) rectangle (15.8109,1.56127);
\draw [color=c, fill=c] (15.8109,1.45542) rectangle (15.8507,1.56127);
\draw [color=c, fill=c] (15.8507,1.45542) rectangle (15.8905,1.56127);
\draw [color=c, fill=c] (15.8905,1.45542) rectangle (15.9303,1.56127);
\draw [color=c, fill=c] (15.9303,1.45542) rectangle (15.9701,1.56127);
\draw [color=c, fill=c] (15.9701,1.45542) rectangle (16.01,1.56127);
\draw [color=c, fill=c] (16.01,1.45542) rectangle (16.0498,1.56127);
\draw [color=c, fill=c] (16.0498,1.45542) rectangle (16.0896,1.56127);
\draw [color=c, fill=c] (16.0896,1.45542) rectangle (16.1294,1.56127);
\draw [color=c, fill=c] (16.1294,1.45542) rectangle (16.1692,1.56127);
\draw [color=c, fill=c] (16.1692,1.45542) rectangle (16.209,1.56127);
\draw [color=c, fill=c] (16.209,1.45542) rectangle (16.2488,1.56127);
\draw [color=c, fill=c] (16.2488,1.45542) rectangle (16.2886,1.56127);
\draw [color=c, fill=c] (16.2886,1.45542) rectangle (16.3284,1.56127);
\draw [color=c, fill=c] (16.3284,1.45542) rectangle (16.3682,1.56127);
\draw [color=c, fill=c] (16.3682,1.45542) rectangle (16.408,1.56127);
\draw [color=c, fill=c] (16.408,1.45542) rectangle (16.4478,1.56127);
\draw [color=c, fill=c] (16.4478,1.45542) rectangle (16.4876,1.56127);
\draw [color=c, fill=c] (16.4876,1.45542) rectangle (16.5274,1.56127);
\draw [color=c, fill=c] (16.5274,1.45542) rectangle (16.5672,1.56127);
\draw [color=c, fill=c] (16.5672,1.45542) rectangle (16.607,1.56127);
\draw [color=c, fill=c] (16.607,1.45542) rectangle (16.6468,1.56127);
\draw [color=c, fill=c] (16.6468,1.45542) rectangle (16.6866,1.56127);
\draw [color=c, fill=c] (16.6866,1.45542) rectangle (16.7264,1.56127);
\draw [color=c, fill=c] (16.7264,1.45542) rectangle (16.7662,1.56127);
\draw [color=c, fill=c] (16.7662,1.45542) rectangle (16.806,1.56127);
\draw [color=c, fill=c] (16.806,1.45542) rectangle (16.8458,1.56127);
\draw [color=c, fill=c] (16.8458,1.45542) rectangle (16.8856,1.56127);
\draw [color=c, fill=c] (16.8856,1.45542) rectangle (16.9254,1.56127);
\draw [color=c, fill=c] (16.9254,1.45542) rectangle (16.9652,1.56127);
\draw [color=c, fill=c] (16.9652,1.45542) rectangle (17.005,1.56127);
\draw [color=c, fill=c] (17.005,1.45542) rectangle (17.0448,1.56127);
\draw [color=c, fill=c] (17.0448,1.45542) rectangle (17.0846,1.56127);
\draw [color=c, fill=c] (17.0846,1.45542) rectangle (17.1244,1.56127);
\draw [color=c, fill=c] (17.1244,1.45542) rectangle (17.1642,1.56127);
\draw [color=c, fill=c] (17.1642,1.45542) rectangle (17.204,1.56127);
\draw [color=c, fill=c] (17.204,1.45542) rectangle (17.2438,1.56127);
\draw [color=c, fill=c] (17.2438,1.45542) rectangle (17.2836,1.56127);
\draw [color=c, fill=c] (17.2836,1.45542) rectangle (17.3234,1.56127);
\draw [color=c, fill=c] (17.3234,1.45542) rectangle (17.3632,1.56127);
\draw [color=c, fill=c] (17.3632,1.45542) rectangle (17.403,1.56127);
\draw [color=c, fill=c] (17.403,1.45542) rectangle (17.4428,1.56127);
\draw [color=c, fill=c] (17.4428,1.45542) rectangle (17.4826,1.56127);
\draw [color=c, fill=c] (17.4826,1.45542) rectangle (17.5224,1.56127);
\draw [color=c, fill=c] (17.5224,1.45542) rectangle (17.5622,1.56127);
\draw [color=c, fill=c] (17.5622,1.45542) rectangle (17.602,1.56127);
\draw [color=c, fill=c] (17.602,1.45542) rectangle (17.6418,1.56127);
\draw [color=c, fill=c] (17.6418,1.45542) rectangle (17.6816,1.56127);
\draw [color=c, fill=c] (17.6816,1.45542) rectangle (17.7214,1.56127);
\draw [color=c, fill=c] (17.7214,1.45542) rectangle (17.7612,1.56127);
\draw [color=c, fill=c] (17.7612,1.45542) rectangle (17.801,1.56127);
\draw [color=c, fill=c] (17.801,1.45542) rectangle (17.8408,1.56127);
\draw [color=c, fill=c] (17.8408,1.45542) rectangle (17.8806,1.56127);
\draw [color=c, fill=c] (17.8806,1.45542) rectangle (17.9204,1.56127);
\draw [color=c, fill=c] (17.9204,1.45542) rectangle (17.9602,1.56127);
\draw [color=c, fill=c] (17.9602,1.45542) rectangle (18,1.56127);
\definecolor{c}{rgb}{1,0,0};
\draw [color=c, fill=c] (2,1.56127) rectangle (2.0398,1.66712);
\draw [color=c, fill=c] (2.0398,1.56127) rectangle (2.0796,1.66712);
\draw [color=c, fill=c] (2.0796,1.56127) rectangle (2.1194,1.66712);
\draw [color=c, fill=c] (2.1194,1.56127) rectangle (2.1592,1.66712);
\draw [color=c, fill=c] (2.1592,1.56127) rectangle (2.19901,1.66712);
\draw [color=c, fill=c] (2.19901,1.56127) rectangle (2.23881,1.66712);
\draw [color=c, fill=c] (2.23881,1.56127) rectangle (2.27861,1.66712);
\draw [color=c, fill=c] (2.27861,1.56127) rectangle (2.31841,1.66712);
\draw [color=c, fill=c] (2.31841,1.56127) rectangle (2.35821,1.66712);
\draw [color=c, fill=c] (2.35821,1.56127) rectangle (2.39801,1.66712);
\draw [color=c, fill=c] (2.39801,1.56127) rectangle (2.43781,1.66712);
\draw [color=c, fill=c] (2.43781,1.56127) rectangle (2.47761,1.66712);
\draw [color=c, fill=c] (2.47761,1.56127) rectangle (2.51741,1.66712);
\draw [color=c, fill=c] (2.51741,1.56127) rectangle (2.55721,1.66712);
\draw [color=c, fill=c] (2.55721,1.56127) rectangle (2.59702,1.66712);
\draw [color=c, fill=c] (2.59702,1.56127) rectangle (2.63682,1.66712);
\draw [color=c, fill=c] (2.63682,1.56127) rectangle (2.67662,1.66712);
\draw [color=c, fill=c] (2.67662,1.56127) rectangle (2.71642,1.66712);
\draw [color=c, fill=c] (2.71642,1.56127) rectangle (2.75622,1.66712);
\draw [color=c, fill=c] (2.75622,1.56127) rectangle (2.79602,1.66712);
\draw [color=c, fill=c] (2.79602,1.56127) rectangle (2.83582,1.66712);
\draw [color=c, fill=c] (2.83582,1.56127) rectangle (2.87562,1.66712);
\draw [color=c, fill=c] (2.87562,1.56127) rectangle (2.91542,1.66712);
\draw [color=c, fill=c] (2.91542,1.56127) rectangle (2.95522,1.66712);
\draw [color=c, fill=c] (2.95522,1.56127) rectangle (2.99502,1.66712);
\draw [color=c, fill=c] (2.99502,1.56127) rectangle (3.03483,1.66712);
\draw [color=c, fill=c] (3.03483,1.56127) rectangle (3.07463,1.66712);
\draw [color=c, fill=c] (3.07463,1.56127) rectangle (3.11443,1.66712);
\draw [color=c, fill=c] (3.11443,1.56127) rectangle (3.15423,1.66712);
\draw [color=c, fill=c] (3.15423,1.56127) rectangle (3.19403,1.66712);
\draw [color=c, fill=c] (3.19403,1.56127) rectangle (3.23383,1.66712);
\draw [color=c, fill=c] (3.23383,1.56127) rectangle (3.27363,1.66712);
\draw [color=c, fill=c] (3.27363,1.56127) rectangle (3.31343,1.66712);
\draw [color=c, fill=c] (3.31343,1.56127) rectangle (3.35323,1.66712);
\draw [color=c, fill=c] (3.35323,1.56127) rectangle (3.39303,1.66712);
\draw [color=c, fill=c] (3.39303,1.56127) rectangle (3.43284,1.66712);
\draw [color=c, fill=c] (3.43284,1.56127) rectangle (3.47264,1.66712);
\draw [color=c, fill=c] (3.47264,1.56127) rectangle (3.51244,1.66712);
\draw [color=c, fill=c] (3.51244,1.56127) rectangle (3.55224,1.66712);
\draw [color=c, fill=c] (3.55224,1.56127) rectangle (3.59204,1.66712);
\draw [color=c, fill=c] (3.59204,1.56127) rectangle (3.63184,1.66712);
\draw [color=c, fill=c] (3.63184,1.56127) rectangle (3.67164,1.66712);
\draw [color=c, fill=c] (3.67164,1.56127) rectangle (3.71144,1.66712);
\draw [color=c, fill=c] (3.71144,1.56127) rectangle (3.75124,1.66712);
\draw [color=c, fill=c] (3.75124,1.56127) rectangle (3.79104,1.66712);
\draw [color=c, fill=c] (3.79104,1.56127) rectangle (3.83085,1.66712);
\draw [color=c, fill=c] (3.83085,1.56127) rectangle (3.87065,1.66712);
\draw [color=c, fill=c] (3.87065,1.56127) rectangle (3.91045,1.66712);
\draw [color=c, fill=c] (3.91045,1.56127) rectangle (3.95025,1.66712);
\draw [color=c, fill=c] (3.95025,1.56127) rectangle (3.99005,1.66712);
\draw [color=c, fill=c] (3.99005,1.56127) rectangle (4.02985,1.66712);
\draw [color=c, fill=c] (4.02985,1.56127) rectangle (4.06965,1.66712);
\draw [color=c, fill=c] (4.06965,1.56127) rectangle (4.10945,1.66712);
\draw [color=c, fill=c] (4.10945,1.56127) rectangle (4.14925,1.66712);
\draw [color=c, fill=c] (4.14925,1.56127) rectangle (4.18905,1.66712);
\draw [color=c, fill=c] (4.18905,1.56127) rectangle (4.22886,1.66712);
\draw [color=c, fill=c] (4.22886,1.56127) rectangle (4.26866,1.66712);
\draw [color=c, fill=c] (4.26866,1.56127) rectangle (4.30846,1.66712);
\draw [color=c, fill=c] (4.30846,1.56127) rectangle (4.34826,1.66712);
\draw [color=c, fill=c] (4.34826,1.56127) rectangle (4.38806,1.66712);
\draw [color=c, fill=c] (4.38806,1.56127) rectangle (4.42786,1.66712);
\draw [color=c, fill=c] (4.42786,1.56127) rectangle (4.46766,1.66712);
\draw [color=c, fill=c] (4.46766,1.56127) rectangle (4.50746,1.66712);
\draw [color=c, fill=c] (4.50746,1.56127) rectangle (4.54726,1.66712);
\draw [color=c, fill=c] (4.54726,1.56127) rectangle (4.58706,1.66712);
\draw [color=c, fill=c] (4.58706,1.56127) rectangle (4.62687,1.66712);
\draw [color=c, fill=c] (4.62687,1.56127) rectangle (4.66667,1.66712);
\draw [color=c, fill=c] (4.66667,1.56127) rectangle (4.70647,1.66712);
\draw [color=c, fill=c] (4.70647,1.56127) rectangle (4.74627,1.66712);
\draw [color=c, fill=c] (4.74627,1.56127) rectangle (4.78607,1.66712);
\draw [color=c, fill=c] (4.78607,1.56127) rectangle (4.82587,1.66712);
\draw [color=c, fill=c] (4.82587,1.56127) rectangle (4.86567,1.66712);
\draw [color=c, fill=c] (4.86567,1.56127) rectangle (4.90547,1.66712);
\draw [color=c, fill=c] (4.90547,1.56127) rectangle (4.94527,1.66712);
\draw [color=c, fill=c] (4.94527,1.56127) rectangle (4.98507,1.66712);
\draw [color=c, fill=c] (4.98507,1.56127) rectangle (5.02488,1.66712);
\draw [color=c, fill=c] (5.02488,1.56127) rectangle (5.06468,1.66712);
\draw [color=c, fill=c] (5.06468,1.56127) rectangle (5.10448,1.66712);
\draw [color=c, fill=c] (5.10448,1.56127) rectangle (5.14428,1.66712);
\draw [color=c, fill=c] (5.14428,1.56127) rectangle (5.18408,1.66712);
\draw [color=c, fill=c] (5.18408,1.56127) rectangle (5.22388,1.66712);
\draw [color=c, fill=c] (5.22388,1.56127) rectangle (5.26368,1.66712);
\draw [color=c, fill=c] (5.26368,1.56127) rectangle (5.30348,1.66712);
\draw [color=c, fill=c] (5.30348,1.56127) rectangle (5.34328,1.66712);
\draw [color=c, fill=c] (5.34328,1.56127) rectangle (5.38308,1.66712);
\draw [color=c, fill=c] (5.38308,1.56127) rectangle (5.42289,1.66712);
\draw [color=c, fill=c] (5.42289,1.56127) rectangle (5.46269,1.66712);
\draw [color=c, fill=c] (5.46269,1.56127) rectangle (5.50249,1.66712);
\draw [color=c, fill=c] (5.50249,1.56127) rectangle (5.54229,1.66712);
\draw [color=c, fill=c] (5.54229,1.56127) rectangle (5.58209,1.66712);
\draw [color=c, fill=c] (5.58209,1.56127) rectangle (5.62189,1.66712);
\draw [color=c, fill=c] (5.62189,1.56127) rectangle (5.66169,1.66712);
\draw [color=c, fill=c] (5.66169,1.56127) rectangle (5.70149,1.66712);
\draw [color=c, fill=c] (5.70149,1.56127) rectangle (5.74129,1.66712);
\draw [color=c, fill=c] (5.74129,1.56127) rectangle (5.78109,1.66712);
\draw [color=c, fill=c] (5.78109,1.56127) rectangle (5.8209,1.66712);
\draw [color=c, fill=c] (5.8209,1.56127) rectangle (5.8607,1.66712);
\draw [color=c, fill=c] (5.8607,1.56127) rectangle (5.9005,1.66712);
\draw [color=c, fill=c] (5.9005,1.56127) rectangle (5.9403,1.66712);
\draw [color=c, fill=c] (5.9403,1.56127) rectangle (5.9801,1.66712);
\draw [color=c, fill=c] (5.9801,1.56127) rectangle (6.0199,1.66712);
\draw [color=c, fill=c] (6.0199,1.56127) rectangle (6.0597,1.66712);
\draw [color=c, fill=c] (6.0597,1.56127) rectangle (6.0995,1.66712);
\draw [color=c, fill=c] (6.0995,1.56127) rectangle (6.1393,1.66712);
\draw [color=c, fill=c] (6.1393,1.56127) rectangle (6.1791,1.66712);
\draw [color=c, fill=c] (6.1791,1.56127) rectangle (6.21891,1.66712);
\draw [color=c, fill=c] (6.21891,1.56127) rectangle (6.25871,1.66712);
\draw [color=c, fill=c] (6.25871,1.56127) rectangle (6.29851,1.66712);
\draw [color=c, fill=c] (6.29851,1.56127) rectangle (6.33831,1.66712);
\draw [color=c, fill=c] (6.33831,1.56127) rectangle (6.37811,1.66712);
\draw [color=c, fill=c] (6.37811,1.56127) rectangle (6.41791,1.66712);
\draw [color=c, fill=c] (6.41791,1.56127) rectangle (6.45771,1.66712);
\draw [color=c, fill=c] (6.45771,1.56127) rectangle (6.49751,1.66712);
\draw [color=c, fill=c] (6.49751,1.56127) rectangle (6.53731,1.66712);
\draw [color=c, fill=c] (6.53731,1.56127) rectangle (6.57711,1.66712);
\draw [color=c, fill=c] (6.57711,1.56127) rectangle (6.61692,1.66712);
\draw [color=c, fill=c] (6.61692,1.56127) rectangle (6.65672,1.66712);
\draw [color=c, fill=c] (6.65672,1.56127) rectangle (6.69652,1.66712);
\draw [color=c, fill=c] (6.69652,1.56127) rectangle (6.73632,1.66712);
\draw [color=c, fill=c] (6.73632,1.56127) rectangle (6.77612,1.66712);
\draw [color=c, fill=c] (6.77612,1.56127) rectangle (6.81592,1.66712);
\draw [color=c, fill=c] (6.81592,1.56127) rectangle (6.85572,1.66712);
\draw [color=c, fill=c] (6.85572,1.56127) rectangle (6.89552,1.66712);
\draw [color=c, fill=c] (6.89552,1.56127) rectangle (6.93532,1.66712);
\draw [color=c, fill=c] (6.93532,1.56127) rectangle (6.97512,1.66712);
\draw [color=c, fill=c] (6.97512,1.56127) rectangle (7.01493,1.66712);
\draw [color=c, fill=c] (7.01493,1.56127) rectangle (7.05473,1.66712);
\draw [color=c, fill=c] (7.05473,1.56127) rectangle (7.09453,1.66712);
\draw [color=c, fill=c] (7.09453,1.56127) rectangle (7.13433,1.66712);
\draw [color=c, fill=c] (7.13433,1.56127) rectangle (7.17413,1.66712);
\draw [color=c, fill=c] (7.17413,1.56127) rectangle (7.21393,1.66712);
\draw [color=c, fill=c] (7.21393,1.56127) rectangle (7.25373,1.66712);
\draw [color=c, fill=c] (7.25373,1.56127) rectangle (7.29353,1.66712);
\draw [color=c, fill=c] (7.29353,1.56127) rectangle (7.33333,1.66712);
\draw [color=c, fill=c] (7.33333,1.56127) rectangle (7.37313,1.66712);
\draw [color=c, fill=c] (7.37313,1.56127) rectangle (7.41294,1.66712);
\draw [color=c, fill=c] (7.41294,1.56127) rectangle (7.45274,1.66712);
\draw [color=c, fill=c] (7.45274,1.56127) rectangle (7.49254,1.66712);
\draw [color=c, fill=c] (7.49254,1.56127) rectangle (7.53234,1.66712);
\draw [color=c, fill=c] (7.53234,1.56127) rectangle (7.57214,1.66712);
\draw [color=c, fill=c] (7.57214,1.56127) rectangle (7.61194,1.66712);
\draw [color=c, fill=c] (7.61194,1.56127) rectangle (7.65174,1.66712);
\draw [color=c, fill=c] (7.65174,1.56127) rectangle (7.69154,1.66712);
\draw [color=c, fill=c] (7.69154,1.56127) rectangle (7.73134,1.66712);
\draw [color=c, fill=c] (7.73134,1.56127) rectangle (7.77114,1.66712);
\definecolor{c}{rgb}{1,0.186667,0};
\draw [color=c, fill=c] (7.77114,1.56127) rectangle (7.81095,1.66712);
\draw [color=c, fill=c] (7.81095,1.56127) rectangle (7.85075,1.66712);
\draw [color=c, fill=c] (7.85075,1.56127) rectangle (7.89055,1.66712);
\draw [color=c, fill=c] (7.89055,1.56127) rectangle (7.93035,1.66712);
\draw [color=c, fill=c] (7.93035,1.56127) rectangle (7.97015,1.66712);
\draw [color=c, fill=c] (7.97015,1.56127) rectangle (8.00995,1.66712);
\draw [color=c, fill=c] (8.00995,1.56127) rectangle (8.04975,1.66712);
\draw [color=c, fill=c] (8.04975,1.56127) rectangle (8.08955,1.66712);
\draw [color=c, fill=c] (8.08955,1.56127) rectangle (8.12935,1.66712);
\draw [color=c, fill=c] (8.12935,1.56127) rectangle (8.16915,1.66712);
\draw [color=c, fill=c] (8.16915,1.56127) rectangle (8.20895,1.66712);
\draw [color=c, fill=c] (8.20895,1.56127) rectangle (8.24876,1.66712);
\draw [color=c, fill=c] (8.24876,1.56127) rectangle (8.28856,1.66712);
\draw [color=c, fill=c] (8.28856,1.56127) rectangle (8.32836,1.66712);
\draw [color=c, fill=c] (8.32836,1.56127) rectangle (8.36816,1.66712);
\draw [color=c, fill=c] (8.36816,1.56127) rectangle (8.40796,1.66712);
\draw [color=c, fill=c] (8.40796,1.56127) rectangle (8.44776,1.66712);
\draw [color=c, fill=c] (8.44776,1.56127) rectangle (8.48756,1.66712);
\draw [color=c, fill=c] (8.48756,1.56127) rectangle (8.52736,1.66712);
\draw [color=c, fill=c] (8.52736,1.56127) rectangle (8.56716,1.66712);
\draw [color=c, fill=c] (8.56716,1.56127) rectangle (8.60697,1.66712);
\definecolor{c}{rgb}{1,0.466667,0};
\draw [color=c, fill=c] (8.60697,1.56127) rectangle (8.64677,1.66712);
\draw [color=c, fill=c] (8.64677,1.56127) rectangle (8.68657,1.66712);
\draw [color=c, fill=c] (8.68657,1.56127) rectangle (8.72637,1.66712);
\draw [color=c, fill=c] (8.72637,1.56127) rectangle (8.76617,1.66712);
\draw [color=c, fill=c] (8.76617,1.56127) rectangle (8.80597,1.66712);
\draw [color=c, fill=c] (8.80597,1.56127) rectangle (8.84577,1.66712);
\draw [color=c, fill=c] (8.84577,1.56127) rectangle (8.88557,1.66712);
\draw [color=c, fill=c] (8.88557,1.56127) rectangle (8.92537,1.66712);
\draw [color=c, fill=c] (8.92537,1.56127) rectangle (8.96517,1.66712);
\draw [color=c, fill=c] (8.96517,1.56127) rectangle (9.00498,1.66712);
\draw [color=c, fill=c] (9.00498,1.56127) rectangle (9.04478,1.66712);
\definecolor{c}{rgb}{1,0.653333,0};
\draw [color=c, fill=c] (9.04478,1.56127) rectangle (9.08458,1.66712);
\draw [color=c, fill=c] (9.08458,1.56127) rectangle (9.12438,1.66712);
\draw [color=c, fill=c] (9.12438,1.56127) rectangle (9.16418,1.66712);
\draw [color=c, fill=c] (9.16418,1.56127) rectangle (9.20398,1.66712);
\draw [color=c, fill=c] (9.20398,1.56127) rectangle (9.24378,1.66712);
\draw [color=c, fill=c] (9.24378,1.56127) rectangle (9.28358,1.66712);
\draw [color=c, fill=c] (9.28358,1.56127) rectangle (9.32338,1.66712);
\definecolor{c}{rgb}{1,0.933333,0};
\draw [color=c, fill=c] (9.32338,1.56127) rectangle (9.36318,1.66712);
\draw [color=c, fill=c] (9.36318,1.56127) rectangle (9.40298,1.66712);
\draw [color=c, fill=c] (9.40298,1.56127) rectangle (9.44279,1.66712);
\draw [color=c, fill=c] (9.44279,1.56127) rectangle (9.48259,1.66712);
\draw [color=c, fill=c] (9.48259,1.56127) rectangle (9.52239,1.66712);
\definecolor{c}{rgb}{0.88,1,0};
\draw [color=c, fill=c] (9.52239,1.56127) rectangle (9.56219,1.66712);
\draw [color=c, fill=c] (9.56219,1.56127) rectangle (9.60199,1.66712);
\draw [color=c, fill=c] (9.60199,1.56127) rectangle (9.64179,1.66712);
\draw [color=c, fill=c] (9.64179,1.56127) rectangle (9.68159,1.66712);
\draw [color=c, fill=c] (9.68159,1.56127) rectangle (9.72139,1.66712);
\definecolor{c}{rgb}{0.6,1,0};
\draw [color=c, fill=c] (9.72139,1.56127) rectangle (9.76119,1.66712);
\draw [color=c, fill=c] (9.76119,1.56127) rectangle (9.80099,1.66712);
\draw [color=c, fill=c] (9.80099,1.56127) rectangle (9.8408,1.66712);
\draw [color=c, fill=c] (9.8408,1.56127) rectangle (9.8806,1.66712);
\definecolor{c}{rgb}{0.413333,1,0};
\draw [color=c, fill=c] (9.8806,1.56127) rectangle (9.9204,1.66712);
\draw [color=c, fill=c] (9.9204,1.56127) rectangle (9.9602,1.66712);
\draw [color=c, fill=c] (9.9602,1.56127) rectangle (10,1.66712);
\draw [color=c, fill=c] (10,1.56127) rectangle (10.0398,1.66712);
\definecolor{c}{rgb}{0.133333,1,0};
\draw [color=c, fill=c] (10.0398,1.56127) rectangle (10.0796,1.66712);
\draw [color=c, fill=c] (10.0796,1.56127) rectangle (10.1194,1.66712);
\draw [color=c, fill=c] (10.1194,1.56127) rectangle (10.1592,1.66712);
\draw [color=c, fill=c] (10.1592,1.56127) rectangle (10.199,1.66712);
\draw [color=c, fill=c] (10.199,1.56127) rectangle (10.2388,1.66712);
\definecolor{c}{rgb}{0,1,0.0533333};
\draw [color=c, fill=c] (10.2388,1.56127) rectangle (10.2786,1.66712);
\draw [color=c, fill=c] (10.2786,1.56127) rectangle (10.3184,1.66712);
\draw [color=c, fill=c] (10.3184,1.56127) rectangle (10.3582,1.66712);
\draw [color=c, fill=c] (10.3582,1.56127) rectangle (10.398,1.66712);
\definecolor{c}{rgb}{0,1,0.333333};
\draw [color=c, fill=c] (10.398,1.56127) rectangle (10.4378,1.66712);
\draw [color=c, fill=c] (10.4378,1.56127) rectangle (10.4776,1.66712);
\draw [color=c, fill=c] (10.4776,1.56127) rectangle (10.5174,1.66712);
\draw [color=c, fill=c] (10.5174,1.56127) rectangle (10.5572,1.66712);
\draw [color=c, fill=c] (10.5572,1.56127) rectangle (10.597,1.66712);
\draw [color=c, fill=c] (10.597,1.56127) rectangle (10.6368,1.66712);
\definecolor{c}{rgb}{0,1,0.52};
\draw [color=c, fill=c] (10.6368,1.56127) rectangle (10.6766,1.66712);
\draw [color=c, fill=c] (10.6766,1.56127) rectangle (10.7164,1.66712);
\draw [color=c, fill=c] (10.7164,1.56127) rectangle (10.7562,1.66712);
\draw [color=c, fill=c] (10.7562,1.56127) rectangle (10.796,1.66712);
\draw [color=c, fill=c] (10.796,1.56127) rectangle (10.8358,1.66712);
\draw [color=c, fill=c] (10.8358,1.56127) rectangle (10.8756,1.66712);
\draw [color=c, fill=c] (10.8756,1.56127) rectangle (10.9154,1.66712);
\draw [color=c, fill=c] (10.9154,1.56127) rectangle (10.9552,1.66712);
\definecolor{c}{rgb}{0,1,0.8};
\draw [color=c, fill=c] (10.9552,1.56127) rectangle (10.995,1.66712);
\draw [color=c, fill=c] (10.995,1.56127) rectangle (11.0348,1.66712);
\draw [color=c, fill=c] (11.0348,1.56127) rectangle (11.0746,1.66712);
\draw [color=c, fill=c] (11.0746,1.56127) rectangle (11.1144,1.66712);
\draw [color=c, fill=c] (11.1144,1.56127) rectangle (11.1542,1.66712);
\draw [color=c, fill=c] (11.1542,1.56127) rectangle (11.194,1.66712);
\draw [color=c, fill=c] (11.194,1.56127) rectangle (11.2338,1.66712);
\draw [color=c, fill=c] (11.2338,1.56127) rectangle (11.2736,1.66712);
\draw [color=c, fill=c] (11.2736,1.56127) rectangle (11.3134,1.66712);
\draw [color=c, fill=c] (11.3134,1.56127) rectangle (11.3532,1.66712);
\draw [color=c, fill=c] (11.3532,1.56127) rectangle (11.393,1.66712);
\draw [color=c, fill=c] (11.393,1.56127) rectangle (11.4328,1.66712);
\definecolor{c}{rgb}{0,1,0.986667};
\draw [color=c, fill=c] (11.4328,1.56127) rectangle (11.4726,1.66712);
\draw [color=c, fill=c] (11.4726,1.56127) rectangle (11.5124,1.66712);
\draw [color=c, fill=c] (11.5124,1.56127) rectangle (11.5522,1.66712);
\draw [color=c, fill=c] (11.5522,1.56127) rectangle (11.592,1.66712);
\draw [color=c, fill=c] (11.592,1.56127) rectangle (11.6318,1.66712);
\draw [color=c, fill=c] (11.6318,1.56127) rectangle (11.6716,1.66712);
\draw [color=c, fill=c] (11.6716,1.56127) rectangle (11.7114,1.66712);
\draw [color=c, fill=c] (11.7114,1.56127) rectangle (11.7512,1.66712);
\draw [color=c, fill=c] (11.7512,1.56127) rectangle (11.791,1.66712);
\draw [color=c, fill=c] (11.791,1.56127) rectangle (11.8308,1.66712);
\draw [color=c, fill=c] (11.8308,1.56127) rectangle (11.8706,1.66712);
\draw [color=c, fill=c] (11.8706,1.56127) rectangle (11.9104,1.66712);
\draw [color=c, fill=c] (11.9104,1.56127) rectangle (11.9502,1.66712);
\draw [color=c, fill=c] (11.9502,1.56127) rectangle (11.99,1.66712);
\draw [color=c, fill=c] (11.99,1.56127) rectangle (12.0299,1.66712);
\draw [color=c, fill=c] (12.0299,1.56127) rectangle (12.0697,1.66712);
\draw [color=c, fill=c] (12.0697,1.56127) rectangle (12.1095,1.66712);
\draw [color=c, fill=c] (12.1095,1.56127) rectangle (12.1493,1.66712);
\draw [color=c, fill=c] (12.1493,1.56127) rectangle (12.1891,1.66712);
\draw [color=c, fill=c] (12.1891,1.56127) rectangle (12.2289,1.66712);
\draw [color=c, fill=c] (12.2289,1.56127) rectangle (12.2687,1.66712);
\draw [color=c, fill=c] (12.2687,1.56127) rectangle (12.3085,1.66712);
\draw [color=c, fill=c] (12.3085,1.56127) rectangle (12.3483,1.66712);
\definecolor{c}{rgb}{0,0.733333,1};
\draw [color=c, fill=c] (12.3483,1.56127) rectangle (12.3881,1.66712);
\draw [color=c, fill=c] (12.3881,1.56127) rectangle (12.4279,1.66712);
\draw [color=c, fill=c] (12.4279,1.56127) rectangle (12.4677,1.66712);
\draw [color=c, fill=c] (12.4677,1.56127) rectangle (12.5075,1.66712);
\draw [color=c, fill=c] (12.5075,1.56127) rectangle (12.5473,1.66712);
\draw [color=c, fill=c] (12.5473,1.56127) rectangle (12.5871,1.66712);
\draw [color=c, fill=c] (12.5871,1.56127) rectangle (12.6269,1.66712);
\draw [color=c, fill=c] (12.6269,1.56127) rectangle (12.6667,1.66712);
\draw [color=c, fill=c] (12.6667,1.56127) rectangle (12.7065,1.66712);
\draw [color=c, fill=c] (12.7065,1.56127) rectangle (12.7463,1.66712);
\draw [color=c, fill=c] (12.7463,1.56127) rectangle (12.7861,1.66712);
\draw [color=c, fill=c] (12.7861,1.56127) rectangle (12.8259,1.66712);
\draw [color=c, fill=c] (12.8259,1.56127) rectangle (12.8657,1.66712);
\draw [color=c, fill=c] (12.8657,1.56127) rectangle (12.9055,1.66712);
\draw [color=c, fill=c] (12.9055,1.56127) rectangle (12.9453,1.66712);
\draw [color=c, fill=c] (12.9453,1.56127) rectangle (12.9851,1.66712);
\draw [color=c, fill=c] (12.9851,1.56127) rectangle (13.0249,1.66712);
\draw [color=c, fill=c] (13.0249,1.56127) rectangle (13.0647,1.66712);
\draw [color=c, fill=c] (13.0647,1.56127) rectangle (13.1045,1.66712);
\draw [color=c, fill=c] (13.1045,1.56127) rectangle (13.1443,1.66712);
\draw [color=c, fill=c] (13.1443,1.56127) rectangle (13.1841,1.66712);
\draw [color=c, fill=c] (13.1841,1.56127) rectangle (13.2239,1.66712);
\draw [color=c, fill=c] (13.2239,1.56127) rectangle (13.2637,1.66712);
\draw [color=c, fill=c] (13.2637,1.56127) rectangle (13.3035,1.66712);
\draw [color=c, fill=c] (13.3035,1.56127) rectangle (13.3433,1.66712);
\draw [color=c, fill=c] (13.3433,1.56127) rectangle (13.3831,1.66712);
\draw [color=c, fill=c] (13.3831,1.56127) rectangle (13.4229,1.66712);
\draw [color=c, fill=c] (13.4229,1.56127) rectangle (13.4627,1.66712);
\draw [color=c, fill=c] (13.4627,1.56127) rectangle (13.5025,1.66712);
\draw [color=c, fill=c] (13.5025,1.56127) rectangle (13.5423,1.66712);
\draw [color=c, fill=c] (13.5423,1.56127) rectangle (13.5821,1.66712);
\draw [color=c, fill=c] (13.5821,1.56127) rectangle (13.6219,1.66712);
\draw [color=c, fill=c] (13.6219,1.56127) rectangle (13.6617,1.66712);
\draw [color=c, fill=c] (13.6617,1.56127) rectangle (13.7015,1.66712);
\draw [color=c, fill=c] (13.7015,1.56127) rectangle (13.7413,1.66712);
\draw [color=c, fill=c] (13.7413,1.56127) rectangle (13.7811,1.66712);
\draw [color=c, fill=c] (13.7811,1.56127) rectangle (13.8209,1.66712);
\draw [color=c, fill=c] (13.8209,1.56127) rectangle (13.8607,1.66712);
\draw [color=c, fill=c] (13.8607,1.56127) rectangle (13.9005,1.66712);
\draw [color=c, fill=c] (13.9005,1.56127) rectangle (13.9403,1.66712);
\draw [color=c, fill=c] (13.9403,1.56127) rectangle (13.9801,1.66712);
\draw [color=c, fill=c] (13.9801,1.56127) rectangle (14.0199,1.66712);
\draw [color=c, fill=c] (14.0199,1.56127) rectangle (14.0597,1.66712);
\draw [color=c, fill=c] (14.0597,1.56127) rectangle (14.0995,1.66712);
\draw [color=c, fill=c] (14.0995,1.56127) rectangle (14.1393,1.66712);
\draw [color=c, fill=c] (14.1393,1.56127) rectangle (14.1791,1.66712);
\draw [color=c, fill=c] (14.1791,1.56127) rectangle (14.2189,1.66712);
\draw [color=c, fill=c] (14.2189,1.56127) rectangle (14.2587,1.66712);
\draw [color=c, fill=c] (14.2587,1.56127) rectangle (14.2985,1.66712);
\draw [color=c, fill=c] (14.2985,1.56127) rectangle (14.3383,1.66712);
\draw [color=c, fill=c] (14.3383,1.56127) rectangle (14.3781,1.66712);
\draw [color=c, fill=c] (14.3781,1.56127) rectangle (14.4179,1.66712);
\draw [color=c, fill=c] (14.4179,1.56127) rectangle (14.4577,1.66712);
\draw [color=c, fill=c] (14.4577,1.56127) rectangle (14.4975,1.66712);
\draw [color=c, fill=c] (14.4975,1.56127) rectangle (14.5373,1.66712);
\draw [color=c, fill=c] (14.5373,1.56127) rectangle (14.5771,1.66712);
\draw [color=c, fill=c] (14.5771,1.56127) rectangle (14.6169,1.66712);
\draw [color=c, fill=c] (14.6169,1.56127) rectangle (14.6567,1.66712);
\draw [color=c, fill=c] (14.6567,1.56127) rectangle (14.6965,1.66712);
\draw [color=c, fill=c] (14.6965,1.56127) rectangle (14.7363,1.66712);
\draw [color=c, fill=c] (14.7363,1.56127) rectangle (14.7761,1.66712);
\draw [color=c, fill=c] (14.7761,1.56127) rectangle (14.8159,1.66712);
\draw [color=c, fill=c] (14.8159,1.56127) rectangle (14.8557,1.66712);
\draw [color=c, fill=c] (14.8557,1.56127) rectangle (14.8955,1.66712);
\draw [color=c, fill=c] (14.8955,1.56127) rectangle (14.9353,1.66712);
\draw [color=c, fill=c] (14.9353,1.56127) rectangle (14.9751,1.66712);
\draw [color=c, fill=c] (14.9751,1.56127) rectangle (15.0149,1.66712);
\draw [color=c, fill=c] (15.0149,1.56127) rectangle (15.0547,1.66712);
\draw [color=c, fill=c] (15.0547,1.56127) rectangle (15.0945,1.66712);
\draw [color=c, fill=c] (15.0945,1.56127) rectangle (15.1343,1.66712);
\draw [color=c, fill=c] (15.1343,1.56127) rectangle (15.1741,1.66712);
\draw [color=c, fill=c] (15.1741,1.56127) rectangle (15.2139,1.66712);
\draw [color=c, fill=c] (15.2139,1.56127) rectangle (15.2537,1.66712);
\draw [color=c, fill=c] (15.2537,1.56127) rectangle (15.2935,1.66712);
\draw [color=c, fill=c] (15.2935,1.56127) rectangle (15.3333,1.66712);
\draw [color=c, fill=c] (15.3333,1.56127) rectangle (15.3731,1.66712);
\draw [color=c, fill=c] (15.3731,1.56127) rectangle (15.4129,1.66712);
\draw [color=c, fill=c] (15.4129,1.56127) rectangle (15.4527,1.66712);
\draw [color=c, fill=c] (15.4527,1.56127) rectangle (15.4925,1.66712);
\draw [color=c, fill=c] (15.4925,1.56127) rectangle (15.5323,1.66712);
\draw [color=c, fill=c] (15.5323,1.56127) rectangle (15.5721,1.66712);
\draw [color=c, fill=c] (15.5721,1.56127) rectangle (15.6119,1.66712);
\draw [color=c, fill=c] (15.6119,1.56127) rectangle (15.6517,1.66712);
\draw [color=c, fill=c] (15.6517,1.56127) rectangle (15.6915,1.66712);
\draw [color=c, fill=c] (15.6915,1.56127) rectangle (15.7313,1.66712);
\draw [color=c, fill=c] (15.7313,1.56127) rectangle (15.7711,1.66712);
\draw [color=c, fill=c] (15.7711,1.56127) rectangle (15.8109,1.66712);
\draw [color=c, fill=c] (15.8109,1.56127) rectangle (15.8507,1.66712);
\draw [color=c, fill=c] (15.8507,1.56127) rectangle (15.8905,1.66712);
\draw [color=c, fill=c] (15.8905,1.56127) rectangle (15.9303,1.66712);
\draw [color=c, fill=c] (15.9303,1.56127) rectangle (15.9701,1.66712);
\draw [color=c, fill=c] (15.9701,1.56127) rectangle (16.01,1.66712);
\draw [color=c, fill=c] (16.01,1.56127) rectangle (16.0498,1.66712);
\draw [color=c, fill=c] (16.0498,1.56127) rectangle (16.0896,1.66712);
\draw [color=c, fill=c] (16.0896,1.56127) rectangle (16.1294,1.66712);
\draw [color=c, fill=c] (16.1294,1.56127) rectangle (16.1692,1.66712);
\draw [color=c, fill=c] (16.1692,1.56127) rectangle (16.209,1.66712);
\draw [color=c, fill=c] (16.209,1.56127) rectangle (16.2488,1.66712);
\draw [color=c, fill=c] (16.2488,1.56127) rectangle (16.2886,1.66712);
\draw [color=c, fill=c] (16.2886,1.56127) rectangle (16.3284,1.66712);
\draw [color=c, fill=c] (16.3284,1.56127) rectangle (16.3682,1.66712);
\draw [color=c, fill=c] (16.3682,1.56127) rectangle (16.408,1.66712);
\draw [color=c, fill=c] (16.408,1.56127) rectangle (16.4478,1.66712);
\draw [color=c, fill=c] (16.4478,1.56127) rectangle (16.4876,1.66712);
\draw [color=c, fill=c] (16.4876,1.56127) rectangle (16.5274,1.66712);
\draw [color=c, fill=c] (16.5274,1.56127) rectangle (16.5672,1.66712);
\draw [color=c, fill=c] (16.5672,1.56127) rectangle (16.607,1.66712);
\draw [color=c, fill=c] (16.607,1.56127) rectangle (16.6468,1.66712);
\draw [color=c, fill=c] (16.6468,1.56127) rectangle (16.6866,1.66712);
\draw [color=c, fill=c] (16.6866,1.56127) rectangle (16.7264,1.66712);
\draw [color=c, fill=c] (16.7264,1.56127) rectangle (16.7662,1.66712);
\draw [color=c, fill=c] (16.7662,1.56127) rectangle (16.806,1.66712);
\draw [color=c, fill=c] (16.806,1.56127) rectangle (16.8458,1.66712);
\draw [color=c, fill=c] (16.8458,1.56127) rectangle (16.8856,1.66712);
\draw [color=c, fill=c] (16.8856,1.56127) rectangle (16.9254,1.66712);
\draw [color=c, fill=c] (16.9254,1.56127) rectangle (16.9652,1.66712);
\draw [color=c, fill=c] (16.9652,1.56127) rectangle (17.005,1.66712);
\draw [color=c, fill=c] (17.005,1.56127) rectangle (17.0448,1.66712);
\draw [color=c, fill=c] (17.0448,1.56127) rectangle (17.0846,1.66712);
\draw [color=c, fill=c] (17.0846,1.56127) rectangle (17.1244,1.66712);
\draw [color=c, fill=c] (17.1244,1.56127) rectangle (17.1642,1.66712);
\draw [color=c, fill=c] (17.1642,1.56127) rectangle (17.204,1.66712);
\draw [color=c, fill=c] (17.204,1.56127) rectangle (17.2438,1.66712);
\draw [color=c, fill=c] (17.2438,1.56127) rectangle (17.2836,1.66712);
\draw [color=c, fill=c] (17.2836,1.56127) rectangle (17.3234,1.66712);
\draw [color=c, fill=c] (17.3234,1.56127) rectangle (17.3632,1.66712);
\draw [color=c, fill=c] (17.3632,1.56127) rectangle (17.403,1.66712);
\draw [color=c, fill=c] (17.403,1.56127) rectangle (17.4428,1.66712);
\draw [color=c, fill=c] (17.4428,1.56127) rectangle (17.4826,1.66712);
\draw [color=c, fill=c] (17.4826,1.56127) rectangle (17.5224,1.66712);
\draw [color=c, fill=c] (17.5224,1.56127) rectangle (17.5622,1.66712);
\draw [color=c, fill=c] (17.5622,1.56127) rectangle (17.602,1.66712);
\draw [color=c, fill=c] (17.602,1.56127) rectangle (17.6418,1.66712);
\draw [color=c, fill=c] (17.6418,1.56127) rectangle (17.6816,1.66712);
\draw [color=c, fill=c] (17.6816,1.56127) rectangle (17.7214,1.66712);
\draw [color=c, fill=c] (17.7214,1.56127) rectangle (17.7612,1.66712);
\draw [color=c, fill=c] (17.7612,1.56127) rectangle (17.801,1.66712);
\draw [color=c, fill=c] (17.801,1.56127) rectangle (17.8408,1.66712);
\draw [color=c, fill=c] (17.8408,1.56127) rectangle (17.8806,1.66712);
\draw [color=c, fill=c] (17.8806,1.56127) rectangle (17.9204,1.66712);
\draw [color=c, fill=c] (17.9204,1.56127) rectangle (17.9602,1.66712);
\draw [color=c, fill=c] (17.9602,1.56127) rectangle (18,1.66712);
\definecolor{c}{rgb}{1,0,0};
\draw [color=c, fill=c] (2,1.66712) rectangle (2.0398,1.77296);
\draw [color=c, fill=c] (2.0398,1.66712) rectangle (2.0796,1.77296);
\draw [color=c, fill=c] (2.0796,1.66712) rectangle (2.1194,1.77296);
\draw [color=c, fill=c] (2.1194,1.66712) rectangle (2.1592,1.77296);
\draw [color=c, fill=c] (2.1592,1.66712) rectangle (2.19901,1.77296);
\draw [color=c, fill=c] (2.19901,1.66712) rectangle (2.23881,1.77296);
\draw [color=c, fill=c] (2.23881,1.66712) rectangle (2.27861,1.77296);
\draw [color=c, fill=c] (2.27861,1.66712) rectangle (2.31841,1.77296);
\draw [color=c, fill=c] (2.31841,1.66712) rectangle (2.35821,1.77296);
\draw [color=c, fill=c] (2.35821,1.66712) rectangle (2.39801,1.77296);
\draw [color=c, fill=c] (2.39801,1.66712) rectangle (2.43781,1.77296);
\draw [color=c, fill=c] (2.43781,1.66712) rectangle (2.47761,1.77296);
\draw [color=c, fill=c] (2.47761,1.66712) rectangle (2.51741,1.77296);
\draw [color=c, fill=c] (2.51741,1.66712) rectangle (2.55721,1.77296);
\draw [color=c, fill=c] (2.55721,1.66712) rectangle (2.59702,1.77296);
\draw [color=c, fill=c] (2.59702,1.66712) rectangle (2.63682,1.77296);
\draw [color=c, fill=c] (2.63682,1.66712) rectangle (2.67662,1.77296);
\draw [color=c, fill=c] (2.67662,1.66712) rectangle (2.71642,1.77296);
\draw [color=c, fill=c] (2.71642,1.66712) rectangle (2.75622,1.77296);
\draw [color=c, fill=c] (2.75622,1.66712) rectangle (2.79602,1.77296);
\draw [color=c, fill=c] (2.79602,1.66712) rectangle (2.83582,1.77296);
\draw [color=c, fill=c] (2.83582,1.66712) rectangle (2.87562,1.77296);
\draw [color=c, fill=c] (2.87562,1.66712) rectangle (2.91542,1.77296);
\draw [color=c, fill=c] (2.91542,1.66712) rectangle (2.95522,1.77296);
\draw [color=c, fill=c] (2.95522,1.66712) rectangle (2.99502,1.77296);
\draw [color=c, fill=c] (2.99502,1.66712) rectangle (3.03483,1.77296);
\draw [color=c, fill=c] (3.03483,1.66712) rectangle (3.07463,1.77296);
\draw [color=c, fill=c] (3.07463,1.66712) rectangle (3.11443,1.77296);
\draw [color=c, fill=c] (3.11443,1.66712) rectangle (3.15423,1.77296);
\draw [color=c, fill=c] (3.15423,1.66712) rectangle (3.19403,1.77296);
\draw [color=c, fill=c] (3.19403,1.66712) rectangle (3.23383,1.77296);
\draw [color=c, fill=c] (3.23383,1.66712) rectangle (3.27363,1.77296);
\draw [color=c, fill=c] (3.27363,1.66712) rectangle (3.31343,1.77296);
\draw [color=c, fill=c] (3.31343,1.66712) rectangle (3.35323,1.77296);
\draw [color=c, fill=c] (3.35323,1.66712) rectangle (3.39303,1.77296);
\draw [color=c, fill=c] (3.39303,1.66712) rectangle (3.43284,1.77296);
\draw [color=c, fill=c] (3.43284,1.66712) rectangle (3.47264,1.77296);
\draw [color=c, fill=c] (3.47264,1.66712) rectangle (3.51244,1.77296);
\draw [color=c, fill=c] (3.51244,1.66712) rectangle (3.55224,1.77296);
\draw [color=c, fill=c] (3.55224,1.66712) rectangle (3.59204,1.77296);
\draw [color=c, fill=c] (3.59204,1.66712) rectangle (3.63184,1.77296);
\draw [color=c, fill=c] (3.63184,1.66712) rectangle (3.67164,1.77296);
\draw [color=c, fill=c] (3.67164,1.66712) rectangle (3.71144,1.77296);
\draw [color=c, fill=c] (3.71144,1.66712) rectangle (3.75124,1.77296);
\draw [color=c, fill=c] (3.75124,1.66712) rectangle (3.79104,1.77296);
\draw [color=c, fill=c] (3.79104,1.66712) rectangle (3.83085,1.77296);
\draw [color=c, fill=c] (3.83085,1.66712) rectangle (3.87065,1.77296);
\draw [color=c, fill=c] (3.87065,1.66712) rectangle (3.91045,1.77296);
\draw [color=c, fill=c] (3.91045,1.66712) rectangle (3.95025,1.77296);
\draw [color=c, fill=c] (3.95025,1.66712) rectangle (3.99005,1.77296);
\draw [color=c, fill=c] (3.99005,1.66712) rectangle (4.02985,1.77296);
\draw [color=c, fill=c] (4.02985,1.66712) rectangle (4.06965,1.77296);
\draw [color=c, fill=c] (4.06965,1.66712) rectangle (4.10945,1.77296);
\draw [color=c, fill=c] (4.10945,1.66712) rectangle (4.14925,1.77296);
\draw [color=c, fill=c] (4.14925,1.66712) rectangle (4.18905,1.77296);
\draw [color=c, fill=c] (4.18905,1.66712) rectangle (4.22886,1.77296);
\draw [color=c, fill=c] (4.22886,1.66712) rectangle (4.26866,1.77296);
\draw [color=c, fill=c] (4.26866,1.66712) rectangle (4.30846,1.77296);
\draw [color=c, fill=c] (4.30846,1.66712) rectangle (4.34826,1.77296);
\draw [color=c, fill=c] (4.34826,1.66712) rectangle (4.38806,1.77296);
\draw [color=c, fill=c] (4.38806,1.66712) rectangle (4.42786,1.77296);
\draw [color=c, fill=c] (4.42786,1.66712) rectangle (4.46766,1.77296);
\draw [color=c, fill=c] (4.46766,1.66712) rectangle (4.50746,1.77296);
\draw [color=c, fill=c] (4.50746,1.66712) rectangle (4.54726,1.77296);
\draw [color=c, fill=c] (4.54726,1.66712) rectangle (4.58706,1.77296);
\draw [color=c, fill=c] (4.58706,1.66712) rectangle (4.62687,1.77296);
\draw [color=c, fill=c] (4.62687,1.66712) rectangle (4.66667,1.77296);
\draw [color=c, fill=c] (4.66667,1.66712) rectangle (4.70647,1.77296);
\draw [color=c, fill=c] (4.70647,1.66712) rectangle (4.74627,1.77296);
\draw [color=c, fill=c] (4.74627,1.66712) rectangle (4.78607,1.77296);
\draw [color=c, fill=c] (4.78607,1.66712) rectangle (4.82587,1.77296);
\draw [color=c, fill=c] (4.82587,1.66712) rectangle (4.86567,1.77296);
\draw [color=c, fill=c] (4.86567,1.66712) rectangle (4.90547,1.77296);
\draw [color=c, fill=c] (4.90547,1.66712) rectangle (4.94527,1.77296);
\draw [color=c, fill=c] (4.94527,1.66712) rectangle (4.98507,1.77296);
\draw [color=c, fill=c] (4.98507,1.66712) rectangle (5.02488,1.77296);
\draw [color=c, fill=c] (5.02488,1.66712) rectangle (5.06468,1.77296);
\draw [color=c, fill=c] (5.06468,1.66712) rectangle (5.10448,1.77296);
\draw [color=c, fill=c] (5.10448,1.66712) rectangle (5.14428,1.77296);
\draw [color=c, fill=c] (5.14428,1.66712) rectangle (5.18408,1.77296);
\draw [color=c, fill=c] (5.18408,1.66712) rectangle (5.22388,1.77296);
\draw [color=c, fill=c] (5.22388,1.66712) rectangle (5.26368,1.77296);
\draw [color=c, fill=c] (5.26368,1.66712) rectangle (5.30348,1.77296);
\draw [color=c, fill=c] (5.30348,1.66712) rectangle (5.34328,1.77296);
\draw [color=c, fill=c] (5.34328,1.66712) rectangle (5.38308,1.77296);
\draw [color=c, fill=c] (5.38308,1.66712) rectangle (5.42289,1.77296);
\draw [color=c, fill=c] (5.42289,1.66712) rectangle (5.46269,1.77296);
\draw [color=c, fill=c] (5.46269,1.66712) rectangle (5.50249,1.77296);
\draw [color=c, fill=c] (5.50249,1.66712) rectangle (5.54229,1.77296);
\draw [color=c, fill=c] (5.54229,1.66712) rectangle (5.58209,1.77296);
\draw [color=c, fill=c] (5.58209,1.66712) rectangle (5.62189,1.77296);
\draw [color=c, fill=c] (5.62189,1.66712) rectangle (5.66169,1.77296);
\draw [color=c, fill=c] (5.66169,1.66712) rectangle (5.70149,1.77296);
\draw [color=c, fill=c] (5.70149,1.66712) rectangle (5.74129,1.77296);
\draw [color=c, fill=c] (5.74129,1.66712) rectangle (5.78109,1.77296);
\draw [color=c, fill=c] (5.78109,1.66712) rectangle (5.8209,1.77296);
\draw [color=c, fill=c] (5.8209,1.66712) rectangle (5.8607,1.77296);
\draw [color=c, fill=c] (5.8607,1.66712) rectangle (5.9005,1.77296);
\draw [color=c, fill=c] (5.9005,1.66712) rectangle (5.9403,1.77296);
\draw [color=c, fill=c] (5.9403,1.66712) rectangle (5.9801,1.77296);
\draw [color=c, fill=c] (5.9801,1.66712) rectangle (6.0199,1.77296);
\draw [color=c, fill=c] (6.0199,1.66712) rectangle (6.0597,1.77296);
\draw [color=c, fill=c] (6.0597,1.66712) rectangle (6.0995,1.77296);
\draw [color=c, fill=c] (6.0995,1.66712) rectangle (6.1393,1.77296);
\draw [color=c, fill=c] (6.1393,1.66712) rectangle (6.1791,1.77296);
\draw [color=c, fill=c] (6.1791,1.66712) rectangle (6.21891,1.77296);
\draw [color=c, fill=c] (6.21891,1.66712) rectangle (6.25871,1.77296);
\draw [color=c, fill=c] (6.25871,1.66712) rectangle (6.29851,1.77296);
\draw [color=c, fill=c] (6.29851,1.66712) rectangle (6.33831,1.77296);
\draw [color=c, fill=c] (6.33831,1.66712) rectangle (6.37811,1.77296);
\draw [color=c, fill=c] (6.37811,1.66712) rectangle (6.41791,1.77296);
\draw [color=c, fill=c] (6.41791,1.66712) rectangle (6.45771,1.77296);
\draw [color=c, fill=c] (6.45771,1.66712) rectangle (6.49751,1.77296);
\draw [color=c, fill=c] (6.49751,1.66712) rectangle (6.53731,1.77296);
\draw [color=c, fill=c] (6.53731,1.66712) rectangle (6.57711,1.77296);
\draw [color=c, fill=c] (6.57711,1.66712) rectangle (6.61692,1.77296);
\draw [color=c, fill=c] (6.61692,1.66712) rectangle (6.65672,1.77296);
\draw [color=c, fill=c] (6.65672,1.66712) rectangle (6.69652,1.77296);
\draw [color=c, fill=c] (6.69652,1.66712) rectangle (6.73632,1.77296);
\draw [color=c, fill=c] (6.73632,1.66712) rectangle (6.77612,1.77296);
\draw [color=c, fill=c] (6.77612,1.66712) rectangle (6.81592,1.77296);
\draw [color=c, fill=c] (6.81592,1.66712) rectangle (6.85572,1.77296);
\draw [color=c, fill=c] (6.85572,1.66712) rectangle (6.89552,1.77296);
\draw [color=c, fill=c] (6.89552,1.66712) rectangle (6.93532,1.77296);
\draw [color=c, fill=c] (6.93532,1.66712) rectangle (6.97512,1.77296);
\draw [color=c, fill=c] (6.97512,1.66712) rectangle (7.01493,1.77296);
\draw [color=c, fill=c] (7.01493,1.66712) rectangle (7.05473,1.77296);
\draw [color=c, fill=c] (7.05473,1.66712) rectangle (7.09453,1.77296);
\draw [color=c, fill=c] (7.09453,1.66712) rectangle (7.13433,1.77296);
\draw [color=c, fill=c] (7.13433,1.66712) rectangle (7.17413,1.77296);
\draw [color=c, fill=c] (7.17413,1.66712) rectangle (7.21393,1.77296);
\draw [color=c, fill=c] (7.21393,1.66712) rectangle (7.25373,1.77296);
\draw [color=c, fill=c] (7.25373,1.66712) rectangle (7.29353,1.77296);
\draw [color=c, fill=c] (7.29353,1.66712) rectangle (7.33333,1.77296);
\draw [color=c, fill=c] (7.33333,1.66712) rectangle (7.37313,1.77296);
\draw [color=c, fill=c] (7.37313,1.66712) rectangle (7.41294,1.77296);
\draw [color=c, fill=c] (7.41294,1.66712) rectangle (7.45274,1.77296);
\draw [color=c, fill=c] (7.45274,1.66712) rectangle (7.49254,1.77296);
\draw [color=c, fill=c] (7.49254,1.66712) rectangle (7.53234,1.77296);
\draw [color=c, fill=c] (7.53234,1.66712) rectangle (7.57214,1.77296);
\draw [color=c, fill=c] (7.57214,1.66712) rectangle (7.61194,1.77296);
\draw [color=c, fill=c] (7.61194,1.66712) rectangle (7.65174,1.77296);
\draw [color=c, fill=c] (7.65174,1.66712) rectangle (7.69154,1.77296);
\draw [color=c, fill=c] (7.69154,1.66712) rectangle (7.73134,1.77296);
\draw [color=c, fill=c] (7.73134,1.66712) rectangle (7.77114,1.77296);
\definecolor{c}{rgb}{1,0.186667,0};
\draw [color=c, fill=c] (7.77114,1.66712) rectangle (7.81095,1.77296);
\draw [color=c, fill=c] (7.81095,1.66712) rectangle (7.85075,1.77296);
\draw [color=c, fill=c] (7.85075,1.66712) rectangle (7.89055,1.77296);
\draw [color=c, fill=c] (7.89055,1.66712) rectangle (7.93035,1.77296);
\draw [color=c, fill=c] (7.93035,1.66712) rectangle (7.97015,1.77296);
\draw [color=c, fill=c] (7.97015,1.66712) rectangle (8.00995,1.77296);
\draw [color=c, fill=c] (8.00995,1.66712) rectangle (8.04975,1.77296);
\draw [color=c, fill=c] (8.04975,1.66712) rectangle (8.08955,1.77296);
\draw [color=c, fill=c] (8.08955,1.66712) rectangle (8.12935,1.77296);
\draw [color=c, fill=c] (8.12935,1.66712) rectangle (8.16915,1.77296);
\draw [color=c, fill=c] (8.16915,1.66712) rectangle (8.20895,1.77296);
\draw [color=c, fill=c] (8.20895,1.66712) rectangle (8.24876,1.77296);
\draw [color=c, fill=c] (8.24876,1.66712) rectangle (8.28856,1.77296);
\draw [color=c, fill=c] (8.28856,1.66712) rectangle (8.32836,1.77296);
\draw [color=c, fill=c] (8.32836,1.66712) rectangle (8.36816,1.77296);
\draw [color=c, fill=c] (8.36816,1.66712) rectangle (8.40796,1.77296);
\draw [color=c, fill=c] (8.40796,1.66712) rectangle (8.44776,1.77296);
\draw [color=c, fill=c] (8.44776,1.66712) rectangle (8.48756,1.77296);
\draw [color=c, fill=c] (8.48756,1.66712) rectangle (8.52736,1.77296);
\draw [color=c, fill=c] (8.52736,1.66712) rectangle (8.56716,1.77296);
\draw [color=c, fill=c] (8.56716,1.66712) rectangle (8.60697,1.77296);
\definecolor{c}{rgb}{1,0.466667,0};
\draw [color=c, fill=c] (8.60697,1.66712) rectangle (8.64677,1.77296);
\draw [color=c, fill=c] (8.64677,1.66712) rectangle (8.68657,1.77296);
\draw [color=c, fill=c] (8.68657,1.66712) rectangle (8.72637,1.77296);
\draw [color=c, fill=c] (8.72637,1.66712) rectangle (8.76617,1.77296);
\draw [color=c, fill=c] (8.76617,1.66712) rectangle (8.80597,1.77296);
\draw [color=c, fill=c] (8.80597,1.66712) rectangle (8.84577,1.77296);
\draw [color=c, fill=c] (8.84577,1.66712) rectangle (8.88557,1.77296);
\draw [color=c, fill=c] (8.88557,1.66712) rectangle (8.92537,1.77296);
\draw [color=c, fill=c] (8.92537,1.66712) rectangle (8.96517,1.77296);
\draw [color=c, fill=c] (8.96517,1.66712) rectangle (9.00498,1.77296);
\draw [color=c, fill=c] (9.00498,1.66712) rectangle (9.04478,1.77296);
\definecolor{c}{rgb}{1,0.653333,0};
\draw [color=c, fill=c] (9.04478,1.66712) rectangle (9.08458,1.77296);
\draw [color=c, fill=c] (9.08458,1.66712) rectangle (9.12438,1.77296);
\draw [color=c, fill=c] (9.12438,1.66712) rectangle (9.16418,1.77296);
\draw [color=c, fill=c] (9.16418,1.66712) rectangle (9.20398,1.77296);
\draw [color=c, fill=c] (9.20398,1.66712) rectangle (9.24378,1.77296);
\draw [color=c, fill=c] (9.24378,1.66712) rectangle (9.28358,1.77296);
\draw [color=c, fill=c] (9.28358,1.66712) rectangle (9.32338,1.77296);
\definecolor{c}{rgb}{1,0.933333,0};
\draw [color=c, fill=c] (9.32338,1.66712) rectangle (9.36318,1.77296);
\draw [color=c, fill=c] (9.36318,1.66712) rectangle (9.40298,1.77296);
\draw [color=c, fill=c] (9.40298,1.66712) rectangle (9.44279,1.77296);
\draw [color=c, fill=c] (9.44279,1.66712) rectangle (9.48259,1.77296);
\draw [color=c, fill=c] (9.48259,1.66712) rectangle (9.52239,1.77296);
\definecolor{c}{rgb}{0.88,1,0};
\draw [color=c, fill=c] (9.52239,1.66712) rectangle (9.56219,1.77296);
\draw [color=c, fill=c] (9.56219,1.66712) rectangle (9.60199,1.77296);
\draw [color=c, fill=c] (9.60199,1.66712) rectangle (9.64179,1.77296);
\draw [color=c, fill=c] (9.64179,1.66712) rectangle (9.68159,1.77296);
\draw [color=c, fill=c] (9.68159,1.66712) rectangle (9.72139,1.77296);
\definecolor{c}{rgb}{0.6,1,0};
\draw [color=c, fill=c] (9.72139,1.66712) rectangle (9.76119,1.77296);
\draw [color=c, fill=c] (9.76119,1.66712) rectangle (9.80099,1.77296);
\draw [color=c, fill=c] (9.80099,1.66712) rectangle (9.8408,1.77296);
\draw [color=c, fill=c] (9.8408,1.66712) rectangle (9.8806,1.77296);
\definecolor{c}{rgb}{0.413333,1,0};
\draw [color=c, fill=c] (9.8806,1.66712) rectangle (9.9204,1.77296);
\draw [color=c, fill=c] (9.9204,1.66712) rectangle (9.9602,1.77296);
\draw [color=c, fill=c] (9.9602,1.66712) rectangle (10,1.77296);
\draw [color=c, fill=c] (10,1.66712) rectangle (10.0398,1.77296);
\definecolor{c}{rgb}{0.133333,1,0};
\draw [color=c, fill=c] (10.0398,1.66712) rectangle (10.0796,1.77296);
\draw [color=c, fill=c] (10.0796,1.66712) rectangle (10.1194,1.77296);
\draw [color=c, fill=c] (10.1194,1.66712) rectangle (10.1592,1.77296);
\draw [color=c, fill=c] (10.1592,1.66712) rectangle (10.199,1.77296);
\draw [color=c, fill=c] (10.199,1.66712) rectangle (10.2388,1.77296);
\definecolor{c}{rgb}{0,1,0.0533333};
\draw [color=c, fill=c] (10.2388,1.66712) rectangle (10.2786,1.77296);
\draw [color=c, fill=c] (10.2786,1.66712) rectangle (10.3184,1.77296);
\draw [color=c, fill=c] (10.3184,1.66712) rectangle (10.3582,1.77296);
\draw [color=c, fill=c] (10.3582,1.66712) rectangle (10.398,1.77296);
\definecolor{c}{rgb}{0,1,0.333333};
\draw [color=c, fill=c] (10.398,1.66712) rectangle (10.4378,1.77296);
\draw [color=c, fill=c] (10.4378,1.66712) rectangle (10.4776,1.77296);
\draw [color=c, fill=c] (10.4776,1.66712) rectangle (10.5174,1.77296);
\draw [color=c, fill=c] (10.5174,1.66712) rectangle (10.5572,1.77296);
\draw [color=c, fill=c] (10.5572,1.66712) rectangle (10.597,1.77296);
\draw [color=c, fill=c] (10.597,1.66712) rectangle (10.6368,1.77296);
\definecolor{c}{rgb}{0,1,0.52};
\draw [color=c, fill=c] (10.6368,1.66712) rectangle (10.6766,1.77296);
\draw [color=c, fill=c] (10.6766,1.66712) rectangle (10.7164,1.77296);
\draw [color=c, fill=c] (10.7164,1.66712) rectangle (10.7562,1.77296);
\draw [color=c, fill=c] (10.7562,1.66712) rectangle (10.796,1.77296);
\draw [color=c, fill=c] (10.796,1.66712) rectangle (10.8358,1.77296);
\draw [color=c, fill=c] (10.8358,1.66712) rectangle (10.8756,1.77296);
\draw [color=c, fill=c] (10.8756,1.66712) rectangle (10.9154,1.77296);
\draw [color=c, fill=c] (10.9154,1.66712) rectangle (10.9552,1.77296);
\definecolor{c}{rgb}{0,1,0.8};
\draw [color=c, fill=c] (10.9552,1.66712) rectangle (10.995,1.77296);
\draw [color=c, fill=c] (10.995,1.66712) rectangle (11.0348,1.77296);
\draw [color=c, fill=c] (11.0348,1.66712) rectangle (11.0746,1.77296);
\draw [color=c, fill=c] (11.0746,1.66712) rectangle (11.1144,1.77296);
\draw [color=c, fill=c] (11.1144,1.66712) rectangle (11.1542,1.77296);
\draw [color=c, fill=c] (11.1542,1.66712) rectangle (11.194,1.77296);
\draw [color=c, fill=c] (11.194,1.66712) rectangle (11.2338,1.77296);
\draw [color=c, fill=c] (11.2338,1.66712) rectangle (11.2736,1.77296);
\draw [color=c, fill=c] (11.2736,1.66712) rectangle (11.3134,1.77296);
\draw [color=c, fill=c] (11.3134,1.66712) rectangle (11.3532,1.77296);
\draw [color=c, fill=c] (11.3532,1.66712) rectangle (11.393,1.77296);
\definecolor{c}{rgb}{0,1,0.986667};
\draw [color=c, fill=c] (11.393,1.66712) rectangle (11.4328,1.77296);
\draw [color=c, fill=c] (11.4328,1.66712) rectangle (11.4726,1.77296);
\draw [color=c, fill=c] (11.4726,1.66712) rectangle (11.5124,1.77296);
\draw [color=c, fill=c] (11.5124,1.66712) rectangle (11.5522,1.77296);
\draw [color=c, fill=c] (11.5522,1.66712) rectangle (11.592,1.77296);
\draw [color=c, fill=c] (11.592,1.66712) rectangle (11.6318,1.77296);
\draw [color=c, fill=c] (11.6318,1.66712) rectangle (11.6716,1.77296);
\draw [color=c, fill=c] (11.6716,1.66712) rectangle (11.7114,1.77296);
\draw [color=c, fill=c] (11.7114,1.66712) rectangle (11.7512,1.77296);
\draw [color=c, fill=c] (11.7512,1.66712) rectangle (11.791,1.77296);
\draw [color=c, fill=c] (11.791,1.66712) rectangle (11.8308,1.77296);
\draw [color=c, fill=c] (11.8308,1.66712) rectangle (11.8706,1.77296);
\draw [color=c, fill=c] (11.8706,1.66712) rectangle (11.9104,1.77296);
\draw [color=c, fill=c] (11.9104,1.66712) rectangle (11.9502,1.77296);
\draw [color=c, fill=c] (11.9502,1.66712) rectangle (11.99,1.77296);
\draw [color=c, fill=c] (11.99,1.66712) rectangle (12.0299,1.77296);
\draw [color=c, fill=c] (12.0299,1.66712) rectangle (12.0697,1.77296);
\draw [color=c, fill=c] (12.0697,1.66712) rectangle (12.1095,1.77296);
\draw [color=c, fill=c] (12.1095,1.66712) rectangle (12.1493,1.77296);
\draw [color=c, fill=c] (12.1493,1.66712) rectangle (12.1891,1.77296);
\draw [color=c, fill=c] (12.1891,1.66712) rectangle (12.2289,1.77296);
\draw [color=c, fill=c] (12.2289,1.66712) rectangle (12.2687,1.77296);
\draw [color=c, fill=c] (12.2687,1.66712) rectangle (12.3085,1.77296);
\draw [color=c, fill=c] (12.3085,1.66712) rectangle (12.3483,1.77296);
\definecolor{c}{rgb}{0,0.733333,1};
\draw [color=c, fill=c] (12.3483,1.66712) rectangle (12.3881,1.77296);
\draw [color=c, fill=c] (12.3881,1.66712) rectangle (12.4279,1.77296);
\draw [color=c, fill=c] (12.4279,1.66712) rectangle (12.4677,1.77296);
\draw [color=c, fill=c] (12.4677,1.66712) rectangle (12.5075,1.77296);
\draw [color=c, fill=c] (12.5075,1.66712) rectangle (12.5473,1.77296);
\draw [color=c, fill=c] (12.5473,1.66712) rectangle (12.5871,1.77296);
\draw [color=c, fill=c] (12.5871,1.66712) rectangle (12.6269,1.77296);
\draw [color=c, fill=c] (12.6269,1.66712) rectangle (12.6667,1.77296);
\draw [color=c, fill=c] (12.6667,1.66712) rectangle (12.7065,1.77296);
\draw [color=c, fill=c] (12.7065,1.66712) rectangle (12.7463,1.77296);
\draw [color=c, fill=c] (12.7463,1.66712) rectangle (12.7861,1.77296);
\draw [color=c, fill=c] (12.7861,1.66712) rectangle (12.8259,1.77296);
\draw [color=c, fill=c] (12.8259,1.66712) rectangle (12.8657,1.77296);
\draw [color=c, fill=c] (12.8657,1.66712) rectangle (12.9055,1.77296);
\draw [color=c, fill=c] (12.9055,1.66712) rectangle (12.9453,1.77296);
\draw [color=c, fill=c] (12.9453,1.66712) rectangle (12.9851,1.77296);
\draw [color=c, fill=c] (12.9851,1.66712) rectangle (13.0249,1.77296);
\draw [color=c, fill=c] (13.0249,1.66712) rectangle (13.0647,1.77296);
\draw [color=c, fill=c] (13.0647,1.66712) rectangle (13.1045,1.77296);
\draw [color=c, fill=c] (13.1045,1.66712) rectangle (13.1443,1.77296);
\draw [color=c, fill=c] (13.1443,1.66712) rectangle (13.1841,1.77296);
\draw [color=c, fill=c] (13.1841,1.66712) rectangle (13.2239,1.77296);
\draw [color=c, fill=c] (13.2239,1.66712) rectangle (13.2637,1.77296);
\draw [color=c, fill=c] (13.2637,1.66712) rectangle (13.3035,1.77296);
\draw [color=c, fill=c] (13.3035,1.66712) rectangle (13.3433,1.77296);
\draw [color=c, fill=c] (13.3433,1.66712) rectangle (13.3831,1.77296);
\draw [color=c, fill=c] (13.3831,1.66712) rectangle (13.4229,1.77296);
\draw [color=c, fill=c] (13.4229,1.66712) rectangle (13.4627,1.77296);
\draw [color=c, fill=c] (13.4627,1.66712) rectangle (13.5025,1.77296);
\draw [color=c, fill=c] (13.5025,1.66712) rectangle (13.5423,1.77296);
\draw [color=c, fill=c] (13.5423,1.66712) rectangle (13.5821,1.77296);
\draw [color=c, fill=c] (13.5821,1.66712) rectangle (13.6219,1.77296);
\draw [color=c, fill=c] (13.6219,1.66712) rectangle (13.6617,1.77296);
\draw [color=c, fill=c] (13.6617,1.66712) rectangle (13.7015,1.77296);
\draw [color=c, fill=c] (13.7015,1.66712) rectangle (13.7413,1.77296);
\draw [color=c, fill=c] (13.7413,1.66712) rectangle (13.7811,1.77296);
\draw [color=c, fill=c] (13.7811,1.66712) rectangle (13.8209,1.77296);
\draw [color=c, fill=c] (13.8209,1.66712) rectangle (13.8607,1.77296);
\draw [color=c, fill=c] (13.8607,1.66712) rectangle (13.9005,1.77296);
\draw [color=c, fill=c] (13.9005,1.66712) rectangle (13.9403,1.77296);
\draw [color=c, fill=c] (13.9403,1.66712) rectangle (13.9801,1.77296);
\draw [color=c, fill=c] (13.9801,1.66712) rectangle (14.0199,1.77296);
\draw [color=c, fill=c] (14.0199,1.66712) rectangle (14.0597,1.77296);
\draw [color=c, fill=c] (14.0597,1.66712) rectangle (14.0995,1.77296);
\draw [color=c, fill=c] (14.0995,1.66712) rectangle (14.1393,1.77296);
\draw [color=c, fill=c] (14.1393,1.66712) rectangle (14.1791,1.77296);
\draw [color=c, fill=c] (14.1791,1.66712) rectangle (14.2189,1.77296);
\draw [color=c, fill=c] (14.2189,1.66712) rectangle (14.2587,1.77296);
\draw [color=c, fill=c] (14.2587,1.66712) rectangle (14.2985,1.77296);
\draw [color=c, fill=c] (14.2985,1.66712) rectangle (14.3383,1.77296);
\draw [color=c, fill=c] (14.3383,1.66712) rectangle (14.3781,1.77296);
\draw [color=c, fill=c] (14.3781,1.66712) rectangle (14.4179,1.77296);
\draw [color=c, fill=c] (14.4179,1.66712) rectangle (14.4577,1.77296);
\draw [color=c, fill=c] (14.4577,1.66712) rectangle (14.4975,1.77296);
\draw [color=c, fill=c] (14.4975,1.66712) rectangle (14.5373,1.77296);
\draw [color=c, fill=c] (14.5373,1.66712) rectangle (14.5771,1.77296);
\draw [color=c, fill=c] (14.5771,1.66712) rectangle (14.6169,1.77296);
\draw [color=c, fill=c] (14.6169,1.66712) rectangle (14.6567,1.77296);
\draw [color=c, fill=c] (14.6567,1.66712) rectangle (14.6965,1.77296);
\draw [color=c, fill=c] (14.6965,1.66712) rectangle (14.7363,1.77296);
\draw [color=c, fill=c] (14.7363,1.66712) rectangle (14.7761,1.77296);
\draw [color=c, fill=c] (14.7761,1.66712) rectangle (14.8159,1.77296);
\draw [color=c, fill=c] (14.8159,1.66712) rectangle (14.8557,1.77296);
\draw [color=c, fill=c] (14.8557,1.66712) rectangle (14.8955,1.77296);
\draw [color=c, fill=c] (14.8955,1.66712) rectangle (14.9353,1.77296);
\draw [color=c, fill=c] (14.9353,1.66712) rectangle (14.9751,1.77296);
\draw [color=c, fill=c] (14.9751,1.66712) rectangle (15.0149,1.77296);
\draw [color=c, fill=c] (15.0149,1.66712) rectangle (15.0547,1.77296);
\draw [color=c, fill=c] (15.0547,1.66712) rectangle (15.0945,1.77296);
\draw [color=c, fill=c] (15.0945,1.66712) rectangle (15.1343,1.77296);
\draw [color=c, fill=c] (15.1343,1.66712) rectangle (15.1741,1.77296);
\draw [color=c, fill=c] (15.1741,1.66712) rectangle (15.2139,1.77296);
\draw [color=c, fill=c] (15.2139,1.66712) rectangle (15.2537,1.77296);
\draw [color=c, fill=c] (15.2537,1.66712) rectangle (15.2935,1.77296);
\draw [color=c, fill=c] (15.2935,1.66712) rectangle (15.3333,1.77296);
\draw [color=c, fill=c] (15.3333,1.66712) rectangle (15.3731,1.77296);
\draw [color=c, fill=c] (15.3731,1.66712) rectangle (15.4129,1.77296);
\draw [color=c, fill=c] (15.4129,1.66712) rectangle (15.4527,1.77296);
\draw [color=c, fill=c] (15.4527,1.66712) rectangle (15.4925,1.77296);
\draw [color=c, fill=c] (15.4925,1.66712) rectangle (15.5323,1.77296);
\draw [color=c, fill=c] (15.5323,1.66712) rectangle (15.5721,1.77296);
\draw [color=c, fill=c] (15.5721,1.66712) rectangle (15.6119,1.77296);
\draw [color=c, fill=c] (15.6119,1.66712) rectangle (15.6517,1.77296);
\draw [color=c, fill=c] (15.6517,1.66712) rectangle (15.6915,1.77296);
\draw [color=c, fill=c] (15.6915,1.66712) rectangle (15.7313,1.77296);
\draw [color=c, fill=c] (15.7313,1.66712) rectangle (15.7711,1.77296);
\draw [color=c, fill=c] (15.7711,1.66712) rectangle (15.8109,1.77296);
\draw [color=c, fill=c] (15.8109,1.66712) rectangle (15.8507,1.77296);
\draw [color=c, fill=c] (15.8507,1.66712) rectangle (15.8905,1.77296);
\draw [color=c, fill=c] (15.8905,1.66712) rectangle (15.9303,1.77296);
\draw [color=c, fill=c] (15.9303,1.66712) rectangle (15.9701,1.77296);
\draw [color=c, fill=c] (15.9701,1.66712) rectangle (16.01,1.77296);
\draw [color=c, fill=c] (16.01,1.66712) rectangle (16.0498,1.77296);
\draw [color=c, fill=c] (16.0498,1.66712) rectangle (16.0896,1.77296);
\draw [color=c, fill=c] (16.0896,1.66712) rectangle (16.1294,1.77296);
\draw [color=c, fill=c] (16.1294,1.66712) rectangle (16.1692,1.77296);
\draw [color=c, fill=c] (16.1692,1.66712) rectangle (16.209,1.77296);
\draw [color=c, fill=c] (16.209,1.66712) rectangle (16.2488,1.77296);
\draw [color=c, fill=c] (16.2488,1.66712) rectangle (16.2886,1.77296);
\draw [color=c, fill=c] (16.2886,1.66712) rectangle (16.3284,1.77296);
\draw [color=c, fill=c] (16.3284,1.66712) rectangle (16.3682,1.77296);
\draw [color=c, fill=c] (16.3682,1.66712) rectangle (16.408,1.77296);
\draw [color=c, fill=c] (16.408,1.66712) rectangle (16.4478,1.77296);
\draw [color=c, fill=c] (16.4478,1.66712) rectangle (16.4876,1.77296);
\draw [color=c, fill=c] (16.4876,1.66712) rectangle (16.5274,1.77296);
\draw [color=c, fill=c] (16.5274,1.66712) rectangle (16.5672,1.77296);
\draw [color=c, fill=c] (16.5672,1.66712) rectangle (16.607,1.77296);
\draw [color=c, fill=c] (16.607,1.66712) rectangle (16.6468,1.77296);
\draw [color=c, fill=c] (16.6468,1.66712) rectangle (16.6866,1.77296);
\draw [color=c, fill=c] (16.6866,1.66712) rectangle (16.7264,1.77296);
\draw [color=c, fill=c] (16.7264,1.66712) rectangle (16.7662,1.77296);
\draw [color=c, fill=c] (16.7662,1.66712) rectangle (16.806,1.77296);
\draw [color=c, fill=c] (16.806,1.66712) rectangle (16.8458,1.77296);
\draw [color=c, fill=c] (16.8458,1.66712) rectangle (16.8856,1.77296);
\draw [color=c, fill=c] (16.8856,1.66712) rectangle (16.9254,1.77296);
\draw [color=c, fill=c] (16.9254,1.66712) rectangle (16.9652,1.77296);
\draw [color=c, fill=c] (16.9652,1.66712) rectangle (17.005,1.77296);
\draw [color=c, fill=c] (17.005,1.66712) rectangle (17.0448,1.77296);
\draw [color=c, fill=c] (17.0448,1.66712) rectangle (17.0846,1.77296);
\draw [color=c, fill=c] (17.0846,1.66712) rectangle (17.1244,1.77296);
\draw [color=c, fill=c] (17.1244,1.66712) rectangle (17.1642,1.77296);
\draw [color=c, fill=c] (17.1642,1.66712) rectangle (17.204,1.77296);
\draw [color=c, fill=c] (17.204,1.66712) rectangle (17.2438,1.77296);
\draw [color=c, fill=c] (17.2438,1.66712) rectangle (17.2836,1.77296);
\draw [color=c, fill=c] (17.2836,1.66712) rectangle (17.3234,1.77296);
\draw [color=c, fill=c] (17.3234,1.66712) rectangle (17.3632,1.77296);
\draw [color=c, fill=c] (17.3632,1.66712) rectangle (17.403,1.77296);
\draw [color=c, fill=c] (17.403,1.66712) rectangle (17.4428,1.77296);
\draw [color=c, fill=c] (17.4428,1.66712) rectangle (17.4826,1.77296);
\draw [color=c, fill=c] (17.4826,1.66712) rectangle (17.5224,1.77296);
\draw [color=c, fill=c] (17.5224,1.66712) rectangle (17.5622,1.77296);
\draw [color=c, fill=c] (17.5622,1.66712) rectangle (17.602,1.77296);
\draw [color=c, fill=c] (17.602,1.66712) rectangle (17.6418,1.77296);
\draw [color=c, fill=c] (17.6418,1.66712) rectangle (17.6816,1.77296);
\draw [color=c, fill=c] (17.6816,1.66712) rectangle (17.7214,1.77296);
\draw [color=c, fill=c] (17.7214,1.66712) rectangle (17.7612,1.77296);
\draw [color=c, fill=c] (17.7612,1.66712) rectangle (17.801,1.77296);
\draw [color=c, fill=c] (17.801,1.66712) rectangle (17.8408,1.77296);
\draw [color=c, fill=c] (17.8408,1.66712) rectangle (17.8806,1.77296);
\draw [color=c, fill=c] (17.8806,1.66712) rectangle (17.9204,1.77296);
\draw [color=c, fill=c] (17.9204,1.66712) rectangle (17.9602,1.77296);
\draw [color=c, fill=c] (17.9602,1.66712) rectangle (18,1.77296);
\definecolor{c}{rgb}{1,0,0};
\draw [color=c, fill=c] (2,1.77296) rectangle (2.0398,1.87881);
\draw [color=c, fill=c] (2.0398,1.77296) rectangle (2.0796,1.87881);
\draw [color=c, fill=c] (2.0796,1.77296) rectangle (2.1194,1.87881);
\draw [color=c, fill=c] (2.1194,1.77296) rectangle (2.1592,1.87881);
\draw [color=c, fill=c] (2.1592,1.77296) rectangle (2.19901,1.87881);
\draw [color=c, fill=c] (2.19901,1.77296) rectangle (2.23881,1.87881);
\draw [color=c, fill=c] (2.23881,1.77296) rectangle (2.27861,1.87881);
\draw [color=c, fill=c] (2.27861,1.77296) rectangle (2.31841,1.87881);
\draw [color=c, fill=c] (2.31841,1.77296) rectangle (2.35821,1.87881);
\draw [color=c, fill=c] (2.35821,1.77296) rectangle (2.39801,1.87881);
\draw [color=c, fill=c] (2.39801,1.77296) rectangle (2.43781,1.87881);
\draw [color=c, fill=c] (2.43781,1.77296) rectangle (2.47761,1.87881);
\draw [color=c, fill=c] (2.47761,1.77296) rectangle (2.51741,1.87881);
\draw [color=c, fill=c] (2.51741,1.77296) rectangle (2.55721,1.87881);
\draw [color=c, fill=c] (2.55721,1.77296) rectangle (2.59702,1.87881);
\draw [color=c, fill=c] (2.59702,1.77296) rectangle (2.63682,1.87881);
\draw [color=c, fill=c] (2.63682,1.77296) rectangle (2.67662,1.87881);
\draw [color=c, fill=c] (2.67662,1.77296) rectangle (2.71642,1.87881);
\draw [color=c, fill=c] (2.71642,1.77296) rectangle (2.75622,1.87881);
\draw [color=c, fill=c] (2.75622,1.77296) rectangle (2.79602,1.87881);
\draw [color=c, fill=c] (2.79602,1.77296) rectangle (2.83582,1.87881);
\draw [color=c, fill=c] (2.83582,1.77296) rectangle (2.87562,1.87881);
\draw [color=c, fill=c] (2.87562,1.77296) rectangle (2.91542,1.87881);
\draw [color=c, fill=c] (2.91542,1.77296) rectangle (2.95522,1.87881);
\draw [color=c, fill=c] (2.95522,1.77296) rectangle (2.99502,1.87881);
\draw [color=c, fill=c] (2.99502,1.77296) rectangle (3.03483,1.87881);
\draw [color=c, fill=c] (3.03483,1.77296) rectangle (3.07463,1.87881);
\draw [color=c, fill=c] (3.07463,1.77296) rectangle (3.11443,1.87881);
\draw [color=c, fill=c] (3.11443,1.77296) rectangle (3.15423,1.87881);
\draw [color=c, fill=c] (3.15423,1.77296) rectangle (3.19403,1.87881);
\draw [color=c, fill=c] (3.19403,1.77296) rectangle (3.23383,1.87881);
\draw [color=c, fill=c] (3.23383,1.77296) rectangle (3.27363,1.87881);
\draw [color=c, fill=c] (3.27363,1.77296) rectangle (3.31343,1.87881);
\draw [color=c, fill=c] (3.31343,1.77296) rectangle (3.35323,1.87881);
\draw [color=c, fill=c] (3.35323,1.77296) rectangle (3.39303,1.87881);
\draw [color=c, fill=c] (3.39303,1.77296) rectangle (3.43284,1.87881);
\draw [color=c, fill=c] (3.43284,1.77296) rectangle (3.47264,1.87881);
\draw [color=c, fill=c] (3.47264,1.77296) rectangle (3.51244,1.87881);
\draw [color=c, fill=c] (3.51244,1.77296) rectangle (3.55224,1.87881);
\draw [color=c, fill=c] (3.55224,1.77296) rectangle (3.59204,1.87881);
\draw [color=c, fill=c] (3.59204,1.77296) rectangle (3.63184,1.87881);
\draw [color=c, fill=c] (3.63184,1.77296) rectangle (3.67164,1.87881);
\draw [color=c, fill=c] (3.67164,1.77296) rectangle (3.71144,1.87881);
\draw [color=c, fill=c] (3.71144,1.77296) rectangle (3.75124,1.87881);
\draw [color=c, fill=c] (3.75124,1.77296) rectangle (3.79104,1.87881);
\draw [color=c, fill=c] (3.79104,1.77296) rectangle (3.83085,1.87881);
\draw [color=c, fill=c] (3.83085,1.77296) rectangle (3.87065,1.87881);
\draw [color=c, fill=c] (3.87065,1.77296) rectangle (3.91045,1.87881);
\draw [color=c, fill=c] (3.91045,1.77296) rectangle (3.95025,1.87881);
\draw [color=c, fill=c] (3.95025,1.77296) rectangle (3.99005,1.87881);
\draw [color=c, fill=c] (3.99005,1.77296) rectangle (4.02985,1.87881);
\draw [color=c, fill=c] (4.02985,1.77296) rectangle (4.06965,1.87881);
\draw [color=c, fill=c] (4.06965,1.77296) rectangle (4.10945,1.87881);
\draw [color=c, fill=c] (4.10945,1.77296) rectangle (4.14925,1.87881);
\draw [color=c, fill=c] (4.14925,1.77296) rectangle (4.18905,1.87881);
\draw [color=c, fill=c] (4.18905,1.77296) rectangle (4.22886,1.87881);
\draw [color=c, fill=c] (4.22886,1.77296) rectangle (4.26866,1.87881);
\draw [color=c, fill=c] (4.26866,1.77296) rectangle (4.30846,1.87881);
\draw [color=c, fill=c] (4.30846,1.77296) rectangle (4.34826,1.87881);
\draw [color=c, fill=c] (4.34826,1.77296) rectangle (4.38806,1.87881);
\draw [color=c, fill=c] (4.38806,1.77296) rectangle (4.42786,1.87881);
\draw [color=c, fill=c] (4.42786,1.77296) rectangle (4.46766,1.87881);
\draw [color=c, fill=c] (4.46766,1.77296) rectangle (4.50746,1.87881);
\draw [color=c, fill=c] (4.50746,1.77296) rectangle (4.54726,1.87881);
\draw [color=c, fill=c] (4.54726,1.77296) rectangle (4.58706,1.87881);
\draw [color=c, fill=c] (4.58706,1.77296) rectangle (4.62687,1.87881);
\draw [color=c, fill=c] (4.62687,1.77296) rectangle (4.66667,1.87881);
\draw [color=c, fill=c] (4.66667,1.77296) rectangle (4.70647,1.87881);
\draw [color=c, fill=c] (4.70647,1.77296) rectangle (4.74627,1.87881);
\draw [color=c, fill=c] (4.74627,1.77296) rectangle (4.78607,1.87881);
\draw [color=c, fill=c] (4.78607,1.77296) rectangle (4.82587,1.87881);
\draw [color=c, fill=c] (4.82587,1.77296) rectangle (4.86567,1.87881);
\draw [color=c, fill=c] (4.86567,1.77296) rectangle (4.90547,1.87881);
\draw [color=c, fill=c] (4.90547,1.77296) rectangle (4.94527,1.87881);
\draw [color=c, fill=c] (4.94527,1.77296) rectangle (4.98507,1.87881);
\draw [color=c, fill=c] (4.98507,1.77296) rectangle (5.02488,1.87881);
\draw [color=c, fill=c] (5.02488,1.77296) rectangle (5.06468,1.87881);
\draw [color=c, fill=c] (5.06468,1.77296) rectangle (5.10448,1.87881);
\draw [color=c, fill=c] (5.10448,1.77296) rectangle (5.14428,1.87881);
\draw [color=c, fill=c] (5.14428,1.77296) rectangle (5.18408,1.87881);
\draw [color=c, fill=c] (5.18408,1.77296) rectangle (5.22388,1.87881);
\draw [color=c, fill=c] (5.22388,1.77296) rectangle (5.26368,1.87881);
\draw [color=c, fill=c] (5.26368,1.77296) rectangle (5.30348,1.87881);
\draw [color=c, fill=c] (5.30348,1.77296) rectangle (5.34328,1.87881);
\draw [color=c, fill=c] (5.34328,1.77296) rectangle (5.38308,1.87881);
\draw [color=c, fill=c] (5.38308,1.77296) rectangle (5.42289,1.87881);
\draw [color=c, fill=c] (5.42289,1.77296) rectangle (5.46269,1.87881);
\draw [color=c, fill=c] (5.46269,1.77296) rectangle (5.50249,1.87881);
\draw [color=c, fill=c] (5.50249,1.77296) rectangle (5.54229,1.87881);
\draw [color=c, fill=c] (5.54229,1.77296) rectangle (5.58209,1.87881);
\draw [color=c, fill=c] (5.58209,1.77296) rectangle (5.62189,1.87881);
\draw [color=c, fill=c] (5.62189,1.77296) rectangle (5.66169,1.87881);
\draw [color=c, fill=c] (5.66169,1.77296) rectangle (5.70149,1.87881);
\draw [color=c, fill=c] (5.70149,1.77296) rectangle (5.74129,1.87881);
\draw [color=c, fill=c] (5.74129,1.77296) rectangle (5.78109,1.87881);
\draw [color=c, fill=c] (5.78109,1.77296) rectangle (5.8209,1.87881);
\draw [color=c, fill=c] (5.8209,1.77296) rectangle (5.8607,1.87881);
\draw [color=c, fill=c] (5.8607,1.77296) rectangle (5.9005,1.87881);
\draw [color=c, fill=c] (5.9005,1.77296) rectangle (5.9403,1.87881);
\draw [color=c, fill=c] (5.9403,1.77296) rectangle (5.9801,1.87881);
\draw [color=c, fill=c] (5.9801,1.77296) rectangle (6.0199,1.87881);
\draw [color=c, fill=c] (6.0199,1.77296) rectangle (6.0597,1.87881);
\draw [color=c, fill=c] (6.0597,1.77296) rectangle (6.0995,1.87881);
\draw [color=c, fill=c] (6.0995,1.77296) rectangle (6.1393,1.87881);
\draw [color=c, fill=c] (6.1393,1.77296) rectangle (6.1791,1.87881);
\draw [color=c, fill=c] (6.1791,1.77296) rectangle (6.21891,1.87881);
\draw [color=c, fill=c] (6.21891,1.77296) rectangle (6.25871,1.87881);
\draw [color=c, fill=c] (6.25871,1.77296) rectangle (6.29851,1.87881);
\draw [color=c, fill=c] (6.29851,1.77296) rectangle (6.33831,1.87881);
\draw [color=c, fill=c] (6.33831,1.77296) rectangle (6.37811,1.87881);
\draw [color=c, fill=c] (6.37811,1.77296) rectangle (6.41791,1.87881);
\draw [color=c, fill=c] (6.41791,1.77296) rectangle (6.45771,1.87881);
\draw [color=c, fill=c] (6.45771,1.77296) rectangle (6.49751,1.87881);
\draw [color=c, fill=c] (6.49751,1.77296) rectangle (6.53731,1.87881);
\draw [color=c, fill=c] (6.53731,1.77296) rectangle (6.57711,1.87881);
\draw [color=c, fill=c] (6.57711,1.77296) rectangle (6.61692,1.87881);
\draw [color=c, fill=c] (6.61692,1.77296) rectangle (6.65672,1.87881);
\draw [color=c, fill=c] (6.65672,1.77296) rectangle (6.69652,1.87881);
\draw [color=c, fill=c] (6.69652,1.77296) rectangle (6.73632,1.87881);
\draw [color=c, fill=c] (6.73632,1.77296) rectangle (6.77612,1.87881);
\draw [color=c, fill=c] (6.77612,1.77296) rectangle (6.81592,1.87881);
\draw [color=c, fill=c] (6.81592,1.77296) rectangle (6.85572,1.87881);
\draw [color=c, fill=c] (6.85572,1.77296) rectangle (6.89552,1.87881);
\draw [color=c, fill=c] (6.89552,1.77296) rectangle (6.93532,1.87881);
\draw [color=c, fill=c] (6.93532,1.77296) rectangle (6.97512,1.87881);
\draw [color=c, fill=c] (6.97512,1.77296) rectangle (7.01493,1.87881);
\draw [color=c, fill=c] (7.01493,1.77296) rectangle (7.05473,1.87881);
\draw [color=c, fill=c] (7.05473,1.77296) rectangle (7.09453,1.87881);
\draw [color=c, fill=c] (7.09453,1.77296) rectangle (7.13433,1.87881);
\draw [color=c, fill=c] (7.13433,1.77296) rectangle (7.17413,1.87881);
\draw [color=c, fill=c] (7.17413,1.77296) rectangle (7.21393,1.87881);
\draw [color=c, fill=c] (7.21393,1.77296) rectangle (7.25373,1.87881);
\draw [color=c, fill=c] (7.25373,1.77296) rectangle (7.29353,1.87881);
\draw [color=c, fill=c] (7.29353,1.77296) rectangle (7.33333,1.87881);
\draw [color=c, fill=c] (7.33333,1.77296) rectangle (7.37313,1.87881);
\draw [color=c, fill=c] (7.37313,1.77296) rectangle (7.41294,1.87881);
\draw [color=c, fill=c] (7.41294,1.77296) rectangle (7.45274,1.87881);
\draw [color=c, fill=c] (7.45274,1.77296) rectangle (7.49254,1.87881);
\draw [color=c, fill=c] (7.49254,1.77296) rectangle (7.53234,1.87881);
\draw [color=c, fill=c] (7.53234,1.77296) rectangle (7.57214,1.87881);
\draw [color=c, fill=c] (7.57214,1.77296) rectangle (7.61194,1.87881);
\draw [color=c, fill=c] (7.61194,1.77296) rectangle (7.65174,1.87881);
\draw [color=c, fill=c] (7.65174,1.77296) rectangle (7.69154,1.87881);
\draw [color=c, fill=c] (7.69154,1.77296) rectangle (7.73134,1.87881);
\draw [color=c, fill=c] (7.73134,1.77296) rectangle (7.77114,1.87881);
\draw [color=c, fill=c] (7.77114,1.77296) rectangle (7.81095,1.87881);
\definecolor{c}{rgb}{1,0.186667,0};
\draw [color=c, fill=c] (7.81095,1.77296) rectangle (7.85075,1.87881);
\draw [color=c, fill=c] (7.85075,1.77296) rectangle (7.89055,1.87881);
\draw [color=c, fill=c] (7.89055,1.77296) rectangle (7.93035,1.87881);
\draw [color=c, fill=c] (7.93035,1.77296) rectangle (7.97015,1.87881);
\draw [color=c, fill=c] (7.97015,1.77296) rectangle (8.00995,1.87881);
\draw [color=c, fill=c] (8.00995,1.77296) rectangle (8.04975,1.87881);
\draw [color=c, fill=c] (8.04975,1.77296) rectangle (8.08955,1.87881);
\draw [color=c, fill=c] (8.08955,1.77296) rectangle (8.12935,1.87881);
\draw [color=c, fill=c] (8.12935,1.77296) rectangle (8.16915,1.87881);
\draw [color=c, fill=c] (8.16915,1.77296) rectangle (8.20895,1.87881);
\draw [color=c, fill=c] (8.20895,1.77296) rectangle (8.24876,1.87881);
\draw [color=c, fill=c] (8.24876,1.77296) rectangle (8.28856,1.87881);
\draw [color=c, fill=c] (8.28856,1.77296) rectangle (8.32836,1.87881);
\draw [color=c, fill=c] (8.32836,1.77296) rectangle (8.36816,1.87881);
\draw [color=c, fill=c] (8.36816,1.77296) rectangle (8.40796,1.87881);
\draw [color=c, fill=c] (8.40796,1.77296) rectangle (8.44776,1.87881);
\draw [color=c, fill=c] (8.44776,1.77296) rectangle (8.48756,1.87881);
\draw [color=c, fill=c] (8.48756,1.77296) rectangle (8.52736,1.87881);
\draw [color=c, fill=c] (8.52736,1.77296) rectangle (8.56716,1.87881);
\draw [color=c, fill=c] (8.56716,1.77296) rectangle (8.60697,1.87881);
\definecolor{c}{rgb}{1,0.466667,0};
\draw [color=c, fill=c] (8.60697,1.77296) rectangle (8.64677,1.87881);
\draw [color=c, fill=c] (8.64677,1.77296) rectangle (8.68657,1.87881);
\draw [color=c, fill=c] (8.68657,1.77296) rectangle (8.72637,1.87881);
\draw [color=c, fill=c] (8.72637,1.77296) rectangle (8.76617,1.87881);
\draw [color=c, fill=c] (8.76617,1.77296) rectangle (8.80597,1.87881);
\draw [color=c, fill=c] (8.80597,1.77296) rectangle (8.84577,1.87881);
\draw [color=c, fill=c] (8.84577,1.77296) rectangle (8.88557,1.87881);
\draw [color=c, fill=c] (8.88557,1.77296) rectangle (8.92537,1.87881);
\draw [color=c, fill=c] (8.92537,1.77296) rectangle (8.96517,1.87881);
\draw [color=c, fill=c] (8.96517,1.77296) rectangle (9.00498,1.87881);
\draw [color=c, fill=c] (9.00498,1.77296) rectangle (9.04478,1.87881);
\definecolor{c}{rgb}{1,0.653333,0};
\draw [color=c, fill=c] (9.04478,1.77296) rectangle (9.08458,1.87881);
\draw [color=c, fill=c] (9.08458,1.77296) rectangle (9.12438,1.87881);
\draw [color=c, fill=c] (9.12438,1.77296) rectangle (9.16418,1.87881);
\draw [color=c, fill=c] (9.16418,1.77296) rectangle (9.20398,1.87881);
\draw [color=c, fill=c] (9.20398,1.77296) rectangle (9.24378,1.87881);
\draw [color=c, fill=c] (9.24378,1.77296) rectangle (9.28358,1.87881);
\draw [color=c, fill=c] (9.28358,1.77296) rectangle (9.32338,1.87881);
\definecolor{c}{rgb}{1,0.933333,0};
\draw [color=c, fill=c] (9.32338,1.77296) rectangle (9.36318,1.87881);
\draw [color=c, fill=c] (9.36318,1.77296) rectangle (9.40298,1.87881);
\draw [color=c, fill=c] (9.40298,1.77296) rectangle (9.44279,1.87881);
\draw [color=c, fill=c] (9.44279,1.77296) rectangle (9.48259,1.87881);
\draw [color=c, fill=c] (9.48259,1.77296) rectangle (9.52239,1.87881);
\draw [color=c, fill=c] (9.52239,1.77296) rectangle (9.56219,1.87881);
\definecolor{c}{rgb}{0.88,1,0};
\draw [color=c, fill=c] (9.56219,1.77296) rectangle (9.60199,1.87881);
\draw [color=c, fill=c] (9.60199,1.77296) rectangle (9.64179,1.87881);
\draw [color=c, fill=c] (9.64179,1.77296) rectangle (9.68159,1.87881);
\draw [color=c, fill=c] (9.68159,1.77296) rectangle (9.72139,1.87881);
\definecolor{c}{rgb}{0.6,1,0};
\draw [color=c, fill=c] (9.72139,1.77296) rectangle (9.76119,1.87881);
\draw [color=c, fill=c] (9.76119,1.77296) rectangle (9.80099,1.87881);
\draw [color=c, fill=c] (9.80099,1.77296) rectangle (9.8408,1.87881);
\draw [color=c, fill=c] (9.8408,1.77296) rectangle (9.8806,1.87881);
\definecolor{c}{rgb}{0.413333,1,0};
\draw [color=c, fill=c] (9.8806,1.77296) rectangle (9.9204,1.87881);
\draw [color=c, fill=c] (9.9204,1.77296) rectangle (9.9602,1.87881);
\draw [color=c, fill=c] (9.9602,1.77296) rectangle (10,1.87881);
\draw [color=c, fill=c] (10,1.77296) rectangle (10.0398,1.87881);
\definecolor{c}{rgb}{0.133333,1,0};
\draw [color=c, fill=c] (10.0398,1.77296) rectangle (10.0796,1.87881);
\draw [color=c, fill=c] (10.0796,1.77296) rectangle (10.1194,1.87881);
\draw [color=c, fill=c] (10.1194,1.77296) rectangle (10.1592,1.87881);
\draw [color=c, fill=c] (10.1592,1.77296) rectangle (10.199,1.87881);
\definecolor{c}{rgb}{0,1,0.0533333};
\draw [color=c, fill=c] (10.199,1.77296) rectangle (10.2388,1.87881);
\draw [color=c, fill=c] (10.2388,1.77296) rectangle (10.2786,1.87881);
\draw [color=c, fill=c] (10.2786,1.77296) rectangle (10.3184,1.87881);
\draw [color=c, fill=c] (10.3184,1.77296) rectangle (10.3582,1.87881);
\draw [color=c, fill=c] (10.3582,1.77296) rectangle (10.398,1.87881);
\definecolor{c}{rgb}{0,1,0.333333};
\draw [color=c, fill=c] (10.398,1.77296) rectangle (10.4378,1.87881);
\draw [color=c, fill=c] (10.4378,1.77296) rectangle (10.4776,1.87881);
\draw [color=c, fill=c] (10.4776,1.77296) rectangle (10.5174,1.87881);
\draw [color=c, fill=c] (10.5174,1.77296) rectangle (10.5572,1.87881);
\draw [color=c, fill=c] (10.5572,1.77296) rectangle (10.597,1.87881);
\draw [color=c, fill=c] (10.597,1.77296) rectangle (10.6368,1.87881);
\definecolor{c}{rgb}{0,1,0.52};
\draw [color=c, fill=c] (10.6368,1.77296) rectangle (10.6766,1.87881);
\draw [color=c, fill=c] (10.6766,1.77296) rectangle (10.7164,1.87881);
\draw [color=c, fill=c] (10.7164,1.77296) rectangle (10.7562,1.87881);
\draw [color=c, fill=c] (10.7562,1.77296) rectangle (10.796,1.87881);
\draw [color=c, fill=c] (10.796,1.77296) rectangle (10.8358,1.87881);
\draw [color=c, fill=c] (10.8358,1.77296) rectangle (10.8756,1.87881);
\draw [color=c, fill=c] (10.8756,1.77296) rectangle (10.9154,1.87881);
\draw [color=c, fill=c] (10.9154,1.77296) rectangle (10.9552,1.87881);
\definecolor{c}{rgb}{0,1,0.8};
\draw [color=c, fill=c] (10.9552,1.77296) rectangle (10.995,1.87881);
\draw [color=c, fill=c] (10.995,1.77296) rectangle (11.0348,1.87881);
\draw [color=c, fill=c] (11.0348,1.77296) rectangle (11.0746,1.87881);
\draw [color=c, fill=c] (11.0746,1.77296) rectangle (11.1144,1.87881);
\draw [color=c, fill=c] (11.1144,1.77296) rectangle (11.1542,1.87881);
\draw [color=c, fill=c] (11.1542,1.77296) rectangle (11.194,1.87881);
\draw [color=c, fill=c] (11.194,1.77296) rectangle (11.2338,1.87881);
\draw [color=c, fill=c] (11.2338,1.77296) rectangle (11.2736,1.87881);
\draw [color=c, fill=c] (11.2736,1.77296) rectangle (11.3134,1.87881);
\draw [color=c, fill=c] (11.3134,1.77296) rectangle (11.3532,1.87881);
\draw [color=c, fill=c] (11.3532,1.77296) rectangle (11.393,1.87881);
\definecolor{c}{rgb}{0,1,0.986667};
\draw [color=c, fill=c] (11.393,1.77296) rectangle (11.4328,1.87881);
\draw [color=c, fill=c] (11.4328,1.77296) rectangle (11.4726,1.87881);
\draw [color=c, fill=c] (11.4726,1.77296) rectangle (11.5124,1.87881);
\draw [color=c, fill=c] (11.5124,1.77296) rectangle (11.5522,1.87881);
\draw [color=c, fill=c] (11.5522,1.77296) rectangle (11.592,1.87881);
\draw [color=c, fill=c] (11.592,1.77296) rectangle (11.6318,1.87881);
\draw [color=c, fill=c] (11.6318,1.77296) rectangle (11.6716,1.87881);
\draw [color=c, fill=c] (11.6716,1.77296) rectangle (11.7114,1.87881);
\draw [color=c, fill=c] (11.7114,1.77296) rectangle (11.7512,1.87881);
\draw [color=c, fill=c] (11.7512,1.77296) rectangle (11.791,1.87881);
\draw [color=c, fill=c] (11.791,1.77296) rectangle (11.8308,1.87881);
\draw [color=c, fill=c] (11.8308,1.77296) rectangle (11.8706,1.87881);
\draw [color=c, fill=c] (11.8706,1.77296) rectangle (11.9104,1.87881);
\draw [color=c, fill=c] (11.9104,1.77296) rectangle (11.9502,1.87881);
\draw [color=c, fill=c] (11.9502,1.77296) rectangle (11.99,1.87881);
\draw [color=c, fill=c] (11.99,1.77296) rectangle (12.0299,1.87881);
\draw [color=c, fill=c] (12.0299,1.77296) rectangle (12.0697,1.87881);
\draw [color=c, fill=c] (12.0697,1.77296) rectangle (12.1095,1.87881);
\draw [color=c, fill=c] (12.1095,1.77296) rectangle (12.1493,1.87881);
\draw [color=c, fill=c] (12.1493,1.77296) rectangle (12.1891,1.87881);
\draw [color=c, fill=c] (12.1891,1.77296) rectangle (12.2289,1.87881);
\draw [color=c, fill=c] (12.2289,1.77296) rectangle (12.2687,1.87881);
\draw [color=c, fill=c] (12.2687,1.77296) rectangle (12.3085,1.87881);
\draw [color=c, fill=c] (12.3085,1.77296) rectangle (12.3483,1.87881);
\definecolor{c}{rgb}{0,0.733333,1};
\draw [color=c, fill=c] (12.3483,1.77296) rectangle (12.3881,1.87881);
\draw [color=c, fill=c] (12.3881,1.77296) rectangle (12.4279,1.87881);
\draw [color=c, fill=c] (12.4279,1.77296) rectangle (12.4677,1.87881);
\draw [color=c, fill=c] (12.4677,1.77296) rectangle (12.5075,1.87881);
\draw [color=c, fill=c] (12.5075,1.77296) rectangle (12.5473,1.87881);
\draw [color=c, fill=c] (12.5473,1.77296) rectangle (12.5871,1.87881);
\draw [color=c, fill=c] (12.5871,1.77296) rectangle (12.6269,1.87881);
\draw [color=c, fill=c] (12.6269,1.77296) rectangle (12.6667,1.87881);
\draw [color=c, fill=c] (12.6667,1.77296) rectangle (12.7065,1.87881);
\draw [color=c, fill=c] (12.7065,1.77296) rectangle (12.7463,1.87881);
\draw [color=c, fill=c] (12.7463,1.77296) rectangle (12.7861,1.87881);
\draw [color=c, fill=c] (12.7861,1.77296) rectangle (12.8259,1.87881);
\draw [color=c, fill=c] (12.8259,1.77296) rectangle (12.8657,1.87881);
\draw [color=c, fill=c] (12.8657,1.77296) rectangle (12.9055,1.87881);
\draw [color=c, fill=c] (12.9055,1.77296) rectangle (12.9453,1.87881);
\draw [color=c, fill=c] (12.9453,1.77296) rectangle (12.9851,1.87881);
\draw [color=c, fill=c] (12.9851,1.77296) rectangle (13.0249,1.87881);
\draw [color=c, fill=c] (13.0249,1.77296) rectangle (13.0647,1.87881);
\draw [color=c, fill=c] (13.0647,1.77296) rectangle (13.1045,1.87881);
\draw [color=c, fill=c] (13.1045,1.77296) rectangle (13.1443,1.87881);
\draw [color=c, fill=c] (13.1443,1.77296) rectangle (13.1841,1.87881);
\draw [color=c, fill=c] (13.1841,1.77296) rectangle (13.2239,1.87881);
\draw [color=c, fill=c] (13.2239,1.77296) rectangle (13.2637,1.87881);
\draw [color=c, fill=c] (13.2637,1.77296) rectangle (13.3035,1.87881);
\draw [color=c, fill=c] (13.3035,1.77296) rectangle (13.3433,1.87881);
\draw [color=c, fill=c] (13.3433,1.77296) rectangle (13.3831,1.87881);
\draw [color=c, fill=c] (13.3831,1.77296) rectangle (13.4229,1.87881);
\draw [color=c, fill=c] (13.4229,1.77296) rectangle (13.4627,1.87881);
\draw [color=c, fill=c] (13.4627,1.77296) rectangle (13.5025,1.87881);
\draw [color=c, fill=c] (13.5025,1.77296) rectangle (13.5423,1.87881);
\draw [color=c, fill=c] (13.5423,1.77296) rectangle (13.5821,1.87881);
\draw [color=c, fill=c] (13.5821,1.77296) rectangle (13.6219,1.87881);
\draw [color=c, fill=c] (13.6219,1.77296) rectangle (13.6617,1.87881);
\draw [color=c, fill=c] (13.6617,1.77296) rectangle (13.7015,1.87881);
\draw [color=c, fill=c] (13.7015,1.77296) rectangle (13.7413,1.87881);
\draw [color=c, fill=c] (13.7413,1.77296) rectangle (13.7811,1.87881);
\draw [color=c, fill=c] (13.7811,1.77296) rectangle (13.8209,1.87881);
\draw [color=c, fill=c] (13.8209,1.77296) rectangle (13.8607,1.87881);
\draw [color=c, fill=c] (13.8607,1.77296) rectangle (13.9005,1.87881);
\draw [color=c, fill=c] (13.9005,1.77296) rectangle (13.9403,1.87881);
\draw [color=c, fill=c] (13.9403,1.77296) rectangle (13.9801,1.87881);
\draw [color=c, fill=c] (13.9801,1.77296) rectangle (14.0199,1.87881);
\draw [color=c, fill=c] (14.0199,1.77296) rectangle (14.0597,1.87881);
\draw [color=c, fill=c] (14.0597,1.77296) rectangle (14.0995,1.87881);
\draw [color=c, fill=c] (14.0995,1.77296) rectangle (14.1393,1.87881);
\draw [color=c, fill=c] (14.1393,1.77296) rectangle (14.1791,1.87881);
\draw [color=c, fill=c] (14.1791,1.77296) rectangle (14.2189,1.87881);
\draw [color=c, fill=c] (14.2189,1.77296) rectangle (14.2587,1.87881);
\draw [color=c, fill=c] (14.2587,1.77296) rectangle (14.2985,1.87881);
\draw [color=c, fill=c] (14.2985,1.77296) rectangle (14.3383,1.87881);
\draw [color=c, fill=c] (14.3383,1.77296) rectangle (14.3781,1.87881);
\draw [color=c, fill=c] (14.3781,1.77296) rectangle (14.4179,1.87881);
\draw [color=c, fill=c] (14.4179,1.77296) rectangle (14.4577,1.87881);
\draw [color=c, fill=c] (14.4577,1.77296) rectangle (14.4975,1.87881);
\draw [color=c, fill=c] (14.4975,1.77296) rectangle (14.5373,1.87881);
\draw [color=c, fill=c] (14.5373,1.77296) rectangle (14.5771,1.87881);
\draw [color=c, fill=c] (14.5771,1.77296) rectangle (14.6169,1.87881);
\draw [color=c, fill=c] (14.6169,1.77296) rectangle (14.6567,1.87881);
\draw [color=c, fill=c] (14.6567,1.77296) rectangle (14.6965,1.87881);
\draw [color=c, fill=c] (14.6965,1.77296) rectangle (14.7363,1.87881);
\draw [color=c, fill=c] (14.7363,1.77296) rectangle (14.7761,1.87881);
\draw [color=c, fill=c] (14.7761,1.77296) rectangle (14.8159,1.87881);
\draw [color=c, fill=c] (14.8159,1.77296) rectangle (14.8557,1.87881);
\draw [color=c, fill=c] (14.8557,1.77296) rectangle (14.8955,1.87881);
\draw [color=c, fill=c] (14.8955,1.77296) rectangle (14.9353,1.87881);
\draw [color=c, fill=c] (14.9353,1.77296) rectangle (14.9751,1.87881);
\draw [color=c, fill=c] (14.9751,1.77296) rectangle (15.0149,1.87881);
\draw [color=c, fill=c] (15.0149,1.77296) rectangle (15.0547,1.87881);
\draw [color=c, fill=c] (15.0547,1.77296) rectangle (15.0945,1.87881);
\draw [color=c, fill=c] (15.0945,1.77296) rectangle (15.1343,1.87881);
\draw [color=c, fill=c] (15.1343,1.77296) rectangle (15.1741,1.87881);
\draw [color=c, fill=c] (15.1741,1.77296) rectangle (15.2139,1.87881);
\draw [color=c, fill=c] (15.2139,1.77296) rectangle (15.2537,1.87881);
\draw [color=c, fill=c] (15.2537,1.77296) rectangle (15.2935,1.87881);
\draw [color=c, fill=c] (15.2935,1.77296) rectangle (15.3333,1.87881);
\draw [color=c, fill=c] (15.3333,1.77296) rectangle (15.3731,1.87881);
\draw [color=c, fill=c] (15.3731,1.77296) rectangle (15.4129,1.87881);
\draw [color=c, fill=c] (15.4129,1.77296) rectangle (15.4527,1.87881);
\draw [color=c, fill=c] (15.4527,1.77296) rectangle (15.4925,1.87881);
\draw [color=c, fill=c] (15.4925,1.77296) rectangle (15.5323,1.87881);
\draw [color=c, fill=c] (15.5323,1.77296) rectangle (15.5721,1.87881);
\draw [color=c, fill=c] (15.5721,1.77296) rectangle (15.6119,1.87881);
\draw [color=c, fill=c] (15.6119,1.77296) rectangle (15.6517,1.87881);
\draw [color=c, fill=c] (15.6517,1.77296) rectangle (15.6915,1.87881);
\draw [color=c, fill=c] (15.6915,1.77296) rectangle (15.7313,1.87881);
\draw [color=c, fill=c] (15.7313,1.77296) rectangle (15.7711,1.87881);
\draw [color=c, fill=c] (15.7711,1.77296) rectangle (15.8109,1.87881);
\draw [color=c, fill=c] (15.8109,1.77296) rectangle (15.8507,1.87881);
\draw [color=c, fill=c] (15.8507,1.77296) rectangle (15.8905,1.87881);
\draw [color=c, fill=c] (15.8905,1.77296) rectangle (15.9303,1.87881);
\draw [color=c, fill=c] (15.9303,1.77296) rectangle (15.9701,1.87881);
\draw [color=c, fill=c] (15.9701,1.77296) rectangle (16.01,1.87881);
\draw [color=c, fill=c] (16.01,1.77296) rectangle (16.0498,1.87881);
\draw [color=c, fill=c] (16.0498,1.77296) rectangle (16.0896,1.87881);
\draw [color=c, fill=c] (16.0896,1.77296) rectangle (16.1294,1.87881);
\draw [color=c, fill=c] (16.1294,1.77296) rectangle (16.1692,1.87881);
\draw [color=c, fill=c] (16.1692,1.77296) rectangle (16.209,1.87881);
\draw [color=c, fill=c] (16.209,1.77296) rectangle (16.2488,1.87881);
\draw [color=c, fill=c] (16.2488,1.77296) rectangle (16.2886,1.87881);
\draw [color=c, fill=c] (16.2886,1.77296) rectangle (16.3284,1.87881);
\draw [color=c, fill=c] (16.3284,1.77296) rectangle (16.3682,1.87881);
\draw [color=c, fill=c] (16.3682,1.77296) rectangle (16.408,1.87881);
\draw [color=c, fill=c] (16.408,1.77296) rectangle (16.4478,1.87881);
\draw [color=c, fill=c] (16.4478,1.77296) rectangle (16.4876,1.87881);
\draw [color=c, fill=c] (16.4876,1.77296) rectangle (16.5274,1.87881);
\draw [color=c, fill=c] (16.5274,1.77296) rectangle (16.5672,1.87881);
\draw [color=c, fill=c] (16.5672,1.77296) rectangle (16.607,1.87881);
\draw [color=c, fill=c] (16.607,1.77296) rectangle (16.6468,1.87881);
\draw [color=c, fill=c] (16.6468,1.77296) rectangle (16.6866,1.87881);
\draw [color=c, fill=c] (16.6866,1.77296) rectangle (16.7264,1.87881);
\draw [color=c, fill=c] (16.7264,1.77296) rectangle (16.7662,1.87881);
\draw [color=c, fill=c] (16.7662,1.77296) rectangle (16.806,1.87881);
\draw [color=c, fill=c] (16.806,1.77296) rectangle (16.8458,1.87881);
\draw [color=c, fill=c] (16.8458,1.77296) rectangle (16.8856,1.87881);
\draw [color=c, fill=c] (16.8856,1.77296) rectangle (16.9254,1.87881);
\draw [color=c, fill=c] (16.9254,1.77296) rectangle (16.9652,1.87881);
\draw [color=c, fill=c] (16.9652,1.77296) rectangle (17.005,1.87881);
\draw [color=c, fill=c] (17.005,1.77296) rectangle (17.0448,1.87881);
\draw [color=c, fill=c] (17.0448,1.77296) rectangle (17.0846,1.87881);
\draw [color=c, fill=c] (17.0846,1.77296) rectangle (17.1244,1.87881);
\draw [color=c, fill=c] (17.1244,1.77296) rectangle (17.1642,1.87881);
\draw [color=c, fill=c] (17.1642,1.77296) rectangle (17.204,1.87881);
\draw [color=c, fill=c] (17.204,1.77296) rectangle (17.2438,1.87881);
\draw [color=c, fill=c] (17.2438,1.77296) rectangle (17.2836,1.87881);
\draw [color=c, fill=c] (17.2836,1.77296) rectangle (17.3234,1.87881);
\draw [color=c, fill=c] (17.3234,1.77296) rectangle (17.3632,1.87881);
\draw [color=c, fill=c] (17.3632,1.77296) rectangle (17.403,1.87881);
\draw [color=c, fill=c] (17.403,1.77296) rectangle (17.4428,1.87881);
\draw [color=c, fill=c] (17.4428,1.77296) rectangle (17.4826,1.87881);
\draw [color=c, fill=c] (17.4826,1.77296) rectangle (17.5224,1.87881);
\draw [color=c, fill=c] (17.5224,1.77296) rectangle (17.5622,1.87881);
\draw [color=c, fill=c] (17.5622,1.77296) rectangle (17.602,1.87881);
\draw [color=c, fill=c] (17.602,1.77296) rectangle (17.6418,1.87881);
\draw [color=c, fill=c] (17.6418,1.77296) rectangle (17.6816,1.87881);
\draw [color=c, fill=c] (17.6816,1.77296) rectangle (17.7214,1.87881);
\draw [color=c, fill=c] (17.7214,1.77296) rectangle (17.7612,1.87881);
\draw [color=c, fill=c] (17.7612,1.77296) rectangle (17.801,1.87881);
\draw [color=c, fill=c] (17.801,1.77296) rectangle (17.8408,1.87881);
\draw [color=c, fill=c] (17.8408,1.77296) rectangle (17.8806,1.87881);
\draw [color=c, fill=c] (17.8806,1.77296) rectangle (17.9204,1.87881);
\draw [color=c, fill=c] (17.9204,1.77296) rectangle (17.9602,1.87881);
\draw [color=c, fill=c] (17.9602,1.77296) rectangle (18,1.87881);
\definecolor{c}{rgb}{1,0,0};
\draw [color=c, fill=c] (2,1.87881) rectangle (2.0398,1.98466);
\draw [color=c, fill=c] (2.0398,1.87881) rectangle (2.0796,1.98466);
\draw [color=c, fill=c] (2.0796,1.87881) rectangle (2.1194,1.98466);
\draw [color=c, fill=c] (2.1194,1.87881) rectangle (2.1592,1.98466);
\draw [color=c, fill=c] (2.1592,1.87881) rectangle (2.19901,1.98466);
\draw [color=c, fill=c] (2.19901,1.87881) rectangle (2.23881,1.98466);
\draw [color=c, fill=c] (2.23881,1.87881) rectangle (2.27861,1.98466);
\draw [color=c, fill=c] (2.27861,1.87881) rectangle (2.31841,1.98466);
\draw [color=c, fill=c] (2.31841,1.87881) rectangle (2.35821,1.98466);
\draw [color=c, fill=c] (2.35821,1.87881) rectangle (2.39801,1.98466);
\draw [color=c, fill=c] (2.39801,1.87881) rectangle (2.43781,1.98466);
\draw [color=c, fill=c] (2.43781,1.87881) rectangle (2.47761,1.98466);
\draw [color=c, fill=c] (2.47761,1.87881) rectangle (2.51741,1.98466);
\draw [color=c, fill=c] (2.51741,1.87881) rectangle (2.55721,1.98466);
\draw [color=c, fill=c] (2.55721,1.87881) rectangle (2.59702,1.98466);
\draw [color=c, fill=c] (2.59702,1.87881) rectangle (2.63682,1.98466);
\draw [color=c, fill=c] (2.63682,1.87881) rectangle (2.67662,1.98466);
\draw [color=c, fill=c] (2.67662,1.87881) rectangle (2.71642,1.98466);
\draw [color=c, fill=c] (2.71642,1.87881) rectangle (2.75622,1.98466);
\draw [color=c, fill=c] (2.75622,1.87881) rectangle (2.79602,1.98466);
\draw [color=c, fill=c] (2.79602,1.87881) rectangle (2.83582,1.98466);
\draw [color=c, fill=c] (2.83582,1.87881) rectangle (2.87562,1.98466);
\draw [color=c, fill=c] (2.87562,1.87881) rectangle (2.91542,1.98466);
\draw [color=c, fill=c] (2.91542,1.87881) rectangle (2.95522,1.98466);
\draw [color=c, fill=c] (2.95522,1.87881) rectangle (2.99502,1.98466);
\draw [color=c, fill=c] (2.99502,1.87881) rectangle (3.03483,1.98466);
\draw [color=c, fill=c] (3.03483,1.87881) rectangle (3.07463,1.98466);
\draw [color=c, fill=c] (3.07463,1.87881) rectangle (3.11443,1.98466);
\draw [color=c, fill=c] (3.11443,1.87881) rectangle (3.15423,1.98466);
\draw [color=c, fill=c] (3.15423,1.87881) rectangle (3.19403,1.98466);
\draw [color=c, fill=c] (3.19403,1.87881) rectangle (3.23383,1.98466);
\draw [color=c, fill=c] (3.23383,1.87881) rectangle (3.27363,1.98466);
\draw [color=c, fill=c] (3.27363,1.87881) rectangle (3.31343,1.98466);
\draw [color=c, fill=c] (3.31343,1.87881) rectangle (3.35323,1.98466);
\draw [color=c, fill=c] (3.35323,1.87881) rectangle (3.39303,1.98466);
\draw [color=c, fill=c] (3.39303,1.87881) rectangle (3.43284,1.98466);
\draw [color=c, fill=c] (3.43284,1.87881) rectangle (3.47264,1.98466);
\draw [color=c, fill=c] (3.47264,1.87881) rectangle (3.51244,1.98466);
\draw [color=c, fill=c] (3.51244,1.87881) rectangle (3.55224,1.98466);
\draw [color=c, fill=c] (3.55224,1.87881) rectangle (3.59204,1.98466);
\draw [color=c, fill=c] (3.59204,1.87881) rectangle (3.63184,1.98466);
\draw [color=c, fill=c] (3.63184,1.87881) rectangle (3.67164,1.98466);
\draw [color=c, fill=c] (3.67164,1.87881) rectangle (3.71144,1.98466);
\draw [color=c, fill=c] (3.71144,1.87881) rectangle (3.75124,1.98466);
\draw [color=c, fill=c] (3.75124,1.87881) rectangle (3.79104,1.98466);
\draw [color=c, fill=c] (3.79104,1.87881) rectangle (3.83085,1.98466);
\draw [color=c, fill=c] (3.83085,1.87881) rectangle (3.87065,1.98466);
\draw [color=c, fill=c] (3.87065,1.87881) rectangle (3.91045,1.98466);
\draw [color=c, fill=c] (3.91045,1.87881) rectangle (3.95025,1.98466);
\draw [color=c, fill=c] (3.95025,1.87881) rectangle (3.99005,1.98466);
\draw [color=c, fill=c] (3.99005,1.87881) rectangle (4.02985,1.98466);
\draw [color=c, fill=c] (4.02985,1.87881) rectangle (4.06965,1.98466);
\draw [color=c, fill=c] (4.06965,1.87881) rectangle (4.10945,1.98466);
\draw [color=c, fill=c] (4.10945,1.87881) rectangle (4.14925,1.98466);
\draw [color=c, fill=c] (4.14925,1.87881) rectangle (4.18905,1.98466);
\draw [color=c, fill=c] (4.18905,1.87881) rectangle (4.22886,1.98466);
\draw [color=c, fill=c] (4.22886,1.87881) rectangle (4.26866,1.98466);
\draw [color=c, fill=c] (4.26866,1.87881) rectangle (4.30846,1.98466);
\draw [color=c, fill=c] (4.30846,1.87881) rectangle (4.34826,1.98466);
\draw [color=c, fill=c] (4.34826,1.87881) rectangle (4.38806,1.98466);
\draw [color=c, fill=c] (4.38806,1.87881) rectangle (4.42786,1.98466);
\draw [color=c, fill=c] (4.42786,1.87881) rectangle (4.46766,1.98466);
\draw [color=c, fill=c] (4.46766,1.87881) rectangle (4.50746,1.98466);
\draw [color=c, fill=c] (4.50746,1.87881) rectangle (4.54726,1.98466);
\draw [color=c, fill=c] (4.54726,1.87881) rectangle (4.58706,1.98466);
\draw [color=c, fill=c] (4.58706,1.87881) rectangle (4.62687,1.98466);
\draw [color=c, fill=c] (4.62687,1.87881) rectangle (4.66667,1.98466);
\draw [color=c, fill=c] (4.66667,1.87881) rectangle (4.70647,1.98466);
\draw [color=c, fill=c] (4.70647,1.87881) rectangle (4.74627,1.98466);
\draw [color=c, fill=c] (4.74627,1.87881) rectangle (4.78607,1.98466);
\draw [color=c, fill=c] (4.78607,1.87881) rectangle (4.82587,1.98466);
\draw [color=c, fill=c] (4.82587,1.87881) rectangle (4.86567,1.98466);
\draw [color=c, fill=c] (4.86567,1.87881) rectangle (4.90547,1.98466);
\draw [color=c, fill=c] (4.90547,1.87881) rectangle (4.94527,1.98466);
\draw [color=c, fill=c] (4.94527,1.87881) rectangle (4.98507,1.98466);
\draw [color=c, fill=c] (4.98507,1.87881) rectangle (5.02488,1.98466);
\draw [color=c, fill=c] (5.02488,1.87881) rectangle (5.06468,1.98466);
\draw [color=c, fill=c] (5.06468,1.87881) rectangle (5.10448,1.98466);
\draw [color=c, fill=c] (5.10448,1.87881) rectangle (5.14428,1.98466);
\draw [color=c, fill=c] (5.14428,1.87881) rectangle (5.18408,1.98466);
\draw [color=c, fill=c] (5.18408,1.87881) rectangle (5.22388,1.98466);
\draw [color=c, fill=c] (5.22388,1.87881) rectangle (5.26368,1.98466);
\draw [color=c, fill=c] (5.26368,1.87881) rectangle (5.30348,1.98466);
\draw [color=c, fill=c] (5.30348,1.87881) rectangle (5.34328,1.98466);
\draw [color=c, fill=c] (5.34328,1.87881) rectangle (5.38308,1.98466);
\draw [color=c, fill=c] (5.38308,1.87881) rectangle (5.42289,1.98466);
\draw [color=c, fill=c] (5.42289,1.87881) rectangle (5.46269,1.98466);
\draw [color=c, fill=c] (5.46269,1.87881) rectangle (5.50249,1.98466);
\draw [color=c, fill=c] (5.50249,1.87881) rectangle (5.54229,1.98466);
\draw [color=c, fill=c] (5.54229,1.87881) rectangle (5.58209,1.98466);
\draw [color=c, fill=c] (5.58209,1.87881) rectangle (5.62189,1.98466);
\draw [color=c, fill=c] (5.62189,1.87881) rectangle (5.66169,1.98466);
\draw [color=c, fill=c] (5.66169,1.87881) rectangle (5.70149,1.98466);
\draw [color=c, fill=c] (5.70149,1.87881) rectangle (5.74129,1.98466);
\draw [color=c, fill=c] (5.74129,1.87881) rectangle (5.78109,1.98466);
\draw [color=c, fill=c] (5.78109,1.87881) rectangle (5.8209,1.98466);
\draw [color=c, fill=c] (5.8209,1.87881) rectangle (5.8607,1.98466);
\draw [color=c, fill=c] (5.8607,1.87881) rectangle (5.9005,1.98466);
\draw [color=c, fill=c] (5.9005,1.87881) rectangle (5.9403,1.98466);
\draw [color=c, fill=c] (5.9403,1.87881) rectangle (5.9801,1.98466);
\draw [color=c, fill=c] (5.9801,1.87881) rectangle (6.0199,1.98466);
\draw [color=c, fill=c] (6.0199,1.87881) rectangle (6.0597,1.98466);
\draw [color=c, fill=c] (6.0597,1.87881) rectangle (6.0995,1.98466);
\draw [color=c, fill=c] (6.0995,1.87881) rectangle (6.1393,1.98466);
\draw [color=c, fill=c] (6.1393,1.87881) rectangle (6.1791,1.98466);
\draw [color=c, fill=c] (6.1791,1.87881) rectangle (6.21891,1.98466);
\draw [color=c, fill=c] (6.21891,1.87881) rectangle (6.25871,1.98466);
\draw [color=c, fill=c] (6.25871,1.87881) rectangle (6.29851,1.98466);
\draw [color=c, fill=c] (6.29851,1.87881) rectangle (6.33831,1.98466);
\draw [color=c, fill=c] (6.33831,1.87881) rectangle (6.37811,1.98466);
\draw [color=c, fill=c] (6.37811,1.87881) rectangle (6.41791,1.98466);
\draw [color=c, fill=c] (6.41791,1.87881) rectangle (6.45771,1.98466);
\draw [color=c, fill=c] (6.45771,1.87881) rectangle (6.49751,1.98466);
\draw [color=c, fill=c] (6.49751,1.87881) rectangle (6.53731,1.98466);
\draw [color=c, fill=c] (6.53731,1.87881) rectangle (6.57711,1.98466);
\draw [color=c, fill=c] (6.57711,1.87881) rectangle (6.61692,1.98466);
\draw [color=c, fill=c] (6.61692,1.87881) rectangle (6.65672,1.98466);
\draw [color=c, fill=c] (6.65672,1.87881) rectangle (6.69652,1.98466);
\draw [color=c, fill=c] (6.69652,1.87881) rectangle (6.73632,1.98466);
\draw [color=c, fill=c] (6.73632,1.87881) rectangle (6.77612,1.98466);
\draw [color=c, fill=c] (6.77612,1.87881) rectangle (6.81592,1.98466);
\draw [color=c, fill=c] (6.81592,1.87881) rectangle (6.85572,1.98466);
\draw [color=c, fill=c] (6.85572,1.87881) rectangle (6.89552,1.98466);
\draw [color=c, fill=c] (6.89552,1.87881) rectangle (6.93532,1.98466);
\draw [color=c, fill=c] (6.93532,1.87881) rectangle (6.97512,1.98466);
\draw [color=c, fill=c] (6.97512,1.87881) rectangle (7.01493,1.98466);
\draw [color=c, fill=c] (7.01493,1.87881) rectangle (7.05473,1.98466);
\draw [color=c, fill=c] (7.05473,1.87881) rectangle (7.09453,1.98466);
\draw [color=c, fill=c] (7.09453,1.87881) rectangle (7.13433,1.98466);
\draw [color=c, fill=c] (7.13433,1.87881) rectangle (7.17413,1.98466);
\draw [color=c, fill=c] (7.17413,1.87881) rectangle (7.21393,1.98466);
\draw [color=c, fill=c] (7.21393,1.87881) rectangle (7.25373,1.98466);
\draw [color=c, fill=c] (7.25373,1.87881) rectangle (7.29353,1.98466);
\draw [color=c, fill=c] (7.29353,1.87881) rectangle (7.33333,1.98466);
\draw [color=c, fill=c] (7.33333,1.87881) rectangle (7.37313,1.98466);
\draw [color=c, fill=c] (7.37313,1.87881) rectangle (7.41294,1.98466);
\draw [color=c, fill=c] (7.41294,1.87881) rectangle (7.45274,1.98466);
\draw [color=c, fill=c] (7.45274,1.87881) rectangle (7.49254,1.98466);
\draw [color=c, fill=c] (7.49254,1.87881) rectangle (7.53234,1.98466);
\draw [color=c, fill=c] (7.53234,1.87881) rectangle (7.57214,1.98466);
\draw [color=c, fill=c] (7.57214,1.87881) rectangle (7.61194,1.98466);
\draw [color=c, fill=c] (7.61194,1.87881) rectangle (7.65174,1.98466);
\draw [color=c, fill=c] (7.65174,1.87881) rectangle (7.69154,1.98466);
\draw [color=c, fill=c] (7.69154,1.87881) rectangle (7.73134,1.98466);
\draw [color=c, fill=c] (7.73134,1.87881) rectangle (7.77114,1.98466);
\draw [color=c, fill=c] (7.77114,1.87881) rectangle (7.81095,1.98466);
\definecolor{c}{rgb}{1,0.186667,0};
\draw [color=c, fill=c] (7.81095,1.87881) rectangle (7.85075,1.98466);
\draw [color=c, fill=c] (7.85075,1.87881) rectangle (7.89055,1.98466);
\draw [color=c, fill=c] (7.89055,1.87881) rectangle (7.93035,1.98466);
\draw [color=c, fill=c] (7.93035,1.87881) rectangle (7.97015,1.98466);
\draw [color=c, fill=c] (7.97015,1.87881) rectangle (8.00995,1.98466);
\draw [color=c, fill=c] (8.00995,1.87881) rectangle (8.04975,1.98466);
\draw [color=c, fill=c] (8.04975,1.87881) rectangle (8.08955,1.98466);
\draw [color=c, fill=c] (8.08955,1.87881) rectangle (8.12935,1.98466);
\draw [color=c, fill=c] (8.12935,1.87881) rectangle (8.16915,1.98466);
\draw [color=c, fill=c] (8.16915,1.87881) rectangle (8.20895,1.98466);
\draw [color=c, fill=c] (8.20895,1.87881) rectangle (8.24876,1.98466);
\draw [color=c, fill=c] (8.24876,1.87881) rectangle (8.28856,1.98466);
\draw [color=c, fill=c] (8.28856,1.87881) rectangle (8.32836,1.98466);
\draw [color=c, fill=c] (8.32836,1.87881) rectangle (8.36816,1.98466);
\draw [color=c, fill=c] (8.36816,1.87881) rectangle (8.40796,1.98466);
\draw [color=c, fill=c] (8.40796,1.87881) rectangle (8.44776,1.98466);
\draw [color=c, fill=c] (8.44776,1.87881) rectangle (8.48756,1.98466);
\draw [color=c, fill=c] (8.48756,1.87881) rectangle (8.52736,1.98466);
\draw [color=c, fill=c] (8.52736,1.87881) rectangle (8.56716,1.98466);
\draw [color=c, fill=c] (8.56716,1.87881) rectangle (8.60697,1.98466);
\definecolor{c}{rgb}{1,0.466667,0};
\draw [color=c, fill=c] (8.60697,1.87881) rectangle (8.64677,1.98466);
\draw [color=c, fill=c] (8.64677,1.87881) rectangle (8.68657,1.98466);
\draw [color=c, fill=c] (8.68657,1.87881) rectangle (8.72637,1.98466);
\draw [color=c, fill=c] (8.72637,1.87881) rectangle (8.76617,1.98466);
\draw [color=c, fill=c] (8.76617,1.87881) rectangle (8.80597,1.98466);
\draw [color=c, fill=c] (8.80597,1.87881) rectangle (8.84577,1.98466);
\draw [color=c, fill=c] (8.84577,1.87881) rectangle (8.88557,1.98466);
\draw [color=c, fill=c] (8.88557,1.87881) rectangle (8.92537,1.98466);
\draw [color=c, fill=c] (8.92537,1.87881) rectangle (8.96517,1.98466);
\draw [color=c, fill=c] (8.96517,1.87881) rectangle (9.00498,1.98466);
\draw [color=c, fill=c] (9.00498,1.87881) rectangle (9.04478,1.98466);
\definecolor{c}{rgb}{1,0.653333,0};
\draw [color=c, fill=c] (9.04478,1.87881) rectangle (9.08458,1.98466);
\draw [color=c, fill=c] (9.08458,1.87881) rectangle (9.12438,1.98466);
\draw [color=c, fill=c] (9.12438,1.87881) rectangle (9.16418,1.98466);
\draw [color=c, fill=c] (9.16418,1.87881) rectangle (9.20398,1.98466);
\draw [color=c, fill=c] (9.20398,1.87881) rectangle (9.24378,1.98466);
\draw [color=c, fill=c] (9.24378,1.87881) rectangle (9.28358,1.98466);
\draw [color=c, fill=c] (9.28358,1.87881) rectangle (9.32338,1.98466);
\definecolor{c}{rgb}{1,0.933333,0};
\draw [color=c, fill=c] (9.32338,1.87881) rectangle (9.36318,1.98466);
\draw [color=c, fill=c] (9.36318,1.87881) rectangle (9.40298,1.98466);
\draw [color=c, fill=c] (9.40298,1.87881) rectangle (9.44279,1.98466);
\draw [color=c, fill=c] (9.44279,1.87881) rectangle (9.48259,1.98466);
\draw [color=c, fill=c] (9.48259,1.87881) rectangle (9.52239,1.98466);
\draw [color=c, fill=c] (9.52239,1.87881) rectangle (9.56219,1.98466);
\definecolor{c}{rgb}{0.88,1,0};
\draw [color=c, fill=c] (9.56219,1.87881) rectangle (9.60199,1.98466);
\draw [color=c, fill=c] (9.60199,1.87881) rectangle (9.64179,1.98466);
\draw [color=c, fill=c] (9.64179,1.87881) rectangle (9.68159,1.98466);
\draw [color=c, fill=c] (9.68159,1.87881) rectangle (9.72139,1.98466);
\definecolor{c}{rgb}{0.6,1,0};
\draw [color=c, fill=c] (9.72139,1.87881) rectangle (9.76119,1.98466);
\draw [color=c, fill=c] (9.76119,1.87881) rectangle (9.80099,1.98466);
\draw [color=c, fill=c] (9.80099,1.87881) rectangle (9.8408,1.98466);
\draw [color=c, fill=c] (9.8408,1.87881) rectangle (9.8806,1.98466);
\definecolor{c}{rgb}{0.413333,1,0};
\draw [color=c, fill=c] (9.8806,1.87881) rectangle (9.9204,1.98466);
\draw [color=c, fill=c] (9.9204,1.87881) rectangle (9.9602,1.98466);
\draw [color=c, fill=c] (9.9602,1.87881) rectangle (10,1.98466);
\draw [color=c, fill=c] (10,1.87881) rectangle (10.0398,1.98466);
\definecolor{c}{rgb}{0.133333,1,0};
\draw [color=c, fill=c] (10.0398,1.87881) rectangle (10.0796,1.98466);
\draw [color=c, fill=c] (10.0796,1.87881) rectangle (10.1194,1.98466);
\draw [color=c, fill=c] (10.1194,1.87881) rectangle (10.1592,1.98466);
\draw [color=c, fill=c] (10.1592,1.87881) rectangle (10.199,1.98466);
\definecolor{c}{rgb}{0,1,0.0533333};
\draw [color=c, fill=c] (10.199,1.87881) rectangle (10.2388,1.98466);
\draw [color=c, fill=c] (10.2388,1.87881) rectangle (10.2786,1.98466);
\draw [color=c, fill=c] (10.2786,1.87881) rectangle (10.3184,1.98466);
\draw [color=c, fill=c] (10.3184,1.87881) rectangle (10.3582,1.98466);
\draw [color=c, fill=c] (10.3582,1.87881) rectangle (10.398,1.98466);
\definecolor{c}{rgb}{0,1,0.333333};
\draw [color=c, fill=c] (10.398,1.87881) rectangle (10.4378,1.98466);
\draw [color=c, fill=c] (10.4378,1.87881) rectangle (10.4776,1.98466);
\draw [color=c, fill=c] (10.4776,1.87881) rectangle (10.5174,1.98466);
\draw [color=c, fill=c] (10.5174,1.87881) rectangle (10.5572,1.98466);
\draw [color=c, fill=c] (10.5572,1.87881) rectangle (10.597,1.98466);
\draw [color=c, fill=c] (10.597,1.87881) rectangle (10.6368,1.98466);
\definecolor{c}{rgb}{0,1,0.52};
\draw [color=c, fill=c] (10.6368,1.87881) rectangle (10.6766,1.98466);
\draw [color=c, fill=c] (10.6766,1.87881) rectangle (10.7164,1.98466);
\draw [color=c, fill=c] (10.7164,1.87881) rectangle (10.7562,1.98466);
\draw [color=c, fill=c] (10.7562,1.87881) rectangle (10.796,1.98466);
\draw [color=c, fill=c] (10.796,1.87881) rectangle (10.8358,1.98466);
\draw [color=c, fill=c] (10.8358,1.87881) rectangle (10.8756,1.98466);
\draw [color=c, fill=c] (10.8756,1.87881) rectangle (10.9154,1.98466);
\definecolor{c}{rgb}{0,1,0.8};
\draw [color=c, fill=c] (10.9154,1.87881) rectangle (10.9552,1.98466);
\draw [color=c, fill=c] (10.9552,1.87881) rectangle (10.995,1.98466);
\draw [color=c, fill=c] (10.995,1.87881) rectangle (11.0348,1.98466);
\draw [color=c, fill=c] (11.0348,1.87881) rectangle (11.0746,1.98466);
\draw [color=c, fill=c] (11.0746,1.87881) rectangle (11.1144,1.98466);
\draw [color=c, fill=c] (11.1144,1.87881) rectangle (11.1542,1.98466);
\draw [color=c, fill=c] (11.1542,1.87881) rectangle (11.194,1.98466);
\draw [color=c, fill=c] (11.194,1.87881) rectangle (11.2338,1.98466);
\draw [color=c, fill=c] (11.2338,1.87881) rectangle (11.2736,1.98466);
\draw [color=c, fill=c] (11.2736,1.87881) rectangle (11.3134,1.98466);
\draw [color=c, fill=c] (11.3134,1.87881) rectangle (11.3532,1.98466);
\draw [color=c, fill=c] (11.3532,1.87881) rectangle (11.393,1.98466);
\definecolor{c}{rgb}{0,1,0.986667};
\draw [color=c, fill=c] (11.393,1.87881) rectangle (11.4328,1.98466);
\draw [color=c, fill=c] (11.4328,1.87881) rectangle (11.4726,1.98466);
\draw [color=c, fill=c] (11.4726,1.87881) rectangle (11.5124,1.98466);
\draw [color=c, fill=c] (11.5124,1.87881) rectangle (11.5522,1.98466);
\draw [color=c, fill=c] (11.5522,1.87881) rectangle (11.592,1.98466);
\draw [color=c, fill=c] (11.592,1.87881) rectangle (11.6318,1.98466);
\draw [color=c, fill=c] (11.6318,1.87881) rectangle (11.6716,1.98466);
\draw [color=c, fill=c] (11.6716,1.87881) rectangle (11.7114,1.98466);
\draw [color=c, fill=c] (11.7114,1.87881) rectangle (11.7512,1.98466);
\draw [color=c, fill=c] (11.7512,1.87881) rectangle (11.791,1.98466);
\draw [color=c, fill=c] (11.791,1.87881) rectangle (11.8308,1.98466);
\draw [color=c, fill=c] (11.8308,1.87881) rectangle (11.8706,1.98466);
\draw [color=c, fill=c] (11.8706,1.87881) rectangle (11.9104,1.98466);
\draw [color=c, fill=c] (11.9104,1.87881) rectangle (11.9502,1.98466);
\draw [color=c, fill=c] (11.9502,1.87881) rectangle (11.99,1.98466);
\draw [color=c, fill=c] (11.99,1.87881) rectangle (12.0299,1.98466);
\draw [color=c, fill=c] (12.0299,1.87881) rectangle (12.0697,1.98466);
\draw [color=c, fill=c] (12.0697,1.87881) rectangle (12.1095,1.98466);
\draw [color=c, fill=c] (12.1095,1.87881) rectangle (12.1493,1.98466);
\draw [color=c, fill=c] (12.1493,1.87881) rectangle (12.1891,1.98466);
\draw [color=c, fill=c] (12.1891,1.87881) rectangle (12.2289,1.98466);
\draw [color=c, fill=c] (12.2289,1.87881) rectangle (12.2687,1.98466);
\draw [color=c, fill=c] (12.2687,1.87881) rectangle (12.3085,1.98466);
\definecolor{c}{rgb}{0,0.733333,1};
\draw [color=c, fill=c] (12.3085,1.87881) rectangle (12.3483,1.98466);
\draw [color=c, fill=c] (12.3483,1.87881) rectangle (12.3881,1.98466);
\draw [color=c, fill=c] (12.3881,1.87881) rectangle (12.4279,1.98466);
\draw [color=c, fill=c] (12.4279,1.87881) rectangle (12.4677,1.98466);
\draw [color=c, fill=c] (12.4677,1.87881) rectangle (12.5075,1.98466);
\draw [color=c, fill=c] (12.5075,1.87881) rectangle (12.5473,1.98466);
\draw [color=c, fill=c] (12.5473,1.87881) rectangle (12.5871,1.98466);
\draw [color=c, fill=c] (12.5871,1.87881) rectangle (12.6269,1.98466);
\draw [color=c, fill=c] (12.6269,1.87881) rectangle (12.6667,1.98466);
\draw [color=c, fill=c] (12.6667,1.87881) rectangle (12.7065,1.98466);
\draw [color=c, fill=c] (12.7065,1.87881) rectangle (12.7463,1.98466);
\draw [color=c, fill=c] (12.7463,1.87881) rectangle (12.7861,1.98466);
\draw [color=c, fill=c] (12.7861,1.87881) rectangle (12.8259,1.98466);
\draw [color=c, fill=c] (12.8259,1.87881) rectangle (12.8657,1.98466);
\draw [color=c, fill=c] (12.8657,1.87881) rectangle (12.9055,1.98466);
\draw [color=c, fill=c] (12.9055,1.87881) rectangle (12.9453,1.98466);
\draw [color=c, fill=c] (12.9453,1.87881) rectangle (12.9851,1.98466);
\draw [color=c, fill=c] (12.9851,1.87881) rectangle (13.0249,1.98466);
\draw [color=c, fill=c] (13.0249,1.87881) rectangle (13.0647,1.98466);
\draw [color=c, fill=c] (13.0647,1.87881) rectangle (13.1045,1.98466);
\draw [color=c, fill=c] (13.1045,1.87881) rectangle (13.1443,1.98466);
\draw [color=c, fill=c] (13.1443,1.87881) rectangle (13.1841,1.98466);
\draw [color=c, fill=c] (13.1841,1.87881) rectangle (13.2239,1.98466);
\draw [color=c, fill=c] (13.2239,1.87881) rectangle (13.2637,1.98466);
\draw [color=c, fill=c] (13.2637,1.87881) rectangle (13.3035,1.98466);
\draw [color=c, fill=c] (13.3035,1.87881) rectangle (13.3433,1.98466);
\draw [color=c, fill=c] (13.3433,1.87881) rectangle (13.3831,1.98466);
\draw [color=c, fill=c] (13.3831,1.87881) rectangle (13.4229,1.98466);
\draw [color=c, fill=c] (13.4229,1.87881) rectangle (13.4627,1.98466);
\draw [color=c, fill=c] (13.4627,1.87881) rectangle (13.5025,1.98466);
\draw [color=c, fill=c] (13.5025,1.87881) rectangle (13.5423,1.98466);
\draw [color=c, fill=c] (13.5423,1.87881) rectangle (13.5821,1.98466);
\draw [color=c, fill=c] (13.5821,1.87881) rectangle (13.6219,1.98466);
\draw [color=c, fill=c] (13.6219,1.87881) rectangle (13.6617,1.98466);
\draw [color=c, fill=c] (13.6617,1.87881) rectangle (13.7015,1.98466);
\draw [color=c, fill=c] (13.7015,1.87881) rectangle (13.7413,1.98466);
\draw [color=c, fill=c] (13.7413,1.87881) rectangle (13.7811,1.98466);
\draw [color=c, fill=c] (13.7811,1.87881) rectangle (13.8209,1.98466);
\draw [color=c, fill=c] (13.8209,1.87881) rectangle (13.8607,1.98466);
\draw [color=c, fill=c] (13.8607,1.87881) rectangle (13.9005,1.98466);
\draw [color=c, fill=c] (13.9005,1.87881) rectangle (13.9403,1.98466);
\draw [color=c, fill=c] (13.9403,1.87881) rectangle (13.9801,1.98466);
\draw [color=c, fill=c] (13.9801,1.87881) rectangle (14.0199,1.98466);
\draw [color=c, fill=c] (14.0199,1.87881) rectangle (14.0597,1.98466);
\draw [color=c, fill=c] (14.0597,1.87881) rectangle (14.0995,1.98466);
\draw [color=c, fill=c] (14.0995,1.87881) rectangle (14.1393,1.98466);
\draw [color=c, fill=c] (14.1393,1.87881) rectangle (14.1791,1.98466);
\draw [color=c, fill=c] (14.1791,1.87881) rectangle (14.2189,1.98466);
\draw [color=c, fill=c] (14.2189,1.87881) rectangle (14.2587,1.98466);
\draw [color=c, fill=c] (14.2587,1.87881) rectangle (14.2985,1.98466);
\draw [color=c, fill=c] (14.2985,1.87881) rectangle (14.3383,1.98466);
\draw [color=c, fill=c] (14.3383,1.87881) rectangle (14.3781,1.98466);
\draw [color=c, fill=c] (14.3781,1.87881) rectangle (14.4179,1.98466);
\draw [color=c, fill=c] (14.4179,1.87881) rectangle (14.4577,1.98466);
\draw [color=c, fill=c] (14.4577,1.87881) rectangle (14.4975,1.98466);
\draw [color=c, fill=c] (14.4975,1.87881) rectangle (14.5373,1.98466);
\draw [color=c, fill=c] (14.5373,1.87881) rectangle (14.5771,1.98466);
\draw [color=c, fill=c] (14.5771,1.87881) rectangle (14.6169,1.98466);
\draw [color=c, fill=c] (14.6169,1.87881) rectangle (14.6567,1.98466);
\draw [color=c, fill=c] (14.6567,1.87881) rectangle (14.6965,1.98466);
\draw [color=c, fill=c] (14.6965,1.87881) rectangle (14.7363,1.98466);
\draw [color=c, fill=c] (14.7363,1.87881) rectangle (14.7761,1.98466);
\draw [color=c, fill=c] (14.7761,1.87881) rectangle (14.8159,1.98466);
\draw [color=c, fill=c] (14.8159,1.87881) rectangle (14.8557,1.98466);
\draw [color=c, fill=c] (14.8557,1.87881) rectangle (14.8955,1.98466);
\draw [color=c, fill=c] (14.8955,1.87881) rectangle (14.9353,1.98466);
\draw [color=c, fill=c] (14.9353,1.87881) rectangle (14.9751,1.98466);
\draw [color=c, fill=c] (14.9751,1.87881) rectangle (15.0149,1.98466);
\draw [color=c, fill=c] (15.0149,1.87881) rectangle (15.0547,1.98466);
\draw [color=c, fill=c] (15.0547,1.87881) rectangle (15.0945,1.98466);
\draw [color=c, fill=c] (15.0945,1.87881) rectangle (15.1343,1.98466);
\draw [color=c, fill=c] (15.1343,1.87881) rectangle (15.1741,1.98466);
\draw [color=c, fill=c] (15.1741,1.87881) rectangle (15.2139,1.98466);
\draw [color=c, fill=c] (15.2139,1.87881) rectangle (15.2537,1.98466);
\draw [color=c, fill=c] (15.2537,1.87881) rectangle (15.2935,1.98466);
\draw [color=c, fill=c] (15.2935,1.87881) rectangle (15.3333,1.98466);
\draw [color=c, fill=c] (15.3333,1.87881) rectangle (15.3731,1.98466);
\draw [color=c, fill=c] (15.3731,1.87881) rectangle (15.4129,1.98466);
\draw [color=c, fill=c] (15.4129,1.87881) rectangle (15.4527,1.98466);
\draw [color=c, fill=c] (15.4527,1.87881) rectangle (15.4925,1.98466);
\draw [color=c, fill=c] (15.4925,1.87881) rectangle (15.5323,1.98466);
\draw [color=c, fill=c] (15.5323,1.87881) rectangle (15.5721,1.98466);
\draw [color=c, fill=c] (15.5721,1.87881) rectangle (15.6119,1.98466);
\draw [color=c, fill=c] (15.6119,1.87881) rectangle (15.6517,1.98466);
\draw [color=c, fill=c] (15.6517,1.87881) rectangle (15.6915,1.98466);
\draw [color=c, fill=c] (15.6915,1.87881) rectangle (15.7313,1.98466);
\draw [color=c, fill=c] (15.7313,1.87881) rectangle (15.7711,1.98466);
\draw [color=c, fill=c] (15.7711,1.87881) rectangle (15.8109,1.98466);
\draw [color=c, fill=c] (15.8109,1.87881) rectangle (15.8507,1.98466);
\draw [color=c, fill=c] (15.8507,1.87881) rectangle (15.8905,1.98466);
\draw [color=c, fill=c] (15.8905,1.87881) rectangle (15.9303,1.98466);
\draw [color=c, fill=c] (15.9303,1.87881) rectangle (15.9701,1.98466);
\draw [color=c, fill=c] (15.9701,1.87881) rectangle (16.01,1.98466);
\draw [color=c, fill=c] (16.01,1.87881) rectangle (16.0498,1.98466);
\draw [color=c, fill=c] (16.0498,1.87881) rectangle (16.0896,1.98466);
\draw [color=c, fill=c] (16.0896,1.87881) rectangle (16.1294,1.98466);
\draw [color=c, fill=c] (16.1294,1.87881) rectangle (16.1692,1.98466);
\draw [color=c, fill=c] (16.1692,1.87881) rectangle (16.209,1.98466);
\draw [color=c, fill=c] (16.209,1.87881) rectangle (16.2488,1.98466);
\draw [color=c, fill=c] (16.2488,1.87881) rectangle (16.2886,1.98466);
\draw [color=c, fill=c] (16.2886,1.87881) rectangle (16.3284,1.98466);
\draw [color=c, fill=c] (16.3284,1.87881) rectangle (16.3682,1.98466);
\draw [color=c, fill=c] (16.3682,1.87881) rectangle (16.408,1.98466);
\draw [color=c, fill=c] (16.408,1.87881) rectangle (16.4478,1.98466);
\draw [color=c, fill=c] (16.4478,1.87881) rectangle (16.4876,1.98466);
\draw [color=c, fill=c] (16.4876,1.87881) rectangle (16.5274,1.98466);
\draw [color=c, fill=c] (16.5274,1.87881) rectangle (16.5672,1.98466);
\draw [color=c, fill=c] (16.5672,1.87881) rectangle (16.607,1.98466);
\draw [color=c, fill=c] (16.607,1.87881) rectangle (16.6468,1.98466);
\draw [color=c, fill=c] (16.6468,1.87881) rectangle (16.6866,1.98466);
\draw [color=c, fill=c] (16.6866,1.87881) rectangle (16.7264,1.98466);
\draw [color=c, fill=c] (16.7264,1.87881) rectangle (16.7662,1.98466);
\draw [color=c, fill=c] (16.7662,1.87881) rectangle (16.806,1.98466);
\draw [color=c, fill=c] (16.806,1.87881) rectangle (16.8458,1.98466);
\draw [color=c, fill=c] (16.8458,1.87881) rectangle (16.8856,1.98466);
\draw [color=c, fill=c] (16.8856,1.87881) rectangle (16.9254,1.98466);
\draw [color=c, fill=c] (16.9254,1.87881) rectangle (16.9652,1.98466);
\draw [color=c, fill=c] (16.9652,1.87881) rectangle (17.005,1.98466);
\draw [color=c, fill=c] (17.005,1.87881) rectangle (17.0448,1.98466);
\draw [color=c, fill=c] (17.0448,1.87881) rectangle (17.0846,1.98466);
\draw [color=c, fill=c] (17.0846,1.87881) rectangle (17.1244,1.98466);
\draw [color=c, fill=c] (17.1244,1.87881) rectangle (17.1642,1.98466);
\draw [color=c, fill=c] (17.1642,1.87881) rectangle (17.204,1.98466);
\draw [color=c, fill=c] (17.204,1.87881) rectangle (17.2438,1.98466);
\draw [color=c, fill=c] (17.2438,1.87881) rectangle (17.2836,1.98466);
\draw [color=c, fill=c] (17.2836,1.87881) rectangle (17.3234,1.98466);
\draw [color=c, fill=c] (17.3234,1.87881) rectangle (17.3632,1.98466);
\draw [color=c, fill=c] (17.3632,1.87881) rectangle (17.403,1.98466);
\draw [color=c, fill=c] (17.403,1.87881) rectangle (17.4428,1.98466);
\draw [color=c, fill=c] (17.4428,1.87881) rectangle (17.4826,1.98466);
\draw [color=c, fill=c] (17.4826,1.87881) rectangle (17.5224,1.98466);
\draw [color=c, fill=c] (17.5224,1.87881) rectangle (17.5622,1.98466);
\draw [color=c, fill=c] (17.5622,1.87881) rectangle (17.602,1.98466);
\draw [color=c, fill=c] (17.602,1.87881) rectangle (17.6418,1.98466);
\draw [color=c, fill=c] (17.6418,1.87881) rectangle (17.6816,1.98466);
\draw [color=c, fill=c] (17.6816,1.87881) rectangle (17.7214,1.98466);
\draw [color=c, fill=c] (17.7214,1.87881) rectangle (17.7612,1.98466);
\draw [color=c, fill=c] (17.7612,1.87881) rectangle (17.801,1.98466);
\draw [color=c, fill=c] (17.801,1.87881) rectangle (17.8408,1.98466);
\draw [color=c, fill=c] (17.8408,1.87881) rectangle (17.8806,1.98466);
\draw [color=c, fill=c] (17.8806,1.87881) rectangle (17.9204,1.98466);
\draw [color=c, fill=c] (17.9204,1.87881) rectangle (17.9602,1.98466);
\draw [color=c, fill=c] (17.9602,1.87881) rectangle (18,1.98466);
\definecolor{c}{rgb}{1,0,0};
\draw [color=c, fill=c] (2,1.98466) rectangle (2.0398,2.09051);
\draw [color=c, fill=c] (2.0398,1.98466) rectangle (2.0796,2.09051);
\draw [color=c, fill=c] (2.0796,1.98466) rectangle (2.1194,2.09051);
\draw [color=c, fill=c] (2.1194,1.98466) rectangle (2.1592,2.09051);
\draw [color=c, fill=c] (2.1592,1.98466) rectangle (2.19901,2.09051);
\draw [color=c, fill=c] (2.19901,1.98466) rectangle (2.23881,2.09051);
\draw [color=c, fill=c] (2.23881,1.98466) rectangle (2.27861,2.09051);
\draw [color=c, fill=c] (2.27861,1.98466) rectangle (2.31841,2.09051);
\draw [color=c, fill=c] (2.31841,1.98466) rectangle (2.35821,2.09051);
\draw [color=c, fill=c] (2.35821,1.98466) rectangle (2.39801,2.09051);
\draw [color=c, fill=c] (2.39801,1.98466) rectangle (2.43781,2.09051);
\draw [color=c, fill=c] (2.43781,1.98466) rectangle (2.47761,2.09051);
\draw [color=c, fill=c] (2.47761,1.98466) rectangle (2.51741,2.09051);
\draw [color=c, fill=c] (2.51741,1.98466) rectangle (2.55721,2.09051);
\draw [color=c, fill=c] (2.55721,1.98466) rectangle (2.59702,2.09051);
\draw [color=c, fill=c] (2.59702,1.98466) rectangle (2.63682,2.09051);
\draw [color=c, fill=c] (2.63682,1.98466) rectangle (2.67662,2.09051);
\draw [color=c, fill=c] (2.67662,1.98466) rectangle (2.71642,2.09051);
\draw [color=c, fill=c] (2.71642,1.98466) rectangle (2.75622,2.09051);
\draw [color=c, fill=c] (2.75622,1.98466) rectangle (2.79602,2.09051);
\draw [color=c, fill=c] (2.79602,1.98466) rectangle (2.83582,2.09051);
\draw [color=c, fill=c] (2.83582,1.98466) rectangle (2.87562,2.09051);
\draw [color=c, fill=c] (2.87562,1.98466) rectangle (2.91542,2.09051);
\draw [color=c, fill=c] (2.91542,1.98466) rectangle (2.95522,2.09051);
\draw [color=c, fill=c] (2.95522,1.98466) rectangle (2.99502,2.09051);
\draw [color=c, fill=c] (2.99502,1.98466) rectangle (3.03483,2.09051);
\draw [color=c, fill=c] (3.03483,1.98466) rectangle (3.07463,2.09051);
\draw [color=c, fill=c] (3.07463,1.98466) rectangle (3.11443,2.09051);
\draw [color=c, fill=c] (3.11443,1.98466) rectangle (3.15423,2.09051);
\draw [color=c, fill=c] (3.15423,1.98466) rectangle (3.19403,2.09051);
\draw [color=c, fill=c] (3.19403,1.98466) rectangle (3.23383,2.09051);
\draw [color=c, fill=c] (3.23383,1.98466) rectangle (3.27363,2.09051);
\draw [color=c, fill=c] (3.27363,1.98466) rectangle (3.31343,2.09051);
\draw [color=c, fill=c] (3.31343,1.98466) rectangle (3.35323,2.09051);
\draw [color=c, fill=c] (3.35323,1.98466) rectangle (3.39303,2.09051);
\draw [color=c, fill=c] (3.39303,1.98466) rectangle (3.43284,2.09051);
\draw [color=c, fill=c] (3.43284,1.98466) rectangle (3.47264,2.09051);
\draw [color=c, fill=c] (3.47264,1.98466) rectangle (3.51244,2.09051);
\draw [color=c, fill=c] (3.51244,1.98466) rectangle (3.55224,2.09051);
\draw [color=c, fill=c] (3.55224,1.98466) rectangle (3.59204,2.09051);
\draw [color=c, fill=c] (3.59204,1.98466) rectangle (3.63184,2.09051);
\draw [color=c, fill=c] (3.63184,1.98466) rectangle (3.67164,2.09051);
\draw [color=c, fill=c] (3.67164,1.98466) rectangle (3.71144,2.09051);
\draw [color=c, fill=c] (3.71144,1.98466) rectangle (3.75124,2.09051);
\draw [color=c, fill=c] (3.75124,1.98466) rectangle (3.79104,2.09051);
\draw [color=c, fill=c] (3.79104,1.98466) rectangle (3.83085,2.09051);
\draw [color=c, fill=c] (3.83085,1.98466) rectangle (3.87065,2.09051);
\draw [color=c, fill=c] (3.87065,1.98466) rectangle (3.91045,2.09051);
\draw [color=c, fill=c] (3.91045,1.98466) rectangle (3.95025,2.09051);
\draw [color=c, fill=c] (3.95025,1.98466) rectangle (3.99005,2.09051);
\draw [color=c, fill=c] (3.99005,1.98466) rectangle (4.02985,2.09051);
\draw [color=c, fill=c] (4.02985,1.98466) rectangle (4.06965,2.09051);
\draw [color=c, fill=c] (4.06965,1.98466) rectangle (4.10945,2.09051);
\draw [color=c, fill=c] (4.10945,1.98466) rectangle (4.14925,2.09051);
\draw [color=c, fill=c] (4.14925,1.98466) rectangle (4.18905,2.09051);
\draw [color=c, fill=c] (4.18905,1.98466) rectangle (4.22886,2.09051);
\draw [color=c, fill=c] (4.22886,1.98466) rectangle (4.26866,2.09051);
\draw [color=c, fill=c] (4.26866,1.98466) rectangle (4.30846,2.09051);
\draw [color=c, fill=c] (4.30846,1.98466) rectangle (4.34826,2.09051);
\draw [color=c, fill=c] (4.34826,1.98466) rectangle (4.38806,2.09051);
\draw [color=c, fill=c] (4.38806,1.98466) rectangle (4.42786,2.09051);
\draw [color=c, fill=c] (4.42786,1.98466) rectangle (4.46766,2.09051);
\draw [color=c, fill=c] (4.46766,1.98466) rectangle (4.50746,2.09051);
\draw [color=c, fill=c] (4.50746,1.98466) rectangle (4.54726,2.09051);
\draw [color=c, fill=c] (4.54726,1.98466) rectangle (4.58706,2.09051);
\draw [color=c, fill=c] (4.58706,1.98466) rectangle (4.62687,2.09051);
\draw [color=c, fill=c] (4.62687,1.98466) rectangle (4.66667,2.09051);
\draw [color=c, fill=c] (4.66667,1.98466) rectangle (4.70647,2.09051);
\draw [color=c, fill=c] (4.70647,1.98466) rectangle (4.74627,2.09051);
\draw [color=c, fill=c] (4.74627,1.98466) rectangle (4.78607,2.09051);
\draw [color=c, fill=c] (4.78607,1.98466) rectangle (4.82587,2.09051);
\draw [color=c, fill=c] (4.82587,1.98466) rectangle (4.86567,2.09051);
\draw [color=c, fill=c] (4.86567,1.98466) rectangle (4.90547,2.09051);
\draw [color=c, fill=c] (4.90547,1.98466) rectangle (4.94527,2.09051);
\draw [color=c, fill=c] (4.94527,1.98466) rectangle (4.98507,2.09051);
\draw [color=c, fill=c] (4.98507,1.98466) rectangle (5.02488,2.09051);
\draw [color=c, fill=c] (5.02488,1.98466) rectangle (5.06468,2.09051);
\draw [color=c, fill=c] (5.06468,1.98466) rectangle (5.10448,2.09051);
\draw [color=c, fill=c] (5.10448,1.98466) rectangle (5.14428,2.09051);
\draw [color=c, fill=c] (5.14428,1.98466) rectangle (5.18408,2.09051);
\draw [color=c, fill=c] (5.18408,1.98466) rectangle (5.22388,2.09051);
\draw [color=c, fill=c] (5.22388,1.98466) rectangle (5.26368,2.09051);
\draw [color=c, fill=c] (5.26368,1.98466) rectangle (5.30348,2.09051);
\draw [color=c, fill=c] (5.30348,1.98466) rectangle (5.34328,2.09051);
\draw [color=c, fill=c] (5.34328,1.98466) rectangle (5.38308,2.09051);
\draw [color=c, fill=c] (5.38308,1.98466) rectangle (5.42289,2.09051);
\draw [color=c, fill=c] (5.42289,1.98466) rectangle (5.46269,2.09051);
\draw [color=c, fill=c] (5.46269,1.98466) rectangle (5.50249,2.09051);
\draw [color=c, fill=c] (5.50249,1.98466) rectangle (5.54229,2.09051);
\draw [color=c, fill=c] (5.54229,1.98466) rectangle (5.58209,2.09051);
\draw [color=c, fill=c] (5.58209,1.98466) rectangle (5.62189,2.09051);
\draw [color=c, fill=c] (5.62189,1.98466) rectangle (5.66169,2.09051);
\draw [color=c, fill=c] (5.66169,1.98466) rectangle (5.70149,2.09051);
\draw [color=c, fill=c] (5.70149,1.98466) rectangle (5.74129,2.09051);
\draw [color=c, fill=c] (5.74129,1.98466) rectangle (5.78109,2.09051);
\draw [color=c, fill=c] (5.78109,1.98466) rectangle (5.8209,2.09051);
\draw [color=c, fill=c] (5.8209,1.98466) rectangle (5.8607,2.09051);
\draw [color=c, fill=c] (5.8607,1.98466) rectangle (5.9005,2.09051);
\draw [color=c, fill=c] (5.9005,1.98466) rectangle (5.9403,2.09051);
\draw [color=c, fill=c] (5.9403,1.98466) rectangle (5.9801,2.09051);
\draw [color=c, fill=c] (5.9801,1.98466) rectangle (6.0199,2.09051);
\draw [color=c, fill=c] (6.0199,1.98466) rectangle (6.0597,2.09051);
\draw [color=c, fill=c] (6.0597,1.98466) rectangle (6.0995,2.09051);
\draw [color=c, fill=c] (6.0995,1.98466) rectangle (6.1393,2.09051);
\draw [color=c, fill=c] (6.1393,1.98466) rectangle (6.1791,2.09051);
\draw [color=c, fill=c] (6.1791,1.98466) rectangle (6.21891,2.09051);
\draw [color=c, fill=c] (6.21891,1.98466) rectangle (6.25871,2.09051);
\draw [color=c, fill=c] (6.25871,1.98466) rectangle (6.29851,2.09051);
\draw [color=c, fill=c] (6.29851,1.98466) rectangle (6.33831,2.09051);
\draw [color=c, fill=c] (6.33831,1.98466) rectangle (6.37811,2.09051);
\draw [color=c, fill=c] (6.37811,1.98466) rectangle (6.41791,2.09051);
\draw [color=c, fill=c] (6.41791,1.98466) rectangle (6.45771,2.09051);
\draw [color=c, fill=c] (6.45771,1.98466) rectangle (6.49751,2.09051);
\draw [color=c, fill=c] (6.49751,1.98466) rectangle (6.53731,2.09051);
\draw [color=c, fill=c] (6.53731,1.98466) rectangle (6.57711,2.09051);
\draw [color=c, fill=c] (6.57711,1.98466) rectangle (6.61692,2.09051);
\draw [color=c, fill=c] (6.61692,1.98466) rectangle (6.65672,2.09051);
\draw [color=c, fill=c] (6.65672,1.98466) rectangle (6.69652,2.09051);
\draw [color=c, fill=c] (6.69652,1.98466) rectangle (6.73632,2.09051);
\draw [color=c, fill=c] (6.73632,1.98466) rectangle (6.77612,2.09051);
\draw [color=c, fill=c] (6.77612,1.98466) rectangle (6.81592,2.09051);
\draw [color=c, fill=c] (6.81592,1.98466) rectangle (6.85572,2.09051);
\draw [color=c, fill=c] (6.85572,1.98466) rectangle (6.89552,2.09051);
\draw [color=c, fill=c] (6.89552,1.98466) rectangle (6.93532,2.09051);
\draw [color=c, fill=c] (6.93532,1.98466) rectangle (6.97512,2.09051);
\draw [color=c, fill=c] (6.97512,1.98466) rectangle (7.01493,2.09051);
\draw [color=c, fill=c] (7.01493,1.98466) rectangle (7.05473,2.09051);
\draw [color=c, fill=c] (7.05473,1.98466) rectangle (7.09453,2.09051);
\draw [color=c, fill=c] (7.09453,1.98466) rectangle (7.13433,2.09051);
\draw [color=c, fill=c] (7.13433,1.98466) rectangle (7.17413,2.09051);
\draw [color=c, fill=c] (7.17413,1.98466) rectangle (7.21393,2.09051);
\draw [color=c, fill=c] (7.21393,1.98466) rectangle (7.25373,2.09051);
\draw [color=c, fill=c] (7.25373,1.98466) rectangle (7.29353,2.09051);
\draw [color=c, fill=c] (7.29353,1.98466) rectangle (7.33333,2.09051);
\draw [color=c, fill=c] (7.33333,1.98466) rectangle (7.37313,2.09051);
\draw [color=c, fill=c] (7.37313,1.98466) rectangle (7.41294,2.09051);
\draw [color=c, fill=c] (7.41294,1.98466) rectangle (7.45274,2.09051);
\draw [color=c, fill=c] (7.45274,1.98466) rectangle (7.49254,2.09051);
\draw [color=c, fill=c] (7.49254,1.98466) rectangle (7.53234,2.09051);
\draw [color=c, fill=c] (7.53234,1.98466) rectangle (7.57214,2.09051);
\draw [color=c, fill=c] (7.57214,1.98466) rectangle (7.61194,2.09051);
\draw [color=c, fill=c] (7.61194,1.98466) rectangle (7.65174,2.09051);
\draw [color=c, fill=c] (7.65174,1.98466) rectangle (7.69154,2.09051);
\draw [color=c, fill=c] (7.69154,1.98466) rectangle (7.73134,2.09051);
\draw [color=c, fill=c] (7.73134,1.98466) rectangle (7.77114,2.09051);
\draw [color=c, fill=c] (7.77114,1.98466) rectangle (7.81095,2.09051);
\definecolor{c}{rgb}{1,0.186667,0};
\draw [color=c, fill=c] (7.81095,1.98466) rectangle (7.85075,2.09051);
\draw [color=c, fill=c] (7.85075,1.98466) rectangle (7.89055,2.09051);
\draw [color=c, fill=c] (7.89055,1.98466) rectangle (7.93035,2.09051);
\draw [color=c, fill=c] (7.93035,1.98466) rectangle (7.97015,2.09051);
\draw [color=c, fill=c] (7.97015,1.98466) rectangle (8.00995,2.09051);
\draw [color=c, fill=c] (8.00995,1.98466) rectangle (8.04975,2.09051);
\draw [color=c, fill=c] (8.04975,1.98466) rectangle (8.08955,2.09051);
\draw [color=c, fill=c] (8.08955,1.98466) rectangle (8.12935,2.09051);
\draw [color=c, fill=c] (8.12935,1.98466) rectangle (8.16915,2.09051);
\draw [color=c, fill=c] (8.16915,1.98466) rectangle (8.20895,2.09051);
\draw [color=c, fill=c] (8.20895,1.98466) rectangle (8.24876,2.09051);
\draw [color=c, fill=c] (8.24876,1.98466) rectangle (8.28856,2.09051);
\draw [color=c, fill=c] (8.28856,1.98466) rectangle (8.32836,2.09051);
\draw [color=c, fill=c] (8.32836,1.98466) rectangle (8.36816,2.09051);
\draw [color=c, fill=c] (8.36816,1.98466) rectangle (8.40796,2.09051);
\draw [color=c, fill=c] (8.40796,1.98466) rectangle (8.44776,2.09051);
\draw [color=c, fill=c] (8.44776,1.98466) rectangle (8.48756,2.09051);
\draw [color=c, fill=c] (8.48756,1.98466) rectangle (8.52736,2.09051);
\draw [color=c, fill=c] (8.52736,1.98466) rectangle (8.56716,2.09051);
\draw [color=c, fill=c] (8.56716,1.98466) rectangle (8.60697,2.09051);
\definecolor{c}{rgb}{1,0.466667,0};
\draw [color=c, fill=c] (8.60697,1.98466) rectangle (8.64677,2.09051);
\draw [color=c, fill=c] (8.64677,1.98466) rectangle (8.68657,2.09051);
\draw [color=c, fill=c] (8.68657,1.98466) rectangle (8.72637,2.09051);
\draw [color=c, fill=c] (8.72637,1.98466) rectangle (8.76617,2.09051);
\draw [color=c, fill=c] (8.76617,1.98466) rectangle (8.80597,2.09051);
\draw [color=c, fill=c] (8.80597,1.98466) rectangle (8.84577,2.09051);
\draw [color=c, fill=c] (8.84577,1.98466) rectangle (8.88557,2.09051);
\draw [color=c, fill=c] (8.88557,1.98466) rectangle (8.92537,2.09051);
\draw [color=c, fill=c] (8.92537,1.98466) rectangle (8.96517,2.09051);
\draw [color=c, fill=c] (8.96517,1.98466) rectangle (9.00498,2.09051);
\draw [color=c, fill=c] (9.00498,1.98466) rectangle (9.04478,2.09051);
\definecolor{c}{rgb}{1,0.653333,0};
\draw [color=c, fill=c] (9.04478,1.98466) rectangle (9.08458,2.09051);
\draw [color=c, fill=c] (9.08458,1.98466) rectangle (9.12438,2.09051);
\draw [color=c, fill=c] (9.12438,1.98466) rectangle (9.16418,2.09051);
\draw [color=c, fill=c] (9.16418,1.98466) rectangle (9.20398,2.09051);
\draw [color=c, fill=c] (9.20398,1.98466) rectangle (9.24378,2.09051);
\draw [color=c, fill=c] (9.24378,1.98466) rectangle (9.28358,2.09051);
\draw [color=c, fill=c] (9.28358,1.98466) rectangle (9.32338,2.09051);
\draw [color=c, fill=c] (9.32338,1.98466) rectangle (9.36318,2.09051);
\definecolor{c}{rgb}{1,0.933333,0};
\draw [color=c, fill=c] (9.36318,1.98466) rectangle (9.40298,2.09051);
\draw [color=c, fill=c] (9.40298,1.98466) rectangle (9.44279,2.09051);
\draw [color=c, fill=c] (9.44279,1.98466) rectangle (9.48259,2.09051);
\draw [color=c, fill=c] (9.48259,1.98466) rectangle (9.52239,2.09051);
\draw [color=c, fill=c] (9.52239,1.98466) rectangle (9.56219,2.09051);
\definecolor{c}{rgb}{0.88,1,0};
\draw [color=c, fill=c] (9.56219,1.98466) rectangle (9.60199,2.09051);
\draw [color=c, fill=c] (9.60199,1.98466) rectangle (9.64179,2.09051);
\draw [color=c, fill=c] (9.64179,1.98466) rectangle (9.68159,2.09051);
\draw [color=c, fill=c] (9.68159,1.98466) rectangle (9.72139,2.09051);
\definecolor{c}{rgb}{0.6,1,0};
\draw [color=c, fill=c] (9.72139,1.98466) rectangle (9.76119,2.09051);
\draw [color=c, fill=c] (9.76119,1.98466) rectangle (9.80099,2.09051);
\draw [color=c, fill=c] (9.80099,1.98466) rectangle (9.8408,2.09051);
\draw [color=c, fill=c] (9.8408,1.98466) rectangle (9.8806,2.09051);
\definecolor{c}{rgb}{0.413333,1,0};
\draw [color=c, fill=c] (9.8806,1.98466) rectangle (9.9204,2.09051);
\draw [color=c, fill=c] (9.9204,1.98466) rectangle (9.9602,2.09051);
\draw [color=c, fill=c] (9.9602,1.98466) rectangle (10,2.09051);
\draw [color=c, fill=c] (10,1.98466) rectangle (10.0398,2.09051);
\definecolor{c}{rgb}{0.133333,1,0};
\draw [color=c, fill=c] (10.0398,1.98466) rectangle (10.0796,2.09051);
\draw [color=c, fill=c] (10.0796,1.98466) rectangle (10.1194,2.09051);
\draw [color=c, fill=c] (10.1194,1.98466) rectangle (10.1592,2.09051);
\draw [color=c, fill=c] (10.1592,1.98466) rectangle (10.199,2.09051);
\definecolor{c}{rgb}{0,1,0.0533333};
\draw [color=c, fill=c] (10.199,1.98466) rectangle (10.2388,2.09051);
\draw [color=c, fill=c] (10.2388,1.98466) rectangle (10.2786,2.09051);
\draw [color=c, fill=c] (10.2786,1.98466) rectangle (10.3184,2.09051);
\draw [color=c, fill=c] (10.3184,1.98466) rectangle (10.3582,2.09051);
\draw [color=c, fill=c] (10.3582,1.98466) rectangle (10.398,2.09051);
\definecolor{c}{rgb}{0,1,0.333333};
\draw [color=c, fill=c] (10.398,1.98466) rectangle (10.4378,2.09051);
\draw [color=c, fill=c] (10.4378,1.98466) rectangle (10.4776,2.09051);
\draw [color=c, fill=c] (10.4776,1.98466) rectangle (10.5174,2.09051);
\draw [color=c, fill=c] (10.5174,1.98466) rectangle (10.5572,2.09051);
\draw [color=c, fill=c] (10.5572,1.98466) rectangle (10.597,2.09051);
\definecolor{c}{rgb}{0,1,0.52};
\draw [color=c, fill=c] (10.597,1.98466) rectangle (10.6368,2.09051);
\draw [color=c, fill=c] (10.6368,1.98466) rectangle (10.6766,2.09051);
\draw [color=c, fill=c] (10.6766,1.98466) rectangle (10.7164,2.09051);
\draw [color=c, fill=c] (10.7164,1.98466) rectangle (10.7562,2.09051);
\draw [color=c, fill=c] (10.7562,1.98466) rectangle (10.796,2.09051);
\draw [color=c, fill=c] (10.796,1.98466) rectangle (10.8358,2.09051);
\draw [color=c, fill=c] (10.8358,1.98466) rectangle (10.8756,2.09051);
\draw [color=c, fill=c] (10.8756,1.98466) rectangle (10.9154,2.09051);
\definecolor{c}{rgb}{0,1,0.8};
\draw [color=c, fill=c] (10.9154,1.98466) rectangle (10.9552,2.09051);
\draw [color=c, fill=c] (10.9552,1.98466) rectangle (10.995,2.09051);
\draw [color=c, fill=c] (10.995,1.98466) rectangle (11.0348,2.09051);
\draw [color=c, fill=c] (11.0348,1.98466) rectangle (11.0746,2.09051);
\draw [color=c, fill=c] (11.0746,1.98466) rectangle (11.1144,2.09051);
\draw [color=c, fill=c] (11.1144,1.98466) rectangle (11.1542,2.09051);
\draw [color=c, fill=c] (11.1542,1.98466) rectangle (11.194,2.09051);
\draw [color=c, fill=c] (11.194,1.98466) rectangle (11.2338,2.09051);
\draw [color=c, fill=c] (11.2338,1.98466) rectangle (11.2736,2.09051);
\draw [color=c, fill=c] (11.2736,1.98466) rectangle (11.3134,2.09051);
\draw [color=c, fill=c] (11.3134,1.98466) rectangle (11.3532,2.09051);
\draw [color=c, fill=c] (11.3532,1.98466) rectangle (11.393,2.09051);
\definecolor{c}{rgb}{0,1,0.986667};
\draw [color=c, fill=c] (11.393,1.98466) rectangle (11.4328,2.09051);
\draw [color=c, fill=c] (11.4328,1.98466) rectangle (11.4726,2.09051);
\draw [color=c, fill=c] (11.4726,1.98466) rectangle (11.5124,2.09051);
\draw [color=c, fill=c] (11.5124,1.98466) rectangle (11.5522,2.09051);
\draw [color=c, fill=c] (11.5522,1.98466) rectangle (11.592,2.09051);
\draw [color=c, fill=c] (11.592,1.98466) rectangle (11.6318,2.09051);
\draw [color=c, fill=c] (11.6318,1.98466) rectangle (11.6716,2.09051);
\draw [color=c, fill=c] (11.6716,1.98466) rectangle (11.7114,2.09051);
\draw [color=c, fill=c] (11.7114,1.98466) rectangle (11.7512,2.09051);
\draw [color=c, fill=c] (11.7512,1.98466) rectangle (11.791,2.09051);
\draw [color=c, fill=c] (11.791,1.98466) rectangle (11.8308,2.09051);
\draw [color=c, fill=c] (11.8308,1.98466) rectangle (11.8706,2.09051);
\draw [color=c, fill=c] (11.8706,1.98466) rectangle (11.9104,2.09051);
\draw [color=c, fill=c] (11.9104,1.98466) rectangle (11.9502,2.09051);
\draw [color=c, fill=c] (11.9502,1.98466) rectangle (11.99,2.09051);
\draw [color=c, fill=c] (11.99,1.98466) rectangle (12.0299,2.09051);
\draw [color=c, fill=c] (12.0299,1.98466) rectangle (12.0697,2.09051);
\draw [color=c, fill=c] (12.0697,1.98466) rectangle (12.1095,2.09051);
\draw [color=c, fill=c] (12.1095,1.98466) rectangle (12.1493,2.09051);
\draw [color=c, fill=c] (12.1493,1.98466) rectangle (12.1891,2.09051);
\draw [color=c, fill=c] (12.1891,1.98466) rectangle (12.2289,2.09051);
\draw [color=c, fill=c] (12.2289,1.98466) rectangle (12.2687,2.09051);
\draw [color=c, fill=c] (12.2687,1.98466) rectangle (12.3085,2.09051);
\definecolor{c}{rgb}{0,0.733333,1};
\draw [color=c, fill=c] (12.3085,1.98466) rectangle (12.3483,2.09051);
\draw [color=c, fill=c] (12.3483,1.98466) rectangle (12.3881,2.09051);
\draw [color=c, fill=c] (12.3881,1.98466) rectangle (12.4279,2.09051);
\draw [color=c, fill=c] (12.4279,1.98466) rectangle (12.4677,2.09051);
\draw [color=c, fill=c] (12.4677,1.98466) rectangle (12.5075,2.09051);
\draw [color=c, fill=c] (12.5075,1.98466) rectangle (12.5473,2.09051);
\draw [color=c, fill=c] (12.5473,1.98466) rectangle (12.5871,2.09051);
\draw [color=c, fill=c] (12.5871,1.98466) rectangle (12.6269,2.09051);
\draw [color=c, fill=c] (12.6269,1.98466) rectangle (12.6667,2.09051);
\draw [color=c, fill=c] (12.6667,1.98466) rectangle (12.7065,2.09051);
\draw [color=c, fill=c] (12.7065,1.98466) rectangle (12.7463,2.09051);
\draw [color=c, fill=c] (12.7463,1.98466) rectangle (12.7861,2.09051);
\draw [color=c, fill=c] (12.7861,1.98466) rectangle (12.8259,2.09051);
\draw [color=c, fill=c] (12.8259,1.98466) rectangle (12.8657,2.09051);
\draw [color=c, fill=c] (12.8657,1.98466) rectangle (12.9055,2.09051);
\draw [color=c, fill=c] (12.9055,1.98466) rectangle (12.9453,2.09051);
\draw [color=c, fill=c] (12.9453,1.98466) rectangle (12.9851,2.09051);
\draw [color=c, fill=c] (12.9851,1.98466) rectangle (13.0249,2.09051);
\draw [color=c, fill=c] (13.0249,1.98466) rectangle (13.0647,2.09051);
\draw [color=c, fill=c] (13.0647,1.98466) rectangle (13.1045,2.09051);
\draw [color=c, fill=c] (13.1045,1.98466) rectangle (13.1443,2.09051);
\draw [color=c, fill=c] (13.1443,1.98466) rectangle (13.1841,2.09051);
\draw [color=c, fill=c] (13.1841,1.98466) rectangle (13.2239,2.09051);
\draw [color=c, fill=c] (13.2239,1.98466) rectangle (13.2637,2.09051);
\draw [color=c, fill=c] (13.2637,1.98466) rectangle (13.3035,2.09051);
\draw [color=c, fill=c] (13.3035,1.98466) rectangle (13.3433,2.09051);
\draw [color=c, fill=c] (13.3433,1.98466) rectangle (13.3831,2.09051);
\draw [color=c, fill=c] (13.3831,1.98466) rectangle (13.4229,2.09051);
\draw [color=c, fill=c] (13.4229,1.98466) rectangle (13.4627,2.09051);
\draw [color=c, fill=c] (13.4627,1.98466) rectangle (13.5025,2.09051);
\draw [color=c, fill=c] (13.5025,1.98466) rectangle (13.5423,2.09051);
\draw [color=c, fill=c] (13.5423,1.98466) rectangle (13.5821,2.09051);
\draw [color=c, fill=c] (13.5821,1.98466) rectangle (13.6219,2.09051);
\draw [color=c, fill=c] (13.6219,1.98466) rectangle (13.6617,2.09051);
\draw [color=c, fill=c] (13.6617,1.98466) rectangle (13.7015,2.09051);
\draw [color=c, fill=c] (13.7015,1.98466) rectangle (13.7413,2.09051);
\draw [color=c, fill=c] (13.7413,1.98466) rectangle (13.7811,2.09051);
\draw [color=c, fill=c] (13.7811,1.98466) rectangle (13.8209,2.09051);
\draw [color=c, fill=c] (13.8209,1.98466) rectangle (13.8607,2.09051);
\draw [color=c, fill=c] (13.8607,1.98466) rectangle (13.9005,2.09051);
\draw [color=c, fill=c] (13.9005,1.98466) rectangle (13.9403,2.09051);
\draw [color=c, fill=c] (13.9403,1.98466) rectangle (13.9801,2.09051);
\draw [color=c, fill=c] (13.9801,1.98466) rectangle (14.0199,2.09051);
\draw [color=c, fill=c] (14.0199,1.98466) rectangle (14.0597,2.09051);
\draw [color=c, fill=c] (14.0597,1.98466) rectangle (14.0995,2.09051);
\draw [color=c, fill=c] (14.0995,1.98466) rectangle (14.1393,2.09051);
\draw [color=c, fill=c] (14.1393,1.98466) rectangle (14.1791,2.09051);
\draw [color=c, fill=c] (14.1791,1.98466) rectangle (14.2189,2.09051);
\draw [color=c, fill=c] (14.2189,1.98466) rectangle (14.2587,2.09051);
\draw [color=c, fill=c] (14.2587,1.98466) rectangle (14.2985,2.09051);
\draw [color=c, fill=c] (14.2985,1.98466) rectangle (14.3383,2.09051);
\draw [color=c, fill=c] (14.3383,1.98466) rectangle (14.3781,2.09051);
\draw [color=c, fill=c] (14.3781,1.98466) rectangle (14.4179,2.09051);
\draw [color=c, fill=c] (14.4179,1.98466) rectangle (14.4577,2.09051);
\draw [color=c, fill=c] (14.4577,1.98466) rectangle (14.4975,2.09051);
\draw [color=c, fill=c] (14.4975,1.98466) rectangle (14.5373,2.09051);
\draw [color=c, fill=c] (14.5373,1.98466) rectangle (14.5771,2.09051);
\draw [color=c, fill=c] (14.5771,1.98466) rectangle (14.6169,2.09051);
\draw [color=c, fill=c] (14.6169,1.98466) rectangle (14.6567,2.09051);
\draw [color=c, fill=c] (14.6567,1.98466) rectangle (14.6965,2.09051);
\draw [color=c, fill=c] (14.6965,1.98466) rectangle (14.7363,2.09051);
\draw [color=c, fill=c] (14.7363,1.98466) rectangle (14.7761,2.09051);
\draw [color=c, fill=c] (14.7761,1.98466) rectangle (14.8159,2.09051);
\draw [color=c, fill=c] (14.8159,1.98466) rectangle (14.8557,2.09051);
\draw [color=c, fill=c] (14.8557,1.98466) rectangle (14.8955,2.09051);
\draw [color=c, fill=c] (14.8955,1.98466) rectangle (14.9353,2.09051);
\draw [color=c, fill=c] (14.9353,1.98466) rectangle (14.9751,2.09051);
\draw [color=c, fill=c] (14.9751,1.98466) rectangle (15.0149,2.09051);
\draw [color=c, fill=c] (15.0149,1.98466) rectangle (15.0547,2.09051);
\draw [color=c, fill=c] (15.0547,1.98466) rectangle (15.0945,2.09051);
\draw [color=c, fill=c] (15.0945,1.98466) rectangle (15.1343,2.09051);
\draw [color=c, fill=c] (15.1343,1.98466) rectangle (15.1741,2.09051);
\draw [color=c, fill=c] (15.1741,1.98466) rectangle (15.2139,2.09051);
\draw [color=c, fill=c] (15.2139,1.98466) rectangle (15.2537,2.09051);
\draw [color=c, fill=c] (15.2537,1.98466) rectangle (15.2935,2.09051);
\draw [color=c, fill=c] (15.2935,1.98466) rectangle (15.3333,2.09051);
\draw [color=c, fill=c] (15.3333,1.98466) rectangle (15.3731,2.09051);
\draw [color=c, fill=c] (15.3731,1.98466) rectangle (15.4129,2.09051);
\draw [color=c, fill=c] (15.4129,1.98466) rectangle (15.4527,2.09051);
\draw [color=c, fill=c] (15.4527,1.98466) rectangle (15.4925,2.09051);
\draw [color=c, fill=c] (15.4925,1.98466) rectangle (15.5323,2.09051);
\draw [color=c, fill=c] (15.5323,1.98466) rectangle (15.5721,2.09051);
\draw [color=c, fill=c] (15.5721,1.98466) rectangle (15.6119,2.09051);
\draw [color=c, fill=c] (15.6119,1.98466) rectangle (15.6517,2.09051);
\draw [color=c, fill=c] (15.6517,1.98466) rectangle (15.6915,2.09051);
\draw [color=c, fill=c] (15.6915,1.98466) rectangle (15.7313,2.09051);
\draw [color=c, fill=c] (15.7313,1.98466) rectangle (15.7711,2.09051);
\draw [color=c, fill=c] (15.7711,1.98466) rectangle (15.8109,2.09051);
\draw [color=c, fill=c] (15.8109,1.98466) rectangle (15.8507,2.09051);
\draw [color=c, fill=c] (15.8507,1.98466) rectangle (15.8905,2.09051);
\draw [color=c, fill=c] (15.8905,1.98466) rectangle (15.9303,2.09051);
\draw [color=c, fill=c] (15.9303,1.98466) rectangle (15.9701,2.09051);
\draw [color=c, fill=c] (15.9701,1.98466) rectangle (16.01,2.09051);
\draw [color=c, fill=c] (16.01,1.98466) rectangle (16.0498,2.09051);
\draw [color=c, fill=c] (16.0498,1.98466) rectangle (16.0896,2.09051);
\draw [color=c, fill=c] (16.0896,1.98466) rectangle (16.1294,2.09051);
\draw [color=c, fill=c] (16.1294,1.98466) rectangle (16.1692,2.09051);
\draw [color=c, fill=c] (16.1692,1.98466) rectangle (16.209,2.09051);
\draw [color=c, fill=c] (16.209,1.98466) rectangle (16.2488,2.09051);
\draw [color=c, fill=c] (16.2488,1.98466) rectangle (16.2886,2.09051);
\draw [color=c, fill=c] (16.2886,1.98466) rectangle (16.3284,2.09051);
\draw [color=c, fill=c] (16.3284,1.98466) rectangle (16.3682,2.09051);
\draw [color=c, fill=c] (16.3682,1.98466) rectangle (16.408,2.09051);
\draw [color=c, fill=c] (16.408,1.98466) rectangle (16.4478,2.09051);
\draw [color=c, fill=c] (16.4478,1.98466) rectangle (16.4876,2.09051);
\draw [color=c, fill=c] (16.4876,1.98466) rectangle (16.5274,2.09051);
\draw [color=c, fill=c] (16.5274,1.98466) rectangle (16.5672,2.09051);
\draw [color=c, fill=c] (16.5672,1.98466) rectangle (16.607,2.09051);
\draw [color=c, fill=c] (16.607,1.98466) rectangle (16.6468,2.09051);
\draw [color=c, fill=c] (16.6468,1.98466) rectangle (16.6866,2.09051);
\draw [color=c, fill=c] (16.6866,1.98466) rectangle (16.7264,2.09051);
\draw [color=c, fill=c] (16.7264,1.98466) rectangle (16.7662,2.09051);
\draw [color=c, fill=c] (16.7662,1.98466) rectangle (16.806,2.09051);
\draw [color=c, fill=c] (16.806,1.98466) rectangle (16.8458,2.09051);
\draw [color=c, fill=c] (16.8458,1.98466) rectangle (16.8856,2.09051);
\draw [color=c, fill=c] (16.8856,1.98466) rectangle (16.9254,2.09051);
\draw [color=c, fill=c] (16.9254,1.98466) rectangle (16.9652,2.09051);
\draw [color=c, fill=c] (16.9652,1.98466) rectangle (17.005,2.09051);
\draw [color=c, fill=c] (17.005,1.98466) rectangle (17.0448,2.09051);
\draw [color=c, fill=c] (17.0448,1.98466) rectangle (17.0846,2.09051);
\draw [color=c, fill=c] (17.0846,1.98466) rectangle (17.1244,2.09051);
\draw [color=c, fill=c] (17.1244,1.98466) rectangle (17.1642,2.09051);
\draw [color=c, fill=c] (17.1642,1.98466) rectangle (17.204,2.09051);
\draw [color=c, fill=c] (17.204,1.98466) rectangle (17.2438,2.09051);
\draw [color=c, fill=c] (17.2438,1.98466) rectangle (17.2836,2.09051);
\draw [color=c, fill=c] (17.2836,1.98466) rectangle (17.3234,2.09051);
\draw [color=c, fill=c] (17.3234,1.98466) rectangle (17.3632,2.09051);
\draw [color=c, fill=c] (17.3632,1.98466) rectangle (17.403,2.09051);
\draw [color=c, fill=c] (17.403,1.98466) rectangle (17.4428,2.09051);
\draw [color=c, fill=c] (17.4428,1.98466) rectangle (17.4826,2.09051);
\draw [color=c, fill=c] (17.4826,1.98466) rectangle (17.5224,2.09051);
\draw [color=c, fill=c] (17.5224,1.98466) rectangle (17.5622,2.09051);
\draw [color=c, fill=c] (17.5622,1.98466) rectangle (17.602,2.09051);
\draw [color=c, fill=c] (17.602,1.98466) rectangle (17.6418,2.09051);
\draw [color=c, fill=c] (17.6418,1.98466) rectangle (17.6816,2.09051);
\draw [color=c, fill=c] (17.6816,1.98466) rectangle (17.7214,2.09051);
\draw [color=c, fill=c] (17.7214,1.98466) rectangle (17.7612,2.09051);
\draw [color=c, fill=c] (17.7612,1.98466) rectangle (17.801,2.09051);
\draw [color=c, fill=c] (17.801,1.98466) rectangle (17.8408,2.09051);
\draw [color=c, fill=c] (17.8408,1.98466) rectangle (17.8806,2.09051);
\draw [color=c, fill=c] (17.8806,1.98466) rectangle (17.9204,2.09051);
\draw [color=c, fill=c] (17.9204,1.98466) rectangle (17.9602,2.09051);
\draw [color=c, fill=c] (17.9602,1.98466) rectangle (18,2.09051);
\definecolor{c}{rgb}{1,0,0};
\draw [color=c, fill=c] (2,2.09051) rectangle (2.0398,2.19636);
\draw [color=c, fill=c] (2.0398,2.09051) rectangle (2.0796,2.19636);
\draw [color=c, fill=c] (2.0796,2.09051) rectangle (2.1194,2.19636);
\draw [color=c, fill=c] (2.1194,2.09051) rectangle (2.1592,2.19636);
\draw [color=c, fill=c] (2.1592,2.09051) rectangle (2.19901,2.19636);
\draw [color=c, fill=c] (2.19901,2.09051) rectangle (2.23881,2.19636);
\draw [color=c, fill=c] (2.23881,2.09051) rectangle (2.27861,2.19636);
\draw [color=c, fill=c] (2.27861,2.09051) rectangle (2.31841,2.19636);
\draw [color=c, fill=c] (2.31841,2.09051) rectangle (2.35821,2.19636);
\draw [color=c, fill=c] (2.35821,2.09051) rectangle (2.39801,2.19636);
\draw [color=c, fill=c] (2.39801,2.09051) rectangle (2.43781,2.19636);
\draw [color=c, fill=c] (2.43781,2.09051) rectangle (2.47761,2.19636);
\draw [color=c, fill=c] (2.47761,2.09051) rectangle (2.51741,2.19636);
\draw [color=c, fill=c] (2.51741,2.09051) rectangle (2.55721,2.19636);
\draw [color=c, fill=c] (2.55721,2.09051) rectangle (2.59702,2.19636);
\draw [color=c, fill=c] (2.59702,2.09051) rectangle (2.63682,2.19636);
\draw [color=c, fill=c] (2.63682,2.09051) rectangle (2.67662,2.19636);
\draw [color=c, fill=c] (2.67662,2.09051) rectangle (2.71642,2.19636);
\draw [color=c, fill=c] (2.71642,2.09051) rectangle (2.75622,2.19636);
\draw [color=c, fill=c] (2.75622,2.09051) rectangle (2.79602,2.19636);
\draw [color=c, fill=c] (2.79602,2.09051) rectangle (2.83582,2.19636);
\draw [color=c, fill=c] (2.83582,2.09051) rectangle (2.87562,2.19636);
\draw [color=c, fill=c] (2.87562,2.09051) rectangle (2.91542,2.19636);
\draw [color=c, fill=c] (2.91542,2.09051) rectangle (2.95522,2.19636);
\draw [color=c, fill=c] (2.95522,2.09051) rectangle (2.99502,2.19636);
\draw [color=c, fill=c] (2.99502,2.09051) rectangle (3.03483,2.19636);
\draw [color=c, fill=c] (3.03483,2.09051) rectangle (3.07463,2.19636);
\draw [color=c, fill=c] (3.07463,2.09051) rectangle (3.11443,2.19636);
\draw [color=c, fill=c] (3.11443,2.09051) rectangle (3.15423,2.19636);
\draw [color=c, fill=c] (3.15423,2.09051) rectangle (3.19403,2.19636);
\draw [color=c, fill=c] (3.19403,2.09051) rectangle (3.23383,2.19636);
\draw [color=c, fill=c] (3.23383,2.09051) rectangle (3.27363,2.19636);
\draw [color=c, fill=c] (3.27363,2.09051) rectangle (3.31343,2.19636);
\draw [color=c, fill=c] (3.31343,2.09051) rectangle (3.35323,2.19636);
\draw [color=c, fill=c] (3.35323,2.09051) rectangle (3.39303,2.19636);
\draw [color=c, fill=c] (3.39303,2.09051) rectangle (3.43284,2.19636);
\draw [color=c, fill=c] (3.43284,2.09051) rectangle (3.47264,2.19636);
\draw [color=c, fill=c] (3.47264,2.09051) rectangle (3.51244,2.19636);
\draw [color=c, fill=c] (3.51244,2.09051) rectangle (3.55224,2.19636);
\draw [color=c, fill=c] (3.55224,2.09051) rectangle (3.59204,2.19636);
\draw [color=c, fill=c] (3.59204,2.09051) rectangle (3.63184,2.19636);
\draw [color=c, fill=c] (3.63184,2.09051) rectangle (3.67164,2.19636);
\draw [color=c, fill=c] (3.67164,2.09051) rectangle (3.71144,2.19636);
\draw [color=c, fill=c] (3.71144,2.09051) rectangle (3.75124,2.19636);
\draw [color=c, fill=c] (3.75124,2.09051) rectangle (3.79104,2.19636);
\draw [color=c, fill=c] (3.79104,2.09051) rectangle (3.83085,2.19636);
\draw [color=c, fill=c] (3.83085,2.09051) rectangle (3.87065,2.19636);
\draw [color=c, fill=c] (3.87065,2.09051) rectangle (3.91045,2.19636);
\draw [color=c, fill=c] (3.91045,2.09051) rectangle (3.95025,2.19636);
\draw [color=c, fill=c] (3.95025,2.09051) rectangle (3.99005,2.19636);
\draw [color=c, fill=c] (3.99005,2.09051) rectangle (4.02985,2.19636);
\draw [color=c, fill=c] (4.02985,2.09051) rectangle (4.06965,2.19636);
\draw [color=c, fill=c] (4.06965,2.09051) rectangle (4.10945,2.19636);
\draw [color=c, fill=c] (4.10945,2.09051) rectangle (4.14925,2.19636);
\draw [color=c, fill=c] (4.14925,2.09051) rectangle (4.18905,2.19636);
\draw [color=c, fill=c] (4.18905,2.09051) rectangle (4.22886,2.19636);
\draw [color=c, fill=c] (4.22886,2.09051) rectangle (4.26866,2.19636);
\draw [color=c, fill=c] (4.26866,2.09051) rectangle (4.30846,2.19636);
\draw [color=c, fill=c] (4.30846,2.09051) rectangle (4.34826,2.19636);
\draw [color=c, fill=c] (4.34826,2.09051) rectangle (4.38806,2.19636);
\draw [color=c, fill=c] (4.38806,2.09051) rectangle (4.42786,2.19636);
\draw [color=c, fill=c] (4.42786,2.09051) rectangle (4.46766,2.19636);
\draw [color=c, fill=c] (4.46766,2.09051) rectangle (4.50746,2.19636);
\draw [color=c, fill=c] (4.50746,2.09051) rectangle (4.54726,2.19636);
\draw [color=c, fill=c] (4.54726,2.09051) rectangle (4.58706,2.19636);
\draw [color=c, fill=c] (4.58706,2.09051) rectangle (4.62687,2.19636);
\draw [color=c, fill=c] (4.62687,2.09051) rectangle (4.66667,2.19636);
\draw [color=c, fill=c] (4.66667,2.09051) rectangle (4.70647,2.19636);
\draw [color=c, fill=c] (4.70647,2.09051) rectangle (4.74627,2.19636);
\draw [color=c, fill=c] (4.74627,2.09051) rectangle (4.78607,2.19636);
\draw [color=c, fill=c] (4.78607,2.09051) rectangle (4.82587,2.19636);
\draw [color=c, fill=c] (4.82587,2.09051) rectangle (4.86567,2.19636);
\draw [color=c, fill=c] (4.86567,2.09051) rectangle (4.90547,2.19636);
\draw [color=c, fill=c] (4.90547,2.09051) rectangle (4.94527,2.19636);
\draw [color=c, fill=c] (4.94527,2.09051) rectangle (4.98507,2.19636);
\draw [color=c, fill=c] (4.98507,2.09051) rectangle (5.02488,2.19636);
\draw [color=c, fill=c] (5.02488,2.09051) rectangle (5.06468,2.19636);
\draw [color=c, fill=c] (5.06468,2.09051) rectangle (5.10448,2.19636);
\draw [color=c, fill=c] (5.10448,2.09051) rectangle (5.14428,2.19636);
\draw [color=c, fill=c] (5.14428,2.09051) rectangle (5.18408,2.19636);
\draw [color=c, fill=c] (5.18408,2.09051) rectangle (5.22388,2.19636);
\draw [color=c, fill=c] (5.22388,2.09051) rectangle (5.26368,2.19636);
\draw [color=c, fill=c] (5.26368,2.09051) rectangle (5.30348,2.19636);
\draw [color=c, fill=c] (5.30348,2.09051) rectangle (5.34328,2.19636);
\draw [color=c, fill=c] (5.34328,2.09051) rectangle (5.38308,2.19636);
\draw [color=c, fill=c] (5.38308,2.09051) rectangle (5.42289,2.19636);
\draw [color=c, fill=c] (5.42289,2.09051) rectangle (5.46269,2.19636);
\draw [color=c, fill=c] (5.46269,2.09051) rectangle (5.50249,2.19636);
\draw [color=c, fill=c] (5.50249,2.09051) rectangle (5.54229,2.19636);
\draw [color=c, fill=c] (5.54229,2.09051) rectangle (5.58209,2.19636);
\draw [color=c, fill=c] (5.58209,2.09051) rectangle (5.62189,2.19636);
\draw [color=c, fill=c] (5.62189,2.09051) rectangle (5.66169,2.19636);
\draw [color=c, fill=c] (5.66169,2.09051) rectangle (5.70149,2.19636);
\draw [color=c, fill=c] (5.70149,2.09051) rectangle (5.74129,2.19636);
\draw [color=c, fill=c] (5.74129,2.09051) rectangle (5.78109,2.19636);
\draw [color=c, fill=c] (5.78109,2.09051) rectangle (5.8209,2.19636);
\draw [color=c, fill=c] (5.8209,2.09051) rectangle (5.8607,2.19636);
\draw [color=c, fill=c] (5.8607,2.09051) rectangle (5.9005,2.19636);
\draw [color=c, fill=c] (5.9005,2.09051) rectangle (5.9403,2.19636);
\draw [color=c, fill=c] (5.9403,2.09051) rectangle (5.9801,2.19636);
\draw [color=c, fill=c] (5.9801,2.09051) rectangle (6.0199,2.19636);
\draw [color=c, fill=c] (6.0199,2.09051) rectangle (6.0597,2.19636);
\draw [color=c, fill=c] (6.0597,2.09051) rectangle (6.0995,2.19636);
\draw [color=c, fill=c] (6.0995,2.09051) rectangle (6.1393,2.19636);
\draw [color=c, fill=c] (6.1393,2.09051) rectangle (6.1791,2.19636);
\draw [color=c, fill=c] (6.1791,2.09051) rectangle (6.21891,2.19636);
\draw [color=c, fill=c] (6.21891,2.09051) rectangle (6.25871,2.19636);
\draw [color=c, fill=c] (6.25871,2.09051) rectangle (6.29851,2.19636);
\draw [color=c, fill=c] (6.29851,2.09051) rectangle (6.33831,2.19636);
\draw [color=c, fill=c] (6.33831,2.09051) rectangle (6.37811,2.19636);
\draw [color=c, fill=c] (6.37811,2.09051) rectangle (6.41791,2.19636);
\draw [color=c, fill=c] (6.41791,2.09051) rectangle (6.45771,2.19636);
\draw [color=c, fill=c] (6.45771,2.09051) rectangle (6.49751,2.19636);
\draw [color=c, fill=c] (6.49751,2.09051) rectangle (6.53731,2.19636);
\draw [color=c, fill=c] (6.53731,2.09051) rectangle (6.57711,2.19636);
\draw [color=c, fill=c] (6.57711,2.09051) rectangle (6.61692,2.19636);
\draw [color=c, fill=c] (6.61692,2.09051) rectangle (6.65672,2.19636);
\draw [color=c, fill=c] (6.65672,2.09051) rectangle (6.69652,2.19636);
\draw [color=c, fill=c] (6.69652,2.09051) rectangle (6.73632,2.19636);
\draw [color=c, fill=c] (6.73632,2.09051) rectangle (6.77612,2.19636);
\draw [color=c, fill=c] (6.77612,2.09051) rectangle (6.81592,2.19636);
\draw [color=c, fill=c] (6.81592,2.09051) rectangle (6.85572,2.19636);
\draw [color=c, fill=c] (6.85572,2.09051) rectangle (6.89552,2.19636);
\draw [color=c, fill=c] (6.89552,2.09051) rectangle (6.93532,2.19636);
\draw [color=c, fill=c] (6.93532,2.09051) rectangle (6.97512,2.19636);
\draw [color=c, fill=c] (6.97512,2.09051) rectangle (7.01493,2.19636);
\draw [color=c, fill=c] (7.01493,2.09051) rectangle (7.05473,2.19636);
\draw [color=c, fill=c] (7.05473,2.09051) rectangle (7.09453,2.19636);
\draw [color=c, fill=c] (7.09453,2.09051) rectangle (7.13433,2.19636);
\draw [color=c, fill=c] (7.13433,2.09051) rectangle (7.17413,2.19636);
\draw [color=c, fill=c] (7.17413,2.09051) rectangle (7.21393,2.19636);
\draw [color=c, fill=c] (7.21393,2.09051) rectangle (7.25373,2.19636);
\draw [color=c, fill=c] (7.25373,2.09051) rectangle (7.29353,2.19636);
\draw [color=c, fill=c] (7.29353,2.09051) rectangle (7.33333,2.19636);
\draw [color=c, fill=c] (7.33333,2.09051) rectangle (7.37313,2.19636);
\draw [color=c, fill=c] (7.37313,2.09051) rectangle (7.41294,2.19636);
\draw [color=c, fill=c] (7.41294,2.09051) rectangle (7.45274,2.19636);
\draw [color=c, fill=c] (7.45274,2.09051) rectangle (7.49254,2.19636);
\draw [color=c, fill=c] (7.49254,2.09051) rectangle (7.53234,2.19636);
\draw [color=c, fill=c] (7.53234,2.09051) rectangle (7.57214,2.19636);
\draw [color=c, fill=c] (7.57214,2.09051) rectangle (7.61194,2.19636);
\draw [color=c, fill=c] (7.61194,2.09051) rectangle (7.65174,2.19636);
\draw [color=c, fill=c] (7.65174,2.09051) rectangle (7.69154,2.19636);
\draw [color=c, fill=c] (7.69154,2.09051) rectangle (7.73134,2.19636);
\draw [color=c, fill=c] (7.73134,2.09051) rectangle (7.77114,2.19636);
\draw [color=c, fill=c] (7.77114,2.09051) rectangle (7.81095,2.19636);
\definecolor{c}{rgb}{1,0.186667,0};
\draw [color=c, fill=c] (7.81095,2.09051) rectangle (7.85075,2.19636);
\draw [color=c, fill=c] (7.85075,2.09051) rectangle (7.89055,2.19636);
\draw [color=c, fill=c] (7.89055,2.09051) rectangle (7.93035,2.19636);
\draw [color=c, fill=c] (7.93035,2.09051) rectangle (7.97015,2.19636);
\draw [color=c, fill=c] (7.97015,2.09051) rectangle (8.00995,2.19636);
\draw [color=c, fill=c] (8.00995,2.09051) rectangle (8.04975,2.19636);
\draw [color=c, fill=c] (8.04975,2.09051) rectangle (8.08955,2.19636);
\draw [color=c, fill=c] (8.08955,2.09051) rectangle (8.12935,2.19636);
\draw [color=c, fill=c] (8.12935,2.09051) rectangle (8.16915,2.19636);
\draw [color=c, fill=c] (8.16915,2.09051) rectangle (8.20895,2.19636);
\draw [color=c, fill=c] (8.20895,2.09051) rectangle (8.24876,2.19636);
\draw [color=c, fill=c] (8.24876,2.09051) rectangle (8.28856,2.19636);
\draw [color=c, fill=c] (8.28856,2.09051) rectangle (8.32836,2.19636);
\draw [color=c, fill=c] (8.32836,2.09051) rectangle (8.36816,2.19636);
\draw [color=c, fill=c] (8.36816,2.09051) rectangle (8.40796,2.19636);
\draw [color=c, fill=c] (8.40796,2.09051) rectangle (8.44776,2.19636);
\draw [color=c, fill=c] (8.44776,2.09051) rectangle (8.48756,2.19636);
\draw [color=c, fill=c] (8.48756,2.09051) rectangle (8.52736,2.19636);
\draw [color=c, fill=c] (8.52736,2.09051) rectangle (8.56716,2.19636);
\draw [color=c, fill=c] (8.56716,2.09051) rectangle (8.60697,2.19636);
\draw [color=c, fill=c] (8.60697,2.09051) rectangle (8.64677,2.19636);
\definecolor{c}{rgb}{1,0.466667,0};
\draw [color=c, fill=c] (8.64677,2.09051) rectangle (8.68657,2.19636);
\draw [color=c, fill=c] (8.68657,2.09051) rectangle (8.72637,2.19636);
\draw [color=c, fill=c] (8.72637,2.09051) rectangle (8.76617,2.19636);
\draw [color=c, fill=c] (8.76617,2.09051) rectangle (8.80597,2.19636);
\draw [color=c, fill=c] (8.80597,2.09051) rectangle (8.84577,2.19636);
\draw [color=c, fill=c] (8.84577,2.09051) rectangle (8.88557,2.19636);
\draw [color=c, fill=c] (8.88557,2.09051) rectangle (8.92537,2.19636);
\draw [color=c, fill=c] (8.92537,2.09051) rectangle (8.96517,2.19636);
\draw [color=c, fill=c] (8.96517,2.09051) rectangle (9.00498,2.19636);
\draw [color=c, fill=c] (9.00498,2.09051) rectangle (9.04478,2.19636);
\draw [color=c, fill=c] (9.04478,2.09051) rectangle (9.08458,2.19636);
\definecolor{c}{rgb}{1,0.653333,0};
\draw [color=c, fill=c] (9.08458,2.09051) rectangle (9.12438,2.19636);
\draw [color=c, fill=c] (9.12438,2.09051) rectangle (9.16418,2.19636);
\draw [color=c, fill=c] (9.16418,2.09051) rectangle (9.20398,2.19636);
\draw [color=c, fill=c] (9.20398,2.09051) rectangle (9.24378,2.19636);
\draw [color=c, fill=c] (9.24378,2.09051) rectangle (9.28358,2.19636);
\draw [color=c, fill=c] (9.28358,2.09051) rectangle (9.32338,2.19636);
\draw [color=c, fill=c] (9.32338,2.09051) rectangle (9.36318,2.19636);
\definecolor{c}{rgb}{1,0.933333,0};
\draw [color=c, fill=c] (9.36318,2.09051) rectangle (9.40298,2.19636);
\draw [color=c, fill=c] (9.40298,2.09051) rectangle (9.44279,2.19636);
\draw [color=c, fill=c] (9.44279,2.09051) rectangle (9.48259,2.19636);
\draw [color=c, fill=c] (9.48259,2.09051) rectangle (9.52239,2.19636);
\draw [color=c, fill=c] (9.52239,2.09051) rectangle (9.56219,2.19636);
\definecolor{c}{rgb}{0.88,1,0};
\draw [color=c, fill=c] (9.56219,2.09051) rectangle (9.60199,2.19636);
\draw [color=c, fill=c] (9.60199,2.09051) rectangle (9.64179,2.19636);
\draw [color=c, fill=c] (9.64179,2.09051) rectangle (9.68159,2.19636);
\draw [color=c, fill=c] (9.68159,2.09051) rectangle (9.72139,2.19636);
\draw [color=c, fill=c] (9.72139,2.09051) rectangle (9.76119,2.19636);
\definecolor{c}{rgb}{0.6,1,0};
\draw [color=c, fill=c] (9.76119,2.09051) rectangle (9.80099,2.19636);
\draw [color=c, fill=c] (9.80099,2.09051) rectangle (9.8408,2.19636);
\draw [color=c, fill=c] (9.8408,2.09051) rectangle (9.8806,2.19636);
\definecolor{c}{rgb}{0.413333,1,0};
\draw [color=c, fill=c] (9.8806,2.09051) rectangle (9.9204,2.19636);
\draw [color=c, fill=c] (9.9204,2.09051) rectangle (9.9602,2.19636);
\draw [color=c, fill=c] (9.9602,2.09051) rectangle (10,2.19636);
\draw [color=c, fill=c] (10,2.09051) rectangle (10.0398,2.19636);
\definecolor{c}{rgb}{0.133333,1,0};
\draw [color=c, fill=c] (10.0398,2.09051) rectangle (10.0796,2.19636);
\draw [color=c, fill=c] (10.0796,2.09051) rectangle (10.1194,2.19636);
\draw [color=c, fill=c] (10.1194,2.09051) rectangle (10.1592,2.19636);
\draw [color=c, fill=c] (10.1592,2.09051) rectangle (10.199,2.19636);
\definecolor{c}{rgb}{0,1,0.0533333};
\draw [color=c, fill=c] (10.199,2.09051) rectangle (10.2388,2.19636);
\draw [color=c, fill=c] (10.2388,2.09051) rectangle (10.2786,2.19636);
\draw [color=c, fill=c] (10.2786,2.09051) rectangle (10.3184,2.19636);
\draw [color=c, fill=c] (10.3184,2.09051) rectangle (10.3582,2.19636);
\draw [color=c, fill=c] (10.3582,2.09051) rectangle (10.398,2.19636);
\definecolor{c}{rgb}{0,1,0.333333};
\draw [color=c, fill=c] (10.398,2.09051) rectangle (10.4378,2.19636);
\draw [color=c, fill=c] (10.4378,2.09051) rectangle (10.4776,2.19636);
\draw [color=c, fill=c] (10.4776,2.09051) rectangle (10.5174,2.19636);
\draw [color=c, fill=c] (10.5174,2.09051) rectangle (10.5572,2.19636);
\draw [color=c, fill=c] (10.5572,2.09051) rectangle (10.597,2.19636);
\definecolor{c}{rgb}{0,1,0.52};
\draw [color=c, fill=c] (10.597,2.09051) rectangle (10.6368,2.19636);
\draw [color=c, fill=c] (10.6368,2.09051) rectangle (10.6766,2.19636);
\draw [color=c, fill=c] (10.6766,2.09051) rectangle (10.7164,2.19636);
\draw [color=c, fill=c] (10.7164,2.09051) rectangle (10.7562,2.19636);
\draw [color=c, fill=c] (10.7562,2.09051) rectangle (10.796,2.19636);
\draw [color=c, fill=c] (10.796,2.09051) rectangle (10.8358,2.19636);
\draw [color=c, fill=c] (10.8358,2.09051) rectangle (10.8756,2.19636);
\draw [color=c, fill=c] (10.8756,2.09051) rectangle (10.9154,2.19636);
\definecolor{c}{rgb}{0,1,0.8};
\draw [color=c, fill=c] (10.9154,2.09051) rectangle (10.9552,2.19636);
\draw [color=c, fill=c] (10.9552,2.09051) rectangle (10.995,2.19636);
\draw [color=c, fill=c] (10.995,2.09051) rectangle (11.0348,2.19636);
\draw [color=c, fill=c] (11.0348,2.09051) rectangle (11.0746,2.19636);
\draw [color=c, fill=c] (11.0746,2.09051) rectangle (11.1144,2.19636);
\draw [color=c, fill=c] (11.1144,2.09051) rectangle (11.1542,2.19636);
\draw [color=c, fill=c] (11.1542,2.09051) rectangle (11.194,2.19636);
\draw [color=c, fill=c] (11.194,2.09051) rectangle (11.2338,2.19636);
\draw [color=c, fill=c] (11.2338,2.09051) rectangle (11.2736,2.19636);
\draw [color=c, fill=c] (11.2736,2.09051) rectangle (11.3134,2.19636);
\draw [color=c, fill=c] (11.3134,2.09051) rectangle (11.3532,2.19636);
\draw [color=c, fill=c] (11.3532,2.09051) rectangle (11.393,2.19636);
\definecolor{c}{rgb}{0,1,0.986667};
\draw [color=c, fill=c] (11.393,2.09051) rectangle (11.4328,2.19636);
\draw [color=c, fill=c] (11.4328,2.09051) rectangle (11.4726,2.19636);
\draw [color=c, fill=c] (11.4726,2.09051) rectangle (11.5124,2.19636);
\draw [color=c, fill=c] (11.5124,2.09051) rectangle (11.5522,2.19636);
\draw [color=c, fill=c] (11.5522,2.09051) rectangle (11.592,2.19636);
\draw [color=c, fill=c] (11.592,2.09051) rectangle (11.6318,2.19636);
\draw [color=c, fill=c] (11.6318,2.09051) rectangle (11.6716,2.19636);
\draw [color=c, fill=c] (11.6716,2.09051) rectangle (11.7114,2.19636);
\draw [color=c, fill=c] (11.7114,2.09051) rectangle (11.7512,2.19636);
\draw [color=c, fill=c] (11.7512,2.09051) rectangle (11.791,2.19636);
\draw [color=c, fill=c] (11.791,2.09051) rectangle (11.8308,2.19636);
\draw [color=c, fill=c] (11.8308,2.09051) rectangle (11.8706,2.19636);
\draw [color=c, fill=c] (11.8706,2.09051) rectangle (11.9104,2.19636);
\draw [color=c, fill=c] (11.9104,2.09051) rectangle (11.9502,2.19636);
\draw [color=c, fill=c] (11.9502,2.09051) rectangle (11.99,2.19636);
\draw [color=c, fill=c] (11.99,2.09051) rectangle (12.0299,2.19636);
\draw [color=c, fill=c] (12.0299,2.09051) rectangle (12.0697,2.19636);
\draw [color=c, fill=c] (12.0697,2.09051) rectangle (12.1095,2.19636);
\draw [color=c, fill=c] (12.1095,2.09051) rectangle (12.1493,2.19636);
\draw [color=c, fill=c] (12.1493,2.09051) rectangle (12.1891,2.19636);
\draw [color=c, fill=c] (12.1891,2.09051) rectangle (12.2289,2.19636);
\draw [color=c, fill=c] (12.2289,2.09051) rectangle (12.2687,2.19636);
\draw [color=c, fill=c] (12.2687,2.09051) rectangle (12.3085,2.19636);
\definecolor{c}{rgb}{0,0.733333,1};
\draw [color=c, fill=c] (12.3085,2.09051) rectangle (12.3483,2.19636);
\draw [color=c, fill=c] (12.3483,2.09051) rectangle (12.3881,2.19636);
\draw [color=c, fill=c] (12.3881,2.09051) rectangle (12.4279,2.19636);
\draw [color=c, fill=c] (12.4279,2.09051) rectangle (12.4677,2.19636);
\draw [color=c, fill=c] (12.4677,2.09051) rectangle (12.5075,2.19636);
\draw [color=c, fill=c] (12.5075,2.09051) rectangle (12.5473,2.19636);
\draw [color=c, fill=c] (12.5473,2.09051) rectangle (12.5871,2.19636);
\draw [color=c, fill=c] (12.5871,2.09051) rectangle (12.6269,2.19636);
\draw [color=c, fill=c] (12.6269,2.09051) rectangle (12.6667,2.19636);
\draw [color=c, fill=c] (12.6667,2.09051) rectangle (12.7065,2.19636);
\draw [color=c, fill=c] (12.7065,2.09051) rectangle (12.7463,2.19636);
\draw [color=c, fill=c] (12.7463,2.09051) rectangle (12.7861,2.19636);
\draw [color=c, fill=c] (12.7861,2.09051) rectangle (12.8259,2.19636);
\draw [color=c, fill=c] (12.8259,2.09051) rectangle (12.8657,2.19636);
\draw [color=c, fill=c] (12.8657,2.09051) rectangle (12.9055,2.19636);
\draw [color=c, fill=c] (12.9055,2.09051) rectangle (12.9453,2.19636);
\draw [color=c, fill=c] (12.9453,2.09051) rectangle (12.9851,2.19636);
\draw [color=c, fill=c] (12.9851,2.09051) rectangle (13.0249,2.19636);
\draw [color=c, fill=c] (13.0249,2.09051) rectangle (13.0647,2.19636);
\draw [color=c, fill=c] (13.0647,2.09051) rectangle (13.1045,2.19636);
\draw [color=c, fill=c] (13.1045,2.09051) rectangle (13.1443,2.19636);
\draw [color=c, fill=c] (13.1443,2.09051) rectangle (13.1841,2.19636);
\draw [color=c, fill=c] (13.1841,2.09051) rectangle (13.2239,2.19636);
\draw [color=c, fill=c] (13.2239,2.09051) rectangle (13.2637,2.19636);
\draw [color=c, fill=c] (13.2637,2.09051) rectangle (13.3035,2.19636);
\draw [color=c, fill=c] (13.3035,2.09051) rectangle (13.3433,2.19636);
\draw [color=c, fill=c] (13.3433,2.09051) rectangle (13.3831,2.19636);
\draw [color=c, fill=c] (13.3831,2.09051) rectangle (13.4229,2.19636);
\draw [color=c, fill=c] (13.4229,2.09051) rectangle (13.4627,2.19636);
\draw [color=c, fill=c] (13.4627,2.09051) rectangle (13.5025,2.19636);
\draw [color=c, fill=c] (13.5025,2.09051) rectangle (13.5423,2.19636);
\draw [color=c, fill=c] (13.5423,2.09051) rectangle (13.5821,2.19636);
\draw [color=c, fill=c] (13.5821,2.09051) rectangle (13.6219,2.19636);
\draw [color=c, fill=c] (13.6219,2.09051) rectangle (13.6617,2.19636);
\draw [color=c, fill=c] (13.6617,2.09051) rectangle (13.7015,2.19636);
\draw [color=c, fill=c] (13.7015,2.09051) rectangle (13.7413,2.19636);
\draw [color=c, fill=c] (13.7413,2.09051) rectangle (13.7811,2.19636);
\draw [color=c, fill=c] (13.7811,2.09051) rectangle (13.8209,2.19636);
\draw [color=c, fill=c] (13.8209,2.09051) rectangle (13.8607,2.19636);
\draw [color=c, fill=c] (13.8607,2.09051) rectangle (13.9005,2.19636);
\draw [color=c, fill=c] (13.9005,2.09051) rectangle (13.9403,2.19636);
\draw [color=c, fill=c] (13.9403,2.09051) rectangle (13.9801,2.19636);
\draw [color=c, fill=c] (13.9801,2.09051) rectangle (14.0199,2.19636);
\draw [color=c, fill=c] (14.0199,2.09051) rectangle (14.0597,2.19636);
\draw [color=c, fill=c] (14.0597,2.09051) rectangle (14.0995,2.19636);
\draw [color=c, fill=c] (14.0995,2.09051) rectangle (14.1393,2.19636);
\draw [color=c, fill=c] (14.1393,2.09051) rectangle (14.1791,2.19636);
\draw [color=c, fill=c] (14.1791,2.09051) rectangle (14.2189,2.19636);
\draw [color=c, fill=c] (14.2189,2.09051) rectangle (14.2587,2.19636);
\draw [color=c, fill=c] (14.2587,2.09051) rectangle (14.2985,2.19636);
\draw [color=c, fill=c] (14.2985,2.09051) rectangle (14.3383,2.19636);
\draw [color=c, fill=c] (14.3383,2.09051) rectangle (14.3781,2.19636);
\draw [color=c, fill=c] (14.3781,2.09051) rectangle (14.4179,2.19636);
\draw [color=c, fill=c] (14.4179,2.09051) rectangle (14.4577,2.19636);
\draw [color=c, fill=c] (14.4577,2.09051) rectangle (14.4975,2.19636);
\draw [color=c, fill=c] (14.4975,2.09051) rectangle (14.5373,2.19636);
\draw [color=c, fill=c] (14.5373,2.09051) rectangle (14.5771,2.19636);
\draw [color=c, fill=c] (14.5771,2.09051) rectangle (14.6169,2.19636);
\draw [color=c, fill=c] (14.6169,2.09051) rectangle (14.6567,2.19636);
\draw [color=c, fill=c] (14.6567,2.09051) rectangle (14.6965,2.19636);
\draw [color=c, fill=c] (14.6965,2.09051) rectangle (14.7363,2.19636);
\draw [color=c, fill=c] (14.7363,2.09051) rectangle (14.7761,2.19636);
\draw [color=c, fill=c] (14.7761,2.09051) rectangle (14.8159,2.19636);
\draw [color=c, fill=c] (14.8159,2.09051) rectangle (14.8557,2.19636);
\draw [color=c, fill=c] (14.8557,2.09051) rectangle (14.8955,2.19636);
\draw [color=c, fill=c] (14.8955,2.09051) rectangle (14.9353,2.19636);
\draw [color=c, fill=c] (14.9353,2.09051) rectangle (14.9751,2.19636);
\draw [color=c, fill=c] (14.9751,2.09051) rectangle (15.0149,2.19636);
\draw [color=c, fill=c] (15.0149,2.09051) rectangle (15.0547,2.19636);
\draw [color=c, fill=c] (15.0547,2.09051) rectangle (15.0945,2.19636);
\draw [color=c, fill=c] (15.0945,2.09051) rectangle (15.1343,2.19636);
\draw [color=c, fill=c] (15.1343,2.09051) rectangle (15.1741,2.19636);
\draw [color=c, fill=c] (15.1741,2.09051) rectangle (15.2139,2.19636);
\draw [color=c, fill=c] (15.2139,2.09051) rectangle (15.2537,2.19636);
\draw [color=c, fill=c] (15.2537,2.09051) rectangle (15.2935,2.19636);
\draw [color=c, fill=c] (15.2935,2.09051) rectangle (15.3333,2.19636);
\draw [color=c, fill=c] (15.3333,2.09051) rectangle (15.3731,2.19636);
\draw [color=c, fill=c] (15.3731,2.09051) rectangle (15.4129,2.19636);
\draw [color=c, fill=c] (15.4129,2.09051) rectangle (15.4527,2.19636);
\draw [color=c, fill=c] (15.4527,2.09051) rectangle (15.4925,2.19636);
\draw [color=c, fill=c] (15.4925,2.09051) rectangle (15.5323,2.19636);
\draw [color=c, fill=c] (15.5323,2.09051) rectangle (15.5721,2.19636);
\draw [color=c, fill=c] (15.5721,2.09051) rectangle (15.6119,2.19636);
\draw [color=c, fill=c] (15.6119,2.09051) rectangle (15.6517,2.19636);
\draw [color=c, fill=c] (15.6517,2.09051) rectangle (15.6915,2.19636);
\draw [color=c, fill=c] (15.6915,2.09051) rectangle (15.7313,2.19636);
\draw [color=c, fill=c] (15.7313,2.09051) rectangle (15.7711,2.19636);
\draw [color=c, fill=c] (15.7711,2.09051) rectangle (15.8109,2.19636);
\draw [color=c, fill=c] (15.8109,2.09051) rectangle (15.8507,2.19636);
\draw [color=c, fill=c] (15.8507,2.09051) rectangle (15.8905,2.19636);
\draw [color=c, fill=c] (15.8905,2.09051) rectangle (15.9303,2.19636);
\draw [color=c, fill=c] (15.9303,2.09051) rectangle (15.9701,2.19636);
\draw [color=c, fill=c] (15.9701,2.09051) rectangle (16.01,2.19636);
\draw [color=c, fill=c] (16.01,2.09051) rectangle (16.0498,2.19636);
\draw [color=c, fill=c] (16.0498,2.09051) rectangle (16.0896,2.19636);
\draw [color=c, fill=c] (16.0896,2.09051) rectangle (16.1294,2.19636);
\draw [color=c, fill=c] (16.1294,2.09051) rectangle (16.1692,2.19636);
\draw [color=c, fill=c] (16.1692,2.09051) rectangle (16.209,2.19636);
\draw [color=c, fill=c] (16.209,2.09051) rectangle (16.2488,2.19636);
\draw [color=c, fill=c] (16.2488,2.09051) rectangle (16.2886,2.19636);
\draw [color=c, fill=c] (16.2886,2.09051) rectangle (16.3284,2.19636);
\draw [color=c, fill=c] (16.3284,2.09051) rectangle (16.3682,2.19636);
\draw [color=c, fill=c] (16.3682,2.09051) rectangle (16.408,2.19636);
\draw [color=c, fill=c] (16.408,2.09051) rectangle (16.4478,2.19636);
\draw [color=c, fill=c] (16.4478,2.09051) rectangle (16.4876,2.19636);
\draw [color=c, fill=c] (16.4876,2.09051) rectangle (16.5274,2.19636);
\draw [color=c, fill=c] (16.5274,2.09051) rectangle (16.5672,2.19636);
\draw [color=c, fill=c] (16.5672,2.09051) rectangle (16.607,2.19636);
\draw [color=c, fill=c] (16.607,2.09051) rectangle (16.6468,2.19636);
\draw [color=c, fill=c] (16.6468,2.09051) rectangle (16.6866,2.19636);
\draw [color=c, fill=c] (16.6866,2.09051) rectangle (16.7264,2.19636);
\draw [color=c, fill=c] (16.7264,2.09051) rectangle (16.7662,2.19636);
\draw [color=c, fill=c] (16.7662,2.09051) rectangle (16.806,2.19636);
\draw [color=c, fill=c] (16.806,2.09051) rectangle (16.8458,2.19636);
\draw [color=c, fill=c] (16.8458,2.09051) rectangle (16.8856,2.19636);
\draw [color=c, fill=c] (16.8856,2.09051) rectangle (16.9254,2.19636);
\draw [color=c, fill=c] (16.9254,2.09051) rectangle (16.9652,2.19636);
\draw [color=c, fill=c] (16.9652,2.09051) rectangle (17.005,2.19636);
\draw [color=c, fill=c] (17.005,2.09051) rectangle (17.0448,2.19636);
\draw [color=c, fill=c] (17.0448,2.09051) rectangle (17.0846,2.19636);
\draw [color=c, fill=c] (17.0846,2.09051) rectangle (17.1244,2.19636);
\draw [color=c, fill=c] (17.1244,2.09051) rectangle (17.1642,2.19636);
\draw [color=c, fill=c] (17.1642,2.09051) rectangle (17.204,2.19636);
\draw [color=c, fill=c] (17.204,2.09051) rectangle (17.2438,2.19636);
\draw [color=c, fill=c] (17.2438,2.09051) rectangle (17.2836,2.19636);
\draw [color=c, fill=c] (17.2836,2.09051) rectangle (17.3234,2.19636);
\draw [color=c, fill=c] (17.3234,2.09051) rectangle (17.3632,2.19636);
\draw [color=c, fill=c] (17.3632,2.09051) rectangle (17.403,2.19636);
\draw [color=c, fill=c] (17.403,2.09051) rectangle (17.4428,2.19636);
\draw [color=c, fill=c] (17.4428,2.09051) rectangle (17.4826,2.19636);
\draw [color=c, fill=c] (17.4826,2.09051) rectangle (17.5224,2.19636);
\draw [color=c, fill=c] (17.5224,2.09051) rectangle (17.5622,2.19636);
\draw [color=c, fill=c] (17.5622,2.09051) rectangle (17.602,2.19636);
\draw [color=c, fill=c] (17.602,2.09051) rectangle (17.6418,2.19636);
\draw [color=c, fill=c] (17.6418,2.09051) rectangle (17.6816,2.19636);
\draw [color=c, fill=c] (17.6816,2.09051) rectangle (17.7214,2.19636);
\draw [color=c, fill=c] (17.7214,2.09051) rectangle (17.7612,2.19636);
\draw [color=c, fill=c] (17.7612,2.09051) rectangle (17.801,2.19636);
\draw [color=c, fill=c] (17.801,2.09051) rectangle (17.8408,2.19636);
\draw [color=c, fill=c] (17.8408,2.09051) rectangle (17.8806,2.19636);
\draw [color=c, fill=c] (17.8806,2.09051) rectangle (17.9204,2.19636);
\draw [color=c, fill=c] (17.9204,2.09051) rectangle (17.9602,2.19636);
\draw [color=c, fill=c] (17.9602,2.09051) rectangle (18,2.19636);
\definecolor{c}{rgb}{1,0,0};
\draw [color=c, fill=c] (2,2.19636) rectangle (2.0398,2.30221);
\draw [color=c, fill=c] (2.0398,2.19636) rectangle (2.0796,2.30221);
\draw [color=c, fill=c] (2.0796,2.19636) rectangle (2.1194,2.30221);
\draw [color=c, fill=c] (2.1194,2.19636) rectangle (2.1592,2.30221);
\draw [color=c, fill=c] (2.1592,2.19636) rectangle (2.19901,2.30221);
\draw [color=c, fill=c] (2.19901,2.19636) rectangle (2.23881,2.30221);
\draw [color=c, fill=c] (2.23881,2.19636) rectangle (2.27861,2.30221);
\draw [color=c, fill=c] (2.27861,2.19636) rectangle (2.31841,2.30221);
\draw [color=c, fill=c] (2.31841,2.19636) rectangle (2.35821,2.30221);
\draw [color=c, fill=c] (2.35821,2.19636) rectangle (2.39801,2.30221);
\draw [color=c, fill=c] (2.39801,2.19636) rectangle (2.43781,2.30221);
\draw [color=c, fill=c] (2.43781,2.19636) rectangle (2.47761,2.30221);
\draw [color=c, fill=c] (2.47761,2.19636) rectangle (2.51741,2.30221);
\draw [color=c, fill=c] (2.51741,2.19636) rectangle (2.55721,2.30221);
\draw [color=c, fill=c] (2.55721,2.19636) rectangle (2.59702,2.30221);
\draw [color=c, fill=c] (2.59702,2.19636) rectangle (2.63682,2.30221);
\draw [color=c, fill=c] (2.63682,2.19636) rectangle (2.67662,2.30221);
\draw [color=c, fill=c] (2.67662,2.19636) rectangle (2.71642,2.30221);
\draw [color=c, fill=c] (2.71642,2.19636) rectangle (2.75622,2.30221);
\draw [color=c, fill=c] (2.75622,2.19636) rectangle (2.79602,2.30221);
\draw [color=c, fill=c] (2.79602,2.19636) rectangle (2.83582,2.30221);
\draw [color=c, fill=c] (2.83582,2.19636) rectangle (2.87562,2.30221);
\draw [color=c, fill=c] (2.87562,2.19636) rectangle (2.91542,2.30221);
\draw [color=c, fill=c] (2.91542,2.19636) rectangle (2.95522,2.30221);
\draw [color=c, fill=c] (2.95522,2.19636) rectangle (2.99502,2.30221);
\draw [color=c, fill=c] (2.99502,2.19636) rectangle (3.03483,2.30221);
\draw [color=c, fill=c] (3.03483,2.19636) rectangle (3.07463,2.30221);
\draw [color=c, fill=c] (3.07463,2.19636) rectangle (3.11443,2.30221);
\draw [color=c, fill=c] (3.11443,2.19636) rectangle (3.15423,2.30221);
\draw [color=c, fill=c] (3.15423,2.19636) rectangle (3.19403,2.30221);
\draw [color=c, fill=c] (3.19403,2.19636) rectangle (3.23383,2.30221);
\draw [color=c, fill=c] (3.23383,2.19636) rectangle (3.27363,2.30221);
\draw [color=c, fill=c] (3.27363,2.19636) rectangle (3.31343,2.30221);
\draw [color=c, fill=c] (3.31343,2.19636) rectangle (3.35323,2.30221);
\draw [color=c, fill=c] (3.35323,2.19636) rectangle (3.39303,2.30221);
\draw [color=c, fill=c] (3.39303,2.19636) rectangle (3.43284,2.30221);
\draw [color=c, fill=c] (3.43284,2.19636) rectangle (3.47264,2.30221);
\draw [color=c, fill=c] (3.47264,2.19636) rectangle (3.51244,2.30221);
\draw [color=c, fill=c] (3.51244,2.19636) rectangle (3.55224,2.30221);
\draw [color=c, fill=c] (3.55224,2.19636) rectangle (3.59204,2.30221);
\draw [color=c, fill=c] (3.59204,2.19636) rectangle (3.63184,2.30221);
\draw [color=c, fill=c] (3.63184,2.19636) rectangle (3.67164,2.30221);
\draw [color=c, fill=c] (3.67164,2.19636) rectangle (3.71144,2.30221);
\draw [color=c, fill=c] (3.71144,2.19636) rectangle (3.75124,2.30221);
\draw [color=c, fill=c] (3.75124,2.19636) rectangle (3.79104,2.30221);
\draw [color=c, fill=c] (3.79104,2.19636) rectangle (3.83085,2.30221);
\draw [color=c, fill=c] (3.83085,2.19636) rectangle (3.87065,2.30221);
\draw [color=c, fill=c] (3.87065,2.19636) rectangle (3.91045,2.30221);
\draw [color=c, fill=c] (3.91045,2.19636) rectangle (3.95025,2.30221);
\draw [color=c, fill=c] (3.95025,2.19636) rectangle (3.99005,2.30221);
\draw [color=c, fill=c] (3.99005,2.19636) rectangle (4.02985,2.30221);
\draw [color=c, fill=c] (4.02985,2.19636) rectangle (4.06965,2.30221);
\draw [color=c, fill=c] (4.06965,2.19636) rectangle (4.10945,2.30221);
\draw [color=c, fill=c] (4.10945,2.19636) rectangle (4.14925,2.30221);
\draw [color=c, fill=c] (4.14925,2.19636) rectangle (4.18905,2.30221);
\draw [color=c, fill=c] (4.18905,2.19636) rectangle (4.22886,2.30221);
\draw [color=c, fill=c] (4.22886,2.19636) rectangle (4.26866,2.30221);
\draw [color=c, fill=c] (4.26866,2.19636) rectangle (4.30846,2.30221);
\draw [color=c, fill=c] (4.30846,2.19636) rectangle (4.34826,2.30221);
\draw [color=c, fill=c] (4.34826,2.19636) rectangle (4.38806,2.30221);
\draw [color=c, fill=c] (4.38806,2.19636) rectangle (4.42786,2.30221);
\draw [color=c, fill=c] (4.42786,2.19636) rectangle (4.46766,2.30221);
\draw [color=c, fill=c] (4.46766,2.19636) rectangle (4.50746,2.30221);
\draw [color=c, fill=c] (4.50746,2.19636) rectangle (4.54726,2.30221);
\draw [color=c, fill=c] (4.54726,2.19636) rectangle (4.58706,2.30221);
\draw [color=c, fill=c] (4.58706,2.19636) rectangle (4.62687,2.30221);
\draw [color=c, fill=c] (4.62687,2.19636) rectangle (4.66667,2.30221);
\draw [color=c, fill=c] (4.66667,2.19636) rectangle (4.70647,2.30221);
\draw [color=c, fill=c] (4.70647,2.19636) rectangle (4.74627,2.30221);
\draw [color=c, fill=c] (4.74627,2.19636) rectangle (4.78607,2.30221);
\draw [color=c, fill=c] (4.78607,2.19636) rectangle (4.82587,2.30221);
\draw [color=c, fill=c] (4.82587,2.19636) rectangle (4.86567,2.30221);
\draw [color=c, fill=c] (4.86567,2.19636) rectangle (4.90547,2.30221);
\draw [color=c, fill=c] (4.90547,2.19636) rectangle (4.94527,2.30221);
\draw [color=c, fill=c] (4.94527,2.19636) rectangle (4.98507,2.30221);
\draw [color=c, fill=c] (4.98507,2.19636) rectangle (5.02488,2.30221);
\draw [color=c, fill=c] (5.02488,2.19636) rectangle (5.06468,2.30221);
\draw [color=c, fill=c] (5.06468,2.19636) rectangle (5.10448,2.30221);
\draw [color=c, fill=c] (5.10448,2.19636) rectangle (5.14428,2.30221);
\draw [color=c, fill=c] (5.14428,2.19636) rectangle (5.18408,2.30221);
\draw [color=c, fill=c] (5.18408,2.19636) rectangle (5.22388,2.30221);
\draw [color=c, fill=c] (5.22388,2.19636) rectangle (5.26368,2.30221);
\draw [color=c, fill=c] (5.26368,2.19636) rectangle (5.30348,2.30221);
\draw [color=c, fill=c] (5.30348,2.19636) rectangle (5.34328,2.30221);
\draw [color=c, fill=c] (5.34328,2.19636) rectangle (5.38308,2.30221);
\draw [color=c, fill=c] (5.38308,2.19636) rectangle (5.42289,2.30221);
\draw [color=c, fill=c] (5.42289,2.19636) rectangle (5.46269,2.30221);
\draw [color=c, fill=c] (5.46269,2.19636) rectangle (5.50249,2.30221);
\draw [color=c, fill=c] (5.50249,2.19636) rectangle (5.54229,2.30221);
\draw [color=c, fill=c] (5.54229,2.19636) rectangle (5.58209,2.30221);
\draw [color=c, fill=c] (5.58209,2.19636) rectangle (5.62189,2.30221);
\draw [color=c, fill=c] (5.62189,2.19636) rectangle (5.66169,2.30221);
\draw [color=c, fill=c] (5.66169,2.19636) rectangle (5.70149,2.30221);
\draw [color=c, fill=c] (5.70149,2.19636) rectangle (5.74129,2.30221);
\draw [color=c, fill=c] (5.74129,2.19636) rectangle (5.78109,2.30221);
\draw [color=c, fill=c] (5.78109,2.19636) rectangle (5.8209,2.30221);
\draw [color=c, fill=c] (5.8209,2.19636) rectangle (5.8607,2.30221);
\draw [color=c, fill=c] (5.8607,2.19636) rectangle (5.9005,2.30221);
\draw [color=c, fill=c] (5.9005,2.19636) rectangle (5.9403,2.30221);
\draw [color=c, fill=c] (5.9403,2.19636) rectangle (5.9801,2.30221);
\draw [color=c, fill=c] (5.9801,2.19636) rectangle (6.0199,2.30221);
\draw [color=c, fill=c] (6.0199,2.19636) rectangle (6.0597,2.30221);
\draw [color=c, fill=c] (6.0597,2.19636) rectangle (6.0995,2.30221);
\draw [color=c, fill=c] (6.0995,2.19636) rectangle (6.1393,2.30221);
\draw [color=c, fill=c] (6.1393,2.19636) rectangle (6.1791,2.30221);
\draw [color=c, fill=c] (6.1791,2.19636) rectangle (6.21891,2.30221);
\draw [color=c, fill=c] (6.21891,2.19636) rectangle (6.25871,2.30221);
\draw [color=c, fill=c] (6.25871,2.19636) rectangle (6.29851,2.30221);
\draw [color=c, fill=c] (6.29851,2.19636) rectangle (6.33831,2.30221);
\draw [color=c, fill=c] (6.33831,2.19636) rectangle (6.37811,2.30221);
\draw [color=c, fill=c] (6.37811,2.19636) rectangle (6.41791,2.30221);
\draw [color=c, fill=c] (6.41791,2.19636) rectangle (6.45771,2.30221);
\draw [color=c, fill=c] (6.45771,2.19636) rectangle (6.49751,2.30221);
\draw [color=c, fill=c] (6.49751,2.19636) rectangle (6.53731,2.30221);
\draw [color=c, fill=c] (6.53731,2.19636) rectangle (6.57711,2.30221);
\draw [color=c, fill=c] (6.57711,2.19636) rectangle (6.61692,2.30221);
\draw [color=c, fill=c] (6.61692,2.19636) rectangle (6.65672,2.30221);
\draw [color=c, fill=c] (6.65672,2.19636) rectangle (6.69652,2.30221);
\draw [color=c, fill=c] (6.69652,2.19636) rectangle (6.73632,2.30221);
\draw [color=c, fill=c] (6.73632,2.19636) rectangle (6.77612,2.30221);
\draw [color=c, fill=c] (6.77612,2.19636) rectangle (6.81592,2.30221);
\draw [color=c, fill=c] (6.81592,2.19636) rectangle (6.85572,2.30221);
\draw [color=c, fill=c] (6.85572,2.19636) rectangle (6.89552,2.30221);
\draw [color=c, fill=c] (6.89552,2.19636) rectangle (6.93532,2.30221);
\draw [color=c, fill=c] (6.93532,2.19636) rectangle (6.97512,2.30221);
\draw [color=c, fill=c] (6.97512,2.19636) rectangle (7.01493,2.30221);
\draw [color=c, fill=c] (7.01493,2.19636) rectangle (7.05473,2.30221);
\draw [color=c, fill=c] (7.05473,2.19636) rectangle (7.09453,2.30221);
\draw [color=c, fill=c] (7.09453,2.19636) rectangle (7.13433,2.30221);
\draw [color=c, fill=c] (7.13433,2.19636) rectangle (7.17413,2.30221);
\draw [color=c, fill=c] (7.17413,2.19636) rectangle (7.21393,2.30221);
\draw [color=c, fill=c] (7.21393,2.19636) rectangle (7.25373,2.30221);
\draw [color=c, fill=c] (7.25373,2.19636) rectangle (7.29353,2.30221);
\draw [color=c, fill=c] (7.29353,2.19636) rectangle (7.33333,2.30221);
\draw [color=c, fill=c] (7.33333,2.19636) rectangle (7.37313,2.30221);
\draw [color=c, fill=c] (7.37313,2.19636) rectangle (7.41294,2.30221);
\draw [color=c, fill=c] (7.41294,2.19636) rectangle (7.45274,2.30221);
\draw [color=c, fill=c] (7.45274,2.19636) rectangle (7.49254,2.30221);
\draw [color=c, fill=c] (7.49254,2.19636) rectangle (7.53234,2.30221);
\draw [color=c, fill=c] (7.53234,2.19636) rectangle (7.57214,2.30221);
\draw [color=c, fill=c] (7.57214,2.19636) rectangle (7.61194,2.30221);
\draw [color=c, fill=c] (7.61194,2.19636) rectangle (7.65174,2.30221);
\draw [color=c, fill=c] (7.65174,2.19636) rectangle (7.69154,2.30221);
\draw [color=c, fill=c] (7.69154,2.19636) rectangle (7.73134,2.30221);
\draw [color=c, fill=c] (7.73134,2.19636) rectangle (7.77114,2.30221);
\draw [color=c, fill=c] (7.77114,2.19636) rectangle (7.81095,2.30221);
\definecolor{c}{rgb}{1,0.186667,0};
\draw [color=c, fill=c] (7.81095,2.19636) rectangle (7.85075,2.30221);
\draw [color=c, fill=c] (7.85075,2.19636) rectangle (7.89055,2.30221);
\draw [color=c, fill=c] (7.89055,2.19636) rectangle (7.93035,2.30221);
\draw [color=c, fill=c] (7.93035,2.19636) rectangle (7.97015,2.30221);
\draw [color=c, fill=c] (7.97015,2.19636) rectangle (8.00995,2.30221);
\draw [color=c, fill=c] (8.00995,2.19636) rectangle (8.04975,2.30221);
\draw [color=c, fill=c] (8.04975,2.19636) rectangle (8.08955,2.30221);
\draw [color=c, fill=c] (8.08955,2.19636) rectangle (8.12935,2.30221);
\draw [color=c, fill=c] (8.12935,2.19636) rectangle (8.16915,2.30221);
\draw [color=c, fill=c] (8.16915,2.19636) rectangle (8.20895,2.30221);
\draw [color=c, fill=c] (8.20895,2.19636) rectangle (8.24876,2.30221);
\draw [color=c, fill=c] (8.24876,2.19636) rectangle (8.28856,2.30221);
\draw [color=c, fill=c] (8.28856,2.19636) rectangle (8.32836,2.30221);
\draw [color=c, fill=c] (8.32836,2.19636) rectangle (8.36816,2.30221);
\draw [color=c, fill=c] (8.36816,2.19636) rectangle (8.40796,2.30221);
\draw [color=c, fill=c] (8.40796,2.19636) rectangle (8.44776,2.30221);
\draw [color=c, fill=c] (8.44776,2.19636) rectangle (8.48756,2.30221);
\draw [color=c, fill=c] (8.48756,2.19636) rectangle (8.52736,2.30221);
\draw [color=c, fill=c] (8.52736,2.19636) rectangle (8.56716,2.30221);
\draw [color=c, fill=c] (8.56716,2.19636) rectangle (8.60697,2.30221);
\draw [color=c, fill=c] (8.60697,2.19636) rectangle (8.64677,2.30221);
\definecolor{c}{rgb}{1,0.466667,0};
\draw [color=c, fill=c] (8.64677,2.19636) rectangle (8.68657,2.30221);
\draw [color=c, fill=c] (8.68657,2.19636) rectangle (8.72637,2.30221);
\draw [color=c, fill=c] (8.72637,2.19636) rectangle (8.76617,2.30221);
\draw [color=c, fill=c] (8.76617,2.19636) rectangle (8.80597,2.30221);
\draw [color=c, fill=c] (8.80597,2.19636) rectangle (8.84577,2.30221);
\draw [color=c, fill=c] (8.84577,2.19636) rectangle (8.88557,2.30221);
\draw [color=c, fill=c] (8.88557,2.19636) rectangle (8.92537,2.30221);
\draw [color=c, fill=c] (8.92537,2.19636) rectangle (8.96517,2.30221);
\draw [color=c, fill=c] (8.96517,2.19636) rectangle (9.00498,2.30221);
\draw [color=c, fill=c] (9.00498,2.19636) rectangle (9.04478,2.30221);
\draw [color=c, fill=c] (9.04478,2.19636) rectangle (9.08458,2.30221);
\definecolor{c}{rgb}{1,0.653333,0};
\draw [color=c, fill=c] (9.08458,2.19636) rectangle (9.12438,2.30221);
\draw [color=c, fill=c] (9.12438,2.19636) rectangle (9.16418,2.30221);
\draw [color=c, fill=c] (9.16418,2.19636) rectangle (9.20398,2.30221);
\draw [color=c, fill=c] (9.20398,2.19636) rectangle (9.24378,2.30221);
\draw [color=c, fill=c] (9.24378,2.19636) rectangle (9.28358,2.30221);
\draw [color=c, fill=c] (9.28358,2.19636) rectangle (9.32338,2.30221);
\draw [color=c, fill=c] (9.32338,2.19636) rectangle (9.36318,2.30221);
\definecolor{c}{rgb}{1,0.933333,0};
\draw [color=c, fill=c] (9.36318,2.19636) rectangle (9.40298,2.30221);
\draw [color=c, fill=c] (9.40298,2.19636) rectangle (9.44279,2.30221);
\draw [color=c, fill=c] (9.44279,2.19636) rectangle (9.48259,2.30221);
\draw [color=c, fill=c] (9.48259,2.19636) rectangle (9.52239,2.30221);
\draw [color=c, fill=c] (9.52239,2.19636) rectangle (9.56219,2.30221);
\definecolor{c}{rgb}{0.88,1,0};
\draw [color=c, fill=c] (9.56219,2.19636) rectangle (9.60199,2.30221);
\draw [color=c, fill=c] (9.60199,2.19636) rectangle (9.64179,2.30221);
\draw [color=c, fill=c] (9.64179,2.19636) rectangle (9.68159,2.30221);
\draw [color=c, fill=c] (9.68159,2.19636) rectangle (9.72139,2.30221);
\draw [color=c, fill=c] (9.72139,2.19636) rectangle (9.76119,2.30221);
\definecolor{c}{rgb}{0.6,1,0};
\draw [color=c, fill=c] (9.76119,2.19636) rectangle (9.80099,2.30221);
\draw [color=c, fill=c] (9.80099,2.19636) rectangle (9.8408,2.30221);
\draw [color=c, fill=c] (9.8408,2.19636) rectangle (9.8806,2.30221);
\definecolor{c}{rgb}{0.413333,1,0};
\draw [color=c, fill=c] (9.8806,2.19636) rectangle (9.9204,2.30221);
\draw [color=c, fill=c] (9.9204,2.19636) rectangle (9.9602,2.30221);
\draw [color=c, fill=c] (9.9602,2.19636) rectangle (10,2.30221);
\draw [color=c, fill=c] (10,2.19636) rectangle (10.0398,2.30221);
\definecolor{c}{rgb}{0.133333,1,0};
\draw [color=c, fill=c] (10.0398,2.19636) rectangle (10.0796,2.30221);
\draw [color=c, fill=c] (10.0796,2.19636) rectangle (10.1194,2.30221);
\draw [color=c, fill=c] (10.1194,2.19636) rectangle (10.1592,2.30221);
\draw [color=c, fill=c] (10.1592,2.19636) rectangle (10.199,2.30221);
\definecolor{c}{rgb}{0,1,0.0533333};
\draw [color=c, fill=c] (10.199,2.19636) rectangle (10.2388,2.30221);
\draw [color=c, fill=c] (10.2388,2.19636) rectangle (10.2786,2.30221);
\draw [color=c, fill=c] (10.2786,2.19636) rectangle (10.3184,2.30221);
\draw [color=c, fill=c] (10.3184,2.19636) rectangle (10.3582,2.30221);
\definecolor{c}{rgb}{0,1,0.333333};
\draw [color=c, fill=c] (10.3582,2.19636) rectangle (10.398,2.30221);
\draw [color=c, fill=c] (10.398,2.19636) rectangle (10.4378,2.30221);
\draw [color=c, fill=c] (10.4378,2.19636) rectangle (10.4776,2.30221);
\draw [color=c, fill=c] (10.4776,2.19636) rectangle (10.5174,2.30221);
\draw [color=c, fill=c] (10.5174,2.19636) rectangle (10.5572,2.30221);
\draw [color=c, fill=c] (10.5572,2.19636) rectangle (10.597,2.30221);
\definecolor{c}{rgb}{0,1,0.52};
\draw [color=c, fill=c] (10.597,2.19636) rectangle (10.6368,2.30221);
\draw [color=c, fill=c] (10.6368,2.19636) rectangle (10.6766,2.30221);
\draw [color=c, fill=c] (10.6766,2.19636) rectangle (10.7164,2.30221);
\draw [color=c, fill=c] (10.7164,2.19636) rectangle (10.7562,2.30221);
\draw [color=c, fill=c] (10.7562,2.19636) rectangle (10.796,2.30221);
\draw [color=c, fill=c] (10.796,2.19636) rectangle (10.8358,2.30221);
\draw [color=c, fill=c] (10.8358,2.19636) rectangle (10.8756,2.30221);
\definecolor{c}{rgb}{0,1,0.8};
\draw [color=c, fill=c] (10.8756,2.19636) rectangle (10.9154,2.30221);
\draw [color=c, fill=c] (10.9154,2.19636) rectangle (10.9552,2.30221);
\draw [color=c, fill=c] (10.9552,2.19636) rectangle (10.995,2.30221);
\draw [color=c, fill=c] (10.995,2.19636) rectangle (11.0348,2.30221);
\draw [color=c, fill=c] (11.0348,2.19636) rectangle (11.0746,2.30221);
\draw [color=c, fill=c] (11.0746,2.19636) rectangle (11.1144,2.30221);
\draw [color=c, fill=c] (11.1144,2.19636) rectangle (11.1542,2.30221);
\draw [color=c, fill=c] (11.1542,2.19636) rectangle (11.194,2.30221);
\draw [color=c, fill=c] (11.194,2.19636) rectangle (11.2338,2.30221);
\draw [color=c, fill=c] (11.2338,2.19636) rectangle (11.2736,2.30221);
\draw [color=c, fill=c] (11.2736,2.19636) rectangle (11.3134,2.30221);
\draw [color=c, fill=c] (11.3134,2.19636) rectangle (11.3532,2.30221);
\definecolor{c}{rgb}{0,1,0.986667};
\draw [color=c, fill=c] (11.3532,2.19636) rectangle (11.393,2.30221);
\draw [color=c, fill=c] (11.393,2.19636) rectangle (11.4328,2.30221);
\draw [color=c, fill=c] (11.4328,2.19636) rectangle (11.4726,2.30221);
\draw [color=c, fill=c] (11.4726,2.19636) rectangle (11.5124,2.30221);
\draw [color=c, fill=c] (11.5124,2.19636) rectangle (11.5522,2.30221);
\draw [color=c, fill=c] (11.5522,2.19636) rectangle (11.592,2.30221);
\draw [color=c, fill=c] (11.592,2.19636) rectangle (11.6318,2.30221);
\draw [color=c, fill=c] (11.6318,2.19636) rectangle (11.6716,2.30221);
\draw [color=c, fill=c] (11.6716,2.19636) rectangle (11.7114,2.30221);
\draw [color=c, fill=c] (11.7114,2.19636) rectangle (11.7512,2.30221);
\draw [color=c, fill=c] (11.7512,2.19636) rectangle (11.791,2.30221);
\draw [color=c, fill=c] (11.791,2.19636) rectangle (11.8308,2.30221);
\draw [color=c, fill=c] (11.8308,2.19636) rectangle (11.8706,2.30221);
\draw [color=c, fill=c] (11.8706,2.19636) rectangle (11.9104,2.30221);
\draw [color=c, fill=c] (11.9104,2.19636) rectangle (11.9502,2.30221);
\draw [color=c, fill=c] (11.9502,2.19636) rectangle (11.99,2.30221);
\draw [color=c, fill=c] (11.99,2.19636) rectangle (12.0299,2.30221);
\draw [color=c, fill=c] (12.0299,2.19636) rectangle (12.0697,2.30221);
\draw [color=c, fill=c] (12.0697,2.19636) rectangle (12.1095,2.30221);
\draw [color=c, fill=c] (12.1095,2.19636) rectangle (12.1493,2.30221);
\draw [color=c, fill=c] (12.1493,2.19636) rectangle (12.1891,2.30221);
\draw [color=c, fill=c] (12.1891,2.19636) rectangle (12.2289,2.30221);
\draw [color=c, fill=c] (12.2289,2.19636) rectangle (12.2687,2.30221);
\draw [color=c, fill=c] (12.2687,2.19636) rectangle (12.3085,2.30221);
\definecolor{c}{rgb}{0,0.733333,1};
\draw [color=c, fill=c] (12.3085,2.19636) rectangle (12.3483,2.30221);
\draw [color=c, fill=c] (12.3483,2.19636) rectangle (12.3881,2.30221);
\draw [color=c, fill=c] (12.3881,2.19636) rectangle (12.4279,2.30221);
\draw [color=c, fill=c] (12.4279,2.19636) rectangle (12.4677,2.30221);
\draw [color=c, fill=c] (12.4677,2.19636) rectangle (12.5075,2.30221);
\draw [color=c, fill=c] (12.5075,2.19636) rectangle (12.5473,2.30221);
\draw [color=c, fill=c] (12.5473,2.19636) rectangle (12.5871,2.30221);
\draw [color=c, fill=c] (12.5871,2.19636) rectangle (12.6269,2.30221);
\draw [color=c, fill=c] (12.6269,2.19636) rectangle (12.6667,2.30221);
\draw [color=c, fill=c] (12.6667,2.19636) rectangle (12.7065,2.30221);
\draw [color=c, fill=c] (12.7065,2.19636) rectangle (12.7463,2.30221);
\draw [color=c, fill=c] (12.7463,2.19636) rectangle (12.7861,2.30221);
\draw [color=c, fill=c] (12.7861,2.19636) rectangle (12.8259,2.30221);
\draw [color=c, fill=c] (12.8259,2.19636) rectangle (12.8657,2.30221);
\draw [color=c, fill=c] (12.8657,2.19636) rectangle (12.9055,2.30221);
\draw [color=c, fill=c] (12.9055,2.19636) rectangle (12.9453,2.30221);
\draw [color=c, fill=c] (12.9453,2.19636) rectangle (12.9851,2.30221);
\draw [color=c, fill=c] (12.9851,2.19636) rectangle (13.0249,2.30221);
\draw [color=c, fill=c] (13.0249,2.19636) rectangle (13.0647,2.30221);
\draw [color=c, fill=c] (13.0647,2.19636) rectangle (13.1045,2.30221);
\draw [color=c, fill=c] (13.1045,2.19636) rectangle (13.1443,2.30221);
\draw [color=c, fill=c] (13.1443,2.19636) rectangle (13.1841,2.30221);
\draw [color=c, fill=c] (13.1841,2.19636) rectangle (13.2239,2.30221);
\draw [color=c, fill=c] (13.2239,2.19636) rectangle (13.2637,2.30221);
\draw [color=c, fill=c] (13.2637,2.19636) rectangle (13.3035,2.30221);
\draw [color=c, fill=c] (13.3035,2.19636) rectangle (13.3433,2.30221);
\draw [color=c, fill=c] (13.3433,2.19636) rectangle (13.3831,2.30221);
\draw [color=c, fill=c] (13.3831,2.19636) rectangle (13.4229,2.30221);
\draw [color=c, fill=c] (13.4229,2.19636) rectangle (13.4627,2.30221);
\draw [color=c, fill=c] (13.4627,2.19636) rectangle (13.5025,2.30221);
\draw [color=c, fill=c] (13.5025,2.19636) rectangle (13.5423,2.30221);
\draw [color=c, fill=c] (13.5423,2.19636) rectangle (13.5821,2.30221);
\draw [color=c, fill=c] (13.5821,2.19636) rectangle (13.6219,2.30221);
\draw [color=c, fill=c] (13.6219,2.19636) rectangle (13.6617,2.30221);
\draw [color=c, fill=c] (13.6617,2.19636) rectangle (13.7015,2.30221);
\draw [color=c, fill=c] (13.7015,2.19636) rectangle (13.7413,2.30221);
\draw [color=c, fill=c] (13.7413,2.19636) rectangle (13.7811,2.30221);
\draw [color=c, fill=c] (13.7811,2.19636) rectangle (13.8209,2.30221);
\draw [color=c, fill=c] (13.8209,2.19636) rectangle (13.8607,2.30221);
\draw [color=c, fill=c] (13.8607,2.19636) rectangle (13.9005,2.30221);
\draw [color=c, fill=c] (13.9005,2.19636) rectangle (13.9403,2.30221);
\draw [color=c, fill=c] (13.9403,2.19636) rectangle (13.9801,2.30221);
\draw [color=c, fill=c] (13.9801,2.19636) rectangle (14.0199,2.30221);
\draw [color=c, fill=c] (14.0199,2.19636) rectangle (14.0597,2.30221);
\draw [color=c, fill=c] (14.0597,2.19636) rectangle (14.0995,2.30221);
\draw [color=c, fill=c] (14.0995,2.19636) rectangle (14.1393,2.30221);
\draw [color=c, fill=c] (14.1393,2.19636) rectangle (14.1791,2.30221);
\draw [color=c, fill=c] (14.1791,2.19636) rectangle (14.2189,2.30221);
\draw [color=c, fill=c] (14.2189,2.19636) rectangle (14.2587,2.30221);
\draw [color=c, fill=c] (14.2587,2.19636) rectangle (14.2985,2.30221);
\draw [color=c, fill=c] (14.2985,2.19636) rectangle (14.3383,2.30221);
\draw [color=c, fill=c] (14.3383,2.19636) rectangle (14.3781,2.30221);
\draw [color=c, fill=c] (14.3781,2.19636) rectangle (14.4179,2.30221);
\draw [color=c, fill=c] (14.4179,2.19636) rectangle (14.4577,2.30221);
\draw [color=c, fill=c] (14.4577,2.19636) rectangle (14.4975,2.30221);
\draw [color=c, fill=c] (14.4975,2.19636) rectangle (14.5373,2.30221);
\draw [color=c, fill=c] (14.5373,2.19636) rectangle (14.5771,2.30221);
\draw [color=c, fill=c] (14.5771,2.19636) rectangle (14.6169,2.30221);
\draw [color=c, fill=c] (14.6169,2.19636) rectangle (14.6567,2.30221);
\draw [color=c, fill=c] (14.6567,2.19636) rectangle (14.6965,2.30221);
\draw [color=c, fill=c] (14.6965,2.19636) rectangle (14.7363,2.30221);
\draw [color=c, fill=c] (14.7363,2.19636) rectangle (14.7761,2.30221);
\draw [color=c, fill=c] (14.7761,2.19636) rectangle (14.8159,2.30221);
\draw [color=c, fill=c] (14.8159,2.19636) rectangle (14.8557,2.30221);
\draw [color=c, fill=c] (14.8557,2.19636) rectangle (14.8955,2.30221);
\draw [color=c, fill=c] (14.8955,2.19636) rectangle (14.9353,2.30221);
\draw [color=c, fill=c] (14.9353,2.19636) rectangle (14.9751,2.30221);
\draw [color=c, fill=c] (14.9751,2.19636) rectangle (15.0149,2.30221);
\draw [color=c, fill=c] (15.0149,2.19636) rectangle (15.0547,2.30221);
\draw [color=c, fill=c] (15.0547,2.19636) rectangle (15.0945,2.30221);
\draw [color=c, fill=c] (15.0945,2.19636) rectangle (15.1343,2.30221);
\draw [color=c, fill=c] (15.1343,2.19636) rectangle (15.1741,2.30221);
\draw [color=c, fill=c] (15.1741,2.19636) rectangle (15.2139,2.30221);
\draw [color=c, fill=c] (15.2139,2.19636) rectangle (15.2537,2.30221);
\draw [color=c, fill=c] (15.2537,2.19636) rectangle (15.2935,2.30221);
\draw [color=c, fill=c] (15.2935,2.19636) rectangle (15.3333,2.30221);
\draw [color=c, fill=c] (15.3333,2.19636) rectangle (15.3731,2.30221);
\draw [color=c, fill=c] (15.3731,2.19636) rectangle (15.4129,2.30221);
\draw [color=c, fill=c] (15.4129,2.19636) rectangle (15.4527,2.30221);
\draw [color=c, fill=c] (15.4527,2.19636) rectangle (15.4925,2.30221);
\draw [color=c, fill=c] (15.4925,2.19636) rectangle (15.5323,2.30221);
\draw [color=c, fill=c] (15.5323,2.19636) rectangle (15.5721,2.30221);
\draw [color=c, fill=c] (15.5721,2.19636) rectangle (15.6119,2.30221);
\draw [color=c, fill=c] (15.6119,2.19636) rectangle (15.6517,2.30221);
\draw [color=c, fill=c] (15.6517,2.19636) rectangle (15.6915,2.30221);
\draw [color=c, fill=c] (15.6915,2.19636) rectangle (15.7313,2.30221);
\draw [color=c, fill=c] (15.7313,2.19636) rectangle (15.7711,2.30221);
\draw [color=c, fill=c] (15.7711,2.19636) rectangle (15.8109,2.30221);
\draw [color=c, fill=c] (15.8109,2.19636) rectangle (15.8507,2.30221);
\draw [color=c, fill=c] (15.8507,2.19636) rectangle (15.8905,2.30221);
\draw [color=c, fill=c] (15.8905,2.19636) rectangle (15.9303,2.30221);
\draw [color=c, fill=c] (15.9303,2.19636) rectangle (15.9701,2.30221);
\draw [color=c, fill=c] (15.9701,2.19636) rectangle (16.01,2.30221);
\draw [color=c, fill=c] (16.01,2.19636) rectangle (16.0498,2.30221);
\draw [color=c, fill=c] (16.0498,2.19636) rectangle (16.0896,2.30221);
\draw [color=c, fill=c] (16.0896,2.19636) rectangle (16.1294,2.30221);
\draw [color=c, fill=c] (16.1294,2.19636) rectangle (16.1692,2.30221);
\draw [color=c, fill=c] (16.1692,2.19636) rectangle (16.209,2.30221);
\draw [color=c, fill=c] (16.209,2.19636) rectangle (16.2488,2.30221);
\draw [color=c, fill=c] (16.2488,2.19636) rectangle (16.2886,2.30221);
\draw [color=c, fill=c] (16.2886,2.19636) rectangle (16.3284,2.30221);
\draw [color=c, fill=c] (16.3284,2.19636) rectangle (16.3682,2.30221);
\draw [color=c, fill=c] (16.3682,2.19636) rectangle (16.408,2.30221);
\draw [color=c, fill=c] (16.408,2.19636) rectangle (16.4478,2.30221);
\draw [color=c, fill=c] (16.4478,2.19636) rectangle (16.4876,2.30221);
\draw [color=c, fill=c] (16.4876,2.19636) rectangle (16.5274,2.30221);
\draw [color=c, fill=c] (16.5274,2.19636) rectangle (16.5672,2.30221);
\draw [color=c, fill=c] (16.5672,2.19636) rectangle (16.607,2.30221);
\draw [color=c, fill=c] (16.607,2.19636) rectangle (16.6468,2.30221);
\draw [color=c, fill=c] (16.6468,2.19636) rectangle (16.6866,2.30221);
\draw [color=c, fill=c] (16.6866,2.19636) rectangle (16.7264,2.30221);
\draw [color=c, fill=c] (16.7264,2.19636) rectangle (16.7662,2.30221);
\draw [color=c, fill=c] (16.7662,2.19636) rectangle (16.806,2.30221);
\draw [color=c, fill=c] (16.806,2.19636) rectangle (16.8458,2.30221);
\draw [color=c, fill=c] (16.8458,2.19636) rectangle (16.8856,2.30221);
\draw [color=c, fill=c] (16.8856,2.19636) rectangle (16.9254,2.30221);
\draw [color=c, fill=c] (16.9254,2.19636) rectangle (16.9652,2.30221);
\draw [color=c, fill=c] (16.9652,2.19636) rectangle (17.005,2.30221);
\draw [color=c, fill=c] (17.005,2.19636) rectangle (17.0448,2.30221);
\draw [color=c, fill=c] (17.0448,2.19636) rectangle (17.0846,2.30221);
\draw [color=c, fill=c] (17.0846,2.19636) rectangle (17.1244,2.30221);
\draw [color=c, fill=c] (17.1244,2.19636) rectangle (17.1642,2.30221);
\draw [color=c, fill=c] (17.1642,2.19636) rectangle (17.204,2.30221);
\draw [color=c, fill=c] (17.204,2.19636) rectangle (17.2438,2.30221);
\draw [color=c, fill=c] (17.2438,2.19636) rectangle (17.2836,2.30221);
\draw [color=c, fill=c] (17.2836,2.19636) rectangle (17.3234,2.30221);
\draw [color=c, fill=c] (17.3234,2.19636) rectangle (17.3632,2.30221);
\draw [color=c, fill=c] (17.3632,2.19636) rectangle (17.403,2.30221);
\draw [color=c, fill=c] (17.403,2.19636) rectangle (17.4428,2.30221);
\draw [color=c, fill=c] (17.4428,2.19636) rectangle (17.4826,2.30221);
\draw [color=c, fill=c] (17.4826,2.19636) rectangle (17.5224,2.30221);
\draw [color=c, fill=c] (17.5224,2.19636) rectangle (17.5622,2.30221);
\draw [color=c, fill=c] (17.5622,2.19636) rectangle (17.602,2.30221);
\draw [color=c, fill=c] (17.602,2.19636) rectangle (17.6418,2.30221);
\draw [color=c, fill=c] (17.6418,2.19636) rectangle (17.6816,2.30221);
\draw [color=c, fill=c] (17.6816,2.19636) rectangle (17.7214,2.30221);
\draw [color=c, fill=c] (17.7214,2.19636) rectangle (17.7612,2.30221);
\draw [color=c, fill=c] (17.7612,2.19636) rectangle (17.801,2.30221);
\draw [color=c, fill=c] (17.801,2.19636) rectangle (17.8408,2.30221);
\draw [color=c, fill=c] (17.8408,2.19636) rectangle (17.8806,2.30221);
\draw [color=c, fill=c] (17.8806,2.19636) rectangle (17.9204,2.30221);
\draw [color=c, fill=c] (17.9204,2.19636) rectangle (17.9602,2.30221);
\draw [color=c, fill=c] (17.9602,2.19636) rectangle (18,2.30221);
\definecolor{c}{rgb}{1,0,0};
\draw [color=c, fill=c] (2,2.30221) rectangle (2.0398,2.40806);
\draw [color=c, fill=c] (2.0398,2.30221) rectangle (2.0796,2.40806);
\draw [color=c, fill=c] (2.0796,2.30221) rectangle (2.1194,2.40806);
\draw [color=c, fill=c] (2.1194,2.30221) rectangle (2.1592,2.40806);
\draw [color=c, fill=c] (2.1592,2.30221) rectangle (2.19901,2.40806);
\draw [color=c, fill=c] (2.19901,2.30221) rectangle (2.23881,2.40806);
\draw [color=c, fill=c] (2.23881,2.30221) rectangle (2.27861,2.40806);
\draw [color=c, fill=c] (2.27861,2.30221) rectangle (2.31841,2.40806);
\draw [color=c, fill=c] (2.31841,2.30221) rectangle (2.35821,2.40806);
\draw [color=c, fill=c] (2.35821,2.30221) rectangle (2.39801,2.40806);
\draw [color=c, fill=c] (2.39801,2.30221) rectangle (2.43781,2.40806);
\draw [color=c, fill=c] (2.43781,2.30221) rectangle (2.47761,2.40806);
\draw [color=c, fill=c] (2.47761,2.30221) rectangle (2.51741,2.40806);
\draw [color=c, fill=c] (2.51741,2.30221) rectangle (2.55721,2.40806);
\draw [color=c, fill=c] (2.55721,2.30221) rectangle (2.59702,2.40806);
\draw [color=c, fill=c] (2.59702,2.30221) rectangle (2.63682,2.40806);
\draw [color=c, fill=c] (2.63682,2.30221) rectangle (2.67662,2.40806);
\draw [color=c, fill=c] (2.67662,2.30221) rectangle (2.71642,2.40806);
\draw [color=c, fill=c] (2.71642,2.30221) rectangle (2.75622,2.40806);
\draw [color=c, fill=c] (2.75622,2.30221) rectangle (2.79602,2.40806);
\draw [color=c, fill=c] (2.79602,2.30221) rectangle (2.83582,2.40806);
\draw [color=c, fill=c] (2.83582,2.30221) rectangle (2.87562,2.40806);
\draw [color=c, fill=c] (2.87562,2.30221) rectangle (2.91542,2.40806);
\draw [color=c, fill=c] (2.91542,2.30221) rectangle (2.95522,2.40806);
\draw [color=c, fill=c] (2.95522,2.30221) rectangle (2.99502,2.40806);
\draw [color=c, fill=c] (2.99502,2.30221) rectangle (3.03483,2.40806);
\draw [color=c, fill=c] (3.03483,2.30221) rectangle (3.07463,2.40806);
\draw [color=c, fill=c] (3.07463,2.30221) rectangle (3.11443,2.40806);
\draw [color=c, fill=c] (3.11443,2.30221) rectangle (3.15423,2.40806);
\draw [color=c, fill=c] (3.15423,2.30221) rectangle (3.19403,2.40806);
\draw [color=c, fill=c] (3.19403,2.30221) rectangle (3.23383,2.40806);
\draw [color=c, fill=c] (3.23383,2.30221) rectangle (3.27363,2.40806);
\draw [color=c, fill=c] (3.27363,2.30221) rectangle (3.31343,2.40806);
\draw [color=c, fill=c] (3.31343,2.30221) rectangle (3.35323,2.40806);
\draw [color=c, fill=c] (3.35323,2.30221) rectangle (3.39303,2.40806);
\draw [color=c, fill=c] (3.39303,2.30221) rectangle (3.43284,2.40806);
\draw [color=c, fill=c] (3.43284,2.30221) rectangle (3.47264,2.40806);
\draw [color=c, fill=c] (3.47264,2.30221) rectangle (3.51244,2.40806);
\draw [color=c, fill=c] (3.51244,2.30221) rectangle (3.55224,2.40806);
\draw [color=c, fill=c] (3.55224,2.30221) rectangle (3.59204,2.40806);
\draw [color=c, fill=c] (3.59204,2.30221) rectangle (3.63184,2.40806);
\draw [color=c, fill=c] (3.63184,2.30221) rectangle (3.67164,2.40806);
\draw [color=c, fill=c] (3.67164,2.30221) rectangle (3.71144,2.40806);
\draw [color=c, fill=c] (3.71144,2.30221) rectangle (3.75124,2.40806);
\draw [color=c, fill=c] (3.75124,2.30221) rectangle (3.79104,2.40806);
\draw [color=c, fill=c] (3.79104,2.30221) rectangle (3.83085,2.40806);
\draw [color=c, fill=c] (3.83085,2.30221) rectangle (3.87065,2.40806);
\draw [color=c, fill=c] (3.87065,2.30221) rectangle (3.91045,2.40806);
\draw [color=c, fill=c] (3.91045,2.30221) rectangle (3.95025,2.40806);
\draw [color=c, fill=c] (3.95025,2.30221) rectangle (3.99005,2.40806);
\draw [color=c, fill=c] (3.99005,2.30221) rectangle (4.02985,2.40806);
\draw [color=c, fill=c] (4.02985,2.30221) rectangle (4.06965,2.40806);
\draw [color=c, fill=c] (4.06965,2.30221) rectangle (4.10945,2.40806);
\draw [color=c, fill=c] (4.10945,2.30221) rectangle (4.14925,2.40806);
\draw [color=c, fill=c] (4.14925,2.30221) rectangle (4.18905,2.40806);
\draw [color=c, fill=c] (4.18905,2.30221) rectangle (4.22886,2.40806);
\draw [color=c, fill=c] (4.22886,2.30221) rectangle (4.26866,2.40806);
\draw [color=c, fill=c] (4.26866,2.30221) rectangle (4.30846,2.40806);
\draw [color=c, fill=c] (4.30846,2.30221) rectangle (4.34826,2.40806);
\draw [color=c, fill=c] (4.34826,2.30221) rectangle (4.38806,2.40806);
\draw [color=c, fill=c] (4.38806,2.30221) rectangle (4.42786,2.40806);
\draw [color=c, fill=c] (4.42786,2.30221) rectangle (4.46766,2.40806);
\draw [color=c, fill=c] (4.46766,2.30221) rectangle (4.50746,2.40806);
\draw [color=c, fill=c] (4.50746,2.30221) rectangle (4.54726,2.40806);
\draw [color=c, fill=c] (4.54726,2.30221) rectangle (4.58706,2.40806);
\draw [color=c, fill=c] (4.58706,2.30221) rectangle (4.62687,2.40806);
\draw [color=c, fill=c] (4.62687,2.30221) rectangle (4.66667,2.40806);
\draw [color=c, fill=c] (4.66667,2.30221) rectangle (4.70647,2.40806);
\draw [color=c, fill=c] (4.70647,2.30221) rectangle (4.74627,2.40806);
\draw [color=c, fill=c] (4.74627,2.30221) rectangle (4.78607,2.40806);
\draw [color=c, fill=c] (4.78607,2.30221) rectangle (4.82587,2.40806);
\draw [color=c, fill=c] (4.82587,2.30221) rectangle (4.86567,2.40806);
\draw [color=c, fill=c] (4.86567,2.30221) rectangle (4.90547,2.40806);
\draw [color=c, fill=c] (4.90547,2.30221) rectangle (4.94527,2.40806);
\draw [color=c, fill=c] (4.94527,2.30221) rectangle (4.98507,2.40806);
\draw [color=c, fill=c] (4.98507,2.30221) rectangle (5.02488,2.40806);
\draw [color=c, fill=c] (5.02488,2.30221) rectangle (5.06468,2.40806);
\draw [color=c, fill=c] (5.06468,2.30221) rectangle (5.10448,2.40806);
\draw [color=c, fill=c] (5.10448,2.30221) rectangle (5.14428,2.40806);
\draw [color=c, fill=c] (5.14428,2.30221) rectangle (5.18408,2.40806);
\draw [color=c, fill=c] (5.18408,2.30221) rectangle (5.22388,2.40806);
\draw [color=c, fill=c] (5.22388,2.30221) rectangle (5.26368,2.40806);
\draw [color=c, fill=c] (5.26368,2.30221) rectangle (5.30348,2.40806);
\draw [color=c, fill=c] (5.30348,2.30221) rectangle (5.34328,2.40806);
\draw [color=c, fill=c] (5.34328,2.30221) rectangle (5.38308,2.40806);
\draw [color=c, fill=c] (5.38308,2.30221) rectangle (5.42289,2.40806);
\draw [color=c, fill=c] (5.42289,2.30221) rectangle (5.46269,2.40806);
\draw [color=c, fill=c] (5.46269,2.30221) rectangle (5.50249,2.40806);
\draw [color=c, fill=c] (5.50249,2.30221) rectangle (5.54229,2.40806);
\draw [color=c, fill=c] (5.54229,2.30221) rectangle (5.58209,2.40806);
\draw [color=c, fill=c] (5.58209,2.30221) rectangle (5.62189,2.40806);
\draw [color=c, fill=c] (5.62189,2.30221) rectangle (5.66169,2.40806);
\draw [color=c, fill=c] (5.66169,2.30221) rectangle (5.70149,2.40806);
\draw [color=c, fill=c] (5.70149,2.30221) rectangle (5.74129,2.40806);
\draw [color=c, fill=c] (5.74129,2.30221) rectangle (5.78109,2.40806);
\draw [color=c, fill=c] (5.78109,2.30221) rectangle (5.8209,2.40806);
\draw [color=c, fill=c] (5.8209,2.30221) rectangle (5.8607,2.40806);
\draw [color=c, fill=c] (5.8607,2.30221) rectangle (5.9005,2.40806);
\draw [color=c, fill=c] (5.9005,2.30221) rectangle (5.9403,2.40806);
\draw [color=c, fill=c] (5.9403,2.30221) rectangle (5.9801,2.40806);
\draw [color=c, fill=c] (5.9801,2.30221) rectangle (6.0199,2.40806);
\draw [color=c, fill=c] (6.0199,2.30221) rectangle (6.0597,2.40806);
\draw [color=c, fill=c] (6.0597,2.30221) rectangle (6.0995,2.40806);
\draw [color=c, fill=c] (6.0995,2.30221) rectangle (6.1393,2.40806);
\draw [color=c, fill=c] (6.1393,2.30221) rectangle (6.1791,2.40806);
\draw [color=c, fill=c] (6.1791,2.30221) rectangle (6.21891,2.40806);
\draw [color=c, fill=c] (6.21891,2.30221) rectangle (6.25871,2.40806);
\draw [color=c, fill=c] (6.25871,2.30221) rectangle (6.29851,2.40806);
\draw [color=c, fill=c] (6.29851,2.30221) rectangle (6.33831,2.40806);
\draw [color=c, fill=c] (6.33831,2.30221) rectangle (6.37811,2.40806);
\draw [color=c, fill=c] (6.37811,2.30221) rectangle (6.41791,2.40806);
\draw [color=c, fill=c] (6.41791,2.30221) rectangle (6.45771,2.40806);
\draw [color=c, fill=c] (6.45771,2.30221) rectangle (6.49751,2.40806);
\draw [color=c, fill=c] (6.49751,2.30221) rectangle (6.53731,2.40806);
\draw [color=c, fill=c] (6.53731,2.30221) rectangle (6.57711,2.40806);
\draw [color=c, fill=c] (6.57711,2.30221) rectangle (6.61692,2.40806);
\draw [color=c, fill=c] (6.61692,2.30221) rectangle (6.65672,2.40806);
\draw [color=c, fill=c] (6.65672,2.30221) rectangle (6.69652,2.40806);
\draw [color=c, fill=c] (6.69652,2.30221) rectangle (6.73632,2.40806);
\draw [color=c, fill=c] (6.73632,2.30221) rectangle (6.77612,2.40806);
\draw [color=c, fill=c] (6.77612,2.30221) rectangle (6.81592,2.40806);
\draw [color=c, fill=c] (6.81592,2.30221) rectangle (6.85572,2.40806);
\draw [color=c, fill=c] (6.85572,2.30221) rectangle (6.89552,2.40806);
\draw [color=c, fill=c] (6.89552,2.30221) rectangle (6.93532,2.40806);
\draw [color=c, fill=c] (6.93532,2.30221) rectangle (6.97512,2.40806);
\draw [color=c, fill=c] (6.97512,2.30221) rectangle (7.01493,2.40806);
\draw [color=c, fill=c] (7.01493,2.30221) rectangle (7.05473,2.40806);
\draw [color=c, fill=c] (7.05473,2.30221) rectangle (7.09453,2.40806);
\draw [color=c, fill=c] (7.09453,2.30221) rectangle (7.13433,2.40806);
\draw [color=c, fill=c] (7.13433,2.30221) rectangle (7.17413,2.40806);
\draw [color=c, fill=c] (7.17413,2.30221) rectangle (7.21393,2.40806);
\draw [color=c, fill=c] (7.21393,2.30221) rectangle (7.25373,2.40806);
\draw [color=c, fill=c] (7.25373,2.30221) rectangle (7.29353,2.40806);
\draw [color=c, fill=c] (7.29353,2.30221) rectangle (7.33333,2.40806);
\draw [color=c, fill=c] (7.33333,2.30221) rectangle (7.37313,2.40806);
\draw [color=c, fill=c] (7.37313,2.30221) rectangle (7.41294,2.40806);
\draw [color=c, fill=c] (7.41294,2.30221) rectangle (7.45274,2.40806);
\draw [color=c, fill=c] (7.45274,2.30221) rectangle (7.49254,2.40806);
\draw [color=c, fill=c] (7.49254,2.30221) rectangle (7.53234,2.40806);
\draw [color=c, fill=c] (7.53234,2.30221) rectangle (7.57214,2.40806);
\draw [color=c, fill=c] (7.57214,2.30221) rectangle (7.61194,2.40806);
\draw [color=c, fill=c] (7.61194,2.30221) rectangle (7.65174,2.40806);
\draw [color=c, fill=c] (7.65174,2.30221) rectangle (7.69154,2.40806);
\draw [color=c, fill=c] (7.69154,2.30221) rectangle (7.73134,2.40806);
\draw [color=c, fill=c] (7.73134,2.30221) rectangle (7.77114,2.40806);
\draw [color=c, fill=c] (7.77114,2.30221) rectangle (7.81095,2.40806);
\definecolor{c}{rgb}{1,0.186667,0};
\draw [color=c, fill=c] (7.81095,2.30221) rectangle (7.85075,2.40806);
\draw [color=c, fill=c] (7.85075,2.30221) rectangle (7.89055,2.40806);
\draw [color=c, fill=c] (7.89055,2.30221) rectangle (7.93035,2.40806);
\draw [color=c, fill=c] (7.93035,2.30221) rectangle (7.97015,2.40806);
\draw [color=c, fill=c] (7.97015,2.30221) rectangle (8.00995,2.40806);
\draw [color=c, fill=c] (8.00995,2.30221) rectangle (8.04975,2.40806);
\draw [color=c, fill=c] (8.04975,2.30221) rectangle (8.08955,2.40806);
\draw [color=c, fill=c] (8.08955,2.30221) rectangle (8.12935,2.40806);
\draw [color=c, fill=c] (8.12935,2.30221) rectangle (8.16915,2.40806);
\draw [color=c, fill=c] (8.16915,2.30221) rectangle (8.20895,2.40806);
\draw [color=c, fill=c] (8.20895,2.30221) rectangle (8.24876,2.40806);
\draw [color=c, fill=c] (8.24876,2.30221) rectangle (8.28856,2.40806);
\draw [color=c, fill=c] (8.28856,2.30221) rectangle (8.32836,2.40806);
\draw [color=c, fill=c] (8.32836,2.30221) rectangle (8.36816,2.40806);
\draw [color=c, fill=c] (8.36816,2.30221) rectangle (8.40796,2.40806);
\draw [color=c, fill=c] (8.40796,2.30221) rectangle (8.44776,2.40806);
\draw [color=c, fill=c] (8.44776,2.30221) rectangle (8.48756,2.40806);
\draw [color=c, fill=c] (8.48756,2.30221) rectangle (8.52736,2.40806);
\draw [color=c, fill=c] (8.52736,2.30221) rectangle (8.56716,2.40806);
\draw [color=c, fill=c] (8.56716,2.30221) rectangle (8.60697,2.40806);
\draw [color=c, fill=c] (8.60697,2.30221) rectangle (8.64677,2.40806);
\definecolor{c}{rgb}{1,0.466667,0};
\draw [color=c, fill=c] (8.64677,2.30221) rectangle (8.68657,2.40806);
\draw [color=c, fill=c] (8.68657,2.30221) rectangle (8.72637,2.40806);
\draw [color=c, fill=c] (8.72637,2.30221) rectangle (8.76617,2.40806);
\draw [color=c, fill=c] (8.76617,2.30221) rectangle (8.80597,2.40806);
\draw [color=c, fill=c] (8.80597,2.30221) rectangle (8.84577,2.40806);
\draw [color=c, fill=c] (8.84577,2.30221) rectangle (8.88557,2.40806);
\draw [color=c, fill=c] (8.88557,2.30221) rectangle (8.92537,2.40806);
\draw [color=c, fill=c] (8.92537,2.30221) rectangle (8.96517,2.40806);
\draw [color=c, fill=c] (8.96517,2.30221) rectangle (9.00498,2.40806);
\draw [color=c, fill=c] (9.00498,2.30221) rectangle (9.04478,2.40806);
\draw [color=c, fill=c] (9.04478,2.30221) rectangle (9.08458,2.40806);
\definecolor{c}{rgb}{1,0.653333,0};
\draw [color=c, fill=c] (9.08458,2.30221) rectangle (9.12438,2.40806);
\draw [color=c, fill=c] (9.12438,2.30221) rectangle (9.16418,2.40806);
\draw [color=c, fill=c] (9.16418,2.30221) rectangle (9.20398,2.40806);
\draw [color=c, fill=c] (9.20398,2.30221) rectangle (9.24378,2.40806);
\draw [color=c, fill=c] (9.24378,2.30221) rectangle (9.28358,2.40806);
\draw [color=c, fill=c] (9.28358,2.30221) rectangle (9.32338,2.40806);
\draw [color=c, fill=c] (9.32338,2.30221) rectangle (9.36318,2.40806);
\definecolor{c}{rgb}{1,0.933333,0};
\draw [color=c, fill=c] (9.36318,2.30221) rectangle (9.40298,2.40806);
\draw [color=c, fill=c] (9.40298,2.30221) rectangle (9.44279,2.40806);
\draw [color=c, fill=c] (9.44279,2.30221) rectangle (9.48259,2.40806);
\draw [color=c, fill=c] (9.48259,2.30221) rectangle (9.52239,2.40806);
\draw [color=c, fill=c] (9.52239,2.30221) rectangle (9.56219,2.40806);
\draw [color=c, fill=c] (9.56219,2.30221) rectangle (9.60199,2.40806);
\definecolor{c}{rgb}{0.88,1,0};
\draw [color=c, fill=c] (9.60199,2.30221) rectangle (9.64179,2.40806);
\draw [color=c, fill=c] (9.64179,2.30221) rectangle (9.68159,2.40806);
\draw [color=c, fill=c] (9.68159,2.30221) rectangle (9.72139,2.40806);
\draw [color=c, fill=c] (9.72139,2.30221) rectangle (9.76119,2.40806);
\definecolor{c}{rgb}{0.6,1,0};
\draw [color=c, fill=c] (9.76119,2.30221) rectangle (9.80099,2.40806);
\draw [color=c, fill=c] (9.80099,2.30221) rectangle (9.8408,2.40806);
\draw [color=c, fill=c] (9.8408,2.30221) rectangle (9.8806,2.40806);
\draw [color=c, fill=c] (9.8806,2.30221) rectangle (9.9204,2.40806);
\definecolor{c}{rgb}{0.413333,1,0};
\draw [color=c, fill=c] (9.9204,2.30221) rectangle (9.9602,2.40806);
\draw [color=c, fill=c] (9.9602,2.30221) rectangle (10,2.40806);
\draw [color=c, fill=c] (10,2.30221) rectangle (10.0398,2.40806);
\definecolor{c}{rgb}{0.133333,1,0};
\draw [color=c, fill=c] (10.0398,2.30221) rectangle (10.0796,2.40806);
\draw [color=c, fill=c] (10.0796,2.30221) rectangle (10.1194,2.40806);
\draw [color=c, fill=c] (10.1194,2.30221) rectangle (10.1592,2.40806);
\draw [color=c, fill=c] (10.1592,2.30221) rectangle (10.199,2.40806);
\definecolor{c}{rgb}{0,1,0.0533333};
\draw [color=c, fill=c] (10.199,2.30221) rectangle (10.2388,2.40806);
\draw [color=c, fill=c] (10.2388,2.30221) rectangle (10.2786,2.40806);
\draw [color=c, fill=c] (10.2786,2.30221) rectangle (10.3184,2.40806);
\draw [color=c, fill=c] (10.3184,2.30221) rectangle (10.3582,2.40806);
\definecolor{c}{rgb}{0,1,0.333333};
\draw [color=c, fill=c] (10.3582,2.30221) rectangle (10.398,2.40806);
\draw [color=c, fill=c] (10.398,2.30221) rectangle (10.4378,2.40806);
\draw [color=c, fill=c] (10.4378,2.30221) rectangle (10.4776,2.40806);
\draw [color=c, fill=c] (10.4776,2.30221) rectangle (10.5174,2.40806);
\draw [color=c, fill=c] (10.5174,2.30221) rectangle (10.5572,2.40806);
\definecolor{c}{rgb}{0,1,0.52};
\draw [color=c, fill=c] (10.5572,2.30221) rectangle (10.597,2.40806);
\draw [color=c, fill=c] (10.597,2.30221) rectangle (10.6368,2.40806);
\draw [color=c, fill=c] (10.6368,2.30221) rectangle (10.6766,2.40806);
\draw [color=c, fill=c] (10.6766,2.30221) rectangle (10.7164,2.40806);
\draw [color=c, fill=c] (10.7164,2.30221) rectangle (10.7562,2.40806);
\draw [color=c, fill=c] (10.7562,2.30221) rectangle (10.796,2.40806);
\draw [color=c, fill=c] (10.796,2.30221) rectangle (10.8358,2.40806);
\draw [color=c, fill=c] (10.8358,2.30221) rectangle (10.8756,2.40806);
\definecolor{c}{rgb}{0,1,0.8};
\draw [color=c, fill=c] (10.8756,2.30221) rectangle (10.9154,2.40806);
\draw [color=c, fill=c] (10.9154,2.30221) rectangle (10.9552,2.40806);
\draw [color=c, fill=c] (10.9552,2.30221) rectangle (10.995,2.40806);
\draw [color=c, fill=c] (10.995,2.30221) rectangle (11.0348,2.40806);
\draw [color=c, fill=c] (11.0348,2.30221) rectangle (11.0746,2.40806);
\draw [color=c, fill=c] (11.0746,2.30221) rectangle (11.1144,2.40806);
\draw [color=c, fill=c] (11.1144,2.30221) rectangle (11.1542,2.40806);
\draw [color=c, fill=c] (11.1542,2.30221) rectangle (11.194,2.40806);
\draw [color=c, fill=c] (11.194,2.30221) rectangle (11.2338,2.40806);
\draw [color=c, fill=c] (11.2338,2.30221) rectangle (11.2736,2.40806);
\draw [color=c, fill=c] (11.2736,2.30221) rectangle (11.3134,2.40806);
\draw [color=c, fill=c] (11.3134,2.30221) rectangle (11.3532,2.40806);
\definecolor{c}{rgb}{0,1,0.986667};
\draw [color=c, fill=c] (11.3532,2.30221) rectangle (11.393,2.40806);
\draw [color=c, fill=c] (11.393,2.30221) rectangle (11.4328,2.40806);
\draw [color=c, fill=c] (11.4328,2.30221) rectangle (11.4726,2.40806);
\draw [color=c, fill=c] (11.4726,2.30221) rectangle (11.5124,2.40806);
\draw [color=c, fill=c] (11.5124,2.30221) rectangle (11.5522,2.40806);
\draw [color=c, fill=c] (11.5522,2.30221) rectangle (11.592,2.40806);
\draw [color=c, fill=c] (11.592,2.30221) rectangle (11.6318,2.40806);
\draw [color=c, fill=c] (11.6318,2.30221) rectangle (11.6716,2.40806);
\draw [color=c, fill=c] (11.6716,2.30221) rectangle (11.7114,2.40806);
\draw [color=c, fill=c] (11.7114,2.30221) rectangle (11.7512,2.40806);
\draw [color=c, fill=c] (11.7512,2.30221) rectangle (11.791,2.40806);
\draw [color=c, fill=c] (11.791,2.30221) rectangle (11.8308,2.40806);
\draw [color=c, fill=c] (11.8308,2.30221) rectangle (11.8706,2.40806);
\draw [color=c, fill=c] (11.8706,2.30221) rectangle (11.9104,2.40806);
\draw [color=c, fill=c] (11.9104,2.30221) rectangle (11.9502,2.40806);
\draw [color=c, fill=c] (11.9502,2.30221) rectangle (11.99,2.40806);
\draw [color=c, fill=c] (11.99,2.30221) rectangle (12.0299,2.40806);
\draw [color=c, fill=c] (12.0299,2.30221) rectangle (12.0697,2.40806);
\draw [color=c, fill=c] (12.0697,2.30221) rectangle (12.1095,2.40806);
\draw [color=c, fill=c] (12.1095,2.30221) rectangle (12.1493,2.40806);
\draw [color=c, fill=c] (12.1493,2.30221) rectangle (12.1891,2.40806);
\draw [color=c, fill=c] (12.1891,2.30221) rectangle (12.2289,2.40806);
\draw [color=c, fill=c] (12.2289,2.30221) rectangle (12.2687,2.40806);
\definecolor{c}{rgb}{0,0.733333,1};
\draw [color=c, fill=c] (12.2687,2.30221) rectangle (12.3085,2.40806);
\draw [color=c, fill=c] (12.3085,2.30221) rectangle (12.3483,2.40806);
\draw [color=c, fill=c] (12.3483,2.30221) rectangle (12.3881,2.40806);
\draw [color=c, fill=c] (12.3881,2.30221) rectangle (12.4279,2.40806);
\draw [color=c, fill=c] (12.4279,2.30221) rectangle (12.4677,2.40806);
\draw [color=c, fill=c] (12.4677,2.30221) rectangle (12.5075,2.40806);
\draw [color=c, fill=c] (12.5075,2.30221) rectangle (12.5473,2.40806);
\draw [color=c, fill=c] (12.5473,2.30221) rectangle (12.5871,2.40806);
\draw [color=c, fill=c] (12.5871,2.30221) rectangle (12.6269,2.40806);
\draw [color=c, fill=c] (12.6269,2.30221) rectangle (12.6667,2.40806);
\draw [color=c, fill=c] (12.6667,2.30221) rectangle (12.7065,2.40806);
\draw [color=c, fill=c] (12.7065,2.30221) rectangle (12.7463,2.40806);
\draw [color=c, fill=c] (12.7463,2.30221) rectangle (12.7861,2.40806);
\draw [color=c, fill=c] (12.7861,2.30221) rectangle (12.8259,2.40806);
\draw [color=c, fill=c] (12.8259,2.30221) rectangle (12.8657,2.40806);
\draw [color=c, fill=c] (12.8657,2.30221) rectangle (12.9055,2.40806);
\draw [color=c, fill=c] (12.9055,2.30221) rectangle (12.9453,2.40806);
\draw [color=c, fill=c] (12.9453,2.30221) rectangle (12.9851,2.40806);
\draw [color=c, fill=c] (12.9851,2.30221) rectangle (13.0249,2.40806);
\draw [color=c, fill=c] (13.0249,2.30221) rectangle (13.0647,2.40806);
\draw [color=c, fill=c] (13.0647,2.30221) rectangle (13.1045,2.40806);
\draw [color=c, fill=c] (13.1045,2.30221) rectangle (13.1443,2.40806);
\draw [color=c, fill=c] (13.1443,2.30221) rectangle (13.1841,2.40806);
\draw [color=c, fill=c] (13.1841,2.30221) rectangle (13.2239,2.40806);
\draw [color=c, fill=c] (13.2239,2.30221) rectangle (13.2637,2.40806);
\draw [color=c, fill=c] (13.2637,2.30221) rectangle (13.3035,2.40806);
\draw [color=c, fill=c] (13.3035,2.30221) rectangle (13.3433,2.40806);
\draw [color=c, fill=c] (13.3433,2.30221) rectangle (13.3831,2.40806);
\draw [color=c, fill=c] (13.3831,2.30221) rectangle (13.4229,2.40806);
\draw [color=c, fill=c] (13.4229,2.30221) rectangle (13.4627,2.40806);
\draw [color=c, fill=c] (13.4627,2.30221) rectangle (13.5025,2.40806);
\draw [color=c, fill=c] (13.5025,2.30221) rectangle (13.5423,2.40806);
\draw [color=c, fill=c] (13.5423,2.30221) rectangle (13.5821,2.40806);
\draw [color=c, fill=c] (13.5821,2.30221) rectangle (13.6219,2.40806);
\draw [color=c, fill=c] (13.6219,2.30221) rectangle (13.6617,2.40806);
\draw [color=c, fill=c] (13.6617,2.30221) rectangle (13.7015,2.40806);
\draw [color=c, fill=c] (13.7015,2.30221) rectangle (13.7413,2.40806);
\draw [color=c, fill=c] (13.7413,2.30221) rectangle (13.7811,2.40806);
\draw [color=c, fill=c] (13.7811,2.30221) rectangle (13.8209,2.40806);
\draw [color=c, fill=c] (13.8209,2.30221) rectangle (13.8607,2.40806);
\draw [color=c, fill=c] (13.8607,2.30221) rectangle (13.9005,2.40806);
\draw [color=c, fill=c] (13.9005,2.30221) rectangle (13.9403,2.40806);
\draw [color=c, fill=c] (13.9403,2.30221) rectangle (13.9801,2.40806);
\draw [color=c, fill=c] (13.9801,2.30221) rectangle (14.0199,2.40806);
\draw [color=c, fill=c] (14.0199,2.30221) rectangle (14.0597,2.40806);
\draw [color=c, fill=c] (14.0597,2.30221) rectangle (14.0995,2.40806);
\draw [color=c, fill=c] (14.0995,2.30221) rectangle (14.1393,2.40806);
\draw [color=c, fill=c] (14.1393,2.30221) rectangle (14.1791,2.40806);
\draw [color=c, fill=c] (14.1791,2.30221) rectangle (14.2189,2.40806);
\draw [color=c, fill=c] (14.2189,2.30221) rectangle (14.2587,2.40806);
\draw [color=c, fill=c] (14.2587,2.30221) rectangle (14.2985,2.40806);
\draw [color=c, fill=c] (14.2985,2.30221) rectangle (14.3383,2.40806);
\draw [color=c, fill=c] (14.3383,2.30221) rectangle (14.3781,2.40806);
\draw [color=c, fill=c] (14.3781,2.30221) rectangle (14.4179,2.40806);
\draw [color=c, fill=c] (14.4179,2.30221) rectangle (14.4577,2.40806);
\draw [color=c, fill=c] (14.4577,2.30221) rectangle (14.4975,2.40806);
\draw [color=c, fill=c] (14.4975,2.30221) rectangle (14.5373,2.40806);
\draw [color=c, fill=c] (14.5373,2.30221) rectangle (14.5771,2.40806);
\draw [color=c, fill=c] (14.5771,2.30221) rectangle (14.6169,2.40806);
\draw [color=c, fill=c] (14.6169,2.30221) rectangle (14.6567,2.40806);
\draw [color=c, fill=c] (14.6567,2.30221) rectangle (14.6965,2.40806);
\draw [color=c, fill=c] (14.6965,2.30221) rectangle (14.7363,2.40806);
\draw [color=c, fill=c] (14.7363,2.30221) rectangle (14.7761,2.40806);
\draw [color=c, fill=c] (14.7761,2.30221) rectangle (14.8159,2.40806);
\draw [color=c, fill=c] (14.8159,2.30221) rectangle (14.8557,2.40806);
\draw [color=c, fill=c] (14.8557,2.30221) rectangle (14.8955,2.40806);
\draw [color=c, fill=c] (14.8955,2.30221) rectangle (14.9353,2.40806);
\draw [color=c, fill=c] (14.9353,2.30221) rectangle (14.9751,2.40806);
\draw [color=c, fill=c] (14.9751,2.30221) rectangle (15.0149,2.40806);
\draw [color=c, fill=c] (15.0149,2.30221) rectangle (15.0547,2.40806);
\draw [color=c, fill=c] (15.0547,2.30221) rectangle (15.0945,2.40806);
\draw [color=c, fill=c] (15.0945,2.30221) rectangle (15.1343,2.40806);
\draw [color=c, fill=c] (15.1343,2.30221) rectangle (15.1741,2.40806);
\draw [color=c, fill=c] (15.1741,2.30221) rectangle (15.2139,2.40806);
\draw [color=c, fill=c] (15.2139,2.30221) rectangle (15.2537,2.40806);
\draw [color=c, fill=c] (15.2537,2.30221) rectangle (15.2935,2.40806);
\draw [color=c, fill=c] (15.2935,2.30221) rectangle (15.3333,2.40806);
\draw [color=c, fill=c] (15.3333,2.30221) rectangle (15.3731,2.40806);
\draw [color=c, fill=c] (15.3731,2.30221) rectangle (15.4129,2.40806);
\draw [color=c, fill=c] (15.4129,2.30221) rectangle (15.4527,2.40806);
\draw [color=c, fill=c] (15.4527,2.30221) rectangle (15.4925,2.40806);
\draw [color=c, fill=c] (15.4925,2.30221) rectangle (15.5323,2.40806);
\draw [color=c, fill=c] (15.5323,2.30221) rectangle (15.5721,2.40806);
\draw [color=c, fill=c] (15.5721,2.30221) rectangle (15.6119,2.40806);
\draw [color=c, fill=c] (15.6119,2.30221) rectangle (15.6517,2.40806);
\draw [color=c, fill=c] (15.6517,2.30221) rectangle (15.6915,2.40806);
\draw [color=c, fill=c] (15.6915,2.30221) rectangle (15.7313,2.40806);
\draw [color=c, fill=c] (15.7313,2.30221) rectangle (15.7711,2.40806);
\draw [color=c, fill=c] (15.7711,2.30221) rectangle (15.8109,2.40806);
\draw [color=c, fill=c] (15.8109,2.30221) rectangle (15.8507,2.40806);
\draw [color=c, fill=c] (15.8507,2.30221) rectangle (15.8905,2.40806);
\draw [color=c, fill=c] (15.8905,2.30221) rectangle (15.9303,2.40806);
\draw [color=c, fill=c] (15.9303,2.30221) rectangle (15.9701,2.40806);
\draw [color=c, fill=c] (15.9701,2.30221) rectangle (16.01,2.40806);
\draw [color=c, fill=c] (16.01,2.30221) rectangle (16.0498,2.40806);
\draw [color=c, fill=c] (16.0498,2.30221) rectangle (16.0896,2.40806);
\draw [color=c, fill=c] (16.0896,2.30221) rectangle (16.1294,2.40806);
\draw [color=c, fill=c] (16.1294,2.30221) rectangle (16.1692,2.40806);
\draw [color=c, fill=c] (16.1692,2.30221) rectangle (16.209,2.40806);
\draw [color=c, fill=c] (16.209,2.30221) rectangle (16.2488,2.40806);
\draw [color=c, fill=c] (16.2488,2.30221) rectangle (16.2886,2.40806);
\draw [color=c, fill=c] (16.2886,2.30221) rectangle (16.3284,2.40806);
\draw [color=c, fill=c] (16.3284,2.30221) rectangle (16.3682,2.40806);
\draw [color=c, fill=c] (16.3682,2.30221) rectangle (16.408,2.40806);
\draw [color=c, fill=c] (16.408,2.30221) rectangle (16.4478,2.40806);
\draw [color=c, fill=c] (16.4478,2.30221) rectangle (16.4876,2.40806);
\draw [color=c, fill=c] (16.4876,2.30221) rectangle (16.5274,2.40806);
\draw [color=c, fill=c] (16.5274,2.30221) rectangle (16.5672,2.40806);
\draw [color=c, fill=c] (16.5672,2.30221) rectangle (16.607,2.40806);
\draw [color=c, fill=c] (16.607,2.30221) rectangle (16.6468,2.40806);
\draw [color=c, fill=c] (16.6468,2.30221) rectangle (16.6866,2.40806);
\draw [color=c, fill=c] (16.6866,2.30221) rectangle (16.7264,2.40806);
\draw [color=c, fill=c] (16.7264,2.30221) rectangle (16.7662,2.40806);
\draw [color=c, fill=c] (16.7662,2.30221) rectangle (16.806,2.40806);
\draw [color=c, fill=c] (16.806,2.30221) rectangle (16.8458,2.40806);
\draw [color=c, fill=c] (16.8458,2.30221) rectangle (16.8856,2.40806);
\draw [color=c, fill=c] (16.8856,2.30221) rectangle (16.9254,2.40806);
\draw [color=c, fill=c] (16.9254,2.30221) rectangle (16.9652,2.40806);
\draw [color=c, fill=c] (16.9652,2.30221) rectangle (17.005,2.40806);
\draw [color=c, fill=c] (17.005,2.30221) rectangle (17.0448,2.40806);
\draw [color=c, fill=c] (17.0448,2.30221) rectangle (17.0846,2.40806);
\draw [color=c, fill=c] (17.0846,2.30221) rectangle (17.1244,2.40806);
\draw [color=c, fill=c] (17.1244,2.30221) rectangle (17.1642,2.40806);
\draw [color=c, fill=c] (17.1642,2.30221) rectangle (17.204,2.40806);
\draw [color=c, fill=c] (17.204,2.30221) rectangle (17.2438,2.40806);
\draw [color=c, fill=c] (17.2438,2.30221) rectangle (17.2836,2.40806);
\draw [color=c, fill=c] (17.2836,2.30221) rectangle (17.3234,2.40806);
\draw [color=c, fill=c] (17.3234,2.30221) rectangle (17.3632,2.40806);
\draw [color=c, fill=c] (17.3632,2.30221) rectangle (17.403,2.40806);
\draw [color=c, fill=c] (17.403,2.30221) rectangle (17.4428,2.40806);
\draw [color=c, fill=c] (17.4428,2.30221) rectangle (17.4826,2.40806);
\draw [color=c, fill=c] (17.4826,2.30221) rectangle (17.5224,2.40806);
\draw [color=c, fill=c] (17.5224,2.30221) rectangle (17.5622,2.40806);
\draw [color=c, fill=c] (17.5622,2.30221) rectangle (17.602,2.40806);
\draw [color=c, fill=c] (17.602,2.30221) rectangle (17.6418,2.40806);
\draw [color=c, fill=c] (17.6418,2.30221) rectangle (17.6816,2.40806);
\draw [color=c, fill=c] (17.6816,2.30221) rectangle (17.7214,2.40806);
\draw [color=c, fill=c] (17.7214,2.30221) rectangle (17.7612,2.40806);
\draw [color=c, fill=c] (17.7612,2.30221) rectangle (17.801,2.40806);
\draw [color=c, fill=c] (17.801,2.30221) rectangle (17.8408,2.40806);
\draw [color=c, fill=c] (17.8408,2.30221) rectangle (17.8806,2.40806);
\draw [color=c, fill=c] (17.8806,2.30221) rectangle (17.9204,2.40806);
\draw [color=c, fill=c] (17.9204,2.30221) rectangle (17.9602,2.40806);
\draw [color=c, fill=c] (17.9602,2.30221) rectangle (18,2.40806);
\definecolor{c}{rgb}{1,0,0};
\draw [color=c, fill=c] (2,2.40806) rectangle (2.0398,2.51391);
\draw [color=c, fill=c] (2.0398,2.40806) rectangle (2.0796,2.51391);
\draw [color=c, fill=c] (2.0796,2.40806) rectangle (2.1194,2.51391);
\draw [color=c, fill=c] (2.1194,2.40806) rectangle (2.1592,2.51391);
\draw [color=c, fill=c] (2.1592,2.40806) rectangle (2.19901,2.51391);
\draw [color=c, fill=c] (2.19901,2.40806) rectangle (2.23881,2.51391);
\draw [color=c, fill=c] (2.23881,2.40806) rectangle (2.27861,2.51391);
\draw [color=c, fill=c] (2.27861,2.40806) rectangle (2.31841,2.51391);
\draw [color=c, fill=c] (2.31841,2.40806) rectangle (2.35821,2.51391);
\draw [color=c, fill=c] (2.35821,2.40806) rectangle (2.39801,2.51391);
\draw [color=c, fill=c] (2.39801,2.40806) rectangle (2.43781,2.51391);
\draw [color=c, fill=c] (2.43781,2.40806) rectangle (2.47761,2.51391);
\draw [color=c, fill=c] (2.47761,2.40806) rectangle (2.51741,2.51391);
\draw [color=c, fill=c] (2.51741,2.40806) rectangle (2.55721,2.51391);
\draw [color=c, fill=c] (2.55721,2.40806) rectangle (2.59702,2.51391);
\draw [color=c, fill=c] (2.59702,2.40806) rectangle (2.63682,2.51391);
\draw [color=c, fill=c] (2.63682,2.40806) rectangle (2.67662,2.51391);
\draw [color=c, fill=c] (2.67662,2.40806) rectangle (2.71642,2.51391);
\draw [color=c, fill=c] (2.71642,2.40806) rectangle (2.75622,2.51391);
\draw [color=c, fill=c] (2.75622,2.40806) rectangle (2.79602,2.51391);
\draw [color=c, fill=c] (2.79602,2.40806) rectangle (2.83582,2.51391);
\draw [color=c, fill=c] (2.83582,2.40806) rectangle (2.87562,2.51391);
\draw [color=c, fill=c] (2.87562,2.40806) rectangle (2.91542,2.51391);
\draw [color=c, fill=c] (2.91542,2.40806) rectangle (2.95522,2.51391);
\draw [color=c, fill=c] (2.95522,2.40806) rectangle (2.99502,2.51391);
\draw [color=c, fill=c] (2.99502,2.40806) rectangle (3.03483,2.51391);
\draw [color=c, fill=c] (3.03483,2.40806) rectangle (3.07463,2.51391);
\draw [color=c, fill=c] (3.07463,2.40806) rectangle (3.11443,2.51391);
\draw [color=c, fill=c] (3.11443,2.40806) rectangle (3.15423,2.51391);
\draw [color=c, fill=c] (3.15423,2.40806) rectangle (3.19403,2.51391);
\draw [color=c, fill=c] (3.19403,2.40806) rectangle (3.23383,2.51391);
\draw [color=c, fill=c] (3.23383,2.40806) rectangle (3.27363,2.51391);
\draw [color=c, fill=c] (3.27363,2.40806) rectangle (3.31343,2.51391);
\draw [color=c, fill=c] (3.31343,2.40806) rectangle (3.35323,2.51391);
\draw [color=c, fill=c] (3.35323,2.40806) rectangle (3.39303,2.51391);
\draw [color=c, fill=c] (3.39303,2.40806) rectangle (3.43284,2.51391);
\draw [color=c, fill=c] (3.43284,2.40806) rectangle (3.47264,2.51391);
\draw [color=c, fill=c] (3.47264,2.40806) rectangle (3.51244,2.51391);
\draw [color=c, fill=c] (3.51244,2.40806) rectangle (3.55224,2.51391);
\draw [color=c, fill=c] (3.55224,2.40806) rectangle (3.59204,2.51391);
\draw [color=c, fill=c] (3.59204,2.40806) rectangle (3.63184,2.51391);
\draw [color=c, fill=c] (3.63184,2.40806) rectangle (3.67164,2.51391);
\draw [color=c, fill=c] (3.67164,2.40806) rectangle (3.71144,2.51391);
\draw [color=c, fill=c] (3.71144,2.40806) rectangle (3.75124,2.51391);
\draw [color=c, fill=c] (3.75124,2.40806) rectangle (3.79104,2.51391);
\draw [color=c, fill=c] (3.79104,2.40806) rectangle (3.83085,2.51391);
\draw [color=c, fill=c] (3.83085,2.40806) rectangle (3.87065,2.51391);
\draw [color=c, fill=c] (3.87065,2.40806) rectangle (3.91045,2.51391);
\draw [color=c, fill=c] (3.91045,2.40806) rectangle (3.95025,2.51391);
\draw [color=c, fill=c] (3.95025,2.40806) rectangle (3.99005,2.51391);
\draw [color=c, fill=c] (3.99005,2.40806) rectangle (4.02985,2.51391);
\draw [color=c, fill=c] (4.02985,2.40806) rectangle (4.06965,2.51391);
\draw [color=c, fill=c] (4.06965,2.40806) rectangle (4.10945,2.51391);
\draw [color=c, fill=c] (4.10945,2.40806) rectangle (4.14925,2.51391);
\draw [color=c, fill=c] (4.14925,2.40806) rectangle (4.18905,2.51391);
\draw [color=c, fill=c] (4.18905,2.40806) rectangle (4.22886,2.51391);
\draw [color=c, fill=c] (4.22886,2.40806) rectangle (4.26866,2.51391);
\draw [color=c, fill=c] (4.26866,2.40806) rectangle (4.30846,2.51391);
\draw [color=c, fill=c] (4.30846,2.40806) rectangle (4.34826,2.51391);
\draw [color=c, fill=c] (4.34826,2.40806) rectangle (4.38806,2.51391);
\draw [color=c, fill=c] (4.38806,2.40806) rectangle (4.42786,2.51391);
\draw [color=c, fill=c] (4.42786,2.40806) rectangle (4.46766,2.51391);
\draw [color=c, fill=c] (4.46766,2.40806) rectangle (4.50746,2.51391);
\draw [color=c, fill=c] (4.50746,2.40806) rectangle (4.54726,2.51391);
\draw [color=c, fill=c] (4.54726,2.40806) rectangle (4.58706,2.51391);
\draw [color=c, fill=c] (4.58706,2.40806) rectangle (4.62687,2.51391);
\draw [color=c, fill=c] (4.62687,2.40806) rectangle (4.66667,2.51391);
\draw [color=c, fill=c] (4.66667,2.40806) rectangle (4.70647,2.51391);
\draw [color=c, fill=c] (4.70647,2.40806) rectangle (4.74627,2.51391);
\draw [color=c, fill=c] (4.74627,2.40806) rectangle (4.78607,2.51391);
\draw [color=c, fill=c] (4.78607,2.40806) rectangle (4.82587,2.51391);
\draw [color=c, fill=c] (4.82587,2.40806) rectangle (4.86567,2.51391);
\draw [color=c, fill=c] (4.86567,2.40806) rectangle (4.90547,2.51391);
\draw [color=c, fill=c] (4.90547,2.40806) rectangle (4.94527,2.51391);
\draw [color=c, fill=c] (4.94527,2.40806) rectangle (4.98507,2.51391);
\draw [color=c, fill=c] (4.98507,2.40806) rectangle (5.02488,2.51391);
\draw [color=c, fill=c] (5.02488,2.40806) rectangle (5.06468,2.51391);
\draw [color=c, fill=c] (5.06468,2.40806) rectangle (5.10448,2.51391);
\draw [color=c, fill=c] (5.10448,2.40806) rectangle (5.14428,2.51391);
\draw [color=c, fill=c] (5.14428,2.40806) rectangle (5.18408,2.51391);
\draw [color=c, fill=c] (5.18408,2.40806) rectangle (5.22388,2.51391);
\draw [color=c, fill=c] (5.22388,2.40806) rectangle (5.26368,2.51391);
\draw [color=c, fill=c] (5.26368,2.40806) rectangle (5.30348,2.51391);
\draw [color=c, fill=c] (5.30348,2.40806) rectangle (5.34328,2.51391);
\draw [color=c, fill=c] (5.34328,2.40806) rectangle (5.38308,2.51391);
\draw [color=c, fill=c] (5.38308,2.40806) rectangle (5.42289,2.51391);
\draw [color=c, fill=c] (5.42289,2.40806) rectangle (5.46269,2.51391);
\draw [color=c, fill=c] (5.46269,2.40806) rectangle (5.50249,2.51391);
\draw [color=c, fill=c] (5.50249,2.40806) rectangle (5.54229,2.51391);
\draw [color=c, fill=c] (5.54229,2.40806) rectangle (5.58209,2.51391);
\draw [color=c, fill=c] (5.58209,2.40806) rectangle (5.62189,2.51391);
\draw [color=c, fill=c] (5.62189,2.40806) rectangle (5.66169,2.51391);
\draw [color=c, fill=c] (5.66169,2.40806) rectangle (5.70149,2.51391);
\draw [color=c, fill=c] (5.70149,2.40806) rectangle (5.74129,2.51391);
\draw [color=c, fill=c] (5.74129,2.40806) rectangle (5.78109,2.51391);
\draw [color=c, fill=c] (5.78109,2.40806) rectangle (5.8209,2.51391);
\draw [color=c, fill=c] (5.8209,2.40806) rectangle (5.8607,2.51391);
\draw [color=c, fill=c] (5.8607,2.40806) rectangle (5.9005,2.51391);
\draw [color=c, fill=c] (5.9005,2.40806) rectangle (5.9403,2.51391);
\draw [color=c, fill=c] (5.9403,2.40806) rectangle (5.9801,2.51391);
\draw [color=c, fill=c] (5.9801,2.40806) rectangle (6.0199,2.51391);
\draw [color=c, fill=c] (6.0199,2.40806) rectangle (6.0597,2.51391);
\draw [color=c, fill=c] (6.0597,2.40806) rectangle (6.0995,2.51391);
\draw [color=c, fill=c] (6.0995,2.40806) rectangle (6.1393,2.51391);
\draw [color=c, fill=c] (6.1393,2.40806) rectangle (6.1791,2.51391);
\draw [color=c, fill=c] (6.1791,2.40806) rectangle (6.21891,2.51391);
\draw [color=c, fill=c] (6.21891,2.40806) rectangle (6.25871,2.51391);
\draw [color=c, fill=c] (6.25871,2.40806) rectangle (6.29851,2.51391);
\draw [color=c, fill=c] (6.29851,2.40806) rectangle (6.33831,2.51391);
\draw [color=c, fill=c] (6.33831,2.40806) rectangle (6.37811,2.51391);
\draw [color=c, fill=c] (6.37811,2.40806) rectangle (6.41791,2.51391);
\draw [color=c, fill=c] (6.41791,2.40806) rectangle (6.45771,2.51391);
\draw [color=c, fill=c] (6.45771,2.40806) rectangle (6.49751,2.51391);
\draw [color=c, fill=c] (6.49751,2.40806) rectangle (6.53731,2.51391);
\draw [color=c, fill=c] (6.53731,2.40806) rectangle (6.57711,2.51391);
\draw [color=c, fill=c] (6.57711,2.40806) rectangle (6.61692,2.51391);
\draw [color=c, fill=c] (6.61692,2.40806) rectangle (6.65672,2.51391);
\draw [color=c, fill=c] (6.65672,2.40806) rectangle (6.69652,2.51391);
\draw [color=c, fill=c] (6.69652,2.40806) rectangle (6.73632,2.51391);
\draw [color=c, fill=c] (6.73632,2.40806) rectangle (6.77612,2.51391);
\draw [color=c, fill=c] (6.77612,2.40806) rectangle (6.81592,2.51391);
\draw [color=c, fill=c] (6.81592,2.40806) rectangle (6.85572,2.51391);
\draw [color=c, fill=c] (6.85572,2.40806) rectangle (6.89552,2.51391);
\draw [color=c, fill=c] (6.89552,2.40806) rectangle (6.93532,2.51391);
\draw [color=c, fill=c] (6.93532,2.40806) rectangle (6.97512,2.51391);
\draw [color=c, fill=c] (6.97512,2.40806) rectangle (7.01493,2.51391);
\draw [color=c, fill=c] (7.01493,2.40806) rectangle (7.05473,2.51391);
\draw [color=c, fill=c] (7.05473,2.40806) rectangle (7.09453,2.51391);
\draw [color=c, fill=c] (7.09453,2.40806) rectangle (7.13433,2.51391);
\draw [color=c, fill=c] (7.13433,2.40806) rectangle (7.17413,2.51391);
\draw [color=c, fill=c] (7.17413,2.40806) rectangle (7.21393,2.51391);
\draw [color=c, fill=c] (7.21393,2.40806) rectangle (7.25373,2.51391);
\draw [color=c, fill=c] (7.25373,2.40806) rectangle (7.29353,2.51391);
\draw [color=c, fill=c] (7.29353,2.40806) rectangle (7.33333,2.51391);
\draw [color=c, fill=c] (7.33333,2.40806) rectangle (7.37313,2.51391);
\draw [color=c, fill=c] (7.37313,2.40806) rectangle (7.41294,2.51391);
\draw [color=c, fill=c] (7.41294,2.40806) rectangle (7.45274,2.51391);
\draw [color=c, fill=c] (7.45274,2.40806) rectangle (7.49254,2.51391);
\draw [color=c, fill=c] (7.49254,2.40806) rectangle (7.53234,2.51391);
\draw [color=c, fill=c] (7.53234,2.40806) rectangle (7.57214,2.51391);
\draw [color=c, fill=c] (7.57214,2.40806) rectangle (7.61194,2.51391);
\draw [color=c, fill=c] (7.61194,2.40806) rectangle (7.65174,2.51391);
\draw [color=c, fill=c] (7.65174,2.40806) rectangle (7.69154,2.51391);
\draw [color=c, fill=c] (7.69154,2.40806) rectangle (7.73134,2.51391);
\draw [color=c, fill=c] (7.73134,2.40806) rectangle (7.77114,2.51391);
\draw [color=c, fill=c] (7.77114,2.40806) rectangle (7.81095,2.51391);
\draw [color=c, fill=c] (7.81095,2.40806) rectangle (7.85075,2.51391);
\definecolor{c}{rgb}{1,0.186667,0};
\draw [color=c, fill=c] (7.85075,2.40806) rectangle (7.89055,2.51391);
\draw [color=c, fill=c] (7.89055,2.40806) rectangle (7.93035,2.51391);
\draw [color=c, fill=c] (7.93035,2.40806) rectangle (7.97015,2.51391);
\draw [color=c, fill=c] (7.97015,2.40806) rectangle (8.00995,2.51391);
\draw [color=c, fill=c] (8.00995,2.40806) rectangle (8.04975,2.51391);
\draw [color=c, fill=c] (8.04975,2.40806) rectangle (8.08955,2.51391);
\draw [color=c, fill=c] (8.08955,2.40806) rectangle (8.12935,2.51391);
\draw [color=c, fill=c] (8.12935,2.40806) rectangle (8.16915,2.51391);
\draw [color=c, fill=c] (8.16915,2.40806) rectangle (8.20895,2.51391);
\draw [color=c, fill=c] (8.20895,2.40806) rectangle (8.24876,2.51391);
\draw [color=c, fill=c] (8.24876,2.40806) rectangle (8.28856,2.51391);
\draw [color=c, fill=c] (8.28856,2.40806) rectangle (8.32836,2.51391);
\draw [color=c, fill=c] (8.32836,2.40806) rectangle (8.36816,2.51391);
\draw [color=c, fill=c] (8.36816,2.40806) rectangle (8.40796,2.51391);
\draw [color=c, fill=c] (8.40796,2.40806) rectangle (8.44776,2.51391);
\draw [color=c, fill=c] (8.44776,2.40806) rectangle (8.48756,2.51391);
\draw [color=c, fill=c] (8.48756,2.40806) rectangle (8.52736,2.51391);
\draw [color=c, fill=c] (8.52736,2.40806) rectangle (8.56716,2.51391);
\draw [color=c, fill=c] (8.56716,2.40806) rectangle (8.60697,2.51391);
\draw [color=c, fill=c] (8.60697,2.40806) rectangle (8.64677,2.51391);
\draw [color=c, fill=c] (8.64677,2.40806) rectangle (8.68657,2.51391);
\definecolor{c}{rgb}{1,0.466667,0};
\draw [color=c, fill=c] (8.68657,2.40806) rectangle (8.72637,2.51391);
\draw [color=c, fill=c] (8.72637,2.40806) rectangle (8.76617,2.51391);
\draw [color=c, fill=c] (8.76617,2.40806) rectangle (8.80597,2.51391);
\draw [color=c, fill=c] (8.80597,2.40806) rectangle (8.84577,2.51391);
\draw [color=c, fill=c] (8.84577,2.40806) rectangle (8.88557,2.51391);
\draw [color=c, fill=c] (8.88557,2.40806) rectangle (8.92537,2.51391);
\draw [color=c, fill=c] (8.92537,2.40806) rectangle (8.96517,2.51391);
\draw [color=c, fill=c] (8.96517,2.40806) rectangle (9.00498,2.51391);
\draw [color=c, fill=c] (9.00498,2.40806) rectangle (9.04478,2.51391);
\draw [color=c, fill=c] (9.04478,2.40806) rectangle (9.08458,2.51391);
\draw [color=c, fill=c] (9.08458,2.40806) rectangle (9.12438,2.51391);
\definecolor{c}{rgb}{1,0.653333,0};
\draw [color=c, fill=c] (9.12438,2.40806) rectangle (9.16418,2.51391);
\draw [color=c, fill=c] (9.16418,2.40806) rectangle (9.20398,2.51391);
\draw [color=c, fill=c] (9.20398,2.40806) rectangle (9.24378,2.51391);
\draw [color=c, fill=c] (9.24378,2.40806) rectangle (9.28358,2.51391);
\draw [color=c, fill=c] (9.28358,2.40806) rectangle (9.32338,2.51391);
\draw [color=c, fill=c] (9.32338,2.40806) rectangle (9.36318,2.51391);
\draw [color=c, fill=c] (9.36318,2.40806) rectangle (9.40298,2.51391);
\definecolor{c}{rgb}{1,0.933333,0};
\draw [color=c, fill=c] (9.40298,2.40806) rectangle (9.44279,2.51391);
\draw [color=c, fill=c] (9.44279,2.40806) rectangle (9.48259,2.51391);
\draw [color=c, fill=c] (9.48259,2.40806) rectangle (9.52239,2.51391);
\draw [color=c, fill=c] (9.52239,2.40806) rectangle (9.56219,2.51391);
\draw [color=c, fill=c] (9.56219,2.40806) rectangle (9.60199,2.51391);
\definecolor{c}{rgb}{0.88,1,0};
\draw [color=c, fill=c] (9.60199,2.40806) rectangle (9.64179,2.51391);
\draw [color=c, fill=c] (9.64179,2.40806) rectangle (9.68159,2.51391);
\draw [color=c, fill=c] (9.68159,2.40806) rectangle (9.72139,2.51391);
\draw [color=c, fill=c] (9.72139,2.40806) rectangle (9.76119,2.51391);
\definecolor{c}{rgb}{0.6,1,0};
\draw [color=c, fill=c] (9.76119,2.40806) rectangle (9.80099,2.51391);
\draw [color=c, fill=c] (9.80099,2.40806) rectangle (9.8408,2.51391);
\draw [color=c, fill=c] (9.8408,2.40806) rectangle (9.8806,2.51391);
\draw [color=c, fill=c] (9.8806,2.40806) rectangle (9.9204,2.51391);
\definecolor{c}{rgb}{0.413333,1,0};
\draw [color=c, fill=c] (9.9204,2.40806) rectangle (9.9602,2.51391);
\draw [color=c, fill=c] (9.9602,2.40806) rectangle (10,2.51391);
\draw [color=c, fill=c] (10,2.40806) rectangle (10.0398,2.51391);
\definecolor{c}{rgb}{0.133333,1,0};
\draw [color=c, fill=c] (10.0398,2.40806) rectangle (10.0796,2.51391);
\draw [color=c, fill=c] (10.0796,2.40806) rectangle (10.1194,2.51391);
\draw [color=c, fill=c] (10.1194,2.40806) rectangle (10.1592,2.51391);
\definecolor{c}{rgb}{0,1,0.0533333};
\draw [color=c, fill=c] (10.1592,2.40806) rectangle (10.199,2.51391);
\draw [color=c, fill=c] (10.199,2.40806) rectangle (10.2388,2.51391);
\draw [color=c, fill=c] (10.2388,2.40806) rectangle (10.2786,2.51391);
\draw [color=c, fill=c] (10.2786,2.40806) rectangle (10.3184,2.51391);
\draw [color=c, fill=c] (10.3184,2.40806) rectangle (10.3582,2.51391);
\definecolor{c}{rgb}{0,1,0.333333};
\draw [color=c, fill=c] (10.3582,2.40806) rectangle (10.398,2.51391);
\draw [color=c, fill=c] (10.398,2.40806) rectangle (10.4378,2.51391);
\draw [color=c, fill=c] (10.4378,2.40806) rectangle (10.4776,2.51391);
\draw [color=c, fill=c] (10.4776,2.40806) rectangle (10.5174,2.51391);
\draw [color=c, fill=c] (10.5174,2.40806) rectangle (10.5572,2.51391);
\definecolor{c}{rgb}{0,1,0.52};
\draw [color=c, fill=c] (10.5572,2.40806) rectangle (10.597,2.51391);
\draw [color=c, fill=c] (10.597,2.40806) rectangle (10.6368,2.51391);
\draw [color=c, fill=c] (10.6368,2.40806) rectangle (10.6766,2.51391);
\draw [color=c, fill=c] (10.6766,2.40806) rectangle (10.7164,2.51391);
\draw [color=c, fill=c] (10.7164,2.40806) rectangle (10.7562,2.51391);
\draw [color=c, fill=c] (10.7562,2.40806) rectangle (10.796,2.51391);
\draw [color=c, fill=c] (10.796,2.40806) rectangle (10.8358,2.51391);
\draw [color=c, fill=c] (10.8358,2.40806) rectangle (10.8756,2.51391);
\definecolor{c}{rgb}{0,1,0.8};
\draw [color=c, fill=c] (10.8756,2.40806) rectangle (10.9154,2.51391);
\draw [color=c, fill=c] (10.9154,2.40806) rectangle (10.9552,2.51391);
\draw [color=c, fill=c] (10.9552,2.40806) rectangle (10.995,2.51391);
\draw [color=c, fill=c] (10.995,2.40806) rectangle (11.0348,2.51391);
\draw [color=c, fill=c] (11.0348,2.40806) rectangle (11.0746,2.51391);
\draw [color=c, fill=c] (11.0746,2.40806) rectangle (11.1144,2.51391);
\draw [color=c, fill=c] (11.1144,2.40806) rectangle (11.1542,2.51391);
\draw [color=c, fill=c] (11.1542,2.40806) rectangle (11.194,2.51391);
\draw [color=c, fill=c] (11.194,2.40806) rectangle (11.2338,2.51391);
\draw [color=c, fill=c] (11.2338,2.40806) rectangle (11.2736,2.51391);
\draw [color=c, fill=c] (11.2736,2.40806) rectangle (11.3134,2.51391);
\definecolor{c}{rgb}{0,1,0.986667};
\draw [color=c, fill=c] (11.3134,2.40806) rectangle (11.3532,2.51391);
\draw [color=c, fill=c] (11.3532,2.40806) rectangle (11.393,2.51391);
\draw [color=c, fill=c] (11.393,2.40806) rectangle (11.4328,2.51391);
\draw [color=c, fill=c] (11.4328,2.40806) rectangle (11.4726,2.51391);
\draw [color=c, fill=c] (11.4726,2.40806) rectangle (11.5124,2.51391);
\draw [color=c, fill=c] (11.5124,2.40806) rectangle (11.5522,2.51391);
\draw [color=c, fill=c] (11.5522,2.40806) rectangle (11.592,2.51391);
\draw [color=c, fill=c] (11.592,2.40806) rectangle (11.6318,2.51391);
\draw [color=c, fill=c] (11.6318,2.40806) rectangle (11.6716,2.51391);
\draw [color=c, fill=c] (11.6716,2.40806) rectangle (11.7114,2.51391);
\draw [color=c, fill=c] (11.7114,2.40806) rectangle (11.7512,2.51391);
\draw [color=c, fill=c] (11.7512,2.40806) rectangle (11.791,2.51391);
\draw [color=c, fill=c] (11.791,2.40806) rectangle (11.8308,2.51391);
\draw [color=c, fill=c] (11.8308,2.40806) rectangle (11.8706,2.51391);
\draw [color=c, fill=c] (11.8706,2.40806) rectangle (11.9104,2.51391);
\draw [color=c, fill=c] (11.9104,2.40806) rectangle (11.9502,2.51391);
\draw [color=c, fill=c] (11.9502,2.40806) rectangle (11.99,2.51391);
\draw [color=c, fill=c] (11.99,2.40806) rectangle (12.0299,2.51391);
\draw [color=c, fill=c] (12.0299,2.40806) rectangle (12.0697,2.51391);
\draw [color=c, fill=c] (12.0697,2.40806) rectangle (12.1095,2.51391);
\draw [color=c, fill=c] (12.1095,2.40806) rectangle (12.1493,2.51391);
\draw [color=c, fill=c] (12.1493,2.40806) rectangle (12.1891,2.51391);
\draw [color=c, fill=c] (12.1891,2.40806) rectangle (12.2289,2.51391);
\draw [color=c, fill=c] (12.2289,2.40806) rectangle (12.2687,2.51391);
\definecolor{c}{rgb}{0,0.733333,1};
\draw [color=c, fill=c] (12.2687,2.40806) rectangle (12.3085,2.51391);
\draw [color=c, fill=c] (12.3085,2.40806) rectangle (12.3483,2.51391);
\draw [color=c, fill=c] (12.3483,2.40806) rectangle (12.3881,2.51391);
\draw [color=c, fill=c] (12.3881,2.40806) rectangle (12.4279,2.51391);
\draw [color=c, fill=c] (12.4279,2.40806) rectangle (12.4677,2.51391);
\draw [color=c, fill=c] (12.4677,2.40806) rectangle (12.5075,2.51391);
\draw [color=c, fill=c] (12.5075,2.40806) rectangle (12.5473,2.51391);
\draw [color=c, fill=c] (12.5473,2.40806) rectangle (12.5871,2.51391);
\draw [color=c, fill=c] (12.5871,2.40806) rectangle (12.6269,2.51391);
\draw [color=c, fill=c] (12.6269,2.40806) rectangle (12.6667,2.51391);
\draw [color=c, fill=c] (12.6667,2.40806) rectangle (12.7065,2.51391);
\draw [color=c, fill=c] (12.7065,2.40806) rectangle (12.7463,2.51391);
\draw [color=c, fill=c] (12.7463,2.40806) rectangle (12.7861,2.51391);
\draw [color=c, fill=c] (12.7861,2.40806) rectangle (12.8259,2.51391);
\draw [color=c, fill=c] (12.8259,2.40806) rectangle (12.8657,2.51391);
\draw [color=c, fill=c] (12.8657,2.40806) rectangle (12.9055,2.51391);
\draw [color=c, fill=c] (12.9055,2.40806) rectangle (12.9453,2.51391);
\draw [color=c, fill=c] (12.9453,2.40806) rectangle (12.9851,2.51391);
\draw [color=c, fill=c] (12.9851,2.40806) rectangle (13.0249,2.51391);
\draw [color=c, fill=c] (13.0249,2.40806) rectangle (13.0647,2.51391);
\draw [color=c, fill=c] (13.0647,2.40806) rectangle (13.1045,2.51391);
\draw [color=c, fill=c] (13.1045,2.40806) rectangle (13.1443,2.51391);
\draw [color=c, fill=c] (13.1443,2.40806) rectangle (13.1841,2.51391);
\draw [color=c, fill=c] (13.1841,2.40806) rectangle (13.2239,2.51391);
\draw [color=c, fill=c] (13.2239,2.40806) rectangle (13.2637,2.51391);
\draw [color=c, fill=c] (13.2637,2.40806) rectangle (13.3035,2.51391);
\draw [color=c, fill=c] (13.3035,2.40806) rectangle (13.3433,2.51391);
\draw [color=c, fill=c] (13.3433,2.40806) rectangle (13.3831,2.51391);
\draw [color=c, fill=c] (13.3831,2.40806) rectangle (13.4229,2.51391);
\draw [color=c, fill=c] (13.4229,2.40806) rectangle (13.4627,2.51391);
\draw [color=c, fill=c] (13.4627,2.40806) rectangle (13.5025,2.51391);
\draw [color=c, fill=c] (13.5025,2.40806) rectangle (13.5423,2.51391);
\draw [color=c, fill=c] (13.5423,2.40806) rectangle (13.5821,2.51391);
\draw [color=c, fill=c] (13.5821,2.40806) rectangle (13.6219,2.51391);
\draw [color=c, fill=c] (13.6219,2.40806) rectangle (13.6617,2.51391);
\draw [color=c, fill=c] (13.6617,2.40806) rectangle (13.7015,2.51391);
\draw [color=c, fill=c] (13.7015,2.40806) rectangle (13.7413,2.51391);
\draw [color=c, fill=c] (13.7413,2.40806) rectangle (13.7811,2.51391);
\draw [color=c, fill=c] (13.7811,2.40806) rectangle (13.8209,2.51391);
\draw [color=c, fill=c] (13.8209,2.40806) rectangle (13.8607,2.51391);
\draw [color=c, fill=c] (13.8607,2.40806) rectangle (13.9005,2.51391);
\draw [color=c, fill=c] (13.9005,2.40806) rectangle (13.9403,2.51391);
\draw [color=c, fill=c] (13.9403,2.40806) rectangle (13.9801,2.51391);
\draw [color=c, fill=c] (13.9801,2.40806) rectangle (14.0199,2.51391);
\draw [color=c, fill=c] (14.0199,2.40806) rectangle (14.0597,2.51391);
\draw [color=c, fill=c] (14.0597,2.40806) rectangle (14.0995,2.51391);
\draw [color=c, fill=c] (14.0995,2.40806) rectangle (14.1393,2.51391);
\draw [color=c, fill=c] (14.1393,2.40806) rectangle (14.1791,2.51391);
\draw [color=c, fill=c] (14.1791,2.40806) rectangle (14.2189,2.51391);
\draw [color=c, fill=c] (14.2189,2.40806) rectangle (14.2587,2.51391);
\draw [color=c, fill=c] (14.2587,2.40806) rectangle (14.2985,2.51391);
\draw [color=c, fill=c] (14.2985,2.40806) rectangle (14.3383,2.51391);
\draw [color=c, fill=c] (14.3383,2.40806) rectangle (14.3781,2.51391);
\draw [color=c, fill=c] (14.3781,2.40806) rectangle (14.4179,2.51391);
\draw [color=c, fill=c] (14.4179,2.40806) rectangle (14.4577,2.51391);
\draw [color=c, fill=c] (14.4577,2.40806) rectangle (14.4975,2.51391);
\draw [color=c, fill=c] (14.4975,2.40806) rectangle (14.5373,2.51391);
\draw [color=c, fill=c] (14.5373,2.40806) rectangle (14.5771,2.51391);
\draw [color=c, fill=c] (14.5771,2.40806) rectangle (14.6169,2.51391);
\draw [color=c, fill=c] (14.6169,2.40806) rectangle (14.6567,2.51391);
\draw [color=c, fill=c] (14.6567,2.40806) rectangle (14.6965,2.51391);
\draw [color=c, fill=c] (14.6965,2.40806) rectangle (14.7363,2.51391);
\draw [color=c, fill=c] (14.7363,2.40806) rectangle (14.7761,2.51391);
\draw [color=c, fill=c] (14.7761,2.40806) rectangle (14.8159,2.51391);
\draw [color=c, fill=c] (14.8159,2.40806) rectangle (14.8557,2.51391);
\draw [color=c, fill=c] (14.8557,2.40806) rectangle (14.8955,2.51391);
\draw [color=c, fill=c] (14.8955,2.40806) rectangle (14.9353,2.51391);
\draw [color=c, fill=c] (14.9353,2.40806) rectangle (14.9751,2.51391);
\draw [color=c, fill=c] (14.9751,2.40806) rectangle (15.0149,2.51391);
\draw [color=c, fill=c] (15.0149,2.40806) rectangle (15.0547,2.51391);
\draw [color=c, fill=c] (15.0547,2.40806) rectangle (15.0945,2.51391);
\draw [color=c, fill=c] (15.0945,2.40806) rectangle (15.1343,2.51391);
\draw [color=c, fill=c] (15.1343,2.40806) rectangle (15.1741,2.51391);
\draw [color=c, fill=c] (15.1741,2.40806) rectangle (15.2139,2.51391);
\draw [color=c, fill=c] (15.2139,2.40806) rectangle (15.2537,2.51391);
\draw [color=c, fill=c] (15.2537,2.40806) rectangle (15.2935,2.51391);
\draw [color=c, fill=c] (15.2935,2.40806) rectangle (15.3333,2.51391);
\draw [color=c, fill=c] (15.3333,2.40806) rectangle (15.3731,2.51391);
\draw [color=c, fill=c] (15.3731,2.40806) rectangle (15.4129,2.51391);
\draw [color=c, fill=c] (15.4129,2.40806) rectangle (15.4527,2.51391);
\draw [color=c, fill=c] (15.4527,2.40806) rectangle (15.4925,2.51391);
\draw [color=c, fill=c] (15.4925,2.40806) rectangle (15.5323,2.51391);
\draw [color=c, fill=c] (15.5323,2.40806) rectangle (15.5721,2.51391);
\draw [color=c, fill=c] (15.5721,2.40806) rectangle (15.6119,2.51391);
\draw [color=c, fill=c] (15.6119,2.40806) rectangle (15.6517,2.51391);
\draw [color=c, fill=c] (15.6517,2.40806) rectangle (15.6915,2.51391);
\draw [color=c, fill=c] (15.6915,2.40806) rectangle (15.7313,2.51391);
\draw [color=c, fill=c] (15.7313,2.40806) rectangle (15.7711,2.51391);
\draw [color=c, fill=c] (15.7711,2.40806) rectangle (15.8109,2.51391);
\draw [color=c, fill=c] (15.8109,2.40806) rectangle (15.8507,2.51391);
\draw [color=c, fill=c] (15.8507,2.40806) rectangle (15.8905,2.51391);
\draw [color=c, fill=c] (15.8905,2.40806) rectangle (15.9303,2.51391);
\draw [color=c, fill=c] (15.9303,2.40806) rectangle (15.9701,2.51391);
\draw [color=c, fill=c] (15.9701,2.40806) rectangle (16.01,2.51391);
\draw [color=c, fill=c] (16.01,2.40806) rectangle (16.0498,2.51391);
\draw [color=c, fill=c] (16.0498,2.40806) rectangle (16.0896,2.51391);
\draw [color=c, fill=c] (16.0896,2.40806) rectangle (16.1294,2.51391);
\draw [color=c, fill=c] (16.1294,2.40806) rectangle (16.1692,2.51391);
\draw [color=c, fill=c] (16.1692,2.40806) rectangle (16.209,2.51391);
\draw [color=c, fill=c] (16.209,2.40806) rectangle (16.2488,2.51391);
\draw [color=c, fill=c] (16.2488,2.40806) rectangle (16.2886,2.51391);
\draw [color=c, fill=c] (16.2886,2.40806) rectangle (16.3284,2.51391);
\draw [color=c, fill=c] (16.3284,2.40806) rectangle (16.3682,2.51391);
\draw [color=c, fill=c] (16.3682,2.40806) rectangle (16.408,2.51391);
\draw [color=c, fill=c] (16.408,2.40806) rectangle (16.4478,2.51391);
\draw [color=c, fill=c] (16.4478,2.40806) rectangle (16.4876,2.51391);
\draw [color=c, fill=c] (16.4876,2.40806) rectangle (16.5274,2.51391);
\draw [color=c, fill=c] (16.5274,2.40806) rectangle (16.5672,2.51391);
\draw [color=c, fill=c] (16.5672,2.40806) rectangle (16.607,2.51391);
\draw [color=c, fill=c] (16.607,2.40806) rectangle (16.6468,2.51391);
\draw [color=c, fill=c] (16.6468,2.40806) rectangle (16.6866,2.51391);
\draw [color=c, fill=c] (16.6866,2.40806) rectangle (16.7264,2.51391);
\draw [color=c, fill=c] (16.7264,2.40806) rectangle (16.7662,2.51391);
\draw [color=c, fill=c] (16.7662,2.40806) rectangle (16.806,2.51391);
\draw [color=c, fill=c] (16.806,2.40806) rectangle (16.8458,2.51391);
\draw [color=c, fill=c] (16.8458,2.40806) rectangle (16.8856,2.51391);
\draw [color=c, fill=c] (16.8856,2.40806) rectangle (16.9254,2.51391);
\draw [color=c, fill=c] (16.9254,2.40806) rectangle (16.9652,2.51391);
\draw [color=c, fill=c] (16.9652,2.40806) rectangle (17.005,2.51391);
\draw [color=c, fill=c] (17.005,2.40806) rectangle (17.0448,2.51391);
\draw [color=c, fill=c] (17.0448,2.40806) rectangle (17.0846,2.51391);
\draw [color=c, fill=c] (17.0846,2.40806) rectangle (17.1244,2.51391);
\draw [color=c, fill=c] (17.1244,2.40806) rectangle (17.1642,2.51391);
\draw [color=c, fill=c] (17.1642,2.40806) rectangle (17.204,2.51391);
\draw [color=c, fill=c] (17.204,2.40806) rectangle (17.2438,2.51391);
\draw [color=c, fill=c] (17.2438,2.40806) rectangle (17.2836,2.51391);
\draw [color=c, fill=c] (17.2836,2.40806) rectangle (17.3234,2.51391);
\draw [color=c, fill=c] (17.3234,2.40806) rectangle (17.3632,2.51391);
\draw [color=c, fill=c] (17.3632,2.40806) rectangle (17.403,2.51391);
\draw [color=c, fill=c] (17.403,2.40806) rectangle (17.4428,2.51391);
\draw [color=c, fill=c] (17.4428,2.40806) rectangle (17.4826,2.51391);
\draw [color=c, fill=c] (17.4826,2.40806) rectangle (17.5224,2.51391);
\draw [color=c, fill=c] (17.5224,2.40806) rectangle (17.5622,2.51391);
\draw [color=c, fill=c] (17.5622,2.40806) rectangle (17.602,2.51391);
\draw [color=c, fill=c] (17.602,2.40806) rectangle (17.6418,2.51391);
\draw [color=c, fill=c] (17.6418,2.40806) rectangle (17.6816,2.51391);
\draw [color=c, fill=c] (17.6816,2.40806) rectangle (17.7214,2.51391);
\draw [color=c, fill=c] (17.7214,2.40806) rectangle (17.7612,2.51391);
\draw [color=c, fill=c] (17.7612,2.40806) rectangle (17.801,2.51391);
\draw [color=c, fill=c] (17.801,2.40806) rectangle (17.8408,2.51391);
\draw [color=c, fill=c] (17.8408,2.40806) rectangle (17.8806,2.51391);
\draw [color=c, fill=c] (17.8806,2.40806) rectangle (17.9204,2.51391);
\draw [color=c, fill=c] (17.9204,2.40806) rectangle (17.9602,2.51391);
\draw [color=c, fill=c] (17.9602,2.40806) rectangle (18,2.51391);
\definecolor{c}{rgb}{1,0,0};
\draw [color=c, fill=c] (2,2.51391) rectangle (2.0398,2.61975);
\draw [color=c, fill=c] (2.0398,2.51391) rectangle (2.0796,2.61975);
\draw [color=c, fill=c] (2.0796,2.51391) rectangle (2.1194,2.61975);
\draw [color=c, fill=c] (2.1194,2.51391) rectangle (2.1592,2.61975);
\draw [color=c, fill=c] (2.1592,2.51391) rectangle (2.19901,2.61975);
\draw [color=c, fill=c] (2.19901,2.51391) rectangle (2.23881,2.61975);
\draw [color=c, fill=c] (2.23881,2.51391) rectangle (2.27861,2.61975);
\draw [color=c, fill=c] (2.27861,2.51391) rectangle (2.31841,2.61975);
\draw [color=c, fill=c] (2.31841,2.51391) rectangle (2.35821,2.61975);
\draw [color=c, fill=c] (2.35821,2.51391) rectangle (2.39801,2.61975);
\draw [color=c, fill=c] (2.39801,2.51391) rectangle (2.43781,2.61975);
\draw [color=c, fill=c] (2.43781,2.51391) rectangle (2.47761,2.61975);
\draw [color=c, fill=c] (2.47761,2.51391) rectangle (2.51741,2.61975);
\draw [color=c, fill=c] (2.51741,2.51391) rectangle (2.55721,2.61975);
\draw [color=c, fill=c] (2.55721,2.51391) rectangle (2.59702,2.61975);
\draw [color=c, fill=c] (2.59702,2.51391) rectangle (2.63682,2.61975);
\draw [color=c, fill=c] (2.63682,2.51391) rectangle (2.67662,2.61975);
\draw [color=c, fill=c] (2.67662,2.51391) rectangle (2.71642,2.61975);
\draw [color=c, fill=c] (2.71642,2.51391) rectangle (2.75622,2.61975);
\draw [color=c, fill=c] (2.75622,2.51391) rectangle (2.79602,2.61975);
\draw [color=c, fill=c] (2.79602,2.51391) rectangle (2.83582,2.61975);
\draw [color=c, fill=c] (2.83582,2.51391) rectangle (2.87562,2.61975);
\draw [color=c, fill=c] (2.87562,2.51391) rectangle (2.91542,2.61975);
\draw [color=c, fill=c] (2.91542,2.51391) rectangle (2.95522,2.61975);
\draw [color=c, fill=c] (2.95522,2.51391) rectangle (2.99502,2.61975);
\draw [color=c, fill=c] (2.99502,2.51391) rectangle (3.03483,2.61975);
\draw [color=c, fill=c] (3.03483,2.51391) rectangle (3.07463,2.61975);
\draw [color=c, fill=c] (3.07463,2.51391) rectangle (3.11443,2.61975);
\draw [color=c, fill=c] (3.11443,2.51391) rectangle (3.15423,2.61975);
\draw [color=c, fill=c] (3.15423,2.51391) rectangle (3.19403,2.61975);
\draw [color=c, fill=c] (3.19403,2.51391) rectangle (3.23383,2.61975);
\draw [color=c, fill=c] (3.23383,2.51391) rectangle (3.27363,2.61975);
\draw [color=c, fill=c] (3.27363,2.51391) rectangle (3.31343,2.61975);
\draw [color=c, fill=c] (3.31343,2.51391) rectangle (3.35323,2.61975);
\draw [color=c, fill=c] (3.35323,2.51391) rectangle (3.39303,2.61975);
\draw [color=c, fill=c] (3.39303,2.51391) rectangle (3.43284,2.61975);
\draw [color=c, fill=c] (3.43284,2.51391) rectangle (3.47264,2.61975);
\draw [color=c, fill=c] (3.47264,2.51391) rectangle (3.51244,2.61975);
\draw [color=c, fill=c] (3.51244,2.51391) rectangle (3.55224,2.61975);
\draw [color=c, fill=c] (3.55224,2.51391) rectangle (3.59204,2.61975);
\draw [color=c, fill=c] (3.59204,2.51391) rectangle (3.63184,2.61975);
\draw [color=c, fill=c] (3.63184,2.51391) rectangle (3.67164,2.61975);
\draw [color=c, fill=c] (3.67164,2.51391) rectangle (3.71144,2.61975);
\draw [color=c, fill=c] (3.71144,2.51391) rectangle (3.75124,2.61975);
\draw [color=c, fill=c] (3.75124,2.51391) rectangle (3.79104,2.61975);
\draw [color=c, fill=c] (3.79104,2.51391) rectangle (3.83085,2.61975);
\draw [color=c, fill=c] (3.83085,2.51391) rectangle (3.87065,2.61975);
\draw [color=c, fill=c] (3.87065,2.51391) rectangle (3.91045,2.61975);
\draw [color=c, fill=c] (3.91045,2.51391) rectangle (3.95025,2.61975);
\draw [color=c, fill=c] (3.95025,2.51391) rectangle (3.99005,2.61975);
\draw [color=c, fill=c] (3.99005,2.51391) rectangle (4.02985,2.61975);
\draw [color=c, fill=c] (4.02985,2.51391) rectangle (4.06965,2.61975);
\draw [color=c, fill=c] (4.06965,2.51391) rectangle (4.10945,2.61975);
\draw [color=c, fill=c] (4.10945,2.51391) rectangle (4.14925,2.61975);
\draw [color=c, fill=c] (4.14925,2.51391) rectangle (4.18905,2.61975);
\draw [color=c, fill=c] (4.18905,2.51391) rectangle (4.22886,2.61975);
\draw [color=c, fill=c] (4.22886,2.51391) rectangle (4.26866,2.61975);
\draw [color=c, fill=c] (4.26866,2.51391) rectangle (4.30846,2.61975);
\draw [color=c, fill=c] (4.30846,2.51391) rectangle (4.34826,2.61975);
\draw [color=c, fill=c] (4.34826,2.51391) rectangle (4.38806,2.61975);
\draw [color=c, fill=c] (4.38806,2.51391) rectangle (4.42786,2.61975);
\draw [color=c, fill=c] (4.42786,2.51391) rectangle (4.46766,2.61975);
\draw [color=c, fill=c] (4.46766,2.51391) rectangle (4.50746,2.61975);
\draw [color=c, fill=c] (4.50746,2.51391) rectangle (4.54726,2.61975);
\draw [color=c, fill=c] (4.54726,2.51391) rectangle (4.58706,2.61975);
\draw [color=c, fill=c] (4.58706,2.51391) rectangle (4.62687,2.61975);
\draw [color=c, fill=c] (4.62687,2.51391) rectangle (4.66667,2.61975);
\draw [color=c, fill=c] (4.66667,2.51391) rectangle (4.70647,2.61975);
\draw [color=c, fill=c] (4.70647,2.51391) rectangle (4.74627,2.61975);
\draw [color=c, fill=c] (4.74627,2.51391) rectangle (4.78607,2.61975);
\draw [color=c, fill=c] (4.78607,2.51391) rectangle (4.82587,2.61975);
\draw [color=c, fill=c] (4.82587,2.51391) rectangle (4.86567,2.61975);
\draw [color=c, fill=c] (4.86567,2.51391) rectangle (4.90547,2.61975);
\draw [color=c, fill=c] (4.90547,2.51391) rectangle (4.94527,2.61975);
\draw [color=c, fill=c] (4.94527,2.51391) rectangle (4.98507,2.61975);
\draw [color=c, fill=c] (4.98507,2.51391) rectangle (5.02488,2.61975);
\draw [color=c, fill=c] (5.02488,2.51391) rectangle (5.06468,2.61975);
\draw [color=c, fill=c] (5.06468,2.51391) rectangle (5.10448,2.61975);
\draw [color=c, fill=c] (5.10448,2.51391) rectangle (5.14428,2.61975);
\draw [color=c, fill=c] (5.14428,2.51391) rectangle (5.18408,2.61975);
\draw [color=c, fill=c] (5.18408,2.51391) rectangle (5.22388,2.61975);
\draw [color=c, fill=c] (5.22388,2.51391) rectangle (5.26368,2.61975);
\draw [color=c, fill=c] (5.26368,2.51391) rectangle (5.30348,2.61975);
\draw [color=c, fill=c] (5.30348,2.51391) rectangle (5.34328,2.61975);
\draw [color=c, fill=c] (5.34328,2.51391) rectangle (5.38308,2.61975);
\draw [color=c, fill=c] (5.38308,2.51391) rectangle (5.42289,2.61975);
\draw [color=c, fill=c] (5.42289,2.51391) rectangle (5.46269,2.61975);
\draw [color=c, fill=c] (5.46269,2.51391) rectangle (5.50249,2.61975);
\draw [color=c, fill=c] (5.50249,2.51391) rectangle (5.54229,2.61975);
\draw [color=c, fill=c] (5.54229,2.51391) rectangle (5.58209,2.61975);
\draw [color=c, fill=c] (5.58209,2.51391) rectangle (5.62189,2.61975);
\draw [color=c, fill=c] (5.62189,2.51391) rectangle (5.66169,2.61975);
\draw [color=c, fill=c] (5.66169,2.51391) rectangle (5.70149,2.61975);
\draw [color=c, fill=c] (5.70149,2.51391) rectangle (5.74129,2.61975);
\draw [color=c, fill=c] (5.74129,2.51391) rectangle (5.78109,2.61975);
\draw [color=c, fill=c] (5.78109,2.51391) rectangle (5.8209,2.61975);
\draw [color=c, fill=c] (5.8209,2.51391) rectangle (5.8607,2.61975);
\draw [color=c, fill=c] (5.8607,2.51391) rectangle (5.9005,2.61975);
\draw [color=c, fill=c] (5.9005,2.51391) rectangle (5.9403,2.61975);
\draw [color=c, fill=c] (5.9403,2.51391) rectangle (5.9801,2.61975);
\draw [color=c, fill=c] (5.9801,2.51391) rectangle (6.0199,2.61975);
\draw [color=c, fill=c] (6.0199,2.51391) rectangle (6.0597,2.61975);
\draw [color=c, fill=c] (6.0597,2.51391) rectangle (6.0995,2.61975);
\draw [color=c, fill=c] (6.0995,2.51391) rectangle (6.1393,2.61975);
\draw [color=c, fill=c] (6.1393,2.51391) rectangle (6.1791,2.61975);
\draw [color=c, fill=c] (6.1791,2.51391) rectangle (6.21891,2.61975);
\draw [color=c, fill=c] (6.21891,2.51391) rectangle (6.25871,2.61975);
\draw [color=c, fill=c] (6.25871,2.51391) rectangle (6.29851,2.61975);
\draw [color=c, fill=c] (6.29851,2.51391) rectangle (6.33831,2.61975);
\draw [color=c, fill=c] (6.33831,2.51391) rectangle (6.37811,2.61975);
\draw [color=c, fill=c] (6.37811,2.51391) rectangle (6.41791,2.61975);
\draw [color=c, fill=c] (6.41791,2.51391) rectangle (6.45771,2.61975);
\draw [color=c, fill=c] (6.45771,2.51391) rectangle (6.49751,2.61975);
\draw [color=c, fill=c] (6.49751,2.51391) rectangle (6.53731,2.61975);
\draw [color=c, fill=c] (6.53731,2.51391) rectangle (6.57711,2.61975);
\draw [color=c, fill=c] (6.57711,2.51391) rectangle (6.61692,2.61975);
\draw [color=c, fill=c] (6.61692,2.51391) rectangle (6.65672,2.61975);
\draw [color=c, fill=c] (6.65672,2.51391) rectangle (6.69652,2.61975);
\draw [color=c, fill=c] (6.69652,2.51391) rectangle (6.73632,2.61975);
\draw [color=c, fill=c] (6.73632,2.51391) rectangle (6.77612,2.61975);
\draw [color=c, fill=c] (6.77612,2.51391) rectangle (6.81592,2.61975);
\draw [color=c, fill=c] (6.81592,2.51391) rectangle (6.85572,2.61975);
\draw [color=c, fill=c] (6.85572,2.51391) rectangle (6.89552,2.61975);
\draw [color=c, fill=c] (6.89552,2.51391) rectangle (6.93532,2.61975);
\draw [color=c, fill=c] (6.93532,2.51391) rectangle (6.97512,2.61975);
\draw [color=c, fill=c] (6.97512,2.51391) rectangle (7.01493,2.61975);
\draw [color=c, fill=c] (7.01493,2.51391) rectangle (7.05473,2.61975);
\draw [color=c, fill=c] (7.05473,2.51391) rectangle (7.09453,2.61975);
\draw [color=c, fill=c] (7.09453,2.51391) rectangle (7.13433,2.61975);
\draw [color=c, fill=c] (7.13433,2.51391) rectangle (7.17413,2.61975);
\draw [color=c, fill=c] (7.17413,2.51391) rectangle (7.21393,2.61975);
\draw [color=c, fill=c] (7.21393,2.51391) rectangle (7.25373,2.61975);
\draw [color=c, fill=c] (7.25373,2.51391) rectangle (7.29353,2.61975);
\draw [color=c, fill=c] (7.29353,2.51391) rectangle (7.33333,2.61975);
\draw [color=c, fill=c] (7.33333,2.51391) rectangle (7.37313,2.61975);
\draw [color=c, fill=c] (7.37313,2.51391) rectangle (7.41294,2.61975);
\draw [color=c, fill=c] (7.41294,2.51391) rectangle (7.45274,2.61975);
\draw [color=c, fill=c] (7.45274,2.51391) rectangle (7.49254,2.61975);
\draw [color=c, fill=c] (7.49254,2.51391) rectangle (7.53234,2.61975);
\draw [color=c, fill=c] (7.53234,2.51391) rectangle (7.57214,2.61975);
\draw [color=c, fill=c] (7.57214,2.51391) rectangle (7.61194,2.61975);
\draw [color=c, fill=c] (7.61194,2.51391) rectangle (7.65174,2.61975);
\draw [color=c, fill=c] (7.65174,2.51391) rectangle (7.69154,2.61975);
\draw [color=c, fill=c] (7.69154,2.51391) rectangle (7.73134,2.61975);
\draw [color=c, fill=c] (7.73134,2.51391) rectangle (7.77114,2.61975);
\draw [color=c, fill=c] (7.77114,2.51391) rectangle (7.81095,2.61975);
\draw [color=c, fill=c] (7.81095,2.51391) rectangle (7.85075,2.61975);
\definecolor{c}{rgb}{1,0.186667,0};
\draw [color=c, fill=c] (7.85075,2.51391) rectangle (7.89055,2.61975);
\draw [color=c, fill=c] (7.89055,2.51391) rectangle (7.93035,2.61975);
\draw [color=c, fill=c] (7.93035,2.51391) rectangle (7.97015,2.61975);
\draw [color=c, fill=c] (7.97015,2.51391) rectangle (8.00995,2.61975);
\draw [color=c, fill=c] (8.00995,2.51391) rectangle (8.04975,2.61975);
\draw [color=c, fill=c] (8.04975,2.51391) rectangle (8.08955,2.61975);
\draw [color=c, fill=c] (8.08955,2.51391) rectangle (8.12935,2.61975);
\draw [color=c, fill=c] (8.12935,2.51391) rectangle (8.16915,2.61975);
\draw [color=c, fill=c] (8.16915,2.51391) rectangle (8.20895,2.61975);
\draw [color=c, fill=c] (8.20895,2.51391) rectangle (8.24876,2.61975);
\draw [color=c, fill=c] (8.24876,2.51391) rectangle (8.28856,2.61975);
\draw [color=c, fill=c] (8.28856,2.51391) rectangle (8.32836,2.61975);
\draw [color=c, fill=c] (8.32836,2.51391) rectangle (8.36816,2.61975);
\draw [color=c, fill=c] (8.36816,2.51391) rectangle (8.40796,2.61975);
\draw [color=c, fill=c] (8.40796,2.51391) rectangle (8.44776,2.61975);
\draw [color=c, fill=c] (8.44776,2.51391) rectangle (8.48756,2.61975);
\draw [color=c, fill=c] (8.48756,2.51391) rectangle (8.52736,2.61975);
\draw [color=c, fill=c] (8.52736,2.51391) rectangle (8.56716,2.61975);
\draw [color=c, fill=c] (8.56716,2.51391) rectangle (8.60697,2.61975);
\draw [color=c, fill=c] (8.60697,2.51391) rectangle (8.64677,2.61975);
\draw [color=c, fill=c] (8.64677,2.51391) rectangle (8.68657,2.61975);
\definecolor{c}{rgb}{1,0.466667,0};
\draw [color=c, fill=c] (8.68657,2.51391) rectangle (8.72637,2.61975);
\draw [color=c, fill=c] (8.72637,2.51391) rectangle (8.76617,2.61975);
\draw [color=c, fill=c] (8.76617,2.51391) rectangle (8.80597,2.61975);
\draw [color=c, fill=c] (8.80597,2.51391) rectangle (8.84577,2.61975);
\draw [color=c, fill=c] (8.84577,2.51391) rectangle (8.88557,2.61975);
\draw [color=c, fill=c] (8.88557,2.51391) rectangle (8.92537,2.61975);
\draw [color=c, fill=c] (8.92537,2.51391) rectangle (8.96517,2.61975);
\draw [color=c, fill=c] (8.96517,2.51391) rectangle (9.00498,2.61975);
\draw [color=c, fill=c] (9.00498,2.51391) rectangle (9.04478,2.61975);
\draw [color=c, fill=c] (9.04478,2.51391) rectangle (9.08458,2.61975);
\draw [color=c, fill=c] (9.08458,2.51391) rectangle (9.12438,2.61975);
\definecolor{c}{rgb}{1,0.653333,0};
\draw [color=c, fill=c] (9.12438,2.51391) rectangle (9.16418,2.61975);
\draw [color=c, fill=c] (9.16418,2.51391) rectangle (9.20398,2.61975);
\draw [color=c, fill=c] (9.20398,2.51391) rectangle (9.24378,2.61975);
\draw [color=c, fill=c] (9.24378,2.51391) rectangle (9.28358,2.61975);
\draw [color=c, fill=c] (9.28358,2.51391) rectangle (9.32338,2.61975);
\draw [color=c, fill=c] (9.32338,2.51391) rectangle (9.36318,2.61975);
\draw [color=c, fill=c] (9.36318,2.51391) rectangle (9.40298,2.61975);
\definecolor{c}{rgb}{1,0.933333,0};
\draw [color=c, fill=c] (9.40298,2.51391) rectangle (9.44279,2.61975);
\draw [color=c, fill=c] (9.44279,2.51391) rectangle (9.48259,2.61975);
\draw [color=c, fill=c] (9.48259,2.51391) rectangle (9.52239,2.61975);
\draw [color=c, fill=c] (9.52239,2.51391) rectangle (9.56219,2.61975);
\draw [color=c, fill=c] (9.56219,2.51391) rectangle (9.60199,2.61975);
\definecolor{c}{rgb}{0.88,1,0};
\draw [color=c, fill=c] (9.60199,2.51391) rectangle (9.64179,2.61975);
\draw [color=c, fill=c] (9.64179,2.51391) rectangle (9.68159,2.61975);
\draw [color=c, fill=c] (9.68159,2.51391) rectangle (9.72139,2.61975);
\draw [color=c, fill=c] (9.72139,2.51391) rectangle (9.76119,2.61975);
\definecolor{c}{rgb}{0.6,1,0};
\draw [color=c, fill=c] (9.76119,2.51391) rectangle (9.80099,2.61975);
\draw [color=c, fill=c] (9.80099,2.51391) rectangle (9.8408,2.61975);
\draw [color=c, fill=c] (9.8408,2.51391) rectangle (9.8806,2.61975);
\draw [color=c, fill=c] (9.8806,2.51391) rectangle (9.9204,2.61975);
\definecolor{c}{rgb}{0.413333,1,0};
\draw [color=c, fill=c] (9.9204,2.51391) rectangle (9.9602,2.61975);
\draw [color=c, fill=c] (9.9602,2.51391) rectangle (10,2.61975);
\draw [color=c, fill=c] (10,2.51391) rectangle (10.0398,2.61975);
\definecolor{c}{rgb}{0.133333,1,0};
\draw [color=c, fill=c] (10.0398,2.51391) rectangle (10.0796,2.61975);
\draw [color=c, fill=c] (10.0796,2.51391) rectangle (10.1194,2.61975);
\draw [color=c, fill=c] (10.1194,2.51391) rectangle (10.1592,2.61975);
\definecolor{c}{rgb}{0,1,0.0533333};
\draw [color=c, fill=c] (10.1592,2.51391) rectangle (10.199,2.61975);
\draw [color=c, fill=c] (10.199,2.51391) rectangle (10.2388,2.61975);
\draw [color=c, fill=c] (10.2388,2.51391) rectangle (10.2786,2.61975);
\draw [color=c, fill=c] (10.2786,2.51391) rectangle (10.3184,2.61975);
\definecolor{c}{rgb}{0,1,0.333333};
\draw [color=c, fill=c] (10.3184,2.51391) rectangle (10.3582,2.61975);
\draw [color=c, fill=c] (10.3582,2.51391) rectangle (10.398,2.61975);
\draw [color=c, fill=c] (10.398,2.51391) rectangle (10.4378,2.61975);
\draw [color=c, fill=c] (10.4378,2.51391) rectangle (10.4776,2.61975);
\draw [color=c, fill=c] (10.4776,2.51391) rectangle (10.5174,2.61975);
\definecolor{c}{rgb}{0,1,0.52};
\draw [color=c, fill=c] (10.5174,2.51391) rectangle (10.5572,2.61975);
\draw [color=c, fill=c] (10.5572,2.51391) rectangle (10.597,2.61975);
\draw [color=c, fill=c] (10.597,2.51391) rectangle (10.6368,2.61975);
\draw [color=c, fill=c] (10.6368,2.51391) rectangle (10.6766,2.61975);
\draw [color=c, fill=c] (10.6766,2.51391) rectangle (10.7164,2.61975);
\draw [color=c, fill=c] (10.7164,2.51391) rectangle (10.7562,2.61975);
\draw [color=c, fill=c] (10.7562,2.51391) rectangle (10.796,2.61975);
\draw [color=c, fill=c] (10.796,2.51391) rectangle (10.8358,2.61975);
\definecolor{c}{rgb}{0,1,0.8};
\draw [color=c, fill=c] (10.8358,2.51391) rectangle (10.8756,2.61975);
\draw [color=c, fill=c] (10.8756,2.51391) rectangle (10.9154,2.61975);
\draw [color=c, fill=c] (10.9154,2.51391) rectangle (10.9552,2.61975);
\draw [color=c, fill=c] (10.9552,2.51391) rectangle (10.995,2.61975);
\draw [color=c, fill=c] (10.995,2.51391) rectangle (11.0348,2.61975);
\draw [color=c, fill=c] (11.0348,2.51391) rectangle (11.0746,2.61975);
\draw [color=c, fill=c] (11.0746,2.51391) rectangle (11.1144,2.61975);
\draw [color=c, fill=c] (11.1144,2.51391) rectangle (11.1542,2.61975);
\draw [color=c, fill=c] (11.1542,2.51391) rectangle (11.194,2.61975);
\draw [color=c, fill=c] (11.194,2.51391) rectangle (11.2338,2.61975);
\draw [color=c, fill=c] (11.2338,2.51391) rectangle (11.2736,2.61975);
\draw [color=c, fill=c] (11.2736,2.51391) rectangle (11.3134,2.61975);
\definecolor{c}{rgb}{0,1,0.986667};
\draw [color=c, fill=c] (11.3134,2.51391) rectangle (11.3532,2.61975);
\draw [color=c, fill=c] (11.3532,2.51391) rectangle (11.393,2.61975);
\draw [color=c, fill=c] (11.393,2.51391) rectangle (11.4328,2.61975);
\draw [color=c, fill=c] (11.4328,2.51391) rectangle (11.4726,2.61975);
\draw [color=c, fill=c] (11.4726,2.51391) rectangle (11.5124,2.61975);
\draw [color=c, fill=c] (11.5124,2.51391) rectangle (11.5522,2.61975);
\draw [color=c, fill=c] (11.5522,2.51391) rectangle (11.592,2.61975);
\draw [color=c, fill=c] (11.592,2.51391) rectangle (11.6318,2.61975);
\draw [color=c, fill=c] (11.6318,2.51391) rectangle (11.6716,2.61975);
\draw [color=c, fill=c] (11.6716,2.51391) rectangle (11.7114,2.61975);
\draw [color=c, fill=c] (11.7114,2.51391) rectangle (11.7512,2.61975);
\draw [color=c, fill=c] (11.7512,2.51391) rectangle (11.791,2.61975);
\draw [color=c, fill=c] (11.791,2.51391) rectangle (11.8308,2.61975);
\draw [color=c, fill=c] (11.8308,2.51391) rectangle (11.8706,2.61975);
\draw [color=c, fill=c] (11.8706,2.51391) rectangle (11.9104,2.61975);
\draw [color=c, fill=c] (11.9104,2.51391) rectangle (11.9502,2.61975);
\draw [color=c, fill=c] (11.9502,2.51391) rectangle (11.99,2.61975);
\draw [color=c, fill=c] (11.99,2.51391) rectangle (12.0299,2.61975);
\draw [color=c, fill=c] (12.0299,2.51391) rectangle (12.0697,2.61975);
\draw [color=c, fill=c] (12.0697,2.51391) rectangle (12.1095,2.61975);
\draw [color=c, fill=c] (12.1095,2.51391) rectangle (12.1493,2.61975);
\draw [color=c, fill=c] (12.1493,2.51391) rectangle (12.1891,2.61975);
\draw [color=c, fill=c] (12.1891,2.51391) rectangle (12.2289,2.61975);
\draw [color=c, fill=c] (12.2289,2.51391) rectangle (12.2687,2.61975);
\definecolor{c}{rgb}{0,0.733333,1};
\draw [color=c, fill=c] (12.2687,2.51391) rectangle (12.3085,2.61975);
\draw [color=c, fill=c] (12.3085,2.51391) rectangle (12.3483,2.61975);
\draw [color=c, fill=c] (12.3483,2.51391) rectangle (12.3881,2.61975);
\draw [color=c, fill=c] (12.3881,2.51391) rectangle (12.4279,2.61975);
\draw [color=c, fill=c] (12.4279,2.51391) rectangle (12.4677,2.61975);
\draw [color=c, fill=c] (12.4677,2.51391) rectangle (12.5075,2.61975);
\draw [color=c, fill=c] (12.5075,2.51391) rectangle (12.5473,2.61975);
\draw [color=c, fill=c] (12.5473,2.51391) rectangle (12.5871,2.61975);
\draw [color=c, fill=c] (12.5871,2.51391) rectangle (12.6269,2.61975);
\draw [color=c, fill=c] (12.6269,2.51391) rectangle (12.6667,2.61975);
\draw [color=c, fill=c] (12.6667,2.51391) rectangle (12.7065,2.61975);
\draw [color=c, fill=c] (12.7065,2.51391) rectangle (12.7463,2.61975);
\draw [color=c, fill=c] (12.7463,2.51391) rectangle (12.7861,2.61975);
\draw [color=c, fill=c] (12.7861,2.51391) rectangle (12.8259,2.61975);
\draw [color=c, fill=c] (12.8259,2.51391) rectangle (12.8657,2.61975);
\draw [color=c, fill=c] (12.8657,2.51391) rectangle (12.9055,2.61975);
\draw [color=c, fill=c] (12.9055,2.51391) rectangle (12.9453,2.61975);
\draw [color=c, fill=c] (12.9453,2.51391) rectangle (12.9851,2.61975);
\draw [color=c, fill=c] (12.9851,2.51391) rectangle (13.0249,2.61975);
\draw [color=c, fill=c] (13.0249,2.51391) rectangle (13.0647,2.61975);
\draw [color=c, fill=c] (13.0647,2.51391) rectangle (13.1045,2.61975);
\draw [color=c, fill=c] (13.1045,2.51391) rectangle (13.1443,2.61975);
\draw [color=c, fill=c] (13.1443,2.51391) rectangle (13.1841,2.61975);
\draw [color=c, fill=c] (13.1841,2.51391) rectangle (13.2239,2.61975);
\draw [color=c, fill=c] (13.2239,2.51391) rectangle (13.2637,2.61975);
\draw [color=c, fill=c] (13.2637,2.51391) rectangle (13.3035,2.61975);
\draw [color=c, fill=c] (13.3035,2.51391) rectangle (13.3433,2.61975);
\draw [color=c, fill=c] (13.3433,2.51391) rectangle (13.3831,2.61975);
\draw [color=c, fill=c] (13.3831,2.51391) rectangle (13.4229,2.61975);
\draw [color=c, fill=c] (13.4229,2.51391) rectangle (13.4627,2.61975);
\draw [color=c, fill=c] (13.4627,2.51391) rectangle (13.5025,2.61975);
\draw [color=c, fill=c] (13.5025,2.51391) rectangle (13.5423,2.61975);
\draw [color=c, fill=c] (13.5423,2.51391) rectangle (13.5821,2.61975);
\draw [color=c, fill=c] (13.5821,2.51391) rectangle (13.6219,2.61975);
\draw [color=c, fill=c] (13.6219,2.51391) rectangle (13.6617,2.61975);
\draw [color=c, fill=c] (13.6617,2.51391) rectangle (13.7015,2.61975);
\draw [color=c, fill=c] (13.7015,2.51391) rectangle (13.7413,2.61975);
\draw [color=c, fill=c] (13.7413,2.51391) rectangle (13.7811,2.61975);
\draw [color=c, fill=c] (13.7811,2.51391) rectangle (13.8209,2.61975);
\draw [color=c, fill=c] (13.8209,2.51391) rectangle (13.8607,2.61975);
\draw [color=c, fill=c] (13.8607,2.51391) rectangle (13.9005,2.61975);
\draw [color=c, fill=c] (13.9005,2.51391) rectangle (13.9403,2.61975);
\draw [color=c, fill=c] (13.9403,2.51391) rectangle (13.9801,2.61975);
\draw [color=c, fill=c] (13.9801,2.51391) rectangle (14.0199,2.61975);
\draw [color=c, fill=c] (14.0199,2.51391) rectangle (14.0597,2.61975);
\draw [color=c, fill=c] (14.0597,2.51391) rectangle (14.0995,2.61975);
\draw [color=c, fill=c] (14.0995,2.51391) rectangle (14.1393,2.61975);
\draw [color=c, fill=c] (14.1393,2.51391) rectangle (14.1791,2.61975);
\draw [color=c, fill=c] (14.1791,2.51391) rectangle (14.2189,2.61975);
\draw [color=c, fill=c] (14.2189,2.51391) rectangle (14.2587,2.61975);
\draw [color=c, fill=c] (14.2587,2.51391) rectangle (14.2985,2.61975);
\draw [color=c, fill=c] (14.2985,2.51391) rectangle (14.3383,2.61975);
\draw [color=c, fill=c] (14.3383,2.51391) rectangle (14.3781,2.61975);
\draw [color=c, fill=c] (14.3781,2.51391) rectangle (14.4179,2.61975);
\draw [color=c, fill=c] (14.4179,2.51391) rectangle (14.4577,2.61975);
\draw [color=c, fill=c] (14.4577,2.51391) rectangle (14.4975,2.61975);
\draw [color=c, fill=c] (14.4975,2.51391) rectangle (14.5373,2.61975);
\draw [color=c, fill=c] (14.5373,2.51391) rectangle (14.5771,2.61975);
\draw [color=c, fill=c] (14.5771,2.51391) rectangle (14.6169,2.61975);
\draw [color=c, fill=c] (14.6169,2.51391) rectangle (14.6567,2.61975);
\draw [color=c, fill=c] (14.6567,2.51391) rectangle (14.6965,2.61975);
\draw [color=c, fill=c] (14.6965,2.51391) rectangle (14.7363,2.61975);
\draw [color=c, fill=c] (14.7363,2.51391) rectangle (14.7761,2.61975);
\draw [color=c, fill=c] (14.7761,2.51391) rectangle (14.8159,2.61975);
\draw [color=c, fill=c] (14.8159,2.51391) rectangle (14.8557,2.61975);
\draw [color=c, fill=c] (14.8557,2.51391) rectangle (14.8955,2.61975);
\draw [color=c, fill=c] (14.8955,2.51391) rectangle (14.9353,2.61975);
\draw [color=c, fill=c] (14.9353,2.51391) rectangle (14.9751,2.61975);
\draw [color=c, fill=c] (14.9751,2.51391) rectangle (15.0149,2.61975);
\draw [color=c, fill=c] (15.0149,2.51391) rectangle (15.0547,2.61975);
\draw [color=c, fill=c] (15.0547,2.51391) rectangle (15.0945,2.61975);
\draw [color=c, fill=c] (15.0945,2.51391) rectangle (15.1343,2.61975);
\draw [color=c, fill=c] (15.1343,2.51391) rectangle (15.1741,2.61975);
\draw [color=c, fill=c] (15.1741,2.51391) rectangle (15.2139,2.61975);
\draw [color=c, fill=c] (15.2139,2.51391) rectangle (15.2537,2.61975);
\draw [color=c, fill=c] (15.2537,2.51391) rectangle (15.2935,2.61975);
\draw [color=c, fill=c] (15.2935,2.51391) rectangle (15.3333,2.61975);
\draw [color=c, fill=c] (15.3333,2.51391) rectangle (15.3731,2.61975);
\draw [color=c, fill=c] (15.3731,2.51391) rectangle (15.4129,2.61975);
\draw [color=c, fill=c] (15.4129,2.51391) rectangle (15.4527,2.61975);
\draw [color=c, fill=c] (15.4527,2.51391) rectangle (15.4925,2.61975);
\draw [color=c, fill=c] (15.4925,2.51391) rectangle (15.5323,2.61975);
\draw [color=c, fill=c] (15.5323,2.51391) rectangle (15.5721,2.61975);
\draw [color=c, fill=c] (15.5721,2.51391) rectangle (15.6119,2.61975);
\draw [color=c, fill=c] (15.6119,2.51391) rectangle (15.6517,2.61975);
\draw [color=c, fill=c] (15.6517,2.51391) rectangle (15.6915,2.61975);
\draw [color=c, fill=c] (15.6915,2.51391) rectangle (15.7313,2.61975);
\draw [color=c, fill=c] (15.7313,2.51391) rectangle (15.7711,2.61975);
\draw [color=c, fill=c] (15.7711,2.51391) rectangle (15.8109,2.61975);
\draw [color=c, fill=c] (15.8109,2.51391) rectangle (15.8507,2.61975);
\draw [color=c, fill=c] (15.8507,2.51391) rectangle (15.8905,2.61975);
\draw [color=c, fill=c] (15.8905,2.51391) rectangle (15.9303,2.61975);
\draw [color=c, fill=c] (15.9303,2.51391) rectangle (15.9701,2.61975);
\draw [color=c, fill=c] (15.9701,2.51391) rectangle (16.01,2.61975);
\draw [color=c, fill=c] (16.01,2.51391) rectangle (16.0498,2.61975);
\draw [color=c, fill=c] (16.0498,2.51391) rectangle (16.0896,2.61975);
\draw [color=c, fill=c] (16.0896,2.51391) rectangle (16.1294,2.61975);
\draw [color=c, fill=c] (16.1294,2.51391) rectangle (16.1692,2.61975);
\draw [color=c, fill=c] (16.1692,2.51391) rectangle (16.209,2.61975);
\draw [color=c, fill=c] (16.209,2.51391) rectangle (16.2488,2.61975);
\draw [color=c, fill=c] (16.2488,2.51391) rectangle (16.2886,2.61975);
\draw [color=c, fill=c] (16.2886,2.51391) rectangle (16.3284,2.61975);
\draw [color=c, fill=c] (16.3284,2.51391) rectangle (16.3682,2.61975);
\draw [color=c, fill=c] (16.3682,2.51391) rectangle (16.408,2.61975);
\draw [color=c, fill=c] (16.408,2.51391) rectangle (16.4478,2.61975);
\draw [color=c, fill=c] (16.4478,2.51391) rectangle (16.4876,2.61975);
\draw [color=c, fill=c] (16.4876,2.51391) rectangle (16.5274,2.61975);
\draw [color=c, fill=c] (16.5274,2.51391) rectangle (16.5672,2.61975);
\draw [color=c, fill=c] (16.5672,2.51391) rectangle (16.607,2.61975);
\draw [color=c, fill=c] (16.607,2.51391) rectangle (16.6468,2.61975);
\draw [color=c, fill=c] (16.6468,2.51391) rectangle (16.6866,2.61975);
\draw [color=c, fill=c] (16.6866,2.51391) rectangle (16.7264,2.61975);
\draw [color=c, fill=c] (16.7264,2.51391) rectangle (16.7662,2.61975);
\draw [color=c, fill=c] (16.7662,2.51391) rectangle (16.806,2.61975);
\draw [color=c, fill=c] (16.806,2.51391) rectangle (16.8458,2.61975);
\draw [color=c, fill=c] (16.8458,2.51391) rectangle (16.8856,2.61975);
\draw [color=c, fill=c] (16.8856,2.51391) rectangle (16.9254,2.61975);
\draw [color=c, fill=c] (16.9254,2.51391) rectangle (16.9652,2.61975);
\draw [color=c, fill=c] (16.9652,2.51391) rectangle (17.005,2.61975);
\draw [color=c, fill=c] (17.005,2.51391) rectangle (17.0448,2.61975);
\draw [color=c, fill=c] (17.0448,2.51391) rectangle (17.0846,2.61975);
\draw [color=c, fill=c] (17.0846,2.51391) rectangle (17.1244,2.61975);
\draw [color=c, fill=c] (17.1244,2.51391) rectangle (17.1642,2.61975);
\draw [color=c, fill=c] (17.1642,2.51391) rectangle (17.204,2.61975);
\draw [color=c, fill=c] (17.204,2.51391) rectangle (17.2438,2.61975);
\draw [color=c, fill=c] (17.2438,2.51391) rectangle (17.2836,2.61975);
\draw [color=c, fill=c] (17.2836,2.51391) rectangle (17.3234,2.61975);
\draw [color=c, fill=c] (17.3234,2.51391) rectangle (17.3632,2.61975);
\draw [color=c, fill=c] (17.3632,2.51391) rectangle (17.403,2.61975);
\draw [color=c, fill=c] (17.403,2.51391) rectangle (17.4428,2.61975);
\draw [color=c, fill=c] (17.4428,2.51391) rectangle (17.4826,2.61975);
\draw [color=c, fill=c] (17.4826,2.51391) rectangle (17.5224,2.61975);
\draw [color=c, fill=c] (17.5224,2.51391) rectangle (17.5622,2.61975);
\draw [color=c, fill=c] (17.5622,2.51391) rectangle (17.602,2.61975);
\draw [color=c, fill=c] (17.602,2.51391) rectangle (17.6418,2.61975);
\draw [color=c, fill=c] (17.6418,2.51391) rectangle (17.6816,2.61975);
\draw [color=c, fill=c] (17.6816,2.51391) rectangle (17.7214,2.61975);
\draw [color=c, fill=c] (17.7214,2.51391) rectangle (17.7612,2.61975);
\draw [color=c, fill=c] (17.7612,2.51391) rectangle (17.801,2.61975);
\draw [color=c, fill=c] (17.801,2.51391) rectangle (17.8408,2.61975);
\draw [color=c, fill=c] (17.8408,2.51391) rectangle (17.8806,2.61975);
\draw [color=c, fill=c] (17.8806,2.51391) rectangle (17.9204,2.61975);
\draw [color=c, fill=c] (17.9204,2.51391) rectangle (17.9602,2.61975);
\draw [color=c, fill=c] (17.9602,2.51391) rectangle (18,2.61975);
\definecolor{c}{rgb}{1,0,0};
\draw [color=c, fill=c] (2,2.61975) rectangle (2.0398,2.7256);
\draw [color=c, fill=c] (2.0398,2.61975) rectangle (2.0796,2.7256);
\draw [color=c, fill=c] (2.0796,2.61975) rectangle (2.1194,2.7256);
\draw [color=c, fill=c] (2.1194,2.61975) rectangle (2.1592,2.7256);
\draw [color=c, fill=c] (2.1592,2.61975) rectangle (2.19901,2.7256);
\draw [color=c, fill=c] (2.19901,2.61975) rectangle (2.23881,2.7256);
\draw [color=c, fill=c] (2.23881,2.61975) rectangle (2.27861,2.7256);
\draw [color=c, fill=c] (2.27861,2.61975) rectangle (2.31841,2.7256);
\draw [color=c, fill=c] (2.31841,2.61975) rectangle (2.35821,2.7256);
\draw [color=c, fill=c] (2.35821,2.61975) rectangle (2.39801,2.7256);
\draw [color=c, fill=c] (2.39801,2.61975) rectangle (2.43781,2.7256);
\draw [color=c, fill=c] (2.43781,2.61975) rectangle (2.47761,2.7256);
\draw [color=c, fill=c] (2.47761,2.61975) rectangle (2.51741,2.7256);
\draw [color=c, fill=c] (2.51741,2.61975) rectangle (2.55721,2.7256);
\draw [color=c, fill=c] (2.55721,2.61975) rectangle (2.59702,2.7256);
\draw [color=c, fill=c] (2.59702,2.61975) rectangle (2.63682,2.7256);
\draw [color=c, fill=c] (2.63682,2.61975) rectangle (2.67662,2.7256);
\draw [color=c, fill=c] (2.67662,2.61975) rectangle (2.71642,2.7256);
\draw [color=c, fill=c] (2.71642,2.61975) rectangle (2.75622,2.7256);
\draw [color=c, fill=c] (2.75622,2.61975) rectangle (2.79602,2.7256);
\draw [color=c, fill=c] (2.79602,2.61975) rectangle (2.83582,2.7256);
\draw [color=c, fill=c] (2.83582,2.61975) rectangle (2.87562,2.7256);
\draw [color=c, fill=c] (2.87562,2.61975) rectangle (2.91542,2.7256);
\draw [color=c, fill=c] (2.91542,2.61975) rectangle (2.95522,2.7256);
\draw [color=c, fill=c] (2.95522,2.61975) rectangle (2.99502,2.7256);
\draw [color=c, fill=c] (2.99502,2.61975) rectangle (3.03483,2.7256);
\draw [color=c, fill=c] (3.03483,2.61975) rectangle (3.07463,2.7256);
\draw [color=c, fill=c] (3.07463,2.61975) rectangle (3.11443,2.7256);
\draw [color=c, fill=c] (3.11443,2.61975) rectangle (3.15423,2.7256);
\draw [color=c, fill=c] (3.15423,2.61975) rectangle (3.19403,2.7256);
\draw [color=c, fill=c] (3.19403,2.61975) rectangle (3.23383,2.7256);
\draw [color=c, fill=c] (3.23383,2.61975) rectangle (3.27363,2.7256);
\draw [color=c, fill=c] (3.27363,2.61975) rectangle (3.31343,2.7256);
\draw [color=c, fill=c] (3.31343,2.61975) rectangle (3.35323,2.7256);
\draw [color=c, fill=c] (3.35323,2.61975) rectangle (3.39303,2.7256);
\draw [color=c, fill=c] (3.39303,2.61975) rectangle (3.43284,2.7256);
\draw [color=c, fill=c] (3.43284,2.61975) rectangle (3.47264,2.7256);
\draw [color=c, fill=c] (3.47264,2.61975) rectangle (3.51244,2.7256);
\draw [color=c, fill=c] (3.51244,2.61975) rectangle (3.55224,2.7256);
\draw [color=c, fill=c] (3.55224,2.61975) rectangle (3.59204,2.7256);
\draw [color=c, fill=c] (3.59204,2.61975) rectangle (3.63184,2.7256);
\draw [color=c, fill=c] (3.63184,2.61975) rectangle (3.67164,2.7256);
\draw [color=c, fill=c] (3.67164,2.61975) rectangle (3.71144,2.7256);
\draw [color=c, fill=c] (3.71144,2.61975) rectangle (3.75124,2.7256);
\draw [color=c, fill=c] (3.75124,2.61975) rectangle (3.79104,2.7256);
\draw [color=c, fill=c] (3.79104,2.61975) rectangle (3.83085,2.7256);
\draw [color=c, fill=c] (3.83085,2.61975) rectangle (3.87065,2.7256);
\draw [color=c, fill=c] (3.87065,2.61975) rectangle (3.91045,2.7256);
\draw [color=c, fill=c] (3.91045,2.61975) rectangle (3.95025,2.7256);
\draw [color=c, fill=c] (3.95025,2.61975) rectangle (3.99005,2.7256);
\draw [color=c, fill=c] (3.99005,2.61975) rectangle (4.02985,2.7256);
\draw [color=c, fill=c] (4.02985,2.61975) rectangle (4.06965,2.7256);
\draw [color=c, fill=c] (4.06965,2.61975) rectangle (4.10945,2.7256);
\draw [color=c, fill=c] (4.10945,2.61975) rectangle (4.14925,2.7256);
\draw [color=c, fill=c] (4.14925,2.61975) rectangle (4.18905,2.7256);
\draw [color=c, fill=c] (4.18905,2.61975) rectangle (4.22886,2.7256);
\draw [color=c, fill=c] (4.22886,2.61975) rectangle (4.26866,2.7256);
\draw [color=c, fill=c] (4.26866,2.61975) rectangle (4.30846,2.7256);
\draw [color=c, fill=c] (4.30846,2.61975) rectangle (4.34826,2.7256);
\draw [color=c, fill=c] (4.34826,2.61975) rectangle (4.38806,2.7256);
\draw [color=c, fill=c] (4.38806,2.61975) rectangle (4.42786,2.7256);
\draw [color=c, fill=c] (4.42786,2.61975) rectangle (4.46766,2.7256);
\draw [color=c, fill=c] (4.46766,2.61975) rectangle (4.50746,2.7256);
\draw [color=c, fill=c] (4.50746,2.61975) rectangle (4.54726,2.7256);
\draw [color=c, fill=c] (4.54726,2.61975) rectangle (4.58706,2.7256);
\draw [color=c, fill=c] (4.58706,2.61975) rectangle (4.62687,2.7256);
\draw [color=c, fill=c] (4.62687,2.61975) rectangle (4.66667,2.7256);
\draw [color=c, fill=c] (4.66667,2.61975) rectangle (4.70647,2.7256);
\draw [color=c, fill=c] (4.70647,2.61975) rectangle (4.74627,2.7256);
\draw [color=c, fill=c] (4.74627,2.61975) rectangle (4.78607,2.7256);
\draw [color=c, fill=c] (4.78607,2.61975) rectangle (4.82587,2.7256);
\draw [color=c, fill=c] (4.82587,2.61975) rectangle (4.86567,2.7256);
\draw [color=c, fill=c] (4.86567,2.61975) rectangle (4.90547,2.7256);
\draw [color=c, fill=c] (4.90547,2.61975) rectangle (4.94527,2.7256);
\draw [color=c, fill=c] (4.94527,2.61975) rectangle (4.98507,2.7256);
\draw [color=c, fill=c] (4.98507,2.61975) rectangle (5.02488,2.7256);
\draw [color=c, fill=c] (5.02488,2.61975) rectangle (5.06468,2.7256);
\draw [color=c, fill=c] (5.06468,2.61975) rectangle (5.10448,2.7256);
\draw [color=c, fill=c] (5.10448,2.61975) rectangle (5.14428,2.7256);
\draw [color=c, fill=c] (5.14428,2.61975) rectangle (5.18408,2.7256);
\draw [color=c, fill=c] (5.18408,2.61975) rectangle (5.22388,2.7256);
\draw [color=c, fill=c] (5.22388,2.61975) rectangle (5.26368,2.7256);
\draw [color=c, fill=c] (5.26368,2.61975) rectangle (5.30348,2.7256);
\draw [color=c, fill=c] (5.30348,2.61975) rectangle (5.34328,2.7256);
\draw [color=c, fill=c] (5.34328,2.61975) rectangle (5.38308,2.7256);
\draw [color=c, fill=c] (5.38308,2.61975) rectangle (5.42289,2.7256);
\draw [color=c, fill=c] (5.42289,2.61975) rectangle (5.46269,2.7256);
\draw [color=c, fill=c] (5.46269,2.61975) rectangle (5.50249,2.7256);
\draw [color=c, fill=c] (5.50249,2.61975) rectangle (5.54229,2.7256);
\draw [color=c, fill=c] (5.54229,2.61975) rectangle (5.58209,2.7256);
\draw [color=c, fill=c] (5.58209,2.61975) rectangle (5.62189,2.7256);
\draw [color=c, fill=c] (5.62189,2.61975) rectangle (5.66169,2.7256);
\draw [color=c, fill=c] (5.66169,2.61975) rectangle (5.70149,2.7256);
\draw [color=c, fill=c] (5.70149,2.61975) rectangle (5.74129,2.7256);
\draw [color=c, fill=c] (5.74129,2.61975) rectangle (5.78109,2.7256);
\draw [color=c, fill=c] (5.78109,2.61975) rectangle (5.8209,2.7256);
\draw [color=c, fill=c] (5.8209,2.61975) rectangle (5.8607,2.7256);
\draw [color=c, fill=c] (5.8607,2.61975) rectangle (5.9005,2.7256);
\draw [color=c, fill=c] (5.9005,2.61975) rectangle (5.9403,2.7256);
\draw [color=c, fill=c] (5.9403,2.61975) rectangle (5.9801,2.7256);
\draw [color=c, fill=c] (5.9801,2.61975) rectangle (6.0199,2.7256);
\draw [color=c, fill=c] (6.0199,2.61975) rectangle (6.0597,2.7256);
\draw [color=c, fill=c] (6.0597,2.61975) rectangle (6.0995,2.7256);
\draw [color=c, fill=c] (6.0995,2.61975) rectangle (6.1393,2.7256);
\draw [color=c, fill=c] (6.1393,2.61975) rectangle (6.1791,2.7256);
\draw [color=c, fill=c] (6.1791,2.61975) rectangle (6.21891,2.7256);
\draw [color=c, fill=c] (6.21891,2.61975) rectangle (6.25871,2.7256);
\draw [color=c, fill=c] (6.25871,2.61975) rectangle (6.29851,2.7256);
\draw [color=c, fill=c] (6.29851,2.61975) rectangle (6.33831,2.7256);
\draw [color=c, fill=c] (6.33831,2.61975) rectangle (6.37811,2.7256);
\draw [color=c, fill=c] (6.37811,2.61975) rectangle (6.41791,2.7256);
\draw [color=c, fill=c] (6.41791,2.61975) rectangle (6.45771,2.7256);
\draw [color=c, fill=c] (6.45771,2.61975) rectangle (6.49751,2.7256);
\draw [color=c, fill=c] (6.49751,2.61975) rectangle (6.53731,2.7256);
\draw [color=c, fill=c] (6.53731,2.61975) rectangle (6.57711,2.7256);
\draw [color=c, fill=c] (6.57711,2.61975) rectangle (6.61692,2.7256);
\draw [color=c, fill=c] (6.61692,2.61975) rectangle (6.65672,2.7256);
\draw [color=c, fill=c] (6.65672,2.61975) rectangle (6.69652,2.7256);
\draw [color=c, fill=c] (6.69652,2.61975) rectangle (6.73632,2.7256);
\draw [color=c, fill=c] (6.73632,2.61975) rectangle (6.77612,2.7256);
\draw [color=c, fill=c] (6.77612,2.61975) rectangle (6.81592,2.7256);
\draw [color=c, fill=c] (6.81592,2.61975) rectangle (6.85572,2.7256);
\draw [color=c, fill=c] (6.85572,2.61975) rectangle (6.89552,2.7256);
\draw [color=c, fill=c] (6.89552,2.61975) rectangle (6.93532,2.7256);
\draw [color=c, fill=c] (6.93532,2.61975) rectangle (6.97512,2.7256);
\draw [color=c, fill=c] (6.97512,2.61975) rectangle (7.01493,2.7256);
\draw [color=c, fill=c] (7.01493,2.61975) rectangle (7.05473,2.7256);
\draw [color=c, fill=c] (7.05473,2.61975) rectangle (7.09453,2.7256);
\draw [color=c, fill=c] (7.09453,2.61975) rectangle (7.13433,2.7256);
\draw [color=c, fill=c] (7.13433,2.61975) rectangle (7.17413,2.7256);
\draw [color=c, fill=c] (7.17413,2.61975) rectangle (7.21393,2.7256);
\draw [color=c, fill=c] (7.21393,2.61975) rectangle (7.25373,2.7256);
\draw [color=c, fill=c] (7.25373,2.61975) rectangle (7.29353,2.7256);
\draw [color=c, fill=c] (7.29353,2.61975) rectangle (7.33333,2.7256);
\draw [color=c, fill=c] (7.33333,2.61975) rectangle (7.37313,2.7256);
\draw [color=c, fill=c] (7.37313,2.61975) rectangle (7.41294,2.7256);
\draw [color=c, fill=c] (7.41294,2.61975) rectangle (7.45274,2.7256);
\draw [color=c, fill=c] (7.45274,2.61975) rectangle (7.49254,2.7256);
\draw [color=c, fill=c] (7.49254,2.61975) rectangle (7.53234,2.7256);
\draw [color=c, fill=c] (7.53234,2.61975) rectangle (7.57214,2.7256);
\draw [color=c, fill=c] (7.57214,2.61975) rectangle (7.61194,2.7256);
\draw [color=c, fill=c] (7.61194,2.61975) rectangle (7.65174,2.7256);
\draw [color=c, fill=c] (7.65174,2.61975) rectangle (7.69154,2.7256);
\draw [color=c, fill=c] (7.69154,2.61975) rectangle (7.73134,2.7256);
\draw [color=c, fill=c] (7.73134,2.61975) rectangle (7.77114,2.7256);
\draw [color=c, fill=c] (7.77114,2.61975) rectangle (7.81095,2.7256);
\draw [color=c, fill=c] (7.81095,2.61975) rectangle (7.85075,2.7256);
\definecolor{c}{rgb}{1,0.186667,0};
\draw [color=c, fill=c] (7.85075,2.61975) rectangle (7.89055,2.7256);
\draw [color=c, fill=c] (7.89055,2.61975) rectangle (7.93035,2.7256);
\draw [color=c, fill=c] (7.93035,2.61975) rectangle (7.97015,2.7256);
\draw [color=c, fill=c] (7.97015,2.61975) rectangle (8.00995,2.7256);
\draw [color=c, fill=c] (8.00995,2.61975) rectangle (8.04975,2.7256);
\draw [color=c, fill=c] (8.04975,2.61975) rectangle (8.08955,2.7256);
\draw [color=c, fill=c] (8.08955,2.61975) rectangle (8.12935,2.7256);
\draw [color=c, fill=c] (8.12935,2.61975) rectangle (8.16915,2.7256);
\draw [color=c, fill=c] (8.16915,2.61975) rectangle (8.20895,2.7256);
\draw [color=c, fill=c] (8.20895,2.61975) rectangle (8.24876,2.7256);
\draw [color=c, fill=c] (8.24876,2.61975) rectangle (8.28856,2.7256);
\draw [color=c, fill=c] (8.28856,2.61975) rectangle (8.32836,2.7256);
\draw [color=c, fill=c] (8.32836,2.61975) rectangle (8.36816,2.7256);
\draw [color=c, fill=c] (8.36816,2.61975) rectangle (8.40796,2.7256);
\draw [color=c, fill=c] (8.40796,2.61975) rectangle (8.44776,2.7256);
\draw [color=c, fill=c] (8.44776,2.61975) rectangle (8.48756,2.7256);
\draw [color=c, fill=c] (8.48756,2.61975) rectangle (8.52736,2.7256);
\draw [color=c, fill=c] (8.52736,2.61975) rectangle (8.56716,2.7256);
\draw [color=c, fill=c] (8.56716,2.61975) rectangle (8.60697,2.7256);
\draw [color=c, fill=c] (8.60697,2.61975) rectangle (8.64677,2.7256);
\draw [color=c, fill=c] (8.64677,2.61975) rectangle (8.68657,2.7256);
\draw [color=c, fill=c] (8.68657,2.61975) rectangle (8.72637,2.7256);
\definecolor{c}{rgb}{1,0.466667,0};
\draw [color=c, fill=c] (8.72637,2.61975) rectangle (8.76617,2.7256);
\draw [color=c, fill=c] (8.76617,2.61975) rectangle (8.80597,2.7256);
\draw [color=c, fill=c] (8.80597,2.61975) rectangle (8.84577,2.7256);
\draw [color=c, fill=c] (8.84577,2.61975) rectangle (8.88557,2.7256);
\draw [color=c, fill=c] (8.88557,2.61975) rectangle (8.92537,2.7256);
\draw [color=c, fill=c] (8.92537,2.61975) rectangle (8.96517,2.7256);
\draw [color=c, fill=c] (8.96517,2.61975) rectangle (9.00498,2.7256);
\draw [color=c, fill=c] (9.00498,2.61975) rectangle (9.04478,2.7256);
\draw [color=c, fill=c] (9.04478,2.61975) rectangle (9.08458,2.7256);
\draw [color=c, fill=c] (9.08458,2.61975) rectangle (9.12438,2.7256);
\draw [color=c, fill=c] (9.12438,2.61975) rectangle (9.16418,2.7256);
\definecolor{c}{rgb}{1,0.653333,0};
\draw [color=c, fill=c] (9.16418,2.61975) rectangle (9.20398,2.7256);
\draw [color=c, fill=c] (9.20398,2.61975) rectangle (9.24378,2.7256);
\draw [color=c, fill=c] (9.24378,2.61975) rectangle (9.28358,2.7256);
\draw [color=c, fill=c] (9.28358,2.61975) rectangle (9.32338,2.7256);
\draw [color=c, fill=c] (9.32338,2.61975) rectangle (9.36318,2.7256);
\draw [color=c, fill=c] (9.36318,2.61975) rectangle (9.40298,2.7256);
\draw [color=c, fill=c] (9.40298,2.61975) rectangle (9.44279,2.7256);
\definecolor{c}{rgb}{1,0.933333,0};
\draw [color=c, fill=c] (9.44279,2.61975) rectangle (9.48259,2.7256);
\draw [color=c, fill=c] (9.48259,2.61975) rectangle (9.52239,2.7256);
\draw [color=c, fill=c] (9.52239,2.61975) rectangle (9.56219,2.7256);
\draw [color=c, fill=c] (9.56219,2.61975) rectangle (9.60199,2.7256);
\draw [color=c, fill=c] (9.60199,2.61975) rectangle (9.64179,2.7256);
\definecolor{c}{rgb}{0.88,1,0};
\draw [color=c, fill=c] (9.64179,2.61975) rectangle (9.68159,2.7256);
\draw [color=c, fill=c] (9.68159,2.61975) rectangle (9.72139,2.7256);
\draw [color=c, fill=c] (9.72139,2.61975) rectangle (9.76119,2.7256);
\draw [color=c, fill=c] (9.76119,2.61975) rectangle (9.80099,2.7256);
\definecolor{c}{rgb}{0.6,1,0};
\draw [color=c, fill=c] (9.80099,2.61975) rectangle (9.8408,2.7256);
\draw [color=c, fill=c] (9.8408,2.61975) rectangle (9.8806,2.7256);
\draw [color=c, fill=c] (9.8806,2.61975) rectangle (9.9204,2.7256);
\definecolor{c}{rgb}{0.413333,1,0};
\draw [color=c, fill=c] (9.9204,2.61975) rectangle (9.9602,2.7256);
\draw [color=c, fill=c] (9.9602,2.61975) rectangle (10,2.7256);
\draw [color=c, fill=c] (10,2.61975) rectangle (10.0398,2.7256);
\definecolor{c}{rgb}{0.133333,1,0};
\draw [color=c, fill=c] (10.0398,2.61975) rectangle (10.0796,2.7256);
\draw [color=c, fill=c] (10.0796,2.61975) rectangle (10.1194,2.7256);
\draw [color=c, fill=c] (10.1194,2.61975) rectangle (10.1592,2.7256);
\definecolor{c}{rgb}{0,1,0.0533333};
\draw [color=c, fill=c] (10.1592,2.61975) rectangle (10.199,2.7256);
\draw [color=c, fill=c] (10.199,2.61975) rectangle (10.2388,2.7256);
\draw [color=c, fill=c] (10.2388,2.61975) rectangle (10.2786,2.7256);
\draw [color=c, fill=c] (10.2786,2.61975) rectangle (10.3184,2.7256);
\definecolor{c}{rgb}{0,1,0.333333};
\draw [color=c, fill=c] (10.3184,2.61975) rectangle (10.3582,2.7256);
\draw [color=c, fill=c] (10.3582,2.61975) rectangle (10.398,2.7256);
\draw [color=c, fill=c] (10.398,2.61975) rectangle (10.4378,2.7256);
\draw [color=c, fill=c] (10.4378,2.61975) rectangle (10.4776,2.7256);
\draw [color=c, fill=c] (10.4776,2.61975) rectangle (10.5174,2.7256);
\definecolor{c}{rgb}{0,1,0.52};
\draw [color=c, fill=c] (10.5174,2.61975) rectangle (10.5572,2.7256);
\draw [color=c, fill=c] (10.5572,2.61975) rectangle (10.597,2.7256);
\draw [color=c, fill=c] (10.597,2.61975) rectangle (10.6368,2.7256);
\draw [color=c, fill=c] (10.6368,2.61975) rectangle (10.6766,2.7256);
\draw [color=c, fill=c] (10.6766,2.61975) rectangle (10.7164,2.7256);
\draw [color=c, fill=c] (10.7164,2.61975) rectangle (10.7562,2.7256);
\draw [color=c, fill=c] (10.7562,2.61975) rectangle (10.796,2.7256);
\definecolor{c}{rgb}{0,1,0.8};
\draw [color=c, fill=c] (10.796,2.61975) rectangle (10.8358,2.7256);
\draw [color=c, fill=c] (10.8358,2.61975) rectangle (10.8756,2.7256);
\draw [color=c, fill=c] (10.8756,2.61975) rectangle (10.9154,2.7256);
\draw [color=c, fill=c] (10.9154,2.61975) rectangle (10.9552,2.7256);
\draw [color=c, fill=c] (10.9552,2.61975) rectangle (10.995,2.7256);
\draw [color=c, fill=c] (10.995,2.61975) rectangle (11.0348,2.7256);
\draw [color=c, fill=c] (11.0348,2.61975) rectangle (11.0746,2.7256);
\draw [color=c, fill=c] (11.0746,2.61975) rectangle (11.1144,2.7256);
\draw [color=c, fill=c] (11.1144,2.61975) rectangle (11.1542,2.7256);
\draw [color=c, fill=c] (11.1542,2.61975) rectangle (11.194,2.7256);
\draw [color=c, fill=c] (11.194,2.61975) rectangle (11.2338,2.7256);
\draw [color=c, fill=c] (11.2338,2.61975) rectangle (11.2736,2.7256);
\definecolor{c}{rgb}{0,1,0.986667};
\draw [color=c, fill=c] (11.2736,2.61975) rectangle (11.3134,2.7256);
\draw [color=c, fill=c] (11.3134,2.61975) rectangle (11.3532,2.7256);
\draw [color=c, fill=c] (11.3532,2.61975) rectangle (11.393,2.7256);
\draw [color=c, fill=c] (11.393,2.61975) rectangle (11.4328,2.7256);
\draw [color=c, fill=c] (11.4328,2.61975) rectangle (11.4726,2.7256);
\draw [color=c, fill=c] (11.4726,2.61975) rectangle (11.5124,2.7256);
\draw [color=c, fill=c] (11.5124,2.61975) rectangle (11.5522,2.7256);
\draw [color=c, fill=c] (11.5522,2.61975) rectangle (11.592,2.7256);
\draw [color=c, fill=c] (11.592,2.61975) rectangle (11.6318,2.7256);
\draw [color=c, fill=c] (11.6318,2.61975) rectangle (11.6716,2.7256);
\draw [color=c, fill=c] (11.6716,2.61975) rectangle (11.7114,2.7256);
\draw [color=c, fill=c] (11.7114,2.61975) rectangle (11.7512,2.7256);
\draw [color=c, fill=c] (11.7512,2.61975) rectangle (11.791,2.7256);
\draw [color=c, fill=c] (11.791,2.61975) rectangle (11.8308,2.7256);
\draw [color=c, fill=c] (11.8308,2.61975) rectangle (11.8706,2.7256);
\draw [color=c, fill=c] (11.8706,2.61975) rectangle (11.9104,2.7256);
\draw [color=c, fill=c] (11.9104,2.61975) rectangle (11.9502,2.7256);
\draw [color=c, fill=c] (11.9502,2.61975) rectangle (11.99,2.7256);
\draw [color=c, fill=c] (11.99,2.61975) rectangle (12.0299,2.7256);
\draw [color=c, fill=c] (12.0299,2.61975) rectangle (12.0697,2.7256);
\draw [color=c, fill=c] (12.0697,2.61975) rectangle (12.1095,2.7256);
\draw [color=c, fill=c] (12.1095,2.61975) rectangle (12.1493,2.7256);
\draw [color=c, fill=c] (12.1493,2.61975) rectangle (12.1891,2.7256);
\draw [color=c, fill=c] (12.1891,2.61975) rectangle (12.2289,2.7256);
\definecolor{c}{rgb}{0,0.733333,1};
\draw [color=c, fill=c] (12.2289,2.61975) rectangle (12.2687,2.7256);
\draw [color=c, fill=c] (12.2687,2.61975) rectangle (12.3085,2.7256);
\draw [color=c, fill=c] (12.3085,2.61975) rectangle (12.3483,2.7256);
\draw [color=c, fill=c] (12.3483,2.61975) rectangle (12.3881,2.7256);
\draw [color=c, fill=c] (12.3881,2.61975) rectangle (12.4279,2.7256);
\draw [color=c, fill=c] (12.4279,2.61975) rectangle (12.4677,2.7256);
\draw [color=c, fill=c] (12.4677,2.61975) rectangle (12.5075,2.7256);
\draw [color=c, fill=c] (12.5075,2.61975) rectangle (12.5473,2.7256);
\draw [color=c, fill=c] (12.5473,2.61975) rectangle (12.5871,2.7256);
\draw [color=c, fill=c] (12.5871,2.61975) rectangle (12.6269,2.7256);
\draw [color=c, fill=c] (12.6269,2.61975) rectangle (12.6667,2.7256);
\draw [color=c, fill=c] (12.6667,2.61975) rectangle (12.7065,2.7256);
\draw [color=c, fill=c] (12.7065,2.61975) rectangle (12.7463,2.7256);
\draw [color=c, fill=c] (12.7463,2.61975) rectangle (12.7861,2.7256);
\draw [color=c, fill=c] (12.7861,2.61975) rectangle (12.8259,2.7256);
\draw [color=c, fill=c] (12.8259,2.61975) rectangle (12.8657,2.7256);
\draw [color=c, fill=c] (12.8657,2.61975) rectangle (12.9055,2.7256);
\draw [color=c, fill=c] (12.9055,2.61975) rectangle (12.9453,2.7256);
\draw [color=c, fill=c] (12.9453,2.61975) rectangle (12.9851,2.7256);
\draw [color=c, fill=c] (12.9851,2.61975) rectangle (13.0249,2.7256);
\draw [color=c, fill=c] (13.0249,2.61975) rectangle (13.0647,2.7256);
\draw [color=c, fill=c] (13.0647,2.61975) rectangle (13.1045,2.7256);
\draw [color=c, fill=c] (13.1045,2.61975) rectangle (13.1443,2.7256);
\draw [color=c, fill=c] (13.1443,2.61975) rectangle (13.1841,2.7256);
\draw [color=c, fill=c] (13.1841,2.61975) rectangle (13.2239,2.7256);
\draw [color=c, fill=c] (13.2239,2.61975) rectangle (13.2637,2.7256);
\draw [color=c, fill=c] (13.2637,2.61975) rectangle (13.3035,2.7256);
\draw [color=c, fill=c] (13.3035,2.61975) rectangle (13.3433,2.7256);
\draw [color=c, fill=c] (13.3433,2.61975) rectangle (13.3831,2.7256);
\draw [color=c, fill=c] (13.3831,2.61975) rectangle (13.4229,2.7256);
\draw [color=c, fill=c] (13.4229,2.61975) rectangle (13.4627,2.7256);
\draw [color=c, fill=c] (13.4627,2.61975) rectangle (13.5025,2.7256);
\draw [color=c, fill=c] (13.5025,2.61975) rectangle (13.5423,2.7256);
\draw [color=c, fill=c] (13.5423,2.61975) rectangle (13.5821,2.7256);
\draw [color=c, fill=c] (13.5821,2.61975) rectangle (13.6219,2.7256);
\draw [color=c, fill=c] (13.6219,2.61975) rectangle (13.6617,2.7256);
\draw [color=c, fill=c] (13.6617,2.61975) rectangle (13.7015,2.7256);
\draw [color=c, fill=c] (13.7015,2.61975) rectangle (13.7413,2.7256);
\draw [color=c, fill=c] (13.7413,2.61975) rectangle (13.7811,2.7256);
\draw [color=c, fill=c] (13.7811,2.61975) rectangle (13.8209,2.7256);
\draw [color=c, fill=c] (13.8209,2.61975) rectangle (13.8607,2.7256);
\draw [color=c, fill=c] (13.8607,2.61975) rectangle (13.9005,2.7256);
\draw [color=c, fill=c] (13.9005,2.61975) rectangle (13.9403,2.7256);
\draw [color=c, fill=c] (13.9403,2.61975) rectangle (13.9801,2.7256);
\draw [color=c, fill=c] (13.9801,2.61975) rectangle (14.0199,2.7256);
\draw [color=c, fill=c] (14.0199,2.61975) rectangle (14.0597,2.7256);
\draw [color=c, fill=c] (14.0597,2.61975) rectangle (14.0995,2.7256);
\draw [color=c, fill=c] (14.0995,2.61975) rectangle (14.1393,2.7256);
\draw [color=c, fill=c] (14.1393,2.61975) rectangle (14.1791,2.7256);
\draw [color=c, fill=c] (14.1791,2.61975) rectangle (14.2189,2.7256);
\draw [color=c, fill=c] (14.2189,2.61975) rectangle (14.2587,2.7256);
\draw [color=c, fill=c] (14.2587,2.61975) rectangle (14.2985,2.7256);
\draw [color=c, fill=c] (14.2985,2.61975) rectangle (14.3383,2.7256);
\draw [color=c, fill=c] (14.3383,2.61975) rectangle (14.3781,2.7256);
\draw [color=c, fill=c] (14.3781,2.61975) rectangle (14.4179,2.7256);
\draw [color=c, fill=c] (14.4179,2.61975) rectangle (14.4577,2.7256);
\draw [color=c, fill=c] (14.4577,2.61975) rectangle (14.4975,2.7256);
\draw [color=c, fill=c] (14.4975,2.61975) rectangle (14.5373,2.7256);
\draw [color=c, fill=c] (14.5373,2.61975) rectangle (14.5771,2.7256);
\draw [color=c, fill=c] (14.5771,2.61975) rectangle (14.6169,2.7256);
\draw [color=c, fill=c] (14.6169,2.61975) rectangle (14.6567,2.7256);
\draw [color=c, fill=c] (14.6567,2.61975) rectangle (14.6965,2.7256);
\draw [color=c, fill=c] (14.6965,2.61975) rectangle (14.7363,2.7256);
\draw [color=c, fill=c] (14.7363,2.61975) rectangle (14.7761,2.7256);
\draw [color=c, fill=c] (14.7761,2.61975) rectangle (14.8159,2.7256);
\draw [color=c, fill=c] (14.8159,2.61975) rectangle (14.8557,2.7256);
\draw [color=c, fill=c] (14.8557,2.61975) rectangle (14.8955,2.7256);
\draw [color=c, fill=c] (14.8955,2.61975) rectangle (14.9353,2.7256);
\draw [color=c, fill=c] (14.9353,2.61975) rectangle (14.9751,2.7256);
\draw [color=c, fill=c] (14.9751,2.61975) rectangle (15.0149,2.7256);
\draw [color=c, fill=c] (15.0149,2.61975) rectangle (15.0547,2.7256);
\draw [color=c, fill=c] (15.0547,2.61975) rectangle (15.0945,2.7256);
\draw [color=c, fill=c] (15.0945,2.61975) rectangle (15.1343,2.7256);
\draw [color=c, fill=c] (15.1343,2.61975) rectangle (15.1741,2.7256);
\draw [color=c, fill=c] (15.1741,2.61975) rectangle (15.2139,2.7256);
\draw [color=c, fill=c] (15.2139,2.61975) rectangle (15.2537,2.7256);
\draw [color=c, fill=c] (15.2537,2.61975) rectangle (15.2935,2.7256);
\draw [color=c, fill=c] (15.2935,2.61975) rectangle (15.3333,2.7256);
\draw [color=c, fill=c] (15.3333,2.61975) rectangle (15.3731,2.7256);
\draw [color=c, fill=c] (15.3731,2.61975) rectangle (15.4129,2.7256);
\draw [color=c, fill=c] (15.4129,2.61975) rectangle (15.4527,2.7256);
\draw [color=c, fill=c] (15.4527,2.61975) rectangle (15.4925,2.7256);
\draw [color=c, fill=c] (15.4925,2.61975) rectangle (15.5323,2.7256);
\draw [color=c, fill=c] (15.5323,2.61975) rectangle (15.5721,2.7256);
\draw [color=c, fill=c] (15.5721,2.61975) rectangle (15.6119,2.7256);
\draw [color=c, fill=c] (15.6119,2.61975) rectangle (15.6517,2.7256);
\draw [color=c, fill=c] (15.6517,2.61975) rectangle (15.6915,2.7256);
\draw [color=c, fill=c] (15.6915,2.61975) rectangle (15.7313,2.7256);
\draw [color=c, fill=c] (15.7313,2.61975) rectangle (15.7711,2.7256);
\draw [color=c, fill=c] (15.7711,2.61975) rectangle (15.8109,2.7256);
\draw [color=c, fill=c] (15.8109,2.61975) rectangle (15.8507,2.7256);
\draw [color=c, fill=c] (15.8507,2.61975) rectangle (15.8905,2.7256);
\draw [color=c, fill=c] (15.8905,2.61975) rectangle (15.9303,2.7256);
\draw [color=c, fill=c] (15.9303,2.61975) rectangle (15.9701,2.7256);
\draw [color=c, fill=c] (15.9701,2.61975) rectangle (16.01,2.7256);
\draw [color=c, fill=c] (16.01,2.61975) rectangle (16.0498,2.7256);
\draw [color=c, fill=c] (16.0498,2.61975) rectangle (16.0896,2.7256);
\draw [color=c, fill=c] (16.0896,2.61975) rectangle (16.1294,2.7256);
\draw [color=c, fill=c] (16.1294,2.61975) rectangle (16.1692,2.7256);
\draw [color=c, fill=c] (16.1692,2.61975) rectangle (16.209,2.7256);
\draw [color=c, fill=c] (16.209,2.61975) rectangle (16.2488,2.7256);
\draw [color=c, fill=c] (16.2488,2.61975) rectangle (16.2886,2.7256);
\draw [color=c, fill=c] (16.2886,2.61975) rectangle (16.3284,2.7256);
\draw [color=c, fill=c] (16.3284,2.61975) rectangle (16.3682,2.7256);
\draw [color=c, fill=c] (16.3682,2.61975) rectangle (16.408,2.7256);
\draw [color=c, fill=c] (16.408,2.61975) rectangle (16.4478,2.7256);
\draw [color=c, fill=c] (16.4478,2.61975) rectangle (16.4876,2.7256);
\draw [color=c, fill=c] (16.4876,2.61975) rectangle (16.5274,2.7256);
\draw [color=c, fill=c] (16.5274,2.61975) rectangle (16.5672,2.7256);
\draw [color=c, fill=c] (16.5672,2.61975) rectangle (16.607,2.7256);
\draw [color=c, fill=c] (16.607,2.61975) rectangle (16.6468,2.7256);
\draw [color=c, fill=c] (16.6468,2.61975) rectangle (16.6866,2.7256);
\draw [color=c, fill=c] (16.6866,2.61975) rectangle (16.7264,2.7256);
\draw [color=c, fill=c] (16.7264,2.61975) rectangle (16.7662,2.7256);
\draw [color=c, fill=c] (16.7662,2.61975) rectangle (16.806,2.7256);
\draw [color=c, fill=c] (16.806,2.61975) rectangle (16.8458,2.7256);
\draw [color=c, fill=c] (16.8458,2.61975) rectangle (16.8856,2.7256);
\draw [color=c, fill=c] (16.8856,2.61975) rectangle (16.9254,2.7256);
\draw [color=c, fill=c] (16.9254,2.61975) rectangle (16.9652,2.7256);
\draw [color=c, fill=c] (16.9652,2.61975) rectangle (17.005,2.7256);
\draw [color=c, fill=c] (17.005,2.61975) rectangle (17.0448,2.7256);
\draw [color=c, fill=c] (17.0448,2.61975) rectangle (17.0846,2.7256);
\draw [color=c, fill=c] (17.0846,2.61975) rectangle (17.1244,2.7256);
\draw [color=c, fill=c] (17.1244,2.61975) rectangle (17.1642,2.7256);
\draw [color=c, fill=c] (17.1642,2.61975) rectangle (17.204,2.7256);
\draw [color=c, fill=c] (17.204,2.61975) rectangle (17.2438,2.7256);
\draw [color=c, fill=c] (17.2438,2.61975) rectangle (17.2836,2.7256);
\draw [color=c, fill=c] (17.2836,2.61975) rectangle (17.3234,2.7256);
\draw [color=c, fill=c] (17.3234,2.61975) rectangle (17.3632,2.7256);
\draw [color=c, fill=c] (17.3632,2.61975) rectangle (17.403,2.7256);
\draw [color=c, fill=c] (17.403,2.61975) rectangle (17.4428,2.7256);
\draw [color=c, fill=c] (17.4428,2.61975) rectangle (17.4826,2.7256);
\draw [color=c, fill=c] (17.4826,2.61975) rectangle (17.5224,2.7256);
\draw [color=c, fill=c] (17.5224,2.61975) rectangle (17.5622,2.7256);
\draw [color=c, fill=c] (17.5622,2.61975) rectangle (17.602,2.7256);
\draw [color=c, fill=c] (17.602,2.61975) rectangle (17.6418,2.7256);
\draw [color=c, fill=c] (17.6418,2.61975) rectangle (17.6816,2.7256);
\draw [color=c, fill=c] (17.6816,2.61975) rectangle (17.7214,2.7256);
\draw [color=c, fill=c] (17.7214,2.61975) rectangle (17.7612,2.7256);
\draw [color=c, fill=c] (17.7612,2.61975) rectangle (17.801,2.7256);
\draw [color=c, fill=c] (17.801,2.61975) rectangle (17.8408,2.7256);
\draw [color=c, fill=c] (17.8408,2.61975) rectangle (17.8806,2.7256);
\draw [color=c, fill=c] (17.8806,2.61975) rectangle (17.9204,2.7256);
\draw [color=c, fill=c] (17.9204,2.61975) rectangle (17.9602,2.7256);
\draw [color=c, fill=c] (17.9602,2.61975) rectangle (18,2.7256);
\definecolor{c}{rgb}{1,0,0};
\draw [color=c, fill=c] (2,2.7256) rectangle (2.0398,2.83145);
\draw [color=c, fill=c] (2.0398,2.7256) rectangle (2.0796,2.83145);
\draw [color=c, fill=c] (2.0796,2.7256) rectangle (2.1194,2.83145);
\draw [color=c, fill=c] (2.1194,2.7256) rectangle (2.1592,2.83145);
\draw [color=c, fill=c] (2.1592,2.7256) rectangle (2.19901,2.83145);
\draw [color=c, fill=c] (2.19901,2.7256) rectangle (2.23881,2.83145);
\draw [color=c, fill=c] (2.23881,2.7256) rectangle (2.27861,2.83145);
\draw [color=c, fill=c] (2.27861,2.7256) rectangle (2.31841,2.83145);
\draw [color=c, fill=c] (2.31841,2.7256) rectangle (2.35821,2.83145);
\draw [color=c, fill=c] (2.35821,2.7256) rectangle (2.39801,2.83145);
\draw [color=c, fill=c] (2.39801,2.7256) rectangle (2.43781,2.83145);
\draw [color=c, fill=c] (2.43781,2.7256) rectangle (2.47761,2.83145);
\draw [color=c, fill=c] (2.47761,2.7256) rectangle (2.51741,2.83145);
\draw [color=c, fill=c] (2.51741,2.7256) rectangle (2.55721,2.83145);
\draw [color=c, fill=c] (2.55721,2.7256) rectangle (2.59702,2.83145);
\draw [color=c, fill=c] (2.59702,2.7256) rectangle (2.63682,2.83145);
\draw [color=c, fill=c] (2.63682,2.7256) rectangle (2.67662,2.83145);
\draw [color=c, fill=c] (2.67662,2.7256) rectangle (2.71642,2.83145);
\draw [color=c, fill=c] (2.71642,2.7256) rectangle (2.75622,2.83145);
\draw [color=c, fill=c] (2.75622,2.7256) rectangle (2.79602,2.83145);
\draw [color=c, fill=c] (2.79602,2.7256) rectangle (2.83582,2.83145);
\draw [color=c, fill=c] (2.83582,2.7256) rectangle (2.87562,2.83145);
\draw [color=c, fill=c] (2.87562,2.7256) rectangle (2.91542,2.83145);
\draw [color=c, fill=c] (2.91542,2.7256) rectangle (2.95522,2.83145);
\draw [color=c, fill=c] (2.95522,2.7256) rectangle (2.99502,2.83145);
\draw [color=c, fill=c] (2.99502,2.7256) rectangle (3.03483,2.83145);
\draw [color=c, fill=c] (3.03483,2.7256) rectangle (3.07463,2.83145);
\draw [color=c, fill=c] (3.07463,2.7256) rectangle (3.11443,2.83145);
\draw [color=c, fill=c] (3.11443,2.7256) rectangle (3.15423,2.83145);
\draw [color=c, fill=c] (3.15423,2.7256) rectangle (3.19403,2.83145);
\draw [color=c, fill=c] (3.19403,2.7256) rectangle (3.23383,2.83145);
\draw [color=c, fill=c] (3.23383,2.7256) rectangle (3.27363,2.83145);
\draw [color=c, fill=c] (3.27363,2.7256) rectangle (3.31343,2.83145);
\draw [color=c, fill=c] (3.31343,2.7256) rectangle (3.35323,2.83145);
\draw [color=c, fill=c] (3.35323,2.7256) rectangle (3.39303,2.83145);
\draw [color=c, fill=c] (3.39303,2.7256) rectangle (3.43284,2.83145);
\draw [color=c, fill=c] (3.43284,2.7256) rectangle (3.47264,2.83145);
\draw [color=c, fill=c] (3.47264,2.7256) rectangle (3.51244,2.83145);
\draw [color=c, fill=c] (3.51244,2.7256) rectangle (3.55224,2.83145);
\draw [color=c, fill=c] (3.55224,2.7256) rectangle (3.59204,2.83145);
\draw [color=c, fill=c] (3.59204,2.7256) rectangle (3.63184,2.83145);
\draw [color=c, fill=c] (3.63184,2.7256) rectangle (3.67164,2.83145);
\draw [color=c, fill=c] (3.67164,2.7256) rectangle (3.71144,2.83145);
\draw [color=c, fill=c] (3.71144,2.7256) rectangle (3.75124,2.83145);
\draw [color=c, fill=c] (3.75124,2.7256) rectangle (3.79104,2.83145);
\draw [color=c, fill=c] (3.79104,2.7256) rectangle (3.83085,2.83145);
\draw [color=c, fill=c] (3.83085,2.7256) rectangle (3.87065,2.83145);
\draw [color=c, fill=c] (3.87065,2.7256) rectangle (3.91045,2.83145);
\draw [color=c, fill=c] (3.91045,2.7256) rectangle (3.95025,2.83145);
\draw [color=c, fill=c] (3.95025,2.7256) rectangle (3.99005,2.83145);
\draw [color=c, fill=c] (3.99005,2.7256) rectangle (4.02985,2.83145);
\draw [color=c, fill=c] (4.02985,2.7256) rectangle (4.06965,2.83145);
\draw [color=c, fill=c] (4.06965,2.7256) rectangle (4.10945,2.83145);
\draw [color=c, fill=c] (4.10945,2.7256) rectangle (4.14925,2.83145);
\draw [color=c, fill=c] (4.14925,2.7256) rectangle (4.18905,2.83145);
\draw [color=c, fill=c] (4.18905,2.7256) rectangle (4.22886,2.83145);
\draw [color=c, fill=c] (4.22886,2.7256) rectangle (4.26866,2.83145);
\draw [color=c, fill=c] (4.26866,2.7256) rectangle (4.30846,2.83145);
\draw [color=c, fill=c] (4.30846,2.7256) rectangle (4.34826,2.83145);
\draw [color=c, fill=c] (4.34826,2.7256) rectangle (4.38806,2.83145);
\draw [color=c, fill=c] (4.38806,2.7256) rectangle (4.42786,2.83145);
\draw [color=c, fill=c] (4.42786,2.7256) rectangle (4.46766,2.83145);
\draw [color=c, fill=c] (4.46766,2.7256) rectangle (4.50746,2.83145);
\draw [color=c, fill=c] (4.50746,2.7256) rectangle (4.54726,2.83145);
\draw [color=c, fill=c] (4.54726,2.7256) rectangle (4.58706,2.83145);
\draw [color=c, fill=c] (4.58706,2.7256) rectangle (4.62687,2.83145);
\draw [color=c, fill=c] (4.62687,2.7256) rectangle (4.66667,2.83145);
\draw [color=c, fill=c] (4.66667,2.7256) rectangle (4.70647,2.83145);
\draw [color=c, fill=c] (4.70647,2.7256) rectangle (4.74627,2.83145);
\draw [color=c, fill=c] (4.74627,2.7256) rectangle (4.78607,2.83145);
\draw [color=c, fill=c] (4.78607,2.7256) rectangle (4.82587,2.83145);
\draw [color=c, fill=c] (4.82587,2.7256) rectangle (4.86567,2.83145);
\draw [color=c, fill=c] (4.86567,2.7256) rectangle (4.90547,2.83145);
\draw [color=c, fill=c] (4.90547,2.7256) rectangle (4.94527,2.83145);
\draw [color=c, fill=c] (4.94527,2.7256) rectangle (4.98507,2.83145);
\draw [color=c, fill=c] (4.98507,2.7256) rectangle (5.02488,2.83145);
\draw [color=c, fill=c] (5.02488,2.7256) rectangle (5.06468,2.83145);
\draw [color=c, fill=c] (5.06468,2.7256) rectangle (5.10448,2.83145);
\draw [color=c, fill=c] (5.10448,2.7256) rectangle (5.14428,2.83145);
\draw [color=c, fill=c] (5.14428,2.7256) rectangle (5.18408,2.83145);
\draw [color=c, fill=c] (5.18408,2.7256) rectangle (5.22388,2.83145);
\draw [color=c, fill=c] (5.22388,2.7256) rectangle (5.26368,2.83145);
\draw [color=c, fill=c] (5.26368,2.7256) rectangle (5.30348,2.83145);
\draw [color=c, fill=c] (5.30348,2.7256) rectangle (5.34328,2.83145);
\draw [color=c, fill=c] (5.34328,2.7256) rectangle (5.38308,2.83145);
\draw [color=c, fill=c] (5.38308,2.7256) rectangle (5.42289,2.83145);
\draw [color=c, fill=c] (5.42289,2.7256) rectangle (5.46269,2.83145);
\draw [color=c, fill=c] (5.46269,2.7256) rectangle (5.50249,2.83145);
\draw [color=c, fill=c] (5.50249,2.7256) rectangle (5.54229,2.83145);
\draw [color=c, fill=c] (5.54229,2.7256) rectangle (5.58209,2.83145);
\draw [color=c, fill=c] (5.58209,2.7256) rectangle (5.62189,2.83145);
\draw [color=c, fill=c] (5.62189,2.7256) rectangle (5.66169,2.83145);
\draw [color=c, fill=c] (5.66169,2.7256) rectangle (5.70149,2.83145);
\draw [color=c, fill=c] (5.70149,2.7256) rectangle (5.74129,2.83145);
\draw [color=c, fill=c] (5.74129,2.7256) rectangle (5.78109,2.83145);
\draw [color=c, fill=c] (5.78109,2.7256) rectangle (5.8209,2.83145);
\draw [color=c, fill=c] (5.8209,2.7256) rectangle (5.8607,2.83145);
\draw [color=c, fill=c] (5.8607,2.7256) rectangle (5.9005,2.83145);
\draw [color=c, fill=c] (5.9005,2.7256) rectangle (5.9403,2.83145);
\draw [color=c, fill=c] (5.9403,2.7256) rectangle (5.9801,2.83145);
\draw [color=c, fill=c] (5.9801,2.7256) rectangle (6.0199,2.83145);
\draw [color=c, fill=c] (6.0199,2.7256) rectangle (6.0597,2.83145);
\draw [color=c, fill=c] (6.0597,2.7256) rectangle (6.0995,2.83145);
\draw [color=c, fill=c] (6.0995,2.7256) rectangle (6.1393,2.83145);
\draw [color=c, fill=c] (6.1393,2.7256) rectangle (6.1791,2.83145);
\draw [color=c, fill=c] (6.1791,2.7256) rectangle (6.21891,2.83145);
\draw [color=c, fill=c] (6.21891,2.7256) rectangle (6.25871,2.83145);
\draw [color=c, fill=c] (6.25871,2.7256) rectangle (6.29851,2.83145);
\draw [color=c, fill=c] (6.29851,2.7256) rectangle (6.33831,2.83145);
\draw [color=c, fill=c] (6.33831,2.7256) rectangle (6.37811,2.83145);
\draw [color=c, fill=c] (6.37811,2.7256) rectangle (6.41791,2.83145);
\draw [color=c, fill=c] (6.41791,2.7256) rectangle (6.45771,2.83145);
\draw [color=c, fill=c] (6.45771,2.7256) rectangle (6.49751,2.83145);
\draw [color=c, fill=c] (6.49751,2.7256) rectangle (6.53731,2.83145);
\draw [color=c, fill=c] (6.53731,2.7256) rectangle (6.57711,2.83145);
\draw [color=c, fill=c] (6.57711,2.7256) rectangle (6.61692,2.83145);
\draw [color=c, fill=c] (6.61692,2.7256) rectangle (6.65672,2.83145);
\draw [color=c, fill=c] (6.65672,2.7256) rectangle (6.69652,2.83145);
\draw [color=c, fill=c] (6.69652,2.7256) rectangle (6.73632,2.83145);
\draw [color=c, fill=c] (6.73632,2.7256) rectangle (6.77612,2.83145);
\draw [color=c, fill=c] (6.77612,2.7256) rectangle (6.81592,2.83145);
\draw [color=c, fill=c] (6.81592,2.7256) rectangle (6.85572,2.83145);
\draw [color=c, fill=c] (6.85572,2.7256) rectangle (6.89552,2.83145);
\draw [color=c, fill=c] (6.89552,2.7256) rectangle (6.93532,2.83145);
\draw [color=c, fill=c] (6.93532,2.7256) rectangle (6.97512,2.83145);
\draw [color=c, fill=c] (6.97512,2.7256) rectangle (7.01493,2.83145);
\draw [color=c, fill=c] (7.01493,2.7256) rectangle (7.05473,2.83145);
\draw [color=c, fill=c] (7.05473,2.7256) rectangle (7.09453,2.83145);
\draw [color=c, fill=c] (7.09453,2.7256) rectangle (7.13433,2.83145);
\draw [color=c, fill=c] (7.13433,2.7256) rectangle (7.17413,2.83145);
\draw [color=c, fill=c] (7.17413,2.7256) rectangle (7.21393,2.83145);
\draw [color=c, fill=c] (7.21393,2.7256) rectangle (7.25373,2.83145);
\draw [color=c, fill=c] (7.25373,2.7256) rectangle (7.29353,2.83145);
\draw [color=c, fill=c] (7.29353,2.7256) rectangle (7.33333,2.83145);
\draw [color=c, fill=c] (7.33333,2.7256) rectangle (7.37313,2.83145);
\draw [color=c, fill=c] (7.37313,2.7256) rectangle (7.41294,2.83145);
\draw [color=c, fill=c] (7.41294,2.7256) rectangle (7.45274,2.83145);
\draw [color=c, fill=c] (7.45274,2.7256) rectangle (7.49254,2.83145);
\draw [color=c, fill=c] (7.49254,2.7256) rectangle (7.53234,2.83145);
\draw [color=c, fill=c] (7.53234,2.7256) rectangle (7.57214,2.83145);
\draw [color=c, fill=c] (7.57214,2.7256) rectangle (7.61194,2.83145);
\draw [color=c, fill=c] (7.61194,2.7256) rectangle (7.65174,2.83145);
\draw [color=c, fill=c] (7.65174,2.7256) rectangle (7.69154,2.83145);
\draw [color=c, fill=c] (7.69154,2.7256) rectangle (7.73134,2.83145);
\draw [color=c, fill=c] (7.73134,2.7256) rectangle (7.77114,2.83145);
\draw [color=c, fill=c] (7.77114,2.7256) rectangle (7.81095,2.83145);
\draw [color=c, fill=c] (7.81095,2.7256) rectangle (7.85075,2.83145);
\draw [color=c, fill=c] (7.85075,2.7256) rectangle (7.89055,2.83145);
\definecolor{c}{rgb}{1,0.186667,0};
\draw [color=c, fill=c] (7.89055,2.7256) rectangle (7.93035,2.83145);
\draw [color=c, fill=c] (7.93035,2.7256) rectangle (7.97015,2.83145);
\draw [color=c, fill=c] (7.97015,2.7256) rectangle (8.00995,2.83145);
\draw [color=c, fill=c] (8.00995,2.7256) rectangle (8.04975,2.83145);
\draw [color=c, fill=c] (8.04975,2.7256) rectangle (8.08955,2.83145);
\draw [color=c, fill=c] (8.08955,2.7256) rectangle (8.12935,2.83145);
\draw [color=c, fill=c] (8.12935,2.7256) rectangle (8.16915,2.83145);
\draw [color=c, fill=c] (8.16915,2.7256) rectangle (8.20895,2.83145);
\draw [color=c, fill=c] (8.20895,2.7256) rectangle (8.24876,2.83145);
\draw [color=c, fill=c] (8.24876,2.7256) rectangle (8.28856,2.83145);
\draw [color=c, fill=c] (8.28856,2.7256) rectangle (8.32836,2.83145);
\draw [color=c, fill=c] (8.32836,2.7256) rectangle (8.36816,2.83145);
\draw [color=c, fill=c] (8.36816,2.7256) rectangle (8.40796,2.83145);
\draw [color=c, fill=c] (8.40796,2.7256) rectangle (8.44776,2.83145);
\draw [color=c, fill=c] (8.44776,2.7256) rectangle (8.48756,2.83145);
\draw [color=c, fill=c] (8.48756,2.7256) rectangle (8.52736,2.83145);
\draw [color=c, fill=c] (8.52736,2.7256) rectangle (8.56716,2.83145);
\draw [color=c, fill=c] (8.56716,2.7256) rectangle (8.60697,2.83145);
\draw [color=c, fill=c] (8.60697,2.7256) rectangle (8.64677,2.83145);
\draw [color=c, fill=c] (8.64677,2.7256) rectangle (8.68657,2.83145);
\draw [color=c, fill=c] (8.68657,2.7256) rectangle (8.72637,2.83145);
\definecolor{c}{rgb}{1,0.466667,0};
\draw [color=c, fill=c] (8.72637,2.7256) rectangle (8.76617,2.83145);
\draw [color=c, fill=c] (8.76617,2.7256) rectangle (8.80597,2.83145);
\draw [color=c, fill=c] (8.80597,2.7256) rectangle (8.84577,2.83145);
\draw [color=c, fill=c] (8.84577,2.7256) rectangle (8.88557,2.83145);
\draw [color=c, fill=c] (8.88557,2.7256) rectangle (8.92537,2.83145);
\draw [color=c, fill=c] (8.92537,2.7256) rectangle (8.96517,2.83145);
\draw [color=c, fill=c] (8.96517,2.7256) rectangle (9.00498,2.83145);
\draw [color=c, fill=c] (9.00498,2.7256) rectangle (9.04478,2.83145);
\draw [color=c, fill=c] (9.04478,2.7256) rectangle (9.08458,2.83145);
\draw [color=c, fill=c] (9.08458,2.7256) rectangle (9.12438,2.83145);
\draw [color=c, fill=c] (9.12438,2.7256) rectangle (9.16418,2.83145);
\draw [color=c, fill=c] (9.16418,2.7256) rectangle (9.20398,2.83145);
\definecolor{c}{rgb}{1,0.653333,0};
\draw [color=c, fill=c] (9.20398,2.7256) rectangle (9.24378,2.83145);
\draw [color=c, fill=c] (9.24378,2.7256) rectangle (9.28358,2.83145);
\draw [color=c, fill=c] (9.28358,2.7256) rectangle (9.32338,2.83145);
\draw [color=c, fill=c] (9.32338,2.7256) rectangle (9.36318,2.83145);
\draw [color=c, fill=c] (9.36318,2.7256) rectangle (9.40298,2.83145);
\draw [color=c, fill=c] (9.40298,2.7256) rectangle (9.44279,2.83145);
\draw [color=c, fill=c] (9.44279,2.7256) rectangle (9.48259,2.83145);
\definecolor{c}{rgb}{1,0.933333,0};
\draw [color=c, fill=c] (9.48259,2.7256) rectangle (9.52239,2.83145);
\draw [color=c, fill=c] (9.52239,2.7256) rectangle (9.56219,2.83145);
\draw [color=c, fill=c] (9.56219,2.7256) rectangle (9.60199,2.83145);
\draw [color=c, fill=c] (9.60199,2.7256) rectangle (9.64179,2.83145);
\definecolor{c}{rgb}{0.88,1,0};
\draw [color=c, fill=c] (9.64179,2.7256) rectangle (9.68159,2.83145);
\draw [color=c, fill=c] (9.68159,2.7256) rectangle (9.72139,2.83145);
\draw [color=c, fill=c] (9.72139,2.7256) rectangle (9.76119,2.83145);
\draw [color=c, fill=c] (9.76119,2.7256) rectangle (9.80099,2.83145);
\definecolor{c}{rgb}{0.6,1,0};
\draw [color=c, fill=c] (9.80099,2.7256) rectangle (9.8408,2.83145);
\draw [color=c, fill=c] (9.8408,2.7256) rectangle (9.8806,2.83145);
\draw [color=c, fill=c] (9.8806,2.7256) rectangle (9.9204,2.83145);
\definecolor{c}{rgb}{0.413333,1,0};
\draw [color=c, fill=c] (9.9204,2.7256) rectangle (9.9602,2.83145);
\draw [color=c, fill=c] (9.9602,2.7256) rectangle (10,2.83145);
\draw [color=c, fill=c] (10,2.7256) rectangle (10.0398,2.83145);
\definecolor{c}{rgb}{0.133333,1,0};
\draw [color=c, fill=c] (10.0398,2.7256) rectangle (10.0796,2.83145);
\draw [color=c, fill=c] (10.0796,2.7256) rectangle (10.1194,2.83145);
\draw [color=c, fill=c] (10.1194,2.7256) rectangle (10.1592,2.83145);
\definecolor{c}{rgb}{0,1,0.0533333};
\draw [color=c, fill=c] (10.1592,2.7256) rectangle (10.199,2.83145);
\draw [color=c, fill=c] (10.199,2.7256) rectangle (10.2388,2.83145);
\draw [color=c, fill=c] (10.2388,2.7256) rectangle (10.2786,2.83145);
\definecolor{c}{rgb}{0,1,0.333333};
\draw [color=c, fill=c] (10.2786,2.7256) rectangle (10.3184,2.83145);
\draw [color=c, fill=c] (10.3184,2.7256) rectangle (10.3582,2.83145);
\draw [color=c, fill=c] (10.3582,2.7256) rectangle (10.398,2.83145);
\draw [color=c, fill=c] (10.398,2.7256) rectangle (10.4378,2.83145);
\draw [color=c, fill=c] (10.4378,2.7256) rectangle (10.4776,2.83145);
\definecolor{c}{rgb}{0,1,0.52};
\draw [color=c, fill=c] (10.4776,2.7256) rectangle (10.5174,2.83145);
\draw [color=c, fill=c] (10.5174,2.7256) rectangle (10.5572,2.83145);
\draw [color=c, fill=c] (10.5572,2.7256) rectangle (10.597,2.83145);
\draw [color=c, fill=c] (10.597,2.7256) rectangle (10.6368,2.83145);
\draw [color=c, fill=c] (10.6368,2.7256) rectangle (10.6766,2.83145);
\draw [color=c, fill=c] (10.6766,2.7256) rectangle (10.7164,2.83145);
\draw [color=c, fill=c] (10.7164,2.7256) rectangle (10.7562,2.83145);
\draw [color=c, fill=c] (10.7562,2.7256) rectangle (10.796,2.83145);
\definecolor{c}{rgb}{0,1,0.8};
\draw [color=c, fill=c] (10.796,2.7256) rectangle (10.8358,2.83145);
\draw [color=c, fill=c] (10.8358,2.7256) rectangle (10.8756,2.83145);
\draw [color=c, fill=c] (10.8756,2.7256) rectangle (10.9154,2.83145);
\draw [color=c, fill=c] (10.9154,2.7256) rectangle (10.9552,2.83145);
\draw [color=c, fill=c] (10.9552,2.7256) rectangle (10.995,2.83145);
\draw [color=c, fill=c] (10.995,2.7256) rectangle (11.0348,2.83145);
\draw [color=c, fill=c] (11.0348,2.7256) rectangle (11.0746,2.83145);
\draw [color=c, fill=c] (11.0746,2.7256) rectangle (11.1144,2.83145);
\draw [color=c, fill=c] (11.1144,2.7256) rectangle (11.1542,2.83145);
\draw [color=c, fill=c] (11.1542,2.7256) rectangle (11.194,2.83145);
\draw [color=c, fill=c] (11.194,2.7256) rectangle (11.2338,2.83145);
\draw [color=c, fill=c] (11.2338,2.7256) rectangle (11.2736,2.83145);
\definecolor{c}{rgb}{0,1,0.986667};
\draw [color=c, fill=c] (11.2736,2.7256) rectangle (11.3134,2.83145);
\draw [color=c, fill=c] (11.3134,2.7256) rectangle (11.3532,2.83145);
\draw [color=c, fill=c] (11.3532,2.7256) rectangle (11.393,2.83145);
\draw [color=c, fill=c] (11.393,2.7256) rectangle (11.4328,2.83145);
\draw [color=c, fill=c] (11.4328,2.7256) rectangle (11.4726,2.83145);
\draw [color=c, fill=c] (11.4726,2.7256) rectangle (11.5124,2.83145);
\draw [color=c, fill=c] (11.5124,2.7256) rectangle (11.5522,2.83145);
\draw [color=c, fill=c] (11.5522,2.7256) rectangle (11.592,2.83145);
\draw [color=c, fill=c] (11.592,2.7256) rectangle (11.6318,2.83145);
\draw [color=c, fill=c] (11.6318,2.7256) rectangle (11.6716,2.83145);
\draw [color=c, fill=c] (11.6716,2.7256) rectangle (11.7114,2.83145);
\draw [color=c, fill=c] (11.7114,2.7256) rectangle (11.7512,2.83145);
\draw [color=c, fill=c] (11.7512,2.7256) rectangle (11.791,2.83145);
\draw [color=c, fill=c] (11.791,2.7256) rectangle (11.8308,2.83145);
\draw [color=c, fill=c] (11.8308,2.7256) rectangle (11.8706,2.83145);
\draw [color=c, fill=c] (11.8706,2.7256) rectangle (11.9104,2.83145);
\draw [color=c, fill=c] (11.9104,2.7256) rectangle (11.9502,2.83145);
\draw [color=c, fill=c] (11.9502,2.7256) rectangle (11.99,2.83145);
\draw [color=c, fill=c] (11.99,2.7256) rectangle (12.0299,2.83145);
\draw [color=c, fill=c] (12.0299,2.7256) rectangle (12.0697,2.83145);
\draw [color=c, fill=c] (12.0697,2.7256) rectangle (12.1095,2.83145);
\draw [color=c, fill=c] (12.1095,2.7256) rectangle (12.1493,2.83145);
\draw [color=c, fill=c] (12.1493,2.7256) rectangle (12.1891,2.83145);
\draw [color=c, fill=c] (12.1891,2.7256) rectangle (12.2289,2.83145);
\definecolor{c}{rgb}{0,0.733333,1};
\draw [color=c, fill=c] (12.2289,2.7256) rectangle (12.2687,2.83145);
\draw [color=c, fill=c] (12.2687,2.7256) rectangle (12.3085,2.83145);
\draw [color=c, fill=c] (12.3085,2.7256) rectangle (12.3483,2.83145);
\draw [color=c, fill=c] (12.3483,2.7256) rectangle (12.3881,2.83145);
\draw [color=c, fill=c] (12.3881,2.7256) rectangle (12.4279,2.83145);
\draw [color=c, fill=c] (12.4279,2.7256) rectangle (12.4677,2.83145);
\draw [color=c, fill=c] (12.4677,2.7256) rectangle (12.5075,2.83145);
\draw [color=c, fill=c] (12.5075,2.7256) rectangle (12.5473,2.83145);
\draw [color=c, fill=c] (12.5473,2.7256) rectangle (12.5871,2.83145);
\draw [color=c, fill=c] (12.5871,2.7256) rectangle (12.6269,2.83145);
\draw [color=c, fill=c] (12.6269,2.7256) rectangle (12.6667,2.83145);
\draw [color=c, fill=c] (12.6667,2.7256) rectangle (12.7065,2.83145);
\draw [color=c, fill=c] (12.7065,2.7256) rectangle (12.7463,2.83145);
\draw [color=c, fill=c] (12.7463,2.7256) rectangle (12.7861,2.83145);
\draw [color=c, fill=c] (12.7861,2.7256) rectangle (12.8259,2.83145);
\draw [color=c, fill=c] (12.8259,2.7256) rectangle (12.8657,2.83145);
\draw [color=c, fill=c] (12.8657,2.7256) rectangle (12.9055,2.83145);
\draw [color=c, fill=c] (12.9055,2.7256) rectangle (12.9453,2.83145);
\draw [color=c, fill=c] (12.9453,2.7256) rectangle (12.9851,2.83145);
\draw [color=c, fill=c] (12.9851,2.7256) rectangle (13.0249,2.83145);
\draw [color=c, fill=c] (13.0249,2.7256) rectangle (13.0647,2.83145);
\draw [color=c, fill=c] (13.0647,2.7256) rectangle (13.1045,2.83145);
\draw [color=c, fill=c] (13.1045,2.7256) rectangle (13.1443,2.83145);
\draw [color=c, fill=c] (13.1443,2.7256) rectangle (13.1841,2.83145);
\draw [color=c, fill=c] (13.1841,2.7256) rectangle (13.2239,2.83145);
\draw [color=c, fill=c] (13.2239,2.7256) rectangle (13.2637,2.83145);
\draw [color=c, fill=c] (13.2637,2.7256) rectangle (13.3035,2.83145);
\draw [color=c, fill=c] (13.3035,2.7256) rectangle (13.3433,2.83145);
\draw [color=c, fill=c] (13.3433,2.7256) rectangle (13.3831,2.83145);
\draw [color=c, fill=c] (13.3831,2.7256) rectangle (13.4229,2.83145);
\draw [color=c, fill=c] (13.4229,2.7256) rectangle (13.4627,2.83145);
\draw [color=c, fill=c] (13.4627,2.7256) rectangle (13.5025,2.83145);
\draw [color=c, fill=c] (13.5025,2.7256) rectangle (13.5423,2.83145);
\draw [color=c, fill=c] (13.5423,2.7256) rectangle (13.5821,2.83145);
\draw [color=c, fill=c] (13.5821,2.7256) rectangle (13.6219,2.83145);
\draw [color=c, fill=c] (13.6219,2.7256) rectangle (13.6617,2.83145);
\draw [color=c, fill=c] (13.6617,2.7256) rectangle (13.7015,2.83145);
\draw [color=c, fill=c] (13.7015,2.7256) rectangle (13.7413,2.83145);
\draw [color=c, fill=c] (13.7413,2.7256) rectangle (13.7811,2.83145);
\draw [color=c, fill=c] (13.7811,2.7256) rectangle (13.8209,2.83145);
\draw [color=c, fill=c] (13.8209,2.7256) rectangle (13.8607,2.83145);
\draw [color=c, fill=c] (13.8607,2.7256) rectangle (13.9005,2.83145);
\draw [color=c, fill=c] (13.9005,2.7256) rectangle (13.9403,2.83145);
\draw [color=c, fill=c] (13.9403,2.7256) rectangle (13.9801,2.83145);
\draw [color=c, fill=c] (13.9801,2.7256) rectangle (14.0199,2.83145);
\draw [color=c, fill=c] (14.0199,2.7256) rectangle (14.0597,2.83145);
\draw [color=c, fill=c] (14.0597,2.7256) rectangle (14.0995,2.83145);
\draw [color=c, fill=c] (14.0995,2.7256) rectangle (14.1393,2.83145);
\draw [color=c, fill=c] (14.1393,2.7256) rectangle (14.1791,2.83145);
\draw [color=c, fill=c] (14.1791,2.7256) rectangle (14.2189,2.83145);
\draw [color=c, fill=c] (14.2189,2.7256) rectangle (14.2587,2.83145);
\draw [color=c, fill=c] (14.2587,2.7256) rectangle (14.2985,2.83145);
\draw [color=c, fill=c] (14.2985,2.7256) rectangle (14.3383,2.83145);
\draw [color=c, fill=c] (14.3383,2.7256) rectangle (14.3781,2.83145);
\draw [color=c, fill=c] (14.3781,2.7256) rectangle (14.4179,2.83145);
\draw [color=c, fill=c] (14.4179,2.7256) rectangle (14.4577,2.83145);
\draw [color=c, fill=c] (14.4577,2.7256) rectangle (14.4975,2.83145);
\draw [color=c, fill=c] (14.4975,2.7256) rectangle (14.5373,2.83145);
\draw [color=c, fill=c] (14.5373,2.7256) rectangle (14.5771,2.83145);
\draw [color=c, fill=c] (14.5771,2.7256) rectangle (14.6169,2.83145);
\draw [color=c, fill=c] (14.6169,2.7256) rectangle (14.6567,2.83145);
\draw [color=c, fill=c] (14.6567,2.7256) rectangle (14.6965,2.83145);
\draw [color=c, fill=c] (14.6965,2.7256) rectangle (14.7363,2.83145);
\draw [color=c, fill=c] (14.7363,2.7256) rectangle (14.7761,2.83145);
\draw [color=c, fill=c] (14.7761,2.7256) rectangle (14.8159,2.83145);
\draw [color=c, fill=c] (14.8159,2.7256) rectangle (14.8557,2.83145);
\draw [color=c, fill=c] (14.8557,2.7256) rectangle (14.8955,2.83145);
\draw [color=c, fill=c] (14.8955,2.7256) rectangle (14.9353,2.83145);
\draw [color=c, fill=c] (14.9353,2.7256) rectangle (14.9751,2.83145);
\draw [color=c, fill=c] (14.9751,2.7256) rectangle (15.0149,2.83145);
\draw [color=c, fill=c] (15.0149,2.7256) rectangle (15.0547,2.83145);
\draw [color=c, fill=c] (15.0547,2.7256) rectangle (15.0945,2.83145);
\draw [color=c, fill=c] (15.0945,2.7256) rectangle (15.1343,2.83145);
\draw [color=c, fill=c] (15.1343,2.7256) rectangle (15.1741,2.83145);
\draw [color=c, fill=c] (15.1741,2.7256) rectangle (15.2139,2.83145);
\draw [color=c, fill=c] (15.2139,2.7256) rectangle (15.2537,2.83145);
\draw [color=c, fill=c] (15.2537,2.7256) rectangle (15.2935,2.83145);
\draw [color=c, fill=c] (15.2935,2.7256) rectangle (15.3333,2.83145);
\draw [color=c, fill=c] (15.3333,2.7256) rectangle (15.3731,2.83145);
\draw [color=c, fill=c] (15.3731,2.7256) rectangle (15.4129,2.83145);
\draw [color=c, fill=c] (15.4129,2.7256) rectangle (15.4527,2.83145);
\draw [color=c, fill=c] (15.4527,2.7256) rectangle (15.4925,2.83145);
\draw [color=c, fill=c] (15.4925,2.7256) rectangle (15.5323,2.83145);
\draw [color=c, fill=c] (15.5323,2.7256) rectangle (15.5721,2.83145);
\draw [color=c, fill=c] (15.5721,2.7256) rectangle (15.6119,2.83145);
\draw [color=c, fill=c] (15.6119,2.7256) rectangle (15.6517,2.83145);
\draw [color=c, fill=c] (15.6517,2.7256) rectangle (15.6915,2.83145);
\draw [color=c, fill=c] (15.6915,2.7256) rectangle (15.7313,2.83145);
\draw [color=c, fill=c] (15.7313,2.7256) rectangle (15.7711,2.83145);
\draw [color=c, fill=c] (15.7711,2.7256) rectangle (15.8109,2.83145);
\draw [color=c, fill=c] (15.8109,2.7256) rectangle (15.8507,2.83145);
\draw [color=c, fill=c] (15.8507,2.7256) rectangle (15.8905,2.83145);
\draw [color=c, fill=c] (15.8905,2.7256) rectangle (15.9303,2.83145);
\draw [color=c, fill=c] (15.9303,2.7256) rectangle (15.9701,2.83145);
\draw [color=c, fill=c] (15.9701,2.7256) rectangle (16.01,2.83145);
\draw [color=c, fill=c] (16.01,2.7256) rectangle (16.0498,2.83145);
\draw [color=c, fill=c] (16.0498,2.7256) rectangle (16.0896,2.83145);
\draw [color=c, fill=c] (16.0896,2.7256) rectangle (16.1294,2.83145);
\draw [color=c, fill=c] (16.1294,2.7256) rectangle (16.1692,2.83145);
\draw [color=c, fill=c] (16.1692,2.7256) rectangle (16.209,2.83145);
\draw [color=c, fill=c] (16.209,2.7256) rectangle (16.2488,2.83145);
\draw [color=c, fill=c] (16.2488,2.7256) rectangle (16.2886,2.83145);
\draw [color=c, fill=c] (16.2886,2.7256) rectangle (16.3284,2.83145);
\draw [color=c, fill=c] (16.3284,2.7256) rectangle (16.3682,2.83145);
\draw [color=c, fill=c] (16.3682,2.7256) rectangle (16.408,2.83145);
\draw [color=c, fill=c] (16.408,2.7256) rectangle (16.4478,2.83145);
\draw [color=c, fill=c] (16.4478,2.7256) rectangle (16.4876,2.83145);
\draw [color=c, fill=c] (16.4876,2.7256) rectangle (16.5274,2.83145);
\draw [color=c, fill=c] (16.5274,2.7256) rectangle (16.5672,2.83145);
\draw [color=c, fill=c] (16.5672,2.7256) rectangle (16.607,2.83145);
\draw [color=c, fill=c] (16.607,2.7256) rectangle (16.6468,2.83145);
\draw [color=c, fill=c] (16.6468,2.7256) rectangle (16.6866,2.83145);
\draw [color=c, fill=c] (16.6866,2.7256) rectangle (16.7264,2.83145);
\draw [color=c, fill=c] (16.7264,2.7256) rectangle (16.7662,2.83145);
\draw [color=c, fill=c] (16.7662,2.7256) rectangle (16.806,2.83145);
\draw [color=c, fill=c] (16.806,2.7256) rectangle (16.8458,2.83145);
\draw [color=c, fill=c] (16.8458,2.7256) rectangle (16.8856,2.83145);
\draw [color=c, fill=c] (16.8856,2.7256) rectangle (16.9254,2.83145);
\draw [color=c, fill=c] (16.9254,2.7256) rectangle (16.9652,2.83145);
\draw [color=c, fill=c] (16.9652,2.7256) rectangle (17.005,2.83145);
\draw [color=c, fill=c] (17.005,2.7256) rectangle (17.0448,2.83145);
\draw [color=c, fill=c] (17.0448,2.7256) rectangle (17.0846,2.83145);
\draw [color=c, fill=c] (17.0846,2.7256) rectangle (17.1244,2.83145);
\draw [color=c, fill=c] (17.1244,2.7256) rectangle (17.1642,2.83145);
\draw [color=c, fill=c] (17.1642,2.7256) rectangle (17.204,2.83145);
\draw [color=c, fill=c] (17.204,2.7256) rectangle (17.2438,2.83145);
\draw [color=c, fill=c] (17.2438,2.7256) rectangle (17.2836,2.83145);
\draw [color=c, fill=c] (17.2836,2.7256) rectangle (17.3234,2.83145);
\draw [color=c, fill=c] (17.3234,2.7256) rectangle (17.3632,2.83145);
\draw [color=c, fill=c] (17.3632,2.7256) rectangle (17.403,2.83145);
\draw [color=c, fill=c] (17.403,2.7256) rectangle (17.4428,2.83145);
\draw [color=c, fill=c] (17.4428,2.7256) rectangle (17.4826,2.83145);
\draw [color=c, fill=c] (17.4826,2.7256) rectangle (17.5224,2.83145);
\draw [color=c, fill=c] (17.5224,2.7256) rectangle (17.5622,2.83145);
\draw [color=c, fill=c] (17.5622,2.7256) rectangle (17.602,2.83145);
\draw [color=c, fill=c] (17.602,2.7256) rectangle (17.6418,2.83145);
\draw [color=c, fill=c] (17.6418,2.7256) rectangle (17.6816,2.83145);
\draw [color=c, fill=c] (17.6816,2.7256) rectangle (17.7214,2.83145);
\draw [color=c, fill=c] (17.7214,2.7256) rectangle (17.7612,2.83145);
\draw [color=c, fill=c] (17.7612,2.7256) rectangle (17.801,2.83145);
\draw [color=c, fill=c] (17.801,2.7256) rectangle (17.8408,2.83145);
\draw [color=c, fill=c] (17.8408,2.7256) rectangle (17.8806,2.83145);
\draw [color=c, fill=c] (17.8806,2.7256) rectangle (17.9204,2.83145);
\draw [color=c, fill=c] (17.9204,2.7256) rectangle (17.9602,2.83145);
\draw [color=c, fill=c] (17.9602,2.7256) rectangle (18,2.83145);
\definecolor{c}{rgb}{1,0,0};
\draw [color=c, fill=c] (2,2.83145) rectangle (2.0398,2.9373);
\draw [color=c, fill=c] (2.0398,2.83145) rectangle (2.0796,2.9373);
\draw [color=c, fill=c] (2.0796,2.83145) rectangle (2.1194,2.9373);
\draw [color=c, fill=c] (2.1194,2.83145) rectangle (2.1592,2.9373);
\draw [color=c, fill=c] (2.1592,2.83145) rectangle (2.19901,2.9373);
\draw [color=c, fill=c] (2.19901,2.83145) rectangle (2.23881,2.9373);
\draw [color=c, fill=c] (2.23881,2.83145) rectangle (2.27861,2.9373);
\draw [color=c, fill=c] (2.27861,2.83145) rectangle (2.31841,2.9373);
\draw [color=c, fill=c] (2.31841,2.83145) rectangle (2.35821,2.9373);
\draw [color=c, fill=c] (2.35821,2.83145) rectangle (2.39801,2.9373);
\draw [color=c, fill=c] (2.39801,2.83145) rectangle (2.43781,2.9373);
\draw [color=c, fill=c] (2.43781,2.83145) rectangle (2.47761,2.9373);
\draw [color=c, fill=c] (2.47761,2.83145) rectangle (2.51741,2.9373);
\draw [color=c, fill=c] (2.51741,2.83145) rectangle (2.55721,2.9373);
\draw [color=c, fill=c] (2.55721,2.83145) rectangle (2.59702,2.9373);
\draw [color=c, fill=c] (2.59702,2.83145) rectangle (2.63682,2.9373);
\draw [color=c, fill=c] (2.63682,2.83145) rectangle (2.67662,2.9373);
\draw [color=c, fill=c] (2.67662,2.83145) rectangle (2.71642,2.9373);
\draw [color=c, fill=c] (2.71642,2.83145) rectangle (2.75622,2.9373);
\draw [color=c, fill=c] (2.75622,2.83145) rectangle (2.79602,2.9373);
\draw [color=c, fill=c] (2.79602,2.83145) rectangle (2.83582,2.9373);
\draw [color=c, fill=c] (2.83582,2.83145) rectangle (2.87562,2.9373);
\draw [color=c, fill=c] (2.87562,2.83145) rectangle (2.91542,2.9373);
\draw [color=c, fill=c] (2.91542,2.83145) rectangle (2.95522,2.9373);
\draw [color=c, fill=c] (2.95522,2.83145) rectangle (2.99502,2.9373);
\draw [color=c, fill=c] (2.99502,2.83145) rectangle (3.03483,2.9373);
\draw [color=c, fill=c] (3.03483,2.83145) rectangle (3.07463,2.9373);
\draw [color=c, fill=c] (3.07463,2.83145) rectangle (3.11443,2.9373);
\draw [color=c, fill=c] (3.11443,2.83145) rectangle (3.15423,2.9373);
\draw [color=c, fill=c] (3.15423,2.83145) rectangle (3.19403,2.9373);
\draw [color=c, fill=c] (3.19403,2.83145) rectangle (3.23383,2.9373);
\draw [color=c, fill=c] (3.23383,2.83145) rectangle (3.27363,2.9373);
\draw [color=c, fill=c] (3.27363,2.83145) rectangle (3.31343,2.9373);
\draw [color=c, fill=c] (3.31343,2.83145) rectangle (3.35323,2.9373);
\draw [color=c, fill=c] (3.35323,2.83145) rectangle (3.39303,2.9373);
\draw [color=c, fill=c] (3.39303,2.83145) rectangle (3.43284,2.9373);
\draw [color=c, fill=c] (3.43284,2.83145) rectangle (3.47264,2.9373);
\draw [color=c, fill=c] (3.47264,2.83145) rectangle (3.51244,2.9373);
\draw [color=c, fill=c] (3.51244,2.83145) rectangle (3.55224,2.9373);
\draw [color=c, fill=c] (3.55224,2.83145) rectangle (3.59204,2.9373);
\draw [color=c, fill=c] (3.59204,2.83145) rectangle (3.63184,2.9373);
\draw [color=c, fill=c] (3.63184,2.83145) rectangle (3.67164,2.9373);
\draw [color=c, fill=c] (3.67164,2.83145) rectangle (3.71144,2.9373);
\draw [color=c, fill=c] (3.71144,2.83145) rectangle (3.75124,2.9373);
\draw [color=c, fill=c] (3.75124,2.83145) rectangle (3.79104,2.9373);
\draw [color=c, fill=c] (3.79104,2.83145) rectangle (3.83085,2.9373);
\draw [color=c, fill=c] (3.83085,2.83145) rectangle (3.87065,2.9373);
\draw [color=c, fill=c] (3.87065,2.83145) rectangle (3.91045,2.9373);
\draw [color=c, fill=c] (3.91045,2.83145) rectangle (3.95025,2.9373);
\draw [color=c, fill=c] (3.95025,2.83145) rectangle (3.99005,2.9373);
\draw [color=c, fill=c] (3.99005,2.83145) rectangle (4.02985,2.9373);
\draw [color=c, fill=c] (4.02985,2.83145) rectangle (4.06965,2.9373);
\draw [color=c, fill=c] (4.06965,2.83145) rectangle (4.10945,2.9373);
\draw [color=c, fill=c] (4.10945,2.83145) rectangle (4.14925,2.9373);
\draw [color=c, fill=c] (4.14925,2.83145) rectangle (4.18905,2.9373);
\draw [color=c, fill=c] (4.18905,2.83145) rectangle (4.22886,2.9373);
\draw [color=c, fill=c] (4.22886,2.83145) rectangle (4.26866,2.9373);
\draw [color=c, fill=c] (4.26866,2.83145) rectangle (4.30846,2.9373);
\draw [color=c, fill=c] (4.30846,2.83145) rectangle (4.34826,2.9373);
\draw [color=c, fill=c] (4.34826,2.83145) rectangle (4.38806,2.9373);
\draw [color=c, fill=c] (4.38806,2.83145) rectangle (4.42786,2.9373);
\draw [color=c, fill=c] (4.42786,2.83145) rectangle (4.46766,2.9373);
\draw [color=c, fill=c] (4.46766,2.83145) rectangle (4.50746,2.9373);
\draw [color=c, fill=c] (4.50746,2.83145) rectangle (4.54726,2.9373);
\draw [color=c, fill=c] (4.54726,2.83145) rectangle (4.58706,2.9373);
\draw [color=c, fill=c] (4.58706,2.83145) rectangle (4.62687,2.9373);
\draw [color=c, fill=c] (4.62687,2.83145) rectangle (4.66667,2.9373);
\draw [color=c, fill=c] (4.66667,2.83145) rectangle (4.70647,2.9373);
\draw [color=c, fill=c] (4.70647,2.83145) rectangle (4.74627,2.9373);
\draw [color=c, fill=c] (4.74627,2.83145) rectangle (4.78607,2.9373);
\draw [color=c, fill=c] (4.78607,2.83145) rectangle (4.82587,2.9373);
\draw [color=c, fill=c] (4.82587,2.83145) rectangle (4.86567,2.9373);
\draw [color=c, fill=c] (4.86567,2.83145) rectangle (4.90547,2.9373);
\draw [color=c, fill=c] (4.90547,2.83145) rectangle (4.94527,2.9373);
\draw [color=c, fill=c] (4.94527,2.83145) rectangle (4.98507,2.9373);
\draw [color=c, fill=c] (4.98507,2.83145) rectangle (5.02488,2.9373);
\draw [color=c, fill=c] (5.02488,2.83145) rectangle (5.06468,2.9373);
\draw [color=c, fill=c] (5.06468,2.83145) rectangle (5.10448,2.9373);
\draw [color=c, fill=c] (5.10448,2.83145) rectangle (5.14428,2.9373);
\draw [color=c, fill=c] (5.14428,2.83145) rectangle (5.18408,2.9373);
\draw [color=c, fill=c] (5.18408,2.83145) rectangle (5.22388,2.9373);
\draw [color=c, fill=c] (5.22388,2.83145) rectangle (5.26368,2.9373);
\draw [color=c, fill=c] (5.26368,2.83145) rectangle (5.30348,2.9373);
\draw [color=c, fill=c] (5.30348,2.83145) rectangle (5.34328,2.9373);
\draw [color=c, fill=c] (5.34328,2.83145) rectangle (5.38308,2.9373);
\draw [color=c, fill=c] (5.38308,2.83145) rectangle (5.42289,2.9373);
\draw [color=c, fill=c] (5.42289,2.83145) rectangle (5.46269,2.9373);
\draw [color=c, fill=c] (5.46269,2.83145) rectangle (5.50249,2.9373);
\draw [color=c, fill=c] (5.50249,2.83145) rectangle (5.54229,2.9373);
\draw [color=c, fill=c] (5.54229,2.83145) rectangle (5.58209,2.9373);
\draw [color=c, fill=c] (5.58209,2.83145) rectangle (5.62189,2.9373);
\draw [color=c, fill=c] (5.62189,2.83145) rectangle (5.66169,2.9373);
\draw [color=c, fill=c] (5.66169,2.83145) rectangle (5.70149,2.9373);
\draw [color=c, fill=c] (5.70149,2.83145) rectangle (5.74129,2.9373);
\draw [color=c, fill=c] (5.74129,2.83145) rectangle (5.78109,2.9373);
\draw [color=c, fill=c] (5.78109,2.83145) rectangle (5.8209,2.9373);
\draw [color=c, fill=c] (5.8209,2.83145) rectangle (5.8607,2.9373);
\draw [color=c, fill=c] (5.8607,2.83145) rectangle (5.9005,2.9373);
\draw [color=c, fill=c] (5.9005,2.83145) rectangle (5.9403,2.9373);
\draw [color=c, fill=c] (5.9403,2.83145) rectangle (5.9801,2.9373);
\draw [color=c, fill=c] (5.9801,2.83145) rectangle (6.0199,2.9373);
\draw [color=c, fill=c] (6.0199,2.83145) rectangle (6.0597,2.9373);
\draw [color=c, fill=c] (6.0597,2.83145) rectangle (6.0995,2.9373);
\draw [color=c, fill=c] (6.0995,2.83145) rectangle (6.1393,2.9373);
\draw [color=c, fill=c] (6.1393,2.83145) rectangle (6.1791,2.9373);
\draw [color=c, fill=c] (6.1791,2.83145) rectangle (6.21891,2.9373);
\draw [color=c, fill=c] (6.21891,2.83145) rectangle (6.25871,2.9373);
\draw [color=c, fill=c] (6.25871,2.83145) rectangle (6.29851,2.9373);
\draw [color=c, fill=c] (6.29851,2.83145) rectangle (6.33831,2.9373);
\draw [color=c, fill=c] (6.33831,2.83145) rectangle (6.37811,2.9373);
\draw [color=c, fill=c] (6.37811,2.83145) rectangle (6.41791,2.9373);
\draw [color=c, fill=c] (6.41791,2.83145) rectangle (6.45771,2.9373);
\draw [color=c, fill=c] (6.45771,2.83145) rectangle (6.49751,2.9373);
\draw [color=c, fill=c] (6.49751,2.83145) rectangle (6.53731,2.9373);
\draw [color=c, fill=c] (6.53731,2.83145) rectangle (6.57711,2.9373);
\draw [color=c, fill=c] (6.57711,2.83145) rectangle (6.61692,2.9373);
\draw [color=c, fill=c] (6.61692,2.83145) rectangle (6.65672,2.9373);
\draw [color=c, fill=c] (6.65672,2.83145) rectangle (6.69652,2.9373);
\draw [color=c, fill=c] (6.69652,2.83145) rectangle (6.73632,2.9373);
\draw [color=c, fill=c] (6.73632,2.83145) rectangle (6.77612,2.9373);
\draw [color=c, fill=c] (6.77612,2.83145) rectangle (6.81592,2.9373);
\draw [color=c, fill=c] (6.81592,2.83145) rectangle (6.85572,2.9373);
\draw [color=c, fill=c] (6.85572,2.83145) rectangle (6.89552,2.9373);
\draw [color=c, fill=c] (6.89552,2.83145) rectangle (6.93532,2.9373);
\draw [color=c, fill=c] (6.93532,2.83145) rectangle (6.97512,2.9373);
\draw [color=c, fill=c] (6.97512,2.83145) rectangle (7.01493,2.9373);
\draw [color=c, fill=c] (7.01493,2.83145) rectangle (7.05473,2.9373);
\draw [color=c, fill=c] (7.05473,2.83145) rectangle (7.09453,2.9373);
\draw [color=c, fill=c] (7.09453,2.83145) rectangle (7.13433,2.9373);
\draw [color=c, fill=c] (7.13433,2.83145) rectangle (7.17413,2.9373);
\draw [color=c, fill=c] (7.17413,2.83145) rectangle (7.21393,2.9373);
\draw [color=c, fill=c] (7.21393,2.83145) rectangle (7.25373,2.9373);
\draw [color=c, fill=c] (7.25373,2.83145) rectangle (7.29353,2.9373);
\draw [color=c, fill=c] (7.29353,2.83145) rectangle (7.33333,2.9373);
\draw [color=c, fill=c] (7.33333,2.83145) rectangle (7.37313,2.9373);
\draw [color=c, fill=c] (7.37313,2.83145) rectangle (7.41294,2.9373);
\draw [color=c, fill=c] (7.41294,2.83145) rectangle (7.45274,2.9373);
\draw [color=c, fill=c] (7.45274,2.83145) rectangle (7.49254,2.9373);
\draw [color=c, fill=c] (7.49254,2.83145) rectangle (7.53234,2.9373);
\draw [color=c, fill=c] (7.53234,2.83145) rectangle (7.57214,2.9373);
\draw [color=c, fill=c] (7.57214,2.83145) rectangle (7.61194,2.9373);
\draw [color=c, fill=c] (7.61194,2.83145) rectangle (7.65174,2.9373);
\draw [color=c, fill=c] (7.65174,2.83145) rectangle (7.69154,2.9373);
\draw [color=c, fill=c] (7.69154,2.83145) rectangle (7.73134,2.9373);
\draw [color=c, fill=c] (7.73134,2.83145) rectangle (7.77114,2.9373);
\draw [color=c, fill=c] (7.77114,2.83145) rectangle (7.81095,2.9373);
\draw [color=c, fill=c] (7.81095,2.83145) rectangle (7.85075,2.9373);
\draw [color=c, fill=c] (7.85075,2.83145) rectangle (7.89055,2.9373);
\definecolor{c}{rgb}{1,0.186667,0};
\draw [color=c, fill=c] (7.89055,2.83145) rectangle (7.93035,2.9373);
\draw [color=c, fill=c] (7.93035,2.83145) rectangle (7.97015,2.9373);
\draw [color=c, fill=c] (7.97015,2.83145) rectangle (8.00995,2.9373);
\draw [color=c, fill=c] (8.00995,2.83145) rectangle (8.04975,2.9373);
\draw [color=c, fill=c] (8.04975,2.83145) rectangle (8.08955,2.9373);
\draw [color=c, fill=c] (8.08955,2.83145) rectangle (8.12935,2.9373);
\draw [color=c, fill=c] (8.12935,2.83145) rectangle (8.16915,2.9373);
\draw [color=c, fill=c] (8.16915,2.83145) rectangle (8.20895,2.9373);
\draw [color=c, fill=c] (8.20895,2.83145) rectangle (8.24876,2.9373);
\draw [color=c, fill=c] (8.24876,2.83145) rectangle (8.28856,2.9373);
\draw [color=c, fill=c] (8.28856,2.83145) rectangle (8.32836,2.9373);
\draw [color=c, fill=c] (8.32836,2.83145) rectangle (8.36816,2.9373);
\draw [color=c, fill=c] (8.36816,2.83145) rectangle (8.40796,2.9373);
\draw [color=c, fill=c] (8.40796,2.83145) rectangle (8.44776,2.9373);
\draw [color=c, fill=c] (8.44776,2.83145) rectangle (8.48756,2.9373);
\draw [color=c, fill=c] (8.48756,2.83145) rectangle (8.52736,2.9373);
\draw [color=c, fill=c] (8.52736,2.83145) rectangle (8.56716,2.9373);
\draw [color=c, fill=c] (8.56716,2.83145) rectangle (8.60697,2.9373);
\draw [color=c, fill=c] (8.60697,2.83145) rectangle (8.64677,2.9373);
\draw [color=c, fill=c] (8.64677,2.83145) rectangle (8.68657,2.9373);
\draw [color=c, fill=c] (8.68657,2.83145) rectangle (8.72637,2.9373);
\draw [color=c, fill=c] (8.72637,2.83145) rectangle (8.76617,2.9373);
\definecolor{c}{rgb}{1,0.466667,0};
\draw [color=c, fill=c] (8.76617,2.83145) rectangle (8.80597,2.9373);
\draw [color=c, fill=c] (8.80597,2.83145) rectangle (8.84577,2.9373);
\draw [color=c, fill=c] (8.84577,2.83145) rectangle (8.88557,2.9373);
\draw [color=c, fill=c] (8.88557,2.83145) rectangle (8.92537,2.9373);
\draw [color=c, fill=c] (8.92537,2.83145) rectangle (8.96517,2.9373);
\draw [color=c, fill=c] (8.96517,2.83145) rectangle (9.00498,2.9373);
\draw [color=c, fill=c] (9.00498,2.83145) rectangle (9.04478,2.9373);
\draw [color=c, fill=c] (9.04478,2.83145) rectangle (9.08458,2.9373);
\draw [color=c, fill=c] (9.08458,2.83145) rectangle (9.12438,2.9373);
\draw [color=c, fill=c] (9.12438,2.83145) rectangle (9.16418,2.9373);
\draw [color=c, fill=c] (9.16418,2.83145) rectangle (9.20398,2.9373);
\definecolor{c}{rgb}{1,0.653333,0};
\draw [color=c, fill=c] (9.20398,2.83145) rectangle (9.24378,2.9373);
\draw [color=c, fill=c] (9.24378,2.83145) rectangle (9.28358,2.9373);
\draw [color=c, fill=c] (9.28358,2.83145) rectangle (9.32338,2.9373);
\draw [color=c, fill=c] (9.32338,2.83145) rectangle (9.36318,2.9373);
\draw [color=c, fill=c] (9.36318,2.83145) rectangle (9.40298,2.9373);
\draw [color=c, fill=c] (9.40298,2.83145) rectangle (9.44279,2.9373);
\draw [color=c, fill=c] (9.44279,2.83145) rectangle (9.48259,2.9373);
\definecolor{c}{rgb}{1,0.933333,0};
\draw [color=c, fill=c] (9.48259,2.83145) rectangle (9.52239,2.9373);
\draw [color=c, fill=c] (9.52239,2.83145) rectangle (9.56219,2.9373);
\draw [color=c, fill=c] (9.56219,2.83145) rectangle (9.60199,2.9373);
\draw [color=c, fill=c] (9.60199,2.83145) rectangle (9.64179,2.9373);
\draw [color=c, fill=c] (9.64179,2.83145) rectangle (9.68159,2.9373);
\definecolor{c}{rgb}{0.88,1,0};
\draw [color=c, fill=c] (9.68159,2.83145) rectangle (9.72139,2.9373);
\draw [color=c, fill=c] (9.72139,2.83145) rectangle (9.76119,2.9373);
\draw [color=c, fill=c] (9.76119,2.83145) rectangle (9.80099,2.9373);
\definecolor{c}{rgb}{0.6,1,0};
\draw [color=c, fill=c] (9.80099,2.83145) rectangle (9.8408,2.9373);
\draw [color=c, fill=c] (9.8408,2.83145) rectangle (9.8806,2.9373);
\draw [color=c, fill=c] (9.8806,2.83145) rectangle (9.9204,2.9373);
\definecolor{c}{rgb}{0.413333,1,0};
\draw [color=c, fill=c] (9.9204,2.83145) rectangle (9.9602,2.9373);
\draw [color=c, fill=c] (9.9602,2.83145) rectangle (10,2.9373);
\definecolor{c}{rgb}{0.133333,1,0};
\draw [color=c, fill=c] (10,2.83145) rectangle (10.0398,2.9373);
\draw [color=c, fill=c] (10.0398,2.83145) rectangle (10.0796,2.9373);
\draw [color=c, fill=c] (10.0796,2.83145) rectangle (10.1194,2.9373);
\definecolor{c}{rgb}{0,1,0.0533333};
\draw [color=c, fill=c] (10.1194,2.83145) rectangle (10.1592,2.9373);
\draw [color=c, fill=c] (10.1592,2.83145) rectangle (10.199,2.9373);
\draw [color=c, fill=c] (10.199,2.83145) rectangle (10.2388,2.9373);
\draw [color=c, fill=c] (10.2388,2.83145) rectangle (10.2786,2.9373);
\definecolor{c}{rgb}{0,1,0.333333};
\draw [color=c, fill=c] (10.2786,2.83145) rectangle (10.3184,2.9373);
\draw [color=c, fill=c] (10.3184,2.83145) rectangle (10.3582,2.9373);
\draw [color=c, fill=c] (10.3582,2.83145) rectangle (10.398,2.9373);
\draw [color=c, fill=c] (10.398,2.83145) rectangle (10.4378,2.9373);
\draw [color=c, fill=c] (10.4378,2.83145) rectangle (10.4776,2.9373);
\definecolor{c}{rgb}{0,1,0.52};
\draw [color=c, fill=c] (10.4776,2.83145) rectangle (10.5174,2.9373);
\draw [color=c, fill=c] (10.5174,2.83145) rectangle (10.5572,2.9373);
\draw [color=c, fill=c] (10.5572,2.83145) rectangle (10.597,2.9373);
\draw [color=c, fill=c] (10.597,2.83145) rectangle (10.6368,2.9373);
\draw [color=c, fill=c] (10.6368,2.83145) rectangle (10.6766,2.9373);
\draw [color=c, fill=c] (10.6766,2.83145) rectangle (10.7164,2.9373);
\draw [color=c, fill=c] (10.7164,2.83145) rectangle (10.7562,2.9373);
\definecolor{c}{rgb}{0,1,0.8};
\draw [color=c, fill=c] (10.7562,2.83145) rectangle (10.796,2.9373);
\draw [color=c, fill=c] (10.796,2.83145) rectangle (10.8358,2.9373);
\draw [color=c, fill=c] (10.8358,2.83145) rectangle (10.8756,2.9373);
\draw [color=c, fill=c] (10.8756,2.83145) rectangle (10.9154,2.9373);
\draw [color=c, fill=c] (10.9154,2.83145) rectangle (10.9552,2.9373);
\draw [color=c, fill=c] (10.9552,2.83145) rectangle (10.995,2.9373);
\draw [color=c, fill=c] (10.995,2.83145) rectangle (11.0348,2.9373);
\draw [color=c, fill=c] (11.0348,2.83145) rectangle (11.0746,2.9373);
\draw [color=c, fill=c] (11.0746,2.83145) rectangle (11.1144,2.9373);
\draw [color=c, fill=c] (11.1144,2.83145) rectangle (11.1542,2.9373);
\draw [color=c, fill=c] (11.1542,2.83145) rectangle (11.194,2.9373);
\draw [color=c, fill=c] (11.194,2.83145) rectangle (11.2338,2.9373);
\definecolor{c}{rgb}{0,1,0.986667};
\draw [color=c, fill=c] (11.2338,2.83145) rectangle (11.2736,2.9373);
\draw [color=c, fill=c] (11.2736,2.83145) rectangle (11.3134,2.9373);
\draw [color=c, fill=c] (11.3134,2.83145) rectangle (11.3532,2.9373);
\draw [color=c, fill=c] (11.3532,2.83145) rectangle (11.393,2.9373);
\draw [color=c, fill=c] (11.393,2.83145) rectangle (11.4328,2.9373);
\draw [color=c, fill=c] (11.4328,2.83145) rectangle (11.4726,2.9373);
\draw [color=c, fill=c] (11.4726,2.83145) rectangle (11.5124,2.9373);
\draw [color=c, fill=c] (11.5124,2.83145) rectangle (11.5522,2.9373);
\draw [color=c, fill=c] (11.5522,2.83145) rectangle (11.592,2.9373);
\draw [color=c, fill=c] (11.592,2.83145) rectangle (11.6318,2.9373);
\draw [color=c, fill=c] (11.6318,2.83145) rectangle (11.6716,2.9373);
\draw [color=c, fill=c] (11.6716,2.83145) rectangle (11.7114,2.9373);
\draw [color=c, fill=c] (11.7114,2.83145) rectangle (11.7512,2.9373);
\draw [color=c, fill=c] (11.7512,2.83145) rectangle (11.791,2.9373);
\draw [color=c, fill=c] (11.791,2.83145) rectangle (11.8308,2.9373);
\draw [color=c, fill=c] (11.8308,2.83145) rectangle (11.8706,2.9373);
\draw [color=c, fill=c] (11.8706,2.83145) rectangle (11.9104,2.9373);
\draw [color=c, fill=c] (11.9104,2.83145) rectangle (11.9502,2.9373);
\draw [color=c, fill=c] (11.9502,2.83145) rectangle (11.99,2.9373);
\draw [color=c, fill=c] (11.99,2.83145) rectangle (12.0299,2.9373);
\draw [color=c, fill=c] (12.0299,2.83145) rectangle (12.0697,2.9373);
\draw [color=c, fill=c] (12.0697,2.83145) rectangle (12.1095,2.9373);
\draw [color=c, fill=c] (12.1095,2.83145) rectangle (12.1493,2.9373);
\draw [color=c, fill=c] (12.1493,2.83145) rectangle (12.1891,2.9373);
\definecolor{c}{rgb}{0,0.733333,1};
\draw [color=c, fill=c] (12.1891,2.83145) rectangle (12.2289,2.9373);
\draw [color=c, fill=c] (12.2289,2.83145) rectangle (12.2687,2.9373);
\draw [color=c, fill=c] (12.2687,2.83145) rectangle (12.3085,2.9373);
\draw [color=c, fill=c] (12.3085,2.83145) rectangle (12.3483,2.9373);
\draw [color=c, fill=c] (12.3483,2.83145) rectangle (12.3881,2.9373);
\draw [color=c, fill=c] (12.3881,2.83145) rectangle (12.4279,2.9373);
\draw [color=c, fill=c] (12.4279,2.83145) rectangle (12.4677,2.9373);
\draw [color=c, fill=c] (12.4677,2.83145) rectangle (12.5075,2.9373);
\draw [color=c, fill=c] (12.5075,2.83145) rectangle (12.5473,2.9373);
\draw [color=c, fill=c] (12.5473,2.83145) rectangle (12.5871,2.9373);
\draw [color=c, fill=c] (12.5871,2.83145) rectangle (12.6269,2.9373);
\draw [color=c, fill=c] (12.6269,2.83145) rectangle (12.6667,2.9373);
\draw [color=c, fill=c] (12.6667,2.83145) rectangle (12.7065,2.9373);
\draw [color=c, fill=c] (12.7065,2.83145) rectangle (12.7463,2.9373);
\draw [color=c, fill=c] (12.7463,2.83145) rectangle (12.7861,2.9373);
\draw [color=c, fill=c] (12.7861,2.83145) rectangle (12.8259,2.9373);
\draw [color=c, fill=c] (12.8259,2.83145) rectangle (12.8657,2.9373);
\draw [color=c, fill=c] (12.8657,2.83145) rectangle (12.9055,2.9373);
\draw [color=c, fill=c] (12.9055,2.83145) rectangle (12.9453,2.9373);
\draw [color=c, fill=c] (12.9453,2.83145) rectangle (12.9851,2.9373);
\draw [color=c, fill=c] (12.9851,2.83145) rectangle (13.0249,2.9373);
\draw [color=c, fill=c] (13.0249,2.83145) rectangle (13.0647,2.9373);
\draw [color=c, fill=c] (13.0647,2.83145) rectangle (13.1045,2.9373);
\draw [color=c, fill=c] (13.1045,2.83145) rectangle (13.1443,2.9373);
\draw [color=c, fill=c] (13.1443,2.83145) rectangle (13.1841,2.9373);
\draw [color=c, fill=c] (13.1841,2.83145) rectangle (13.2239,2.9373);
\draw [color=c, fill=c] (13.2239,2.83145) rectangle (13.2637,2.9373);
\draw [color=c, fill=c] (13.2637,2.83145) rectangle (13.3035,2.9373);
\draw [color=c, fill=c] (13.3035,2.83145) rectangle (13.3433,2.9373);
\draw [color=c, fill=c] (13.3433,2.83145) rectangle (13.3831,2.9373);
\draw [color=c, fill=c] (13.3831,2.83145) rectangle (13.4229,2.9373);
\draw [color=c, fill=c] (13.4229,2.83145) rectangle (13.4627,2.9373);
\draw [color=c, fill=c] (13.4627,2.83145) rectangle (13.5025,2.9373);
\draw [color=c, fill=c] (13.5025,2.83145) rectangle (13.5423,2.9373);
\draw [color=c, fill=c] (13.5423,2.83145) rectangle (13.5821,2.9373);
\draw [color=c, fill=c] (13.5821,2.83145) rectangle (13.6219,2.9373);
\draw [color=c, fill=c] (13.6219,2.83145) rectangle (13.6617,2.9373);
\draw [color=c, fill=c] (13.6617,2.83145) rectangle (13.7015,2.9373);
\draw [color=c, fill=c] (13.7015,2.83145) rectangle (13.7413,2.9373);
\draw [color=c, fill=c] (13.7413,2.83145) rectangle (13.7811,2.9373);
\draw [color=c, fill=c] (13.7811,2.83145) rectangle (13.8209,2.9373);
\draw [color=c, fill=c] (13.8209,2.83145) rectangle (13.8607,2.9373);
\draw [color=c, fill=c] (13.8607,2.83145) rectangle (13.9005,2.9373);
\draw [color=c, fill=c] (13.9005,2.83145) rectangle (13.9403,2.9373);
\draw [color=c, fill=c] (13.9403,2.83145) rectangle (13.9801,2.9373);
\draw [color=c, fill=c] (13.9801,2.83145) rectangle (14.0199,2.9373);
\draw [color=c, fill=c] (14.0199,2.83145) rectangle (14.0597,2.9373);
\draw [color=c, fill=c] (14.0597,2.83145) rectangle (14.0995,2.9373);
\draw [color=c, fill=c] (14.0995,2.83145) rectangle (14.1393,2.9373);
\draw [color=c, fill=c] (14.1393,2.83145) rectangle (14.1791,2.9373);
\draw [color=c, fill=c] (14.1791,2.83145) rectangle (14.2189,2.9373);
\draw [color=c, fill=c] (14.2189,2.83145) rectangle (14.2587,2.9373);
\draw [color=c, fill=c] (14.2587,2.83145) rectangle (14.2985,2.9373);
\draw [color=c, fill=c] (14.2985,2.83145) rectangle (14.3383,2.9373);
\draw [color=c, fill=c] (14.3383,2.83145) rectangle (14.3781,2.9373);
\draw [color=c, fill=c] (14.3781,2.83145) rectangle (14.4179,2.9373);
\draw [color=c, fill=c] (14.4179,2.83145) rectangle (14.4577,2.9373);
\draw [color=c, fill=c] (14.4577,2.83145) rectangle (14.4975,2.9373);
\draw [color=c, fill=c] (14.4975,2.83145) rectangle (14.5373,2.9373);
\draw [color=c, fill=c] (14.5373,2.83145) rectangle (14.5771,2.9373);
\draw [color=c, fill=c] (14.5771,2.83145) rectangle (14.6169,2.9373);
\draw [color=c, fill=c] (14.6169,2.83145) rectangle (14.6567,2.9373);
\draw [color=c, fill=c] (14.6567,2.83145) rectangle (14.6965,2.9373);
\draw [color=c, fill=c] (14.6965,2.83145) rectangle (14.7363,2.9373);
\draw [color=c, fill=c] (14.7363,2.83145) rectangle (14.7761,2.9373);
\draw [color=c, fill=c] (14.7761,2.83145) rectangle (14.8159,2.9373);
\draw [color=c, fill=c] (14.8159,2.83145) rectangle (14.8557,2.9373);
\draw [color=c, fill=c] (14.8557,2.83145) rectangle (14.8955,2.9373);
\draw [color=c, fill=c] (14.8955,2.83145) rectangle (14.9353,2.9373);
\draw [color=c, fill=c] (14.9353,2.83145) rectangle (14.9751,2.9373);
\draw [color=c, fill=c] (14.9751,2.83145) rectangle (15.0149,2.9373);
\draw [color=c, fill=c] (15.0149,2.83145) rectangle (15.0547,2.9373);
\draw [color=c, fill=c] (15.0547,2.83145) rectangle (15.0945,2.9373);
\draw [color=c, fill=c] (15.0945,2.83145) rectangle (15.1343,2.9373);
\draw [color=c, fill=c] (15.1343,2.83145) rectangle (15.1741,2.9373);
\draw [color=c, fill=c] (15.1741,2.83145) rectangle (15.2139,2.9373);
\draw [color=c, fill=c] (15.2139,2.83145) rectangle (15.2537,2.9373);
\draw [color=c, fill=c] (15.2537,2.83145) rectangle (15.2935,2.9373);
\draw [color=c, fill=c] (15.2935,2.83145) rectangle (15.3333,2.9373);
\draw [color=c, fill=c] (15.3333,2.83145) rectangle (15.3731,2.9373);
\draw [color=c, fill=c] (15.3731,2.83145) rectangle (15.4129,2.9373);
\draw [color=c, fill=c] (15.4129,2.83145) rectangle (15.4527,2.9373);
\draw [color=c, fill=c] (15.4527,2.83145) rectangle (15.4925,2.9373);
\draw [color=c, fill=c] (15.4925,2.83145) rectangle (15.5323,2.9373);
\draw [color=c, fill=c] (15.5323,2.83145) rectangle (15.5721,2.9373);
\draw [color=c, fill=c] (15.5721,2.83145) rectangle (15.6119,2.9373);
\draw [color=c, fill=c] (15.6119,2.83145) rectangle (15.6517,2.9373);
\draw [color=c, fill=c] (15.6517,2.83145) rectangle (15.6915,2.9373);
\draw [color=c, fill=c] (15.6915,2.83145) rectangle (15.7313,2.9373);
\draw [color=c, fill=c] (15.7313,2.83145) rectangle (15.7711,2.9373);
\draw [color=c, fill=c] (15.7711,2.83145) rectangle (15.8109,2.9373);
\draw [color=c, fill=c] (15.8109,2.83145) rectangle (15.8507,2.9373);
\draw [color=c, fill=c] (15.8507,2.83145) rectangle (15.8905,2.9373);
\draw [color=c, fill=c] (15.8905,2.83145) rectangle (15.9303,2.9373);
\draw [color=c, fill=c] (15.9303,2.83145) rectangle (15.9701,2.9373);
\draw [color=c, fill=c] (15.9701,2.83145) rectangle (16.01,2.9373);
\draw [color=c, fill=c] (16.01,2.83145) rectangle (16.0498,2.9373);
\draw [color=c, fill=c] (16.0498,2.83145) rectangle (16.0896,2.9373);
\draw [color=c, fill=c] (16.0896,2.83145) rectangle (16.1294,2.9373);
\draw [color=c, fill=c] (16.1294,2.83145) rectangle (16.1692,2.9373);
\draw [color=c, fill=c] (16.1692,2.83145) rectangle (16.209,2.9373);
\draw [color=c, fill=c] (16.209,2.83145) rectangle (16.2488,2.9373);
\draw [color=c, fill=c] (16.2488,2.83145) rectangle (16.2886,2.9373);
\draw [color=c, fill=c] (16.2886,2.83145) rectangle (16.3284,2.9373);
\draw [color=c, fill=c] (16.3284,2.83145) rectangle (16.3682,2.9373);
\draw [color=c, fill=c] (16.3682,2.83145) rectangle (16.408,2.9373);
\draw [color=c, fill=c] (16.408,2.83145) rectangle (16.4478,2.9373);
\draw [color=c, fill=c] (16.4478,2.83145) rectangle (16.4876,2.9373);
\draw [color=c, fill=c] (16.4876,2.83145) rectangle (16.5274,2.9373);
\draw [color=c, fill=c] (16.5274,2.83145) rectangle (16.5672,2.9373);
\draw [color=c, fill=c] (16.5672,2.83145) rectangle (16.607,2.9373);
\draw [color=c, fill=c] (16.607,2.83145) rectangle (16.6468,2.9373);
\draw [color=c, fill=c] (16.6468,2.83145) rectangle (16.6866,2.9373);
\draw [color=c, fill=c] (16.6866,2.83145) rectangle (16.7264,2.9373);
\draw [color=c, fill=c] (16.7264,2.83145) rectangle (16.7662,2.9373);
\draw [color=c, fill=c] (16.7662,2.83145) rectangle (16.806,2.9373);
\draw [color=c, fill=c] (16.806,2.83145) rectangle (16.8458,2.9373);
\draw [color=c, fill=c] (16.8458,2.83145) rectangle (16.8856,2.9373);
\draw [color=c, fill=c] (16.8856,2.83145) rectangle (16.9254,2.9373);
\draw [color=c, fill=c] (16.9254,2.83145) rectangle (16.9652,2.9373);
\draw [color=c, fill=c] (16.9652,2.83145) rectangle (17.005,2.9373);
\draw [color=c, fill=c] (17.005,2.83145) rectangle (17.0448,2.9373);
\draw [color=c, fill=c] (17.0448,2.83145) rectangle (17.0846,2.9373);
\draw [color=c, fill=c] (17.0846,2.83145) rectangle (17.1244,2.9373);
\draw [color=c, fill=c] (17.1244,2.83145) rectangle (17.1642,2.9373);
\draw [color=c, fill=c] (17.1642,2.83145) rectangle (17.204,2.9373);
\draw [color=c, fill=c] (17.204,2.83145) rectangle (17.2438,2.9373);
\draw [color=c, fill=c] (17.2438,2.83145) rectangle (17.2836,2.9373);
\draw [color=c, fill=c] (17.2836,2.83145) rectangle (17.3234,2.9373);
\draw [color=c, fill=c] (17.3234,2.83145) rectangle (17.3632,2.9373);
\draw [color=c, fill=c] (17.3632,2.83145) rectangle (17.403,2.9373);
\draw [color=c, fill=c] (17.403,2.83145) rectangle (17.4428,2.9373);
\draw [color=c, fill=c] (17.4428,2.83145) rectangle (17.4826,2.9373);
\draw [color=c, fill=c] (17.4826,2.83145) rectangle (17.5224,2.9373);
\draw [color=c, fill=c] (17.5224,2.83145) rectangle (17.5622,2.9373);
\draw [color=c, fill=c] (17.5622,2.83145) rectangle (17.602,2.9373);
\draw [color=c, fill=c] (17.602,2.83145) rectangle (17.6418,2.9373);
\draw [color=c, fill=c] (17.6418,2.83145) rectangle (17.6816,2.9373);
\draw [color=c, fill=c] (17.6816,2.83145) rectangle (17.7214,2.9373);
\draw [color=c, fill=c] (17.7214,2.83145) rectangle (17.7612,2.9373);
\draw [color=c, fill=c] (17.7612,2.83145) rectangle (17.801,2.9373);
\draw [color=c, fill=c] (17.801,2.83145) rectangle (17.8408,2.9373);
\draw [color=c, fill=c] (17.8408,2.83145) rectangle (17.8806,2.9373);
\draw [color=c, fill=c] (17.8806,2.83145) rectangle (17.9204,2.9373);
\draw [color=c, fill=c] (17.9204,2.83145) rectangle (17.9602,2.9373);
\draw [color=c, fill=c] (17.9602,2.83145) rectangle (18,2.9373);
\definecolor{c}{rgb}{1,0,0};
\draw [color=c, fill=c] (2,2.9373) rectangle (2.0398,3.04315);
\draw [color=c, fill=c] (2.0398,2.9373) rectangle (2.0796,3.04315);
\draw [color=c, fill=c] (2.0796,2.9373) rectangle (2.1194,3.04315);
\draw [color=c, fill=c] (2.1194,2.9373) rectangle (2.1592,3.04315);
\draw [color=c, fill=c] (2.1592,2.9373) rectangle (2.19901,3.04315);
\draw [color=c, fill=c] (2.19901,2.9373) rectangle (2.23881,3.04315);
\draw [color=c, fill=c] (2.23881,2.9373) rectangle (2.27861,3.04315);
\draw [color=c, fill=c] (2.27861,2.9373) rectangle (2.31841,3.04315);
\draw [color=c, fill=c] (2.31841,2.9373) rectangle (2.35821,3.04315);
\draw [color=c, fill=c] (2.35821,2.9373) rectangle (2.39801,3.04315);
\draw [color=c, fill=c] (2.39801,2.9373) rectangle (2.43781,3.04315);
\draw [color=c, fill=c] (2.43781,2.9373) rectangle (2.47761,3.04315);
\draw [color=c, fill=c] (2.47761,2.9373) rectangle (2.51741,3.04315);
\draw [color=c, fill=c] (2.51741,2.9373) rectangle (2.55721,3.04315);
\draw [color=c, fill=c] (2.55721,2.9373) rectangle (2.59702,3.04315);
\draw [color=c, fill=c] (2.59702,2.9373) rectangle (2.63682,3.04315);
\draw [color=c, fill=c] (2.63682,2.9373) rectangle (2.67662,3.04315);
\draw [color=c, fill=c] (2.67662,2.9373) rectangle (2.71642,3.04315);
\draw [color=c, fill=c] (2.71642,2.9373) rectangle (2.75622,3.04315);
\draw [color=c, fill=c] (2.75622,2.9373) rectangle (2.79602,3.04315);
\draw [color=c, fill=c] (2.79602,2.9373) rectangle (2.83582,3.04315);
\draw [color=c, fill=c] (2.83582,2.9373) rectangle (2.87562,3.04315);
\draw [color=c, fill=c] (2.87562,2.9373) rectangle (2.91542,3.04315);
\draw [color=c, fill=c] (2.91542,2.9373) rectangle (2.95522,3.04315);
\draw [color=c, fill=c] (2.95522,2.9373) rectangle (2.99502,3.04315);
\draw [color=c, fill=c] (2.99502,2.9373) rectangle (3.03483,3.04315);
\draw [color=c, fill=c] (3.03483,2.9373) rectangle (3.07463,3.04315);
\draw [color=c, fill=c] (3.07463,2.9373) rectangle (3.11443,3.04315);
\draw [color=c, fill=c] (3.11443,2.9373) rectangle (3.15423,3.04315);
\draw [color=c, fill=c] (3.15423,2.9373) rectangle (3.19403,3.04315);
\draw [color=c, fill=c] (3.19403,2.9373) rectangle (3.23383,3.04315);
\draw [color=c, fill=c] (3.23383,2.9373) rectangle (3.27363,3.04315);
\draw [color=c, fill=c] (3.27363,2.9373) rectangle (3.31343,3.04315);
\draw [color=c, fill=c] (3.31343,2.9373) rectangle (3.35323,3.04315);
\draw [color=c, fill=c] (3.35323,2.9373) rectangle (3.39303,3.04315);
\draw [color=c, fill=c] (3.39303,2.9373) rectangle (3.43284,3.04315);
\draw [color=c, fill=c] (3.43284,2.9373) rectangle (3.47264,3.04315);
\draw [color=c, fill=c] (3.47264,2.9373) rectangle (3.51244,3.04315);
\draw [color=c, fill=c] (3.51244,2.9373) rectangle (3.55224,3.04315);
\draw [color=c, fill=c] (3.55224,2.9373) rectangle (3.59204,3.04315);
\draw [color=c, fill=c] (3.59204,2.9373) rectangle (3.63184,3.04315);
\draw [color=c, fill=c] (3.63184,2.9373) rectangle (3.67164,3.04315);
\draw [color=c, fill=c] (3.67164,2.9373) rectangle (3.71144,3.04315);
\draw [color=c, fill=c] (3.71144,2.9373) rectangle (3.75124,3.04315);
\draw [color=c, fill=c] (3.75124,2.9373) rectangle (3.79104,3.04315);
\draw [color=c, fill=c] (3.79104,2.9373) rectangle (3.83085,3.04315);
\draw [color=c, fill=c] (3.83085,2.9373) rectangle (3.87065,3.04315);
\draw [color=c, fill=c] (3.87065,2.9373) rectangle (3.91045,3.04315);
\draw [color=c, fill=c] (3.91045,2.9373) rectangle (3.95025,3.04315);
\draw [color=c, fill=c] (3.95025,2.9373) rectangle (3.99005,3.04315);
\draw [color=c, fill=c] (3.99005,2.9373) rectangle (4.02985,3.04315);
\draw [color=c, fill=c] (4.02985,2.9373) rectangle (4.06965,3.04315);
\draw [color=c, fill=c] (4.06965,2.9373) rectangle (4.10945,3.04315);
\draw [color=c, fill=c] (4.10945,2.9373) rectangle (4.14925,3.04315);
\draw [color=c, fill=c] (4.14925,2.9373) rectangle (4.18905,3.04315);
\draw [color=c, fill=c] (4.18905,2.9373) rectangle (4.22886,3.04315);
\draw [color=c, fill=c] (4.22886,2.9373) rectangle (4.26866,3.04315);
\draw [color=c, fill=c] (4.26866,2.9373) rectangle (4.30846,3.04315);
\draw [color=c, fill=c] (4.30846,2.9373) rectangle (4.34826,3.04315);
\draw [color=c, fill=c] (4.34826,2.9373) rectangle (4.38806,3.04315);
\draw [color=c, fill=c] (4.38806,2.9373) rectangle (4.42786,3.04315);
\draw [color=c, fill=c] (4.42786,2.9373) rectangle (4.46766,3.04315);
\draw [color=c, fill=c] (4.46766,2.9373) rectangle (4.50746,3.04315);
\draw [color=c, fill=c] (4.50746,2.9373) rectangle (4.54726,3.04315);
\draw [color=c, fill=c] (4.54726,2.9373) rectangle (4.58706,3.04315);
\draw [color=c, fill=c] (4.58706,2.9373) rectangle (4.62687,3.04315);
\draw [color=c, fill=c] (4.62687,2.9373) rectangle (4.66667,3.04315);
\draw [color=c, fill=c] (4.66667,2.9373) rectangle (4.70647,3.04315);
\draw [color=c, fill=c] (4.70647,2.9373) rectangle (4.74627,3.04315);
\draw [color=c, fill=c] (4.74627,2.9373) rectangle (4.78607,3.04315);
\draw [color=c, fill=c] (4.78607,2.9373) rectangle (4.82587,3.04315);
\draw [color=c, fill=c] (4.82587,2.9373) rectangle (4.86567,3.04315);
\draw [color=c, fill=c] (4.86567,2.9373) rectangle (4.90547,3.04315);
\draw [color=c, fill=c] (4.90547,2.9373) rectangle (4.94527,3.04315);
\draw [color=c, fill=c] (4.94527,2.9373) rectangle (4.98507,3.04315);
\draw [color=c, fill=c] (4.98507,2.9373) rectangle (5.02488,3.04315);
\draw [color=c, fill=c] (5.02488,2.9373) rectangle (5.06468,3.04315);
\draw [color=c, fill=c] (5.06468,2.9373) rectangle (5.10448,3.04315);
\draw [color=c, fill=c] (5.10448,2.9373) rectangle (5.14428,3.04315);
\draw [color=c, fill=c] (5.14428,2.9373) rectangle (5.18408,3.04315);
\draw [color=c, fill=c] (5.18408,2.9373) rectangle (5.22388,3.04315);
\draw [color=c, fill=c] (5.22388,2.9373) rectangle (5.26368,3.04315);
\draw [color=c, fill=c] (5.26368,2.9373) rectangle (5.30348,3.04315);
\draw [color=c, fill=c] (5.30348,2.9373) rectangle (5.34328,3.04315);
\draw [color=c, fill=c] (5.34328,2.9373) rectangle (5.38308,3.04315);
\draw [color=c, fill=c] (5.38308,2.9373) rectangle (5.42289,3.04315);
\draw [color=c, fill=c] (5.42289,2.9373) rectangle (5.46269,3.04315);
\draw [color=c, fill=c] (5.46269,2.9373) rectangle (5.50249,3.04315);
\draw [color=c, fill=c] (5.50249,2.9373) rectangle (5.54229,3.04315);
\draw [color=c, fill=c] (5.54229,2.9373) rectangle (5.58209,3.04315);
\draw [color=c, fill=c] (5.58209,2.9373) rectangle (5.62189,3.04315);
\draw [color=c, fill=c] (5.62189,2.9373) rectangle (5.66169,3.04315);
\draw [color=c, fill=c] (5.66169,2.9373) rectangle (5.70149,3.04315);
\draw [color=c, fill=c] (5.70149,2.9373) rectangle (5.74129,3.04315);
\draw [color=c, fill=c] (5.74129,2.9373) rectangle (5.78109,3.04315);
\draw [color=c, fill=c] (5.78109,2.9373) rectangle (5.8209,3.04315);
\draw [color=c, fill=c] (5.8209,2.9373) rectangle (5.8607,3.04315);
\draw [color=c, fill=c] (5.8607,2.9373) rectangle (5.9005,3.04315);
\draw [color=c, fill=c] (5.9005,2.9373) rectangle (5.9403,3.04315);
\draw [color=c, fill=c] (5.9403,2.9373) rectangle (5.9801,3.04315);
\draw [color=c, fill=c] (5.9801,2.9373) rectangle (6.0199,3.04315);
\draw [color=c, fill=c] (6.0199,2.9373) rectangle (6.0597,3.04315);
\draw [color=c, fill=c] (6.0597,2.9373) rectangle (6.0995,3.04315);
\draw [color=c, fill=c] (6.0995,2.9373) rectangle (6.1393,3.04315);
\draw [color=c, fill=c] (6.1393,2.9373) rectangle (6.1791,3.04315);
\draw [color=c, fill=c] (6.1791,2.9373) rectangle (6.21891,3.04315);
\draw [color=c, fill=c] (6.21891,2.9373) rectangle (6.25871,3.04315);
\draw [color=c, fill=c] (6.25871,2.9373) rectangle (6.29851,3.04315);
\draw [color=c, fill=c] (6.29851,2.9373) rectangle (6.33831,3.04315);
\draw [color=c, fill=c] (6.33831,2.9373) rectangle (6.37811,3.04315);
\draw [color=c, fill=c] (6.37811,2.9373) rectangle (6.41791,3.04315);
\draw [color=c, fill=c] (6.41791,2.9373) rectangle (6.45771,3.04315);
\draw [color=c, fill=c] (6.45771,2.9373) rectangle (6.49751,3.04315);
\draw [color=c, fill=c] (6.49751,2.9373) rectangle (6.53731,3.04315);
\draw [color=c, fill=c] (6.53731,2.9373) rectangle (6.57711,3.04315);
\draw [color=c, fill=c] (6.57711,2.9373) rectangle (6.61692,3.04315);
\draw [color=c, fill=c] (6.61692,2.9373) rectangle (6.65672,3.04315);
\draw [color=c, fill=c] (6.65672,2.9373) rectangle (6.69652,3.04315);
\draw [color=c, fill=c] (6.69652,2.9373) rectangle (6.73632,3.04315);
\draw [color=c, fill=c] (6.73632,2.9373) rectangle (6.77612,3.04315);
\draw [color=c, fill=c] (6.77612,2.9373) rectangle (6.81592,3.04315);
\draw [color=c, fill=c] (6.81592,2.9373) rectangle (6.85572,3.04315);
\draw [color=c, fill=c] (6.85572,2.9373) rectangle (6.89552,3.04315);
\draw [color=c, fill=c] (6.89552,2.9373) rectangle (6.93532,3.04315);
\draw [color=c, fill=c] (6.93532,2.9373) rectangle (6.97512,3.04315);
\draw [color=c, fill=c] (6.97512,2.9373) rectangle (7.01493,3.04315);
\draw [color=c, fill=c] (7.01493,2.9373) rectangle (7.05473,3.04315);
\draw [color=c, fill=c] (7.05473,2.9373) rectangle (7.09453,3.04315);
\draw [color=c, fill=c] (7.09453,2.9373) rectangle (7.13433,3.04315);
\draw [color=c, fill=c] (7.13433,2.9373) rectangle (7.17413,3.04315);
\draw [color=c, fill=c] (7.17413,2.9373) rectangle (7.21393,3.04315);
\draw [color=c, fill=c] (7.21393,2.9373) rectangle (7.25373,3.04315);
\draw [color=c, fill=c] (7.25373,2.9373) rectangle (7.29353,3.04315);
\draw [color=c, fill=c] (7.29353,2.9373) rectangle (7.33333,3.04315);
\draw [color=c, fill=c] (7.33333,2.9373) rectangle (7.37313,3.04315);
\draw [color=c, fill=c] (7.37313,2.9373) rectangle (7.41294,3.04315);
\draw [color=c, fill=c] (7.41294,2.9373) rectangle (7.45274,3.04315);
\draw [color=c, fill=c] (7.45274,2.9373) rectangle (7.49254,3.04315);
\draw [color=c, fill=c] (7.49254,2.9373) rectangle (7.53234,3.04315);
\draw [color=c, fill=c] (7.53234,2.9373) rectangle (7.57214,3.04315);
\draw [color=c, fill=c] (7.57214,2.9373) rectangle (7.61194,3.04315);
\draw [color=c, fill=c] (7.61194,2.9373) rectangle (7.65174,3.04315);
\draw [color=c, fill=c] (7.65174,2.9373) rectangle (7.69154,3.04315);
\draw [color=c, fill=c] (7.69154,2.9373) rectangle (7.73134,3.04315);
\draw [color=c, fill=c] (7.73134,2.9373) rectangle (7.77114,3.04315);
\draw [color=c, fill=c] (7.77114,2.9373) rectangle (7.81095,3.04315);
\draw [color=c, fill=c] (7.81095,2.9373) rectangle (7.85075,3.04315);
\draw [color=c, fill=c] (7.85075,2.9373) rectangle (7.89055,3.04315);
\draw [color=c, fill=c] (7.89055,2.9373) rectangle (7.93035,3.04315);
\definecolor{c}{rgb}{1,0.186667,0};
\draw [color=c, fill=c] (7.93035,2.9373) rectangle (7.97015,3.04315);
\draw [color=c, fill=c] (7.97015,2.9373) rectangle (8.00995,3.04315);
\draw [color=c, fill=c] (8.00995,2.9373) rectangle (8.04975,3.04315);
\draw [color=c, fill=c] (8.04975,2.9373) rectangle (8.08955,3.04315);
\draw [color=c, fill=c] (8.08955,2.9373) rectangle (8.12935,3.04315);
\draw [color=c, fill=c] (8.12935,2.9373) rectangle (8.16915,3.04315);
\draw [color=c, fill=c] (8.16915,2.9373) rectangle (8.20895,3.04315);
\draw [color=c, fill=c] (8.20895,2.9373) rectangle (8.24876,3.04315);
\draw [color=c, fill=c] (8.24876,2.9373) rectangle (8.28856,3.04315);
\draw [color=c, fill=c] (8.28856,2.9373) rectangle (8.32836,3.04315);
\draw [color=c, fill=c] (8.32836,2.9373) rectangle (8.36816,3.04315);
\draw [color=c, fill=c] (8.36816,2.9373) rectangle (8.40796,3.04315);
\draw [color=c, fill=c] (8.40796,2.9373) rectangle (8.44776,3.04315);
\draw [color=c, fill=c] (8.44776,2.9373) rectangle (8.48756,3.04315);
\draw [color=c, fill=c] (8.48756,2.9373) rectangle (8.52736,3.04315);
\draw [color=c, fill=c] (8.52736,2.9373) rectangle (8.56716,3.04315);
\draw [color=c, fill=c] (8.56716,2.9373) rectangle (8.60697,3.04315);
\draw [color=c, fill=c] (8.60697,2.9373) rectangle (8.64677,3.04315);
\draw [color=c, fill=c] (8.64677,2.9373) rectangle (8.68657,3.04315);
\draw [color=c, fill=c] (8.68657,2.9373) rectangle (8.72637,3.04315);
\draw [color=c, fill=c] (8.72637,2.9373) rectangle (8.76617,3.04315);
\draw [color=c, fill=c] (8.76617,2.9373) rectangle (8.80597,3.04315);
\definecolor{c}{rgb}{1,0.466667,0};
\draw [color=c, fill=c] (8.80597,2.9373) rectangle (8.84577,3.04315);
\draw [color=c, fill=c] (8.84577,2.9373) rectangle (8.88557,3.04315);
\draw [color=c, fill=c] (8.88557,2.9373) rectangle (8.92537,3.04315);
\draw [color=c, fill=c] (8.92537,2.9373) rectangle (8.96517,3.04315);
\draw [color=c, fill=c] (8.96517,2.9373) rectangle (9.00498,3.04315);
\draw [color=c, fill=c] (9.00498,2.9373) rectangle (9.04478,3.04315);
\draw [color=c, fill=c] (9.04478,2.9373) rectangle (9.08458,3.04315);
\draw [color=c, fill=c] (9.08458,2.9373) rectangle (9.12438,3.04315);
\draw [color=c, fill=c] (9.12438,2.9373) rectangle (9.16418,3.04315);
\draw [color=c, fill=c] (9.16418,2.9373) rectangle (9.20398,3.04315);
\draw [color=c, fill=c] (9.20398,2.9373) rectangle (9.24378,3.04315);
\definecolor{c}{rgb}{1,0.653333,0};
\draw [color=c, fill=c] (9.24378,2.9373) rectangle (9.28358,3.04315);
\draw [color=c, fill=c] (9.28358,2.9373) rectangle (9.32338,3.04315);
\draw [color=c, fill=c] (9.32338,2.9373) rectangle (9.36318,3.04315);
\draw [color=c, fill=c] (9.36318,2.9373) rectangle (9.40298,3.04315);
\draw [color=c, fill=c] (9.40298,2.9373) rectangle (9.44279,3.04315);
\draw [color=c, fill=c] (9.44279,2.9373) rectangle (9.48259,3.04315);
\draw [color=c, fill=c] (9.48259,2.9373) rectangle (9.52239,3.04315);
\definecolor{c}{rgb}{1,0.933333,0};
\draw [color=c, fill=c] (9.52239,2.9373) rectangle (9.56219,3.04315);
\draw [color=c, fill=c] (9.56219,2.9373) rectangle (9.60199,3.04315);
\draw [color=c, fill=c] (9.60199,2.9373) rectangle (9.64179,3.04315);
\draw [color=c, fill=c] (9.64179,2.9373) rectangle (9.68159,3.04315);
\definecolor{c}{rgb}{0.88,1,0};
\draw [color=c, fill=c] (9.68159,2.9373) rectangle (9.72139,3.04315);
\draw [color=c, fill=c] (9.72139,2.9373) rectangle (9.76119,3.04315);
\draw [color=c, fill=c] (9.76119,2.9373) rectangle (9.80099,3.04315);
\definecolor{c}{rgb}{0.6,1,0};
\draw [color=c, fill=c] (9.80099,2.9373) rectangle (9.8408,3.04315);
\draw [color=c, fill=c] (9.8408,2.9373) rectangle (9.8806,3.04315);
\draw [color=c, fill=c] (9.8806,2.9373) rectangle (9.9204,3.04315);
\definecolor{c}{rgb}{0.413333,1,0};
\draw [color=c, fill=c] (9.9204,2.9373) rectangle (9.9602,3.04315);
\draw [color=c, fill=c] (9.9602,2.9373) rectangle (10,3.04315);
\definecolor{c}{rgb}{0.133333,1,0};
\draw [color=c, fill=c] (10,2.9373) rectangle (10.0398,3.04315);
\draw [color=c, fill=c] (10.0398,2.9373) rectangle (10.0796,3.04315);
\draw [color=c, fill=c] (10.0796,2.9373) rectangle (10.1194,3.04315);
\definecolor{c}{rgb}{0,1,0.0533333};
\draw [color=c, fill=c] (10.1194,2.9373) rectangle (10.1592,3.04315);
\draw [color=c, fill=c] (10.1592,2.9373) rectangle (10.199,3.04315);
\draw [color=c, fill=c] (10.199,2.9373) rectangle (10.2388,3.04315);
\definecolor{c}{rgb}{0,1,0.333333};
\draw [color=c, fill=c] (10.2388,2.9373) rectangle (10.2786,3.04315);
\draw [color=c, fill=c] (10.2786,2.9373) rectangle (10.3184,3.04315);
\draw [color=c, fill=c] (10.3184,2.9373) rectangle (10.3582,3.04315);
\draw [color=c, fill=c] (10.3582,2.9373) rectangle (10.398,3.04315);
\draw [color=c, fill=c] (10.398,2.9373) rectangle (10.4378,3.04315);
\definecolor{c}{rgb}{0,1,0.52};
\draw [color=c, fill=c] (10.4378,2.9373) rectangle (10.4776,3.04315);
\draw [color=c, fill=c] (10.4776,2.9373) rectangle (10.5174,3.04315);
\draw [color=c, fill=c] (10.5174,2.9373) rectangle (10.5572,3.04315);
\draw [color=c, fill=c] (10.5572,2.9373) rectangle (10.597,3.04315);
\draw [color=c, fill=c] (10.597,2.9373) rectangle (10.6368,3.04315);
\draw [color=c, fill=c] (10.6368,2.9373) rectangle (10.6766,3.04315);
\draw [color=c, fill=c] (10.6766,2.9373) rectangle (10.7164,3.04315);
\definecolor{c}{rgb}{0,1,0.8};
\draw [color=c, fill=c] (10.7164,2.9373) rectangle (10.7562,3.04315);
\draw [color=c, fill=c] (10.7562,2.9373) rectangle (10.796,3.04315);
\draw [color=c, fill=c] (10.796,2.9373) rectangle (10.8358,3.04315);
\draw [color=c, fill=c] (10.8358,2.9373) rectangle (10.8756,3.04315);
\draw [color=c, fill=c] (10.8756,2.9373) rectangle (10.9154,3.04315);
\draw [color=c, fill=c] (10.9154,2.9373) rectangle (10.9552,3.04315);
\draw [color=c, fill=c] (10.9552,2.9373) rectangle (10.995,3.04315);
\draw [color=c, fill=c] (10.995,2.9373) rectangle (11.0348,3.04315);
\draw [color=c, fill=c] (11.0348,2.9373) rectangle (11.0746,3.04315);
\draw [color=c, fill=c] (11.0746,2.9373) rectangle (11.1144,3.04315);
\draw [color=c, fill=c] (11.1144,2.9373) rectangle (11.1542,3.04315);
\draw [color=c, fill=c] (11.1542,2.9373) rectangle (11.194,3.04315);
\definecolor{c}{rgb}{0,1,0.986667};
\draw [color=c, fill=c] (11.194,2.9373) rectangle (11.2338,3.04315);
\draw [color=c, fill=c] (11.2338,2.9373) rectangle (11.2736,3.04315);
\draw [color=c, fill=c] (11.2736,2.9373) rectangle (11.3134,3.04315);
\draw [color=c, fill=c] (11.3134,2.9373) rectangle (11.3532,3.04315);
\draw [color=c, fill=c] (11.3532,2.9373) rectangle (11.393,3.04315);
\draw [color=c, fill=c] (11.393,2.9373) rectangle (11.4328,3.04315);
\draw [color=c, fill=c] (11.4328,2.9373) rectangle (11.4726,3.04315);
\draw [color=c, fill=c] (11.4726,2.9373) rectangle (11.5124,3.04315);
\draw [color=c, fill=c] (11.5124,2.9373) rectangle (11.5522,3.04315);
\draw [color=c, fill=c] (11.5522,2.9373) rectangle (11.592,3.04315);
\draw [color=c, fill=c] (11.592,2.9373) rectangle (11.6318,3.04315);
\draw [color=c, fill=c] (11.6318,2.9373) rectangle (11.6716,3.04315);
\draw [color=c, fill=c] (11.6716,2.9373) rectangle (11.7114,3.04315);
\draw [color=c, fill=c] (11.7114,2.9373) rectangle (11.7512,3.04315);
\draw [color=c, fill=c] (11.7512,2.9373) rectangle (11.791,3.04315);
\draw [color=c, fill=c] (11.791,2.9373) rectangle (11.8308,3.04315);
\draw [color=c, fill=c] (11.8308,2.9373) rectangle (11.8706,3.04315);
\draw [color=c, fill=c] (11.8706,2.9373) rectangle (11.9104,3.04315);
\draw [color=c, fill=c] (11.9104,2.9373) rectangle (11.9502,3.04315);
\draw [color=c, fill=c] (11.9502,2.9373) rectangle (11.99,3.04315);
\draw [color=c, fill=c] (11.99,2.9373) rectangle (12.0299,3.04315);
\draw [color=c, fill=c] (12.0299,2.9373) rectangle (12.0697,3.04315);
\draw [color=c, fill=c] (12.0697,2.9373) rectangle (12.1095,3.04315);
\draw [color=c, fill=c] (12.1095,2.9373) rectangle (12.1493,3.04315);
\draw [color=c, fill=c] (12.1493,2.9373) rectangle (12.1891,3.04315);
\definecolor{c}{rgb}{0,0.733333,1};
\draw [color=c, fill=c] (12.1891,2.9373) rectangle (12.2289,3.04315);
\draw [color=c, fill=c] (12.2289,2.9373) rectangle (12.2687,3.04315);
\draw [color=c, fill=c] (12.2687,2.9373) rectangle (12.3085,3.04315);
\draw [color=c, fill=c] (12.3085,2.9373) rectangle (12.3483,3.04315);
\draw [color=c, fill=c] (12.3483,2.9373) rectangle (12.3881,3.04315);
\draw [color=c, fill=c] (12.3881,2.9373) rectangle (12.4279,3.04315);
\draw [color=c, fill=c] (12.4279,2.9373) rectangle (12.4677,3.04315);
\draw [color=c, fill=c] (12.4677,2.9373) rectangle (12.5075,3.04315);
\draw [color=c, fill=c] (12.5075,2.9373) rectangle (12.5473,3.04315);
\draw [color=c, fill=c] (12.5473,2.9373) rectangle (12.5871,3.04315);
\draw [color=c, fill=c] (12.5871,2.9373) rectangle (12.6269,3.04315);
\draw [color=c, fill=c] (12.6269,2.9373) rectangle (12.6667,3.04315);
\draw [color=c, fill=c] (12.6667,2.9373) rectangle (12.7065,3.04315);
\draw [color=c, fill=c] (12.7065,2.9373) rectangle (12.7463,3.04315);
\draw [color=c, fill=c] (12.7463,2.9373) rectangle (12.7861,3.04315);
\draw [color=c, fill=c] (12.7861,2.9373) rectangle (12.8259,3.04315);
\draw [color=c, fill=c] (12.8259,2.9373) rectangle (12.8657,3.04315);
\draw [color=c, fill=c] (12.8657,2.9373) rectangle (12.9055,3.04315);
\draw [color=c, fill=c] (12.9055,2.9373) rectangle (12.9453,3.04315);
\draw [color=c, fill=c] (12.9453,2.9373) rectangle (12.9851,3.04315);
\draw [color=c, fill=c] (12.9851,2.9373) rectangle (13.0249,3.04315);
\draw [color=c, fill=c] (13.0249,2.9373) rectangle (13.0647,3.04315);
\draw [color=c, fill=c] (13.0647,2.9373) rectangle (13.1045,3.04315);
\draw [color=c, fill=c] (13.1045,2.9373) rectangle (13.1443,3.04315);
\draw [color=c, fill=c] (13.1443,2.9373) rectangle (13.1841,3.04315);
\draw [color=c, fill=c] (13.1841,2.9373) rectangle (13.2239,3.04315);
\draw [color=c, fill=c] (13.2239,2.9373) rectangle (13.2637,3.04315);
\draw [color=c, fill=c] (13.2637,2.9373) rectangle (13.3035,3.04315);
\draw [color=c, fill=c] (13.3035,2.9373) rectangle (13.3433,3.04315);
\draw [color=c, fill=c] (13.3433,2.9373) rectangle (13.3831,3.04315);
\draw [color=c, fill=c] (13.3831,2.9373) rectangle (13.4229,3.04315);
\draw [color=c, fill=c] (13.4229,2.9373) rectangle (13.4627,3.04315);
\draw [color=c, fill=c] (13.4627,2.9373) rectangle (13.5025,3.04315);
\draw [color=c, fill=c] (13.5025,2.9373) rectangle (13.5423,3.04315);
\draw [color=c, fill=c] (13.5423,2.9373) rectangle (13.5821,3.04315);
\draw [color=c, fill=c] (13.5821,2.9373) rectangle (13.6219,3.04315);
\draw [color=c, fill=c] (13.6219,2.9373) rectangle (13.6617,3.04315);
\draw [color=c, fill=c] (13.6617,2.9373) rectangle (13.7015,3.04315);
\draw [color=c, fill=c] (13.7015,2.9373) rectangle (13.7413,3.04315);
\draw [color=c, fill=c] (13.7413,2.9373) rectangle (13.7811,3.04315);
\draw [color=c, fill=c] (13.7811,2.9373) rectangle (13.8209,3.04315);
\draw [color=c, fill=c] (13.8209,2.9373) rectangle (13.8607,3.04315);
\draw [color=c, fill=c] (13.8607,2.9373) rectangle (13.9005,3.04315);
\draw [color=c, fill=c] (13.9005,2.9373) rectangle (13.9403,3.04315);
\draw [color=c, fill=c] (13.9403,2.9373) rectangle (13.9801,3.04315);
\draw [color=c, fill=c] (13.9801,2.9373) rectangle (14.0199,3.04315);
\draw [color=c, fill=c] (14.0199,2.9373) rectangle (14.0597,3.04315);
\draw [color=c, fill=c] (14.0597,2.9373) rectangle (14.0995,3.04315);
\draw [color=c, fill=c] (14.0995,2.9373) rectangle (14.1393,3.04315);
\draw [color=c, fill=c] (14.1393,2.9373) rectangle (14.1791,3.04315);
\draw [color=c, fill=c] (14.1791,2.9373) rectangle (14.2189,3.04315);
\draw [color=c, fill=c] (14.2189,2.9373) rectangle (14.2587,3.04315);
\draw [color=c, fill=c] (14.2587,2.9373) rectangle (14.2985,3.04315);
\draw [color=c, fill=c] (14.2985,2.9373) rectangle (14.3383,3.04315);
\draw [color=c, fill=c] (14.3383,2.9373) rectangle (14.3781,3.04315);
\draw [color=c, fill=c] (14.3781,2.9373) rectangle (14.4179,3.04315);
\draw [color=c, fill=c] (14.4179,2.9373) rectangle (14.4577,3.04315);
\draw [color=c, fill=c] (14.4577,2.9373) rectangle (14.4975,3.04315);
\draw [color=c, fill=c] (14.4975,2.9373) rectangle (14.5373,3.04315);
\draw [color=c, fill=c] (14.5373,2.9373) rectangle (14.5771,3.04315);
\draw [color=c, fill=c] (14.5771,2.9373) rectangle (14.6169,3.04315);
\draw [color=c, fill=c] (14.6169,2.9373) rectangle (14.6567,3.04315);
\draw [color=c, fill=c] (14.6567,2.9373) rectangle (14.6965,3.04315);
\draw [color=c, fill=c] (14.6965,2.9373) rectangle (14.7363,3.04315);
\draw [color=c, fill=c] (14.7363,2.9373) rectangle (14.7761,3.04315);
\draw [color=c, fill=c] (14.7761,2.9373) rectangle (14.8159,3.04315);
\draw [color=c, fill=c] (14.8159,2.9373) rectangle (14.8557,3.04315);
\draw [color=c, fill=c] (14.8557,2.9373) rectangle (14.8955,3.04315);
\draw [color=c, fill=c] (14.8955,2.9373) rectangle (14.9353,3.04315);
\draw [color=c, fill=c] (14.9353,2.9373) rectangle (14.9751,3.04315);
\draw [color=c, fill=c] (14.9751,2.9373) rectangle (15.0149,3.04315);
\draw [color=c, fill=c] (15.0149,2.9373) rectangle (15.0547,3.04315);
\draw [color=c, fill=c] (15.0547,2.9373) rectangle (15.0945,3.04315);
\draw [color=c, fill=c] (15.0945,2.9373) rectangle (15.1343,3.04315);
\draw [color=c, fill=c] (15.1343,2.9373) rectangle (15.1741,3.04315);
\draw [color=c, fill=c] (15.1741,2.9373) rectangle (15.2139,3.04315);
\draw [color=c, fill=c] (15.2139,2.9373) rectangle (15.2537,3.04315);
\draw [color=c, fill=c] (15.2537,2.9373) rectangle (15.2935,3.04315);
\draw [color=c, fill=c] (15.2935,2.9373) rectangle (15.3333,3.04315);
\draw [color=c, fill=c] (15.3333,2.9373) rectangle (15.3731,3.04315);
\draw [color=c, fill=c] (15.3731,2.9373) rectangle (15.4129,3.04315);
\draw [color=c, fill=c] (15.4129,2.9373) rectangle (15.4527,3.04315);
\draw [color=c, fill=c] (15.4527,2.9373) rectangle (15.4925,3.04315);
\draw [color=c, fill=c] (15.4925,2.9373) rectangle (15.5323,3.04315);
\draw [color=c, fill=c] (15.5323,2.9373) rectangle (15.5721,3.04315);
\draw [color=c, fill=c] (15.5721,2.9373) rectangle (15.6119,3.04315);
\draw [color=c, fill=c] (15.6119,2.9373) rectangle (15.6517,3.04315);
\draw [color=c, fill=c] (15.6517,2.9373) rectangle (15.6915,3.04315);
\draw [color=c, fill=c] (15.6915,2.9373) rectangle (15.7313,3.04315);
\draw [color=c, fill=c] (15.7313,2.9373) rectangle (15.7711,3.04315);
\draw [color=c, fill=c] (15.7711,2.9373) rectangle (15.8109,3.04315);
\draw [color=c, fill=c] (15.8109,2.9373) rectangle (15.8507,3.04315);
\draw [color=c, fill=c] (15.8507,2.9373) rectangle (15.8905,3.04315);
\draw [color=c, fill=c] (15.8905,2.9373) rectangle (15.9303,3.04315);
\draw [color=c, fill=c] (15.9303,2.9373) rectangle (15.9701,3.04315);
\draw [color=c, fill=c] (15.9701,2.9373) rectangle (16.01,3.04315);
\draw [color=c, fill=c] (16.01,2.9373) rectangle (16.0498,3.04315);
\draw [color=c, fill=c] (16.0498,2.9373) rectangle (16.0896,3.04315);
\draw [color=c, fill=c] (16.0896,2.9373) rectangle (16.1294,3.04315);
\draw [color=c, fill=c] (16.1294,2.9373) rectangle (16.1692,3.04315);
\draw [color=c, fill=c] (16.1692,2.9373) rectangle (16.209,3.04315);
\draw [color=c, fill=c] (16.209,2.9373) rectangle (16.2488,3.04315);
\draw [color=c, fill=c] (16.2488,2.9373) rectangle (16.2886,3.04315);
\draw [color=c, fill=c] (16.2886,2.9373) rectangle (16.3284,3.04315);
\draw [color=c, fill=c] (16.3284,2.9373) rectangle (16.3682,3.04315);
\draw [color=c, fill=c] (16.3682,2.9373) rectangle (16.408,3.04315);
\draw [color=c, fill=c] (16.408,2.9373) rectangle (16.4478,3.04315);
\draw [color=c, fill=c] (16.4478,2.9373) rectangle (16.4876,3.04315);
\draw [color=c, fill=c] (16.4876,2.9373) rectangle (16.5274,3.04315);
\draw [color=c, fill=c] (16.5274,2.9373) rectangle (16.5672,3.04315);
\draw [color=c, fill=c] (16.5672,2.9373) rectangle (16.607,3.04315);
\draw [color=c, fill=c] (16.607,2.9373) rectangle (16.6468,3.04315);
\draw [color=c, fill=c] (16.6468,2.9373) rectangle (16.6866,3.04315);
\draw [color=c, fill=c] (16.6866,2.9373) rectangle (16.7264,3.04315);
\draw [color=c, fill=c] (16.7264,2.9373) rectangle (16.7662,3.04315);
\draw [color=c, fill=c] (16.7662,2.9373) rectangle (16.806,3.04315);
\draw [color=c, fill=c] (16.806,2.9373) rectangle (16.8458,3.04315);
\draw [color=c, fill=c] (16.8458,2.9373) rectangle (16.8856,3.04315);
\draw [color=c, fill=c] (16.8856,2.9373) rectangle (16.9254,3.04315);
\draw [color=c, fill=c] (16.9254,2.9373) rectangle (16.9652,3.04315);
\draw [color=c, fill=c] (16.9652,2.9373) rectangle (17.005,3.04315);
\draw [color=c, fill=c] (17.005,2.9373) rectangle (17.0448,3.04315);
\draw [color=c, fill=c] (17.0448,2.9373) rectangle (17.0846,3.04315);
\draw [color=c, fill=c] (17.0846,2.9373) rectangle (17.1244,3.04315);
\draw [color=c, fill=c] (17.1244,2.9373) rectangle (17.1642,3.04315);
\draw [color=c, fill=c] (17.1642,2.9373) rectangle (17.204,3.04315);
\draw [color=c, fill=c] (17.204,2.9373) rectangle (17.2438,3.04315);
\draw [color=c, fill=c] (17.2438,2.9373) rectangle (17.2836,3.04315);
\draw [color=c, fill=c] (17.2836,2.9373) rectangle (17.3234,3.04315);
\draw [color=c, fill=c] (17.3234,2.9373) rectangle (17.3632,3.04315);
\draw [color=c, fill=c] (17.3632,2.9373) rectangle (17.403,3.04315);
\draw [color=c, fill=c] (17.403,2.9373) rectangle (17.4428,3.04315);
\draw [color=c, fill=c] (17.4428,2.9373) rectangle (17.4826,3.04315);
\draw [color=c, fill=c] (17.4826,2.9373) rectangle (17.5224,3.04315);
\draw [color=c, fill=c] (17.5224,2.9373) rectangle (17.5622,3.04315);
\draw [color=c, fill=c] (17.5622,2.9373) rectangle (17.602,3.04315);
\draw [color=c, fill=c] (17.602,2.9373) rectangle (17.6418,3.04315);
\draw [color=c, fill=c] (17.6418,2.9373) rectangle (17.6816,3.04315);
\draw [color=c, fill=c] (17.6816,2.9373) rectangle (17.7214,3.04315);
\draw [color=c, fill=c] (17.7214,2.9373) rectangle (17.7612,3.04315);
\draw [color=c, fill=c] (17.7612,2.9373) rectangle (17.801,3.04315);
\draw [color=c, fill=c] (17.801,2.9373) rectangle (17.8408,3.04315);
\draw [color=c, fill=c] (17.8408,2.9373) rectangle (17.8806,3.04315);
\draw [color=c, fill=c] (17.8806,2.9373) rectangle (17.9204,3.04315);
\draw [color=c, fill=c] (17.9204,2.9373) rectangle (17.9602,3.04315);
\draw [color=c, fill=c] (17.9602,2.9373) rectangle (18,3.04315);
\definecolor{c}{rgb}{1,0,0};
\draw [color=c, fill=c] (2,3.04315) rectangle (2.0398,3.149);
\draw [color=c, fill=c] (2.0398,3.04315) rectangle (2.0796,3.149);
\draw [color=c, fill=c] (2.0796,3.04315) rectangle (2.1194,3.149);
\draw [color=c, fill=c] (2.1194,3.04315) rectangle (2.1592,3.149);
\draw [color=c, fill=c] (2.1592,3.04315) rectangle (2.19901,3.149);
\draw [color=c, fill=c] (2.19901,3.04315) rectangle (2.23881,3.149);
\draw [color=c, fill=c] (2.23881,3.04315) rectangle (2.27861,3.149);
\draw [color=c, fill=c] (2.27861,3.04315) rectangle (2.31841,3.149);
\draw [color=c, fill=c] (2.31841,3.04315) rectangle (2.35821,3.149);
\draw [color=c, fill=c] (2.35821,3.04315) rectangle (2.39801,3.149);
\draw [color=c, fill=c] (2.39801,3.04315) rectangle (2.43781,3.149);
\draw [color=c, fill=c] (2.43781,3.04315) rectangle (2.47761,3.149);
\draw [color=c, fill=c] (2.47761,3.04315) rectangle (2.51741,3.149);
\draw [color=c, fill=c] (2.51741,3.04315) rectangle (2.55721,3.149);
\draw [color=c, fill=c] (2.55721,3.04315) rectangle (2.59702,3.149);
\draw [color=c, fill=c] (2.59702,3.04315) rectangle (2.63682,3.149);
\draw [color=c, fill=c] (2.63682,3.04315) rectangle (2.67662,3.149);
\draw [color=c, fill=c] (2.67662,3.04315) rectangle (2.71642,3.149);
\draw [color=c, fill=c] (2.71642,3.04315) rectangle (2.75622,3.149);
\draw [color=c, fill=c] (2.75622,3.04315) rectangle (2.79602,3.149);
\draw [color=c, fill=c] (2.79602,3.04315) rectangle (2.83582,3.149);
\draw [color=c, fill=c] (2.83582,3.04315) rectangle (2.87562,3.149);
\draw [color=c, fill=c] (2.87562,3.04315) rectangle (2.91542,3.149);
\draw [color=c, fill=c] (2.91542,3.04315) rectangle (2.95522,3.149);
\draw [color=c, fill=c] (2.95522,3.04315) rectangle (2.99502,3.149);
\draw [color=c, fill=c] (2.99502,3.04315) rectangle (3.03483,3.149);
\draw [color=c, fill=c] (3.03483,3.04315) rectangle (3.07463,3.149);
\draw [color=c, fill=c] (3.07463,3.04315) rectangle (3.11443,3.149);
\draw [color=c, fill=c] (3.11443,3.04315) rectangle (3.15423,3.149);
\draw [color=c, fill=c] (3.15423,3.04315) rectangle (3.19403,3.149);
\draw [color=c, fill=c] (3.19403,3.04315) rectangle (3.23383,3.149);
\draw [color=c, fill=c] (3.23383,3.04315) rectangle (3.27363,3.149);
\draw [color=c, fill=c] (3.27363,3.04315) rectangle (3.31343,3.149);
\draw [color=c, fill=c] (3.31343,3.04315) rectangle (3.35323,3.149);
\draw [color=c, fill=c] (3.35323,3.04315) rectangle (3.39303,3.149);
\draw [color=c, fill=c] (3.39303,3.04315) rectangle (3.43284,3.149);
\draw [color=c, fill=c] (3.43284,3.04315) rectangle (3.47264,3.149);
\draw [color=c, fill=c] (3.47264,3.04315) rectangle (3.51244,3.149);
\draw [color=c, fill=c] (3.51244,3.04315) rectangle (3.55224,3.149);
\draw [color=c, fill=c] (3.55224,3.04315) rectangle (3.59204,3.149);
\draw [color=c, fill=c] (3.59204,3.04315) rectangle (3.63184,3.149);
\draw [color=c, fill=c] (3.63184,3.04315) rectangle (3.67164,3.149);
\draw [color=c, fill=c] (3.67164,3.04315) rectangle (3.71144,3.149);
\draw [color=c, fill=c] (3.71144,3.04315) rectangle (3.75124,3.149);
\draw [color=c, fill=c] (3.75124,3.04315) rectangle (3.79104,3.149);
\draw [color=c, fill=c] (3.79104,3.04315) rectangle (3.83085,3.149);
\draw [color=c, fill=c] (3.83085,3.04315) rectangle (3.87065,3.149);
\draw [color=c, fill=c] (3.87065,3.04315) rectangle (3.91045,3.149);
\draw [color=c, fill=c] (3.91045,3.04315) rectangle (3.95025,3.149);
\draw [color=c, fill=c] (3.95025,3.04315) rectangle (3.99005,3.149);
\draw [color=c, fill=c] (3.99005,3.04315) rectangle (4.02985,3.149);
\draw [color=c, fill=c] (4.02985,3.04315) rectangle (4.06965,3.149);
\draw [color=c, fill=c] (4.06965,3.04315) rectangle (4.10945,3.149);
\draw [color=c, fill=c] (4.10945,3.04315) rectangle (4.14925,3.149);
\draw [color=c, fill=c] (4.14925,3.04315) rectangle (4.18905,3.149);
\draw [color=c, fill=c] (4.18905,3.04315) rectangle (4.22886,3.149);
\draw [color=c, fill=c] (4.22886,3.04315) rectangle (4.26866,3.149);
\draw [color=c, fill=c] (4.26866,3.04315) rectangle (4.30846,3.149);
\draw [color=c, fill=c] (4.30846,3.04315) rectangle (4.34826,3.149);
\draw [color=c, fill=c] (4.34826,3.04315) rectangle (4.38806,3.149);
\draw [color=c, fill=c] (4.38806,3.04315) rectangle (4.42786,3.149);
\draw [color=c, fill=c] (4.42786,3.04315) rectangle (4.46766,3.149);
\draw [color=c, fill=c] (4.46766,3.04315) rectangle (4.50746,3.149);
\draw [color=c, fill=c] (4.50746,3.04315) rectangle (4.54726,3.149);
\draw [color=c, fill=c] (4.54726,3.04315) rectangle (4.58706,3.149);
\draw [color=c, fill=c] (4.58706,3.04315) rectangle (4.62687,3.149);
\draw [color=c, fill=c] (4.62687,3.04315) rectangle (4.66667,3.149);
\draw [color=c, fill=c] (4.66667,3.04315) rectangle (4.70647,3.149);
\draw [color=c, fill=c] (4.70647,3.04315) rectangle (4.74627,3.149);
\draw [color=c, fill=c] (4.74627,3.04315) rectangle (4.78607,3.149);
\draw [color=c, fill=c] (4.78607,3.04315) rectangle (4.82587,3.149);
\draw [color=c, fill=c] (4.82587,3.04315) rectangle (4.86567,3.149);
\draw [color=c, fill=c] (4.86567,3.04315) rectangle (4.90547,3.149);
\draw [color=c, fill=c] (4.90547,3.04315) rectangle (4.94527,3.149);
\draw [color=c, fill=c] (4.94527,3.04315) rectangle (4.98507,3.149);
\draw [color=c, fill=c] (4.98507,3.04315) rectangle (5.02488,3.149);
\draw [color=c, fill=c] (5.02488,3.04315) rectangle (5.06468,3.149);
\draw [color=c, fill=c] (5.06468,3.04315) rectangle (5.10448,3.149);
\draw [color=c, fill=c] (5.10448,3.04315) rectangle (5.14428,3.149);
\draw [color=c, fill=c] (5.14428,3.04315) rectangle (5.18408,3.149);
\draw [color=c, fill=c] (5.18408,3.04315) rectangle (5.22388,3.149);
\draw [color=c, fill=c] (5.22388,3.04315) rectangle (5.26368,3.149);
\draw [color=c, fill=c] (5.26368,3.04315) rectangle (5.30348,3.149);
\draw [color=c, fill=c] (5.30348,3.04315) rectangle (5.34328,3.149);
\draw [color=c, fill=c] (5.34328,3.04315) rectangle (5.38308,3.149);
\draw [color=c, fill=c] (5.38308,3.04315) rectangle (5.42289,3.149);
\draw [color=c, fill=c] (5.42289,3.04315) rectangle (5.46269,3.149);
\draw [color=c, fill=c] (5.46269,3.04315) rectangle (5.50249,3.149);
\draw [color=c, fill=c] (5.50249,3.04315) rectangle (5.54229,3.149);
\draw [color=c, fill=c] (5.54229,3.04315) rectangle (5.58209,3.149);
\draw [color=c, fill=c] (5.58209,3.04315) rectangle (5.62189,3.149);
\draw [color=c, fill=c] (5.62189,3.04315) rectangle (5.66169,3.149);
\draw [color=c, fill=c] (5.66169,3.04315) rectangle (5.70149,3.149);
\draw [color=c, fill=c] (5.70149,3.04315) rectangle (5.74129,3.149);
\draw [color=c, fill=c] (5.74129,3.04315) rectangle (5.78109,3.149);
\draw [color=c, fill=c] (5.78109,3.04315) rectangle (5.8209,3.149);
\draw [color=c, fill=c] (5.8209,3.04315) rectangle (5.8607,3.149);
\draw [color=c, fill=c] (5.8607,3.04315) rectangle (5.9005,3.149);
\draw [color=c, fill=c] (5.9005,3.04315) rectangle (5.9403,3.149);
\draw [color=c, fill=c] (5.9403,3.04315) rectangle (5.9801,3.149);
\draw [color=c, fill=c] (5.9801,3.04315) rectangle (6.0199,3.149);
\draw [color=c, fill=c] (6.0199,3.04315) rectangle (6.0597,3.149);
\draw [color=c, fill=c] (6.0597,3.04315) rectangle (6.0995,3.149);
\draw [color=c, fill=c] (6.0995,3.04315) rectangle (6.1393,3.149);
\draw [color=c, fill=c] (6.1393,3.04315) rectangle (6.1791,3.149);
\draw [color=c, fill=c] (6.1791,3.04315) rectangle (6.21891,3.149);
\draw [color=c, fill=c] (6.21891,3.04315) rectangle (6.25871,3.149);
\draw [color=c, fill=c] (6.25871,3.04315) rectangle (6.29851,3.149);
\draw [color=c, fill=c] (6.29851,3.04315) rectangle (6.33831,3.149);
\draw [color=c, fill=c] (6.33831,3.04315) rectangle (6.37811,3.149);
\draw [color=c, fill=c] (6.37811,3.04315) rectangle (6.41791,3.149);
\draw [color=c, fill=c] (6.41791,3.04315) rectangle (6.45771,3.149);
\draw [color=c, fill=c] (6.45771,3.04315) rectangle (6.49751,3.149);
\draw [color=c, fill=c] (6.49751,3.04315) rectangle (6.53731,3.149);
\draw [color=c, fill=c] (6.53731,3.04315) rectangle (6.57711,3.149);
\draw [color=c, fill=c] (6.57711,3.04315) rectangle (6.61692,3.149);
\draw [color=c, fill=c] (6.61692,3.04315) rectangle (6.65672,3.149);
\draw [color=c, fill=c] (6.65672,3.04315) rectangle (6.69652,3.149);
\draw [color=c, fill=c] (6.69652,3.04315) rectangle (6.73632,3.149);
\draw [color=c, fill=c] (6.73632,3.04315) rectangle (6.77612,3.149);
\draw [color=c, fill=c] (6.77612,3.04315) rectangle (6.81592,3.149);
\draw [color=c, fill=c] (6.81592,3.04315) rectangle (6.85572,3.149);
\draw [color=c, fill=c] (6.85572,3.04315) rectangle (6.89552,3.149);
\draw [color=c, fill=c] (6.89552,3.04315) rectangle (6.93532,3.149);
\draw [color=c, fill=c] (6.93532,3.04315) rectangle (6.97512,3.149);
\draw [color=c, fill=c] (6.97512,3.04315) rectangle (7.01493,3.149);
\draw [color=c, fill=c] (7.01493,3.04315) rectangle (7.05473,3.149);
\draw [color=c, fill=c] (7.05473,3.04315) rectangle (7.09453,3.149);
\draw [color=c, fill=c] (7.09453,3.04315) rectangle (7.13433,3.149);
\draw [color=c, fill=c] (7.13433,3.04315) rectangle (7.17413,3.149);
\draw [color=c, fill=c] (7.17413,3.04315) rectangle (7.21393,3.149);
\draw [color=c, fill=c] (7.21393,3.04315) rectangle (7.25373,3.149);
\draw [color=c, fill=c] (7.25373,3.04315) rectangle (7.29353,3.149);
\draw [color=c, fill=c] (7.29353,3.04315) rectangle (7.33333,3.149);
\draw [color=c, fill=c] (7.33333,3.04315) rectangle (7.37313,3.149);
\draw [color=c, fill=c] (7.37313,3.04315) rectangle (7.41294,3.149);
\draw [color=c, fill=c] (7.41294,3.04315) rectangle (7.45274,3.149);
\draw [color=c, fill=c] (7.45274,3.04315) rectangle (7.49254,3.149);
\draw [color=c, fill=c] (7.49254,3.04315) rectangle (7.53234,3.149);
\draw [color=c, fill=c] (7.53234,3.04315) rectangle (7.57214,3.149);
\draw [color=c, fill=c] (7.57214,3.04315) rectangle (7.61194,3.149);
\draw [color=c, fill=c] (7.61194,3.04315) rectangle (7.65174,3.149);
\draw [color=c, fill=c] (7.65174,3.04315) rectangle (7.69154,3.149);
\draw [color=c, fill=c] (7.69154,3.04315) rectangle (7.73134,3.149);
\draw [color=c, fill=c] (7.73134,3.04315) rectangle (7.77114,3.149);
\draw [color=c, fill=c] (7.77114,3.04315) rectangle (7.81095,3.149);
\draw [color=c, fill=c] (7.81095,3.04315) rectangle (7.85075,3.149);
\draw [color=c, fill=c] (7.85075,3.04315) rectangle (7.89055,3.149);
\draw [color=c, fill=c] (7.89055,3.04315) rectangle (7.93035,3.149);
\definecolor{c}{rgb}{1,0.186667,0};
\draw [color=c, fill=c] (7.93035,3.04315) rectangle (7.97015,3.149);
\draw [color=c, fill=c] (7.97015,3.04315) rectangle (8.00995,3.149);
\draw [color=c, fill=c] (8.00995,3.04315) rectangle (8.04975,3.149);
\draw [color=c, fill=c] (8.04975,3.04315) rectangle (8.08955,3.149);
\draw [color=c, fill=c] (8.08955,3.04315) rectangle (8.12935,3.149);
\draw [color=c, fill=c] (8.12935,3.04315) rectangle (8.16915,3.149);
\draw [color=c, fill=c] (8.16915,3.04315) rectangle (8.20895,3.149);
\draw [color=c, fill=c] (8.20895,3.04315) rectangle (8.24876,3.149);
\draw [color=c, fill=c] (8.24876,3.04315) rectangle (8.28856,3.149);
\draw [color=c, fill=c] (8.28856,3.04315) rectangle (8.32836,3.149);
\draw [color=c, fill=c] (8.32836,3.04315) rectangle (8.36816,3.149);
\draw [color=c, fill=c] (8.36816,3.04315) rectangle (8.40796,3.149);
\draw [color=c, fill=c] (8.40796,3.04315) rectangle (8.44776,3.149);
\draw [color=c, fill=c] (8.44776,3.04315) rectangle (8.48756,3.149);
\draw [color=c, fill=c] (8.48756,3.04315) rectangle (8.52736,3.149);
\draw [color=c, fill=c] (8.52736,3.04315) rectangle (8.56716,3.149);
\draw [color=c, fill=c] (8.56716,3.04315) rectangle (8.60697,3.149);
\draw [color=c, fill=c] (8.60697,3.04315) rectangle (8.64677,3.149);
\draw [color=c, fill=c] (8.64677,3.04315) rectangle (8.68657,3.149);
\draw [color=c, fill=c] (8.68657,3.04315) rectangle (8.72637,3.149);
\draw [color=c, fill=c] (8.72637,3.04315) rectangle (8.76617,3.149);
\draw [color=c, fill=c] (8.76617,3.04315) rectangle (8.80597,3.149);
\draw [color=c, fill=c] (8.80597,3.04315) rectangle (8.84577,3.149);
\definecolor{c}{rgb}{1,0.466667,0};
\draw [color=c, fill=c] (8.84577,3.04315) rectangle (8.88557,3.149);
\draw [color=c, fill=c] (8.88557,3.04315) rectangle (8.92537,3.149);
\draw [color=c, fill=c] (8.92537,3.04315) rectangle (8.96517,3.149);
\draw [color=c, fill=c] (8.96517,3.04315) rectangle (9.00498,3.149);
\draw [color=c, fill=c] (9.00498,3.04315) rectangle (9.04478,3.149);
\draw [color=c, fill=c] (9.04478,3.04315) rectangle (9.08458,3.149);
\draw [color=c, fill=c] (9.08458,3.04315) rectangle (9.12438,3.149);
\draw [color=c, fill=c] (9.12438,3.04315) rectangle (9.16418,3.149);
\draw [color=c, fill=c] (9.16418,3.04315) rectangle (9.20398,3.149);
\draw [color=c, fill=c] (9.20398,3.04315) rectangle (9.24378,3.149);
\draw [color=c, fill=c] (9.24378,3.04315) rectangle (9.28358,3.149);
\definecolor{c}{rgb}{1,0.653333,0};
\draw [color=c, fill=c] (9.28358,3.04315) rectangle (9.32338,3.149);
\draw [color=c, fill=c] (9.32338,3.04315) rectangle (9.36318,3.149);
\draw [color=c, fill=c] (9.36318,3.04315) rectangle (9.40298,3.149);
\draw [color=c, fill=c] (9.40298,3.04315) rectangle (9.44279,3.149);
\draw [color=c, fill=c] (9.44279,3.04315) rectangle (9.48259,3.149);
\draw [color=c, fill=c] (9.48259,3.04315) rectangle (9.52239,3.149);
\draw [color=c, fill=c] (9.52239,3.04315) rectangle (9.56219,3.149);
\definecolor{c}{rgb}{1,0.933333,0};
\draw [color=c, fill=c] (9.56219,3.04315) rectangle (9.60199,3.149);
\draw [color=c, fill=c] (9.60199,3.04315) rectangle (9.64179,3.149);
\draw [color=c, fill=c] (9.64179,3.04315) rectangle (9.68159,3.149);
\draw [color=c, fill=c] (9.68159,3.04315) rectangle (9.72139,3.149);
\definecolor{c}{rgb}{0.88,1,0};
\draw [color=c, fill=c] (9.72139,3.04315) rectangle (9.76119,3.149);
\draw [color=c, fill=c] (9.76119,3.04315) rectangle (9.80099,3.149);
\draw [color=c, fill=c] (9.80099,3.04315) rectangle (9.8408,3.149);
\definecolor{c}{rgb}{0.6,1,0};
\draw [color=c, fill=c] (9.8408,3.04315) rectangle (9.8806,3.149);
\draw [color=c, fill=c] (9.8806,3.04315) rectangle (9.9204,3.149);
\definecolor{c}{rgb}{0.413333,1,0};
\draw [color=c, fill=c] (9.9204,3.04315) rectangle (9.9602,3.149);
\draw [color=c, fill=c] (9.9602,3.04315) rectangle (10,3.149);
\definecolor{c}{rgb}{0.133333,1,0};
\draw [color=c, fill=c] (10,3.04315) rectangle (10.0398,3.149);
\draw [color=c, fill=c] (10.0398,3.04315) rectangle (10.0796,3.149);
\draw [color=c, fill=c] (10.0796,3.04315) rectangle (10.1194,3.149);
\definecolor{c}{rgb}{0,1,0.0533333};
\draw [color=c, fill=c] (10.1194,3.04315) rectangle (10.1592,3.149);
\draw [color=c, fill=c] (10.1592,3.04315) rectangle (10.199,3.149);
\draw [color=c, fill=c] (10.199,3.04315) rectangle (10.2388,3.149);
\definecolor{c}{rgb}{0,1,0.333333};
\draw [color=c, fill=c] (10.2388,3.04315) rectangle (10.2786,3.149);
\draw [color=c, fill=c] (10.2786,3.04315) rectangle (10.3184,3.149);
\draw [color=c, fill=c] (10.3184,3.04315) rectangle (10.3582,3.149);
\draw [color=c, fill=c] (10.3582,3.04315) rectangle (10.398,3.149);
\definecolor{c}{rgb}{0,1,0.52};
\draw [color=c, fill=c] (10.398,3.04315) rectangle (10.4378,3.149);
\draw [color=c, fill=c] (10.4378,3.04315) rectangle (10.4776,3.149);
\draw [color=c, fill=c] (10.4776,3.04315) rectangle (10.5174,3.149);
\draw [color=c, fill=c] (10.5174,3.04315) rectangle (10.5572,3.149);
\draw [color=c, fill=c] (10.5572,3.04315) rectangle (10.597,3.149);
\draw [color=c, fill=c] (10.597,3.04315) rectangle (10.6368,3.149);
\draw [color=c, fill=c] (10.6368,3.04315) rectangle (10.6766,3.149);
\definecolor{c}{rgb}{0,1,0.8};
\draw [color=c, fill=c] (10.6766,3.04315) rectangle (10.7164,3.149);
\draw [color=c, fill=c] (10.7164,3.04315) rectangle (10.7562,3.149);
\draw [color=c, fill=c] (10.7562,3.04315) rectangle (10.796,3.149);
\draw [color=c, fill=c] (10.796,3.04315) rectangle (10.8358,3.149);
\draw [color=c, fill=c] (10.8358,3.04315) rectangle (10.8756,3.149);
\draw [color=c, fill=c] (10.8756,3.04315) rectangle (10.9154,3.149);
\draw [color=c, fill=c] (10.9154,3.04315) rectangle (10.9552,3.149);
\draw [color=c, fill=c] (10.9552,3.04315) rectangle (10.995,3.149);
\draw [color=c, fill=c] (10.995,3.04315) rectangle (11.0348,3.149);
\draw [color=c, fill=c] (11.0348,3.04315) rectangle (11.0746,3.149);
\draw [color=c, fill=c] (11.0746,3.04315) rectangle (11.1144,3.149);
\draw [color=c, fill=c] (11.1144,3.04315) rectangle (11.1542,3.149);
\draw [color=c, fill=c] (11.1542,3.04315) rectangle (11.194,3.149);
\definecolor{c}{rgb}{0,1,0.986667};
\draw [color=c, fill=c] (11.194,3.04315) rectangle (11.2338,3.149);
\draw [color=c, fill=c] (11.2338,3.04315) rectangle (11.2736,3.149);
\draw [color=c, fill=c] (11.2736,3.04315) rectangle (11.3134,3.149);
\draw [color=c, fill=c] (11.3134,3.04315) rectangle (11.3532,3.149);
\draw [color=c, fill=c] (11.3532,3.04315) rectangle (11.393,3.149);
\draw [color=c, fill=c] (11.393,3.04315) rectangle (11.4328,3.149);
\draw [color=c, fill=c] (11.4328,3.04315) rectangle (11.4726,3.149);
\draw [color=c, fill=c] (11.4726,3.04315) rectangle (11.5124,3.149);
\draw [color=c, fill=c] (11.5124,3.04315) rectangle (11.5522,3.149);
\draw [color=c, fill=c] (11.5522,3.04315) rectangle (11.592,3.149);
\draw [color=c, fill=c] (11.592,3.04315) rectangle (11.6318,3.149);
\draw [color=c, fill=c] (11.6318,3.04315) rectangle (11.6716,3.149);
\draw [color=c, fill=c] (11.6716,3.04315) rectangle (11.7114,3.149);
\draw [color=c, fill=c] (11.7114,3.04315) rectangle (11.7512,3.149);
\draw [color=c, fill=c] (11.7512,3.04315) rectangle (11.791,3.149);
\draw [color=c, fill=c] (11.791,3.04315) rectangle (11.8308,3.149);
\draw [color=c, fill=c] (11.8308,3.04315) rectangle (11.8706,3.149);
\draw [color=c, fill=c] (11.8706,3.04315) rectangle (11.9104,3.149);
\draw [color=c, fill=c] (11.9104,3.04315) rectangle (11.9502,3.149);
\draw [color=c, fill=c] (11.9502,3.04315) rectangle (11.99,3.149);
\draw [color=c, fill=c] (11.99,3.04315) rectangle (12.0299,3.149);
\draw [color=c, fill=c] (12.0299,3.04315) rectangle (12.0697,3.149);
\draw [color=c, fill=c] (12.0697,3.04315) rectangle (12.1095,3.149);
\draw [color=c, fill=c] (12.1095,3.04315) rectangle (12.1493,3.149);
\definecolor{c}{rgb}{0,0.733333,1};
\draw [color=c, fill=c] (12.1493,3.04315) rectangle (12.1891,3.149);
\draw [color=c, fill=c] (12.1891,3.04315) rectangle (12.2289,3.149);
\draw [color=c, fill=c] (12.2289,3.04315) rectangle (12.2687,3.149);
\draw [color=c, fill=c] (12.2687,3.04315) rectangle (12.3085,3.149);
\draw [color=c, fill=c] (12.3085,3.04315) rectangle (12.3483,3.149);
\draw [color=c, fill=c] (12.3483,3.04315) rectangle (12.3881,3.149);
\draw [color=c, fill=c] (12.3881,3.04315) rectangle (12.4279,3.149);
\draw [color=c, fill=c] (12.4279,3.04315) rectangle (12.4677,3.149);
\draw [color=c, fill=c] (12.4677,3.04315) rectangle (12.5075,3.149);
\draw [color=c, fill=c] (12.5075,3.04315) rectangle (12.5473,3.149);
\draw [color=c, fill=c] (12.5473,3.04315) rectangle (12.5871,3.149);
\draw [color=c, fill=c] (12.5871,3.04315) rectangle (12.6269,3.149);
\draw [color=c, fill=c] (12.6269,3.04315) rectangle (12.6667,3.149);
\draw [color=c, fill=c] (12.6667,3.04315) rectangle (12.7065,3.149);
\draw [color=c, fill=c] (12.7065,3.04315) rectangle (12.7463,3.149);
\draw [color=c, fill=c] (12.7463,3.04315) rectangle (12.7861,3.149);
\draw [color=c, fill=c] (12.7861,3.04315) rectangle (12.8259,3.149);
\draw [color=c, fill=c] (12.8259,3.04315) rectangle (12.8657,3.149);
\draw [color=c, fill=c] (12.8657,3.04315) rectangle (12.9055,3.149);
\draw [color=c, fill=c] (12.9055,3.04315) rectangle (12.9453,3.149);
\draw [color=c, fill=c] (12.9453,3.04315) rectangle (12.9851,3.149);
\draw [color=c, fill=c] (12.9851,3.04315) rectangle (13.0249,3.149);
\draw [color=c, fill=c] (13.0249,3.04315) rectangle (13.0647,3.149);
\draw [color=c, fill=c] (13.0647,3.04315) rectangle (13.1045,3.149);
\draw [color=c, fill=c] (13.1045,3.04315) rectangle (13.1443,3.149);
\draw [color=c, fill=c] (13.1443,3.04315) rectangle (13.1841,3.149);
\draw [color=c, fill=c] (13.1841,3.04315) rectangle (13.2239,3.149);
\draw [color=c, fill=c] (13.2239,3.04315) rectangle (13.2637,3.149);
\draw [color=c, fill=c] (13.2637,3.04315) rectangle (13.3035,3.149);
\draw [color=c, fill=c] (13.3035,3.04315) rectangle (13.3433,3.149);
\draw [color=c, fill=c] (13.3433,3.04315) rectangle (13.3831,3.149);
\draw [color=c, fill=c] (13.3831,3.04315) rectangle (13.4229,3.149);
\draw [color=c, fill=c] (13.4229,3.04315) rectangle (13.4627,3.149);
\draw [color=c, fill=c] (13.4627,3.04315) rectangle (13.5025,3.149);
\draw [color=c, fill=c] (13.5025,3.04315) rectangle (13.5423,3.149);
\draw [color=c, fill=c] (13.5423,3.04315) rectangle (13.5821,3.149);
\draw [color=c, fill=c] (13.5821,3.04315) rectangle (13.6219,3.149);
\draw [color=c, fill=c] (13.6219,3.04315) rectangle (13.6617,3.149);
\draw [color=c, fill=c] (13.6617,3.04315) rectangle (13.7015,3.149);
\draw [color=c, fill=c] (13.7015,3.04315) rectangle (13.7413,3.149);
\draw [color=c, fill=c] (13.7413,3.04315) rectangle (13.7811,3.149);
\draw [color=c, fill=c] (13.7811,3.04315) rectangle (13.8209,3.149);
\draw [color=c, fill=c] (13.8209,3.04315) rectangle (13.8607,3.149);
\draw [color=c, fill=c] (13.8607,3.04315) rectangle (13.9005,3.149);
\draw [color=c, fill=c] (13.9005,3.04315) rectangle (13.9403,3.149);
\draw [color=c, fill=c] (13.9403,3.04315) rectangle (13.9801,3.149);
\draw [color=c, fill=c] (13.9801,3.04315) rectangle (14.0199,3.149);
\draw [color=c, fill=c] (14.0199,3.04315) rectangle (14.0597,3.149);
\draw [color=c, fill=c] (14.0597,3.04315) rectangle (14.0995,3.149);
\draw [color=c, fill=c] (14.0995,3.04315) rectangle (14.1393,3.149);
\draw [color=c, fill=c] (14.1393,3.04315) rectangle (14.1791,3.149);
\draw [color=c, fill=c] (14.1791,3.04315) rectangle (14.2189,3.149);
\draw [color=c, fill=c] (14.2189,3.04315) rectangle (14.2587,3.149);
\draw [color=c, fill=c] (14.2587,3.04315) rectangle (14.2985,3.149);
\draw [color=c, fill=c] (14.2985,3.04315) rectangle (14.3383,3.149);
\draw [color=c, fill=c] (14.3383,3.04315) rectangle (14.3781,3.149);
\draw [color=c, fill=c] (14.3781,3.04315) rectangle (14.4179,3.149);
\draw [color=c, fill=c] (14.4179,3.04315) rectangle (14.4577,3.149);
\draw [color=c, fill=c] (14.4577,3.04315) rectangle (14.4975,3.149);
\draw [color=c, fill=c] (14.4975,3.04315) rectangle (14.5373,3.149);
\draw [color=c, fill=c] (14.5373,3.04315) rectangle (14.5771,3.149);
\draw [color=c, fill=c] (14.5771,3.04315) rectangle (14.6169,3.149);
\draw [color=c, fill=c] (14.6169,3.04315) rectangle (14.6567,3.149);
\draw [color=c, fill=c] (14.6567,3.04315) rectangle (14.6965,3.149);
\draw [color=c, fill=c] (14.6965,3.04315) rectangle (14.7363,3.149);
\draw [color=c, fill=c] (14.7363,3.04315) rectangle (14.7761,3.149);
\draw [color=c, fill=c] (14.7761,3.04315) rectangle (14.8159,3.149);
\draw [color=c, fill=c] (14.8159,3.04315) rectangle (14.8557,3.149);
\draw [color=c, fill=c] (14.8557,3.04315) rectangle (14.8955,3.149);
\draw [color=c, fill=c] (14.8955,3.04315) rectangle (14.9353,3.149);
\draw [color=c, fill=c] (14.9353,3.04315) rectangle (14.9751,3.149);
\draw [color=c, fill=c] (14.9751,3.04315) rectangle (15.0149,3.149);
\draw [color=c, fill=c] (15.0149,3.04315) rectangle (15.0547,3.149);
\draw [color=c, fill=c] (15.0547,3.04315) rectangle (15.0945,3.149);
\draw [color=c, fill=c] (15.0945,3.04315) rectangle (15.1343,3.149);
\draw [color=c, fill=c] (15.1343,3.04315) rectangle (15.1741,3.149);
\draw [color=c, fill=c] (15.1741,3.04315) rectangle (15.2139,3.149);
\draw [color=c, fill=c] (15.2139,3.04315) rectangle (15.2537,3.149);
\draw [color=c, fill=c] (15.2537,3.04315) rectangle (15.2935,3.149);
\draw [color=c, fill=c] (15.2935,3.04315) rectangle (15.3333,3.149);
\draw [color=c, fill=c] (15.3333,3.04315) rectangle (15.3731,3.149);
\draw [color=c, fill=c] (15.3731,3.04315) rectangle (15.4129,3.149);
\draw [color=c, fill=c] (15.4129,3.04315) rectangle (15.4527,3.149);
\draw [color=c, fill=c] (15.4527,3.04315) rectangle (15.4925,3.149);
\draw [color=c, fill=c] (15.4925,3.04315) rectangle (15.5323,3.149);
\draw [color=c, fill=c] (15.5323,3.04315) rectangle (15.5721,3.149);
\draw [color=c, fill=c] (15.5721,3.04315) rectangle (15.6119,3.149);
\draw [color=c, fill=c] (15.6119,3.04315) rectangle (15.6517,3.149);
\draw [color=c, fill=c] (15.6517,3.04315) rectangle (15.6915,3.149);
\draw [color=c, fill=c] (15.6915,3.04315) rectangle (15.7313,3.149);
\draw [color=c, fill=c] (15.7313,3.04315) rectangle (15.7711,3.149);
\draw [color=c, fill=c] (15.7711,3.04315) rectangle (15.8109,3.149);
\draw [color=c, fill=c] (15.8109,3.04315) rectangle (15.8507,3.149);
\draw [color=c, fill=c] (15.8507,3.04315) rectangle (15.8905,3.149);
\draw [color=c, fill=c] (15.8905,3.04315) rectangle (15.9303,3.149);
\draw [color=c, fill=c] (15.9303,3.04315) rectangle (15.9701,3.149);
\draw [color=c, fill=c] (15.9701,3.04315) rectangle (16.01,3.149);
\draw [color=c, fill=c] (16.01,3.04315) rectangle (16.0498,3.149);
\draw [color=c, fill=c] (16.0498,3.04315) rectangle (16.0896,3.149);
\draw [color=c, fill=c] (16.0896,3.04315) rectangle (16.1294,3.149);
\draw [color=c, fill=c] (16.1294,3.04315) rectangle (16.1692,3.149);
\draw [color=c, fill=c] (16.1692,3.04315) rectangle (16.209,3.149);
\draw [color=c, fill=c] (16.209,3.04315) rectangle (16.2488,3.149);
\draw [color=c, fill=c] (16.2488,3.04315) rectangle (16.2886,3.149);
\draw [color=c, fill=c] (16.2886,3.04315) rectangle (16.3284,3.149);
\draw [color=c, fill=c] (16.3284,3.04315) rectangle (16.3682,3.149);
\draw [color=c, fill=c] (16.3682,3.04315) rectangle (16.408,3.149);
\draw [color=c, fill=c] (16.408,3.04315) rectangle (16.4478,3.149);
\draw [color=c, fill=c] (16.4478,3.04315) rectangle (16.4876,3.149);
\draw [color=c, fill=c] (16.4876,3.04315) rectangle (16.5274,3.149);
\draw [color=c, fill=c] (16.5274,3.04315) rectangle (16.5672,3.149);
\draw [color=c, fill=c] (16.5672,3.04315) rectangle (16.607,3.149);
\draw [color=c, fill=c] (16.607,3.04315) rectangle (16.6468,3.149);
\draw [color=c, fill=c] (16.6468,3.04315) rectangle (16.6866,3.149);
\draw [color=c, fill=c] (16.6866,3.04315) rectangle (16.7264,3.149);
\draw [color=c, fill=c] (16.7264,3.04315) rectangle (16.7662,3.149);
\draw [color=c, fill=c] (16.7662,3.04315) rectangle (16.806,3.149);
\draw [color=c, fill=c] (16.806,3.04315) rectangle (16.8458,3.149);
\draw [color=c, fill=c] (16.8458,3.04315) rectangle (16.8856,3.149);
\draw [color=c, fill=c] (16.8856,3.04315) rectangle (16.9254,3.149);
\draw [color=c, fill=c] (16.9254,3.04315) rectangle (16.9652,3.149);
\draw [color=c, fill=c] (16.9652,3.04315) rectangle (17.005,3.149);
\draw [color=c, fill=c] (17.005,3.04315) rectangle (17.0448,3.149);
\draw [color=c, fill=c] (17.0448,3.04315) rectangle (17.0846,3.149);
\draw [color=c, fill=c] (17.0846,3.04315) rectangle (17.1244,3.149);
\draw [color=c, fill=c] (17.1244,3.04315) rectangle (17.1642,3.149);
\draw [color=c, fill=c] (17.1642,3.04315) rectangle (17.204,3.149);
\draw [color=c, fill=c] (17.204,3.04315) rectangle (17.2438,3.149);
\draw [color=c, fill=c] (17.2438,3.04315) rectangle (17.2836,3.149);
\draw [color=c, fill=c] (17.2836,3.04315) rectangle (17.3234,3.149);
\draw [color=c, fill=c] (17.3234,3.04315) rectangle (17.3632,3.149);
\draw [color=c, fill=c] (17.3632,3.04315) rectangle (17.403,3.149);
\draw [color=c, fill=c] (17.403,3.04315) rectangle (17.4428,3.149);
\draw [color=c, fill=c] (17.4428,3.04315) rectangle (17.4826,3.149);
\draw [color=c, fill=c] (17.4826,3.04315) rectangle (17.5224,3.149);
\draw [color=c, fill=c] (17.5224,3.04315) rectangle (17.5622,3.149);
\draw [color=c, fill=c] (17.5622,3.04315) rectangle (17.602,3.149);
\draw [color=c, fill=c] (17.602,3.04315) rectangle (17.6418,3.149);
\draw [color=c, fill=c] (17.6418,3.04315) rectangle (17.6816,3.149);
\draw [color=c, fill=c] (17.6816,3.04315) rectangle (17.7214,3.149);
\draw [color=c, fill=c] (17.7214,3.04315) rectangle (17.7612,3.149);
\draw [color=c, fill=c] (17.7612,3.04315) rectangle (17.801,3.149);
\draw [color=c, fill=c] (17.801,3.04315) rectangle (17.8408,3.149);
\draw [color=c, fill=c] (17.8408,3.04315) rectangle (17.8806,3.149);
\draw [color=c, fill=c] (17.8806,3.04315) rectangle (17.9204,3.149);
\draw [color=c, fill=c] (17.9204,3.04315) rectangle (17.9602,3.149);
\draw [color=c, fill=c] (17.9602,3.04315) rectangle (18,3.149);
\definecolor{c}{rgb}{1,0,0};
\draw [color=c, fill=c] (2,3.149) rectangle (2.0398,3.25485);
\draw [color=c, fill=c] (2.0398,3.149) rectangle (2.0796,3.25485);
\draw [color=c, fill=c] (2.0796,3.149) rectangle (2.1194,3.25485);
\draw [color=c, fill=c] (2.1194,3.149) rectangle (2.1592,3.25485);
\draw [color=c, fill=c] (2.1592,3.149) rectangle (2.19901,3.25485);
\draw [color=c, fill=c] (2.19901,3.149) rectangle (2.23881,3.25485);
\draw [color=c, fill=c] (2.23881,3.149) rectangle (2.27861,3.25485);
\draw [color=c, fill=c] (2.27861,3.149) rectangle (2.31841,3.25485);
\draw [color=c, fill=c] (2.31841,3.149) rectangle (2.35821,3.25485);
\draw [color=c, fill=c] (2.35821,3.149) rectangle (2.39801,3.25485);
\draw [color=c, fill=c] (2.39801,3.149) rectangle (2.43781,3.25485);
\draw [color=c, fill=c] (2.43781,3.149) rectangle (2.47761,3.25485);
\draw [color=c, fill=c] (2.47761,3.149) rectangle (2.51741,3.25485);
\draw [color=c, fill=c] (2.51741,3.149) rectangle (2.55721,3.25485);
\draw [color=c, fill=c] (2.55721,3.149) rectangle (2.59702,3.25485);
\draw [color=c, fill=c] (2.59702,3.149) rectangle (2.63682,3.25485);
\draw [color=c, fill=c] (2.63682,3.149) rectangle (2.67662,3.25485);
\draw [color=c, fill=c] (2.67662,3.149) rectangle (2.71642,3.25485);
\draw [color=c, fill=c] (2.71642,3.149) rectangle (2.75622,3.25485);
\draw [color=c, fill=c] (2.75622,3.149) rectangle (2.79602,3.25485);
\draw [color=c, fill=c] (2.79602,3.149) rectangle (2.83582,3.25485);
\draw [color=c, fill=c] (2.83582,3.149) rectangle (2.87562,3.25485);
\draw [color=c, fill=c] (2.87562,3.149) rectangle (2.91542,3.25485);
\draw [color=c, fill=c] (2.91542,3.149) rectangle (2.95522,3.25485);
\draw [color=c, fill=c] (2.95522,3.149) rectangle (2.99502,3.25485);
\draw [color=c, fill=c] (2.99502,3.149) rectangle (3.03483,3.25485);
\draw [color=c, fill=c] (3.03483,3.149) rectangle (3.07463,3.25485);
\draw [color=c, fill=c] (3.07463,3.149) rectangle (3.11443,3.25485);
\draw [color=c, fill=c] (3.11443,3.149) rectangle (3.15423,3.25485);
\draw [color=c, fill=c] (3.15423,3.149) rectangle (3.19403,3.25485);
\draw [color=c, fill=c] (3.19403,3.149) rectangle (3.23383,3.25485);
\draw [color=c, fill=c] (3.23383,3.149) rectangle (3.27363,3.25485);
\draw [color=c, fill=c] (3.27363,3.149) rectangle (3.31343,3.25485);
\draw [color=c, fill=c] (3.31343,3.149) rectangle (3.35323,3.25485);
\draw [color=c, fill=c] (3.35323,3.149) rectangle (3.39303,3.25485);
\draw [color=c, fill=c] (3.39303,3.149) rectangle (3.43284,3.25485);
\draw [color=c, fill=c] (3.43284,3.149) rectangle (3.47264,3.25485);
\draw [color=c, fill=c] (3.47264,3.149) rectangle (3.51244,3.25485);
\draw [color=c, fill=c] (3.51244,3.149) rectangle (3.55224,3.25485);
\draw [color=c, fill=c] (3.55224,3.149) rectangle (3.59204,3.25485);
\draw [color=c, fill=c] (3.59204,3.149) rectangle (3.63184,3.25485);
\draw [color=c, fill=c] (3.63184,3.149) rectangle (3.67164,3.25485);
\draw [color=c, fill=c] (3.67164,3.149) rectangle (3.71144,3.25485);
\draw [color=c, fill=c] (3.71144,3.149) rectangle (3.75124,3.25485);
\draw [color=c, fill=c] (3.75124,3.149) rectangle (3.79104,3.25485);
\draw [color=c, fill=c] (3.79104,3.149) rectangle (3.83085,3.25485);
\draw [color=c, fill=c] (3.83085,3.149) rectangle (3.87065,3.25485);
\draw [color=c, fill=c] (3.87065,3.149) rectangle (3.91045,3.25485);
\draw [color=c, fill=c] (3.91045,3.149) rectangle (3.95025,3.25485);
\draw [color=c, fill=c] (3.95025,3.149) rectangle (3.99005,3.25485);
\draw [color=c, fill=c] (3.99005,3.149) rectangle (4.02985,3.25485);
\draw [color=c, fill=c] (4.02985,3.149) rectangle (4.06965,3.25485);
\draw [color=c, fill=c] (4.06965,3.149) rectangle (4.10945,3.25485);
\draw [color=c, fill=c] (4.10945,3.149) rectangle (4.14925,3.25485);
\draw [color=c, fill=c] (4.14925,3.149) rectangle (4.18905,3.25485);
\draw [color=c, fill=c] (4.18905,3.149) rectangle (4.22886,3.25485);
\draw [color=c, fill=c] (4.22886,3.149) rectangle (4.26866,3.25485);
\draw [color=c, fill=c] (4.26866,3.149) rectangle (4.30846,3.25485);
\draw [color=c, fill=c] (4.30846,3.149) rectangle (4.34826,3.25485);
\draw [color=c, fill=c] (4.34826,3.149) rectangle (4.38806,3.25485);
\draw [color=c, fill=c] (4.38806,3.149) rectangle (4.42786,3.25485);
\draw [color=c, fill=c] (4.42786,3.149) rectangle (4.46766,3.25485);
\draw [color=c, fill=c] (4.46766,3.149) rectangle (4.50746,3.25485);
\draw [color=c, fill=c] (4.50746,3.149) rectangle (4.54726,3.25485);
\draw [color=c, fill=c] (4.54726,3.149) rectangle (4.58706,3.25485);
\draw [color=c, fill=c] (4.58706,3.149) rectangle (4.62687,3.25485);
\draw [color=c, fill=c] (4.62687,3.149) rectangle (4.66667,3.25485);
\draw [color=c, fill=c] (4.66667,3.149) rectangle (4.70647,3.25485);
\draw [color=c, fill=c] (4.70647,3.149) rectangle (4.74627,3.25485);
\draw [color=c, fill=c] (4.74627,3.149) rectangle (4.78607,3.25485);
\draw [color=c, fill=c] (4.78607,3.149) rectangle (4.82587,3.25485);
\draw [color=c, fill=c] (4.82587,3.149) rectangle (4.86567,3.25485);
\draw [color=c, fill=c] (4.86567,3.149) rectangle (4.90547,3.25485);
\draw [color=c, fill=c] (4.90547,3.149) rectangle (4.94527,3.25485);
\draw [color=c, fill=c] (4.94527,3.149) rectangle (4.98507,3.25485);
\draw [color=c, fill=c] (4.98507,3.149) rectangle (5.02488,3.25485);
\draw [color=c, fill=c] (5.02488,3.149) rectangle (5.06468,3.25485);
\draw [color=c, fill=c] (5.06468,3.149) rectangle (5.10448,3.25485);
\draw [color=c, fill=c] (5.10448,3.149) rectangle (5.14428,3.25485);
\draw [color=c, fill=c] (5.14428,3.149) rectangle (5.18408,3.25485);
\draw [color=c, fill=c] (5.18408,3.149) rectangle (5.22388,3.25485);
\draw [color=c, fill=c] (5.22388,3.149) rectangle (5.26368,3.25485);
\draw [color=c, fill=c] (5.26368,3.149) rectangle (5.30348,3.25485);
\draw [color=c, fill=c] (5.30348,3.149) rectangle (5.34328,3.25485);
\draw [color=c, fill=c] (5.34328,3.149) rectangle (5.38308,3.25485);
\draw [color=c, fill=c] (5.38308,3.149) rectangle (5.42289,3.25485);
\draw [color=c, fill=c] (5.42289,3.149) rectangle (5.46269,3.25485);
\draw [color=c, fill=c] (5.46269,3.149) rectangle (5.50249,3.25485);
\draw [color=c, fill=c] (5.50249,3.149) rectangle (5.54229,3.25485);
\draw [color=c, fill=c] (5.54229,3.149) rectangle (5.58209,3.25485);
\draw [color=c, fill=c] (5.58209,3.149) rectangle (5.62189,3.25485);
\draw [color=c, fill=c] (5.62189,3.149) rectangle (5.66169,3.25485);
\draw [color=c, fill=c] (5.66169,3.149) rectangle (5.70149,3.25485);
\draw [color=c, fill=c] (5.70149,3.149) rectangle (5.74129,3.25485);
\draw [color=c, fill=c] (5.74129,3.149) rectangle (5.78109,3.25485);
\draw [color=c, fill=c] (5.78109,3.149) rectangle (5.8209,3.25485);
\draw [color=c, fill=c] (5.8209,3.149) rectangle (5.8607,3.25485);
\draw [color=c, fill=c] (5.8607,3.149) rectangle (5.9005,3.25485);
\draw [color=c, fill=c] (5.9005,3.149) rectangle (5.9403,3.25485);
\draw [color=c, fill=c] (5.9403,3.149) rectangle (5.9801,3.25485);
\draw [color=c, fill=c] (5.9801,3.149) rectangle (6.0199,3.25485);
\draw [color=c, fill=c] (6.0199,3.149) rectangle (6.0597,3.25485);
\draw [color=c, fill=c] (6.0597,3.149) rectangle (6.0995,3.25485);
\draw [color=c, fill=c] (6.0995,3.149) rectangle (6.1393,3.25485);
\draw [color=c, fill=c] (6.1393,3.149) rectangle (6.1791,3.25485);
\draw [color=c, fill=c] (6.1791,3.149) rectangle (6.21891,3.25485);
\draw [color=c, fill=c] (6.21891,3.149) rectangle (6.25871,3.25485);
\draw [color=c, fill=c] (6.25871,3.149) rectangle (6.29851,3.25485);
\draw [color=c, fill=c] (6.29851,3.149) rectangle (6.33831,3.25485);
\draw [color=c, fill=c] (6.33831,3.149) rectangle (6.37811,3.25485);
\draw [color=c, fill=c] (6.37811,3.149) rectangle (6.41791,3.25485);
\draw [color=c, fill=c] (6.41791,3.149) rectangle (6.45771,3.25485);
\draw [color=c, fill=c] (6.45771,3.149) rectangle (6.49751,3.25485);
\draw [color=c, fill=c] (6.49751,3.149) rectangle (6.53731,3.25485);
\draw [color=c, fill=c] (6.53731,3.149) rectangle (6.57711,3.25485);
\draw [color=c, fill=c] (6.57711,3.149) rectangle (6.61692,3.25485);
\draw [color=c, fill=c] (6.61692,3.149) rectangle (6.65672,3.25485);
\draw [color=c, fill=c] (6.65672,3.149) rectangle (6.69652,3.25485);
\draw [color=c, fill=c] (6.69652,3.149) rectangle (6.73632,3.25485);
\draw [color=c, fill=c] (6.73632,3.149) rectangle (6.77612,3.25485);
\draw [color=c, fill=c] (6.77612,3.149) rectangle (6.81592,3.25485);
\draw [color=c, fill=c] (6.81592,3.149) rectangle (6.85572,3.25485);
\draw [color=c, fill=c] (6.85572,3.149) rectangle (6.89552,3.25485);
\draw [color=c, fill=c] (6.89552,3.149) rectangle (6.93532,3.25485);
\draw [color=c, fill=c] (6.93532,3.149) rectangle (6.97512,3.25485);
\draw [color=c, fill=c] (6.97512,3.149) rectangle (7.01493,3.25485);
\draw [color=c, fill=c] (7.01493,3.149) rectangle (7.05473,3.25485);
\draw [color=c, fill=c] (7.05473,3.149) rectangle (7.09453,3.25485);
\draw [color=c, fill=c] (7.09453,3.149) rectangle (7.13433,3.25485);
\draw [color=c, fill=c] (7.13433,3.149) rectangle (7.17413,3.25485);
\draw [color=c, fill=c] (7.17413,3.149) rectangle (7.21393,3.25485);
\draw [color=c, fill=c] (7.21393,3.149) rectangle (7.25373,3.25485);
\draw [color=c, fill=c] (7.25373,3.149) rectangle (7.29353,3.25485);
\draw [color=c, fill=c] (7.29353,3.149) rectangle (7.33333,3.25485);
\draw [color=c, fill=c] (7.33333,3.149) rectangle (7.37313,3.25485);
\draw [color=c, fill=c] (7.37313,3.149) rectangle (7.41294,3.25485);
\draw [color=c, fill=c] (7.41294,3.149) rectangle (7.45274,3.25485);
\draw [color=c, fill=c] (7.45274,3.149) rectangle (7.49254,3.25485);
\draw [color=c, fill=c] (7.49254,3.149) rectangle (7.53234,3.25485);
\draw [color=c, fill=c] (7.53234,3.149) rectangle (7.57214,3.25485);
\draw [color=c, fill=c] (7.57214,3.149) rectangle (7.61194,3.25485);
\draw [color=c, fill=c] (7.61194,3.149) rectangle (7.65174,3.25485);
\draw [color=c, fill=c] (7.65174,3.149) rectangle (7.69154,3.25485);
\draw [color=c, fill=c] (7.69154,3.149) rectangle (7.73134,3.25485);
\draw [color=c, fill=c] (7.73134,3.149) rectangle (7.77114,3.25485);
\draw [color=c, fill=c] (7.77114,3.149) rectangle (7.81095,3.25485);
\draw [color=c, fill=c] (7.81095,3.149) rectangle (7.85075,3.25485);
\draw [color=c, fill=c] (7.85075,3.149) rectangle (7.89055,3.25485);
\draw [color=c, fill=c] (7.89055,3.149) rectangle (7.93035,3.25485);
\draw [color=c, fill=c] (7.93035,3.149) rectangle (7.97015,3.25485);
\definecolor{c}{rgb}{1,0.186667,0};
\draw [color=c, fill=c] (7.97015,3.149) rectangle (8.00995,3.25485);
\draw [color=c, fill=c] (8.00995,3.149) rectangle (8.04975,3.25485);
\draw [color=c, fill=c] (8.04975,3.149) rectangle (8.08955,3.25485);
\draw [color=c, fill=c] (8.08955,3.149) rectangle (8.12935,3.25485);
\draw [color=c, fill=c] (8.12935,3.149) rectangle (8.16915,3.25485);
\draw [color=c, fill=c] (8.16915,3.149) rectangle (8.20895,3.25485);
\draw [color=c, fill=c] (8.20895,3.149) rectangle (8.24876,3.25485);
\draw [color=c, fill=c] (8.24876,3.149) rectangle (8.28856,3.25485);
\draw [color=c, fill=c] (8.28856,3.149) rectangle (8.32836,3.25485);
\draw [color=c, fill=c] (8.32836,3.149) rectangle (8.36816,3.25485);
\draw [color=c, fill=c] (8.36816,3.149) rectangle (8.40796,3.25485);
\draw [color=c, fill=c] (8.40796,3.149) rectangle (8.44776,3.25485);
\draw [color=c, fill=c] (8.44776,3.149) rectangle (8.48756,3.25485);
\draw [color=c, fill=c] (8.48756,3.149) rectangle (8.52736,3.25485);
\draw [color=c, fill=c] (8.52736,3.149) rectangle (8.56716,3.25485);
\draw [color=c, fill=c] (8.56716,3.149) rectangle (8.60697,3.25485);
\draw [color=c, fill=c] (8.60697,3.149) rectangle (8.64677,3.25485);
\draw [color=c, fill=c] (8.64677,3.149) rectangle (8.68657,3.25485);
\draw [color=c, fill=c] (8.68657,3.149) rectangle (8.72637,3.25485);
\draw [color=c, fill=c] (8.72637,3.149) rectangle (8.76617,3.25485);
\draw [color=c, fill=c] (8.76617,3.149) rectangle (8.80597,3.25485);
\draw [color=c, fill=c] (8.80597,3.149) rectangle (8.84577,3.25485);
\definecolor{c}{rgb}{1,0.466667,0};
\draw [color=c, fill=c] (8.84577,3.149) rectangle (8.88557,3.25485);
\draw [color=c, fill=c] (8.88557,3.149) rectangle (8.92537,3.25485);
\draw [color=c, fill=c] (8.92537,3.149) rectangle (8.96517,3.25485);
\draw [color=c, fill=c] (8.96517,3.149) rectangle (9.00498,3.25485);
\draw [color=c, fill=c] (9.00498,3.149) rectangle (9.04478,3.25485);
\draw [color=c, fill=c] (9.04478,3.149) rectangle (9.08458,3.25485);
\draw [color=c, fill=c] (9.08458,3.149) rectangle (9.12438,3.25485);
\draw [color=c, fill=c] (9.12438,3.149) rectangle (9.16418,3.25485);
\draw [color=c, fill=c] (9.16418,3.149) rectangle (9.20398,3.25485);
\draw [color=c, fill=c] (9.20398,3.149) rectangle (9.24378,3.25485);
\draw [color=c, fill=c] (9.24378,3.149) rectangle (9.28358,3.25485);
\draw [color=c, fill=c] (9.28358,3.149) rectangle (9.32338,3.25485);
\definecolor{c}{rgb}{1,0.653333,0};
\draw [color=c, fill=c] (9.32338,3.149) rectangle (9.36318,3.25485);
\draw [color=c, fill=c] (9.36318,3.149) rectangle (9.40298,3.25485);
\draw [color=c, fill=c] (9.40298,3.149) rectangle (9.44279,3.25485);
\draw [color=c, fill=c] (9.44279,3.149) rectangle (9.48259,3.25485);
\draw [color=c, fill=c] (9.48259,3.149) rectangle (9.52239,3.25485);
\draw [color=c, fill=c] (9.52239,3.149) rectangle (9.56219,3.25485);
\draw [color=c, fill=c] (9.56219,3.149) rectangle (9.60199,3.25485);
\definecolor{c}{rgb}{1,0.933333,0};
\draw [color=c, fill=c] (9.60199,3.149) rectangle (9.64179,3.25485);
\draw [color=c, fill=c] (9.64179,3.149) rectangle (9.68159,3.25485);
\draw [color=c, fill=c] (9.68159,3.149) rectangle (9.72139,3.25485);
\draw [color=c, fill=c] (9.72139,3.149) rectangle (9.76119,3.25485);
\definecolor{c}{rgb}{0.88,1,0};
\draw [color=c, fill=c] (9.76119,3.149) rectangle (9.80099,3.25485);
\draw [color=c, fill=c] (9.80099,3.149) rectangle (9.8408,3.25485);
\definecolor{c}{rgb}{0.6,1,0};
\draw [color=c, fill=c] (9.8408,3.149) rectangle (9.8806,3.25485);
\draw [color=c, fill=c] (9.8806,3.149) rectangle (9.9204,3.25485);
\definecolor{c}{rgb}{0.413333,1,0};
\draw [color=c, fill=c] (9.9204,3.149) rectangle (9.9602,3.25485);
\draw [color=c, fill=c] (9.9602,3.149) rectangle (10,3.25485);
\definecolor{c}{rgb}{0.133333,1,0};
\draw [color=c, fill=c] (10,3.149) rectangle (10.0398,3.25485);
\draw [color=c, fill=c] (10.0398,3.149) rectangle (10.0796,3.25485);
\definecolor{c}{rgb}{0,1,0.0533333};
\draw [color=c, fill=c] (10.0796,3.149) rectangle (10.1194,3.25485);
\draw [color=c, fill=c] (10.1194,3.149) rectangle (10.1592,3.25485);
\draw [color=c, fill=c] (10.1592,3.149) rectangle (10.199,3.25485);
\definecolor{c}{rgb}{0,1,0.333333};
\draw [color=c, fill=c] (10.199,3.149) rectangle (10.2388,3.25485);
\draw [color=c, fill=c] (10.2388,3.149) rectangle (10.2786,3.25485);
\draw [color=c, fill=c] (10.2786,3.149) rectangle (10.3184,3.25485);
\draw [color=c, fill=c] (10.3184,3.149) rectangle (10.3582,3.25485);
\definecolor{c}{rgb}{0,1,0.52};
\draw [color=c, fill=c] (10.3582,3.149) rectangle (10.398,3.25485);
\draw [color=c, fill=c] (10.398,3.149) rectangle (10.4378,3.25485);
\draw [color=c, fill=c] (10.4378,3.149) rectangle (10.4776,3.25485);
\draw [color=c, fill=c] (10.4776,3.149) rectangle (10.5174,3.25485);
\draw [color=c, fill=c] (10.5174,3.149) rectangle (10.5572,3.25485);
\draw [color=c, fill=c] (10.5572,3.149) rectangle (10.597,3.25485);
\draw [color=c, fill=c] (10.597,3.149) rectangle (10.6368,3.25485);
\definecolor{c}{rgb}{0,1,0.8};
\draw [color=c, fill=c] (10.6368,3.149) rectangle (10.6766,3.25485);
\draw [color=c, fill=c] (10.6766,3.149) rectangle (10.7164,3.25485);
\draw [color=c, fill=c] (10.7164,3.149) rectangle (10.7562,3.25485);
\draw [color=c, fill=c] (10.7562,3.149) rectangle (10.796,3.25485);
\draw [color=c, fill=c] (10.796,3.149) rectangle (10.8358,3.25485);
\draw [color=c, fill=c] (10.8358,3.149) rectangle (10.8756,3.25485);
\draw [color=c, fill=c] (10.8756,3.149) rectangle (10.9154,3.25485);
\draw [color=c, fill=c] (10.9154,3.149) rectangle (10.9552,3.25485);
\draw [color=c, fill=c] (10.9552,3.149) rectangle (10.995,3.25485);
\draw [color=c, fill=c] (10.995,3.149) rectangle (11.0348,3.25485);
\draw [color=c, fill=c] (11.0348,3.149) rectangle (11.0746,3.25485);
\draw [color=c, fill=c] (11.0746,3.149) rectangle (11.1144,3.25485);
\draw [color=c, fill=c] (11.1144,3.149) rectangle (11.1542,3.25485);
\definecolor{c}{rgb}{0,1,0.986667};
\draw [color=c, fill=c] (11.1542,3.149) rectangle (11.194,3.25485);
\draw [color=c, fill=c] (11.194,3.149) rectangle (11.2338,3.25485);
\draw [color=c, fill=c] (11.2338,3.149) rectangle (11.2736,3.25485);
\draw [color=c, fill=c] (11.2736,3.149) rectangle (11.3134,3.25485);
\draw [color=c, fill=c] (11.3134,3.149) rectangle (11.3532,3.25485);
\draw [color=c, fill=c] (11.3532,3.149) rectangle (11.393,3.25485);
\draw [color=c, fill=c] (11.393,3.149) rectangle (11.4328,3.25485);
\draw [color=c, fill=c] (11.4328,3.149) rectangle (11.4726,3.25485);
\draw [color=c, fill=c] (11.4726,3.149) rectangle (11.5124,3.25485);
\draw [color=c, fill=c] (11.5124,3.149) rectangle (11.5522,3.25485);
\draw [color=c, fill=c] (11.5522,3.149) rectangle (11.592,3.25485);
\draw [color=c, fill=c] (11.592,3.149) rectangle (11.6318,3.25485);
\draw [color=c, fill=c] (11.6318,3.149) rectangle (11.6716,3.25485);
\draw [color=c, fill=c] (11.6716,3.149) rectangle (11.7114,3.25485);
\draw [color=c, fill=c] (11.7114,3.149) rectangle (11.7512,3.25485);
\draw [color=c, fill=c] (11.7512,3.149) rectangle (11.791,3.25485);
\draw [color=c, fill=c] (11.791,3.149) rectangle (11.8308,3.25485);
\draw [color=c, fill=c] (11.8308,3.149) rectangle (11.8706,3.25485);
\draw [color=c, fill=c] (11.8706,3.149) rectangle (11.9104,3.25485);
\draw [color=c, fill=c] (11.9104,3.149) rectangle (11.9502,3.25485);
\draw [color=c, fill=c] (11.9502,3.149) rectangle (11.99,3.25485);
\draw [color=c, fill=c] (11.99,3.149) rectangle (12.0299,3.25485);
\draw [color=c, fill=c] (12.0299,3.149) rectangle (12.0697,3.25485);
\draw [color=c, fill=c] (12.0697,3.149) rectangle (12.1095,3.25485);
\draw [color=c, fill=c] (12.1095,3.149) rectangle (12.1493,3.25485);
\definecolor{c}{rgb}{0,0.733333,1};
\draw [color=c, fill=c] (12.1493,3.149) rectangle (12.1891,3.25485);
\draw [color=c, fill=c] (12.1891,3.149) rectangle (12.2289,3.25485);
\draw [color=c, fill=c] (12.2289,3.149) rectangle (12.2687,3.25485);
\draw [color=c, fill=c] (12.2687,3.149) rectangle (12.3085,3.25485);
\draw [color=c, fill=c] (12.3085,3.149) rectangle (12.3483,3.25485);
\draw [color=c, fill=c] (12.3483,3.149) rectangle (12.3881,3.25485);
\draw [color=c, fill=c] (12.3881,3.149) rectangle (12.4279,3.25485);
\draw [color=c, fill=c] (12.4279,3.149) rectangle (12.4677,3.25485);
\draw [color=c, fill=c] (12.4677,3.149) rectangle (12.5075,3.25485);
\draw [color=c, fill=c] (12.5075,3.149) rectangle (12.5473,3.25485);
\draw [color=c, fill=c] (12.5473,3.149) rectangle (12.5871,3.25485);
\draw [color=c, fill=c] (12.5871,3.149) rectangle (12.6269,3.25485);
\draw [color=c, fill=c] (12.6269,3.149) rectangle (12.6667,3.25485);
\draw [color=c, fill=c] (12.6667,3.149) rectangle (12.7065,3.25485);
\draw [color=c, fill=c] (12.7065,3.149) rectangle (12.7463,3.25485);
\draw [color=c, fill=c] (12.7463,3.149) rectangle (12.7861,3.25485);
\draw [color=c, fill=c] (12.7861,3.149) rectangle (12.8259,3.25485);
\draw [color=c, fill=c] (12.8259,3.149) rectangle (12.8657,3.25485);
\draw [color=c, fill=c] (12.8657,3.149) rectangle (12.9055,3.25485);
\draw [color=c, fill=c] (12.9055,3.149) rectangle (12.9453,3.25485);
\draw [color=c, fill=c] (12.9453,3.149) rectangle (12.9851,3.25485);
\draw [color=c, fill=c] (12.9851,3.149) rectangle (13.0249,3.25485);
\draw [color=c, fill=c] (13.0249,3.149) rectangle (13.0647,3.25485);
\draw [color=c, fill=c] (13.0647,3.149) rectangle (13.1045,3.25485);
\draw [color=c, fill=c] (13.1045,3.149) rectangle (13.1443,3.25485);
\draw [color=c, fill=c] (13.1443,3.149) rectangle (13.1841,3.25485);
\draw [color=c, fill=c] (13.1841,3.149) rectangle (13.2239,3.25485);
\draw [color=c, fill=c] (13.2239,3.149) rectangle (13.2637,3.25485);
\draw [color=c, fill=c] (13.2637,3.149) rectangle (13.3035,3.25485);
\draw [color=c, fill=c] (13.3035,3.149) rectangle (13.3433,3.25485);
\draw [color=c, fill=c] (13.3433,3.149) rectangle (13.3831,3.25485);
\draw [color=c, fill=c] (13.3831,3.149) rectangle (13.4229,3.25485);
\draw [color=c, fill=c] (13.4229,3.149) rectangle (13.4627,3.25485);
\draw [color=c, fill=c] (13.4627,3.149) rectangle (13.5025,3.25485);
\draw [color=c, fill=c] (13.5025,3.149) rectangle (13.5423,3.25485);
\draw [color=c, fill=c] (13.5423,3.149) rectangle (13.5821,3.25485);
\draw [color=c, fill=c] (13.5821,3.149) rectangle (13.6219,3.25485);
\draw [color=c, fill=c] (13.6219,3.149) rectangle (13.6617,3.25485);
\draw [color=c, fill=c] (13.6617,3.149) rectangle (13.7015,3.25485);
\draw [color=c, fill=c] (13.7015,3.149) rectangle (13.7413,3.25485);
\draw [color=c, fill=c] (13.7413,3.149) rectangle (13.7811,3.25485);
\draw [color=c, fill=c] (13.7811,3.149) rectangle (13.8209,3.25485);
\draw [color=c, fill=c] (13.8209,3.149) rectangle (13.8607,3.25485);
\draw [color=c, fill=c] (13.8607,3.149) rectangle (13.9005,3.25485);
\draw [color=c, fill=c] (13.9005,3.149) rectangle (13.9403,3.25485);
\draw [color=c, fill=c] (13.9403,3.149) rectangle (13.9801,3.25485);
\draw [color=c, fill=c] (13.9801,3.149) rectangle (14.0199,3.25485);
\draw [color=c, fill=c] (14.0199,3.149) rectangle (14.0597,3.25485);
\draw [color=c, fill=c] (14.0597,3.149) rectangle (14.0995,3.25485);
\draw [color=c, fill=c] (14.0995,3.149) rectangle (14.1393,3.25485);
\draw [color=c, fill=c] (14.1393,3.149) rectangle (14.1791,3.25485);
\draw [color=c, fill=c] (14.1791,3.149) rectangle (14.2189,3.25485);
\draw [color=c, fill=c] (14.2189,3.149) rectangle (14.2587,3.25485);
\draw [color=c, fill=c] (14.2587,3.149) rectangle (14.2985,3.25485);
\draw [color=c, fill=c] (14.2985,3.149) rectangle (14.3383,3.25485);
\draw [color=c, fill=c] (14.3383,3.149) rectangle (14.3781,3.25485);
\draw [color=c, fill=c] (14.3781,3.149) rectangle (14.4179,3.25485);
\draw [color=c, fill=c] (14.4179,3.149) rectangle (14.4577,3.25485);
\draw [color=c, fill=c] (14.4577,3.149) rectangle (14.4975,3.25485);
\draw [color=c, fill=c] (14.4975,3.149) rectangle (14.5373,3.25485);
\draw [color=c, fill=c] (14.5373,3.149) rectangle (14.5771,3.25485);
\draw [color=c, fill=c] (14.5771,3.149) rectangle (14.6169,3.25485);
\draw [color=c, fill=c] (14.6169,3.149) rectangle (14.6567,3.25485);
\draw [color=c, fill=c] (14.6567,3.149) rectangle (14.6965,3.25485);
\draw [color=c, fill=c] (14.6965,3.149) rectangle (14.7363,3.25485);
\draw [color=c, fill=c] (14.7363,3.149) rectangle (14.7761,3.25485);
\draw [color=c, fill=c] (14.7761,3.149) rectangle (14.8159,3.25485);
\draw [color=c, fill=c] (14.8159,3.149) rectangle (14.8557,3.25485);
\draw [color=c, fill=c] (14.8557,3.149) rectangle (14.8955,3.25485);
\draw [color=c, fill=c] (14.8955,3.149) rectangle (14.9353,3.25485);
\draw [color=c, fill=c] (14.9353,3.149) rectangle (14.9751,3.25485);
\draw [color=c, fill=c] (14.9751,3.149) rectangle (15.0149,3.25485);
\draw [color=c, fill=c] (15.0149,3.149) rectangle (15.0547,3.25485);
\draw [color=c, fill=c] (15.0547,3.149) rectangle (15.0945,3.25485);
\draw [color=c, fill=c] (15.0945,3.149) rectangle (15.1343,3.25485);
\draw [color=c, fill=c] (15.1343,3.149) rectangle (15.1741,3.25485);
\draw [color=c, fill=c] (15.1741,3.149) rectangle (15.2139,3.25485);
\draw [color=c, fill=c] (15.2139,3.149) rectangle (15.2537,3.25485);
\draw [color=c, fill=c] (15.2537,3.149) rectangle (15.2935,3.25485);
\draw [color=c, fill=c] (15.2935,3.149) rectangle (15.3333,3.25485);
\draw [color=c, fill=c] (15.3333,3.149) rectangle (15.3731,3.25485);
\draw [color=c, fill=c] (15.3731,3.149) rectangle (15.4129,3.25485);
\draw [color=c, fill=c] (15.4129,3.149) rectangle (15.4527,3.25485);
\draw [color=c, fill=c] (15.4527,3.149) rectangle (15.4925,3.25485);
\draw [color=c, fill=c] (15.4925,3.149) rectangle (15.5323,3.25485);
\draw [color=c, fill=c] (15.5323,3.149) rectangle (15.5721,3.25485);
\draw [color=c, fill=c] (15.5721,3.149) rectangle (15.6119,3.25485);
\draw [color=c, fill=c] (15.6119,3.149) rectangle (15.6517,3.25485);
\draw [color=c, fill=c] (15.6517,3.149) rectangle (15.6915,3.25485);
\draw [color=c, fill=c] (15.6915,3.149) rectangle (15.7313,3.25485);
\draw [color=c, fill=c] (15.7313,3.149) rectangle (15.7711,3.25485);
\draw [color=c, fill=c] (15.7711,3.149) rectangle (15.8109,3.25485);
\draw [color=c, fill=c] (15.8109,3.149) rectangle (15.8507,3.25485);
\draw [color=c, fill=c] (15.8507,3.149) rectangle (15.8905,3.25485);
\draw [color=c, fill=c] (15.8905,3.149) rectangle (15.9303,3.25485);
\draw [color=c, fill=c] (15.9303,3.149) rectangle (15.9701,3.25485);
\draw [color=c, fill=c] (15.9701,3.149) rectangle (16.01,3.25485);
\draw [color=c, fill=c] (16.01,3.149) rectangle (16.0498,3.25485);
\draw [color=c, fill=c] (16.0498,3.149) rectangle (16.0896,3.25485);
\draw [color=c, fill=c] (16.0896,3.149) rectangle (16.1294,3.25485);
\draw [color=c, fill=c] (16.1294,3.149) rectangle (16.1692,3.25485);
\draw [color=c, fill=c] (16.1692,3.149) rectangle (16.209,3.25485);
\draw [color=c, fill=c] (16.209,3.149) rectangle (16.2488,3.25485);
\draw [color=c, fill=c] (16.2488,3.149) rectangle (16.2886,3.25485);
\draw [color=c, fill=c] (16.2886,3.149) rectangle (16.3284,3.25485);
\draw [color=c, fill=c] (16.3284,3.149) rectangle (16.3682,3.25485);
\draw [color=c, fill=c] (16.3682,3.149) rectangle (16.408,3.25485);
\draw [color=c, fill=c] (16.408,3.149) rectangle (16.4478,3.25485);
\draw [color=c, fill=c] (16.4478,3.149) rectangle (16.4876,3.25485);
\draw [color=c, fill=c] (16.4876,3.149) rectangle (16.5274,3.25485);
\draw [color=c, fill=c] (16.5274,3.149) rectangle (16.5672,3.25485);
\draw [color=c, fill=c] (16.5672,3.149) rectangle (16.607,3.25485);
\draw [color=c, fill=c] (16.607,3.149) rectangle (16.6468,3.25485);
\draw [color=c, fill=c] (16.6468,3.149) rectangle (16.6866,3.25485);
\draw [color=c, fill=c] (16.6866,3.149) rectangle (16.7264,3.25485);
\draw [color=c, fill=c] (16.7264,3.149) rectangle (16.7662,3.25485);
\draw [color=c, fill=c] (16.7662,3.149) rectangle (16.806,3.25485);
\draw [color=c, fill=c] (16.806,3.149) rectangle (16.8458,3.25485);
\draw [color=c, fill=c] (16.8458,3.149) rectangle (16.8856,3.25485);
\draw [color=c, fill=c] (16.8856,3.149) rectangle (16.9254,3.25485);
\draw [color=c, fill=c] (16.9254,3.149) rectangle (16.9652,3.25485);
\draw [color=c, fill=c] (16.9652,3.149) rectangle (17.005,3.25485);
\draw [color=c, fill=c] (17.005,3.149) rectangle (17.0448,3.25485);
\draw [color=c, fill=c] (17.0448,3.149) rectangle (17.0846,3.25485);
\draw [color=c, fill=c] (17.0846,3.149) rectangle (17.1244,3.25485);
\draw [color=c, fill=c] (17.1244,3.149) rectangle (17.1642,3.25485);
\draw [color=c, fill=c] (17.1642,3.149) rectangle (17.204,3.25485);
\draw [color=c, fill=c] (17.204,3.149) rectangle (17.2438,3.25485);
\draw [color=c, fill=c] (17.2438,3.149) rectangle (17.2836,3.25485);
\draw [color=c, fill=c] (17.2836,3.149) rectangle (17.3234,3.25485);
\draw [color=c, fill=c] (17.3234,3.149) rectangle (17.3632,3.25485);
\draw [color=c, fill=c] (17.3632,3.149) rectangle (17.403,3.25485);
\draw [color=c, fill=c] (17.403,3.149) rectangle (17.4428,3.25485);
\draw [color=c, fill=c] (17.4428,3.149) rectangle (17.4826,3.25485);
\draw [color=c, fill=c] (17.4826,3.149) rectangle (17.5224,3.25485);
\draw [color=c, fill=c] (17.5224,3.149) rectangle (17.5622,3.25485);
\draw [color=c, fill=c] (17.5622,3.149) rectangle (17.602,3.25485);
\draw [color=c, fill=c] (17.602,3.149) rectangle (17.6418,3.25485);
\draw [color=c, fill=c] (17.6418,3.149) rectangle (17.6816,3.25485);
\draw [color=c, fill=c] (17.6816,3.149) rectangle (17.7214,3.25485);
\draw [color=c, fill=c] (17.7214,3.149) rectangle (17.7612,3.25485);
\draw [color=c, fill=c] (17.7612,3.149) rectangle (17.801,3.25485);
\draw [color=c, fill=c] (17.801,3.149) rectangle (17.8408,3.25485);
\draw [color=c, fill=c] (17.8408,3.149) rectangle (17.8806,3.25485);
\draw [color=c, fill=c] (17.8806,3.149) rectangle (17.9204,3.25485);
\draw [color=c, fill=c] (17.9204,3.149) rectangle (17.9602,3.25485);
\draw [color=c, fill=c] (17.9602,3.149) rectangle (18,3.25485);
\definecolor{c}{rgb}{1,0,0};
\draw [color=c, fill=c] (2,3.25485) rectangle (2.0398,3.36069);
\draw [color=c, fill=c] (2.0398,3.25485) rectangle (2.0796,3.36069);
\draw [color=c, fill=c] (2.0796,3.25485) rectangle (2.1194,3.36069);
\draw [color=c, fill=c] (2.1194,3.25485) rectangle (2.1592,3.36069);
\draw [color=c, fill=c] (2.1592,3.25485) rectangle (2.19901,3.36069);
\draw [color=c, fill=c] (2.19901,3.25485) rectangle (2.23881,3.36069);
\draw [color=c, fill=c] (2.23881,3.25485) rectangle (2.27861,3.36069);
\draw [color=c, fill=c] (2.27861,3.25485) rectangle (2.31841,3.36069);
\draw [color=c, fill=c] (2.31841,3.25485) rectangle (2.35821,3.36069);
\draw [color=c, fill=c] (2.35821,3.25485) rectangle (2.39801,3.36069);
\draw [color=c, fill=c] (2.39801,3.25485) rectangle (2.43781,3.36069);
\draw [color=c, fill=c] (2.43781,3.25485) rectangle (2.47761,3.36069);
\draw [color=c, fill=c] (2.47761,3.25485) rectangle (2.51741,3.36069);
\draw [color=c, fill=c] (2.51741,3.25485) rectangle (2.55721,3.36069);
\draw [color=c, fill=c] (2.55721,3.25485) rectangle (2.59702,3.36069);
\draw [color=c, fill=c] (2.59702,3.25485) rectangle (2.63682,3.36069);
\draw [color=c, fill=c] (2.63682,3.25485) rectangle (2.67662,3.36069);
\draw [color=c, fill=c] (2.67662,3.25485) rectangle (2.71642,3.36069);
\draw [color=c, fill=c] (2.71642,3.25485) rectangle (2.75622,3.36069);
\draw [color=c, fill=c] (2.75622,3.25485) rectangle (2.79602,3.36069);
\draw [color=c, fill=c] (2.79602,3.25485) rectangle (2.83582,3.36069);
\draw [color=c, fill=c] (2.83582,3.25485) rectangle (2.87562,3.36069);
\draw [color=c, fill=c] (2.87562,3.25485) rectangle (2.91542,3.36069);
\draw [color=c, fill=c] (2.91542,3.25485) rectangle (2.95522,3.36069);
\draw [color=c, fill=c] (2.95522,3.25485) rectangle (2.99502,3.36069);
\draw [color=c, fill=c] (2.99502,3.25485) rectangle (3.03483,3.36069);
\draw [color=c, fill=c] (3.03483,3.25485) rectangle (3.07463,3.36069);
\draw [color=c, fill=c] (3.07463,3.25485) rectangle (3.11443,3.36069);
\draw [color=c, fill=c] (3.11443,3.25485) rectangle (3.15423,3.36069);
\draw [color=c, fill=c] (3.15423,3.25485) rectangle (3.19403,3.36069);
\draw [color=c, fill=c] (3.19403,3.25485) rectangle (3.23383,3.36069);
\draw [color=c, fill=c] (3.23383,3.25485) rectangle (3.27363,3.36069);
\draw [color=c, fill=c] (3.27363,3.25485) rectangle (3.31343,3.36069);
\draw [color=c, fill=c] (3.31343,3.25485) rectangle (3.35323,3.36069);
\draw [color=c, fill=c] (3.35323,3.25485) rectangle (3.39303,3.36069);
\draw [color=c, fill=c] (3.39303,3.25485) rectangle (3.43284,3.36069);
\draw [color=c, fill=c] (3.43284,3.25485) rectangle (3.47264,3.36069);
\draw [color=c, fill=c] (3.47264,3.25485) rectangle (3.51244,3.36069);
\draw [color=c, fill=c] (3.51244,3.25485) rectangle (3.55224,3.36069);
\draw [color=c, fill=c] (3.55224,3.25485) rectangle (3.59204,3.36069);
\draw [color=c, fill=c] (3.59204,3.25485) rectangle (3.63184,3.36069);
\draw [color=c, fill=c] (3.63184,3.25485) rectangle (3.67164,3.36069);
\draw [color=c, fill=c] (3.67164,3.25485) rectangle (3.71144,3.36069);
\draw [color=c, fill=c] (3.71144,3.25485) rectangle (3.75124,3.36069);
\draw [color=c, fill=c] (3.75124,3.25485) rectangle (3.79104,3.36069);
\draw [color=c, fill=c] (3.79104,3.25485) rectangle (3.83085,3.36069);
\draw [color=c, fill=c] (3.83085,3.25485) rectangle (3.87065,3.36069);
\draw [color=c, fill=c] (3.87065,3.25485) rectangle (3.91045,3.36069);
\draw [color=c, fill=c] (3.91045,3.25485) rectangle (3.95025,3.36069);
\draw [color=c, fill=c] (3.95025,3.25485) rectangle (3.99005,3.36069);
\draw [color=c, fill=c] (3.99005,3.25485) rectangle (4.02985,3.36069);
\draw [color=c, fill=c] (4.02985,3.25485) rectangle (4.06965,3.36069);
\draw [color=c, fill=c] (4.06965,3.25485) rectangle (4.10945,3.36069);
\draw [color=c, fill=c] (4.10945,3.25485) rectangle (4.14925,3.36069);
\draw [color=c, fill=c] (4.14925,3.25485) rectangle (4.18905,3.36069);
\draw [color=c, fill=c] (4.18905,3.25485) rectangle (4.22886,3.36069);
\draw [color=c, fill=c] (4.22886,3.25485) rectangle (4.26866,3.36069);
\draw [color=c, fill=c] (4.26866,3.25485) rectangle (4.30846,3.36069);
\draw [color=c, fill=c] (4.30846,3.25485) rectangle (4.34826,3.36069);
\draw [color=c, fill=c] (4.34826,3.25485) rectangle (4.38806,3.36069);
\draw [color=c, fill=c] (4.38806,3.25485) rectangle (4.42786,3.36069);
\draw [color=c, fill=c] (4.42786,3.25485) rectangle (4.46766,3.36069);
\draw [color=c, fill=c] (4.46766,3.25485) rectangle (4.50746,3.36069);
\draw [color=c, fill=c] (4.50746,3.25485) rectangle (4.54726,3.36069);
\draw [color=c, fill=c] (4.54726,3.25485) rectangle (4.58706,3.36069);
\draw [color=c, fill=c] (4.58706,3.25485) rectangle (4.62687,3.36069);
\draw [color=c, fill=c] (4.62687,3.25485) rectangle (4.66667,3.36069);
\draw [color=c, fill=c] (4.66667,3.25485) rectangle (4.70647,3.36069);
\draw [color=c, fill=c] (4.70647,3.25485) rectangle (4.74627,3.36069);
\draw [color=c, fill=c] (4.74627,3.25485) rectangle (4.78607,3.36069);
\draw [color=c, fill=c] (4.78607,3.25485) rectangle (4.82587,3.36069);
\draw [color=c, fill=c] (4.82587,3.25485) rectangle (4.86567,3.36069);
\draw [color=c, fill=c] (4.86567,3.25485) rectangle (4.90547,3.36069);
\draw [color=c, fill=c] (4.90547,3.25485) rectangle (4.94527,3.36069);
\draw [color=c, fill=c] (4.94527,3.25485) rectangle (4.98507,3.36069);
\draw [color=c, fill=c] (4.98507,3.25485) rectangle (5.02488,3.36069);
\draw [color=c, fill=c] (5.02488,3.25485) rectangle (5.06468,3.36069);
\draw [color=c, fill=c] (5.06468,3.25485) rectangle (5.10448,3.36069);
\draw [color=c, fill=c] (5.10448,3.25485) rectangle (5.14428,3.36069);
\draw [color=c, fill=c] (5.14428,3.25485) rectangle (5.18408,3.36069);
\draw [color=c, fill=c] (5.18408,3.25485) rectangle (5.22388,3.36069);
\draw [color=c, fill=c] (5.22388,3.25485) rectangle (5.26368,3.36069);
\draw [color=c, fill=c] (5.26368,3.25485) rectangle (5.30348,3.36069);
\draw [color=c, fill=c] (5.30348,3.25485) rectangle (5.34328,3.36069);
\draw [color=c, fill=c] (5.34328,3.25485) rectangle (5.38308,3.36069);
\draw [color=c, fill=c] (5.38308,3.25485) rectangle (5.42289,3.36069);
\draw [color=c, fill=c] (5.42289,3.25485) rectangle (5.46269,3.36069);
\draw [color=c, fill=c] (5.46269,3.25485) rectangle (5.50249,3.36069);
\draw [color=c, fill=c] (5.50249,3.25485) rectangle (5.54229,3.36069);
\draw [color=c, fill=c] (5.54229,3.25485) rectangle (5.58209,3.36069);
\draw [color=c, fill=c] (5.58209,3.25485) rectangle (5.62189,3.36069);
\draw [color=c, fill=c] (5.62189,3.25485) rectangle (5.66169,3.36069);
\draw [color=c, fill=c] (5.66169,3.25485) rectangle (5.70149,3.36069);
\draw [color=c, fill=c] (5.70149,3.25485) rectangle (5.74129,3.36069);
\draw [color=c, fill=c] (5.74129,3.25485) rectangle (5.78109,3.36069);
\draw [color=c, fill=c] (5.78109,3.25485) rectangle (5.8209,3.36069);
\draw [color=c, fill=c] (5.8209,3.25485) rectangle (5.8607,3.36069);
\draw [color=c, fill=c] (5.8607,3.25485) rectangle (5.9005,3.36069);
\draw [color=c, fill=c] (5.9005,3.25485) rectangle (5.9403,3.36069);
\draw [color=c, fill=c] (5.9403,3.25485) rectangle (5.9801,3.36069);
\draw [color=c, fill=c] (5.9801,3.25485) rectangle (6.0199,3.36069);
\draw [color=c, fill=c] (6.0199,3.25485) rectangle (6.0597,3.36069);
\draw [color=c, fill=c] (6.0597,3.25485) rectangle (6.0995,3.36069);
\draw [color=c, fill=c] (6.0995,3.25485) rectangle (6.1393,3.36069);
\draw [color=c, fill=c] (6.1393,3.25485) rectangle (6.1791,3.36069);
\draw [color=c, fill=c] (6.1791,3.25485) rectangle (6.21891,3.36069);
\draw [color=c, fill=c] (6.21891,3.25485) rectangle (6.25871,3.36069);
\draw [color=c, fill=c] (6.25871,3.25485) rectangle (6.29851,3.36069);
\draw [color=c, fill=c] (6.29851,3.25485) rectangle (6.33831,3.36069);
\draw [color=c, fill=c] (6.33831,3.25485) rectangle (6.37811,3.36069);
\draw [color=c, fill=c] (6.37811,3.25485) rectangle (6.41791,3.36069);
\draw [color=c, fill=c] (6.41791,3.25485) rectangle (6.45771,3.36069);
\draw [color=c, fill=c] (6.45771,3.25485) rectangle (6.49751,3.36069);
\draw [color=c, fill=c] (6.49751,3.25485) rectangle (6.53731,3.36069);
\draw [color=c, fill=c] (6.53731,3.25485) rectangle (6.57711,3.36069);
\draw [color=c, fill=c] (6.57711,3.25485) rectangle (6.61692,3.36069);
\draw [color=c, fill=c] (6.61692,3.25485) rectangle (6.65672,3.36069);
\draw [color=c, fill=c] (6.65672,3.25485) rectangle (6.69652,3.36069);
\draw [color=c, fill=c] (6.69652,3.25485) rectangle (6.73632,3.36069);
\draw [color=c, fill=c] (6.73632,3.25485) rectangle (6.77612,3.36069);
\draw [color=c, fill=c] (6.77612,3.25485) rectangle (6.81592,3.36069);
\draw [color=c, fill=c] (6.81592,3.25485) rectangle (6.85572,3.36069);
\draw [color=c, fill=c] (6.85572,3.25485) rectangle (6.89552,3.36069);
\draw [color=c, fill=c] (6.89552,3.25485) rectangle (6.93532,3.36069);
\draw [color=c, fill=c] (6.93532,3.25485) rectangle (6.97512,3.36069);
\draw [color=c, fill=c] (6.97512,3.25485) rectangle (7.01493,3.36069);
\draw [color=c, fill=c] (7.01493,3.25485) rectangle (7.05473,3.36069);
\draw [color=c, fill=c] (7.05473,3.25485) rectangle (7.09453,3.36069);
\draw [color=c, fill=c] (7.09453,3.25485) rectangle (7.13433,3.36069);
\draw [color=c, fill=c] (7.13433,3.25485) rectangle (7.17413,3.36069);
\draw [color=c, fill=c] (7.17413,3.25485) rectangle (7.21393,3.36069);
\draw [color=c, fill=c] (7.21393,3.25485) rectangle (7.25373,3.36069);
\draw [color=c, fill=c] (7.25373,3.25485) rectangle (7.29353,3.36069);
\draw [color=c, fill=c] (7.29353,3.25485) rectangle (7.33333,3.36069);
\draw [color=c, fill=c] (7.33333,3.25485) rectangle (7.37313,3.36069);
\draw [color=c, fill=c] (7.37313,3.25485) rectangle (7.41294,3.36069);
\draw [color=c, fill=c] (7.41294,3.25485) rectangle (7.45274,3.36069);
\draw [color=c, fill=c] (7.45274,3.25485) rectangle (7.49254,3.36069);
\draw [color=c, fill=c] (7.49254,3.25485) rectangle (7.53234,3.36069);
\draw [color=c, fill=c] (7.53234,3.25485) rectangle (7.57214,3.36069);
\draw [color=c, fill=c] (7.57214,3.25485) rectangle (7.61194,3.36069);
\draw [color=c, fill=c] (7.61194,3.25485) rectangle (7.65174,3.36069);
\draw [color=c, fill=c] (7.65174,3.25485) rectangle (7.69154,3.36069);
\draw [color=c, fill=c] (7.69154,3.25485) rectangle (7.73134,3.36069);
\draw [color=c, fill=c] (7.73134,3.25485) rectangle (7.77114,3.36069);
\draw [color=c, fill=c] (7.77114,3.25485) rectangle (7.81095,3.36069);
\draw [color=c, fill=c] (7.81095,3.25485) rectangle (7.85075,3.36069);
\draw [color=c, fill=c] (7.85075,3.25485) rectangle (7.89055,3.36069);
\draw [color=c, fill=c] (7.89055,3.25485) rectangle (7.93035,3.36069);
\draw [color=c, fill=c] (7.93035,3.25485) rectangle (7.97015,3.36069);
\draw [color=c, fill=c] (7.97015,3.25485) rectangle (8.00995,3.36069);
\definecolor{c}{rgb}{1,0.186667,0};
\draw [color=c, fill=c] (8.00995,3.25485) rectangle (8.04975,3.36069);
\draw [color=c, fill=c] (8.04975,3.25485) rectangle (8.08955,3.36069);
\draw [color=c, fill=c] (8.08955,3.25485) rectangle (8.12935,3.36069);
\draw [color=c, fill=c] (8.12935,3.25485) rectangle (8.16915,3.36069);
\draw [color=c, fill=c] (8.16915,3.25485) rectangle (8.20895,3.36069);
\draw [color=c, fill=c] (8.20895,3.25485) rectangle (8.24876,3.36069);
\draw [color=c, fill=c] (8.24876,3.25485) rectangle (8.28856,3.36069);
\draw [color=c, fill=c] (8.28856,3.25485) rectangle (8.32836,3.36069);
\draw [color=c, fill=c] (8.32836,3.25485) rectangle (8.36816,3.36069);
\draw [color=c, fill=c] (8.36816,3.25485) rectangle (8.40796,3.36069);
\draw [color=c, fill=c] (8.40796,3.25485) rectangle (8.44776,3.36069);
\draw [color=c, fill=c] (8.44776,3.25485) rectangle (8.48756,3.36069);
\draw [color=c, fill=c] (8.48756,3.25485) rectangle (8.52736,3.36069);
\draw [color=c, fill=c] (8.52736,3.25485) rectangle (8.56716,3.36069);
\draw [color=c, fill=c] (8.56716,3.25485) rectangle (8.60697,3.36069);
\draw [color=c, fill=c] (8.60697,3.25485) rectangle (8.64677,3.36069);
\draw [color=c, fill=c] (8.64677,3.25485) rectangle (8.68657,3.36069);
\draw [color=c, fill=c] (8.68657,3.25485) rectangle (8.72637,3.36069);
\draw [color=c, fill=c] (8.72637,3.25485) rectangle (8.76617,3.36069);
\draw [color=c, fill=c] (8.76617,3.25485) rectangle (8.80597,3.36069);
\draw [color=c, fill=c] (8.80597,3.25485) rectangle (8.84577,3.36069);
\draw [color=c, fill=c] (8.84577,3.25485) rectangle (8.88557,3.36069);
\definecolor{c}{rgb}{1,0.466667,0};
\draw [color=c, fill=c] (8.88557,3.25485) rectangle (8.92537,3.36069);
\draw [color=c, fill=c] (8.92537,3.25485) rectangle (8.96517,3.36069);
\draw [color=c, fill=c] (8.96517,3.25485) rectangle (9.00498,3.36069);
\draw [color=c, fill=c] (9.00498,3.25485) rectangle (9.04478,3.36069);
\draw [color=c, fill=c] (9.04478,3.25485) rectangle (9.08458,3.36069);
\draw [color=c, fill=c] (9.08458,3.25485) rectangle (9.12438,3.36069);
\draw [color=c, fill=c] (9.12438,3.25485) rectangle (9.16418,3.36069);
\draw [color=c, fill=c] (9.16418,3.25485) rectangle (9.20398,3.36069);
\draw [color=c, fill=c] (9.20398,3.25485) rectangle (9.24378,3.36069);
\draw [color=c, fill=c] (9.24378,3.25485) rectangle (9.28358,3.36069);
\draw [color=c, fill=c] (9.28358,3.25485) rectangle (9.32338,3.36069);
\draw [color=c, fill=c] (9.32338,3.25485) rectangle (9.36318,3.36069);
\definecolor{c}{rgb}{1,0.653333,0};
\draw [color=c, fill=c] (9.36318,3.25485) rectangle (9.40298,3.36069);
\draw [color=c, fill=c] (9.40298,3.25485) rectangle (9.44279,3.36069);
\draw [color=c, fill=c] (9.44279,3.25485) rectangle (9.48259,3.36069);
\draw [color=c, fill=c] (9.48259,3.25485) rectangle (9.52239,3.36069);
\draw [color=c, fill=c] (9.52239,3.25485) rectangle (9.56219,3.36069);
\draw [color=c, fill=c] (9.56219,3.25485) rectangle (9.60199,3.36069);
\draw [color=c, fill=c] (9.60199,3.25485) rectangle (9.64179,3.36069);
\definecolor{c}{rgb}{1,0.933333,0};
\draw [color=c, fill=c] (9.64179,3.25485) rectangle (9.68159,3.36069);
\draw [color=c, fill=c] (9.68159,3.25485) rectangle (9.72139,3.36069);
\draw [color=c, fill=c] (9.72139,3.25485) rectangle (9.76119,3.36069);
\definecolor{c}{rgb}{0.88,1,0};
\draw [color=c, fill=c] (9.76119,3.25485) rectangle (9.80099,3.36069);
\draw [color=c, fill=c] (9.80099,3.25485) rectangle (9.8408,3.36069);
\draw [color=c, fill=c] (9.8408,3.25485) rectangle (9.8806,3.36069);
\definecolor{c}{rgb}{0.6,1,0};
\draw [color=c, fill=c] (9.8806,3.25485) rectangle (9.9204,3.36069);
\definecolor{c}{rgb}{0.413333,1,0};
\draw [color=c, fill=c] (9.9204,3.25485) rectangle (9.9602,3.36069);
\draw [color=c, fill=c] (9.9602,3.25485) rectangle (10,3.36069);
\definecolor{c}{rgb}{0.133333,1,0};
\draw [color=c, fill=c] (10,3.25485) rectangle (10.0398,3.36069);
\draw [color=c, fill=c] (10.0398,3.25485) rectangle (10.0796,3.36069);
\definecolor{c}{rgb}{0,1,0.0533333};
\draw [color=c, fill=c] (10.0796,3.25485) rectangle (10.1194,3.36069);
\draw [color=c, fill=c] (10.1194,3.25485) rectangle (10.1592,3.36069);
\definecolor{c}{rgb}{0,1,0.333333};
\draw [color=c, fill=c] (10.1592,3.25485) rectangle (10.199,3.36069);
\draw [color=c, fill=c] (10.199,3.25485) rectangle (10.2388,3.36069);
\draw [color=c, fill=c] (10.2388,3.25485) rectangle (10.2786,3.36069);
\draw [color=c, fill=c] (10.2786,3.25485) rectangle (10.3184,3.36069);
\definecolor{c}{rgb}{0,1,0.52};
\draw [color=c, fill=c] (10.3184,3.25485) rectangle (10.3582,3.36069);
\draw [color=c, fill=c] (10.3582,3.25485) rectangle (10.398,3.36069);
\draw [color=c, fill=c] (10.398,3.25485) rectangle (10.4378,3.36069);
\draw [color=c, fill=c] (10.4378,3.25485) rectangle (10.4776,3.36069);
\draw [color=c, fill=c] (10.4776,3.25485) rectangle (10.5174,3.36069);
\draw [color=c, fill=c] (10.5174,3.25485) rectangle (10.5572,3.36069);
\draw [color=c, fill=c] (10.5572,3.25485) rectangle (10.597,3.36069);
\definecolor{c}{rgb}{0,1,0.8};
\draw [color=c, fill=c] (10.597,3.25485) rectangle (10.6368,3.36069);
\draw [color=c, fill=c] (10.6368,3.25485) rectangle (10.6766,3.36069);
\draw [color=c, fill=c] (10.6766,3.25485) rectangle (10.7164,3.36069);
\draw [color=c, fill=c] (10.7164,3.25485) rectangle (10.7562,3.36069);
\draw [color=c, fill=c] (10.7562,3.25485) rectangle (10.796,3.36069);
\draw [color=c, fill=c] (10.796,3.25485) rectangle (10.8358,3.36069);
\draw [color=c, fill=c] (10.8358,3.25485) rectangle (10.8756,3.36069);
\draw [color=c, fill=c] (10.8756,3.25485) rectangle (10.9154,3.36069);
\draw [color=c, fill=c] (10.9154,3.25485) rectangle (10.9552,3.36069);
\draw [color=c, fill=c] (10.9552,3.25485) rectangle (10.995,3.36069);
\draw [color=c, fill=c] (10.995,3.25485) rectangle (11.0348,3.36069);
\draw [color=c, fill=c] (11.0348,3.25485) rectangle (11.0746,3.36069);
\draw [color=c, fill=c] (11.0746,3.25485) rectangle (11.1144,3.36069);
\definecolor{c}{rgb}{0,1,0.986667};
\draw [color=c, fill=c] (11.1144,3.25485) rectangle (11.1542,3.36069);
\draw [color=c, fill=c] (11.1542,3.25485) rectangle (11.194,3.36069);
\draw [color=c, fill=c] (11.194,3.25485) rectangle (11.2338,3.36069);
\draw [color=c, fill=c] (11.2338,3.25485) rectangle (11.2736,3.36069);
\draw [color=c, fill=c] (11.2736,3.25485) rectangle (11.3134,3.36069);
\draw [color=c, fill=c] (11.3134,3.25485) rectangle (11.3532,3.36069);
\draw [color=c, fill=c] (11.3532,3.25485) rectangle (11.393,3.36069);
\draw [color=c, fill=c] (11.393,3.25485) rectangle (11.4328,3.36069);
\draw [color=c, fill=c] (11.4328,3.25485) rectangle (11.4726,3.36069);
\draw [color=c, fill=c] (11.4726,3.25485) rectangle (11.5124,3.36069);
\draw [color=c, fill=c] (11.5124,3.25485) rectangle (11.5522,3.36069);
\draw [color=c, fill=c] (11.5522,3.25485) rectangle (11.592,3.36069);
\draw [color=c, fill=c] (11.592,3.25485) rectangle (11.6318,3.36069);
\draw [color=c, fill=c] (11.6318,3.25485) rectangle (11.6716,3.36069);
\draw [color=c, fill=c] (11.6716,3.25485) rectangle (11.7114,3.36069);
\draw [color=c, fill=c] (11.7114,3.25485) rectangle (11.7512,3.36069);
\draw [color=c, fill=c] (11.7512,3.25485) rectangle (11.791,3.36069);
\draw [color=c, fill=c] (11.791,3.25485) rectangle (11.8308,3.36069);
\draw [color=c, fill=c] (11.8308,3.25485) rectangle (11.8706,3.36069);
\draw [color=c, fill=c] (11.8706,3.25485) rectangle (11.9104,3.36069);
\draw [color=c, fill=c] (11.9104,3.25485) rectangle (11.9502,3.36069);
\draw [color=c, fill=c] (11.9502,3.25485) rectangle (11.99,3.36069);
\draw [color=c, fill=c] (11.99,3.25485) rectangle (12.0299,3.36069);
\draw [color=c, fill=c] (12.0299,3.25485) rectangle (12.0697,3.36069);
\draw [color=c, fill=c] (12.0697,3.25485) rectangle (12.1095,3.36069);
\definecolor{c}{rgb}{0,0.733333,1};
\draw [color=c, fill=c] (12.1095,3.25485) rectangle (12.1493,3.36069);
\draw [color=c, fill=c] (12.1493,3.25485) rectangle (12.1891,3.36069);
\draw [color=c, fill=c] (12.1891,3.25485) rectangle (12.2289,3.36069);
\draw [color=c, fill=c] (12.2289,3.25485) rectangle (12.2687,3.36069);
\draw [color=c, fill=c] (12.2687,3.25485) rectangle (12.3085,3.36069);
\draw [color=c, fill=c] (12.3085,3.25485) rectangle (12.3483,3.36069);
\draw [color=c, fill=c] (12.3483,3.25485) rectangle (12.3881,3.36069);
\draw [color=c, fill=c] (12.3881,3.25485) rectangle (12.4279,3.36069);
\draw [color=c, fill=c] (12.4279,3.25485) rectangle (12.4677,3.36069);
\draw [color=c, fill=c] (12.4677,3.25485) rectangle (12.5075,3.36069);
\draw [color=c, fill=c] (12.5075,3.25485) rectangle (12.5473,3.36069);
\draw [color=c, fill=c] (12.5473,3.25485) rectangle (12.5871,3.36069);
\draw [color=c, fill=c] (12.5871,3.25485) rectangle (12.6269,3.36069);
\draw [color=c, fill=c] (12.6269,3.25485) rectangle (12.6667,3.36069);
\draw [color=c, fill=c] (12.6667,3.25485) rectangle (12.7065,3.36069);
\draw [color=c, fill=c] (12.7065,3.25485) rectangle (12.7463,3.36069);
\draw [color=c, fill=c] (12.7463,3.25485) rectangle (12.7861,3.36069);
\draw [color=c, fill=c] (12.7861,3.25485) rectangle (12.8259,3.36069);
\draw [color=c, fill=c] (12.8259,3.25485) rectangle (12.8657,3.36069);
\draw [color=c, fill=c] (12.8657,3.25485) rectangle (12.9055,3.36069);
\draw [color=c, fill=c] (12.9055,3.25485) rectangle (12.9453,3.36069);
\draw [color=c, fill=c] (12.9453,3.25485) rectangle (12.9851,3.36069);
\draw [color=c, fill=c] (12.9851,3.25485) rectangle (13.0249,3.36069);
\draw [color=c, fill=c] (13.0249,3.25485) rectangle (13.0647,3.36069);
\draw [color=c, fill=c] (13.0647,3.25485) rectangle (13.1045,3.36069);
\draw [color=c, fill=c] (13.1045,3.25485) rectangle (13.1443,3.36069);
\draw [color=c, fill=c] (13.1443,3.25485) rectangle (13.1841,3.36069);
\draw [color=c, fill=c] (13.1841,3.25485) rectangle (13.2239,3.36069);
\draw [color=c, fill=c] (13.2239,3.25485) rectangle (13.2637,3.36069);
\draw [color=c, fill=c] (13.2637,3.25485) rectangle (13.3035,3.36069);
\draw [color=c, fill=c] (13.3035,3.25485) rectangle (13.3433,3.36069);
\draw [color=c, fill=c] (13.3433,3.25485) rectangle (13.3831,3.36069);
\draw [color=c, fill=c] (13.3831,3.25485) rectangle (13.4229,3.36069);
\draw [color=c, fill=c] (13.4229,3.25485) rectangle (13.4627,3.36069);
\draw [color=c, fill=c] (13.4627,3.25485) rectangle (13.5025,3.36069);
\draw [color=c, fill=c] (13.5025,3.25485) rectangle (13.5423,3.36069);
\draw [color=c, fill=c] (13.5423,3.25485) rectangle (13.5821,3.36069);
\draw [color=c, fill=c] (13.5821,3.25485) rectangle (13.6219,3.36069);
\draw [color=c, fill=c] (13.6219,3.25485) rectangle (13.6617,3.36069);
\draw [color=c, fill=c] (13.6617,3.25485) rectangle (13.7015,3.36069);
\draw [color=c, fill=c] (13.7015,3.25485) rectangle (13.7413,3.36069);
\draw [color=c, fill=c] (13.7413,3.25485) rectangle (13.7811,3.36069);
\draw [color=c, fill=c] (13.7811,3.25485) rectangle (13.8209,3.36069);
\draw [color=c, fill=c] (13.8209,3.25485) rectangle (13.8607,3.36069);
\draw [color=c, fill=c] (13.8607,3.25485) rectangle (13.9005,3.36069);
\draw [color=c, fill=c] (13.9005,3.25485) rectangle (13.9403,3.36069);
\draw [color=c, fill=c] (13.9403,3.25485) rectangle (13.9801,3.36069);
\draw [color=c, fill=c] (13.9801,3.25485) rectangle (14.0199,3.36069);
\draw [color=c, fill=c] (14.0199,3.25485) rectangle (14.0597,3.36069);
\draw [color=c, fill=c] (14.0597,3.25485) rectangle (14.0995,3.36069);
\draw [color=c, fill=c] (14.0995,3.25485) rectangle (14.1393,3.36069);
\draw [color=c, fill=c] (14.1393,3.25485) rectangle (14.1791,3.36069);
\draw [color=c, fill=c] (14.1791,3.25485) rectangle (14.2189,3.36069);
\draw [color=c, fill=c] (14.2189,3.25485) rectangle (14.2587,3.36069);
\draw [color=c, fill=c] (14.2587,3.25485) rectangle (14.2985,3.36069);
\draw [color=c, fill=c] (14.2985,3.25485) rectangle (14.3383,3.36069);
\draw [color=c, fill=c] (14.3383,3.25485) rectangle (14.3781,3.36069);
\draw [color=c, fill=c] (14.3781,3.25485) rectangle (14.4179,3.36069);
\draw [color=c, fill=c] (14.4179,3.25485) rectangle (14.4577,3.36069);
\draw [color=c, fill=c] (14.4577,3.25485) rectangle (14.4975,3.36069);
\draw [color=c, fill=c] (14.4975,3.25485) rectangle (14.5373,3.36069);
\draw [color=c, fill=c] (14.5373,3.25485) rectangle (14.5771,3.36069);
\draw [color=c, fill=c] (14.5771,3.25485) rectangle (14.6169,3.36069);
\draw [color=c, fill=c] (14.6169,3.25485) rectangle (14.6567,3.36069);
\draw [color=c, fill=c] (14.6567,3.25485) rectangle (14.6965,3.36069);
\draw [color=c, fill=c] (14.6965,3.25485) rectangle (14.7363,3.36069);
\draw [color=c, fill=c] (14.7363,3.25485) rectangle (14.7761,3.36069);
\draw [color=c, fill=c] (14.7761,3.25485) rectangle (14.8159,3.36069);
\draw [color=c, fill=c] (14.8159,3.25485) rectangle (14.8557,3.36069);
\draw [color=c, fill=c] (14.8557,3.25485) rectangle (14.8955,3.36069);
\draw [color=c, fill=c] (14.8955,3.25485) rectangle (14.9353,3.36069);
\draw [color=c, fill=c] (14.9353,3.25485) rectangle (14.9751,3.36069);
\draw [color=c, fill=c] (14.9751,3.25485) rectangle (15.0149,3.36069);
\draw [color=c, fill=c] (15.0149,3.25485) rectangle (15.0547,3.36069);
\draw [color=c, fill=c] (15.0547,3.25485) rectangle (15.0945,3.36069);
\draw [color=c, fill=c] (15.0945,3.25485) rectangle (15.1343,3.36069);
\draw [color=c, fill=c] (15.1343,3.25485) rectangle (15.1741,3.36069);
\draw [color=c, fill=c] (15.1741,3.25485) rectangle (15.2139,3.36069);
\draw [color=c, fill=c] (15.2139,3.25485) rectangle (15.2537,3.36069);
\draw [color=c, fill=c] (15.2537,3.25485) rectangle (15.2935,3.36069);
\draw [color=c, fill=c] (15.2935,3.25485) rectangle (15.3333,3.36069);
\draw [color=c, fill=c] (15.3333,3.25485) rectangle (15.3731,3.36069);
\draw [color=c, fill=c] (15.3731,3.25485) rectangle (15.4129,3.36069);
\draw [color=c, fill=c] (15.4129,3.25485) rectangle (15.4527,3.36069);
\draw [color=c, fill=c] (15.4527,3.25485) rectangle (15.4925,3.36069);
\draw [color=c, fill=c] (15.4925,3.25485) rectangle (15.5323,3.36069);
\draw [color=c, fill=c] (15.5323,3.25485) rectangle (15.5721,3.36069);
\draw [color=c, fill=c] (15.5721,3.25485) rectangle (15.6119,3.36069);
\draw [color=c, fill=c] (15.6119,3.25485) rectangle (15.6517,3.36069);
\draw [color=c, fill=c] (15.6517,3.25485) rectangle (15.6915,3.36069);
\draw [color=c, fill=c] (15.6915,3.25485) rectangle (15.7313,3.36069);
\draw [color=c, fill=c] (15.7313,3.25485) rectangle (15.7711,3.36069);
\draw [color=c, fill=c] (15.7711,3.25485) rectangle (15.8109,3.36069);
\draw [color=c, fill=c] (15.8109,3.25485) rectangle (15.8507,3.36069);
\draw [color=c, fill=c] (15.8507,3.25485) rectangle (15.8905,3.36069);
\draw [color=c, fill=c] (15.8905,3.25485) rectangle (15.9303,3.36069);
\draw [color=c, fill=c] (15.9303,3.25485) rectangle (15.9701,3.36069);
\draw [color=c, fill=c] (15.9701,3.25485) rectangle (16.01,3.36069);
\draw [color=c, fill=c] (16.01,3.25485) rectangle (16.0498,3.36069);
\draw [color=c, fill=c] (16.0498,3.25485) rectangle (16.0896,3.36069);
\draw [color=c, fill=c] (16.0896,3.25485) rectangle (16.1294,3.36069);
\draw [color=c, fill=c] (16.1294,3.25485) rectangle (16.1692,3.36069);
\draw [color=c, fill=c] (16.1692,3.25485) rectangle (16.209,3.36069);
\draw [color=c, fill=c] (16.209,3.25485) rectangle (16.2488,3.36069);
\draw [color=c, fill=c] (16.2488,3.25485) rectangle (16.2886,3.36069);
\draw [color=c, fill=c] (16.2886,3.25485) rectangle (16.3284,3.36069);
\draw [color=c, fill=c] (16.3284,3.25485) rectangle (16.3682,3.36069);
\draw [color=c, fill=c] (16.3682,3.25485) rectangle (16.408,3.36069);
\draw [color=c, fill=c] (16.408,3.25485) rectangle (16.4478,3.36069);
\draw [color=c, fill=c] (16.4478,3.25485) rectangle (16.4876,3.36069);
\draw [color=c, fill=c] (16.4876,3.25485) rectangle (16.5274,3.36069);
\draw [color=c, fill=c] (16.5274,3.25485) rectangle (16.5672,3.36069);
\draw [color=c, fill=c] (16.5672,3.25485) rectangle (16.607,3.36069);
\draw [color=c, fill=c] (16.607,3.25485) rectangle (16.6468,3.36069);
\draw [color=c, fill=c] (16.6468,3.25485) rectangle (16.6866,3.36069);
\draw [color=c, fill=c] (16.6866,3.25485) rectangle (16.7264,3.36069);
\draw [color=c, fill=c] (16.7264,3.25485) rectangle (16.7662,3.36069);
\draw [color=c, fill=c] (16.7662,3.25485) rectangle (16.806,3.36069);
\draw [color=c, fill=c] (16.806,3.25485) rectangle (16.8458,3.36069);
\draw [color=c, fill=c] (16.8458,3.25485) rectangle (16.8856,3.36069);
\draw [color=c, fill=c] (16.8856,3.25485) rectangle (16.9254,3.36069);
\draw [color=c, fill=c] (16.9254,3.25485) rectangle (16.9652,3.36069);
\draw [color=c, fill=c] (16.9652,3.25485) rectangle (17.005,3.36069);
\draw [color=c, fill=c] (17.005,3.25485) rectangle (17.0448,3.36069);
\draw [color=c, fill=c] (17.0448,3.25485) rectangle (17.0846,3.36069);
\draw [color=c, fill=c] (17.0846,3.25485) rectangle (17.1244,3.36069);
\draw [color=c, fill=c] (17.1244,3.25485) rectangle (17.1642,3.36069);
\draw [color=c, fill=c] (17.1642,3.25485) rectangle (17.204,3.36069);
\draw [color=c, fill=c] (17.204,3.25485) rectangle (17.2438,3.36069);
\draw [color=c, fill=c] (17.2438,3.25485) rectangle (17.2836,3.36069);
\draw [color=c, fill=c] (17.2836,3.25485) rectangle (17.3234,3.36069);
\draw [color=c, fill=c] (17.3234,3.25485) rectangle (17.3632,3.36069);
\draw [color=c, fill=c] (17.3632,3.25485) rectangle (17.403,3.36069);
\draw [color=c, fill=c] (17.403,3.25485) rectangle (17.4428,3.36069);
\draw [color=c, fill=c] (17.4428,3.25485) rectangle (17.4826,3.36069);
\draw [color=c, fill=c] (17.4826,3.25485) rectangle (17.5224,3.36069);
\draw [color=c, fill=c] (17.5224,3.25485) rectangle (17.5622,3.36069);
\draw [color=c, fill=c] (17.5622,3.25485) rectangle (17.602,3.36069);
\draw [color=c, fill=c] (17.602,3.25485) rectangle (17.6418,3.36069);
\draw [color=c, fill=c] (17.6418,3.25485) rectangle (17.6816,3.36069);
\draw [color=c, fill=c] (17.6816,3.25485) rectangle (17.7214,3.36069);
\draw [color=c, fill=c] (17.7214,3.25485) rectangle (17.7612,3.36069);
\draw [color=c, fill=c] (17.7612,3.25485) rectangle (17.801,3.36069);
\draw [color=c, fill=c] (17.801,3.25485) rectangle (17.8408,3.36069);
\draw [color=c, fill=c] (17.8408,3.25485) rectangle (17.8806,3.36069);
\draw [color=c, fill=c] (17.8806,3.25485) rectangle (17.9204,3.36069);
\draw [color=c, fill=c] (17.9204,3.25485) rectangle (17.9602,3.36069);
\draw [color=c, fill=c] (17.9602,3.25485) rectangle (18,3.36069);
\definecolor{c}{rgb}{1,0,0};
\draw [color=c, fill=c] (2,3.36069) rectangle (2.0398,3.46654);
\draw [color=c, fill=c] (2.0398,3.36069) rectangle (2.0796,3.46654);
\draw [color=c, fill=c] (2.0796,3.36069) rectangle (2.1194,3.46654);
\draw [color=c, fill=c] (2.1194,3.36069) rectangle (2.1592,3.46654);
\draw [color=c, fill=c] (2.1592,3.36069) rectangle (2.19901,3.46654);
\draw [color=c, fill=c] (2.19901,3.36069) rectangle (2.23881,3.46654);
\draw [color=c, fill=c] (2.23881,3.36069) rectangle (2.27861,3.46654);
\draw [color=c, fill=c] (2.27861,3.36069) rectangle (2.31841,3.46654);
\draw [color=c, fill=c] (2.31841,3.36069) rectangle (2.35821,3.46654);
\draw [color=c, fill=c] (2.35821,3.36069) rectangle (2.39801,3.46654);
\draw [color=c, fill=c] (2.39801,3.36069) rectangle (2.43781,3.46654);
\draw [color=c, fill=c] (2.43781,3.36069) rectangle (2.47761,3.46654);
\draw [color=c, fill=c] (2.47761,3.36069) rectangle (2.51741,3.46654);
\draw [color=c, fill=c] (2.51741,3.36069) rectangle (2.55721,3.46654);
\draw [color=c, fill=c] (2.55721,3.36069) rectangle (2.59702,3.46654);
\draw [color=c, fill=c] (2.59702,3.36069) rectangle (2.63682,3.46654);
\draw [color=c, fill=c] (2.63682,3.36069) rectangle (2.67662,3.46654);
\draw [color=c, fill=c] (2.67662,3.36069) rectangle (2.71642,3.46654);
\draw [color=c, fill=c] (2.71642,3.36069) rectangle (2.75622,3.46654);
\draw [color=c, fill=c] (2.75622,3.36069) rectangle (2.79602,3.46654);
\draw [color=c, fill=c] (2.79602,3.36069) rectangle (2.83582,3.46654);
\draw [color=c, fill=c] (2.83582,3.36069) rectangle (2.87562,3.46654);
\draw [color=c, fill=c] (2.87562,3.36069) rectangle (2.91542,3.46654);
\draw [color=c, fill=c] (2.91542,3.36069) rectangle (2.95522,3.46654);
\draw [color=c, fill=c] (2.95522,3.36069) rectangle (2.99502,3.46654);
\draw [color=c, fill=c] (2.99502,3.36069) rectangle (3.03483,3.46654);
\draw [color=c, fill=c] (3.03483,3.36069) rectangle (3.07463,3.46654);
\draw [color=c, fill=c] (3.07463,3.36069) rectangle (3.11443,3.46654);
\draw [color=c, fill=c] (3.11443,3.36069) rectangle (3.15423,3.46654);
\draw [color=c, fill=c] (3.15423,3.36069) rectangle (3.19403,3.46654);
\draw [color=c, fill=c] (3.19403,3.36069) rectangle (3.23383,3.46654);
\draw [color=c, fill=c] (3.23383,3.36069) rectangle (3.27363,3.46654);
\draw [color=c, fill=c] (3.27363,3.36069) rectangle (3.31343,3.46654);
\draw [color=c, fill=c] (3.31343,3.36069) rectangle (3.35323,3.46654);
\draw [color=c, fill=c] (3.35323,3.36069) rectangle (3.39303,3.46654);
\draw [color=c, fill=c] (3.39303,3.36069) rectangle (3.43284,3.46654);
\draw [color=c, fill=c] (3.43284,3.36069) rectangle (3.47264,3.46654);
\draw [color=c, fill=c] (3.47264,3.36069) rectangle (3.51244,3.46654);
\draw [color=c, fill=c] (3.51244,3.36069) rectangle (3.55224,3.46654);
\draw [color=c, fill=c] (3.55224,3.36069) rectangle (3.59204,3.46654);
\draw [color=c, fill=c] (3.59204,3.36069) rectangle (3.63184,3.46654);
\draw [color=c, fill=c] (3.63184,3.36069) rectangle (3.67164,3.46654);
\draw [color=c, fill=c] (3.67164,3.36069) rectangle (3.71144,3.46654);
\draw [color=c, fill=c] (3.71144,3.36069) rectangle (3.75124,3.46654);
\draw [color=c, fill=c] (3.75124,3.36069) rectangle (3.79104,3.46654);
\draw [color=c, fill=c] (3.79104,3.36069) rectangle (3.83085,3.46654);
\draw [color=c, fill=c] (3.83085,3.36069) rectangle (3.87065,3.46654);
\draw [color=c, fill=c] (3.87065,3.36069) rectangle (3.91045,3.46654);
\draw [color=c, fill=c] (3.91045,3.36069) rectangle (3.95025,3.46654);
\draw [color=c, fill=c] (3.95025,3.36069) rectangle (3.99005,3.46654);
\draw [color=c, fill=c] (3.99005,3.36069) rectangle (4.02985,3.46654);
\draw [color=c, fill=c] (4.02985,3.36069) rectangle (4.06965,3.46654);
\draw [color=c, fill=c] (4.06965,3.36069) rectangle (4.10945,3.46654);
\draw [color=c, fill=c] (4.10945,3.36069) rectangle (4.14925,3.46654);
\draw [color=c, fill=c] (4.14925,3.36069) rectangle (4.18905,3.46654);
\draw [color=c, fill=c] (4.18905,3.36069) rectangle (4.22886,3.46654);
\draw [color=c, fill=c] (4.22886,3.36069) rectangle (4.26866,3.46654);
\draw [color=c, fill=c] (4.26866,3.36069) rectangle (4.30846,3.46654);
\draw [color=c, fill=c] (4.30846,3.36069) rectangle (4.34826,3.46654);
\draw [color=c, fill=c] (4.34826,3.36069) rectangle (4.38806,3.46654);
\draw [color=c, fill=c] (4.38806,3.36069) rectangle (4.42786,3.46654);
\draw [color=c, fill=c] (4.42786,3.36069) rectangle (4.46766,3.46654);
\draw [color=c, fill=c] (4.46766,3.36069) rectangle (4.50746,3.46654);
\draw [color=c, fill=c] (4.50746,3.36069) rectangle (4.54726,3.46654);
\draw [color=c, fill=c] (4.54726,3.36069) rectangle (4.58706,3.46654);
\draw [color=c, fill=c] (4.58706,3.36069) rectangle (4.62687,3.46654);
\draw [color=c, fill=c] (4.62687,3.36069) rectangle (4.66667,3.46654);
\draw [color=c, fill=c] (4.66667,3.36069) rectangle (4.70647,3.46654);
\draw [color=c, fill=c] (4.70647,3.36069) rectangle (4.74627,3.46654);
\draw [color=c, fill=c] (4.74627,3.36069) rectangle (4.78607,3.46654);
\draw [color=c, fill=c] (4.78607,3.36069) rectangle (4.82587,3.46654);
\draw [color=c, fill=c] (4.82587,3.36069) rectangle (4.86567,3.46654);
\draw [color=c, fill=c] (4.86567,3.36069) rectangle (4.90547,3.46654);
\draw [color=c, fill=c] (4.90547,3.36069) rectangle (4.94527,3.46654);
\draw [color=c, fill=c] (4.94527,3.36069) rectangle (4.98507,3.46654);
\draw [color=c, fill=c] (4.98507,3.36069) rectangle (5.02488,3.46654);
\draw [color=c, fill=c] (5.02488,3.36069) rectangle (5.06468,3.46654);
\draw [color=c, fill=c] (5.06468,3.36069) rectangle (5.10448,3.46654);
\draw [color=c, fill=c] (5.10448,3.36069) rectangle (5.14428,3.46654);
\draw [color=c, fill=c] (5.14428,3.36069) rectangle (5.18408,3.46654);
\draw [color=c, fill=c] (5.18408,3.36069) rectangle (5.22388,3.46654);
\draw [color=c, fill=c] (5.22388,3.36069) rectangle (5.26368,3.46654);
\draw [color=c, fill=c] (5.26368,3.36069) rectangle (5.30348,3.46654);
\draw [color=c, fill=c] (5.30348,3.36069) rectangle (5.34328,3.46654);
\draw [color=c, fill=c] (5.34328,3.36069) rectangle (5.38308,3.46654);
\draw [color=c, fill=c] (5.38308,3.36069) rectangle (5.42289,3.46654);
\draw [color=c, fill=c] (5.42289,3.36069) rectangle (5.46269,3.46654);
\draw [color=c, fill=c] (5.46269,3.36069) rectangle (5.50249,3.46654);
\draw [color=c, fill=c] (5.50249,3.36069) rectangle (5.54229,3.46654);
\draw [color=c, fill=c] (5.54229,3.36069) rectangle (5.58209,3.46654);
\draw [color=c, fill=c] (5.58209,3.36069) rectangle (5.62189,3.46654);
\draw [color=c, fill=c] (5.62189,3.36069) rectangle (5.66169,3.46654);
\draw [color=c, fill=c] (5.66169,3.36069) rectangle (5.70149,3.46654);
\draw [color=c, fill=c] (5.70149,3.36069) rectangle (5.74129,3.46654);
\draw [color=c, fill=c] (5.74129,3.36069) rectangle (5.78109,3.46654);
\draw [color=c, fill=c] (5.78109,3.36069) rectangle (5.8209,3.46654);
\draw [color=c, fill=c] (5.8209,3.36069) rectangle (5.8607,3.46654);
\draw [color=c, fill=c] (5.8607,3.36069) rectangle (5.9005,3.46654);
\draw [color=c, fill=c] (5.9005,3.36069) rectangle (5.9403,3.46654);
\draw [color=c, fill=c] (5.9403,3.36069) rectangle (5.9801,3.46654);
\draw [color=c, fill=c] (5.9801,3.36069) rectangle (6.0199,3.46654);
\draw [color=c, fill=c] (6.0199,3.36069) rectangle (6.0597,3.46654);
\draw [color=c, fill=c] (6.0597,3.36069) rectangle (6.0995,3.46654);
\draw [color=c, fill=c] (6.0995,3.36069) rectangle (6.1393,3.46654);
\draw [color=c, fill=c] (6.1393,3.36069) rectangle (6.1791,3.46654);
\draw [color=c, fill=c] (6.1791,3.36069) rectangle (6.21891,3.46654);
\draw [color=c, fill=c] (6.21891,3.36069) rectangle (6.25871,3.46654);
\draw [color=c, fill=c] (6.25871,3.36069) rectangle (6.29851,3.46654);
\draw [color=c, fill=c] (6.29851,3.36069) rectangle (6.33831,3.46654);
\draw [color=c, fill=c] (6.33831,3.36069) rectangle (6.37811,3.46654);
\draw [color=c, fill=c] (6.37811,3.36069) rectangle (6.41791,3.46654);
\draw [color=c, fill=c] (6.41791,3.36069) rectangle (6.45771,3.46654);
\draw [color=c, fill=c] (6.45771,3.36069) rectangle (6.49751,3.46654);
\draw [color=c, fill=c] (6.49751,3.36069) rectangle (6.53731,3.46654);
\draw [color=c, fill=c] (6.53731,3.36069) rectangle (6.57711,3.46654);
\draw [color=c, fill=c] (6.57711,3.36069) rectangle (6.61692,3.46654);
\draw [color=c, fill=c] (6.61692,3.36069) rectangle (6.65672,3.46654);
\draw [color=c, fill=c] (6.65672,3.36069) rectangle (6.69652,3.46654);
\draw [color=c, fill=c] (6.69652,3.36069) rectangle (6.73632,3.46654);
\draw [color=c, fill=c] (6.73632,3.36069) rectangle (6.77612,3.46654);
\draw [color=c, fill=c] (6.77612,3.36069) rectangle (6.81592,3.46654);
\draw [color=c, fill=c] (6.81592,3.36069) rectangle (6.85572,3.46654);
\draw [color=c, fill=c] (6.85572,3.36069) rectangle (6.89552,3.46654);
\draw [color=c, fill=c] (6.89552,3.36069) rectangle (6.93532,3.46654);
\draw [color=c, fill=c] (6.93532,3.36069) rectangle (6.97512,3.46654);
\draw [color=c, fill=c] (6.97512,3.36069) rectangle (7.01493,3.46654);
\draw [color=c, fill=c] (7.01493,3.36069) rectangle (7.05473,3.46654);
\draw [color=c, fill=c] (7.05473,3.36069) rectangle (7.09453,3.46654);
\draw [color=c, fill=c] (7.09453,3.36069) rectangle (7.13433,3.46654);
\draw [color=c, fill=c] (7.13433,3.36069) rectangle (7.17413,3.46654);
\draw [color=c, fill=c] (7.17413,3.36069) rectangle (7.21393,3.46654);
\draw [color=c, fill=c] (7.21393,3.36069) rectangle (7.25373,3.46654);
\draw [color=c, fill=c] (7.25373,3.36069) rectangle (7.29353,3.46654);
\draw [color=c, fill=c] (7.29353,3.36069) rectangle (7.33333,3.46654);
\draw [color=c, fill=c] (7.33333,3.36069) rectangle (7.37313,3.46654);
\draw [color=c, fill=c] (7.37313,3.36069) rectangle (7.41294,3.46654);
\draw [color=c, fill=c] (7.41294,3.36069) rectangle (7.45274,3.46654);
\draw [color=c, fill=c] (7.45274,3.36069) rectangle (7.49254,3.46654);
\draw [color=c, fill=c] (7.49254,3.36069) rectangle (7.53234,3.46654);
\draw [color=c, fill=c] (7.53234,3.36069) rectangle (7.57214,3.46654);
\draw [color=c, fill=c] (7.57214,3.36069) rectangle (7.61194,3.46654);
\draw [color=c, fill=c] (7.61194,3.36069) rectangle (7.65174,3.46654);
\draw [color=c, fill=c] (7.65174,3.36069) rectangle (7.69154,3.46654);
\draw [color=c, fill=c] (7.69154,3.36069) rectangle (7.73134,3.46654);
\draw [color=c, fill=c] (7.73134,3.36069) rectangle (7.77114,3.46654);
\draw [color=c, fill=c] (7.77114,3.36069) rectangle (7.81095,3.46654);
\draw [color=c, fill=c] (7.81095,3.36069) rectangle (7.85075,3.46654);
\draw [color=c, fill=c] (7.85075,3.36069) rectangle (7.89055,3.46654);
\draw [color=c, fill=c] (7.89055,3.36069) rectangle (7.93035,3.46654);
\draw [color=c, fill=c] (7.93035,3.36069) rectangle (7.97015,3.46654);
\draw [color=c, fill=c] (7.97015,3.36069) rectangle (8.00995,3.46654);
\definecolor{c}{rgb}{1,0.186667,0};
\draw [color=c, fill=c] (8.00995,3.36069) rectangle (8.04975,3.46654);
\draw [color=c, fill=c] (8.04975,3.36069) rectangle (8.08955,3.46654);
\draw [color=c, fill=c] (8.08955,3.36069) rectangle (8.12935,3.46654);
\draw [color=c, fill=c] (8.12935,3.36069) rectangle (8.16915,3.46654);
\draw [color=c, fill=c] (8.16915,3.36069) rectangle (8.20895,3.46654);
\draw [color=c, fill=c] (8.20895,3.36069) rectangle (8.24876,3.46654);
\draw [color=c, fill=c] (8.24876,3.36069) rectangle (8.28856,3.46654);
\draw [color=c, fill=c] (8.28856,3.36069) rectangle (8.32836,3.46654);
\draw [color=c, fill=c] (8.32836,3.36069) rectangle (8.36816,3.46654);
\draw [color=c, fill=c] (8.36816,3.36069) rectangle (8.40796,3.46654);
\draw [color=c, fill=c] (8.40796,3.36069) rectangle (8.44776,3.46654);
\draw [color=c, fill=c] (8.44776,3.36069) rectangle (8.48756,3.46654);
\draw [color=c, fill=c] (8.48756,3.36069) rectangle (8.52736,3.46654);
\draw [color=c, fill=c] (8.52736,3.36069) rectangle (8.56716,3.46654);
\draw [color=c, fill=c] (8.56716,3.36069) rectangle (8.60697,3.46654);
\draw [color=c, fill=c] (8.60697,3.36069) rectangle (8.64677,3.46654);
\draw [color=c, fill=c] (8.64677,3.36069) rectangle (8.68657,3.46654);
\draw [color=c, fill=c] (8.68657,3.36069) rectangle (8.72637,3.46654);
\draw [color=c, fill=c] (8.72637,3.36069) rectangle (8.76617,3.46654);
\draw [color=c, fill=c] (8.76617,3.36069) rectangle (8.80597,3.46654);
\draw [color=c, fill=c] (8.80597,3.36069) rectangle (8.84577,3.46654);
\draw [color=c, fill=c] (8.84577,3.36069) rectangle (8.88557,3.46654);
\draw [color=c, fill=c] (8.88557,3.36069) rectangle (8.92537,3.46654);
\draw [color=c, fill=c] (8.92537,3.36069) rectangle (8.96517,3.46654);
\definecolor{c}{rgb}{1,0.466667,0};
\draw [color=c, fill=c] (8.96517,3.36069) rectangle (9.00498,3.46654);
\draw [color=c, fill=c] (9.00498,3.36069) rectangle (9.04478,3.46654);
\draw [color=c, fill=c] (9.04478,3.36069) rectangle (9.08458,3.46654);
\draw [color=c, fill=c] (9.08458,3.36069) rectangle (9.12438,3.46654);
\draw [color=c, fill=c] (9.12438,3.36069) rectangle (9.16418,3.46654);
\draw [color=c, fill=c] (9.16418,3.36069) rectangle (9.20398,3.46654);
\draw [color=c, fill=c] (9.20398,3.36069) rectangle (9.24378,3.46654);
\draw [color=c, fill=c] (9.24378,3.36069) rectangle (9.28358,3.46654);
\draw [color=c, fill=c] (9.28358,3.36069) rectangle (9.32338,3.46654);
\draw [color=c, fill=c] (9.32338,3.36069) rectangle (9.36318,3.46654);
\draw [color=c, fill=c] (9.36318,3.36069) rectangle (9.40298,3.46654);
\draw [color=c, fill=c] (9.40298,3.36069) rectangle (9.44279,3.46654);
\definecolor{c}{rgb}{1,0.653333,0};
\draw [color=c, fill=c] (9.44279,3.36069) rectangle (9.48259,3.46654);
\draw [color=c, fill=c] (9.48259,3.36069) rectangle (9.52239,3.46654);
\draw [color=c, fill=c] (9.52239,3.36069) rectangle (9.56219,3.46654);
\draw [color=c, fill=c] (9.56219,3.36069) rectangle (9.60199,3.46654);
\draw [color=c, fill=c] (9.60199,3.36069) rectangle (9.64179,3.46654);
\draw [color=c, fill=c] (9.64179,3.36069) rectangle (9.68159,3.46654);
\definecolor{c}{rgb}{1,0.933333,0};
\draw [color=c, fill=c] (9.68159,3.36069) rectangle (9.72139,3.46654);
\draw [color=c, fill=c] (9.72139,3.36069) rectangle (9.76119,3.46654);
\draw [color=c, fill=c] (9.76119,3.36069) rectangle (9.80099,3.46654);
\definecolor{c}{rgb}{0.88,1,0};
\draw [color=c, fill=c] (9.80099,3.36069) rectangle (9.8408,3.46654);
\draw [color=c, fill=c] (9.8408,3.36069) rectangle (9.8806,3.46654);
\definecolor{c}{rgb}{0.6,1,0};
\draw [color=c, fill=c] (9.8806,3.36069) rectangle (9.9204,3.46654);
\definecolor{c}{rgb}{0.413333,1,0};
\draw [color=c, fill=c] (9.9204,3.36069) rectangle (9.9602,3.46654);
\draw [color=c, fill=c] (9.9602,3.36069) rectangle (10,3.46654);
\definecolor{c}{rgb}{0.133333,1,0};
\draw [color=c, fill=c] (10,3.36069) rectangle (10.0398,3.46654);
\definecolor{c}{rgb}{0,1,0.0533333};
\draw [color=c, fill=c] (10.0398,3.36069) rectangle (10.0796,3.46654);
\draw [color=c, fill=c] (10.0796,3.36069) rectangle (10.1194,3.46654);
\definecolor{c}{rgb}{0,1,0.333333};
\draw [color=c, fill=c] (10.1194,3.36069) rectangle (10.1592,3.46654);
\draw [color=c, fill=c] (10.1592,3.36069) rectangle (10.199,3.46654);
\draw [color=c, fill=c] (10.199,3.36069) rectangle (10.2388,3.46654);
\draw [color=c, fill=c] (10.2388,3.36069) rectangle (10.2786,3.46654);
\definecolor{c}{rgb}{0,1,0.52};
\draw [color=c, fill=c] (10.2786,3.36069) rectangle (10.3184,3.46654);
\draw [color=c, fill=c] (10.3184,3.36069) rectangle (10.3582,3.46654);
\draw [color=c, fill=c] (10.3582,3.36069) rectangle (10.398,3.46654);
\draw [color=c, fill=c] (10.398,3.36069) rectangle (10.4378,3.46654);
\draw [color=c, fill=c] (10.4378,3.36069) rectangle (10.4776,3.46654);
\draw [color=c, fill=c] (10.4776,3.36069) rectangle (10.5174,3.46654);
\definecolor{c}{rgb}{0,1,0.8};
\draw [color=c, fill=c] (10.5174,3.36069) rectangle (10.5572,3.46654);
\draw [color=c, fill=c] (10.5572,3.36069) rectangle (10.597,3.46654);
\draw [color=c, fill=c] (10.597,3.36069) rectangle (10.6368,3.46654);
\draw [color=c, fill=c] (10.6368,3.36069) rectangle (10.6766,3.46654);
\draw [color=c, fill=c] (10.6766,3.36069) rectangle (10.7164,3.46654);
\draw [color=c, fill=c] (10.7164,3.36069) rectangle (10.7562,3.46654);
\draw [color=c, fill=c] (10.7562,3.36069) rectangle (10.796,3.46654);
\draw [color=c, fill=c] (10.796,3.36069) rectangle (10.8358,3.46654);
\draw [color=c, fill=c] (10.8358,3.36069) rectangle (10.8756,3.46654);
\draw [color=c, fill=c] (10.8756,3.36069) rectangle (10.9154,3.46654);
\draw [color=c, fill=c] (10.9154,3.36069) rectangle (10.9552,3.46654);
\draw [color=c, fill=c] (10.9552,3.36069) rectangle (10.995,3.46654);
\draw [color=c, fill=c] (10.995,3.36069) rectangle (11.0348,3.46654);
\definecolor{c}{rgb}{0,1,0.986667};
\draw [color=c, fill=c] (11.0348,3.36069) rectangle (11.0746,3.46654);
\draw [color=c, fill=c] (11.0746,3.36069) rectangle (11.1144,3.46654);
\draw [color=c, fill=c] (11.1144,3.36069) rectangle (11.1542,3.46654);
\draw [color=c, fill=c] (11.1542,3.36069) rectangle (11.194,3.46654);
\draw [color=c, fill=c] (11.194,3.36069) rectangle (11.2338,3.46654);
\draw [color=c, fill=c] (11.2338,3.36069) rectangle (11.2736,3.46654);
\draw [color=c, fill=c] (11.2736,3.36069) rectangle (11.3134,3.46654);
\draw [color=c, fill=c] (11.3134,3.36069) rectangle (11.3532,3.46654);
\draw [color=c, fill=c] (11.3532,3.36069) rectangle (11.393,3.46654);
\draw [color=c, fill=c] (11.393,3.36069) rectangle (11.4328,3.46654);
\draw [color=c, fill=c] (11.4328,3.36069) rectangle (11.4726,3.46654);
\draw [color=c, fill=c] (11.4726,3.36069) rectangle (11.5124,3.46654);
\draw [color=c, fill=c] (11.5124,3.36069) rectangle (11.5522,3.46654);
\draw [color=c, fill=c] (11.5522,3.36069) rectangle (11.592,3.46654);
\draw [color=c, fill=c] (11.592,3.36069) rectangle (11.6318,3.46654);
\draw [color=c, fill=c] (11.6318,3.36069) rectangle (11.6716,3.46654);
\draw [color=c, fill=c] (11.6716,3.36069) rectangle (11.7114,3.46654);
\draw [color=c, fill=c] (11.7114,3.36069) rectangle (11.7512,3.46654);
\draw [color=c, fill=c] (11.7512,3.36069) rectangle (11.791,3.46654);
\draw [color=c, fill=c] (11.791,3.36069) rectangle (11.8308,3.46654);
\draw [color=c, fill=c] (11.8308,3.36069) rectangle (11.8706,3.46654);
\draw [color=c, fill=c] (11.8706,3.36069) rectangle (11.9104,3.46654);
\draw [color=c, fill=c] (11.9104,3.36069) rectangle (11.9502,3.46654);
\draw [color=c, fill=c] (11.9502,3.36069) rectangle (11.99,3.46654);
\draw [color=c, fill=c] (11.99,3.36069) rectangle (12.0299,3.46654);
\draw [color=c, fill=c] (12.0299,3.36069) rectangle (12.0697,3.46654);
\definecolor{c}{rgb}{0,0.733333,1};
\draw [color=c, fill=c] (12.0697,3.36069) rectangle (12.1095,3.46654);
\draw [color=c, fill=c] (12.1095,3.36069) rectangle (12.1493,3.46654);
\draw [color=c, fill=c] (12.1493,3.36069) rectangle (12.1891,3.46654);
\draw [color=c, fill=c] (12.1891,3.36069) rectangle (12.2289,3.46654);
\draw [color=c, fill=c] (12.2289,3.36069) rectangle (12.2687,3.46654);
\draw [color=c, fill=c] (12.2687,3.36069) rectangle (12.3085,3.46654);
\draw [color=c, fill=c] (12.3085,3.36069) rectangle (12.3483,3.46654);
\draw [color=c, fill=c] (12.3483,3.36069) rectangle (12.3881,3.46654);
\draw [color=c, fill=c] (12.3881,3.36069) rectangle (12.4279,3.46654);
\draw [color=c, fill=c] (12.4279,3.36069) rectangle (12.4677,3.46654);
\draw [color=c, fill=c] (12.4677,3.36069) rectangle (12.5075,3.46654);
\draw [color=c, fill=c] (12.5075,3.36069) rectangle (12.5473,3.46654);
\draw [color=c, fill=c] (12.5473,3.36069) rectangle (12.5871,3.46654);
\draw [color=c, fill=c] (12.5871,3.36069) rectangle (12.6269,3.46654);
\draw [color=c, fill=c] (12.6269,3.36069) rectangle (12.6667,3.46654);
\draw [color=c, fill=c] (12.6667,3.36069) rectangle (12.7065,3.46654);
\draw [color=c, fill=c] (12.7065,3.36069) rectangle (12.7463,3.46654);
\draw [color=c, fill=c] (12.7463,3.36069) rectangle (12.7861,3.46654);
\draw [color=c, fill=c] (12.7861,3.36069) rectangle (12.8259,3.46654);
\draw [color=c, fill=c] (12.8259,3.36069) rectangle (12.8657,3.46654);
\draw [color=c, fill=c] (12.8657,3.36069) rectangle (12.9055,3.46654);
\draw [color=c, fill=c] (12.9055,3.36069) rectangle (12.9453,3.46654);
\draw [color=c, fill=c] (12.9453,3.36069) rectangle (12.9851,3.46654);
\draw [color=c, fill=c] (12.9851,3.36069) rectangle (13.0249,3.46654);
\draw [color=c, fill=c] (13.0249,3.36069) rectangle (13.0647,3.46654);
\draw [color=c, fill=c] (13.0647,3.36069) rectangle (13.1045,3.46654);
\draw [color=c, fill=c] (13.1045,3.36069) rectangle (13.1443,3.46654);
\draw [color=c, fill=c] (13.1443,3.36069) rectangle (13.1841,3.46654);
\draw [color=c, fill=c] (13.1841,3.36069) rectangle (13.2239,3.46654);
\draw [color=c, fill=c] (13.2239,3.36069) rectangle (13.2637,3.46654);
\draw [color=c, fill=c] (13.2637,3.36069) rectangle (13.3035,3.46654);
\draw [color=c, fill=c] (13.3035,3.36069) rectangle (13.3433,3.46654);
\draw [color=c, fill=c] (13.3433,3.36069) rectangle (13.3831,3.46654);
\draw [color=c, fill=c] (13.3831,3.36069) rectangle (13.4229,3.46654);
\draw [color=c, fill=c] (13.4229,3.36069) rectangle (13.4627,3.46654);
\draw [color=c, fill=c] (13.4627,3.36069) rectangle (13.5025,3.46654);
\draw [color=c, fill=c] (13.5025,3.36069) rectangle (13.5423,3.46654);
\draw [color=c, fill=c] (13.5423,3.36069) rectangle (13.5821,3.46654);
\draw [color=c, fill=c] (13.5821,3.36069) rectangle (13.6219,3.46654);
\draw [color=c, fill=c] (13.6219,3.36069) rectangle (13.6617,3.46654);
\draw [color=c, fill=c] (13.6617,3.36069) rectangle (13.7015,3.46654);
\draw [color=c, fill=c] (13.7015,3.36069) rectangle (13.7413,3.46654);
\draw [color=c, fill=c] (13.7413,3.36069) rectangle (13.7811,3.46654);
\draw [color=c, fill=c] (13.7811,3.36069) rectangle (13.8209,3.46654);
\draw [color=c, fill=c] (13.8209,3.36069) rectangle (13.8607,3.46654);
\draw [color=c, fill=c] (13.8607,3.36069) rectangle (13.9005,3.46654);
\draw [color=c, fill=c] (13.9005,3.36069) rectangle (13.9403,3.46654);
\draw [color=c, fill=c] (13.9403,3.36069) rectangle (13.9801,3.46654);
\draw [color=c, fill=c] (13.9801,3.36069) rectangle (14.0199,3.46654);
\draw [color=c, fill=c] (14.0199,3.36069) rectangle (14.0597,3.46654);
\draw [color=c, fill=c] (14.0597,3.36069) rectangle (14.0995,3.46654);
\draw [color=c, fill=c] (14.0995,3.36069) rectangle (14.1393,3.46654);
\draw [color=c, fill=c] (14.1393,3.36069) rectangle (14.1791,3.46654);
\draw [color=c, fill=c] (14.1791,3.36069) rectangle (14.2189,3.46654);
\draw [color=c, fill=c] (14.2189,3.36069) rectangle (14.2587,3.46654);
\draw [color=c, fill=c] (14.2587,3.36069) rectangle (14.2985,3.46654);
\draw [color=c, fill=c] (14.2985,3.36069) rectangle (14.3383,3.46654);
\draw [color=c, fill=c] (14.3383,3.36069) rectangle (14.3781,3.46654);
\draw [color=c, fill=c] (14.3781,3.36069) rectangle (14.4179,3.46654);
\draw [color=c, fill=c] (14.4179,3.36069) rectangle (14.4577,3.46654);
\draw [color=c, fill=c] (14.4577,3.36069) rectangle (14.4975,3.46654);
\draw [color=c, fill=c] (14.4975,3.36069) rectangle (14.5373,3.46654);
\draw [color=c, fill=c] (14.5373,3.36069) rectangle (14.5771,3.46654);
\draw [color=c, fill=c] (14.5771,3.36069) rectangle (14.6169,3.46654);
\draw [color=c, fill=c] (14.6169,3.36069) rectangle (14.6567,3.46654);
\draw [color=c, fill=c] (14.6567,3.36069) rectangle (14.6965,3.46654);
\draw [color=c, fill=c] (14.6965,3.36069) rectangle (14.7363,3.46654);
\draw [color=c, fill=c] (14.7363,3.36069) rectangle (14.7761,3.46654);
\draw [color=c, fill=c] (14.7761,3.36069) rectangle (14.8159,3.46654);
\draw [color=c, fill=c] (14.8159,3.36069) rectangle (14.8557,3.46654);
\draw [color=c, fill=c] (14.8557,3.36069) rectangle (14.8955,3.46654);
\draw [color=c, fill=c] (14.8955,3.36069) rectangle (14.9353,3.46654);
\draw [color=c, fill=c] (14.9353,3.36069) rectangle (14.9751,3.46654);
\draw [color=c, fill=c] (14.9751,3.36069) rectangle (15.0149,3.46654);
\draw [color=c, fill=c] (15.0149,3.36069) rectangle (15.0547,3.46654);
\draw [color=c, fill=c] (15.0547,3.36069) rectangle (15.0945,3.46654);
\draw [color=c, fill=c] (15.0945,3.36069) rectangle (15.1343,3.46654);
\draw [color=c, fill=c] (15.1343,3.36069) rectangle (15.1741,3.46654);
\draw [color=c, fill=c] (15.1741,3.36069) rectangle (15.2139,3.46654);
\draw [color=c, fill=c] (15.2139,3.36069) rectangle (15.2537,3.46654);
\draw [color=c, fill=c] (15.2537,3.36069) rectangle (15.2935,3.46654);
\draw [color=c, fill=c] (15.2935,3.36069) rectangle (15.3333,3.46654);
\draw [color=c, fill=c] (15.3333,3.36069) rectangle (15.3731,3.46654);
\draw [color=c, fill=c] (15.3731,3.36069) rectangle (15.4129,3.46654);
\draw [color=c, fill=c] (15.4129,3.36069) rectangle (15.4527,3.46654);
\draw [color=c, fill=c] (15.4527,3.36069) rectangle (15.4925,3.46654);
\draw [color=c, fill=c] (15.4925,3.36069) rectangle (15.5323,3.46654);
\draw [color=c, fill=c] (15.5323,3.36069) rectangle (15.5721,3.46654);
\draw [color=c, fill=c] (15.5721,3.36069) rectangle (15.6119,3.46654);
\draw [color=c, fill=c] (15.6119,3.36069) rectangle (15.6517,3.46654);
\draw [color=c, fill=c] (15.6517,3.36069) rectangle (15.6915,3.46654);
\draw [color=c, fill=c] (15.6915,3.36069) rectangle (15.7313,3.46654);
\draw [color=c, fill=c] (15.7313,3.36069) rectangle (15.7711,3.46654);
\draw [color=c, fill=c] (15.7711,3.36069) rectangle (15.8109,3.46654);
\draw [color=c, fill=c] (15.8109,3.36069) rectangle (15.8507,3.46654);
\draw [color=c, fill=c] (15.8507,3.36069) rectangle (15.8905,3.46654);
\draw [color=c, fill=c] (15.8905,3.36069) rectangle (15.9303,3.46654);
\draw [color=c, fill=c] (15.9303,3.36069) rectangle (15.9701,3.46654);
\draw [color=c, fill=c] (15.9701,3.36069) rectangle (16.01,3.46654);
\draw [color=c, fill=c] (16.01,3.36069) rectangle (16.0498,3.46654);
\draw [color=c, fill=c] (16.0498,3.36069) rectangle (16.0896,3.46654);
\draw [color=c, fill=c] (16.0896,3.36069) rectangle (16.1294,3.46654);
\draw [color=c, fill=c] (16.1294,3.36069) rectangle (16.1692,3.46654);
\draw [color=c, fill=c] (16.1692,3.36069) rectangle (16.209,3.46654);
\draw [color=c, fill=c] (16.209,3.36069) rectangle (16.2488,3.46654);
\draw [color=c, fill=c] (16.2488,3.36069) rectangle (16.2886,3.46654);
\draw [color=c, fill=c] (16.2886,3.36069) rectangle (16.3284,3.46654);
\draw [color=c, fill=c] (16.3284,3.36069) rectangle (16.3682,3.46654);
\draw [color=c, fill=c] (16.3682,3.36069) rectangle (16.408,3.46654);
\draw [color=c, fill=c] (16.408,3.36069) rectangle (16.4478,3.46654);
\draw [color=c, fill=c] (16.4478,3.36069) rectangle (16.4876,3.46654);
\draw [color=c, fill=c] (16.4876,3.36069) rectangle (16.5274,3.46654);
\draw [color=c, fill=c] (16.5274,3.36069) rectangle (16.5672,3.46654);
\draw [color=c, fill=c] (16.5672,3.36069) rectangle (16.607,3.46654);
\draw [color=c, fill=c] (16.607,3.36069) rectangle (16.6468,3.46654);
\draw [color=c, fill=c] (16.6468,3.36069) rectangle (16.6866,3.46654);
\draw [color=c, fill=c] (16.6866,3.36069) rectangle (16.7264,3.46654);
\draw [color=c, fill=c] (16.7264,3.36069) rectangle (16.7662,3.46654);
\draw [color=c, fill=c] (16.7662,3.36069) rectangle (16.806,3.46654);
\draw [color=c, fill=c] (16.806,3.36069) rectangle (16.8458,3.46654);
\draw [color=c, fill=c] (16.8458,3.36069) rectangle (16.8856,3.46654);
\draw [color=c, fill=c] (16.8856,3.36069) rectangle (16.9254,3.46654);
\draw [color=c, fill=c] (16.9254,3.36069) rectangle (16.9652,3.46654);
\draw [color=c, fill=c] (16.9652,3.36069) rectangle (17.005,3.46654);
\draw [color=c, fill=c] (17.005,3.36069) rectangle (17.0448,3.46654);
\draw [color=c, fill=c] (17.0448,3.36069) rectangle (17.0846,3.46654);
\draw [color=c, fill=c] (17.0846,3.36069) rectangle (17.1244,3.46654);
\draw [color=c, fill=c] (17.1244,3.36069) rectangle (17.1642,3.46654);
\draw [color=c, fill=c] (17.1642,3.36069) rectangle (17.204,3.46654);
\draw [color=c, fill=c] (17.204,3.36069) rectangle (17.2438,3.46654);
\draw [color=c, fill=c] (17.2438,3.36069) rectangle (17.2836,3.46654);
\draw [color=c, fill=c] (17.2836,3.36069) rectangle (17.3234,3.46654);
\draw [color=c, fill=c] (17.3234,3.36069) rectangle (17.3632,3.46654);
\draw [color=c, fill=c] (17.3632,3.36069) rectangle (17.403,3.46654);
\draw [color=c, fill=c] (17.403,3.36069) rectangle (17.4428,3.46654);
\draw [color=c, fill=c] (17.4428,3.36069) rectangle (17.4826,3.46654);
\draw [color=c, fill=c] (17.4826,3.36069) rectangle (17.5224,3.46654);
\draw [color=c, fill=c] (17.5224,3.36069) rectangle (17.5622,3.46654);
\draw [color=c, fill=c] (17.5622,3.36069) rectangle (17.602,3.46654);
\draw [color=c, fill=c] (17.602,3.36069) rectangle (17.6418,3.46654);
\draw [color=c, fill=c] (17.6418,3.36069) rectangle (17.6816,3.46654);
\draw [color=c, fill=c] (17.6816,3.36069) rectangle (17.7214,3.46654);
\draw [color=c, fill=c] (17.7214,3.36069) rectangle (17.7612,3.46654);
\draw [color=c, fill=c] (17.7612,3.36069) rectangle (17.801,3.46654);
\draw [color=c, fill=c] (17.801,3.36069) rectangle (17.8408,3.46654);
\draw [color=c, fill=c] (17.8408,3.36069) rectangle (17.8806,3.46654);
\draw [color=c, fill=c] (17.8806,3.36069) rectangle (17.9204,3.46654);
\draw [color=c, fill=c] (17.9204,3.36069) rectangle (17.9602,3.46654);
\draw [color=c, fill=c] (17.9602,3.36069) rectangle (18,3.46654);
\definecolor{c}{rgb}{1,0,0};
\draw [color=c, fill=c] (2,3.46654) rectangle (2.0398,3.57239);
\draw [color=c, fill=c] (2.0398,3.46654) rectangle (2.0796,3.57239);
\draw [color=c, fill=c] (2.0796,3.46654) rectangle (2.1194,3.57239);
\draw [color=c, fill=c] (2.1194,3.46654) rectangle (2.1592,3.57239);
\draw [color=c, fill=c] (2.1592,3.46654) rectangle (2.19901,3.57239);
\draw [color=c, fill=c] (2.19901,3.46654) rectangle (2.23881,3.57239);
\draw [color=c, fill=c] (2.23881,3.46654) rectangle (2.27861,3.57239);
\draw [color=c, fill=c] (2.27861,3.46654) rectangle (2.31841,3.57239);
\draw [color=c, fill=c] (2.31841,3.46654) rectangle (2.35821,3.57239);
\draw [color=c, fill=c] (2.35821,3.46654) rectangle (2.39801,3.57239);
\draw [color=c, fill=c] (2.39801,3.46654) rectangle (2.43781,3.57239);
\draw [color=c, fill=c] (2.43781,3.46654) rectangle (2.47761,3.57239);
\draw [color=c, fill=c] (2.47761,3.46654) rectangle (2.51741,3.57239);
\draw [color=c, fill=c] (2.51741,3.46654) rectangle (2.55721,3.57239);
\draw [color=c, fill=c] (2.55721,3.46654) rectangle (2.59702,3.57239);
\draw [color=c, fill=c] (2.59702,3.46654) rectangle (2.63682,3.57239);
\draw [color=c, fill=c] (2.63682,3.46654) rectangle (2.67662,3.57239);
\draw [color=c, fill=c] (2.67662,3.46654) rectangle (2.71642,3.57239);
\draw [color=c, fill=c] (2.71642,3.46654) rectangle (2.75622,3.57239);
\draw [color=c, fill=c] (2.75622,3.46654) rectangle (2.79602,3.57239);
\draw [color=c, fill=c] (2.79602,3.46654) rectangle (2.83582,3.57239);
\draw [color=c, fill=c] (2.83582,3.46654) rectangle (2.87562,3.57239);
\draw [color=c, fill=c] (2.87562,3.46654) rectangle (2.91542,3.57239);
\draw [color=c, fill=c] (2.91542,3.46654) rectangle (2.95522,3.57239);
\draw [color=c, fill=c] (2.95522,3.46654) rectangle (2.99502,3.57239);
\draw [color=c, fill=c] (2.99502,3.46654) rectangle (3.03483,3.57239);
\draw [color=c, fill=c] (3.03483,3.46654) rectangle (3.07463,3.57239);
\draw [color=c, fill=c] (3.07463,3.46654) rectangle (3.11443,3.57239);
\draw [color=c, fill=c] (3.11443,3.46654) rectangle (3.15423,3.57239);
\draw [color=c, fill=c] (3.15423,3.46654) rectangle (3.19403,3.57239);
\draw [color=c, fill=c] (3.19403,3.46654) rectangle (3.23383,3.57239);
\draw [color=c, fill=c] (3.23383,3.46654) rectangle (3.27363,3.57239);
\draw [color=c, fill=c] (3.27363,3.46654) rectangle (3.31343,3.57239);
\draw [color=c, fill=c] (3.31343,3.46654) rectangle (3.35323,3.57239);
\draw [color=c, fill=c] (3.35323,3.46654) rectangle (3.39303,3.57239);
\draw [color=c, fill=c] (3.39303,3.46654) rectangle (3.43284,3.57239);
\draw [color=c, fill=c] (3.43284,3.46654) rectangle (3.47264,3.57239);
\draw [color=c, fill=c] (3.47264,3.46654) rectangle (3.51244,3.57239);
\draw [color=c, fill=c] (3.51244,3.46654) rectangle (3.55224,3.57239);
\draw [color=c, fill=c] (3.55224,3.46654) rectangle (3.59204,3.57239);
\draw [color=c, fill=c] (3.59204,3.46654) rectangle (3.63184,3.57239);
\draw [color=c, fill=c] (3.63184,3.46654) rectangle (3.67164,3.57239);
\draw [color=c, fill=c] (3.67164,3.46654) rectangle (3.71144,3.57239);
\draw [color=c, fill=c] (3.71144,3.46654) rectangle (3.75124,3.57239);
\draw [color=c, fill=c] (3.75124,3.46654) rectangle (3.79104,3.57239);
\draw [color=c, fill=c] (3.79104,3.46654) rectangle (3.83085,3.57239);
\draw [color=c, fill=c] (3.83085,3.46654) rectangle (3.87065,3.57239);
\draw [color=c, fill=c] (3.87065,3.46654) rectangle (3.91045,3.57239);
\draw [color=c, fill=c] (3.91045,3.46654) rectangle (3.95025,3.57239);
\draw [color=c, fill=c] (3.95025,3.46654) rectangle (3.99005,3.57239);
\draw [color=c, fill=c] (3.99005,3.46654) rectangle (4.02985,3.57239);
\draw [color=c, fill=c] (4.02985,3.46654) rectangle (4.06965,3.57239);
\draw [color=c, fill=c] (4.06965,3.46654) rectangle (4.10945,3.57239);
\draw [color=c, fill=c] (4.10945,3.46654) rectangle (4.14925,3.57239);
\draw [color=c, fill=c] (4.14925,3.46654) rectangle (4.18905,3.57239);
\draw [color=c, fill=c] (4.18905,3.46654) rectangle (4.22886,3.57239);
\draw [color=c, fill=c] (4.22886,3.46654) rectangle (4.26866,3.57239);
\draw [color=c, fill=c] (4.26866,3.46654) rectangle (4.30846,3.57239);
\draw [color=c, fill=c] (4.30846,3.46654) rectangle (4.34826,3.57239);
\draw [color=c, fill=c] (4.34826,3.46654) rectangle (4.38806,3.57239);
\draw [color=c, fill=c] (4.38806,3.46654) rectangle (4.42786,3.57239);
\draw [color=c, fill=c] (4.42786,3.46654) rectangle (4.46766,3.57239);
\draw [color=c, fill=c] (4.46766,3.46654) rectangle (4.50746,3.57239);
\draw [color=c, fill=c] (4.50746,3.46654) rectangle (4.54726,3.57239);
\draw [color=c, fill=c] (4.54726,3.46654) rectangle (4.58706,3.57239);
\draw [color=c, fill=c] (4.58706,3.46654) rectangle (4.62687,3.57239);
\draw [color=c, fill=c] (4.62687,3.46654) rectangle (4.66667,3.57239);
\draw [color=c, fill=c] (4.66667,3.46654) rectangle (4.70647,3.57239);
\draw [color=c, fill=c] (4.70647,3.46654) rectangle (4.74627,3.57239);
\draw [color=c, fill=c] (4.74627,3.46654) rectangle (4.78607,3.57239);
\draw [color=c, fill=c] (4.78607,3.46654) rectangle (4.82587,3.57239);
\draw [color=c, fill=c] (4.82587,3.46654) rectangle (4.86567,3.57239);
\draw [color=c, fill=c] (4.86567,3.46654) rectangle (4.90547,3.57239);
\draw [color=c, fill=c] (4.90547,3.46654) rectangle (4.94527,3.57239);
\draw [color=c, fill=c] (4.94527,3.46654) rectangle (4.98507,3.57239);
\draw [color=c, fill=c] (4.98507,3.46654) rectangle (5.02488,3.57239);
\draw [color=c, fill=c] (5.02488,3.46654) rectangle (5.06468,3.57239);
\draw [color=c, fill=c] (5.06468,3.46654) rectangle (5.10448,3.57239);
\draw [color=c, fill=c] (5.10448,3.46654) rectangle (5.14428,3.57239);
\draw [color=c, fill=c] (5.14428,3.46654) rectangle (5.18408,3.57239);
\draw [color=c, fill=c] (5.18408,3.46654) rectangle (5.22388,3.57239);
\draw [color=c, fill=c] (5.22388,3.46654) rectangle (5.26368,3.57239);
\draw [color=c, fill=c] (5.26368,3.46654) rectangle (5.30348,3.57239);
\draw [color=c, fill=c] (5.30348,3.46654) rectangle (5.34328,3.57239);
\draw [color=c, fill=c] (5.34328,3.46654) rectangle (5.38308,3.57239);
\draw [color=c, fill=c] (5.38308,3.46654) rectangle (5.42289,3.57239);
\draw [color=c, fill=c] (5.42289,3.46654) rectangle (5.46269,3.57239);
\draw [color=c, fill=c] (5.46269,3.46654) rectangle (5.50249,3.57239);
\draw [color=c, fill=c] (5.50249,3.46654) rectangle (5.54229,3.57239);
\draw [color=c, fill=c] (5.54229,3.46654) rectangle (5.58209,3.57239);
\draw [color=c, fill=c] (5.58209,3.46654) rectangle (5.62189,3.57239);
\draw [color=c, fill=c] (5.62189,3.46654) rectangle (5.66169,3.57239);
\draw [color=c, fill=c] (5.66169,3.46654) rectangle (5.70149,3.57239);
\draw [color=c, fill=c] (5.70149,3.46654) rectangle (5.74129,3.57239);
\draw [color=c, fill=c] (5.74129,3.46654) rectangle (5.78109,3.57239);
\draw [color=c, fill=c] (5.78109,3.46654) rectangle (5.8209,3.57239);
\draw [color=c, fill=c] (5.8209,3.46654) rectangle (5.8607,3.57239);
\draw [color=c, fill=c] (5.8607,3.46654) rectangle (5.9005,3.57239);
\draw [color=c, fill=c] (5.9005,3.46654) rectangle (5.9403,3.57239);
\draw [color=c, fill=c] (5.9403,3.46654) rectangle (5.9801,3.57239);
\draw [color=c, fill=c] (5.9801,3.46654) rectangle (6.0199,3.57239);
\draw [color=c, fill=c] (6.0199,3.46654) rectangle (6.0597,3.57239);
\draw [color=c, fill=c] (6.0597,3.46654) rectangle (6.0995,3.57239);
\draw [color=c, fill=c] (6.0995,3.46654) rectangle (6.1393,3.57239);
\draw [color=c, fill=c] (6.1393,3.46654) rectangle (6.1791,3.57239);
\draw [color=c, fill=c] (6.1791,3.46654) rectangle (6.21891,3.57239);
\draw [color=c, fill=c] (6.21891,3.46654) rectangle (6.25871,3.57239);
\draw [color=c, fill=c] (6.25871,3.46654) rectangle (6.29851,3.57239);
\draw [color=c, fill=c] (6.29851,3.46654) rectangle (6.33831,3.57239);
\draw [color=c, fill=c] (6.33831,3.46654) rectangle (6.37811,3.57239);
\draw [color=c, fill=c] (6.37811,3.46654) rectangle (6.41791,3.57239);
\draw [color=c, fill=c] (6.41791,3.46654) rectangle (6.45771,3.57239);
\draw [color=c, fill=c] (6.45771,3.46654) rectangle (6.49751,3.57239);
\draw [color=c, fill=c] (6.49751,3.46654) rectangle (6.53731,3.57239);
\draw [color=c, fill=c] (6.53731,3.46654) rectangle (6.57711,3.57239);
\draw [color=c, fill=c] (6.57711,3.46654) rectangle (6.61692,3.57239);
\draw [color=c, fill=c] (6.61692,3.46654) rectangle (6.65672,3.57239);
\draw [color=c, fill=c] (6.65672,3.46654) rectangle (6.69652,3.57239);
\draw [color=c, fill=c] (6.69652,3.46654) rectangle (6.73632,3.57239);
\draw [color=c, fill=c] (6.73632,3.46654) rectangle (6.77612,3.57239);
\draw [color=c, fill=c] (6.77612,3.46654) rectangle (6.81592,3.57239);
\draw [color=c, fill=c] (6.81592,3.46654) rectangle (6.85572,3.57239);
\draw [color=c, fill=c] (6.85572,3.46654) rectangle (6.89552,3.57239);
\draw [color=c, fill=c] (6.89552,3.46654) rectangle (6.93532,3.57239);
\draw [color=c, fill=c] (6.93532,3.46654) rectangle (6.97512,3.57239);
\draw [color=c, fill=c] (6.97512,3.46654) rectangle (7.01493,3.57239);
\draw [color=c, fill=c] (7.01493,3.46654) rectangle (7.05473,3.57239);
\draw [color=c, fill=c] (7.05473,3.46654) rectangle (7.09453,3.57239);
\draw [color=c, fill=c] (7.09453,3.46654) rectangle (7.13433,3.57239);
\draw [color=c, fill=c] (7.13433,3.46654) rectangle (7.17413,3.57239);
\draw [color=c, fill=c] (7.17413,3.46654) rectangle (7.21393,3.57239);
\draw [color=c, fill=c] (7.21393,3.46654) rectangle (7.25373,3.57239);
\draw [color=c, fill=c] (7.25373,3.46654) rectangle (7.29353,3.57239);
\draw [color=c, fill=c] (7.29353,3.46654) rectangle (7.33333,3.57239);
\draw [color=c, fill=c] (7.33333,3.46654) rectangle (7.37313,3.57239);
\draw [color=c, fill=c] (7.37313,3.46654) rectangle (7.41294,3.57239);
\draw [color=c, fill=c] (7.41294,3.46654) rectangle (7.45274,3.57239);
\draw [color=c, fill=c] (7.45274,3.46654) rectangle (7.49254,3.57239);
\draw [color=c, fill=c] (7.49254,3.46654) rectangle (7.53234,3.57239);
\draw [color=c, fill=c] (7.53234,3.46654) rectangle (7.57214,3.57239);
\draw [color=c, fill=c] (7.57214,3.46654) rectangle (7.61194,3.57239);
\draw [color=c, fill=c] (7.61194,3.46654) rectangle (7.65174,3.57239);
\draw [color=c, fill=c] (7.65174,3.46654) rectangle (7.69154,3.57239);
\draw [color=c, fill=c] (7.69154,3.46654) rectangle (7.73134,3.57239);
\draw [color=c, fill=c] (7.73134,3.46654) rectangle (7.77114,3.57239);
\draw [color=c, fill=c] (7.77114,3.46654) rectangle (7.81095,3.57239);
\draw [color=c, fill=c] (7.81095,3.46654) rectangle (7.85075,3.57239);
\draw [color=c, fill=c] (7.85075,3.46654) rectangle (7.89055,3.57239);
\draw [color=c, fill=c] (7.89055,3.46654) rectangle (7.93035,3.57239);
\draw [color=c, fill=c] (7.93035,3.46654) rectangle (7.97015,3.57239);
\draw [color=c, fill=c] (7.97015,3.46654) rectangle (8.00995,3.57239);
\draw [color=c, fill=c] (8.00995,3.46654) rectangle (8.04975,3.57239);
\definecolor{c}{rgb}{1,0.186667,0};
\draw [color=c, fill=c] (8.04975,3.46654) rectangle (8.08955,3.57239);
\draw [color=c, fill=c] (8.08955,3.46654) rectangle (8.12935,3.57239);
\draw [color=c, fill=c] (8.12935,3.46654) rectangle (8.16915,3.57239);
\draw [color=c, fill=c] (8.16915,3.46654) rectangle (8.20895,3.57239);
\draw [color=c, fill=c] (8.20895,3.46654) rectangle (8.24876,3.57239);
\draw [color=c, fill=c] (8.24876,3.46654) rectangle (8.28856,3.57239);
\draw [color=c, fill=c] (8.28856,3.46654) rectangle (8.32836,3.57239);
\draw [color=c, fill=c] (8.32836,3.46654) rectangle (8.36816,3.57239);
\draw [color=c, fill=c] (8.36816,3.46654) rectangle (8.40796,3.57239);
\draw [color=c, fill=c] (8.40796,3.46654) rectangle (8.44776,3.57239);
\draw [color=c, fill=c] (8.44776,3.46654) rectangle (8.48756,3.57239);
\draw [color=c, fill=c] (8.48756,3.46654) rectangle (8.52736,3.57239);
\draw [color=c, fill=c] (8.52736,3.46654) rectangle (8.56716,3.57239);
\draw [color=c, fill=c] (8.56716,3.46654) rectangle (8.60697,3.57239);
\draw [color=c, fill=c] (8.60697,3.46654) rectangle (8.64677,3.57239);
\draw [color=c, fill=c] (8.64677,3.46654) rectangle (8.68657,3.57239);
\draw [color=c, fill=c] (8.68657,3.46654) rectangle (8.72637,3.57239);
\draw [color=c, fill=c] (8.72637,3.46654) rectangle (8.76617,3.57239);
\draw [color=c, fill=c] (8.76617,3.46654) rectangle (8.80597,3.57239);
\draw [color=c, fill=c] (8.80597,3.46654) rectangle (8.84577,3.57239);
\draw [color=c, fill=c] (8.84577,3.46654) rectangle (8.88557,3.57239);
\draw [color=c, fill=c] (8.88557,3.46654) rectangle (8.92537,3.57239);
\draw [color=c, fill=c] (8.92537,3.46654) rectangle (8.96517,3.57239);
\draw [color=c, fill=c] (8.96517,3.46654) rectangle (9.00498,3.57239);
\definecolor{c}{rgb}{1,0.466667,0};
\draw [color=c, fill=c] (9.00498,3.46654) rectangle (9.04478,3.57239);
\draw [color=c, fill=c] (9.04478,3.46654) rectangle (9.08458,3.57239);
\draw [color=c, fill=c] (9.08458,3.46654) rectangle (9.12438,3.57239);
\draw [color=c, fill=c] (9.12438,3.46654) rectangle (9.16418,3.57239);
\draw [color=c, fill=c] (9.16418,3.46654) rectangle (9.20398,3.57239);
\draw [color=c, fill=c] (9.20398,3.46654) rectangle (9.24378,3.57239);
\draw [color=c, fill=c] (9.24378,3.46654) rectangle (9.28358,3.57239);
\draw [color=c, fill=c] (9.28358,3.46654) rectangle (9.32338,3.57239);
\draw [color=c, fill=c] (9.32338,3.46654) rectangle (9.36318,3.57239);
\draw [color=c, fill=c] (9.36318,3.46654) rectangle (9.40298,3.57239);
\draw [color=c, fill=c] (9.40298,3.46654) rectangle (9.44279,3.57239);
\draw [color=c, fill=c] (9.44279,3.46654) rectangle (9.48259,3.57239);
\definecolor{c}{rgb}{1,0.653333,0};
\draw [color=c, fill=c] (9.48259,3.46654) rectangle (9.52239,3.57239);
\draw [color=c, fill=c] (9.52239,3.46654) rectangle (9.56219,3.57239);
\draw [color=c, fill=c] (9.56219,3.46654) rectangle (9.60199,3.57239);
\draw [color=c, fill=c] (9.60199,3.46654) rectangle (9.64179,3.57239);
\draw [color=c, fill=c] (9.64179,3.46654) rectangle (9.68159,3.57239);
\draw [color=c, fill=c] (9.68159,3.46654) rectangle (9.72139,3.57239);
\definecolor{c}{rgb}{1,0.933333,0};
\draw [color=c, fill=c] (9.72139,3.46654) rectangle (9.76119,3.57239);
\draw [color=c, fill=c] (9.76119,3.46654) rectangle (9.80099,3.57239);
\draw [color=c, fill=c] (9.80099,3.46654) rectangle (9.8408,3.57239);
\definecolor{c}{rgb}{0.88,1,0};
\draw [color=c, fill=c] (9.8408,3.46654) rectangle (9.8806,3.57239);
\draw [color=c, fill=c] (9.8806,3.46654) rectangle (9.9204,3.57239);
\definecolor{c}{rgb}{0.6,1,0};
\draw [color=c, fill=c] (9.9204,3.46654) rectangle (9.9602,3.57239);
\definecolor{c}{rgb}{0.413333,1,0};
\draw [color=c, fill=c] (9.9602,3.46654) rectangle (10,3.57239);
\definecolor{c}{rgb}{0.133333,1,0};
\draw [color=c, fill=c] (10,3.46654) rectangle (10.0398,3.57239);
\definecolor{c}{rgb}{0,1,0.0533333};
\draw [color=c, fill=c] (10.0398,3.46654) rectangle (10.0796,3.57239);
\definecolor{c}{rgb}{0,1,0.333333};
\draw [color=c, fill=c] (10.0796,3.46654) rectangle (10.1194,3.57239);
\draw [color=c, fill=c] (10.1194,3.46654) rectangle (10.1592,3.57239);
\draw [color=c, fill=c] (10.1592,3.46654) rectangle (10.199,3.57239);
\definecolor{c}{rgb}{0,1,0.52};
\draw [color=c, fill=c] (10.199,3.46654) rectangle (10.2388,3.57239);
\draw [color=c, fill=c] (10.2388,3.46654) rectangle (10.2786,3.57239);
\draw [color=c, fill=c] (10.2786,3.46654) rectangle (10.3184,3.57239);
\draw [color=c, fill=c] (10.3184,3.46654) rectangle (10.3582,3.57239);
\draw [color=c, fill=c] (10.3582,3.46654) rectangle (10.398,3.57239);
\draw [color=c, fill=c] (10.398,3.46654) rectangle (10.4378,3.57239);
\draw [color=c, fill=c] (10.4378,3.46654) rectangle (10.4776,3.57239);
\definecolor{c}{rgb}{0,1,0.8};
\draw [color=c, fill=c] (10.4776,3.46654) rectangle (10.5174,3.57239);
\draw [color=c, fill=c] (10.5174,3.46654) rectangle (10.5572,3.57239);
\draw [color=c, fill=c] (10.5572,3.46654) rectangle (10.597,3.57239);
\draw [color=c, fill=c] (10.597,3.46654) rectangle (10.6368,3.57239);
\draw [color=c, fill=c] (10.6368,3.46654) rectangle (10.6766,3.57239);
\draw [color=c, fill=c] (10.6766,3.46654) rectangle (10.7164,3.57239);
\draw [color=c, fill=c] (10.7164,3.46654) rectangle (10.7562,3.57239);
\draw [color=c, fill=c] (10.7562,3.46654) rectangle (10.796,3.57239);
\draw [color=c, fill=c] (10.796,3.46654) rectangle (10.8358,3.57239);
\draw [color=c, fill=c] (10.8358,3.46654) rectangle (10.8756,3.57239);
\draw [color=c, fill=c] (10.8756,3.46654) rectangle (10.9154,3.57239);
\draw [color=c, fill=c] (10.9154,3.46654) rectangle (10.9552,3.57239);
\draw [color=c, fill=c] (10.9552,3.46654) rectangle (10.995,3.57239);
\definecolor{c}{rgb}{0,1,0.986667};
\draw [color=c, fill=c] (10.995,3.46654) rectangle (11.0348,3.57239);
\draw [color=c, fill=c] (11.0348,3.46654) rectangle (11.0746,3.57239);
\draw [color=c, fill=c] (11.0746,3.46654) rectangle (11.1144,3.57239);
\draw [color=c, fill=c] (11.1144,3.46654) rectangle (11.1542,3.57239);
\draw [color=c, fill=c] (11.1542,3.46654) rectangle (11.194,3.57239);
\draw [color=c, fill=c] (11.194,3.46654) rectangle (11.2338,3.57239);
\draw [color=c, fill=c] (11.2338,3.46654) rectangle (11.2736,3.57239);
\draw [color=c, fill=c] (11.2736,3.46654) rectangle (11.3134,3.57239);
\draw [color=c, fill=c] (11.3134,3.46654) rectangle (11.3532,3.57239);
\draw [color=c, fill=c] (11.3532,3.46654) rectangle (11.393,3.57239);
\draw [color=c, fill=c] (11.393,3.46654) rectangle (11.4328,3.57239);
\draw [color=c, fill=c] (11.4328,3.46654) rectangle (11.4726,3.57239);
\draw [color=c, fill=c] (11.4726,3.46654) rectangle (11.5124,3.57239);
\draw [color=c, fill=c] (11.5124,3.46654) rectangle (11.5522,3.57239);
\draw [color=c, fill=c] (11.5522,3.46654) rectangle (11.592,3.57239);
\draw [color=c, fill=c] (11.592,3.46654) rectangle (11.6318,3.57239);
\draw [color=c, fill=c] (11.6318,3.46654) rectangle (11.6716,3.57239);
\draw [color=c, fill=c] (11.6716,3.46654) rectangle (11.7114,3.57239);
\draw [color=c, fill=c] (11.7114,3.46654) rectangle (11.7512,3.57239);
\draw [color=c, fill=c] (11.7512,3.46654) rectangle (11.791,3.57239);
\draw [color=c, fill=c] (11.791,3.46654) rectangle (11.8308,3.57239);
\draw [color=c, fill=c] (11.8308,3.46654) rectangle (11.8706,3.57239);
\draw [color=c, fill=c] (11.8706,3.46654) rectangle (11.9104,3.57239);
\draw [color=c, fill=c] (11.9104,3.46654) rectangle (11.9502,3.57239);
\draw [color=c, fill=c] (11.9502,3.46654) rectangle (11.99,3.57239);
\draw [color=c, fill=c] (11.99,3.46654) rectangle (12.0299,3.57239);
\definecolor{c}{rgb}{0,0.733333,1};
\draw [color=c, fill=c] (12.0299,3.46654) rectangle (12.0697,3.57239);
\draw [color=c, fill=c] (12.0697,3.46654) rectangle (12.1095,3.57239);
\draw [color=c, fill=c] (12.1095,3.46654) rectangle (12.1493,3.57239);
\draw [color=c, fill=c] (12.1493,3.46654) rectangle (12.1891,3.57239);
\draw [color=c, fill=c] (12.1891,3.46654) rectangle (12.2289,3.57239);
\draw [color=c, fill=c] (12.2289,3.46654) rectangle (12.2687,3.57239);
\draw [color=c, fill=c] (12.2687,3.46654) rectangle (12.3085,3.57239);
\draw [color=c, fill=c] (12.3085,3.46654) rectangle (12.3483,3.57239);
\draw [color=c, fill=c] (12.3483,3.46654) rectangle (12.3881,3.57239);
\draw [color=c, fill=c] (12.3881,3.46654) rectangle (12.4279,3.57239);
\draw [color=c, fill=c] (12.4279,3.46654) rectangle (12.4677,3.57239);
\draw [color=c, fill=c] (12.4677,3.46654) rectangle (12.5075,3.57239);
\draw [color=c, fill=c] (12.5075,3.46654) rectangle (12.5473,3.57239);
\draw [color=c, fill=c] (12.5473,3.46654) rectangle (12.5871,3.57239);
\draw [color=c, fill=c] (12.5871,3.46654) rectangle (12.6269,3.57239);
\draw [color=c, fill=c] (12.6269,3.46654) rectangle (12.6667,3.57239);
\draw [color=c, fill=c] (12.6667,3.46654) rectangle (12.7065,3.57239);
\draw [color=c, fill=c] (12.7065,3.46654) rectangle (12.7463,3.57239);
\draw [color=c, fill=c] (12.7463,3.46654) rectangle (12.7861,3.57239);
\draw [color=c, fill=c] (12.7861,3.46654) rectangle (12.8259,3.57239);
\draw [color=c, fill=c] (12.8259,3.46654) rectangle (12.8657,3.57239);
\draw [color=c, fill=c] (12.8657,3.46654) rectangle (12.9055,3.57239);
\draw [color=c, fill=c] (12.9055,3.46654) rectangle (12.9453,3.57239);
\draw [color=c, fill=c] (12.9453,3.46654) rectangle (12.9851,3.57239);
\draw [color=c, fill=c] (12.9851,3.46654) rectangle (13.0249,3.57239);
\draw [color=c, fill=c] (13.0249,3.46654) rectangle (13.0647,3.57239);
\draw [color=c, fill=c] (13.0647,3.46654) rectangle (13.1045,3.57239);
\draw [color=c, fill=c] (13.1045,3.46654) rectangle (13.1443,3.57239);
\draw [color=c, fill=c] (13.1443,3.46654) rectangle (13.1841,3.57239);
\draw [color=c, fill=c] (13.1841,3.46654) rectangle (13.2239,3.57239);
\draw [color=c, fill=c] (13.2239,3.46654) rectangle (13.2637,3.57239);
\draw [color=c, fill=c] (13.2637,3.46654) rectangle (13.3035,3.57239);
\draw [color=c, fill=c] (13.3035,3.46654) rectangle (13.3433,3.57239);
\draw [color=c, fill=c] (13.3433,3.46654) rectangle (13.3831,3.57239);
\draw [color=c, fill=c] (13.3831,3.46654) rectangle (13.4229,3.57239);
\draw [color=c, fill=c] (13.4229,3.46654) rectangle (13.4627,3.57239);
\draw [color=c, fill=c] (13.4627,3.46654) rectangle (13.5025,3.57239);
\draw [color=c, fill=c] (13.5025,3.46654) rectangle (13.5423,3.57239);
\draw [color=c, fill=c] (13.5423,3.46654) rectangle (13.5821,3.57239);
\draw [color=c, fill=c] (13.5821,3.46654) rectangle (13.6219,3.57239);
\draw [color=c, fill=c] (13.6219,3.46654) rectangle (13.6617,3.57239);
\draw [color=c, fill=c] (13.6617,3.46654) rectangle (13.7015,3.57239);
\draw [color=c, fill=c] (13.7015,3.46654) rectangle (13.7413,3.57239);
\draw [color=c, fill=c] (13.7413,3.46654) rectangle (13.7811,3.57239);
\draw [color=c, fill=c] (13.7811,3.46654) rectangle (13.8209,3.57239);
\draw [color=c, fill=c] (13.8209,3.46654) rectangle (13.8607,3.57239);
\draw [color=c, fill=c] (13.8607,3.46654) rectangle (13.9005,3.57239);
\draw [color=c, fill=c] (13.9005,3.46654) rectangle (13.9403,3.57239);
\draw [color=c, fill=c] (13.9403,3.46654) rectangle (13.9801,3.57239);
\draw [color=c, fill=c] (13.9801,3.46654) rectangle (14.0199,3.57239);
\draw [color=c, fill=c] (14.0199,3.46654) rectangle (14.0597,3.57239);
\draw [color=c, fill=c] (14.0597,3.46654) rectangle (14.0995,3.57239);
\draw [color=c, fill=c] (14.0995,3.46654) rectangle (14.1393,3.57239);
\draw [color=c, fill=c] (14.1393,3.46654) rectangle (14.1791,3.57239);
\draw [color=c, fill=c] (14.1791,3.46654) rectangle (14.2189,3.57239);
\draw [color=c, fill=c] (14.2189,3.46654) rectangle (14.2587,3.57239);
\draw [color=c, fill=c] (14.2587,3.46654) rectangle (14.2985,3.57239);
\draw [color=c, fill=c] (14.2985,3.46654) rectangle (14.3383,3.57239);
\draw [color=c, fill=c] (14.3383,3.46654) rectangle (14.3781,3.57239);
\draw [color=c, fill=c] (14.3781,3.46654) rectangle (14.4179,3.57239);
\draw [color=c, fill=c] (14.4179,3.46654) rectangle (14.4577,3.57239);
\draw [color=c, fill=c] (14.4577,3.46654) rectangle (14.4975,3.57239);
\draw [color=c, fill=c] (14.4975,3.46654) rectangle (14.5373,3.57239);
\draw [color=c, fill=c] (14.5373,3.46654) rectangle (14.5771,3.57239);
\draw [color=c, fill=c] (14.5771,3.46654) rectangle (14.6169,3.57239);
\draw [color=c, fill=c] (14.6169,3.46654) rectangle (14.6567,3.57239);
\draw [color=c, fill=c] (14.6567,3.46654) rectangle (14.6965,3.57239);
\draw [color=c, fill=c] (14.6965,3.46654) rectangle (14.7363,3.57239);
\draw [color=c, fill=c] (14.7363,3.46654) rectangle (14.7761,3.57239);
\draw [color=c, fill=c] (14.7761,3.46654) rectangle (14.8159,3.57239);
\draw [color=c, fill=c] (14.8159,3.46654) rectangle (14.8557,3.57239);
\draw [color=c, fill=c] (14.8557,3.46654) rectangle (14.8955,3.57239);
\draw [color=c, fill=c] (14.8955,3.46654) rectangle (14.9353,3.57239);
\draw [color=c, fill=c] (14.9353,3.46654) rectangle (14.9751,3.57239);
\draw [color=c, fill=c] (14.9751,3.46654) rectangle (15.0149,3.57239);
\draw [color=c, fill=c] (15.0149,3.46654) rectangle (15.0547,3.57239);
\draw [color=c, fill=c] (15.0547,3.46654) rectangle (15.0945,3.57239);
\draw [color=c, fill=c] (15.0945,3.46654) rectangle (15.1343,3.57239);
\draw [color=c, fill=c] (15.1343,3.46654) rectangle (15.1741,3.57239);
\draw [color=c, fill=c] (15.1741,3.46654) rectangle (15.2139,3.57239);
\draw [color=c, fill=c] (15.2139,3.46654) rectangle (15.2537,3.57239);
\draw [color=c, fill=c] (15.2537,3.46654) rectangle (15.2935,3.57239);
\draw [color=c, fill=c] (15.2935,3.46654) rectangle (15.3333,3.57239);
\draw [color=c, fill=c] (15.3333,3.46654) rectangle (15.3731,3.57239);
\draw [color=c, fill=c] (15.3731,3.46654) rectangle (15.4129,3.57239);
\draw [color=c, fill=c] (15.4129,3.46654) rectangle (15.4527,3.57239);
\draw [color=c, fill=c] (15.4527,3.46654) rectangle (15.4925,3.57239);
\draw [color=c, fill=c] (15.4925,3.46654) rectangle (15.5323,3.57239);
\draw [color=c, fill=c] (15.5323,3.46654) rectangle (15.5721,3.57239);
\draw [color=c, fill=c] (15.5721,3.46654) rectangle (15.6119,3.57239);
\draw [color=c, fill=c] (15.6119,3.46654) rectangle (15.6517,3.57239);
\draw [color=c, fill=c] (15.6517,3.46654) rectangle (15.6915,3.57239);
\draw [color=c, fill=c] (15.6915,3.46654) rectangle (15.7313,3.57239);
\draw [color=c, fill=c] (15.7313,3.46654) rectangle (15.7711,3.57239);
\draw [color=c, fill=c] (15.7711,3.46654) rectangle (15.8109,3.57239);
\draw [color=c, fill=c] (15.8109,3.46654) rectangle (15.8507,3.57239);
\draw [color=c, fill=c] (15.8507,3.46654) rectangle (15.8905,3.57239);
\draw [color=c, fill=c] (15.8905,3.46654) rectangle (15.9303,3.57239);
\draw [color=c, fill=c] (15.9303,3.46654) rectangle (15.9701,3.57239);
\draw [color=c, fill=c] (15.9701,3.46654) rectangle (16.01,3.57239);
\draw [color=c, fill=c] (16.01,3.46654) rectangle (16.0498,3.57239);
\draw [color=c, fill=c] (16.0498,3.46654) rectangle (16.0896,3.57239);
\draw [color=c, fill=c] (16.0896,3.46654) rectangle (16.1294,3.57239);
\draw [color=c, fill=c] (16.1294,3.46654) rectangle (16.1692,3.57239);
\draw [color=c, fill=c] (16.1692,3.46654) rectangle (16.209,3.57239);
\draw [color=c, fill=c] (16.209,3.46654) rectangle (16.2488,3.57239);
\draw [color=c, fill=c] (16.2488,3.46654) rectangle (16.2886,3.57239);
\draw [color=c, fill=c] (16.2886,3.46654) rectangle (16.3284,3.57239);
\draw [color=c, fill=c] (16.3284,3.46654) rectangle (16.3682,3.57239);
\draw [color=c, fill=c] (16.3682,3.46654) rectangle (16.408,3.57239);
\draw [color=c, fill=c] (16.408,3.46654) rectangle (16.4478,3.57239);
\draw [color=c, fill=c] (16.4478,3.46654) rectangle (16.4876,3.57239);
\draw [color=c, fill=c] (16.4876,3.46654) rectangle (16.5274,3.57239);
\draw [color=c, fill=c] (16.5274,3.46654) rectangle (16.5672,3.57239);
\draw [color=c, fill=c] (16.5672,3.46654) rectangle (16.607,3.57239);
\draw [color=c, fill=c] (16.607,3.46654) rectangle (16.6468,3.57239);
\draw [color=c, fill=c] (16.6468,3.46654) rectangle (16.6866,3.57239);
\draw [color=c, fill=c] (16.6866,3.46654) rectangle (16.7264,3.57239);
\draw [color=c, fill=c] (16.7264,3.46654) rectangle (16.7662,3.57239);
\draw [color=c, fill=c] (16.7662,3.46654) rectangle (16.806,3.57239);
\draw [color=c, fill=c] (16.806,3.46654) rectangle (16.8458,3.57239);
\draw [color=c, fill=c] (16.8458,3.46654) rectangle (16.8856,3.57239);
\draw [color=c, fill=c] (16.8856,3.46654) rectangle (16.9254,3.57239);
\draw [color=c, fill=c] (16.9254,3.46654) rectangle (16.9652,3.57239);
\draw [color=c, fill=c] (16.9652,3.46654) rectangle (17.005,3.57239);
\draw [color=c, fill=c] (17.005,3.46654) rectangle (17.0448,3.57239);
\draw [color=c, fill=c] (17.0448,3.46654) rectangle (17.0846,3.57239);
\draw [color=c, fill=c] (17.0846,3.46654) rectangle (17.1244,3.57239);
\draw [color=c, fill=c] (17.1244,3.46654) rectangle (17.1642,3.57239);
\draw [color=c, fill=c] (17.1642,3.46654) rectangle (17.204,3.57239);
\draw [color=c, fill=c] (17.204,3.46654) rectangle (17.2438,3.57239);
\draw [color=c, fill=c] (17.2438,3.46654) rectangle (17.2836,3.57239);
\draw [color=c, fill=c] (17.2836,3.46654) rectangle (17.3234,3.57239);
\draw [color=c, fill=c] (17.3234,3.46654) rectangle (17.3632,3.57239);
\draw [color=c, fill=c] (17.3632,3.46654) rectangle (17.403,3.57239);
\draw [color=c, fill=c] (17.403,3.46654) rectangle (17.4428,3.57239);
\draw [color=c, fill=c] (17.4428,3.46654) rectangle (17.4826,3.57239);
\draw [color=c, fill=c] (17.4826,3.46654) rectangle (17.5224,3.57239);
\draw [color=c, fill=c] (17.5224,3.46654) rectangle (17.5622,3.57239);
\draw [color=c, fill=c] (17.5622,3.46654) rectangle (17.602,3.57239);
\draw [color=c, fill=c] (17.602,3.46654) rectangle (17.6418,3.57239);
\draw [color=c, fill=c] (17.6418,3.46654) rectangle (17.6816,3.57239);
\draw [color=c, fill=c] (17.6816,3.46654) rectangle (17.7214,3.57239);
\draw [color=c, fill=c] (17.7214,3.46654) rectangle (17.7612,3.57239);
\draw [color=c, fill=c] (17.7612,3.46654) rectangle (17.801,3.57239);
\draw [color=c, fill=c] (17.801,3.46654) rectangle (17.8408,3.57239);
\draw [color=c, fill=c] (17.8408,3.46654) rectangle (17.8806,3.57239);
\draw [color=c, fill=c] (17.8806,3.46654) rectangle (17.9204,3.57239);
\draw [color=c, fill=c] (17.9204,3.46654) rectangle (17.9602,3.57239);
\draw [color=c, fill=c] (17.9602,3.46654) rectangle (18,3.57239);
\definecolor{c}{rgb}{1,0,0};
\draw [color=c, fill=c] (2,3.57239) rectangle (2.0398,3.67824);
\draw [color=c, fill=c] (2.0398,3.57239) rectangle (2.0796,3.67824);
\draw [color=c, fill=c] (2.0796,3.57239) rectangle (2.1194,3.67824);
\draw [color=c, fill=c] (2.1194,3.57239) rectangle (2.1592,3.67824);
\draw [color=c, fill=c] (2.1592,3.57239) rectangle (2.19901,3.67824);
\draw [color=c, fill=c] (2.19901,3.57239) rectangle (2.23881,3.67824);
\draw [color=c, fill=c] (2.23881,3.57239) rectangle (2.27861,3.67824);
\draw [color=c, fill=c] (2.27861,3.57239) rectangle (2.31841,3.67824);
\draw [color=c, fill=c] (2.31841,3.57239) rectangle (2.35821,3.67824);
\draw [color=c, fill=c] (2.35821,3.57239) rectangle (2.39801,3.67824);
\draw [color=c, fill=c] (2.39801,3.57239) rectangle (2.43781,3.67824);
\draw [color=c, fill=c] (2.43781,3.57239) rectangle (2.47761,3.67824);
\draw [color=c, fill=c] (2.47761,3.57239) rectangle (2.51741,3.67824);
\draw [color=c, fill=c] (2.51741,3.57239) rectangle (2.55721,3.67824);
\draw [color=c, fill=c] (2.55721,3.57239) rectangle (2.59702,3.67824);
\draw [color=c, fill=c] (2.59702,3.57239) rectangle (2.63682,3.67824);
\draw [color=c, fill=c] (2.63682,3.57239) rectangle (2.67662,3.67824);
\draw [color=c, fill=c] (2.67662,3.57239) rectangle (2.71642,3.67824);
\draw [color=c, fill=c] (2.71642,3.57239) rectangle (2.75622,3.67824);
\draw [color=c, fill=c] (2.75622,3.57239) rectangle (2.79602,3.67824);
\draw [color=c, fill=c] (2.79602,3.57239) rectangle (2.83582,3.67824);
\draw [color=c, fill=c] (2.83582,3.57239) rectangle (2.87562,3.67824);
\draw [color=c, fill=c] (2.87562,3.57239) rectangle (2.91542,3.67824);
\draw [color=c, fill=c] (2.91542,3.57239) rectangle (2.95522,3.67824);
\draw [color=c, fill=c] (2.95522,3.57239) rectangle (2.99502,3.67824);
\draw [color=c, fill=c] (2.99502,3.57239) rectangle (3.03483,3.67824);
\draw [color=c, fill=c] (3.03483,3.57239) rectangle (3.07463,3.67824);
\draw [color=c, fill=c] (3.07463,3.57239) rectangle (3.11443,3.67824);
\draw [color=c, fill=c] (3.11443,3.57239) rectangle (3.15423,3.67824);
\draw [color=c, fill=c] (3.15423,3.57239) rectangle (3.19403,3.67824);
\draw [color=c, fill=c] (3.19403,3.57239) rectangle (3.23383,3.67824);
\draw [color=c, fill=c] (3.23383,3.57239) rectangle (3.27363,3.67824);
\draw [color=c, fill=c] (3.27363,3.57239) rectangle (3.31343,3.67824);
\draw [color=c, fill=c] (3.31343,3.57239) rectangle (3.35323,3.67824);
\draw [color=c, fill=c] (3.35323,3.57239) rectangle (3.39303,3.67824);
\draw [color=c, fill=c] (3.39303,3.57239) rectangle (3.43284,3.67824);
\draw [color=c, fill=c] (3.43284,3.57239) rectangle (3.47264,3.67824);
\draw [color=c, fill=c] (3.47264,3.57239) rectangle (3.51244,3.67824);
\draw [color=c, fill=c] (3.51244,3.57239) rectangle (3.55224,3.67824);
\draw [color=c, fill=c] (3.55224,3.57239) rectangle (3.59204,3.67824);
\draw [color=c, fill=c] (3.59204,3.57239) rectangle (3.63184,3.67824);
\draw [color=c, fill=c] (3.63184,3.57239) rectangle (3.67164,3.67824);
\draw [color=c, fill=c] (3.67164,3.57239) rectangle (3.71144,3.67824);
\draw [color=c, fill=c] (3.71144,3.57239) rectangle (3.75124,3.67824);
\draw [color=c, fill=c] (3.75124,3.57239) rectangle (3.79104,3.67824);
\draw [color=c, fill=c] (3.79104,3.57239) rectangle (3.83085,3.67824);
\draw [color=c, fill=c] (3.83085,3.57239) rectangle (3.87065,3.67824);
\draw [color=c, fill=c] (3.87065,3.57239) rectangle (3.91045,3.67824);
\draw [color=c, fill=c] (3.91045,3.57239) rectangle (3.95025,3.67824);
\draw [color=c, fill=c] (3.95025,3.57239) rectangle (3.99005,3.67824);
\draw [color=c, fill=c] (3.99005,3.57239) rectangle (4.02985,3.67824);
\draw [color=c, fill=c] (4.02985,3.57239) rectangle (4.06965,3.67824);
\draw [color=c, fill=c] (4.06965,3.57239) rectangle (4.10945,3.67824);
\draw [color=c, fill=c] (4.10945,3.57239) rectangle (4.14925,3.67824);
\draw [color=c, fill=c] (4.14925,3.57239) rectangle (4.18905,3.67824);
\draw [color=c, fill=c] (4.18905,3.57239) rectangle (4.22886,3.67824);
\draw [color=c, fill=c] (4.22886,3.57239) rectangle (4.26866,3.67824);
\draw [color=c, fill=c] (4.26866,3.57239) rectangle (4.30846,3.67824);
\draw [color=c, fill=c] (4.30846,3.57239) rectangle (4.34826,3.67824);
\draw [color=c, fill=c] (4.34826,3.57239) rectangle (4.38806,3.67824);
\draw [color=c, fill=c] (4.38806,3.57239) rectangle (4.42786,3.67824);
\draw [color=c, fill=c] (4.42786,3.57239) rectangle (4.46766,3.67824);
\draw [color=c, fill=c] (4.46766,3.57239) rectangle (4.50746,3.67824);
\draw [color=c, fill=c] (4.50746,3.57239) rectangle (4.54726,3.67824);
\draw [color=c, fill=c] (4.54726,3.57239) rectangle (4.58706,3.67824);
\draw [color=c, fill=c] (4.58706,3.57239) rectangle (4.62687,3.67824);
\draw [color=c, fill=c] (4.62687,3.57239) rectangle (4.66667,3.67824);
\draw [color=c, fill=c] (4.66667,3.57239) rectangle (4.70647,3.67824);
\draw [color=c, fill=c] (4.70647,3.57239) rectangle (4.74627,3.67824);
\draw [color=c, fill=c] (4.74627,3.57239) rectangle (4.78607,3.67824);
\draw [color=c, fill=c] (4.78607,3.57239) rectangle (4.82587,3.67824);
\draw [color=c, fill=c] (4.82587,3.57239) rectangle (4.86567,3.67824);
\draw [color=c, fill=c] (4.86567,3.57239) rectangle (4.90547,3.67824);
\draw [color=c, fill=c] (4.90547,3.57239) rectangle (4.94527,3.67824);
\draw [color=c, fill=c] (4.94527,3.57239) rectangle (4.98507,3.67824);
\draw [color=c, fill=c] (4.98507,3.57239) rectangle (5.02488,3.67824);
\draw [color=c, fill=c] (5.02488,3.57239) rectangle (5.06468,3.67824);
\draw [color=c, fill=c] (5.06468,3.57239) rectangle (5.10448,3.67824);
\draw [color=c, fill=c] (5.10448,3.57239) rectangle (5.14428,3.67824);
\draw [color=c, fill=c] (5.14428,3.57239) rectangle (5.18408,3.67824);
\draw [color=c, fill=c] (5.18408,3.57239) rectangle (5.22388,3.67824);
\draw [color=c, fill=c] (5.22388,3.57239) rectangle (5.26368,3.67824);
\draw [color=c, fill=c] (5.26368,3.57239) rectangle (5.30348,3.67824);
\draw [color=c, fill=c] (5.30348,3.57239) rectangle (5.34328,3.67824);
\draw [color=c, fill=c] (5.34328,3.57239) rectangle (5.38308,3.67824);
\draw [color=c, fill=c] (5.38308,3.57239) rectangle (5.42289,3.67824);
\draw [color=c, fill=c] (5.42289,3.57239) rectangle (5.46269,3.67824);
\draw [color=c, fill=c] (5.46269,3.57239) rectangle (5.50249,3.67824);
\draw [color=c, fill=c] (5.50249,3.57239) rectangle (5.54229,3.67824);
\draw [color=c, fill=c] (5.54229,3.57239) rectangle (5.58209,3.67824);
\draw [color=c, fill=c] (5.58209,3.57239) rectangle (5.62189,3.67824);
\draw [color=c, fill=c] (5.62189,3.57239) rectangle (5.66169,3.67824);
\draw [color=c, fill=c] (5.66169,3.57239) rectangle (5.70149,3.67824);
\draw [color=c, fill=c] (5.70149,3.57239) rectangle (5.74129,3.67824);
\draw [color=c, fill=c] (5.74129,3.57239) rectangle (5.78109,3.67824);
\draw [color=c, fill=c] (5.78109,3.57239) rectangle (5.8209,3.67824);
\draw [color=c, fill=c] (5.8209,3.57239) rectangle (5.8607,3.67824);
\draw [color=c, fill=c] (5.8607,3.57239) rectangle (5.9005,3.67824);
\draw [color=c, fill=c] (5.9005,3.57239) rectangle (5.9403,3.67824);
\draw [color=c, fill=c] (5.9403,3.57239) rectangle (5.9801,3.67824);
\draw [color=c, fill=c] (5.9801,3.57239) rectangle (6.0199,3.67824);
\draw [color=c, fill=c] (6.0199,3.57239) rectangle (6.0597,3.67824);
\draw [color=c, fill=c] (6.0597,3.57239) rectangle (6.0995,3.67824);
\draw [color=c, fill=c] (6.0995,3.57239) rectangle (6.1393,3.67824);
\draw [color=c, fill=c] (6.1393,3.57239) rectangle (6.1791,3.67824);
\draw [color=c, fill=c] (6.1791,3.57239) rectangle (6.21891,3.67824);
\draw [color=c, fill=c] (6.21891,3.57239) rectangle (6.25871,3.67824);
\draw [color=c, fill=c] (6.25871,3.57239) rectangle (6.29851,3.67824);
\draw [color=c, fill=c] (6.29851,3.57239) rectangle (6.33831,3.67824);
\draw [color=c, fill=c] (6.33831,3.57239) rectangle (6.37811,3.67824);
\draw [color=c, fill=c] (6.37811,3.57239) rectangle (6.41791,3.67824);
\draw [color=c, fill=c] (6.41791,3.57239) rectangle (6.45771,3.67824);
\draw [color=c, fill=c] (6.45771,3.57239) rectangle (6.49751,3.67824);
\draw [color=c, fill=c] (6.49751,3.57239) rectangle (6.53731,3.67824);
\draw [color=c, fill=c] (6.53731,3.57239) rectangle (6.57711,3.67824);
\draw [color=c, fill=c] (6.57711,3.57239) rectangle (6.61692,3.67824);
\draw [color=c, fill=c] (6.61692,3.57239) rectangle (6.65672,3.67824);
\draw [color=c, fill=c] (6.65672,3.57239) rectangle (6.69652,3.67824);
\draw [color=c, fill=c] (6.69652,3.57239) rectangle (6.73632,3.67824);
\draw [color=c, fill=c] (6.73632,3.57239) rectangle (6.77612,3.67824);
\draw [color=c, fill=c] (6.77612,3.57239) rectangle (6.81592,3.67824);
\draw [color=c, fill=c] (6.81592,3.57239) rectangle (6.85572,3.67824);
\draw [color=c, fill=c] (6.85572,3.57239) rectangle (6.89552,3.67824);
\draw [color=c, fill=c] (6.89552,3.57239) rectangle (6.93532,3.67824);
\draw [color=c, fill=c] (6.93532,3.57239) rectangle (6.97512,3.67824);
\draw [color=c, fill=c] (6.97512,3.57239) rectangle (7.01493,3.67824);
\draw [color=c, fill=c] (7.01493,3.57239) rectangle (7.05473,3.67824);
\draw [color=c, fill=c] (7.05473,3.57239) rectangle (7.09453,3.67824);
\draw [color=c, fill=c] (7.09453,3.57239) rectangle (7.13433,3.67824);
\draw [color=c, fill=c] (7.13433,3.57239) rectangle (7.17413,3.67824);
\draw [color=c, fill=c] (7.17413,3.57239) rectangle (7.21393,3.67824);
\draw [color=c, fill=c] (7.21393,3.57239) rectangle (7.25373,3.67824);
\draw [color=c, fill=c] (7.25373,3.57239) rectangle (7.29353,3.67824);
\draw [color=c, fill=c] (7.29353,3.57239) rectangle (7.33333,3.67824);
\draw [color=c, fill=c] (7.33333,3.57239) rectangle (7.37313,3.67824);
\draw [color=c, fill=c] (7.37313,3.57239) rectangle (7.41294,3.67824);
\draw [color=c, fill=c] (7.41294,3.57239) rectangle (7.45274,3.67824);
\draw [color=c, fill=c] (7.45274,3.57239) rectangle (7.49254,3.67824);
\draw [color=c, fill=c] (7.49254,3.57239) rectangle (7.53234,3.67824);
\draw [color=c, fill=c] (7.53234,3.57239) rectangle (7.57214,3.67824);
\draw [color=c, fill=c] (7.57214,3.57239) rectangle (7.61194,3.67824);
\draw [color=c, fill=c] (7.61194,3.57239) rectangle (7.65174,3.67824);
\draw [color=c, fill=c] (7.65174,3.57239) rectangle (7.69154,3.67824);
\draw [color=c, fill=c] (7.69154,3.57239) rectangle (7.73134,3.67824);
\draw [color=c, fill=c] (7.73134,3.57239) rectangle (7.77114,3.67824);
\draw [color=c, fill=c] (7.77114,3.57239) rectangle (7.81095,3.67824);
\draw [color=c, fill=c] (7.81095,3.57239) rectangle (7.85075,3.67824);
\draw [color=c, fill=c] (7.85075,3.57239) rectangle (7.89055,3.67824);
\draw [color=c, fill=c] (7.89055,3.57239) rectangle (7.93035,3.67824);
\draw [color=c, fill=c] (7.93035,3.57239) rectangle (7.97015,3.67824);
\draw [color=c, fill=c] (7.97015,3.57239) rectangle (8.00995,3.67824);
\draw [color=c, fill=c] (8.00995,3.57239) rectangle (8.04975,3.67824);
\draw [color=c, fill=c] (8.04975,3.57239) rectangle (8.08955,3.67824);
\definecolor{c}{rgb}{1,0.186667,0};
\draw [color=c, fill=c] (8.08955,3.57239) rectangle (8.12935,3.67824);
\draw [color=c, fill=c] (8.12935,3.57239) rectangle (8.16915,3.67824);
\draw [color=c, fill=c] (8.16915,3.57239) rectangle (8.20895,3.67824);
\draw [color=c, fill=c] (8.20895,3.57239) rectangle (8.24876,3.67824);
\draw [color=c, fill=c] (8.24876,3.57239) rectangle (8.28856,3.67824);
\draw [color=c, fill=c] (8.28856,3.57239) rectangle (8.32836,3.67824);
\draw [color=c, fill=c] (8.32836,3.57239) rectangle (8.36816,3.67824);
\draw [color=c, fill=c] (8.36816,3.57239) rectangle (8.40796,3.67824);
\draw [color=c, fill=c] (8.40796,3.57239) rectangle (8.44776,3.67824);
\draw [color=c, fill=c] (8.44776,3.57239) rectangle (8.48756,3.67824);
\draw [color=c, fill=c] (8.48756,3.57239) rectangle (8.52736,3.67824);
\draw [color=c, fill=c] (8.52736,3.57239) rectangle (8.56716,3.67824);
\draw [color=c, fill=c] (8.56716,3.57239) rectangle (8.60697,3.67824);
\draw [color=c, fill=c] (8.60697,3.57239) rectangle (8.64677,3.67824);
\draw [color=c, fill=c] (8.64677,3.57239) rectangle (8.68657,3.67824);
\draw [color=c, fill=c] (8.68657,3.57239) rectangle (8.72637,3.67824);
\draw [color=c, fill=c] (8.72637,3.57239) rectangle (8.76617,3.67824);
\draw [color=c, fill=c] (8.76617,3.57239) rectangle (8.80597,3.67824);
\draw [color=c, fill=c] (8.80597,3.57239) rectangle (8.84577,3.67824);
\draw [color=c, fill=c] (8.84577,3.57239) rectangle (8.88557,3.67824);
\draw [color=c, fill=c] (8.88557,3.57239) rectangle (8.92537,3.67824);
\draw [color=c, fill=c] (8.92537,3.57239) rectangle (8.96517,3.67824);
\draw [color=c, fill=c] (8.96517,3.57239) rectangle (9.00498,3.67824);
\draw [color=c, fill=c] (9.00498,3.57239) rectangle (9.04478,3.67824);
\definecolor{c}{rgb}{1,0.466667,0};
\draw [color=c, fill=c] (9.04478,3.57239) rectangle (9.08458,3.67824);
\draw [color=c, fill=c] (9.08458,3.57239) rectangle (9.12438,3.67824);
\draw [color=c, fill=c] (9.12438,3.57239) rectangle (9.16418,3.67824);
\draw [color=c, fill=c] (9.16418,3.57239) rectangle (9.20398,3.67824);
\draw [color=c, fill=c] (9.20398,3.57239) rectangle (9.24378,3.67824);
\draw [color=c, fill=c] (9.24378,3.57239) rectangle (9.28358,3.67824);
\draw [color=c, fill=c] (9.28358,3.57239) rectangle (9.32338,3.67824);
\draw [color=c, fill=c] (9.32338,3.57239) rectangle (9.36318,3.67824);
\draw [color=c, fill=c] (9.36318,3.57239) rectangle (9.40298,3.67824);
\draw [color=c, fill=c] (9.40298,3.57239) rectangle (9.44279,3.67824);
\draw [color=c, fill=c] (9.44279,3.57239) rectangle (9.48259,3.67824);
\draw [color=c, fill=c] (9.48259,3.57239) rectangle (9.52239,3.67824);
\draw [color=c, fill=c] (9.52239,3.57239) rectangle (9.56219,3.67824);
\definecolor{c}{rgb}{1,0.653333,0};
\draw [color=c, fill=c] (9.56219,3.57239) rectangle (9.60199,3.67824);
\draw [color=c, fill=c] (9.60199,3.57239) rectangle (9.64179,3.67824);
\draw [color=c, fill=c] (9.64179,3.57239) rectangle (9.68159,3.67824);
\draw [color=c, fill=c] (9.68159,3.57239) rectangle (9.72139,3.67824);
\draw [color=c, fill=c] (9.72139,3.57239) rectangle (9.76119,3.67824);
\draw [color=c, fill=c] (9.76119,3.57239) rectangle (9.80099,3.67824);
\definecolor{c}{rgb}{1,0.933333,0};
\draw [color=c, fill=c] (9.80099,3.57239) rectangle (9.8408,3.67824);
\draw [color=c, fill=c] (9.8408,3.57239) rectangle (9.8806,3.67824);
\definecolor{c}{rgb}{0.88,1,0};
\draw [color=c, fill=c] (9.8806,3.57239) rectangle (9.9204,3.67824);
\definecolor{c}{rgb}{0.6,1,0};
\draw [color=c, fill=c] (9.9204,3.57239) rectangle (9.9602,3.67824);
\definecolor{c}{rgb}{0.133333,1,0};
\draw [color=c, fill=c] (9.9602,3.57239) rectangle (10,3.67824);
\definecolor{c}{rgb}{0,1,0.0533333};
\draw [color=c, fill=c] (10,3.57239) rectangle (10.0398,3.67824);
\definecolor{c}{rgb}{0,1,0.333333};
\draw [color=c, fill=c] (10.0398,3.57239) rectangle (10.0796,3.67824);
\draw [color=c, fill=c] (10.0796,3.57239) rectangle (10.1194,3.67824);
\draw [color=c, fill=c] (10.1194,3.57239) rectangle (10.1592,3.67824);
\definecolor{c}{rgb}{0,1,0.52};
\draw [color=c, fill=c] (10.1592,3.57239) rectangle (10.199,3.67824);
\draw [color=c, fill=c] (10.199,3.57239) rectangle (10.2388,3.67824);
\draw [color=c, fill=c] (10.2388,3.57239) rectangle (10.2786,3.67824);
\draw [color=c, fill=c] (10.2786,3.57239) rectangle (10.3184,3.67824);
\draw [color=c, fill=c] (10.3184,3.57239) rectangle (10.3582,3.67824);
\draw [color=c, fill=c] (10.3582,3.57239) rectangle (10.398,3.67824);
\definecolor{c}{rgb}{0,1,0.8};
\draw [color=c, fill=c] (10.398,3.57239) rectangle (10.4378,3.67824);
\draw [color=c, fill=c] (10.4378,3.57239) rectangle (10.4776,3.67824);
\draw [color=c, fill=c] (10.4776,3.57239) rectangle (10.5174,3.67824);
\draw [color=c, fill=c] (10.5174,3.57239) rectangle (10.5572,3.67824);
\draw [color=c, fill=c] (10.5572,3.57239) rectangle (10.597,3.67824);
\draw [color=c, fill=c] (10.597,3.57239) rectangle (10.6368,3.67824);
\draw [color=c, fill=c] (10.6368,3.57239) rectangle (10.6766,3.67824);
\draw [color=c, fill=c] (10.6766,3.57239) rectangle (10.7164,3.67824);
\draw [color=c, fill=c] (10.7164,3.57239) rectangle (10.7562,3.67824);
\draw [color=c, fill=c] (10.7562,3.57239) rectangle (10.796,3.67824);
\draw [color=c, fill=c] (10.796,3.57239) rectangle (10.8358,3.67824);
\draw [color=c, fill=c] (10.8358,3.57239) rectangle (10.8756,3.67824);
\draw [color=c, fill=c] (10.8756,3.57239) rectangle (10.9154,3.67824);
\draw [color=c, fill=c] (10.9154,3.57239) rectangle (10.9552,3.67824);
\definecolor{c}{rgb}{0,1,0.986667};
\draw [color=c, fill=c] (10.9552,3.57239) rectangle (10.995,3.67824);
\draw [color=c, fill=c] (10.995,3.57239) rectangle (11.0348,3.67824);
\draw [color=c, fill=c] (11.0348,3.57239) rectangle (11.0746,3.67824);
\draw [color=c, fill=c] (11.0746,3.57239) rectangle (11.1144,3.67824);
\draw [color=c, fill=c] (11.1144,3.57239) rectangle (11.1542,3.67824);
\draw [color=c, fill=c] (11.1542,3.57239) rectangle (11.194,3.67824);
\draw [color=c, fill=c] (11.194,3.57239) rectangle (11.2338,3.67824);
\draw [color=c, fill=c] (11.2338,3.57239) rectangle (11.2736,3.67824);
\draw [color=c, fill=c] (11.2736,3.57239) rectangle (11.3134,3.67824);
\draw [color=c, fill=c] (11.3134,3.57239) rectangle (11.3532,3.67824);
\draw [color=c, fill=c] (11.3532,3.57239) rectangle (11.393,3.67824);
\draw [color=c, fill=c] (11.393,3.57239) rectangle (11.4328,3.67824);
\draw [color=c, fill=c] (11.4328,3.57239) rectangle (11.4726,3.67824);
\draw [color=c, fill=c] (11.4726,3.57239) rectangle (11.5124,3.67824);
\draw [color=c, fill=c] (11.5124,3.57239) rectangle (11.5522,3.67824);
\draw [color=c, fill=c] (11.5522,3.57239) rectangle (11.592,3.67824);
\draw [color=c, fill=c] (11.592,3.57239) rectangle (11.6318,3.67824);
\draw [color=c, fill=c] (11.6318,3.57239) rectangle (11.6716,3.67824);
\draw [color=c, fill=c] (11.6716,3.57239) rectangle (11.7114,3.67824);
\draw [color=c, fill=c] (11.7114,3.57239) rectangle (11.7512,3.67824);
\draw [color=c, fill=c] (11.7512,3.57239) rectangle (11.791,3.67824);
\draw [color=c, fill=c] (11.791,3.57239) rectangle (11.8308,3.67824);
\draw [color=c, fill=c] (11.8308,3.57239) rectangle (11.8706,3.67824);
\draw [color=c, fill=c] (11.8706,3.57239) rectangle (11.9104,3.67824);
\draw [color=c, fill=c] (11.9104,3.57239) rectangle (11.9502,3.67824);
\draw [color=c, fill=c] (11.9502,3.57239) rectangle (11.99,3.67824);
\definecolor{c}{rgb}{0,0.733333,1};
\draw [color=c, fill=c] (11.99,3.57239) rectangle (12.0299,3.67824);
\draw [color=c, fill=c] (12.0299,3.57239) rectangle (12.0697,3.67824);
\draw [color=c, fill=c] (12.0697,3.57239) rectangle (12.1095,3.67824);
\draw [color=c, fill=c] (12.1095,3.57239) rectangle (12.1493,3.67824);
\draw [color=c, fill=c] (12.1493,3.57239) rectangle (12.1891,3.67824);
\draw [color=c, fill=c] (12.1891,3.57239) rectangle (12.2289,3.67824);
\draw [color=c, fill=c] (12.2289,3.57239) rectangle (12.2687,3.67824);
\draw [color=c, fill=c] (12.2687,3.57239) rectangle (12.3085,3.67824);
\draw [color=c, fill=c] (12.3085,3.57239) rectangle (12.3483,3.67824);
\draw [color=c, fill=c] (12.3483,3.57239) rectangle (12.3881,3.67824);
\draw [color=c, fill=c] (12.3881,3.57239) rectangle (12.4279,3.67824);
\draw [color=c, fill=c] (12.4279,3.57239) rectangle (12.4677,3.67824);
\draw [color=c, fill=c] (12.4677,3.57239) rectangle (12.5075,3.67824);
\draw [color=c, fill=c] (12.5075,3.57239) rectangle (12.5473,3.67824);
\draw [color=c, fill=c] (12.5473,3.57239) rectangle (12.5871,3.67824);
\draw [color=c, fill=c] (12.5871,3.57239) rectangle (12.6269,3.67824);
\draw [color=c, fill=c] (12.6269,3.57239) rectangle (12.6667,3.67824);
\draw [color=c, fill=c] (12.6667,3.57239) rectangle (12.7065,3.67824);
\draw [color=c, fill=c] (12.7065,3.57239) rectangle (12.7463,3.67824);
\draw [color=c, fill=c] (12.7463,3.57239) rectangle (12.7861,3.67824);
\draw [color=c, fill=c] (12.7861,3.57239) rectangle (12.8259,3.67824);
\draw [color=c, fill=c] (12.8259,3.57239) rectangle (12.8657,3.67824);
\draw [color=c, fill=c] (12.8657,3.57239) rectangle (12.9055,3.67824);
\draw [color=c, fill=c] (12.9055,3.57239) rectangle (12.9453,3.67824);
\draw [color=c, fill=c] (12.9453,3.57239) rectangle (12.9851,3.67824);
\draw [color=c, fill=c] (12.9851,3.57239) rectangle (13.0249,3.67824);
\draw [color=c, fill=c] (13.0249,3.57239) rectangle (13.0647,3.67824);
\draw [color=c, fill=c] (13.0647,3.57239) rectangle (13.1045,3.67824);
\draw [color=c, fill=c] (13.1045,3.57239) rectangle (13.1443,3.67824);
\draw [color=c, fill=c] (13.1443,3.57239) rectangle (13.1841,3.67824);
\draw [color=c, fill=c] (13.1841,3.57239) rectangle (13.2239,3.67824);
\draw [color=c, fill=c] (13.2239,3.57239) rectangle (13.2637,3.67824);
\draw [color=c, fill=c] (13.2637,3.57239) rectangle (13.3035,3.67824);
\draw [color=c, fill=c] (13.3035,3.57239) rectangle (13.3433,3.67824);
\draw [color=c, fill=c] (13.3433,3.57239) rectangle (13.3831,3.67824);
\draw [color=c, fill=c] (13.3831,3.57239) rectangle (13.4229,3.67824);
\draw [color=c, fill=c] (13.4229,3.57239) rectangle (13.4627,3.67824);
\draw [color=c, fill=c] (13.4627,3.57239) rectangle (13.5025,3.67824);
\draw [color=c, fill=c] (13.5025,3.57239) rectangle (13.5423,3.67824);
\draw [color=c, fill=c] (13.5423,3.57239) rectangle (13.5821,3.67824);
\draw [color=c, fill=c] (13.5821,3.57239) rectangle (13.6219,3.67824);
\draw [color=c, fill=c] (13.6219,3.57239) rectangle (13.6617,3.67824);
\draw [color=c, fill=c] (13.6617,3.57239) rectangle (13.7015,3.67824);
\draw [color=c, fill=c] (13.7015,3.57239) rectangle (13.7413,3.67824);
\draw [color=c, fill=c] (13.7413,3.57239) rectangle (13.7811,3.67824);
\draw [color=c, fill=c] (13.7811,3.57239) rectangle (13.8209,3.67824);
\draw [color=c, fill=c] (13.8209,3.57239) rectangle (13.8607,3.67824);
\draw [color=c, fill=c] (13.8607,3.57239) rectangle (13.9005,3.67824);
\draw [color=c, fill=c] (13.9005,3.57239) rectangle (13.9403,3.67824);
\draw [color=c, fill=c] (13.9403,3.57239) rectangle (13.9801,3.67824);
\draw [color=c, fill=c] (13.9801,3.57239) rectangle (14.0199,3.67824);
\draw [color=c, fill=c] (14.0199,3.57239) rectangle (14.0597,3.67824);
\draw [color=c, fill=c] (14.0597,3.57239) rectangle (14.0995,3.67824);
\draw [color=c, fill=c] (14.0995,3.57239) rectangle (14.1393,3.67824);
\draw [color=c, fill=c] (14.1393,3.57239) rectangle (14.1791,3.67824);
\draw [color=c, fill=c] (14.1791,3.57239) rectangle (14.2189,3.67824);
\draw [color=c, fill=c] (14.2189,3.57239) rectangle (14.2587,3.67824);
\draw [color=c, fill=c] (14.2587,3.57239) rectangle (14.2985,3.67824);
\draw [color=c, fill=c] (14.2985,3.57239) rectangle (14.3383,3.67824);
\draw [color=c, fill=c] (14.3383,3.57239) rectangle (14.3781,3.67824);
\draw [color=c, fill=c] (14.3781,3.57239) rectangle (14.4179,3.67824);
\draw [color=c, fill=c] (14.4179,3.57239) rectangle (14.4577,3.67824);
\draw [color=c, fill=c] (14.4577,3.57239) rectangle (14.4975,3.67824);
\draw [color=c, fill=c] (14.4975,3.57239) rectangle (14.5373,3.67824);
\draw [color=c, fill=c] (14.5373,3.57239) rectangle (14.5771,3.67824);
\draw [color=c, fill=c] (14.5771,3.57239) rectangle (14.6169,3.67824);
\draw [color=c, fill=c] (14.6169,3.57239) rectangle (14.6567,3.67824);
\draw [color=c, fill=c] (14.6567,3.57239) rectangle (14.6965,3.67824);
\draw [color=c, fill=c] (14.6965,3.57239) rectangle (14.7363,3.67824);
\draw [color=c, fill=c] (14.7363,3.57239) rectangle (14.7761,3.67824);
\draw [color=c, fill=c] (14.7761,3.57239) rectangle (14.8159,3.67824);
\draw [color=c, fill=c] (14.8159,3.57239) rectangle (14.8557,3.67824);
\draw [color=c, fill=c] (14.8557,3.57239) rectangle (14.8955,3.67824);
\draw [color=c, fill=c] (14.8955,3.57239) rectangle (14.9353,3.67824);
\draw [color=c, fill=c] (14.9353,3.57239) rectangle (14.9751,3.67824);
\draw [color=c, fill=c] (14.9751,3.57239) rectangle (15.0149,3.67824);
\draw [color=c, fill=c] (15.0149,3.57239) rectangle (15.0547,3.67824);
\draw [color=c, fill=c] (15.0547,3.57239) rectangle (15.0945,3.67824);
\draw [color=c, fill=c] (15.0945,3.57239) rectangle (15.1343,3.67824);
\draw [color=c, fill=c] (15.1343,3.57239) rectangle (15.1741,3.67824);
\draw [color=c, fill=c] (15.1741,3.57239) rectangle (15.2139,3.67824);
\draw [color=c, fill=c] (15.2139,3.57239) rectangle (15.2537,3.67824);
\draw [color=c, fill=c] (15.2537,3.57239) rectangle (15.2935,3.67824);
\draw [color=c, fill=c] (15.2935,3.57239) rectangle (15.3333,3.67824);
\draw [color=c, fill=c] (15.3333,3.57239) rectangle (15.3731,3.67824);
\draw [color=c, fill=c] (15.3731,3.57239) rectangle (15.4129,3.67824);
\draw [color=c, fill=c] (15.4129,3.57239) rectangle (15.4527,3.67824);
\draw [color=c, fill=c] (15.4527,3.57239) rectangle (15.4925,3.67824);
\draw [color=c, fill=c] (15.4925,3.57239) rectangle (15.5323,3.67824);
\draw [color=c, fill=c] (15.5323,3.57239) rectangle (15.5721,3.67824);
\draw [color=c, fill=c] (15.5721,3.57239) rectangle (15.6119,3.67824);
\draw [color=c, fill=c] (15.6119,3.57239) rectangle (15.6517,3.67824);
\draw [color=c, fill=c] (15.6517,3.57239) rectangle (15.6915,3.67824);
\draw [color=c, fill=c] (15.6915,3.57239) rectangle (15.7313,3.67824);
\draw [color=c, fill=c] (15.7313,3.57239) rectangle (15.7711,3.67824);
\draw [color=c, fill=c] (15.7711,3.57239) rectangle (15.8109,3.67824);
\draw [color=c, fill=c] (15.8109,3.57239) rectangle (15.8507,3.67824);
\draw [color=c, fill=c] (15.8507,3.57239) rectangle (15.8905,3.67824);
\draw [color=c, fill=c] (15.8905,3.57239) rectangle (15.9303,3.67824);
\draw [color=c, fill=c] (15.9303,3.57239) rectangle (15.9701,3.67824);
\draw [color=c, fill=c] (15.9701,3.57239) rectangle (16.01,3.67824);
\draw [color=c, fill=c] (16.01,3.57239) rectangle (16.0498,3.67824);
\draw [color=c, fill=c] (16.0498,3.57239) rectangle (16.0896,3.67824);
\draw [color=c, fill=c] (16.0896,3.57239) rectangle (16.1294,3.67824);
\draw [color=c, fill=c] (16.1294,3.57239) rectangle (16.1692,3.67824);
\draw [color=c, fill=c] (16.1692,3.57239) rectangle (16.209,3.67824);
\draw [color=c, fill=c] (16.209,3.57239) rectangle (16.2488,3.67824);
\draw [color=c, fill=c] (16.2488,3.57239) rectangle (16.2886,3.67824);
\draw [color=c, fill=c] (16.2886,3.57239) rectangle (16.3284,3.67824);
\draw [color=c, fill=c] (16.3284,3.57239) rectangle (16.3682,3.67824);
\draw [color=c, fill=c] (16.3682,3.57239) rectangle (16.408,3.67824);
\draw [color=c, fill=c] (16.408,3.57239) rectangle (16.4478,3.67824);
\draw [color=c, fill=c] (16.4478,3.57239) rectangle (16.4876,3.67824);
\draw [color=c, fill=c] (16.4876,3.57239) rectangle (16.5274,3.67824);
\draw [color=c, fill=c] (16.5274,3.57239) rectangle (16.5672,3.67824);
\draw [color=c, fill=c] (16.5672,3.57239) rectangle (16.607,3.67824);
\draw [color=c, fill=c] (16.607,3.57239) rectangle (16.6468,3.67824);
\draw [color=c, fill=c] (16.6468,3.57239) rectangle (16.6866,3.67824);
\draw [color=c, fill=c] (16.6866,3.57239) rectangle (16.7264,3.67824);
\draw [color=c, fill=c] (16.7264,3.57239) rectangle (16.7662,3.67824);
\draw [color=c, fill=c] (16.7662,3.57239) rectangle (16.806,3.67824);
\draw [color=c, fill=c] (16.806,3.57239) rectangle (16.8458,3.67824);
\draw [color=c, fill=c] (16.8458,3.57239) rectangle (16.8856,3.67824);
\draw [color=c, fill=c] (16.8856,3.57239) rectangle (16.9254,3.67824);
\draw [color=c, fill=c] (16.9254,3.57239) rectangle (16.9652,3.67824);
\draw [color=c, fill=c] (16.9652,3.57239) rectangle (17.005,3.67824);
\draw [color=c, fill=c] (17.005,3.57239) rectangle (17.0448,3.67824);
\draw [color=c, fill=c] (17.0448,3.57239) rectangle (17.0846,3.67824);
\draw [color=c, fill=c] (17.0846,3.57239) rectangle (17.1244,3.67824);
\draw [color=c, fill=c] (17.1244,3.57239) rectangle (17.1642,3.67824);
\draw [color=c, fill=c] (17.1642,3.57239) rectangle (17.204,3.67824);
\draw [color=c, fill=c] (17.204,3.57239) rectangle (17.2438,3.67824);
\draw [color=c, fill=c] (17.2438,3.57239) rectangle (17.2836,3.67824);
\draw [color=c, fill=c] (17.2836,3.57239) rectangle (17.3234,3.67824);
\draw [color=c, fill=c] (17.3234,3.57239) rectangle (17.3632,3.67824);
\draw [color=c, fill=c] (17.3632,3.57239) rectangle (17.403,3.67824);
\draw [color=c, fill=c] (17.403,3.57239) rectangle (17.4428,3.67824);
\draw [color=c, fill=c] (17.4428,3.57239) rectangle (17.4826,3.67824);
\draw [color=c, fill=c] (17.4826,3.57239) rectangle (17.5224,3.67824);
\draw [color=c, fill=c] (17.5224,3.57239) rectangle (17.5622,3.67824);
\draw [color=c, fill=c] (17.5622,3.57239) rectangle (17.602,3.67824);
\draw [color=c, fill=c] (17.602,3.57239) rectangle (17.6418,3.67824);
\draw [color=c, fill=c] (17.6418,3.57239) rectangle (17.6816,3.67824);
\draw [color=c, fill=c] (17.6816,3.57239) rectangle (17.7214,3.67824);
\draw [color=c, fill=c] (17.7214,3.57239) rectangle (17.7612,3.67824);
\draw [color=c, fill=c] (17.7612,3.57239) rectangle (17.801,3.67824);
\draw [color=c, fill=c] (17.801,3.57239) rectangle (17.8408,3.67824);
\draw [color=c, fill=c] (17.8408,3.57239) rectangle (17.8806,3.67824);
\draw [color=c, fill=c] (17.8806,3.57239) rectangle (17.9204,3.67824);
\draw [color=c, fill=c] (17.9204,3.57239) rectangle (17.9602,3.67824);
\draw [color=c, fill=c] (17.9602,3.57239) rectangle (18,3.67824);
\definecolor{c}{rgb}{1,0,0};
\draw [color=c, fill=c] (2,3.67824) rectangle (2.0398,3.78409);
\draw [color=c, fill=c] (2.0398,3.67824) rectangle (2.0796,3.78409);
\draw [color=c, fill=c] (2.0796,3.67824) rectangle (2.1194,3.78409);
\draw [color=c, fill=c] (2.1194,3.67824) rectangle (2.1592,3.78409);
\draw [color=c, fill=c] (2.1592,3.67824) rectangle (2.19901,3.78409);
\draw [color=c, fill=c] (2.19901,3.67824) rectangle (2.23881,3.78409);
\draw [color=c, fill=c] (2.23881,3.67824) rectangle (2.27861,3.78409);
\draw [color=c, fill=c] (2.27861,3.67824) rectangle (2.31841,3.78409);
\draw [color=c, fill=c] (2.31841,3.67824) rectangle (2.35821,3.78409);
\draw [color=c, fill=c] (2.35821,3.67824) rectangle (2.39801,3.78409);
\draw [color=c, fill=c] (2.39801,3.67824) rectangle (2.43781,3.78409);
\draw [color=c, fill=c] (2.43781,3.67824) rectangle (2.47761,3.78409);
\draw [color=c, fill=c] (2.47761,3.67824) rectangle (2.51741,3.78409);
\draw [color=c, fill=c] (2.51741,3.67824) rectangle (2.55721,3.78409);
\draw [color=c, fill=c] (2.55721,3.67824) rectangle (2.59702,3.78409);
\draw [color=c, fill=c] (2.59702,3.67824) rectangle (2.63682,3.78409);
\draw [color=c, fill=c] (2.63682,3.67824) rectangle (2.67662,3.78409);
\draw [color=c, fill=c] (2.67662,3.67824) rectangle (2.71642,3.78409);
\draw [color=c, fill=c] (2.71642,3.67824) rectangle (2.75622,3.78409);
\draw [color=c, fill=c] (2.75622,3.67824) rectangle (2.79602,3.78409);
\draw [color=c, fill=c] (2.79602,3.67824) rectangle (2.83582,3.78409);
\draw [color=c, fill=c] (2.83582,3.67824) rectangle (2.87562,3.78409);
\draw [color=c, fill=c] (2.87562,3.67824) rectangle (2.91542,3.78409);
\draw [color=c, fill=c] (2.91542,3.67824) rectangle (2.95522,3.78409);
\draw [color=c, fill=c] (2.95522,3.67824) rectangle (2.99502,3.78409);
\draw [color=c, fill=c] (2.99502,3.67824) rectangle (3.03483,3.78409);
\draw [color=c, fill=c] (3.03483,3.67824) rectangle (3.07463,3.78409);
\draw [color=c, fill=c] (3.07463,3.67824) rectangle (3.11443,3.78409);
\draw [color=c, fill=c] (3.11443,3.67824) rectangle (3.15423,3.78409);
\draw [color=c, fill=c] (3.15423,3.67824) rectangle (3.19403,3.78409);
\draw [color=c, fill=c] (3.19403,3.67824) rectangle (3.23383,3.78409);
\draw [color=c, fill=c] (3.23383,3.67824) rectangle (3.27363,3.78409);
\draw [color=c, fill=c] (3.27363,3.67824) rectangle (3.31343,3.78409);
\draw [color=c, fill=c] (3.31343,3.67824) rectangle (3.35323,3.78409);
\draw [color=c, fill=c] (3.35323,3.67824) rectangle (3.39303,3.78409);
\draw [color=c, fill=c] (3.39303,3.67824) rectangle (3.43284,3.78409);
\draw [color=c, fill=c] (3.43284,3.67824) rectangle (3.47264,3.78409);
\draw [color=c, fill=c] (3.47264,3.67824) rectangle (3.51244,3.78409);
\draw [color=c, fill=c] (3.51244,3.67824) rectangle (3.55224,3.78409);
\draw [color=c, fill=c] (3.55224,3.67824) rectangle (3.59204,3.78409);
\draw [color=c, fill=c] (3.59204,3.67824) rectangle (3.63184,3.78409);
\draw [color=c, fill=c] (3.63184,3.67824) rectangle (3.67164,3.78409);
\draw [color=c, fill=c] (3.67164,3.67824) rectangle (3.71144,3.78409);
\draw [color=c, fill=c] (3.71144,3.67824) rectangle (3.75124,3.78409);
\draw [color=c, fill=c] (3.75124,3.67824) rectangle (3.79104,3.78409);
\draw [color=c, fill=c] (3.79104,3.67824) rectangle (3.83085,3.78409);
\draw [color=c, fill=c] (3.83085,3.67824) rectangle (3.87065,3.78409);
\draw [color=c, fill=c] (3.87065,3.67824) rectangle (3.91045,3.78409);
\draw [color=c, fill=c] (3.91045,3.67824) rectangle (3.95025,3.78409);
\draw [color=c, fill=c] (3.95025,3.67824) rectangle (3.99005,3.78409);
\draw [color=c, fill=c] (3.99005,3.67824) rectangle (4.02985,3.78409);
\draw [color=c, fill=c] (4.02985,3.67824) rectangle (4.06965,3.78409);
\draw [color=c, fill=c] (4.06965,3.67824) rectangle (4.10945,3.78409);
\draw [color=c, fill=c] (4.10945,3.67824) rectangle (4.14925,3.78409);
\draw [color=c, fill=c] (4.14925,3.67824) rectangle (4.18905,3.78409);
\draw [color=c, fill=c] (4.18905,3.67824) rectangle (4.22886,3.78409);
\draw [color=c, fill=c] (4.22886,3.67824) rectangle (4.26866,3.78409);
\draw [color=c, fill=c] (4.26866,3.67824) rectangle (4.30846,3.78409);
\draw [color=c, fill=c] (4.30846,3.67824) rectangle (4.34826,3.78409);
\draw [color=c, fill=c] (4.34826,3.67824) rectangle (4.38806,3.78409);
\draw [color=c, fill=c] (4.38806,3.67824) rectangle (4.42786,3.78409);
\draw [color=c, fill=c] (4.42786,3.67824) rectangle (4.46766,3.78409);
\draw [color=c, fill=c] (4.46766,3.67824) rectangle (4.50746,3.78409);
\draw [color=c, fill=c] (4.50746,3.67824) rectangle (4.54726,3.78409);
\draw [color=c, fill=c] (4.54726,3.67824) rectangle (4.58706,3.78409);
\draw [color=c, fill=c] (4.58706,3.67824) rectangle (4.62687,3.78409);
\draw [color=c, fill=c] (4.62687,3.67824) rectangle (4.66667,3.78409);
\draw [color=c, fill=c] (4.66667,3.67824) rectangle (4.70647,3.78409);
\draw [color=c, fill=c] (4.70647,3.67824) rectangle (4.74627,3.78409);
\draw [color=c, fill=c] (4.74627,3.67824) rectangle (4.78607,3.78409);
\draw [color=c, fill=c] (4.78607,3.67824) rectangle (4.82587,3.78409);
\draw [color=c, fill=c] (4.82587,3.67824) rectangle (4.86567,3.78409);
\draw [color=c, fill=c] (4.86567,3.67824) rectangle (4.90547,3.78409);
\draw [color=c, fill=c] (4.90547,3.67824) rectangle (4.94527,3.78409);
\draw [color=c, fill=c] (4.94527,3.67824) rectangle (4.98507,3.78409);
\draw [color=c, fill=c] (4.98507,3.67824) rectangle (5.02488,3.78409);
\draw [color=c, fill=c] (5.02488,3.67824) rectangle (5.06468,3.78409);
\draw [color=c, fill=c] (5.06468,3.67824) rectangle (5.10448,3.78409);
\draw [color=c, fill=c] (5.10448,3.67824) rectangle (5.14428,3.78409);
\draw [color=c, fill=c] (5.14428,3.67824) rectangle (5.18408,3.78409);
\draw [color=c, fill=c] (5.18408,3.67824) rectangle (5.22388,3.78409);
\draw [color=c, fill=c] (5.22388,3.67824) rectangle (5.26368,3.78409);
\draw [color=c, fill=c] (5.26368,3.67824) rectangle (5.30348,3.78409);
\draw [color=c, fill=c] (5.30348,3.67824) rectangle (5.34328,3.78409);
\draw [color=c, fill=c] (5.34328,3.67824) rectangle (5.38308,3.78409);
\draw [color=c, fill=c] (5.38308,3.67824) rectangle (5.42289,3.78409);
\draw [color=c, fill=c] (5.42289,3.67824) rectangle (5.46269,3.78409);
\draw [color=c, fill=c] (5.46269,3.67824) rectangle (5.50249,3.78409);
\draw [color=c, fill=c] (5.50249,3.67824) rectangle (5.54229,3.78409);
\draw [color=c, fill=c] (5.54229,3.67824) rectangle (5.58209,3.78409);
\draw [color=c, fill=c] (5.58209,3.67824) rectangle (5.62189,3.78409);
\draw [color=c, fill=c] (5.62189,3.67824) rectangle (5.66169,3.78409);
\draw [color=c, fill=c] (5.66169,3.67824) rectangle (5.70149,3.78409);
\draw [color=c, fill=c] (5.70149,3.67824) rectangle (5.74129,3.78409);
\draw [color=c, fill=c] (5.74129,3.67824) rectangle (5.78109,3.78409);
\draw [color=c, fill=c] (5.78109,3.67824) rectangle (5.8209,3.78409);
\draw [color=c, fill=c] (5.8209,3.67824) rectangle (5.8607,3.78409);
\draw [color=c, fill=c] (5.8607,3.67824) rectangle (5.9005,3.78409);
\draw [color=c, fill=c] (5.9005,3.67824) rectangle (5.9403,3.78409);
\draw [color=c, fill=c] (5.9403,3.67824) rectangle (5.9801,3.78409);
\draw [color=c, fill=c] (5.9801,3.67824) rectangle (6.0199,3.78409);
\draw [color=c, fill=c] (6.0199,3.67824) rectangle (6.0597,3.78409);
\draw [color=c, fill=c] (6.0597,3.67824) rectangle (6.0995,3.78409);
\draw [color=c, fill=c] (6.0995,3.67824) rectangle (6.1393,3.78409);
\draw [color=c, fill=c] (6.1393,3.67824) rectangle (6.1791,3.78409);
\draw [color=c, fill=c] (6.1791,3.67824) rectangle (6.21891,3.78409);
\draw [color=c, fill=c] (6.21891,3.67824) rectangle (6.25871,3.78409);
\draw [color=c, fill=c] (6.25871,3.67824) rectangle (6.29851,3.78409);
\draw [color=c, fill=c] (6.29851,3.67824) rectangle (6.33831,3.78409);
\draw [color=c, fill=c] (6.33831,3.67824) rectangle (6.37811,3.78409);
\draw [color=c, fill=c] (6.37811,3.67824) rectangle (6.41791,3.78409);
\draw [color=c, fill=c] (6.41791,3.67824) rectangle (6.45771,3.78409);
\draw [color=c, fill=c] (6.45771,3.67824) rectangle (6.49751,3.78409);
\draw [color=c, fill=c] (6.49751,3.67824) rectangle (6.53731,3.78409);
\draw [color=c, fill=c] (6.53731,3.67824) rectangle (6.57711,3.78409);
\draw [color=c, fill=c] (6.57711,3.67824) rectangle (6.61692,3.78409);
\draw [color=c, fill=c] (6.61692,3.67824) rectangle (6.65672,3.78409);
\draw [color=c, fill=c] (6.65672,3.67824) rectangle (6.69652,3.78409);
\draw [color=c, fill=c] (6.69652,3.67824) rectangle (6.73632,3.78409);
\draw [color=c, fill=c] (6.73632,3.67824) rectangle (6.77612,3.78409);
\draw [color=c, fill=c] (6.77612,3.67824) rectangle (6.81592,3.78409);
\draw [color=c, fill=c] (6.81592,3.67824) rectangle (6.85572,3.78409);
\draw [color=c, fill=c] (6.85572,3.67824) rectangle (6.89552,3.78409);
\draw [color=c, fill=c] (6.89552,3.67824) rectangle (6.93532,3.78409);
\draw [color=c, fill=c] (6.93532,3.67824) rectangle (6.97512,3.78409);
\draw [color=c, fill=c] (6.97512,3.67824) rectangle (7.01493,3.78409);
\draw [color=c, fill=c] (7.01493,3.67824) rectangle (7.05473,3.78409);
\draw [color=c, fill=c] (7.05473,3.67824) rectangle (7.09453,3.78409);
\draw [color=c, fill=c] (7.09453,3.67824) rectangle (7.13433,3.78409);
\draw [color=c, fill=c] (7.13433,3.67824) rectangle (7.17413,3.78409);
\draw [color=c, fill=c] (7.17413,3.67824) rectangle (7.21393,3.78409);
\draw [color=c, fill=c] (7.21393,3.67824) rectangle (7.25373,3.78409);
\draw [color=c, fill=c] (7.25373,3.67824) rectangle (7.29353,3.78409);
\draw [color=c, fill=c] (7.29353,3.67824) rectangle (7.33333,3.78409);
\draw [color=c, fill=c] (7.33333,3.67824) rectangle (7.37313,3.78409);
\draw [color=c, fill=c] (7.37313,3.67824) rectangle (7.41294,3.78409);
\draw [color=c, fill=c] (7.41294,3.67824) rectangle (7.45274,3.78409);
\draw [color=c, fill=c] (7.45274,3.67824) rectangle (7.49254,3.78409);
\draw [color=c, fill=c] (7.49254,3.67824) rectangle (7.53234,3.78409);
\draw [color=c, fill=c] (7.53234,3.67824) rectangle (7.57214,3.78409);
\draw [color=c, fill=c] (7.57214,3.67824) rectangle (7.61194,3.78409);
\draw [color=c, fill=c] (7.61194,3.67824) rectangle (7.65174,3.78409);
\draw [color=c, fill=c] (7.65174,3.67824) rectangle (7.69154,3.78409);
\draw [color=c, fill=c] (7.69154,3.67824) rectangle (7.73134,3.78409);
\draw [color=c, fill=c] (7.73134,3.67824) rectangle (7.77114,3.78409);
\draw [color=c, fill=c] (7.77114,3.67824) rectangle (7.81095,3.78409);
\draw [color=c, fill=c] (7.81095,3.67824) rectangle (7.85075,3.78409);
\draw [color=c, fill=c] (7.85075,3.67824) rectangle (7.89055,3.78409);
\draw [color=c, fill=c] (7.89055,3.67824) rectangle (7.93035,3.78409);
\draw [color=c, fill=c] (7.93035,3.67824) rectangle (7.97015,3.78409);
\draw [color=c, fill=c] (7.97015,3.67824) rectangle (8.00995,3.78409);
\draw [color=c, fill=c] (8.00995,3.67824) rectangle (8.04975,3.78409);
\draw [color=c, fill=c] (8.04975,3.67824) rectangle (8.08955,3.78409);
\draw [color=c, fill=c] (8.08955,3.67824) rectangle (8.12935,3.78409);
\definecolor{c}{rgb}{1,0.186667,0};
\draw [color=c, fill=c] (8.12935,3.67824) rectangle (8.16915,3.78409);
\draw [color=c, fill=c] (8.16915,3.67824) rectangle (8.20895,3.78409);
\draw [color=c, fill=c] (8.20895,3.67824) rectangle (8.24876,3.78409);
\draw [color=c, fill=c] (8.24876,3.67824) rectangle (8.28856,3.78409);
\draw [color=c, fill=c] (8.28856,3.67824) rectangle (8.32836,3.78409);
\draw [color=c, fill=c] (8.32836,3.67824) rectangle (8.36816,3.78409);
\draw [color=c, fill=c] (8.36816,3.67824) rectangle (8.40796,3.78409);
\draw [color=c, fill=c] (8.40796,3.67824) rectangle (8.44776,3.78409);
\draw [color=c, fill=c] (8.44776,3.67824) rectangle (8.48756,3.78409);
\draw [color=c, fill=c] (8.48756,3.67824) rectangle (8.52736,3.78409);
\draw [color=c, fill=c] (8.52736,3.67824) rectangle (8.56716,3.78409);
\draw [color=c, fill=c] (8.56716,3.67824) rectangle (8.60697,3.78409);
\draw [color=c, fill=c] (8.60697,3.67824) rectangle (8.64677,3.78409);
\draw [color=c, fill=c] (8.64677,3.67824) rectangle (8.68657,3.78409);
\draw [color=c, fill=c] (8.68657,3.67824) rectangle (8.72637,3.78409);
\draw [color=c, fill=c] (8.72637,3.67824) rectangle (8.76617,3.78409);
\draw [color=c, fill=c] (8.76617,3.67824) rectangle (8.80597,3.78409);
\draw [color=c, fill=c] (8.80597,3.67824) rectangle (8.84577,3.78409);
\draw [color=c, fill=c] (8.84577,3.67824) rectangle (8.88557,3.78409);
\draw [color=c, fill=c] (8.88557,3.67824) rectangle (8.92537,3.78409);
\draw [color=c, fill=c] (8.92537,3.67824) rectangle (8.96517,3.78409);
\draw [color=c, fill=c] (8.96517,3.67824) rectangle (9.00498,3.78409);
\draw [color=c, fill=c] (9.00498,3.67824) rectangle (9.04478,3.78409);
\draw [color=c, fill=c] (9.04478,3.67824) rectangle (9.08458,3.78409);
\draw [color=c, fill=c] (9.08458,3.67824) rectangle (9.12438,3.78409);
\definecolor{c}{rgb}{1,0.466667,0};
\draw [color=c, fill=c] (9.12438,3.67824) rectangle (9.16418,3.78409);
\draw [color=c, fill=c] (9.16418,3.67824) rectangle (9.20398,3.78409);
\draw [color=c, fill=c] (9.20398,3.67824) rectangle (9.24378,3.78409);
\draw [color=c, fill=c] (9.24378,3.67824) rectangle (9.28358,3.78409);
\draw [color=c, fill=c] (9.28358,3.67824) rectangle (9.32338,3.78409);
\draw [color=c, fill=c] (9.32338,3.67824) rectangle (9.36318,3.78409);
\draw [color=c, fill=c] (9.36318,3.67824) rectangle (9.40298,3.78409);
\draw [color=c, fill=c] (9.40298,3.67824) rectangle (9.44279,3.78409);
\draw [color=c, fill=c] (9.44279,3.67824) rectangle (9.48259,3.78409);
\draw [color=c, fill=c] (9.48259,3.67824) rectangle (9.52239,3.78409);
\draw [color=c, fill=c] (9.52239,3.67824) rectangle (9.56219,3.78409);
\draw [color=c, fill=c] (9.56219,3.67824) rectangle (9.60199,3.78409);
\draw [color=c, fill=c] (9.60199,3.67824) rectangle (9.64179,3.78409);
\draw [color=c, fill=c] (9.64179,3.67824) rectangle (9.68159,3.78409);
\definecolor{c}{rgb}{1,0.653333,0};
\draw [color=c, fill=c] (9.68159,3.67824) rectangle (9.72139,3.78409);
\draw [color=c, fill=c] (9.72139,3.67824) rectangle (9.76119,3.78409);
\draw [color=c, fill=c] (9.76119,3.67824) rectangle (9.80099,3.78409);
\draw [color=c, fill=c] (9.80099,3.67824) rectangle (9.8408,3.78409);
\draw [color=c, fill=c] (9.8408,3.67824) rectangle (9.8806,3.78409);
\definecolor{c}{rgb}{1,0.933333,0};
\draw [color=c, fill=c] (9.8806,3.67824) rectangle (9.9204,3.78409);
\definecolor{c}{rgb}{0.88,1,0};
\draw [color=c, fill=c] (9.9204,3.67824) rectangle (9.9602,3.78409);
\definecolor{c}{rgb}{0,1,0.0533333};
\draw [color=c, fill=c] (9.9602,3.67824) rectangle (10,3.78409);
\definecolor{c}{rgb}{0,1,0.333333};
\draw [color=c, fill=c] (10,3.67824) rectangle (10.0398,3.78409);
\draw [color=c, fill=c] (10.0398,3.67824) rectangle (10.0796,3.78409);
\definecolor{c}{rgb}{0,1,0.52};
\draw [color=c, fill=c] (10.0796,3.67824) rectangle (10.1194,3.78409);
\draw [color=c, fill=c] (10.1194,3.67824) rectangle (10.1592,3.78409);
\draw [color=c, fill=c] (10.1592,3.67824) rectangle (10.199,3.78409);
\draw [color=c, fill=c] (10.199,3.67824) rectangle (10.2388,3.78409);
\draw [color=c, fill=c] (10.2388,3.67824) rectangle (10.2786,3.78409);
\definecolor{c}{rgb}{0,1,0.8};
\draw [color=c, fill=c] (10.2786,3.67824) rectangle (10.3184,3.78409);
\draw [color=c, fill=c] (10.3184,3.67824) rectangle (10.3582,3.78409);
\draw [color=c, fill=c] (10.3582,3.67824) rectangle (10.398,3.78409);
\draw [color=c, fill=c] (10.398,3.67824) rectangle (10.4378,3.78409);
\draw [color=c, fill=c] (10.4378,3.67824) rectangle (10.4776,3.78409);
\draw [color=c, fill=c] (10.4776,3.67824) rectangle (10.5174,3.78409);
\draw [color=c, fill=c] (10.5174,3.67824) rectangle (10.5572,3.78409);
\draw [color=c, fill=c] (10.5572,3.67824) rectangle (10.597,3.78409);
\draw [color=c, fill=c] (10.597,3.67824) rectangle (10.6368,3.78409);
\draw [color=c, fill=c] (10.6368,3.67824) rectangle (10.6766,3.78409);
\draw [color=c, fill=c] (10.6766,3.67824) rectangle (10.7164,3.78409);
\draw [color=c, fill=c] (10.7164,3.67824) rectangle (10.7562,3.78409);
\draw [color=c, fill=c] (10.7562,3.67824) rectangle (10.796,3.78409);
\draw [color=c, fill=c] (10.796,3.67824) rectangle (10.8358,3.78409);
\draw [color=c, fill=c] (10.8358,3.67824) rectangle (10.8756,3.78409);
\definecolor{c}{rgb}{0,1,0.986667};
\draw [color=c, fill=c] (10.8756,3.67824) rectangle (10.9154,3.78409);
\draw [color=c, fill=c] (10.9154,3.67824) rectangle (10.9552,3.78409);
\draw [color=c, fill=c] (10.9552,3.67824) rectangle (10.995,3.78409);
\draw [color=c, fill=c] (10.995,3.67824) rectangle (11.0348,3.78409);
\draw [color=c, fill=c] (11.0348,3.67824) rectangle (11.0746,3.78409);
\draw [color=c, fill=c] (11.0746,3.67824) rectangle (11.1144,3.78409);
\draw [color=c, fill=c] (11.1144,3.67824) rectangle (11.1542,3.78409);
\draw [color=c, fill=c] (11.1542,3.67824) rectangle (11.194,3.78409);
\draw [color=c, fill=c] (11.194,3.67824) rectangle (11.2338,3.78409);
\draw [color=c, fill=c] (11.2338,3.67824) rectangle (11.2736,3.78409);
\draw [color=c, fill=c] (11.2736,3.67824) rectangle (11.3134,3.78409);
\draw [color=c, fill=c] (11.3134,3.67824) rectangle (11.3532,3.78409);
\draw [color=c, fill=c] (11.3532,3.67824) rectangle (11.393,3.78409);
\draw [color=c, fill=c] (11.393,3.67824) rectangle (11.4328,3.78409);
\draw [color=c, fill=c] (11.4328,3.67824) rectangle (11.4726,3.78409);
\draw [color=c, fill=c] (11.4726,3.67824) rectangle (11.5124,3.78409);
\draw [color=c, fill=c] (11.5124,3.67824) rectangle (11.5522,3.78409);
\draw [color=c, fill=c] (11.5522,3.67824) rectangle (11.592,3.78409);
\draw [color=c, fill=c] (11.592,3.67824) rectangle (11.6318,3.78409);
\draw [color=c, fill=c] (11.6318,3.67824) rectangle (11.6716,3.78409);
\draw [color=c, fill=c] (11.6716,3.67824) rectangle (11.7114,3.78409);
\draw [color=c, fill=c] (11.7114,3.67824) rectangle (11.7512,3.78409);
\draw [color=c, fill=c] (11.7512,3.67824) rectangle (11.791,3.78409);
\draw [color=c, fill=c] (11.791,3.67824) rectangle (11.8308,3.78409);
\draw [color=c, fill=c] (11.8308,3.67824) rectangle (11.8706,3.78409);
\draw [color=c, fill=c] (11.8706,3.67824) rectangle (11.9104,3.78409);
\draw [color=c, fill=c] (11.9104,3.67824) rectangle (11.9502,3.78409);
\definecolor{c}{rgb}{0,0.733333,1};
\draw [color=c, fill=c] (11.9502,3.67824) rectangle (11.99,3.78409);
\draw [color=c, fill=c] (11.99,3.67824) rectangle (12.0299,3.78409);
\draw [color=c, fill=c] (12.0299,3.67824) rectangle (12.0697,3.78409);
\draw [color=c, fill=c] (12.0697,3.67824) rectangle (12.1095,3.78409);
\draw [color=c, fill=c] (12.1095,3.67824) rectangle (12.1493,3.78409);
\draw [color=c, fill=c] (12.1493,3.67824) rectangle (12.1891,3.78409);
\draw [color=c, fill=c] (12.1891,3.67824) rectangle (12.2289,3.78409);
\draw [color=c, fill=c] (12.2289,3.67824) rectangle (12.2687,3.78409);
\draw [color=c, fill=c] (12.2687,3.67824) rectangle (12.3085,3.78409);
\draw [color=c, fill=c] (12.3085,3.67824) rectangle (12.3483,3.78409);
\draw [color=c, fill=c] (12.3483,3.67824) rectangle (12.3881,3.78409);
\draw [color=c, fill=c] (12.3881,3.67824) rectangle (12.4279,3.78409);
\draw [color=c, fill=c] (12.4279,3.67824) rectangle (12.4677,3.78409);
\draw [color=c, fill=c] (12.4677,3.67824) rectangle (12.5075,3.78409);
\draw [color=c, fill=c] (12.5075,3.67824) rectangle (12.5473,3.78409);
\draw [color=c, fill=c] (12.5473,3.67824) rectangle (12.5871,3.78409);
\draw [color=c, fill=c] (12.5871,3.67824) rectangle (12.6269,3.78409);
\draw [color=c, fill=c] (12.6269,3.67824) rectangle (12.6667,3.78409);
\draw [color=c, fill=c] (12.6667,3.67824) rectangle (12.7065,3.78409);
\draw [color=c, fill=c] (12.7065,3.67824) rectangle (12.7463,3.78409);
\draw [color=c, fill=c] (12.7463,3.67824) rectangle (12.7861,3.78409);
\draw [color=c, fill=c] (12.7861,3.67824) rectangle (12.8259,3.78409);
\draw [color=c, fill=c] (12.8259,3.67824) rectangle (12.8657,3.78409);
\draw [color=c, fill=c] (12.8657,3.67824) rectangle (12.9055,3.78409);
\draw [color=c, fill=c] (12.9055,3.67824) rectangle (12.9453,3.78409);
\draw [color=c, fill=c] (12.9453,3.67824) rectangle (12.9851,3.78409);
\draw [color=c, fill=c] (12.9851,3.67824) rectangle (13.0249,3.78409);
\draw [color=c, fill=c] (13.0249,3.67824) rectangle (13.0647,3.78409);
\draw [color=c, fill=c] (13.0647,3.67824) rectangle (13.1045,3.78409);
\draw [color=c, fill=c] (13.1045,3.67824) rectangle (13.1443,3.78409);
\draw [color=c, fill=c] (13.1443,3.67824) rectangle (13.1841,3.78409);
\draw [color=c, fill=c] (13.1841,3.67824) rectangle (13.2239,3.78409);
\draw [color=c, fill=c] (13.2239,3.67824) rectangle (13.2637,3.78409);
\draw [color=c, fill=c] (13.2637,3.67824) rectangle (13.3035,3.78409);
\draw [color=c, fill=c] (13.3035,3.67824) rectangle (13.3433,3.78409);
\draw [color=c, fill=c] (13.3433,3.67824) rectangle (13.3831,3.78409);
\draw [color=c, fill=c] (13.3831,3.67824) rectangle (13.4229,3.78409);
\draw [color=c, fill=c] (13.4229,3.67824) rectangle (13.4627,3.78409);
\draw [color=c, fill=c] (13.4627,3.67824) rectangle (13.5025,3.78409);
\draw [color=c, fill=c] (13.5025,3.67824) rectangle (13.5423,3.78409);
\draw [color=c, fill=c] (13.5423,3.67824) rectangle (13.5821,3.78409);
\draw [color=c, fill=c] (13.5821,3.67824) rectangle (13.6219,3.78409);
\draw [color=c, fill=c] (13.6219,3.67824) rectangle (13.6617,3.78409);
\draw [color=c, fill=c] (13.6617,3.67824) rectangle (13.7015,3.78409);
\draw [color=c, fill=c] (13.7015,3.67824) rectangle (13.7413,3.78409);
\draw [color=c, fill=c] (13.7413,3.67824) rectangle (13.7811,3.78409);
\draw [color=c, fill=c] (13.7811,3.67824) rectangle (13.8209,3.78409);
\draw [color=c, fill=c] (13.8209,3.67824) rectangle (13.8607,3.78409);
\draw [color=c, fill=c] (13.8607,3.67824) rectangle (13.9005,3.78409);
\draw [color=c, fill=c] (13.9005,3.67824) rectangle (13.9403,3.78409);
\draw [color=c, fill=c] (13.9403,3.67824) rectangle (13.9801,3.78409);
\draw [color=c, fill=c] (13.9801,3.67824) rectangle (14.0199,3.78409);
\draw [color=c, fill=c] (14.0199,3.67824) rectangle (14.0597,3.78409);
\draw [color=c, fill=c] (14.0597,3.67824) rectangle (14.0995,3.78409);
\draw [color=c, fill=c] (14.0995,3.67824) rectangle (14.1393,3.78409);
\draw [color=c, fill=c] (14.1393,3.67824) rectangle (14.1791,3.78409);
\draw [color=c, fill=c] (14.1791,3.67824) rectangle (14.2189,3.78409);
\draw [color=c, fill=c] (14.2189,3.67824) rectangle (14.2587,3.78409);
\draw [color=c, fill=c] (14.2587,3.67824) rectangle (14.2985,3.78409);
\draw [color=c, fill=c] (14.2985,3.67824) rectangle (14.3383,3.78409);
\draw [color=c, fill=c] (14.3383,3.67824) rectangle (14.3781,3.78409);
\draw [color=c, fill=c] (14.3781,3.67824) rectangle (14.4179,3.78409);
\draw [color=c, fill=c] (14.4179,3.67824) rectangle (14.4577,3.78409);
\draw [color=c, fill=c] (14.4577,3.67824) rectangle (14.4975,3.78409);
\draw [color=c, fill=c] (14.4975,3.67824) rectangle (14.5373,3.78409);
\draw [color=c, fill=c] (14.5373,3.67824) rectangle (14.5771,3.78409);
\draw [color=c, fill=c] (14.5771,3.67824) rectangle (14.6169,3.78409);
\draw [color=c, fill=c] (14.6169,3.67824) rectangle (14.6567,3.78409);
\draw [color=c, fill=c] (14.6567,3.67824) rectangle (14.6965,3.78409);
\draw [color=c, fill=c] (14.6965,3.67824) rectangle (14.7363,3.78409);
\draw [color=c, fill=c] (14.7363,3.67824) rectangle (14.7761,3.78409);
\draw [color=c, fill=c] (14.7761,3.67824) rectangle (14.8159,3.78409);
\draw [color=c, fill=c] (14.8159,3.67824) rectangle (14.8557,3.78409);
\draw [color=c, fill=c] (14.8557,3.67824) rectangle (14.8955,3.78409);
\draw [color=c, fill=c] (14.8955,3.67824) rectangle (14.9353,3.78409);
\draw [color=c, fill=c] (14.9353,3.67824) rectangle (14.9751,3.78409);
\draw [color=c, fill=c] (14.9751,3.67824) rectangle (15.0149,3.78409);
\draw [color=c, fill=c] (15.0149,3.67824) rectangle (15.0547,3.78409);
\draw [color=c, fill=c] (15.0547,3.67824) rectangle (15.0945,3.78409);
\draw [color=c, fill=c] (15.0945,3.67824) rectangle (15.1343,3.78409);
\draw [color=c, fill=c] (15.1343,3.67824) rectangle (15.1741,3.78409);
\draw [color=c, fill=c] (15.1741,3.67824) rectangle (15.2139,3.78409);
\draw [color=c, fill=c] (15.2139,3.67824) rectangle (15.2537,3.78409);
\draw [color=c, fill=c] (15.2537,3.67824) rectangle (15.2935,3.78409);
\draw [color=c, fill=c] (15.2935,3.67824) rectangle (15.3333,3.78409);
\draw [color=c, fill=c] (15.3333,3.67824) rectangle (15.3731,3.78409);
\draw [color=c, fill=c] (15.3731,3.67824) rectangle (15.4129,3.78409);
\draw [color=c, fill=c] (15.4129,3.67824) rectangle (15.4527,3.78409);
\draw [color=c, fill=c] (15.4527,3.67824) rectangle (15.4925,3.78409);
\draw [color=c, fill=c] (15.4925,3.67824) rectangle (15.5323,3.78409);
\draw [color=c, fill=c] (15.5323,3.67824) rectangle (15.5721,3.78409);
\draw [color=c, fill=c] (15.5721,3.67824) rectangle (15.6119,3.78409);
\draw [color=c, fill=c] (15.6119,3.67824) rectangle (15.6517,3.78409);
\draw [color=c, fill=c] (15.6517,3.67824) rectangle (15.6915,3.78409);
\draw [color=c, fill=c] (15.6915,3.67824) rectangle (15.7313,3.78409);
\draw [color=c, fill=c] (15.7313,3.67824) rectangle (15.7711,3.78409);
\draw [color=c, fill=c] (15.7711,3.67824) rectangle (15.8109,3.78409);
\draw [color=c, fill=c] (15.8109,3.67824) rectangle (15.8507,3.78409);
\draw [color=c, fill=c] (15.8507,3.67824) rectangle (15.8905,3.78409);
\draw [color=c, fill=c] (15.8905,3.67824) rectangle (15.9303,3.78409);
\draw [color=c, fill=c] (15.9303,3.67824) rectangle (15.9701,3.78409);
\draw [color=c, fill=c] (15.9701,3.67824) rectangle (16.01,3.78409);
\draw [color=c, fill=c] (16.01,3.67824) rectangle (16.0498,3.78409);
\draw [color=c, fill=c] (16.0498,3.67824) rectangle (16.0896,3.78409);
\draw [color=c, fill=c] (16.0896,3.67824) rectangle (16.1294,3.78409);
\draw [color=c, fill=c] (16.1294,3.67824) rectangle (16.1692,3.78409);
\draw [color=c, fill=c] (16.1692,3.67824) rectangle (16.209,3.78409);
\draw [color=c, fill=c] (16.209,3.67824) rectangle (16.2488,3.78409);
\draw [color=c, fill=c] (16.2488,3.67824) rectangle (16.2886,3.78409);
\draw [color=c, fill=c] (16.2886,3.67824) rectangle (16.3284,3.78409);
\draw [color=c, fill=c] (16.3284,3.67824) rectangle (16.3682,3.78409);
\draw [color=c, fill=c] (16.3682,3.67824) rectangle (16.408,3.78409);
\draw [color=c, fill=c] (16.408,3.67824) rectangle (16.4478,3.78409);
\draw [color=c, fill=c] (16.4478,3.67824) rectangle (16.4876,3.78409);
\draw [color=c, fill=c] (16.4876,3.67824) rectangle (16.5274,3.78409);
\draw [color=c, fill=c] (16.5274,3.67824) rectangle (16.5672,3.78409);
\draw [color=c, fill=c] (16.5672,3.67824) rectangle (16.607,3.78409);
\draw [color=c, fill=c] (16.607,3.67824) rectangle (16.6468,3.78409);
\draw [color=c, fill=c] (16.6468,3.67824) rectangle (16.6866,3.78409);
\draw [color=c, fill=c] (16.6866,3.67824) rectangle (16.7264,3.78409);
\draw [color=c, fill=c] (16.7264,3.67824) rectangle (16.7662,3.78409);
\draw [color=c, fill=c] (16.7662,3.67824) rectangle (16.806,3.78409);
\draw [color=c, fill=c] (16.806,3.67824) rectangle (16.8458,3.78409);
\draw [color=c, fill=c] (16.8458,3.67824) rectangle (16.8856,3.78409);
\draw [color=c, fill=c] (16.8856,3.67824) rectangle (16.9254,3.78409);
\draw [color=c, fill=c] (16.9254,3.67824) rectangle (16.9652,3.78409);
\draw [color=c, fill=c] (16.9652,3.67824) rectangle (17.005,3.78409);
\draw [color=c, fill=c] (17.005,3.67824) rectangle (17.0448,3.78409);
\draw [color=c, fill=c] (17.0448,3.67824) rectangle (17.0846,3.78409);
\draw [color=c, fill=c] (17.0846,3.67824) rectangle (17.1244,3.78409);
\draw [color=c, fill=c] (17.1244,3.67824) rectangle (17.1642,3.78409);
\draw [color=c, fill=c] (17.1642,3.67824) rectangle (17.204,3.78409);
\draw [color=c, fill=c] (17.204,3.67824) rectangle (17.2438,3.78409);
\draw [color=c, fill=c] (17.2438,3.67824) rectangle (17.2836,3.78409);
\draw [color=c, fill=c] (17.2836,3.67824) rectangle (17.3234,3.78409);
\draw [color=c, fill=c] (17.3234,3.67824) rectangle (17.3632,3.78409);
\draw [color=c, fill=c] (17.3632,3.67824) rectangle (17.403,3.78409);
\draw [color=c, fill=c] (17.403,3.67824) rectangle (17.4428,3.78409);
\draw [color=c, fill=c] (17.4428,3.67824) rectangle (17.4826,3.78409);
\draw [color=c, fill=c] (17.4826,3.67824) rectangle (17.5224,3.78409);
\draw [color=c, fill=c] (17.5224,3.67824) rectangle (17.5622,3.78409);
\draw [color=c, fill=c] (17.5622,3.67824) rectangle (17.602,3.78409);
\draw [color=c, fill=c] (17.602,3.67824) rectangle (17.6418,3.78409);
\draw [color=c, fill=c] (17.6418,3.67824) rectangle (17.6816,3.78409);
\draw [color=c, fill=c] (17.6816,3.67824) rectangle (17.7214,3.78409);
\draw [color=c, fill=c] (17.7214,3.67824) rectangle (17.7612,3.78409);
\draw [color=c, fill=c] (17.7612,3.67824) rectangle (17.801,3.78409);
\draw [color=c, fill=c] (17.801,3.67824) rectangle (17.8408,3.78409);
\draw [color=c, fill=c] (17.8408,3.67824) rectangle (17.8806,3.78409);
\draw [color=c, fill=c] (17.8806,3.67824) rectangle (17.9204,3.78409);
\draw [color=c, fill=c] (17.9204,3.67824) rectangle (17.9602,3.78409);
\draw [color=c, fill=c] (17.9602,3.67824) rectangle (18,3.78409);
\definecolor{c}{rgb}{0.133333,1,0};
\draw [color=c, fill=c] (2,3.78409) rectangle (2.0398,3.88994);
\draw [color=c, fill=c] (2.0398,3.78409) rectangle (2.0796,3.88994);
\draw [color=c, fill=c] (2.0796,3.78409) rectangle (2.1194,3.88994);
\draw [color=c, fill=c] (2.1194,3.78409) rectangle (2.1592,3.88994);
\draw [color=c, fill=c] (2.1592,3.78409) rectangle (2.19901,3.88994);
\draw [color=c, fill=c] (2.19901,3.78409) rectangle (2.23881,3.88994);
\draw [color=c, fill=c] (2.23881,3.78409) rectangle (2.27861,3.88994);
\draw [color=c, fill=c] (2.27861,3.78409) rectangle (2.31841,3.88994);
\draw [color=c, fill=c] (2.31841,3.78409) rectangle (2.35821,3.88994);
\draw [color=c, fill=c] (2.35821,3.78409) rectangle (2.39801,3.88994);
\draw [color=c, fill=c] (2.39801,3.78409) rectangle (2.43781,3.88994);
\draw [color=c, fill=c] (2.43781,3.78409) rectangle (2.47761,3.88994);
\draw [color=c, fill=c] (2.47761,3.78409) rectangle (2.51741,3.88994);
\draw [color=c, fill=c] (2.51741,3.78409) rectangle (2.55721,3.88994);
\draw [color=c, fill=c] (2.55721,3.78409) rectangle (2.59702,3.88994);
\draw [color=c, fill=c] (2.59702,3.78409) rectangle (2.63682,3.88994);
\draw [color=c, fill=c] (2.63682,3.78409) rectangle (2.67662,3.88994);
\draw [color=c, fill=c] (2.67662,3.78409) rectangle (2.71642,3.88994);
\draw [color=c, fill=c] (2.71642,3.78409) rectangle (2.75622,3.88994);
\draw [color=c, fill=c] (2.75622,3.78409) rectangle (2.79602,3.88994);
\draw [color=c, fill=c] (2.79602,3.78409) rectangle (2.83582,3.88994);
\draw [color=c, fill=c] (2.83582,3.78409) rectangle (2.87562,3.88994);
\draw [color=c, fill=c] (2.87562,3.78409) rectangle (2.91542,3.88994);
\draw [color=c, fill=c] (2.91542,3.78409) rectangle (2.95522,3.88994);
\draw [color=c, fill=c] (2.95522,3.78409) rectangle (2.99502,3.88994);
\draw [color=c, fill=c] (2.99502,3.78409) rectangle (3.03483,3.88994);
\draw [color=c, fill=c] (3.03483,3.78409) rectangle (3.07463,3.88994);
\draw [color=c, fill=c] (3.07463,3.78409) rectangle (3.11443,3.88994);
\draw [color=c, fill=c] (3.11443,3.78409) rectangle (3.15423,3.88994);
\draw [color=c, fill=c] (3.15423,3.78409) rectangle (3.19403,3.88994);
\draw [color=c, fill=c] (3.19403,3.78409) rectangle (3.23383,3.88994);
\draw [color=c, fill=c] (3.23383,3.78409) rectangle (3.27363,3.88994);
\draw [color=c, fill=c] (3.27363,3.78409) rectangle (3.31343,3.88994);
\draw [color=c, fill=c] (3.31343,3.78409) rectangle (3.35323,3.88994);
\draw [color=c, fill=c] (3.35323,3.78409) rectangle (3.39303,3.88994);
\draw [color=c, fill=c] (3.39303,3.78409) rectangle (3.43284,3.88994);
\draw [color=c, fill=c] (3.43284,3.78409) rectangle (3.47264,3.88994);
\draw [color=c, fill=c] (3.47264,3.78409) rectangle (3.51244,3.88994);
\draw [color=c, fill=c] (3.51244,3.78409) rectangle (3.55224,3.88994);
\draw [color=c, fill=c] (3.55224,3.78409) rectangle (3.59204,3.88994);
\draw [color=c, fill=c] (3.59204,3.78409) rectangle (3.63184,3.88994);
\draw [color=c, fill=c] (3.63184,3.78409) rectangle (3.67164,3.88994);
\draw [color=c, fill=c] (3.67164,3.78409) rectangle (3.71144,3.88994);
\draw [color=c, fill=c] (3.71144,3.78409) rectangle (3.75124,3.88994);
\draw [color=c, fill=c] (3.75124,3.78409) rectangle (3.79104,3.88994);
\draw [color=c, fill=c] (3.79104,3.78409) rectangle (3.83085,3.88994);
\draw [color=c, fill=c] (3.83085,3.78409) rectangle (3.87065,3.88994);
\draw [color=c, fill=c] (3.87065,3.78409) rectangle (3.91045,3.88994);
\draw [color=c, fill=c] (3.91045,3.78409) rectangle (3.95025,3.88994);
\draw [color=c, fill=c] (3.95025,3.78409) rectangle (3.99005,3.88994);
\draw [color=c, fill=c] (3.99005,3.78409) rectangle (4.02985,3.88994);
\draw [color=c, fill=c] (4.02985,3.78409) rectangle (4.06965,3.88994);
\draw [color=c, fill=c] (4.06965,3.78409) rectangle (4.10945,3.88994);
\draw [color=c, fill=c] (4.10945,3.78409) rectangle (4.14925,3.88994);
\draw [color=c, fill=c] (4.14925,3.78409) rectangle (4.18905,3.88994);
\draw [color=c, fill=c] (4.18905,3.78409) rectangle (4.22886,3.88994);
\draw [color=c, fill=c] (4.22886,3.78409) rectangle (4.26866,3.88994);
\draw [color=c, fill=c] (4.26866,3.78409) rectangle (4.30846,3.88994);
\draw [color=c, fill=c] (4.30846,3.78409) rectangle (4.34826,3.88994);
\draw [color=c, fill=c] (4.34826,3.78409) rectangle (4.38806,3.88994);
\draw [color=c, fill=c] (4.38806,3.78409) rectangle (4.42786,3.88994);
\draw [color=c, fill=c] (4.42786,3.78409) rectangle (4.46766,3.88994);
\draw [color=c, fill=c] (4.46766,3.78409) rectangle (4.50746,3.88994);
\draw [color=c, fill=c] (4.50746,3.78409) rectangle (4.54726,3.88994);
\draw [color=c, fill=c] (4.54726,3.78409) rectangle (4.58706,3.88994);
\draw [color=c, fill=c] (4.58706,3.78409) rectangle (4.62687,3.88994);
\draw [color=c, fill=c] (4.62687,3.78409) rectangle (4.66667,3.88994);
\draw [color=c, fill=c] (4.66667,3.78409) rectangle (4.70647,3.88994);
\draw [color=c, fill=c] (4.70647,3.78409) rectangle (4.74627,3.88994);
\draw [color=c, fill=c] (4.74627,3.78409) rectangle (4.78607,3.88994);
\draw [color=c, fill=c] (4.78607,3.78409) rectangle (4.82587,3.88994);
\definecolor{c}{rgb}{0,1,0.0533333};
\draw [color=c, fill=c] (4.82587,3.78409) rectangle (4.86567,3.88994);
\draw [color=c, fill=c] (4.86567,3.78409) rectangle (4.90547,3.88994);
\draw [color=c, fill=c] (4.90547,3.78409) rectangle (4.94527,3.88994);
\draw [color=c, fill=c] (4.94527,3.78409) rectangle (4.98507,3.88994);
\draw [color=c, fill=c] (4.98507,3.78409) rectangle (5.02488,3.88994);
\draw [color=c, fill=c] (5.02488,3.78409) rectangle (5.06468,3.88994);
\draw [color=c, fill=c] (5.06468,3.78409) rectangle (5.10448,3.88994);
\draw [color=c, fill=c] (5.10448,3.78409) rectangle (5.14428,3.88994);
\draw [color=c, fill=c] (5.14428,3.78409) rectangle (5.18408,3.88994);
\draw [color=c, fill=c] (5.18408,3.78409) rectangle (5.22388,3.88994);
\draw [color=c, fill=c] (5.22388,3.78409) rectangle (5.26368,3.88994);
\draw [color=c, fill=c] (5.26368,3.78409) rectangle (5.30348,3.88994);
\draw [color=c, fill=c] (5.30348,3.78409) rectangle (5.34328,3.88994);
\draw [color=c, fill=c] (5.34328,3.78409) rectangle (5.38308,3.88994);
\draw [color=c, fill=c] (5.38308,3.78409) rectangle (5.42289,3.88994);
\draw [color=c, fill=c] (5.42289,3.78409) rectangle (5.46269,3.88994);
\draw [color=c, fill=c] (5.46269,3.78409) rectangle (5.50249,3.88994);
\draw [color=c, fill=c] (5.50249,3.78409) rectangle (5.54229,3.88994);
\draw [color=c, fill=c] (5.54229,3.78409) rectangle (5.58209,3.88994);
\draw [color=c, fill=c] (5.58209,3.78409) rectangle (5.62189,3.88994);
\draw [color=c, fill=c] (5.62189,3.78409) rectangle (5.66169,3.88994);
\draw [color=c, fill=c] (5.66169,3.78409) rectangle (5.70149,3.88994);
\draw [color=c, fill=c] (5.70149,3.78409) rectangle (5.74129,3.88994);
\draw [color=c, fill=c] (5.74129,3.78409) rectangle (5.78109,3.88994);
\draw [color=c, fill=c] (5.78109,3.78409) rectangle (5.8209,3.88994);
\draw [color=c, fill=c] (5.8209,3.78409) rectangle (5.8607,3.88994);
\draw [color=c, fill=c] (5.8607,3.78409) rectangle (5.9005,3.88994);
\draw [color=c, fill=c] (5.9005,3.78409) rectangle (5.9403,3.88994);
\draw [color=c, fill=c] (5.9403,3.78409) rectangle (5.9801,3.88994);
\draw [color=c, fill=c] (5.9801,3.78409) rectangle (6.0199,3.88994);
\draw [color=c, fill=c] (6.0199,3.78409) rectangle (6.0597,3.88994);
\draw [color=c, fill=c] (6.0597,3.78409) rectangle (6.0995,3.88994);
\draw [color=c, fill=c] (6.0995,3.78409) rectangle (6.1393,3.88994);
\draw [color=c, fill=c] (6.1393,3.78409) rectangle (6.1791,3.88994);
\draw [color=c, fill=c] (6.1791,3.78409) rectangle (6.21891,3.88994);
\draw [color=c, fill=c] (6.21891,3.78409) rectangle (6.25871,3.88994);
\draw [color=c, fill=c] (6.25871,3.78409) rectangle (6.29851,3.88994);
\draw [color=c, fill=c] (6.29851,3.78409) rectangle (6.33831,3.88994);
\draw [color=c, fill=c] (6.33831,3.78409) rectangle (6.37811,3.88994);
\draw [color=c, fill=c] (6.37811,3.78409) rectangle (6.41791,3.88994);
\draw [color=c, fill=c] (6.41791,3.78409) rectangle (6.45771,3.88994);
\draw [color=c, fill=c] (6.45771,3.78409) rectangle (6.49751,3.88994);
\draw [color=c, fill=c] (6.49751,3.78409) rectangle (6.53731,3.88994);
\draw [color=c, fill=c] (6.53731,3.78409) rectangle (6.57711,3.88994);
\draw [color=c, fill=c] (6.57711,3.78409) rectangle (6.61692,3.88994);
\draw [color=c, fill=c] (6.61692,3.78409) rectangle (6.65672,3.88994);
\draw [color=c, fill=c] (6.65672,3.78409) rectangle (6.69652,3.88994);
\draw [color=c, fill=c] (6.69652,3.78409) rectangle (6.73632,3.88994);
\draw [color=c, fill=c] (6.73632,3.78409) rectangle (6.77612,3.88994);
\draw [color=c, fill=c] (6.77612,3.78409) rectangle (6.81592,3.88994);
\draw [color=c, fill=c] (6.81592,3.78409) rectangle (6.85572,3.88994);
\draw [color=c, fill=c] (6.85572,3.78409) rectangle (6.89552,3.88994);
\draw [color=c, fill=c] (6.89552,3.78409) rectangle (6.93532,3.88994);
\draw [color=c, fill=c] (6.93532,3.78409) rectangle (6.97512,3.88994);
\draw [color=c, fill=c] (6.97512,3.78409) rectangle (7.01493,3.88994);
\draw [color=c, fill=c] (7.01493,3.78409) rectangle (7.05473,3.88994);
\draw [color=c, fill=c] (7.05473,3.78409) rectangle (7.09453,3.88994);
\draw [color=c, fill=c] (7.09453,3.78409) rectangle (7.13433,3.88994);
\draw [color=c, fill=c] (7.13433,3.78409) rectangle (7.17413,3.88994);
\draw [color=c, fill=c] (7.17413,3.78409) rectangle (7.21393,3.88994);
\draw [color=c, fill=c] (7.21393,3.78409) rectangle (7.25373,3.88994);
\draw [color=c, fill=c] (7.25373,3.78409) rectangle (7.29353,3.88994);
\draw [color=c, fill=c] (7.29353,3.78409) rectangle (7.33333,3.88994);
\draw [color=c, fill=c] (7.33333,3.78409) rectangle (7.37313,3.88994);
\draw [color=c, fill=c] (7.37313,3.78409) rectangle (7.41294,3.88994);
\draw [color=c, fill=c] (7.41294,3.78409) rectangle (7.45274,3.88994);
\draw [color=c, fill=c] (7.45274,3.78409) rectangle (7.49254,3.88994);
\draw [color=c, fill=c] (7.49254,3.78409) rectangle (7.53234,3.88994);
\draw [color=c, fill=c] (7.53234,3.78409) rectangle (7.57214,3.88994);
\draw [color=c, fill=c] (7.57214,3.78409) rectangle (7.61194,3.88994);
\draw [color=c, fill=c] (7.61194,3.78409) rectangle (7.65174,3.88994);
\draw [color=c, fill=c] (7.65174,3.78409) rectangle (7.69154,3.88994);
\draw [color=c, fill=c] (7.69154,3.78409) rectangle (7.73134,3.88994);
\draw [color=c, fill=c] (7.73134,3.78409) rectangle (7.77114,3.88994);
\draw [color=c, fill=c] (7.77114,3.78409) rectangle (7.81095,3.88994);
\draw [color=c, fill=c] (7.81095,3.78409) rectangle (7.85075,3.88994);
\draw [color=c, fill=c] (7.85075,3.78409) rectangle (7.89055,3.88994);
\draw [color=c, fill=c] (7.89055,3.78409) rectangle (7.93035,3.88994);
\draw [color=c, fill=c] (7.93035,3.78409) rectangle (7.97015,3.88994);
\draw [color=c, fill=c] (7.97015,3.78409) rectangle (8.00995,3.88994);
\draw [color=c, fill=c] (8.00995,3.78409) rectangle (8.04975,3.88994);
\draw [color=c, fill=c] (8.04975,3.78409) rectangle (8.08955,3.88994);
\draw [color=c, fill=c] (8.08955,3.78409) rectangle (8.12935,3.88994);
\draw [color=c, fill=c] (8.12935,3.78409) rectangle (8.16915,3.88994);
\draw [color=c, fill=c] (8.16915,3.78409) rectangle (8.20895,3.88994);
\draw [color=c, fill=c] (8.20895,3.78409) rectangle (8.24876,3.88994);
\draw [color=c, fill=c] (8.24876,3.78409) rectangle (8.28856,3.88994);
\draw [color=c, fill=c] (8.28856,3.78409) rectangle (8.32836,3.88994);
\definecolor{c}{rgb}{0,1,0.333333};
\draw [color=c, fill=c] (8.32836,3.78409) rectangle (8.36816,3.88994);
\draw [color=c, fill=c] (8.36816,3.78409) rectangle (8.40796,3.88994);
\draw [color=c, fill=c] (8.40796,3.78409) rectangle (8.44776,3.88994);
\draw [color=c, fill=c] (8.44776,3.78409) rectangle (8.48756,3.88994);
\draw [color=c, fill=c] (8.48756,3.78409) rectangle (8.52736,3.88994);
\draw [color=c, fill=c] (8.52736,3.78409) rectangle (8.56716,3.88994);
\draw [color=c, fill=c] (8.56716,3.78409) rectangle (8.60697,3.88994);
\draw [color=c, fill=c] (8.60697,3.78409) rectangle (8.64677,3.88994);
\draw [color=c, fill=c] (8.64677,3.78409) rectangle (8.68657,3.88994);
\draw [color=c, fill=c] (8.68657,3.78409) rectangle (8.72637,3.88994);
\draw [color=c, fill=c] (8.72637,3.78409) rectangle (8.76617,3.88994);
\draw [color=c, fill=c] (8.76617,3.78409) rectangle (8.80597,3.88994);
\draw [color=c, fill=c] (8.80597,3.78409) rectangle (8.84577,3.88994);
\draw [color=c, fill=c] (8.84577,3.78409) rectangle (8.88557,3.88994);
\draw [color=c, fill=c] (8.88557,3.78409) rectangle (8.92537,3.88994);
\draw [color=c, fill=c] (8.92537,3.78409) rectangle (8.96517,3.88994);
\draw [color=c, fill=c] (8.96517,3.78409) rectangle (9.00498,3.88994);
\draw [color=c, fill=c] (9.00498,3.78409) rectangle (9.04478,3.88994);
\draw [color=c, fill=c] (9.04478,3.78409) rectangle (9.08458,3.88994);
\draw [color=c, fill=c] (9.08458,3.78409) rectangle (9.12438,3.88994);
\draw [color=c, fill=c] (9.12438,3.78409) rectangle (9.16418,3.88994);
\draw [color=c, fill=c] (9.16418,3.78409) rectangle (9.20398,3.88994);
\draw [color=c, fill=c] (9.20398,3.78409) rectangle (9.24378,3.88994);
\draw [color=c, fill=c] (9.24378,3.78409) rectangle (9.28358,3.88994);
\draw [color=c, fill=c] (9.28358,3.78409) rectangle (9.32338,3.88994);
\definecolor{c}{rgb}{0,1,0.52};
\draw [color=c, fill=c] (9.32338,3.78409) rectangle (9.36318,3.88994);
\draw [color=c, fill=c] (9.36318,3.78409) rectangle (9.40298,3.88994);
\draw [color=c, fill=c] (9.40298,3.78409) rectangle (9.44279,3.88994);
\draw [color=c, fill=c] (9.44279,3.78409) rectangle (9.48259,3.88994);
\draw [color=c, fill=c] (9.48259,3.78409) rectangle (9.52239,3.88994);
\draw [color=c, fill=c] (9.52239,3.78409) rectangle (9.56219,3.88994);
\draw [color=c, fill=c] (9.56219,3.78409) rectangle (9.60199,3.88994);
\draw [color=c, fill=c] (9.60199,3.78409) rectangle (9.64179,3.88994);
\draw [color=c, fill=c] (9.64179,3.78409) rectangle (9.68159,3.88994);
\draw [color=c, fill=c] (9.68159,3.78409) rectangle (9.72139,3.88994);
\draw [color=c, fill=c] (9.72139,3.78409) rectangle (9.76119,3.88994);
\draw [color=c, fill=c] (9.76119,3.78409) rectangle (9.80099,3.88994);
\draw [color=c, fill=c] (9.80099,3.78409) rectangle (9.8408,3.88994);
\draw [color=c, fill=c] (9.8408,3.78409) rectangle (9.8806,3.88994);
\draw [color=c, fill=c] (9.8806,3.78409) rectangle (9.9204,3.88994);
\draw [color=c, fill=c] (9.9204,3.78409) rectangle (9.9602,3.88994);
\draw [color=c, fill=c] (9.9602,3.78409) rectangle (10,3.88994);
\draw [color=c, fill=c] (10,3.78409) rectangle (10.0398,3.88994);
\definecolor{c}{rgb}{0,1,0.8};
\draw [color=c, fill=c] (10.0398,3.78409) rectangle (10.0796,3.88994);
\draw [color=c, fill=c] (10.0796,3.78409) rectangle (10.1194,3.88994);
\draw [color=c, fill=c] (10.1194,3.78409) rectangle (10.1592,3.88994);
\draw [color=c, fill=c] (10.1592,3.78409) rectangle (10.199,3.88994);
\draw [color=c, fill=c] (10.199,3.78409) rectangle (10.2388,3.88994);
\draw [color=c, fill=c] (10.2388,3.78409) rectangle (10.2786,3.88994);
\draw [color=c, fill=c] (10.2786,3.78409) rectangle (10.3184,3.88994);
\draw [color=c, fill=c] (10.3184,3.78409) rectangle (10.3582,3.88994);
\draw [color=c, fill=c] (10.3582,3.78409) rectangle (10.398,3.88994);
\draw [color=c, fill=c] (10.398,3.78409) rectangle (10.4378,3.88994);
\draw [color=c, fill=c] (10.4378,3.78409) rectangle (10.4776,3.88994);
\draw [color=c, fill=c] (10.4776,3.78409) rectangle (10.5174,3.88994);
\draw [color=c, fill=c] (10.5174,3.78409) rectangle (10.5572,3.88994);
\draw [color=c, fill=c] (10.5572,3.78409) rectangle (10.597,3.88994);
\draw [color=c, fill=c] (10.597,3.78409) rectangle (10.6368,3.88994);
\draw [color=c, fill=c] (10.6368,3.78409) rectangle (10.6766,3.88994);
\draw [color=c, fill=c] (10.6766,3.78409) rectangle (10.7164,3.88994);
\draw [color=c, fill=c] (10.7164,3.78409) rectangle (10.7562,3.88994);
\draw [color=c, fill=c] (10.7562,3.78409) rectangle (10.796,3.88994);
\definecolor{c}{rgb}{0,1,0.986667};
\draw [color=c, fill=c] (10.796,3.78409) rectangle (10.8358,3.88994);
\draw [color=c, fill=c] (10.8358,3.78409) rectangle (10.8756,3.88994);
\draw [color=c, fill=c] (10.8756,3.78409) rectangle (10.9154,3.88994);
\draw [color=c, fill=c] (10.9154,3.78409) rectangle (10.9552,3.88994);
\draw [color=c, fill=c] (10.9552,3.78409) rectangle (10.995,3.88994);
\draw [color=c, fill=c] (10.995,3.78409) rectangle (11.0348,3.88994);
\draw [color=c, fill=c] (11.0348,3.78409) rectangle (11.0746,3.88994);
\draw [color=c, fill=c] (11.0746,3.78409) rectangle (11.1144,3.88994);
\draw [color=c, fill=c] (11.1144,3.78409) rectangle (11.1542,3.88994);
\draw [color=c, fill=c] (11.1542,3.78409) rectangle (11.194,3.88994);
\draw [color=c, fill=c] (11.194,3.78409) rectangle (11.2338,3.88994);
\draw [color=c, fill=c] (11.2338,3.78409) rectangle (11.2736,3.88994);
\draw [color=c, fill=c] (11.2736,3.78409) rectangle (11.3134,3.88994);
\draw [color=c, fill=c] (11.3134,3.78409) rectangle (11.3532,3.88994);
\draw [color=c, fill=c] (11.3532,3.78409) rectangle (11.393,3.88994);
\draw [color=c, fill=c] (11.393,3.78409) rectangle (11.4328,3.88994);
\draw [color=c, fill=c] (11.4328,3.78409) rectangle (11.4726,3.88994);
\draw [color=c, fill=c] (11.4726,3.78409) rectangle (11.5124,3.88994);
\draw [color=c, fill=c] (11.5124,3.78409) rectangle (11.5522,3.88994);
\draw [color=c, fill=c] (11.5522,3.78409) rectangle (11.592,3.88994);
\draw [color=c, fill=c] (11.592,3.78409) rectangle (11.6318,3.88994);
\draw [color=c, fill=c] (11.6318,3.78409) rectangle (11.6716,3.88994);
\draw [color=c, fill=c] (11.6716,3.78409) rectangle (11.7114,3.88994);
\draw [color=c, fill=c] (11.7114,3.78409) rectangle (11.7512,3.88994);
\draw [color=c, fill=c] (11.7512,3.78409) rectangle (11.791,3.88994);
\draw [color=c, fill=c] (11.791,3.78409) rectangle (11.8308,3.88994);
\draw [color=c, fill=c] (11.8308,3.78409) rectangle (11.8706,3.88994);
\draw [color=c, fill=c] (11.8706,3.78409) rectangle (11.9104,3.88994);
\definecolor{c}{rgb}{0,0.733333,1};
\draw [color=c, fill=c] (11.9104,3.78409) rectangle (11.9502,3.88994);
\draw [color=c, fill=c] (11.9502,3.78409) rectangle (11.99,3.88994);
\draw [color=c, fill=c] (11.99,3.78409) rectangle (12.0299,3.88994);
\draw [color=c, fill=c] (12.0299,3.78409) rectangle (12.0697,3.88994);
\draw [color=c, fill=c] (12.0697,3.78409) rectangle (12.1095,3.88994);
\draw [color=c, fill=c] (12.1095,3.78409) rectangle (12.1493,3.88994);
\draw [color=c, fill=c] (12.1493,3.78409) rectangle (12.1891,3.88994);
\draw [color=c, fill=c] (12.1891,3.78409) rectangle (12.2289,3.88994);
\draw [color=c, fill=c] (12.2289,3.78409) rectangle (12.2687,3.88994);
\draw [color=c, fill=c] (12.2687,3.78409) rectangle (12.3085,3.88994);
\draw [color=c, fill=c] (12.3085,3.78409) rectangle (12.3483,3.88994);
\draw [color=c, fill=c] (12.3483,3.78409) rectangle (12.3881,3.88994);
\draw [color=c, fill=c] (12.3881,3.78409) rectangle (12.4279,3.88994);
\draw [color=c, fill=c] (12.4279,3.78409) rectangle (12.4677,3.88994);
\draw [color=c, fill=c] (12.4677,3.78409) rectangle (12.5075,3.88994);
\draw [color=c, fill=c] (12.5075,3.78409) rectangle (12.5473,3.88994);
\draw [color=c, fill=c] (12.5473,3.78409) rectangle (12.5871,3.88994);
\draw [color=c, fill=c] (12.5871,3.78409) rectangle (12.6269,3.88994);
\draw [color=c, fill=c] (12.6269,3.78409) rectangle (12.6667,3.88994);
\draw [color=c, fill=c] (12.6667,3.78409) rectangle (12.7065,3.88994);
\draw [color=c, fill=c] (12.7065,3.78409) rectangle (12.7463,3.88994);
\draw [color=c, fill=c] (12.7463,3.78409) rectangle (12.7861,3.88994);
\draw [color=c, fill=c] (12.7861,3.78409) rectangle (12.8259,3.88994);
\draw [color=c, fill=c] (12.8259,3.78409) rectangle (12.8657,3.88994);
\draw [color=c, fill=c] (12.8657,3.78409) rectangle (12.9055,3.88994);
\draw [color=c, fill=c] (12.9055,3.78409) rectangle (12.9453,3.88994);
\draw [color=c, fill=c] (12.9453,3.78409) rectangle (12.9851,3.88994);
\draw [color=c, fill=c] (12.9851,3.78409) rectangle (13.0249,3.88994);
\draw [color=c, fill=c] (13.0249,3.78409) rectangle (13.0647,3.88994);
\draw [color=c, fill=c] (13.0647,3.78409) rectangle (13.1045,3.88994);
\draw [color=c, fill=c] (13.1045,3.78409) rectangle (13.1443,3.88994);
\draw [color=c, fill=c] (13.1443,3.78409) rectangle (13.1841,3.88994);
\draw [color=c, fill=c] (13.1841,3.78409) rectangle (13.2239,3.88994);
\draw [color=c, fill=c] (13.2239,3.78409) rectangle (13.2637,3.88994);
\draw [color=c, fill=c] (13.2637,3.78409) rectangle (13.3035,3.88994);
\draw [color=c, fill=c] (13.3035,3.78409) rectangle (13.3433,3.88994);
\draw [color=c, fill=c] (13.3433,3.78409) rectangle (13.3831,3.88994);
\draw [color=c, fill=c] (13.3831,3.78409) rectangle (13.4229,3.88994);
\draw [color=c, fill=c] (13.4229,3.78409) rectangle (13.4627,3.88994);
\draw [color=c, fill=c] (13.4627,3.78409) rectangle (13.5025,3.88994);
\draw [color=c, fill=c] (13.5025,3.78409) rectangle (13.5423,3.88994);
\draw [color=c, fill=c] (13.5423,3.78409) rectangle (13.5821,3.88994);
\draw [color=c, fill=c] (13.5821,3.78409) rectangle (13.6219,3.88994);
\draw [color=c, fill=c] (13.6219,3.78409) rectangle (13.6617,3.88994);
\draw [color=c, fill=c] (13.6617,3.78409) rectangle (13.7015,3.88994);
\draw [color=c, fill=c] (13.7015,3.78409) rectangle (13.7413,3.88994);
\draw [color=c, fill=c] (13.7413,3.78409) rectangle (13.7811,3.88994);
\draw [color=c, fill=c] (13.7811,3.78409) rectangle (13.8209,3.88994);
\draw [color=c, fill=c] (13.8209,3.78409) rectangle (13.8607,3.88994);
\draw [color=c, fill=c] (13.8607,3.78409) rectangle (13.9005,3.88994);
\draw [color=c, fill=c] (13.9005,3.78409) rectangle (13.9403,3.88994);
\draw [color=c, fill=c] (13.9403,3.78409) rectangle (13.9801,3.88994);
\draw [color=c, fill=c] (13.9801,3.78409) rectangle (14.0199,3.88994);
\draw [color=c, fill=c] (14.0199,3.78409) rectangle (14.0597,3.88994);
\draw [color=c, fill=c] (14.0597,3.78409) rectangle (14.0995,3.88994);
\draw [color=c, fill=c] (14.0995,3.78409) rectangle (14.1393,3.88994);
\draw [color=c, fill=c] (14.1393,3.78409) rectangle (14.1791,3.88994);
\draw [color=c, fill=c] (14.1791,3.78409) rectangle (14.2189,3.88994);
\draw [color=c, fill=c] (14.2189,3.78409) rectangle (14.2587,3.88994);
\draw [color=c, fill=c] (14.2587,3.78409) rectangle (14.2985,3.88994);
\draw [color=c, fill=c] (14.2985,3.78409) rectangle (14.3383,3.88994);
\draw [color=c, fill=c] (14.3383,3.78409) rectangle (14.3781,3.88994);
\draw [color=c, fill=c] (14.3781,3.78409) rectangle (14.4179,3.88994);
\draw [color=c, fill=c] (14.4179,3.78409) rectangle (14.4577,3.88994);
\draw [color=c, fill=c] (14.4577,3.78409) rectangle (14.4975,3.88994);
\draw [color=c, fill=c] (14.4975,3.78409) rectangle (14.5373,3.88994);
\draw [color=c, fill=c] (14.5373,3.78409) rectangle (14.5771,3.88994);
\draw [color=c, fill=c] (14.5771,3.78409) rectangle (14.6169,3.88994);
\draw [color=c, fill=c] (14.6169,3.78409) rectangle (14.6567,3.88994);
\draw [color=c, fill=c] (14.6567,3.78409) rectangle (14.6965,3.88994);
\draw [color=c, fill=c] (14.6965,3.78409) rectangle (14.7363,3.88994);
\draw [color=c, fill=c] (14.7363,3.78409) rectangle (14.7761,3.88994);
\draw [color=c, fill=c] (14.7761,3.78409) rectangle (14.8159,3.88994);
\draw [color=c, fill=c] (14.8159,3.78409) rectangle (14.8557,3.88994);
\draw [color=c, fill=c] (14.8557,3.78409) rectangle (14.8955,3.88994);
\draw [color=c, fill=c] (14.8955,3.78409) rectangle (14.9353,3.88994);
\draw [color=c, fill=c] (14.9353,3.78409) rectangle (14.9751,3.88994);
\draw [color=c, fill=c] (14.9751,3.78409) rectangle (15.0149,3.88994);
\draw [color=c, fill=c] (15.0149,3.78409) rectangle (15.0547,3.88994);
\draw [color=c, fill=c] (15.0547,3.78409) rectangle (15.0945,3.88994);
\draw [color=c, fill=c] (15.0945,3.78409) rectangle (15.1343,3.88994);
\draw [color=c, fill=c] (15.1343,3.78409) rectangle (15.1741,3.88994);
\draw [color=c, fill=c] (15.1741,3.78409) rectangle (15.2139,3.88994);
\draw [color=c, fill=c] (15.2139,3.78409) rectangle (15.2537,3.88994);
\draw [color=c, fill=c] (15.2537,3.78409) rectangle (15.2935,3.88994);
\draw [color=c, fill=c] (15.2935,3.78409) rectangle (15.3333,3.88994);
\draw [color=c, fill=c] (15.3333,3.78409) rectangle (15.3731,3.88994);
\draw [color=c, fill=c] (15.3731,3.78409) rectangle (15.4129,3.88994);
\draw [color=c, fill=c] (15.4129,3.78409) rectangle (15.4527,3.88994);
\draw [color=c, fill=c] (15.4527,3.78409) rectangle (15.4925,3.88994);
\draw [color=c, fill=c] (15.4925,3.78409) rectangle (15.5323,3.88994);
\draw [color=c, fill=c] (15.5323,3.78409) rectangle (15.5721,3.88994);
\draw [color=c, fill=c] (15.5721,3.78409) rectangle (15.6119,3.88994);
\draw [color=c, fill=c] (15.6119,3.78409) rectangle (15.6517,3.88994);
\draw [color=c, fill=c] (15.6517,3.78409) rectangle (15.6915,3.88994);
\draw [color=c, fill=c] (15.6915,3.78409) rectangle (15.7313,3.88994);
\draw [color=c, fill=c] (15.7313,3.78409) rectangle (15.7711,3.88994);
\draw [color=c, fill=c] (15.7711,3.78409) rectangle (15.8109,3.88994);
\draw [color=c, fill=c] (15.8109,3.78409) rectangle (15.8507,3.88994);
\draw [color=c, fill=c] (15.8507,3.78409) rectangle (15.8905,3.88994);
\draw [color=c, fill=c] (15.8905,3.78409) rectangle (15.9303,3.88994);
\draw [color=c, fill=c] (15.9303,3.78409) rectangle (15.9701,3.88994);
\draw [color=c, fill=c] (15.9701,3.78409) rectangle (16.01,3.88994);
\draw [color=c, fill=c] (16.01,3.78409) rectangle (16.0498,3.88994);
\draw [color=c, fill=c] (16.0498,3.78409) rectangle (16.0896,3.88994);
\draw [color=c, fill=c] (16.0896,3.78409) rectangle (16.1294,3.88994);
\draw [color=c, fill=c] (16.1294,3.78409) rectangle (16.1692,3.88994);
\draw [color=c, fill=c] (16.1692,3.78409) rectangle (16.209,3.88994);
\draw [color=c, fill=c] (16.209,3.78409) rectangle (16.2488,3.88994);
\draw [color=c, fill=c] (16.2488,3.78409) rectangle (16.2886,3.88994);
\draw [color=c, fill=c] (16.2886,3.78409) rectangle (16.3284,3.88994);
\draw [color=c, fill=c] (16.3284,3.78409) rectangle (16.3682,3.88994);
\draw [color=c, fill=c] (16.3682,3.78409) rectangle (16.408,3.88994);
\draw [color=c, fill=c] (16.408,3.78409) rectangle (16.4478,3.88994);
\draw [color=c, fill=c] (16.4478,3.78409) rectangle (16.4876,3.88994);
\draw [color=c, fill=c] (16.4876,3.78409) rectangle (16.5274,3.88994);
\draw [color=c, fill=c] (16.5274,3.78409) rectangle (16.5672,3.88994);
\draw [color=c, fill=c] (16.5672,3.78409) rectangle (16.607,3.88994);
\draw [color=c, fill=c] (16.607,3.78409) rectangle (16.6468,3.88994);
\draw [color=c, fill=c] (16.6468,3.78409) rectangle (16.6866,3.88994);
\draw [color=c, fill=c] (16.6866,3.78409) rectangle (16.7264,3.88994);
\draw [color=c, fill=c] (16.7264,3.78409) rectangle (16.7662,3.88994);
\draw [color=c, fill=c] (16.7662,3.78409) rectangle (16.806,3.88994);
\draw [color=c, fill=c] (16.806,3.78409) rectangle (16.8458,3.88994);
\draw [color=c, fill=c] (16.8458,3.78409) rectangle (16.8856,3.88994);
\draw [color=c, fill=c] (16.8856,3.78409) rectangle (16.9254,3.88994);
\draw [color=c, fill=c] (16.9254,3.78409) rectangle (16.9652,3.88994);
\draw [color=c, fill=c] (16.9652,3.78409) rectangle (17.005,3.88994);
\draw [color=c, fill=c] (17.005,3.78409) rectangle (17.0448,3.88994);
\draw [color=c, fill=c] (17.0448,3.78409) rectangle (17.0846,3.88994);
\draw [color=c, fill=c] (17.0846,3.78409) rectangle (17.1244,3.88994);
\draw [color=c, fill=c] (17.1244,3.78409) rectangle (17.1642,3.88994);
\draw [color=c, fill=c] (17.1642,3.78409) rectangle (17.204,3.88994);
\draw [color=c, fill=c] (17.204,3.78409) rectangle (17.2438,3.88994);
\draw [color=c, fill=c] (17.2438,3.78409) rectangle (17.2836,3.88994);
\draw [color=c, fill=c] (17.2836,3.78409) rectangle (17.3234,3.88994);
\draw [color=c, fill=c] (17.3234,3.78409) rectangle (17.3632,3.88994);
\draw [color=c, fill=c] (17.3632,3.78409) rectangle (17.403,3.88994);
\draw [color=c, fill=c] (17.403,3.78409) rectangle (17.4428,3.88994);
\draw [color=c, fill=c] (17.4428,3.78409) rectangle (17.4826,3.88994);
\draw [color=c, fill=c] (17.4826,3.78409) rectangle (17.5224,3.88994);
\draw [color=c, fill=c] (17.5224,3.78409) rectangle (17.5622,3.88994);
\draw [color=c, fill=c] (17.5622,3.78409) rectangle (17.602,3.88994);
\draw [color=c, fill=c] (17.602,3.78409) rectangle (17.6418,3.88994);
\draw [color=c, fill=c] (17.6418,3.78409) rectangle (17.6816,3.88994);
\draw [color=c, fill=c] (17.6816,3.78409) rectangle (17.7214,3.88994);
\draw [color=c, fill=c] (17.7214,3.78409) rectangle (17.7612,3.88994);
\draw [color=c, fill=c] (17.7612,3.78409) rectangle (17.801,3.88994);
\draw [color=c, fill=c] (17.801,3.78409) rectangle (17.8408,3.88994);
\draw [color=c, fill=c] (17.8408,3.78409) rectangle (17.8806,3.88994);
\draw [color=c, fill=c] (17.8806,3.78409) rectangle (17.9204,3.88994);
\draw [color=c, fill=c] (17.9204,3.78409) rectangle (17.9602,3.88994);
\draw [color=c, fill=c] (17.9602,3.78409) rectangle (18,3.88994);
\definecolor{c}{rgb}{0,0.0800001,1};
\draw [color=c, fill=c] (2,3.88994) rectangle (2.0398,3.99579);
\draw [color=c, fill=c] (2.0398,3.88994) rectangle (2.0796,3.99579);
\draw [color=c, fill=c] (2.0796,3.88994) rectangle (2.1194,3.99579);
\draw [color=c, fill=c] (2.1194,3.88994) rectangle (2.1592,3.99579);
\draw [color=c, fill=c] (2.1592,3.88994) rectangle (2.19901,3.99579);
\draw [color=c, fill=c] (2.19901,3.88994) rectangle (2.23881,3.99579);
\draw [color=c, fill=c] (2.23881,3.88994) rectangle (2.27861,3.99579);
\draw [color=c, fill=c] (2.27861,3.88994) rectangle (2.31841,3.99579);
\draw [color=c, fill=c] (2.31841,3.88994) rectangle (2.35821,3.99579);
\draw [color=c, fill=c] (2.35821,3.88994) rectangle (2.39801,3.99579);
\draw [color=c, fill=c] (2.39801,3.88994) rectangle (2.43781,3.99579);
\draw [color=c, fill=c] (2.43781,3.88994) rectangle (2.47761,3.99579);
\draw [color=c, fill=c] (2.47761,3.88994) rectangle (2.51741,3.99579);
\draw [color=c, fill=c] (2.51741,3.88994) rectangle (2.55721,3.99579);
\draw [color=c, fill=c] (2.55721,3.88994) rectangle (2.59702,3.99579);
\draw [color=c, fill=c] (2.59702,3.88994) rectangle (2.63682,3.99579);
\draw [color=c, fill=c] (2.63682,3.88994) rectangle (2.67662,3.99579);
\draw [color=c, fill=c] (2.67662,3.88994) rectangle (2.71642,3.99579);
\draw [color=c, fill=c] (2.71642,3.88994) rectangle (2.75622,3.99579);
\draw [color=c, fill=c] (2.75622,3.88994) rectangle (2.79602,3.99579);
\draw [color=c, fill=c] (2.79602,3.88994) rectangle (2.83582,3.99579);
\draw [color=c, fill=c] (2.83582,3.88994) rectangle (2.87562,3.99579);
\draw [color=c, fill=c] (2.87562,3.88994) rectangle (2.91542,3.99579);
\draw [color=c, fill=c] (2.91542,3.88994) rectangle (2.95522,3.99579);
\draw [color=c, fill=c] (2.95522,3.88994) rectangle (2.99502,3.99579);
\draw [color=c, fill=c] (2.99502,3.88994) rectangle (3.03483,3.99579);
\draw [color=c, fill=c] (3.03483,3.88994) rectangle (3.07463,3.99579);
\draw [color=c, fill=c] (3.07463,3.88994) rectangle (3.11443,3.99579);
\draw [color=c, fill=c] (3.11443,3.88994) rectangle (3.15423,3.99579);
\draw [color=c, fill=c] (3.15423,3.88994) rectangle (3.19403,3.99579);
\draw [color=c, fill=c] (3.19403,3.88994) rectangle (3.23383,3.99579);
\draw [color=c, fill=c] (3.23383,3.88994) rectangle (3.27363,3.99579);
\draw [color=c, fill=c] (3.27363,3.88994) rectangle (3.31343,3.99579);
\draw [color=c, fill=c] (3.31343,3.88994) rectangle (3.35323,3.99579);
\draw [color=c, fill=c] (3.35323,3.88994) rectangle (3.39303,3.99579);
\draw [color=c, fill=c] (3.39303,3.88994) rectangle (3.43284,3.99579);
\draw [color=c, fill=c] (3.43284,3.88994) rectangle (3.47264,3.99579);
\draw [color=c, fill=c] (3.47264,3.88994) rectangle (3.51244,3.99579);
\draw [color=c, fill=c] (3.51244,3.88994) rectangle (3.55224,3.99579);
\draw [color=c, fill=c] (3.55224,3.88994) rectangle (3.59204,3.99579);
\draw [color=c, fill=c] (3.59204,3.88994) rectangle (3.63184,3.99579);
\draw [color=c, fill=c] (3.63184,3.88994) rectangle (3.67164,3.99579);
\draw [color=c, fill=c] (3.67164,3.88994) rectangle (3.71144,3.99579);
\draw [color=c, fill=c] (3.71144,3.88994) rectangle (3.75124,3.99579);
\draw [color=c, fill=c] (3.75124,3.88994) rectangle (3.79104,3.99579);
\draw [color=c, fill=c] (3.79104,3.88994) rectangle (3.83085,3.99579);
\draw [color=c, fill=c] (3.83085,3.88994) rectangle (3.87065,3.99579);
\draw [color=c, fill=c] (3.87065,3.88994) rectangle (3.91045,3.99579);
\draw [color=c, fill=c] (3.91045,3.88994) rectangle (3.95025,3.99579);
\draw [color=c, fill=c] (3.95025,3.88994) rectangle (3.99005,3.99579);
\draw [color=c, fill=c] (3.99005,3.88994) rectangle (4.02985,3.99579);
\draw [color=c, fill=c] (4.02985,3.88994) rectangle (4.06965,3.99579);
\draw [color=c, fill=c] (4.06965,3.88994) rectangle (4.10945,3.99579);
\draw [color=c, fill=c] (4.10945,3.88994) rectangle (4.14925,3.99579);
\draw [color=c, fill=c] (4.14925,3.88994) rectangle (4.18905,3.99579);
\draw [color=c, fill=c] (4.18905,3.88994) rectangle (4.22886,3.99579);
\draw [color=c, fill=c] (4.22886,3.88994) rectangle (4.26866,3.99579);
\draw [color=c, fill=c] (4.26866,3.88994) rectangle (4.30846,3.99579);
\draw [color=c, fill=c] (4.30846,3.88994) rectangle (4.34826,3.99579);
\draw [color=c, fill=c] (4.34826,3.88994) rectangle (4.38806,3.99579);
\draw [color=c, fill=c] (4.38806,3.88994) rectangle (4.42786,3.99579);
\draw [color=c, fill=c] (4.42786,3.88994) rectangle (4.46766,3.99579);
\draw [color=c, fill=c] (4.46766,3.88994) rectangle (4.50746,3.99579);
\draw [color=c, fill=c] (4.50746,3.88994) rectangle (4.54726,3.99579);
\draw [color=c, fill=c] (4.54726,3.88994) rectangle (4.58706,3.99579);
\draw [color=c, fill=c] (4.58706,3.88994) rectangle (4.62687,3.99579);
\draw [color=c, fill=c] (4.62687,3.88994) rectangle (4.66667,3.99579);
\draw [color=c, fill=c] (4.66667,3.88994) rectangle (4.70647,3.99579);
\draw [color=c, fill=c] (4.70647,3.88994) rectangle (4.74627,3.99579);
\draw [color=c, fill=c] (4.74627,3.88994) rectangle (4.78607,3.99579);
\draw [color=c, fill=c] (4.78607,3.88994) rectangle (4.82587,3.99579);
\draw [color=c, fill=c] (4.82587,3.88994) rectangle (4.86567,3.99579);
\draw [color=c, fill=c] (4.86567,3.88994) rectangle (4.90547,3.99579);
\draw [color=c, fill=c] (4.90547,3.88994) rectangle (4.94527,3.99579);
\draw [color=c, fill=c] (4.94527,3.88994) rectangle (4.98507,3.99579);
\draw [color=c, fill=c] (4.98507,3.88994) rectangle (5.02488,3.99579);
\draw [color=c, fill=c] (5.02488,3.88994) rectangle (5.06468,3.99579);
\draw [color=c, fill=c] (5.06468,3.88994) rectangle (5.10448,3.99579);
\draw [color=c, fill=c] (5.10448,3.88994) rectangle (5.14428,3.99579);
\draw [color=c, fill=c] (5.14428,3.88994) rectangle (5.18408,3.99579);
\draw [color=c, fill=c] (5.18408,3.88994) rectangle (5.22388,3.99579);
\draw [color=c, fill=c] (5.22388,3.88994) rectangle (5.26368,3.99579);
\draw [color=c, fill=c] (5.26368,3.88994) rectangle (5.30348,3.99579);
\draw [color=c, fill=c] (5.30348,3.88994) rectangle (5.34328,3.99579);
\draw [color=c, fill=c] (5.34328,3.88994) rectangle (5.38308,3.99579);
\draw [color=c, fill=c] (5.38308,3.88994) rectangle (5.42289,3.99579);
\draw [color=c, fill=c] (5.42289,3.88994) rectangle (5.46269,3.99579);
\draw [color=c, fill=c] (5.46269,3.88994) rectangle (5.50249,3.99579);
\draw [color=c, fill=c] (5.50249,3.88994) rectangle (5.54229,3.99579);
\draw [color=c, fill=c] (5.54229,3.88994) rectangle (5.58209,3.99579);
\draw [color=c, fill=c] (5.58209,3.88994) rectangle (5.62189,3.99579);
\draw [color=c, fill=c] (5.62189,3.88994) rectangle (5.66169,3.99579);
\draw [color=c, fill=c] (5.66169,3.88994) rectangle (5.70149,3.99579);
\draw [color=c, fill=c] (5.70149,3.88994) rectangle (5.74129,3.99579);
\draw [color=c, fill=c] (5.74129,3.88994) rectangle (5.78109,3.99579);
\draw [color=c, fill=c] (5.78109,3.88994) rectangle (5.8209,3.99579);
\draw [color=c, fill=c] (5.8209,3.88994) rectangle (5.8607,3.99579);
\draw [color=c, fill=c] (5.8607,3.88994) rectangle (5.9005,3.99579);
\definecolor{c}{rgb}{0.2,0,1};
\draw [color=c, fill=c] (5.9005,3.88994) rectangle (5.9403,3.99579);
\draw [color=c, fill=c] (5.9403,3.88994) rectangle (5.9801,3.99579);
\draw [color=c, fill=c] (5.9801,3.88994) rectangle (6.0199,3.99579);
\draw [color=c, fill=c] (6.0199,3.88994) rectangle (6.0597,3.99579);
\draw [color=c, fill=c] (6.0597,3.88994) rectangle (6.0995,3.99579);
\draw [color=c, fill=c] (6.0995,3.88994) rectangle (6.1393,3.99579);
\draw [color=c, fill=c] (6.1393,3.88994) rectangle (6.1791,3.99579);
\draw [color=c, fill=c] (6.1791,3.88994) rectangle (6.21891,3.99579);
\draw [color=c, fill=c] (6.21891,3.88994) rectangle (6.25871,3.99579);
\draw [color=c, fill=c] (6.25871,3.88994) rectangle (6.29851,3.99579);
\draw [color=c, fill=c] (6.29851,3.88994) rectangle (6.33831,3.99579);
\draw [color=c, fill=c] (6.33831,3.88994) rectangle (6.37811,3.99579);
\draw [color=c, fill=c] (6.37811,3.88994) rectangle (6.41791,3.99579);
\draw [color=c, fill=c] (6.41791,3.88994) rectangle (6.45771,3.99579);
\draw [color=c, fill=c] (6.45771,3.88994) rectangle (6.49751,3.99579);
\draw [color=c, fill=c] (6.49751,3.88994) rectangle (6.53731,3.99579);
\draw [color=c, fill=c] (6.53731,3.88994) rectangle (6.57711,3.99579);
\draw [color=c, fill=c] (6.57711,3.88994) rectangle (6.61692,3.99579);
\draw [color=c, fill=c] (6.61692,3.88994) rectangle (6.65672,3.99579);
\draw [color=c, fill=c] (6.65672,3.88994) rectangle (6.69652,3.99579);
\draw [color=c, fill=c] (6.69652,3.88994) rectangle (6.73632,3.99579);
\draw [color=c, fill=c] (6.73632,3.88994) rectangle (6.77612,3.99579);
\draw [color=c, fill=c] (6.77612,3.88994) rectangle (6.81592,3.99579);
\draw [color=c, fill=c] (6.81592,3.88994) rectangle (6.85572,3.99579);
\draw [color=c, fill=c] (6.85572,3.88994) rectangle (6.89552,3.99579);
\draw [color=c, fill=c] (6.89552,3.88994) rectangle (6.93532,3.99579);
\draw [color=c, fill=c] (6.93532,3.88994) rectangle (6.97512,3.99579);
\draw [color=c, fill=c] (6.97512,3.88994) rectangle (7.01493,3.99579);
\draw [color=c, fill=c] (7.01493,3.88994) rectangle (7.05473,3.99579);
\draw [color=c, fill=c] (7.05473,3.88994) rectangle (7.09453,3.99579);
\draw [color=c, fill=c] (7.09453,3.88994) rectangle (7.13433,3.99579);
\draw [color=c, fill=c] (7.13433,3.88994) rectangle (7.17413,3.99579);
\draw [color=c, fill=c] (7.17413,3.88994) rectangle (7.21393,3.99579);
\draw [color=c, fill=c] (7.21393,3.88994) rectangle (7.25373,3.99579);
\draw [color=c, fill=c] (7.25373,3.88994) rectangle (7.29353,3.99579);
\draw [color=c, fill=c] (7.29353,3.88994) rectangle (7.33333,3.99579);
\draw [color=c, fill=c] (7.33333,3.88994) rectangle (7.37313,3.99579);
\draw [color=c, fill=c] (7.37313,3.88994) rectangle (7.41294,3.99579);
\draw [color=c, fill=c] (7.41294,3.88994) rectangle (7.45274,3.99579);
\draw [color=c, fill=c] (7.45274,3.88994) rectangle (7.49254,3.99579);
\draw [color=c, fill=c] (7.49254,3.88994) rectangle (7.53234,3.99579);
\draw [color=c, fill=c] (7.53234,3.88994) rectangle (7.57214,3.99579);
\draw [color=c, fill=c] (7.57214,3.88994) rectangle (7.61194,3.99579);
\draw [color=c, fill=c] (7.61194,3.88994) rectangle (7.65174,3.99579);
\draw [color=c, fill=c] (7.65174,3.88994) rectangle (7.69154,3.99579);
\draw [color=c, fill=c] (7.69154,3.88994) rectangle (7.73134,3.99579);
\draw [color=c, fill=c] (7.73134,3.88994) rectangle (7.77114,3.99579);
\draw [color=c, fill=c] (7.77114,3.88994) rectangle (7.81095,3.99579);
\draw [color=c, fill=c] (7.81095,3.88994) rectangle (7.85075,3.99579);
\draw [color=c, fill=c] (7.85075,3.88994) rectangle (7.89055,3.99579);
\draw [color=c, fill=c] (7.89055,3.88994) rectangle (7.93035,3.99579);
\draw [color=c, fill=c] (7.93035,3.88994) rectangle (7.97015,3.99579);
\draw [color=c, fill=c] (7.97015,3.88994) rectangle (8.00995,3.99579);
\draw [color=c, fill=c] (8.00995,3.88994) rectangle (8.04975,3.99579);
\draw [color=c, fill=c] (8.04975,3.88994) rectangle (8.08955,3.99579);
\draw [color=c, fill=c] (8.08955,3.88994) rectangle (8.12935,3.99579);
\draw [color=c, fill=c] (8.12935,3.88994) rectangle (8.16915,3.99579);
\draw [color=c, fill=c] (8.16915,3.88994) rectangle (8.20895,3.99579);
\draw [color=c, fill=c] (8.20895,3.88994) rectangle (8.24876,3.99579);
\draw [color=c, fill=c] (8.24876,3.88994) rectangle (8.28856,3.99579);
\draw [color=c, fill=c] (8.28856,3.88994) rectangle (8.32836,3.99579);
\draw [color=c, fill=c] (8.32836,3.88994) rectangle (8.36816,3.99579);
\draw [color=c, fill=c] (8.36816,3.88994) rectangle (8.40796,3.99579);
\draw [color=c, fill=c] (8.40796,3.88994) rectangle (8.44776,3.99579);
\draw [color=c, fill=c] (8.44776,3.88994) rectangle (8.48756,3.99579);
\draw [color=c, fill=c] (8.48756,3.88994) rectangle (8.52736,3.99579);
\definecolor{c}{rgb}{0.386667,0,1};
\draw [color=c, fill=c] (8.52736,3.88994) rectangle (8.56716,3.99579);
\draw [color=c, fill=c] (8.56716,3.88994) rectangle (8.60697,3.99579);
\draw [color=c, fill=c] (8.60697,3.88994) rectangle (8.64677,3.99579);
\draw [color=c, fill=c] (8.64677,3.88994) rectangle (8.68657,3.99579);
\draw [color=c, fill=c] (8.68657,3.88994) rectangle (8.72637,3.99579);
\draw [color=c, fill=c] (8.72637,3.88994) rectangle (8.76617,3.99579);
\draw [color=c, fill=c] (8.76617,3.88994) rectangle (8.80597,3.99579);
\draw [color=c, fill=c] (8.80597,3.88994) rectangle (8.84577,3.99579);
\draw [color=c, fill=c] (8.84577,3.88994) rectangle (8.88557,3.99579);
\draw [color=c, fill=c] (8.88557,3.88994) rectangle (8.92537,3.99579);
\draw [color=c, fill=c] (8.92537,3.88994) rectangle (8.96517,3.99579);
\draw [color=c, fill=c] (8.96517,3.88994) rectangle (9.00498,3.99579);
\draw [color=c, fill=c] (9.00498,3.88994) rectangle (9.04478,3.99579);
\draw [color=c, fill=c] (9.04478,3.88994) rectangle (9.08458,3.99579);
\draw [color=c, fill=c] (9.08458,3.88994) rectangle (9.12438,3.99579);
\draw [color=c, fill=c] (9.12438,3.88994) rectangle (9.16418,3.99579);
\draw [color=c, fill=c] (9.16418,3.88994) rectangle (9.20398,3.99579);
\draw [color=c, fill=c] (9.20398,3.88994) rectangle (9.24378,3.99579);
\draw [color=c, fill=c] (9.24378,3.88994) rectangle (9.28358,3.99579);
\draw [color=c, fill=c] (9.28358,3.88994) rectangle (9.32338,3.99579);
\draw [color=c, fill=c] (9.32338,3.88994) rectangle (9.36318,3.99579);
\draw [color=c, fill=c] (9.36318,3.88994) rectangle (9.40298,3.99579);
\draw [color=c, fill=c] (9.40298,3.88994) rectangle (9.44279,3.99579);
\draw [color=c, fill=c] (9.44279,3.88994) rectangle (9.48259,3.99579);
\draw [color=c, fill=c] (9.48259,3.88994) rectangle (9.52239,3.99579);
\draw [color=c, fill=c] (9.52239,3.88994) rectangle (9.56219,3.99579);
\draw [color=c, fill=c] (9.56219,3.88994) rectangle (9.60199,3.99579);
\draw [color=c, fill=c] (9.60199,3.88994) rectangle (9.64179,3.99579);
\draw [color=c, fill=c] (9.64179,3.88994) rectangle (9.68159,3.99579);
\draw [color=c, fill=c] (9.68159,3.88994) rectangle (9.72139,3.99579);
\draw [color=c, fill=c] (9.72139,3.88994) rectangle (9.76119,3.99579);
\draw [color=c, fill=c] (9.76119,3.88994) rectangle (9.80099,3.99579);
\draw [color=c, fill=c] (9.80099,3.88994) rectangle (9.8408,3.99579);
\draw [color=c, fill=c] (9.8408,3.88994) rectangle (9.8806,3.99579);
\draw [color=c, fill=c] (9.8806,3.88994) rectangle (9.9204,3.99579);
\definecolor{c}{rgb}{0,0.0800001,1};
\draw [color=c, fill=c] (9.9204,3.88994) rectangle (9.9602,3.99579);
\definecolor{c}{rgb}{0,0.733333,1};
\draw [color=c, fill=c] (9.9602,3.88994) rectangle (10,3.99579);
\definecolor{c}{rgb}{0,1,0.986667};
\draw [color=c, fill=c] (10,3.88994) rectangle (10.0398,3.99579);
\draw [color=c, fill=c] (10.0398,3.88994) rectangle (10.0796,3.99579);
\definecolor{c}{rgb}{0,1,0.8};
\draw [color=c, fill=c] (10.0796,3.88994) rectangle (10.1194,3.99579);
\draw [color=c, fill=c] (10.1194,3.88994) rectangle (10.1592,3.99579);
\draw [color=c, fill=c] (10.1592,3.88994) rectangle (10.199,3.99579);
\draw [color=c, fill=c] (10.199,3.88994) rectangle (10.2388,3.99579);
\draw [color=c, fill=c] (10.2388,3.88994) rectangle (10.2786,3.99579);
\draw [color=c, fill=c] (10.2786,3.88994) rectangle (10.3184,3.99579);
\draw [color=c, fill=c] (10.3184,3.88994) rectangle (10.3582,3.99579);
\draw [color=c, fill=c] (10.3582,3.88994) rectangle (10.398,3.99579);
\draw [color=c, fill=c] (10.398,3.88994) rectangle (10.4378,3.99579);
\draw [color=c, fill=c] (10.4378,3.88994) rectangle (10.4776,3.99579);
\draw [color=c, fill=c] (10.4776,3.88994) rectangle (10.5174,3.99579);
\draw [color=c, fill=c] (10.5174,3.88994) rectangle (10.5572,3.99579);
\draw [color=c, fill=c] (10.5572,3.88994) rectangle (10.597,3.99579);
\draw [color=c, fill=c] (10.597,3.88994) rectangle (10.6368,3.99579);
\draw [color=c, fill=c] (10.6368,3.88994) rectangle (10.6766,3.99579);
\definecolor{c}{rgb}{0,1,0.986667};
\draw [color=c, fill=c] (10.6766,3.88994) rectangle (10.7164,3.99579);
\draw [color=c, fill=c] (10.7164,3.88994) rectangle (10.7562,3.99579);
\draw [color=c, fill=c] (10.7562,3.88994) rectangle (10.796,3.99579);
\draw [color=c, fill=c] (10.796,3.88994) rectangle (10.8358,3.99579);
\draw [color=c, fill=c] (10.8358,3.88994) rectangle (10.8756,3.99579);
\draw [color=c, fill=c] (10.8756,3.88994) rectangle (10.9154,3.99579);
\draw [color=c, fill=c] (10.9154,3.88994) rectangle (10.9552,3.99579);
\draw [color=c, fill=c] (10.9552,3.88994) rectangle (10.995,3.99579);
\draw [color=c, fill=c] (10.995,3.88994) rectangle (11.0348,3.99579);
\draw [color=c, fill=c] (11.0348,3.88994) rectangle (11.0746,3.99579);
\draw [color=c, fill=c] (11.0746,3.88994) rectangle (11.1144,3.99579);
\draw [color=c, fill=c] (11.1144,3.88994) rectangle (11.1542,3.99579);
\draw [color=c, fill=c] (11.1542,3.88994) rectangle (11.194,3.99579);
\draw [color=c, fill=c] (11.194,3.88994) rectangle (11.2338,3.99579);
\draw [color=c, fill=c] (11.2338,3.88994) rectangle (11.2736,3.99579);
\draw [color=c, fill=c] (11.2736,3.88994) rectangle (11.3134,3.99579);
\draw [color=c, fill=c] (11.3134,3.88994) rectangle (11.3532,3.99579);
\draw [color=c, fill=c] (11.3532,3.88994) rectangle (11.393,3.99579);
\draw [color=c, fill=c] (11.393,3.88994) rectangle (11.4328,3.99579);
\draw [color=c, fill=c] (11.4328,3.88994) rectangle (11.4726,3.99579);
\draw [color=c, fill=c] (11.4726,3.88994) rectangle (11.5124,3.99579);
\draw [color=c, fill=c] (11.5124,3.88994) rectangle (11.5522,3.99579);
\draw [color=c, fill=c] (11.5522,3.88994) rectangle (11.592,3.99579);
\draw [color=c, fill=c] (11.592,3.88994) rectangle (11.6318,3.99579);
\draw [color=c, fill=c] (11.6318,3.88994) rectangle (11.6716,3.99579);
\draw [color=c, fill=c] (11.6716,3.88994) rectangle (11.7114,3.99579);
\draw [color=c, fill=c] (11.7114,3.88994) rectangle (11.7512,3.99579);
\draw [color=c, fill=c] (11.7512,3.88994) rectangle (11.791,3.99579);
\draw [color=c, fill=c] (11.791,3.88994) rectangle (11.8308,3.99579);
\draw [color=c, fill=c] (11.8308,3.88994) rectangle (11.8706,3.99579);
\definecolor{c}{rgb}{0,0.733333,1};
\draw [color=c, fill=c] (11.8706,3.88994) rectangle (11.9104,3.99579);
\draw [color=c, fill=c] (11.9104,3.88994) rectangle (11.9502,3.99579);
\draw [color=c, fill=c] (11.9502,3.88994) rectangle (11.99,3.99579);
\draw [color=c, fill=c] (11.99,3.88994) rectangle (12.0299,3.99579);
\draw [color=c, fill=c] (12.0299,3.88994) rectangle (12.0697,3.99579);
\draw [color=c, fill=c] (12.0697,3.88994) rectangle (12.1095,3.99579);
\draw [color=c, fill=c] (12.1095,3.88994) rectangle (12.1493,3.99579);
\draw [color=c, fill=c] (12.1493,3.88994) rectangle (12.1891,3.99579);
\draw [color=c, fill=c] (12.1891,3.88994) rectangle (12.2289,3.99579);
\draw [color=c, fill=c] (12.2289,3.88994) rectangle (12.2687,3.99579);
\draw [color=c, fill=c] (12.2687,3.88994) rectangle (12.3085,3.99579);
\draw [color=c, fill=c] (12.3085,3.88994) rectangle (12.3483,3.99579);
\draw [color=c, fill=c] (12.3483,3.88994) rectangle (12.3881,3.99579);
\draw [color=c, fill=c] (12.3881,3.88994) rectangle (12.4279,3.99579);
\draw [color=c, fill=c] (12.4279,3.88994) rectangle (12.4677,3.99579);
\draw [color=c, fill=c] (12.4677,3.88994) rectangle (12.5075,3.99579);
\draw [color=c, fill=c] (12.5075,3.88994) rectangle (12.5473,3.99579);
\draw [color=c, fill=c] (12.5473,3.88994) rectangle (12.5871,3.99579);
\draw [color=c, fill=c] (12.5871,3.88994) rectangle (12.6269,3.99579);
\draw [color=c, fill=c] (12.6269,3.88994) rectangle (12.6667,3.99579);
\draw [color=c, fill=c] (12.6667,3.88994) rectangle (12.7065,3.99579);
\draw [color=c, fill=c] (12.7065,3.88994) rectangle (12.7463,3.99579);
\draw [color=c, fill=c] (12.7463,3.88994) rectangle (12.7861,3.99579);
\draw [color=c, fill=c] (12.7861,3.88994) rectangle (12.8259,3.99579);
\draw [color=c, fill=c] (12.8259,3.88994) rectangle (12.8657,3.99579);
\draw [color=c, fill=c] (12.8657,3.88994) rectangle (12.9055,3.99579);
\draw [color=c, fill=c] (12.9055,3.88994) rectangle (12.9453,3.99579);
\draw [color=c, fill=c] (12.9453,3.88994) rectangle (12.9851,3.99579);
\draw [color=c, fill=c] (12.9851,3.88994) rectangle (13.0249,3.99579);
\draw [color=c, fill=c] (13.0249,3.88994) rectangle (13.0647,3.99579);
\draw [color=c, fill=c] (13.0647,3.88994) rectangle (13.1045,3.99579);
\draw [color=c, fill=c] (13.1045,3.88994) rectangle (13.1443,3.99579);
\draw [color=c, fill=c] (13.1443,3.88994) rectangle (13.1841,3.99579);
\draw [color=c, fill=c] (13.1841,3.88994) rectangle (13.2239,3.99579);
\draw [color=c, fill=c] (13.2239,3.88994) rectangle (13.2637,3.99579);
\draw [color=c, fill=c] (13.2637,3.88994) rectangle (13.3035,3.99579);
\draw [color=c, fill=c] (13.3035,3.88994) rectangle (13.3433,3.99579);
\draw [color=c, fill=c] (13.3433,3.88994) rectangle (13.3831,3.99579);
\draw [color=c, fill=c] (13.3831,3.88994) rectangle (13.4229,3.99579);
\draw [color=c, fill=c] (13.4229,3.88994) rectangle (13.4627,3.99579);
\draw [color=c, fill=c] (13.4627,3.88994) rectangle (13.5025,3.99579);
\draw [color=c, fill=c] (13.5025,3.88994) rectangle (13.5423,3.99579);
\draw [color=c, fill=c] (13.5423,3.88994) rectangle (13.5821,3.99579);
\draw [color=c, fill=c] (13.5821,3.88994) rectangle (13.6219,3.99579);
\draw [color=c, fill=c] (13.6219,3.88994) rectangle (13.6617,3.99579);
\draw [color=c, fill=c] (13.6617,3.88994) rectangle (13.7015,3.99579);
\draw [color=c, fill=c] (13.7015,3.88994) rectangle (13.7413,3.99579);
\draw [color=c, fill=c] (13.7413,3.88994) rectangle (13.7811,3.99579);
\draw [color=c, fill=c] (13.7811,3.88994) rectangle (13.8209,3.99579);
\draw [color=c, fill=c] (13.8209,3.88994) rectangle (13.8607,3.99579);
\draw [color=c, fill=c] (13.8607,3.88994) rectangle (13.9005,3.99579);
\draw [color=c, fill=c] (13.9005,3.88994) rectangle (13.9403,3.99579);
\draw [color=c, fill=c] (13.9403,3.88994) rectangle (13.9801,3.99579);
\draw [color=c, fill=c] (13.9801,3.88994) rectangle (14.0199,3.99579);
\draw [color=c, fill=c] (14.0199,3.88994) rectangle (14.0597,3.99579);
\draw [color=c, fill=c] (14.0597,3.88994) rectangle (14.0995,3.99579);
\draw [color=c, fill=c] (14.0995,3.88994) rectangle (14.1393,3.99579);
\draw [color=c, fill=c] (14.1393,3.88994) rectangle (14.1791,3.99579);
\draw [color=c, fill=c] (14.1791,3.88994) rectangle (14.2189,3.99579);
\draw [color=c, fill=c] (14.2189,3.88994) rectangle (14.2587,3.99579);
\draw [color=c, fill=c] (14.2587,3.88994) rectangle (14.2985,3.99579);
\draw [color=c, fill=c] (14.2985,3.88994) rectangle (14.3383,3.99579);
\draw [color=c, fill=c] (14.3383,3.88994) rectangle (14.3781,3.99579);
\draw [color=c, fill=c] (14.3781,3.88994) rectangle (14.4179,3.99579);
\draw [color=c, fill=c] (14.4179,3.88994) rectangle (14.4577,3.99579);
\draw [color=c, fill=c] (14.4577,3.88994) rectangle (14.4975,3.99579);
\draw [color=c, fill=c] (14.4975,3.88994) rectangle (14.5373,3.99579);
\draw [color=c, fill=c] (14.5373,3.88994) rectangle (14.5771,3.99579);
\draw [color=c, fill=c] (14.5771,3.88994) rectangle (14.6169,3.99579);
\draw [color=c, fill=c] (14.6169,3.88994) rectangle (14.6567,3.99579);
\draw [color=c, fill=c] (14.6567,3.88994) rectangle (14.6965,3.99579);
\draw [color=c, fill=c] (14.6965,3.88994) rectangle (14.7363,3.99579);
\draw [color=c, fill=c] (14.7363,3.88994) rectangle (14.7761,3.99579);
\draw [color=c, fill=c] (14.7761,3.88994) rectangle (14.8159,3.99579);
\draw [color=c, fill=c] (14.8159,3.88994) rectangle (14.8557,3.99579);
\draw [color=c, fill=c] (14.8557,3.88994) rectangle (14.8955,3.99579);
\draw [color=c, fill=c] (14.8955,3.88994) rectangle (14.9353,3.99579);
\draw [color=c, fill=c] (14.9353,3.88994) rectangle (14.9751,3.99579);
\draw [color=c, fill=c] (14.9751,3.88994) rectangle (15.0149,3.99579);
\draw [color=c, fill=c] (15.0149,3.88994) rectangle (15.0547,3.99579);
\draw [color=c, fill=c] (15.0547,3.88994) rectangle (15.0945,3.99579);
\draw [color=c, fill=c] (15.0945,3.88994) rectangle (15.1343,3.99579);
\draw [color=c, fill=c] (15.1343,3.88994) rectangle (15.1741,3.99579);
\draw [color=c, fill=c] (15.1741,3.88994) rectangle (15.2139,3.99579);
\draw [color=c, fill=c] (15.2139,3.88994) rectangle (15.2537,3.99579);
\draw [color=c, fill=c] (15.2537,3.88994) rectangle (15.2935,3.99579);
\draw [color=c, fill=c] (15.2935,3.88994) rectangle (15.3333,3.99579);
\draw [color=c, fill=c] (15.3333,3.88994) rectangle (15.3731,3.99579);
\draw [color=c, fill=c] (15.3731,3.88994) rectangle (15.4129,3.99579);
\draw [color=c, fill=c] (15.4129,3.88994) rectangle (15.4527,3.99579);
\draw [color=c, fill=c] (15.4527,3.88994) rectangle (15.4925,3.99579);
\draw [color=c, fill=c] (15.4925,3.88994) rectangle (15.5323,3.99579);
\draw [color=c, fill=c] (15.5323,3.88994) rectangle (15.5721,3.99579);
\draw [color=c, fill=c] (15.5721,3.88994) rectangle (15.6119,3.99579);
\draw [color=c, fill=c] (15.6119,3.88994) rectangle (15.6517,3.99579);
\draw [color=c, fill=c] (15.6517,3.88994) rectangle (15.6915,3.99579);
\draw [color=c, fill=c] (15.6915,3.88994) rectangle (15.7313,3.99579);
\draw [color=c, fill=c] (15.7313,3.88994) rectangle (15.7711,3.99579);
\draw [color=c, fill=c] (15.7711,3.88994) rectangle (15.8109,3.99579);
\draw [color=c, fill=c] (15.8109,3.88994) rectangle (15.8507,3.99579);
\draw [color=c, fill=c] (15.8507,3.88994) rectangle (15.8905,3.99579);
\draw [color=c, fill=c] (15.8905,3.88994) rectangle (15.9303,3.99579);
\draw [color=c, fill=c] (15.9303,3.88994) rectangle (15.9701,3.99579);
\draw [color=c, fill=c] (15.9701,3.88994) rectangle (16.01,3.99579);
\draw [color=c, fill=c] (16.01,3.88994) rectangle (16.0498,3.99579);
\draw [color=c, fill=c] (16.0498,3.88994) rectangle (16.0896,3.99579);
\draw [color=c, fill=c] (16.0896,3.88994) rectangle (16.1294,3.99579);
\draw [color=c, fill=c] (16.1294,3.88994) rectangle (16.1692,3.99579);
\draw [color=c, fill=c] (16.1692,3.88994) rectangle (16.209,3.99579);
\draw [color=c, fill=c] (16.209,3.88994) rectangle (16.2488,3.99579);
\draw [color=c, fill=c] (16.2488,3.88994) rectangle (16.2886,3.99579);
\draw [color=c, fill=c] (16.2886,3.88994) rectangle (16.3284,3.99579);
\draw [color=c, fill=c] (16.3284,3.88994) rectangle (16.3682,3.99579);
\draw [color=c, fill=c] (16.3682,3.88994) rectangle (16.408,3.99579);
\draw [color=c, fill=c] (16.408,3.88994) rectangle (16.4478,3.99579);
\draw [color=c, fill=c] (16.4478,3.88994) rectangle (16.4876,3.99579);
\draw [color=c, fill=c] (16.4876,3.88994) rectangle (16.5274,3.99579);
\draw [color=c, fill=c] (16.5274,3.88994) rectangle (16.5672,3.99579);
\draw [color=c, fill=c] (16.5672,3.88994) rectangle (16.607,3.99579);
\draw [color=c, fill=c] (16.607,3.88994) rectangle (16.6468,3.99579);
\draw [color=c, fill=c] (16.6468,3.88994) rectangle (16.6866,3.99579);
\draw [color=c, fill=c] (16.6866,3.88994) rectangle (16.7264,3.99579);
\draw [color=c, fill=c] (16.7264,3.88994) rectangle (16.7662,3.99579);
\draw [color=c, fill=c] (16.7662,3.88994) rectangle (16.806,3.99579);
\draw [color=c, fill=c] (16.806,3.88994) rectangle (16.8458,3.99579);
\draw [color=c, fill=c] (16.8458,3.88994) rectangle (16.8856,3.99579);
\draw [color=c, fill=c] (16.8856,3.88994) rectangle (16.9254,3.99579);
\draw [color=c, fill=c] (16.9254,3.88994) rectangle (16.9652,3.99579);
\draw [color=c, fill=c] (16.9652,3.88994) rectangle (17.005,3.99579);
\draw [color=c, fill=c] (17.005,3.88994) rectangle (17.0448,3.99579);
\draw [color=c, fill=c] (17.0448,3.88994) rectangle (17.0846,3.99579);
\draw [color=c, fill=c] (17.0846,3.88994) rectangle (17.1244,3.99579);
\draw [color=c, fill=c] (17.1244,3.88994) rectangle (17.1642,3.99579);
\draw [color=c, fill=c] (17.1642,3.88994) rectangle (17.204,3.99579);
\draw [color=c, fill=c] (17.204,3.88994) rectangle (17.2438,3.99579);
\draw [color=c, fill=c] (17.2438,3.88994) rectangle (17.2836,3.99579);
\draw [color=c, fill=c] (17.2836,3.88994) rectangle (17.3234,3.99579);
\draw [color=c, fill=c] (17.3234,3.88994) rectangle (17.3632,3.99579);
\draw [color=c, fill=c] (17.3632,3.88994) rectangle (17.403,3.99579);
\draw [color=c, fill=c] (17.403,3.88994) rectangle (17.4428,3.99579);
\draw [color=c, fill=c] (17.4428,3.88994) rectangle (17.4826,3.99579);
\draw [color=c, fill=c] (17.4826,3.88994) rectangle (17.5224,3.99579);
\draw [color=c, fill=c] (17.5224,3.88994) rectangle (17.5622,3.99579);
\draw [color=c, fill=c] (17.5622,3.88994) rectangle (17.602,3.99579);
\draw [color=c, fill=c] (17.602,3.88994) rectangle (17.6418,3.99579);
\draw [color=c, fill=c] (17.6418,3.88994) rectangle (17.6816,3.99579);
\draw [color=c, fill=c] (17.6816,3.88994) rectangle (17.7214,3.99579);
\draw [color=c, fill=c] (17.7214,3.88994) rectangle (17.7612,3.99579);
\draw [color=c, fill=c] (17.7612,3.88994) rectangle (17.801,3.99579);
\draw [color=c, fill=c] (17.801,3.88994) rectangle (17.8408,3.99579);
\draw [color=c, fill=c] (17.8408,3.88994) rectangle (17.8806,3.99579);
\draw [color=c, fill=c] (17.8806,3.88994) rectangle (17.9204,3.99579);
\draw [color=c, fill=c] (17.9204,3.88994) rectangle (17.9602,3.99579);
\draw [color=c, fill=c] (17.9602,3.88994) rectangle (18,3.99579);
\definecolor{c}{rgb}{0,0.0800001,1};
\draw [color=c, fill=c] (2,3.99579) rectangle (2.0398,4.10163);
\draw [color=c, fill=c] (2.0398,3.99579) rectangle (2.0796,4.10163);
\draw [color=c, fill=c] (2.0796,3.99579) rectangle (2.1194,4.10163);
\draw [color=c, fill=c] (2.1194,3.99579) rectangle (2.1592,4.10163);
\draw [color=c, fill=c] (2.1592,3.99579) rectangle (2.19901,4.10163);
\draw [color=c, fill=c] (2.19901,3.99579) rectangle (2.23881,4.10163);
\draw [color=c, fill=c] (2.23881,3.99579) rectangle (2.27861,4.10163);
\draw [color=c, fill=c] (2.27861,3.99579) rectangle (2.31841,4.10163);
\draw [color=c, fill=c] (2.31841,3.99579) rectangle (2.35821,4.10163);
\draw [color=c, fill=c] (2.35821,3.99579) rectangle (2.39801,4.10163);
\draw [color=c, fill=c] (2.39801,3.99579) rectangle (2.43781,4.10163);
\draw [color=c, fill=c] (2.43781,3.99579) rectangle (2.47761,4.10163);
\draw [color=c, fill=c] (2.47761,3.99579) rectangle (2.51741,4.10163);
\draw [color=c, fill=c] (2.51741,3.99579) rectangle (2.55721,4.10163);
\draw [color=c, fill=c] (2.55721,3.99579) rectangle (2.59702,4.10163);
\draw [color=c, fill=c] (2.59702,3.99579) rectangle (2.63682,4.10163);
\draw [color=c, fill=c] (2.63682,3.99579) rectangle (2.67662,4.10163);
\draw [color=c, fill=c] (2.67662,3.99579) rectangle (2.71642,4.10163);
\draw [color=c, fill=c] (2.71642,3.99579) rectangle (2.75622,4.10163);
\draw [color=c, fill=c] (2.75622,3.99579) rectangle (2.79602,4.10163);
\draw [color=c, fill=c] (2.79602,3.99579) rectangle (2.83582,4.10163);
\draw [color=c, fill=c] (2.83582,3.99579) rectangle (2.87562,4.10163);
\draw [color=c, fill=c] (2.87562,3.99579) rectangle (2.91542,4.10163);
\draw [color=c, fill=c] (2.91542,3.99579) rectangle (2.95522,4.10163);
\draw [color=c, fill=c] (2.95522,3.99579) rectangle (2.99502,4.10163);
\draw [color=c, fill=c] (2.99502,3.99579) rectangle (3.03483,4.10163);
\draw [color=c, fill=c] (3.03483,3.99579) rectangle (3.07463,4.10163);
\draw [color=c, fill=c] (3.07463,3.99579) rectangle (3.11443,4.10163);
\draw [color=c, fill=c] (3.11443,3.99579) rectangle (3.15423,4.10163);
\draw [color=c, fill=c] (3.15423,3.99579) rectangle (3.19403,4.10163);
\draw [color=c, fill=c] (3.19403,3.99579) rectangle (3.23383,4.10163);
\draw [color=c, fill=c] (3.23383,3.99579) rectangle (3.27363,4.10163);
\draw [color=c, fill=c] (3.27363,3.99579) rectangle (3.31343,4.10163);
\draw [color=c, fill=c] (3.31343,3.99579) rectangle (3.35323,4.10163);
\draw [color=c, fill=c] (3.35323,3.99579) rectangle (3.39303,4.10163);
\draw [color=c, fill=c] (3.39303,3.99579) rectangle (3.43284,4.10163);
\draw [color=c, fill=c] (3.43284,3.99579) rectangle (3.47264,4.10163);
\draw [color=c, fill=c] (3.47264,3.99579) rectangle (3.51244,4.10163);
\draw [color=c, fill=c] (3.51244,3.99579) rectangle (3.55224,4.10163);
\draw [color=c, fill=c] (3.55224,3.99579) rectangle (3.59204,4.10163);
\draw [color=c, fill=c] (3.59204,3.99579) rectangle (3.63184,4.10163);
\draw [color=c, fill=c] (3.63184,3.99579) rectangle (3.67164,4.10163);
\draw [color=c, fill=c] (3.67164,3.99579) rectangle (3.71144,4.10163);
\draw [color=c, fill=c] (3.71144,3.99579) rectangle (3.75124,4.10163);
\draw [color=c, fill=c] (3.75124,3.99579) rectangle (3.79104,4.10163);
\draw [color=c, fill=c] (3.79104,3.99579) rectangle (3.83085,4.10163);
\draw [color=c, fill=c] (3.83085,3.99579) rectangle (3.87065,4.10163);
\draw [color=c, fill=c] (3.87065,3.99579) rectangle (3.91045,4.10163);
\draw [color=c, fill=c] (3.91045,3.99579) rectangle (3.95025,4.10163);
\draw [color=c, fill=c] (3.95025,3.99579) rectangle (3.99005,4.10163);
\draw [color=c, fill=c] (3.99005,3.99579) rectangle (4.02985,4.10163);
\draw [color=c, fill=c] (4.02985,3.99579) rectangle (4.06965,4.10163);
\draw [color=c, fill=c] (4.06965,3.99579) rectangle (4.10945,4.10163);
\draw [color=c, fill=c] (4.10945,3.99579) rectangle (4.14925,4.10163);
\draw [color=c, fill=c] (4.14925,3.99579) rectangle (4.18905,4.10163);
\draw [color=c, fill=c] (4.18905,3.99579) rectangle (4.22886,4.10163);
\draw [color=c, fill=c] (4.22886,3.99579) rectangle (4.26866,4.10163);
\draw [color=c, fill=c] (4.26866,3.99579) rectangle (4.30846,4.10163);
\draw [color=c, fill=c] (4.30846,3.99579) rectangle (4.34826,4.10163);
\draw [color=c, fill=c] (4.34826,3.99579) rectangle (4.38806,4.10163);
\draw [color=c, fill=c] (4.38806,3.99579) rectangle (4.42786,4.10163);
\draw [color=c, fill=c] (4.42786,3.99579) rectangle (4.46766,4.10163);
\draw [color=c, fill=c] (4.46766,3.99579) rectangle (4.50746,4.10163);
\draw [color=c, fill=c] (4.50746,3.99579) rectangle (4.54726,4.10163);
\draw [color=c, fill=c] (4.54726,3.99579) rectangle (4.58706,4.10163);
\draw [color=c, fill=c] (4.58706,3.99579) rectangle (4.62687,4.10163);
\draw [color=c, fill=c] (4.62687,3.99579) rectangle (4.66667,4.10163);
\draw [color=c, fill=c] (4.66667,3.99579) rectangle (4.70647,4.10163);
\draw [color=c, fill=c] (4.70647,3.99579) rectangle (4.74627,4.10163);
\draw [color=c, fill=c] (4.74627,3.99579) rectangle (4.78607,4.10163);
\draw [color=c, fill=c] (4.78607,3.99579) rectangle (4.82587,4.10163);
\draw [color=c, fill=c] (4.82587,3.99579) rectangle (4.86567,4.10163);
\draw [color=c, fill=c] (4.86567,3.99579) rectangle (4.90547,4.10163);
\draw [color=c, fill=c] (4.90547,3.99579) rectangle (4.94527,4.10163);
\draw [color=c, fill=c] (4.94527,3.99579) rectangle (4.98507,4.10163);
\draw [color=c, fill=c] (4.98507,3.99579) rectangle (5.02488,4.10163);
\draw [color=c, fill=c] (5.02488,3.99579) rectangle (5.06468,4.10163);
\draw [color=c, fill=c] (5.06468,3.99579) rectangle (5.10448,4.10163);
\draw [color=c, fill=c] (5.10448,3.99579) rectangle (5.14428,4.10163);
\draw [color=c, fill=c] (5.14428,3.99579) rectangle (5.18408,4.10163);
\draw [color=c, fill=c] (5.18408,3.99579) rectangle (5.22388,4.10163);
\draw [color=c, fill=c] (5.22388,3.99579) rectangle (5.26368,4.10163);
\draw [color=c, fill=c] (5.26368,3.99579) rectangle (5.30348,4.10163);
\draw [color=c, fill=c] (5.30348,3.99579) rectangle (5.34328,4.10163);
\draw [color=c, fill=c] (5.34328,3.99579) rectangle (5.38308,4.10163);
\draw [color=c, fill=c] (5.38308,3.99579) rectangle (5.42289,4.10163);
\draw [color=c, fill=c] (5.42289,3.99579) rectangle (5.46269,4.10163);
\draw [color=c, fill=c] (5.46269,3.99579) rectangle (5.50249,4.10163);
\draw [color=c, fill=c] (5.50249,3.99579) rectangle (5.54229,4.10163);
\draw [color=c, fill=c] (5.54229,3.99579) rectangle (5.58209,4.10163);
\draw [color=c, fill=c] (5.58209,3.99579) rectangle (5.62189,4.10163);
\draw [color=c, fill=c] (5.62189,3.99579) rectangle (5.66169,4.10163);
\draw [color=c, fill=c] (5.66169,3.99579) rectangle (5.70149,4.10163);
\draw [color=c, fill=c] (5.70149,3.99579) rectangle (5.74129,4.10163);
\draw [color=c, fill=c] (5.74129,3.99579) rectangle (5.78109,4.10163);
\draw [color=c, fill=c] (5.78109,3.99579) rectangle (5.8209,4.10163);
\draw [color=c, fill=c] (5.8209,3.99579) rectangle (5.8607,4.10163);
\draw [color=c, fill=c] (5.8607,3.99579) rectangle (5.9005,4.10163);
\definecolor{c}{rgb}{0.2,0,1};
\draw [color=c, fill=c] (5.9005,3.99579) rectangle (5.9403,4.10163);
\draw [color=c, fill=c] (5.9403,3.99579) rectangle (5.9801,4.10163);
\draw [color=c, fill=c] (5.9801,3.99579) rectangle (6.0199,4.10163);
\draw [color=c, fill=c] (6.0199,3.99579) rectangle (6.0597,4.10163);
\draw [color=c, fill=c] (6.0597,3.99579) rectangle (6.0995,4.10163);
\draw [color=c, fill=c] (6.0995,3.99579) rectangle (6.1393,4.10163);
\draw [color=c, fill=c] (6.1393,3.99579) rectangle (6.1791,4.10163);
\draw [color=c, fill=c] (6.1791,3.99579) rectangle (6.21891,4.10163);
\draw [color=c, fill=c] (6.21891,3.99579) rectangle (6.25871,4.10163);
\draw [color=c, fill=c] (6.25871,3.99579) rectangle (6.29851,4.10163);
\draw [color=c, fill=c] (6.29851,3.99579) rectangle (6.33831,4.10163);
\draw [color=c, fill=c] (6.33831,3.99579) rectangle (6.37811,4.10163);
\draw [color=c, fill=c] (6.37811,3.99579) rectangle (6.41791,4.10163);
\draw [color=c, fill=c] (6.41791,3.99579) rectangle (6.45771,4.10163);
\draw [color=c, fill=c] (6.45771,3.99579) rectangle (6.49751,4.10163);
\draw [color=c, fill=c] (6.49751,3.99579) rectangle (6.53731,4.10163);
\draw [color=c, fill=c] (6.53731,3.99579) rectangle (6.57711,4.10163);
\draw [color=c, fill=c] (6.57711,3.99579) rectangle (6.61692,4.10163);
\draw [color=c, fill=c] (6.61692,3.99579) rectangle (6.65672,4.10163);
\draw [color=c, fill=c] (6.65672,3.99579) rectangle (6.69652,4.10163);
\draw [color=c, fill=c] (6.69652,3.99579) rectangle (6.73632,4.10163);
\draw [color=c, fill=c] (6.73632,3.99579) rectangle (6.77612,4.10163);
\draw [color=c, fill=c] (6.77612,3.99579) rectangle (6.81592,4.10163);
\draw [color=c, fill=c] (6.81592,3.99579) rectangle (6.85572,4.10163);
\draw [color=c, fill=c] (6.85572,3.99579) rectangle (6.89552,4.10163);
\draw [color=c, fill=c] (6.89552,3.99579) rectangle (6.93532,4.10163);
\draw [color=c, fill=c] (6.93532,3.99579) rectangle (6.97512,4.10163);
\draw [color=c, fill=c] (6.97512,3.99579) rectangle (7.01493,4.10163);
\draw [color=c, fill=c] (7.01493,3.99579) rectangle (7.05473,4.10163);
\draw [color=c, fill=c] (7.05473,3.99579) rectangle (7.09453,4.10163);
\draw [color=c, fill=c] (7.09453,3.99579) rectangle (7.13433,4.10163);
\draw [color=c, fill=c] (7.13433,3.99579) rectangle (7.17413,4.10163);
\draw [color=c, fill=c] (7.17413,3.99579) rectangle (7.21393,4.10163);
\draw [color=c, fill=c] (7.21393,3.99579) rectangle (7.25373,4.10163);
\draw [color=c, fill=c] (7.25373,3.99579) rectangle (7.29353,4.10163);
\draw [color=c, fill=c] (7.29353,3.99579) rectangle (7.33333,4.10163);
\draw [color=c, fill=c] (7.33333,3.99579) rectangle (7.37313,4.10163);
\draw [color=c, fill=c] (7.37313,3.99579) rectangle (7.41294,4.10163);
\draw [color=c, fill=c] (7.41294,3.99579) rectangle (7.45274,4.10163);
\draw [color=c, fill=c] (7.45274,3.99579) rectangle (7.49254,4.10163);
\draw [color=c, fill=c] (7.49254,3.99579) rectangle (7.53234,4.10163);
\draw [color=c, fill=c] (7.53234,3.99579) rectangle (7.57214,4.10163);
\draw [color=c, fill=c] (7.57214,3.99579) rectangle (7.61194,4.10163);
\draw [color=c, fill=c] (7.61194,3.99579) rectangle (7.65174,4.10163);
\draw [color=c, fill=c] (7.65174,3.99579) rectangle (7.69154,4.10163);
\draw [color=c, fill=c] (7.69154,3.99579) rectangle (7.73134,4.10163);
\draw [color=c, fill=c] (7.73134,3.99579) rectangle (7.77114,4.10163);
\draw [color=c, fill=c] (7.77114,3.99579) rectangle (7.81095,4.10163);
\draw [color=c, fill=c] (7.81095,3.99579) rectangle (7.85075,4.10163);
\draw [color=c, fill=c] (7.85075,3.99579) rectangle (7.89055,4.10163);
\draw [color=c, fill=c] (7.89055,3.99579) rectangle (7.93035,4.10163);
\draw [color=c, fill=c] (7.93035,3.99579) rectangle (7.97015,4.10163);
\draw [color=c, fill=c] (7.97015,3.99579) rectangle (8.00995,4.10163);
\draw [color=c, fill=c] (8.00995,3.99579) rectangle (8.04975,4.10163);
\draw [color=c, fill=c] (8.04975,3.99579) rectangle (8.08955,4.10163);
\draw [color=c, fill=c] (8.08955,3.99579) rectangle (8.12935,4.10163);
\draw [color=c, fill=c] (8.12935,3.99579) rectangle (8.16915,4.10163);
\draw [color=c, fill=c] (8.16915,3.99579) rectangle (8.20895,4.10163);
\draw [color=c, fill=c] (8.20895,3.99579) rectangle (8.24876,4.10163);
\draw [color=c, fill=c] (8.24876,3.99579) rectangle (8.28856,4.10163);
\draw [color=c, fill=c] (8.28856,3.99579) rectangle (8.32836,4.10163);
\draw [color=c, fill=c] (8.32836,3.99579) rectangle (8.36816,4.10163);
\draw [color=c, fill=c] (8.36816,3.99579) rectangle (8.40796,4.10163);
\draw [color=c, fill=c] (8.40796,3.99579) rectangle (8.44776,4.10163);
\draw [color=c, fill=c] (8.44776,3.99579) rectangle (8.48756,4.10163);
\draw [color=c, fill=c] (8.48756,3.99579) rectangle (8.52736,4.10163);
\draw [color=c, fill=c] (8.52736,3.99579) rectangle (8.56716,4.10163);
\definecolor{c}{rgb}{0.386667,0,1};
\draw [color=c, fill=c] (8.56716,3.99579) rectangle (8.60697,4.10163);
\draw [color=c, fill=c] (8.60697,3.99579) rectangle (8.64677,4.10163);
\draw [color=c, fill=c] (8.64677,3.99579) rectangle (8.68657,4.10163);
\draw [color=c, fill=c] (8.68657,3.99579) rectangle (8.72637,4.10163);
\draw [color=c, fill=c] (8.72637,3.99579) rectangle (8.76617,4.10163);
\draw [color=c, fill=c] (8.76617,3.99579) rectangle (8.80597,4.10163);
\draw [color=c, fill=c] (8.80597,3.99579) rectangle (8.84577,4.10163);
\draw [color=c, fill=c] (8.84577,3.99579) rectangle (8.88557,4.10163);
\draw [color=c, fill=c] (8.88557,3.99579) rectangle (8.92537,4.10163);
\draw [color=c, fill=c] (8.92537,3.99579) rectangle (8.96517,4.10163);
\draw [color=c, fill=c] (8.96517,3.99579) rectangle (9.00498,4.10163);
\draw [color=c, fill=c] (9.00498,3.99579) rectangle (9.04478,4.10163);
\draw [color=c, fill=c] (9.04478,3.99579) rectangle (9.08458,4.10163);
\draw [color=c, fill=c] (9.08458,3.99579) rectangle (9.12438,4.10163);
\draw [color=c, fill=c] (9.12438,3.99579) rectangle (9.16418,4.10163);
\draw [color=c, fill=c] (9.16418,3.99579) rectangle (9.20398,4.10163);
\draw [color=c, fill=c] (9.20398,3.99579) rectangle (9.24378,4.10163);
\draw [color=c, fill=c] (9.24378,3.99579) rectangle (9.28358,4.10163);
\draw [color=c, fill=c] (9.28358,3.99579) rectangle (9.32338,4.10163);
\draw [color=c, fill=c] (9.32338,3.99579) rectangle (9.36318,4.10163);
\draw [color=c, fill=c] (9.36318,3.99579) rectangle (9.40298,4.10163);
\draw [color=c, fill=c] (9.40298,3.99579) rectangle (9.44279,4.10163);
\draw [color=c, fill=c] (9.44279,3.99579) rectangle (9.48259,4.10163);
\draw [color=c, fill=c] (9.48259,3.99579) rectangle (9.52239,4.10163);
\draw [color=c, fill=c] (9.52239,3.99579) rectangle (9.56219,4.10163);
\draw [color=c, fill=c] (9.56219,3.99579) rectangle (9.60199,4.10163);
\draw [color=c, fill=c] (9.60199,3.99579) rectangle (9.64179,4.10163);
\draw [color=c, fill=c] (9.64179,3.99579) rectangle (9.68159,4.10163);
\draw [color=c, fill=c] (9.68159,3.99579) rectangle (9.72139,4.10163);
\draw [color=c, fill=c] (9.72139,3.99579) rectangle (9.76119,4.10163);
\draw [color=c, fill=c] (9.76119,3.99579) rectangle (9.80099,4.10163);
\draw [color=c, fill=c] (9.80099,3.99579) rectangle (9.8408,4.10163);
\definecolor{c}{rgb}{0.2,0,1};
\draw [color=c, fill=c] (9.8408,3.99579) rectangle (9.8806,4.10163);
\draw [color=c, fill=c] (9.8806,3.99579) rectangle (9.9204,4.10163);
\definecolor{c}{rgb}{0,0.0800001,1};
\draw [color=c, fill=c] (9.9204,3.99579) rectangle (9.9602,4.10163);
\definecolor{c}{rgb}{0,0.546666,1};
\draw [color=c, fill=c] (9.9602,3.99579) rectangle (10,4.10163);
\definecolor{c}{rgb}{0,0.733333,1};
\draw [color=c, fill=c] (10,3.99579) rectangle (10.0398,4.10163);
\definecolor{c}{rgb}{0,1,0.986667};
\draw [color=c, fill=c] (10.0398,3.99579) rectangle (10.0796,4.10163);
\draw [color=c, fill=c] (10.0796,3.99579) rectangle (10.1194,4.10163);
\draw [color=c, fill=c] (10.1194,3.99579) rectangle (10.1592,4.10163);
\draw [color=c, fill=c] (10.1592,3.99579) rectangle (10.199,4.10163);
\draw [color=c, fill=c] (10.199,3.99579) rectangle (10.2388,4.10163);
\draw [color=c, fill=c] (10.2388,3.99579) rectangle (10.2786,4.10163);
\draw [color=c, fill=c] (10.2786,3.99579) rectangle (10.3184,4.10163);
\definecolor{c}{rgb}{0,1,0.8};
\draw [color=c, fill=c] (10.3184,3.99579) rectangle (10.3582,4.10163);
\draw [color=c, fill=c] (10.3582,3.99579) rectangle (10.398,4.10163);
\definecolor{c}{rgb}{0,1,0.986667};
\draw [color=c, fill=c] (10.398,3.99579) rectangle (10.4378,4.10163);
\draw [color=c, fill=c] (10.4378,3.99579) rectangle (10.4776,4.10163);
\draw [color=c, fill=c] (10.4776,3.99579) rectangle (10.5174,4.10163);
\draw [color=c, fill=c] (10.5174,3.99579) rectangle (10.5572,4.10163);
\draw [color=c, fill=c] (10.5572,3.99579) rectangle (10.597,4.10163);
\draw [color=c, fill=c] (10.597,3.99579) rectangle (10.6368,4.10163);
\draw [color=c, fill=c] (10.6368,3.99579) rectangle (10.6766,4.10163);
\draw [color=c, fill=c] (10.6766,3.99579) rectangle (10.7164,4.10163);
\draw [color=c, fill=c] (10.7164,3.99579) rectangle (10.7562,4.10163);
\draw [color=c, fill=c] (10.7562,3.99579) rectangle (10.796,4.10163);
\draw [color=c, fill=c] (10.796,3.99579) rectangle (10.8358,4.10163);
\draw [color=c, fill=c] (10.8358,3.99579) rectangle (10.8756,4.10163);
\draw [color=c, fill=c] (10.8756,3.99579) rectangle (10.9154,4.10163);
\draw [color=c, fill=c] (10.9154,3.99579) rectangle (10.9552,4.10163);
\draw [color=c, fill=c] (10.9552,3.99579) rectangle (10.995,4.10163);
\draw [color=c, fill=c] (10.995,3.99579) rectangle (11.0348,4.10163);
\draw [color=c, fill=c] (11.0348,3.99579) rectangle (11.0746,4.10163);
\draw [color=c, fill=c] (11.0746,3.99579) rectangle (11.1144,4.10163);
\draw [color=c, fill=c] (11.1144,3.99579) rectangle (11.1542,4.10163);
\draw [color=c, fill=c] (11.1542,3.99579) rectangle (11.194,4.10163);
\draw [color=c, fill=c] (11.194,3.99579) rectangle (11.2338,4.10163);
\draw [color=c, fill=c] (11.2338,3.99579) rectangle (11.2736,4.10163);
\draw [color=c, fill=c] (11.2736,3.99579) rectangle (11.3134,4.10163);
\draw [color=c, fill=c] (11.3134,3.99579) rectangle (11.3532,4.10163);
\draw [color=c, fill=c] (11.3532,3.99579) rectangle (11.393,4.10163);
\draw [color=c, fill=c] (11.393,3.99579) rectangle (11.4328,4.10163);
\draw [color=c, fill=c] (11.4328,3.99579) rectangle (11.4726,4.10163);
\draw [color=c, fill=c] (11.4726,3.99579) rectangle (11.5124,4.10163);
\draw [color=c, fill=c] (11.5124,3.99579) rectangle (11.5522,4.10163);
\draw [color=c, fill=c] (11.5522,3.99579) rectangle (11.592,4.10163);
\draw [color=c, fill=c] (11.592,3.99579) rectangle (11.6318,4.10163);
\draw [color=c, fill=c] (11.6318,3.99579) rectangle (11.6716,4.10163);
\draw [color=c, fill=c] (11.6716,3.99579) rectangle (11.7114,4.10163);
\draw [color=c, fill=c] (11.7114,3.99579) rectangle (11.7512,4.10163);
\draw [color=c, fill=c] (11.7512,3.99579) rectangle (11.791,4.10163);
\draw [color=c, fill=c] (11.791,3.99579) rectangle (11.8308,4.10163);
\definecolor{c}{rgb}{0,0.733333,1};
\draw [color=c, fill=c] (11.8308,3.99579) rectangle (11.8706,4.10163);
\draw [color=c, fill=c] (11.8706,3.99579) rectangle (11.9104,4.10163);
\draw [color=c, fill=c] (11.9104,3.99579) rectangle (11.9502,4.10163);
\draw [color=c, fill=c] (11.9502,3.99579) rectangle (11.99,4.10163);
\draw [color=c, fill=c] (11.99,3.99579) rectangle (12.0299,4.10163);
\draw [color=c, fill=c] (12.0299,3.99579) rectangle (12.0697,4.10163);
\draw [color=c, fill=c] (12.0697,3.99579) rectangle (12.1095,4.10163);
\draw [color=c, fill=c] (12.1095,3.99579) rectangle (12.1493,4.10163);
\draw [color=c, fill=c] (12.1493,3.99579) rectangle (12.1891,4.10163);
\draw [color=c, fill=c] (12.1891,3.99579) rectangle (12.2289,4.10163);
\draw [color=c, fill=c] (12.2289,3.99579) rectangle (12.2687,4.10163);
\draw [color=c, fill=c] (12.2687,3.99579) rectangle (12.3085,4.10163);
\draw [color=c, fill=c] (12.3085,3.99579) rectangle (12.3483,4.10163);
\draw [color=c, fill=c] (12.3483,3.99579) rectangle (12.3881,4.10163);
\draw [color=c, fill=c] (12.3881,3.99579) rectangle (12.4279,4.10163);
\draw [color=c, fill=c] (12.4279,3.99579) rectangle (12.4677,4.10163);
\draw [color=c, fill=c] (12.4677,3.99579) rectangle (12.5075,4.10163);
\draw [color=c, fill=c] (12.5075,3.99579) rectangle (12.5473,4.10163);
\draw [color=c, fill=c] (12.5473,3.99579) rectangle (12.5871,4.10163);
\draw [color=c, fill=c] (12.5871,3.99579) rectangle (12.6269,4.10163);
\draw [color=c, fill=c] (12.6269,3.99579) rectangle (12.6667,4.10163);
\draw [color=c, fill=c] (12.6667,3.99579) rectangle (12.7065,4.10163);
\draw [color=c, fill=c] (12.7065,3.99579) rectangle (12.7463,4.10163);
\draw [color=c, fill=c] (12.7463,3.99579) rectangle (12.7861,4.10163);
\draw [color=c, fill=c] (12.7861,3.99579) rectangle (12.8259,4.10163);
\draw [color=c, fill=c] (12.8259,3.99579) rectangle (12.8657,4.10163);
\draw [color=c, fill=c] (12.8657,3.99579) rectangle (12.9055,4.10163);
\draw [color=c, fill=c] (12.9055,3.99579) rectangle (12.9453,4.10163);
\draw [color=c, fill=c] (12.9453,3.99579) rectangle (12.9851,4.10163);
\draw [color=c, fill=c] (12.9851,3.99579) rectangle (13.0249,4.10163);
\draw [color=c, fill=c] (13.0249,3.99579) rectangle (13.0647,4.10163);
\draw [color=c, fill=c] (13.0647,3.99579) rectangle (13.1045,4.10163);
\draw [color=c, fill=c] (13.1045,3.99579) rectangle (13.1443,4.10163);
\draw [color=c, fill=c] (13.1443,3.99579) rectangle (13.1841,4.10163);
\draw [color=c, fill=c] (13.1841,3.99579) rectangle (13.2239,4.10163);
\draw [color=c, fill=c] (13.2239,3.99579) rectangle (13.2637,4.10163);
\draw [color=c, fill=c] (13.2637,3.99579) rectangle (13.3035,4.10163);
\draw [color=c, fill=c] (13.3035,3.99579) rectangle (13.3433,4.10163);
\draw [color=c, fill=c] (13.3433,3.99579) rectangle (13.3831,4.10163);
\draw [color=c, fill=c] (13.3831,3.99579) rectangle (13.4229,4.10163);
\draw [color=c, fill=c] (13.4229,3.99579) rectangle (13.4627,4.10163);
\draw [color=c, fill=c] (13.4627,3.99579) rectangle (13.5025,4.10163);
\draw [color=c, fill=c] (13.5025,3.99579) rectangle (13.5423,4.10163);
\draw [color=c, fill=c] (13.5423,3.99579) rectangle (13.5821,4.10163);
\draw [color=c, fill=c] (13.5821,3.99579) rectangle (13.6219,4.10163);
\draw [color=c, fill=c] (13.6219,3.99579) rectangle (13.6617,4.10163);
\draw [color=c, fill=c] (13.6617,3.99579) rectangle (13.7015,4.10163);
\draw [color=c, fill=c] (13.7015,3.99579) rectangle (13.7413,4.10163);
\draw [color=c, fill=c] (13.7413,3.99579) rectangle (13.7811,4.10163);
\draw [color=c, fill=c] (13.7811,3.99579) rectangle (13.8209,4.10163);
\draw [color=c, fill=c] (13.8209,3.99579) rectangle (13.8607,4.10163);
\draw [color=c, fill=c] (13.8607,3.99579) rectangle (13.9005,4.10163);
\draw [color=c, fill=c] (13.9005,3.99579) rectangle (13.9403,4.10163);
\draw [color=c, fill=c] (13.9403,3.99579) rectangle (13.9801,4.10163);
\draw [color=c, fill=c] (13.9801,3.99579) rectangle (14.0199,4.10163);
\draw [color=c, fill=c] (14.0199,3.99579) rectangle (14.0597,4.10163);
\draw [color=c, fill=c] (14.0597,3.99579) rectangle (14.0995,4.10163);
\draw [color=c, fill=c] (14.0995,3.99579) rectangle (14.1393,4.10163);
\draw [color=c, fill=c] (14.1393,3.99579) rectangle (14.1791,4.10163);
\draw [color=c, fill=c] (14.1791,3.99579) rectangle (14.2189,4.10163);
\draw [color=c, fill=c] (14.2189,3.99579) rectangle (14.2587,4.10163);
\draw [color=c, fill=c] (14.2587,3.99579) rectangle (14.2985,4.10163);
\draw [color=c, fill=c] (14.2985,3.99579) rectangle (14.3383,4.10163);
\draw [color=c, fill=c] (14.3383,3.99579) rectangle (14.3781,4.10163);
\draw [color=c, fill=c] (14.3781,3.99579) rectangle (14.4179,4.10163);
\draw [color=c, fill=c] (14.4179,3.99579) rectangle (14.4577,4.10163);
\draw [color=c, fill=c] (14.4577,3.99579) rectangle (14.4975,4.10163);
\draw [color=c, fill=c] (14.4975,3.99579) rectangle (14.5373,4.10163);
\draw [color=c, fill=c] (14.5373,3.99579) rectangle (14.5771,4.10163);
\draw [color=c, fill=c] (14.5771,3.99579) rectangle (14.6169,4.10163);
\draw [color=c, fill=c] (14.6169,3.99579) rectangle (14.6567,4.10163);
\draw [color=c, fill=c] (14.6567,3.99579) rectangle (14.6965,4.10163);
\draw [color=c, fill=c] (14.6965,3.99579) rectangle (14.7363,4.10163);
\draw [color=c, fill=c] (14.7363,3.99579) rectangle (14.7761,4.10163);
\draw [color=c, fill=c] (14.7761,3.99579) rectangle (14.8159,4.10163);
\draw [color=c, fill=c] (14.8159,3.99579) rectangle (14.8557,4.10163);
\draw [color=c, fill=c] (14.8557,3.99579) rectangle (14.8955,4.10163);
\draw [color=c, fill=c] (14.8955,3.99579) rectangle (14.9353,4.10163);
\draw [color=c, fill=c] (14.9353,3.99579) rectangle (14.9751,4.10163);
\draw [color=c, fill=c] (14.9751,3.99579) rectangle (15.0149,4.10163);
\draw [color=c, fill=c] (15.0149,3.99579) rectangle (15.0547,4.10163);
\draw [color=c, fill=c] (15.0547,3.99579) rectangle (15.0945,4.10163);
\draw [color=c, fill=c] (15.0945,3.99579) rectangle (15.1343,4.10163);
\draw [color=c, fill=c] (15.1343,3.99579) rectangle (15.1741,4.10163);
\draw [color=c, fill=c] (15.1741,3.99579) rectangle (15.2139,4.10163);
\draw [color=c, fill=c] (15.2139,3.99579) rectangle (15.2537,4.10163);
\draw [color=c, fill=c] (15.2537,3.99579) rectangle (15.2935,4.10163);
\draw [color=c, fill=c] (15.2935,3.99579) rectangle (15.3333,4.10163);
\draw [color=c, fill=c] (15.3333,3.99579) rectangle (15.3731,4.10163);
\draw [color=c, fill=c] (15.3731,3.99579) rectangle (15.4129,4.10163);
\draw [color=c, fill=c] (15.4129,3.99579) rectangle (15.4527,4.10163);
\draw [color=c, fill=c] (15.4527,3.99579) rectangle (15.4925,4.10163);
\draw [color=c, fill=c] (15.4925,3.99579) rectangle (15.5323,4.10163);
\draw [color=c, fill=c] (15.5323,3.99579) rectangle (15.5721,4.10163);
\draw [color=c, fill=c] (15.5721,3.99579) rectangle (15.6119,4.10163);
\draw [color=c, fill=c] (15.6119,3.99579) rectangle (15.6517,4.10163);
\draw [color=c, fill=c] (15.6517,3.99579) rectangle (15.6915,4.10163);
\draw [color=c, fill=c] (15.6915,3.99579) rectangle (15.7313,4.10163);
\draw [color=c, fill=c] (15.7313,3.99579) rectangle (15.7711,4.10163);
\draw [color=c, fill=c] (15.7711,3.99579) rectangle (15.8109,4.10163);
\draw [color=c, fill=c] (15.8109,3.99579) rectangle (15.8507,4.10163);
\draw [color=c, fill=c] (15.8507,3.99579) rectangle (15.8905,4.10163);
\draw [color=c, fill=c] (15.8905,3.99579) rectangle (15.9303,4.10163);
\draw [color=c, fill=c] (15.9303,3.99579) rectangle (15.9701,4.10163);
\draw [color=c, fill=c] (15.9701,3.99579) rectangle (16.01,4.10163);
\draw [color=c, fill=c] (16.01,3.99579) rectangle (16.0498,4.10163);
\draw [color=c, fill=c] (16.0498,3.99579) rectangle (16.0896,4.10163);
\draw [color=c, fill=c] (16.0896,3.99579) rectangle (16.1294,4.10163);
\draw [color=c, fill=c] (16.1294,3.99579) rectangle (16.1692,4.10163);
\draw [color=c, fill=c] (16.1692,3.99579) rectangle (16.209,4.10163);
\draw [color=c, fill=c] (16.209,3.99579) rectangle (16.2488,4.10163);
\draw [color=c, fill=c] (16.2488,3.99579) rectangle (16.2886,4.10163);
\draw [color=c, fill=c] (16.2886,3.99579) rectangle (16.3284,4.10163);
\draw [color=c, fill=c] (16.3284,3.99579) rectangle (16.3682,4.10163);
\draw [color=c, fill=c] (16.3682,3.99579) rectangle (16.408,4.10163);
\draw [color=c, fill=c] (16.408,3.99579) rectangle (16.4478,4.10163);
\draw [color=c, fill=c] (16.4478,3.99579) rectangle (16.4876,4.10163);
\draw [color=c, fill=c] (16.4876,3.99579) rectangle (16.5274,4.10163);
\draw [color=c, fill=c] (16.5274,3.99579) rectangle (16.5672,4.10163);
\draw [color=c, fill=c] (16.5672,3.99579) rectangle (16.607,4.10163);
\draw [color=c, fill=c] (16.607,3.99579) rectangle (16.6468,4.10163);
\draw [color=c, fill=c] (16.6468,3.99579) rectangle (16.6866,4.10163);
\draw [color=c, fill=c] (16.6866,3.99579) rectangle (16.7264,4.10163);
\draw [color=c, fill=c] (16.7264,3.99579) rectangle (16.7662,4.10163);
\draw [color=c, fill=c] (16.7662,3.99579) rectangle (16.806,4.10163);
\draw [color=c, fill=c] (16.806,3.99579) rectangle (16.8458,4.10163);
\draw [color=c, fill=c] (16.8458,3.99579) rectangle (16.8856,4.10163);
\draw [color=c, fill=c] (16.8856,3.99579) rectangle (16.9254,4.10163);
\draw [color=c, fill=c] (16.9254,3.99579) rectangle (16.9652,4.10163);
\draw [color=c, fill=c] (16.9652,3.99579) rectangle (17.005,4.10163);
\draw [color=c, fill=c] (17.005,3.99579) rectangle (17.0448,4.10163);
\draw [color=c, fill=c] (17.0448,3.99579) rectangle (17.0846,4.10163);
\draw [color=c, fill=c] (17.0846,3.99579) rectangle (17.1244,4.10163);
\draw [color=c, fill=c] (17.1244,3.99579) rectangle (17.1642,4.10163);
\draw [color=c, fill=c] (17.1642,3.99579) rectangle (17.204,4.10163);
\draw [color=c, fill=c] (17.204,3.99579) rectangle (17.2438,4.10163);
\draw [color=c, fill=c] (17.2438,3.99579) rectangle (17.2836,4.10163);
\draw [color=c, fill=c] (17.2836,3.99579) rectangle (17.3234,4.10163);
\draw [color=c, fill=c] (17.3234,3.99579) rectangle (17.3632,4.10163);
\draw [color=c, fill=c] (17.3632,3.99579) rectangle (17.403,4.10163);
\draw [color=c, fill=c] (17.403,3.99579) rectangle (17.4428,4.10163);
\draw [color=c, fill=c] (17.4428,3.99579) rectangle (17.4826,4.10163);
\draw [color=c, fill=c] (17.4826,3.99579) rectangle (17.5224,4.10163);
\draw [color=c, fill=c] (17.5224,3.99579) rectangle (17.5622,4.10163);
\draw [color=c, fill=c] (17.5622,3.99579) rectangle (17.602,4.10163);
\draw [color=c, fill=c] (17.602,3.99579) rectangle (17.6418,4.10163);
\draw [color=c, fill=c] (17.6418,3.99579) rectangle (17.6816,4.10163);
\draw [color=c, fill=c] (17.6816,3.99579) rectangle (17.7214,4.10163);
\draw [color=c, fill=c] (17.7214,3.99579) rectangle (17.7612,4.10163);
\draw [color=c, fill=c] (17.7612,3.99579) rectangle (17.801,4.10163);
\draw [color=c, fill=c] (17.801,3.99579) rectangle (17.8408,4.10163);
\draw [color=c, fill=c] (17.8408,3.99579) rectangle (17.8806,4.10163);
\draw [color=c, fill=c] (17.8806,3.99579) rectangle (17.9204,4.10163);
\draw [color=c, fill=c] (17.9204,3.99579) rectangle (17.9602,4.10163);
\draw [color=c, fill=c] (17.9602,3.99579) rectangle (18,4.10163);
\definecolor{c}{rgb}{0,0.0800001,1};
\draw [color=c, fill=c] (2,4.10163) rectangle (2.0398,4.20748);
\draw [color=c, fill=c] (2.0398,4.10163) rectangle (2.0796,4.20748);
\draw [color=c, fill=c] (2.0796,4.10163) rectangle (2.1194,4.20748);
\draw [color=c, fill=c] (2.1194,4.10163) rectangle (2.1592,4.20748);
\draw [color=c, fill=c] (2.1592,4.10163) rectangle (2.19901,4.20748);
\draw [color=c, fill=c] (2.19901,4.10163) rectangle (2.23881,4.20748);
\draw [color=c, fill=c] (2.23881,4.10163) rectangle (2.27861,4.20748);
\draw [color=c, fill=c] (2.27861,4.10163) rectangle (2.31841,4.20748);
\draw [color=c, fill=c] (2.31841,4.10163) rectangle (2.35821,4.20748);
\draw [color=c, fill=c] (2.35821,4.10163) rectangle (2.39801,4.20748);
\draw [color=c, fill=c] (2.39801,4.10163) rectangle (2.43781,4.20748);
\draw [color=c, fill=c] (2.43781,4.10163) rectangle (2.47761,4.20748);
\draw [color=c, fill=c] (2.47761,4.10163) rectangle (2.51741,4.20748);
\draw [color=c, fill=c] (2.51741,4.10163) rectangle (2.55721,4.20748);
\draw [color=c, fill=c] (2.55721,4.10163) rectangle (2.59702,4.20748);
\draw [color=c, fill=c] (2.59702,4.10163) rectangle (2.63682,4.20748);
\draw [color=c, fill=c] (2.63682,4.10163) rectangle (2.67662,4.20748);
\draw [color=c, fill=c] (2.67662,4.10163) rectangle (2.71642,4.20748);
\draw [color=c, fill=c] (2.71642,4.10163) rectangle (2.75622,4.20748);
\draw [color=c, fill=c] (2.75622,4.10163) rectangle (2.79602,4.20748);
\draw [color=c, fill=c] (2.79602,4.10163) rectangle (2.83582,4.20748);
\draw [color=c, fill=c] (2.83582,4.10163) rectangle (2.87562,4.20748);
\draw [color=c, fill=c] (2.87562,4.10163) rectangle (2.91542,4.20748);
\draw [color=c, fill=c] (2.91542,4.10163) rectangle (2.95522,4.20748);
\draw [color=c, fill=c] (2.95522,4.10163) rectangle (2.99502,4.20748);
\draw [color=c, fill=c] (2.99502,4.10163) rectangle (3.03483,4.20748);
\draw [color=c, fill=c] (3.03483,4.10163) rectangle (3.07463,4.20748);
\draw [color=c, fill=c] (3.07463,4.10163) rectangle (3.11443,4.20748);
\draw [color=c, fill=c] (3.11443,4.10163) rectangle (3.15423,4.20748);
\draw [color=c, fill=c] (3.15423,4.10163) rectangle (3.19403,4.20748);
\draw [color=c, fill=c] (3.19403,4.10163) rectangle (3.23383,4.20748);
\draw [color=c, fill=c] (3.23383,4.10163) rectangle (3.27363,4.20748);
\draw [color=c, fill=c] (3.27363,4.10163) rectangle (3.31343,4.20748);
\draw [color=c, fill=c] (3.31343,4.10163) rectangle (3.35323,4.20748);
\draw [color=c, fill=c] (3.35323,4.10163) rectangle (3.39303,4.20748);
\draw [color=c, fill=c] (3.39303,4.10163) rectangle (3.43284,4.20748);
\draw [color=c, fill=c] (3.43284,4.10163) rectangle (3.47264,4.20748);
\draw [color=c, fill=c] (3.47264,4.10163) rectangle (3.51244,4.20748);
\draw [color=c, fill=c] (3.51244,4.10163) rectangle (3.55224,4.20748);
\draw [color=c, fill=c] (3.55224,4.10163) rectangle (3.59204,4.20748);
\draw [color=c, fill=c] (3.59204,4.10163) rectangle (3.63184,4.20748);
\draw [color=c, fill=c] (3.63184,4.10163) rectangle (3.67164,4.20748);
\draw [color=c, fill=c] (3.67164,4.10163) rectangle (3.71144,4.20748);
\draw [color=c, fill=c] (3.71144,4.10163) rectangle (3.75124,4.20748);
\draw [color=c, fill=c] (3.75124,4.10163) rectangle (3.79104,4.20748);
\draw [color=c, fill=c] (3.79104,4.10163) rectangle (3.83085,4.20748);
\draw [color=c, fill=c] (3.83085,4.10163) rectangle (3.87065,4.20748);
\draw [color=c, fill=c] (3.87065,4.10163) rectangle (3.91045,4.20748);
\draw [color=c, fill=c] (3.91045,4.10163) rectangle (3.95025,4.20748);
\draw [color=c, fill=c] (3.95025,4.10163) rectangle (3.99005,4.20748);
\draw [color=c, fill=c] (3.99005,4.10163) rectangle (4.02985,4.20748);
\draw [color=c, fill=c] (4.02985,4.10163) rectangle (4.06965,4.20748);
\draw [color=c, fill=c] (4.06965,4.10163) rectangle (4.10945,4.20748);
\draw [color=c, fill=c] (4.10945,4.10163) rectangle (4.14925,4.20748);
\draw [color=c, fill=c] (4.14925,4.10163) rectangle (4.18905,4.20748);
\draw [color=c, fill=c] (4.18905,4.10163) rectangle (4.22886,4.20748);
\draw [color=c, fill=c] (4.22886,4.10163) rectangle (4.26866,4.20748);
\draw [color=c, fill=c] (4.26866,4.10163) rectangle (4.30846,4.20748);
\draw [color=c, fill=c] (4.30846,4.10163) rectangle (4.34826,4.20748);
\draw [color=c, fill=c] (4.34826,4.10163) rectangle (4.38806,4.20748);
\draw [color=c, fill=c] (4.38806,4.10163) rectangle (4.42786,4.20748);
\draw [color=c, fill=c] (4.42786,4.10163) rectangle (4.46766,4.20748);
\draw [color=c, fill=c] (4.46766,4.10163) rectangle (4.50746,4.20748);
\draw [color=c, fill=c] (4.50746,4.10163) rectangle (4.54726,4.20748);
\draw [color=c, fill=c] (4.54726,4.10163) rectangle (4.58706,4.20748);
\draw [color=c, fill=c] (4.58706,4.10163) rectangle (4.62687,4.20748);
\draw [color=c, fill=c] (4.62687,4.10163) rectangle (4.66667,4.20748);
\draw [color=c, fill=c] (4.66667,4.10163) rectangle (4.70647,4.20748);
\draw [color=c, fill=c] (4.70647,4.10163) rectangle (4.74627,4.20748);
\draw [color=c, fill=c] (4.74627,4.10163) rectangle (4.78607,4.20748);
\draw [color=c, fill=c] (4.78607,4.10163) rectangle (4.82587,4.20748);
\draw [color=c, fill=c] (4.82587,4.10163) rectangle (4.86567,4.20748);
\draw [color=c, fill=c] (4.86567,4.10163) rectangle (4.90547,4.20748);
\draw [color=c, fill=c] (4.90547,4.10163) rectangle (4.94527,4.20748);
\draw [color=c, fill=c] (4.94527,4.10163) rectangle (4.98507,4.20748);
\draw [color=c, fill=c] (4.98507,4.10163) rectangle (5.02488,4.20748);
\draw [color=c, fill=c] (5.02488,4.10163) rectangle (5.06468,4.20748);
\draw [color=c, fill=c] (5.06468,4.10163) rectangle (5.10448,4.20748);
\draw [color=c, fill=c] (5.10448,4.10163) rectangle (5.14428,4.20748);
\draw [color=c, fill=c] (5.14428,4.10163) rectangle (5.18408,4.20748);
\draw [color=c, fill=c] (5.18408,4.10163) rectangle (5.22388,4.20748);
\draw [color=c, fill=c] (5.22388,4.10163) rectangle (5.26368,4.20748);
\draw [color=c, fill=c] (5.26368,4.10163) rectangle (5.30348,4.20748);
\draw [color=c, fill=c] (5.30348,4.10163) rectangle (5.34328,4.20748);
\draw [color=c, fill=c] (5.34328,4.10163) rectangle (5.38308,4.20748);
\draw [color=c, fill=c] (5.38308,4.10163) rectangle (5.42289,4.20748);
\draw [color=c, fill=c] (5.42289,4.10163) rectangle (5.46269,4.20748);
\draw [color=c, fill=c] (5.46269,4.10163) rectangle (5.50249,4.20748);
\draw [color=c, fill=c] (5.50249,4.10163) rectangle (5.54229,4.20748);
\draw [color=c, fill=c] (5.54229,4.10163) rectangle (5.58209,4.20748);
\draw [color=c, fill=c] (5.58209,4.10163) rectangle (5.62189,4.20748);
\draw [color=c, fill=c] (5.62189,4.10163) rectangle (5.66169,4.20748);
\draw [color=c, fill=c] (5.66169,4.10163) rectangle (5.70149,4.20748);
\draw [color=c, fill=c] (5.70149,4.10163) rectangle (5.74129,4.20748);
\draw [color=c, fill=c] (5.74129,4.10163) rectangle (5.78109,4.20748);
\draw [color=c, fill=c] (5.78109,4.10163) rectangle (5.8209,4.20748);
\draw [color=c, fill=c] (5.8209,4.10163) rectangle (5.8607,4.20748);
\draw [color=c, fill=c] (5.8607,4.10163) rectangle (5.9005,4.20748);
\definecolor{c}{rgb}{0.2,0,1};
\draw [color=c, fill=c] (5.9005,4.10163) rectangle (5.9403,4.20748);
\draw [color=c, fill=c] (5.9403,4.10163) rectangle (5.9801,4.20748);
\draw [color=c, fill=c] (5.9801,4.10163) rectangle (6.0199,4.20748);
\draw [color=c, fill=c] (6.0199,4.10163) rectangle (6.0597,4.20748);
\draw [color=c, fill=c] (6.0597,4.10163) rectangle (6.0995,4.20748);
\draw [color=c, fill=c] (6.0995,4.10163) rectangle (6.1393,4.20748);
\draw [color=c, fill=c] (6.1393,4.10163) rectangle (6.1791,4.20748);
\draw [color=c, fill=c] (6.1791,4.10163) rectangle (6.21891,4.20748);
\draw [color=c, fill=c] (6.21891,4.10163) rectangle (6.25871,4.20748);
\draw [color=c, fill=c] (6.25871,4.10163) rectangle (6.29851,4.20748);
\draw [color=c, fill=c] (6.29851,4.10163) rectangle (6.33831,4.20748);
\draw [color=c, fill=c] (6.33831,4.10163) rectangle (6.37811,4.20748);
\draw [color=c, fill=c] (6.37811,4.10163) rectangle (6.41791,4.20748);
\draw [color=c, fill=c] (6.41791,4.10163) rectangle (6.45771,4.20748);
\draw [color=c, fill=c] (6.45771,4.10163) rectangle (6.49751,4.20748);
\draw [color=c, fill=c] (6.49751,4.10163) rectangle (6.53731,4.20748);
\draw [color=c, fill=c] (6.53731,4.10163) rectangle (6.57711,4.20748);
\draw [color=c, fill=c] (6.57711,4.10163) rectangle (6.61692,4.20748);
\draw [color=c, fill=c] (6.61692,4.10163) rectangle (6.65672,4.20748);
\draw [color=c, fill=c] (6.65672,4.10163) rectangle (6.69652,4.20748);
\draw [color=c, fill=c] (6.69652,4.10163) rectangle (6.73632,4.20748);
\draw [color=c, fill=c] (6.73632,4.10163) rectangle (6.77612,4.20748);
\draw [color=c, fill=c] (6.77612,4.10163) rectangle (6.81592,4.20748);
\draw [color=c, fill=c] (6.81592,4.10163) rectangle (6.85572,4.20748);
\draw [color=c, fill=c] (6.85572,4.10163) rectangle (6.89552,4.20748);
\draw [color=c, fill=c] (6.89552,4.10163) rectangle (6.93532,4.20748);
\draw [color=c, fill=c] (6.93532,4.10163) rectangle (6.97512,4.20748);
\draw [color=c, fill=c] (6.97512,4.10163) rectangle (7.01493,4.20748);
\draw [color=c, fill=c] (7.01493,4.10163) rectangle (7.05473,4.20748);
\draw [color=c, fill=c] (7.05473,4.10163) rectangle (7.09453,4.20748);
\draw [color=c, fill=c] (7.09453,4.10163) rectangle (7.13433,4.20748);
\draw [color=c, fill=c] (7.13433,4.10163) rectangle (7.17413,4.20748);
\draw [color=c, fill=c] (7.17413,4.10163) rectangle (7.21393,4.20748);
\draw [color=c, fill=c] (7.21393,4.10163) rectangle (7.25373,4.20748);
\draw [color=c, fill=c] (7.25373,4.10163) rectangle (7.29353,4.20748);
\draw [color=c, fill=c] (7.29353,4.10163) rectangle (7.33333,4.20748);
\draw [color=c, fill=c] (7.33333,4.10163) rectangle (7.37313,4.20748);
\draw [color=c, fill=c] (7.37313,4.10163) rectangle (7.41294,4.20748);
\draw [color=c, fill=c] (7.41294,4.10163) rectangle (7.45274,4.20748);
\draw [color=c, fill=c] (7.45274,4.10163) rectangle (7.49254,4.20748);
\draw [color=c, fill=c] (7.49254,4.10163) rectangle (7.53234,4.20748);
\draw [color=c, fill=c] (7.53234,4.10163) rectangle (7.57214,4.20748);
\draw [color=c, fill=c] (7.57214,4.10163) rectangle (7.61194,4.20748);
\draw [color=c, fill=c] (7.61194,4.10163) rectangle (7.65174,4.20748);
\draw [color=c, fill=c] (7.65174,4.10163) rectangle (7.69154,4.20748);
\draw [color=c, fill=c] (7.69154,4.10163) rectangle (7.73134,4.20748);
\draw [color=c, fill=c] (7.73134,4.10163) rectangle (7.77114,4.20748);
\draw [color=c, fill=c] (7.77114,4.10163) rectangle (7.81095,4.20748);
\draw [color=c, fill=c] (7.81095,4.10163) rectangle (7.85075,4.20748);
\draw [color=c, fill=c] (7.85075,4.10163) rectangle (7.89055,4.20748);
\draw [color=c, fill=c] (7.89055,4.10163) rectangle (7.93035,4.20748);
\draw [color=c, fill=c] (7.93035,4.10163) rectangle (7.97015,4.20748);
\draw [color=c, fill=c] (7.97015,4.10163) rectangle (8.00995,4.20748);
\draw [color=c, fill=c] (8.00995,4.10163) rectangle (8.04975,4.20748);
\draw [color=c, fill=c] (8.04975,4.10163) rectangle (8.08955,4.20748);
\draw [color=c, fill=c] (8.08955,4.10163) rectangle (8.12935,4.20748);
\draw [color=c, fill=c] (8.12935,4.10163) rectangle (8.16915,4.20748);
\draw [color=c, fill=c] (8.16915,4.10163) rectangle (8.20895,4.20748);
\draw [color=c, fill=c] (8.20895,4.10163) rectangle (8.24876,4.20748);
\draw [color=c, fill=c] (8.24876,4.10163) rectangle (8.28856,4.20748);
\draw [color=c, fill=c] (8.28856,4.10163) rectangle (8.32836,4.20748);
\draw [color=c, fill=c] (8.32836,4.10163) rectangle (8.36816,4.20748);
\draw [color=c, fill=c] (8.36816,4.10163) rectangle (8.40796,4.20748);
\draw [color=c, fill=c] (8.40796,4.10163) rectangle (8.44776,4.20748);
\draw [color=c, fill=c] (8.44776,4.10163) rectangle (8.48756,4.20748);
\draw [color=c, fill=c] (8.48756,4.10163) rectangle (8.52736,4.20748);
\draw [color=c, fill=c] (8.52736,4.10163) rectangle (8.56716,4.20748);
\draw [color=c, fill=c] (8.56716,4.10163) rectangle (8.60697,4.20748);
\draw [color=c, fill=c] (8.60697,4.10163) rectangle (8.64677,4.20748);
\definecolor{c}{rgb}{0.386667,0,1};
\draw [color=c, fill=c] (8.64677,4.10163) rectangle (8.68657,4.20748);
\draw [color=c, fill=c] (8.68657,4.10163) rectangle (8.72637,4.20748);
\draw [color=c, fill=c] (8.72637,4.10163) rectangle (8.76617,4.20748);
\draw [color=c, fill=c] (8.76617,4.10163) rectangle (8.80597,4.20748);
\draw [color=c, fill=c] (8.80597,4.10163) rectangle (8.84577,4.20748);
\draw [color=c, fill=c] (8.84577,4.10163) rectangle (8.88557,4.20748);
\draw [color=c, fill=c] (8.88557,4.10163) rectangle (8.92537,4.20748);
\draw [color=c, fill=c] (8.92537,4.10163) rectangle (8.96517,4.20748);
\draw [color=c, fill=c] (8.96517,4.10163) rectangle (9.00498,4.20748);
\draw [color=c, fill=c] (9.00498,4.10163) rectangle (9.04478,4.20748);
\draw [color=c, fill=c] (9.04478,4.10163) rectangle (9.08458,4.20748);
\draw [color=c, fill=c] (9.08458,4.10163) rectangle (9.12438,4.20748);
\draw [color=c, fill=c] (9.12438,4.10163) rectangle (9.16418,4.20748);
\draw [color=c, fill=c] (9.16418,4.10163) rectangle (9.20398,4.20748);
\draw [color=c, fill=c] (9.20398,4.10163) rectangle (9.24378,4.20748);
\draw [color=c, fill=c] (9.24378,4.10163) rectangle (9.28358,4.20748);
\draw [color=c, fill=c] (9.28358,4.10163) rectangle (9.32338,4.20748);
\draw [color=c, fill=c] (9.32338,4.10163) rectangle (9.36318,4.20748);
\draw [color=c, fill=c] (9.36318,4.10163) rectangle (9.40298,4.20748);
\draw [color=c, fill=c] (9.40298,4.10163) rectangle (9.44279,4.20748);
\draw [color=c, fill=c] (9.44279,4.10163) rectangle (9.48259,4.20748);
\draw [color=c, fill=c] (9.48259,4.10163) rectangle (9.52239,4.20748);
\draw [color=c, fill=c] (9.52239,4.10163) rectangle (9.56219,4.20748);
\draw [color=c, fill=c] (9.56219,4.10163) rectangle (9.60199,4.20748);
\draw [color=c, fill=c] (9.60199,4.10163) rectangle (9.64179,4.20748);
\draw [color=c, fill=c] (9.64179,4.10163) rectangle (9.68159,4.20748);
\draw [color=c, fill=c] (9.68159,4.10163) rectangle (9.72139,4.20748);
\draw [color=c, fill=c] (9.72139,4.10163) rectangle (9.76119,4.20748);
\definecolor{c}{rgb}{0.2,0,1};
\draw [color=c, fill=c] (9.76119,4.10163) rectangle (9.80099,4.20748);
\draw [color=c, fill=c] (9.80099,4.10163) rectangle (9.8408,4.20748);
\draw [color=c, fill=c] (9.8408,4.10163) rectangle (9.8806,4.20748);
\definecolor{c}{rgb}{0,0.0800001,1};
\draw [color=c, fill=c] (9.8806,4.10163) rectangle (9.9204,4.20748);
\definecolor{c}{rgb}{0,0.266667,1};
\draw [color=c, fill=c] (9.9204,4.10163) rectangle (9.9602,4.20748);
\definecolor{c}{rgb}{0,0.546666,1};
\draw [color=c, fill=c] (9.9602,4.10163) rectangle (10,4.20748);
\draw [color=c, fill=c] (10,4.10163) rectangle (10.0398,4.20748);
\definecolor{c}{rgb}{0,0.733333,1};
\draw [color=c, fill=c] (10.0398,4.10163) rectangle (10.0796,4.20748);
\draw [color=c, fill=c] (10.0796,4.10163) rectangle (10.1194,4.20748);
\definecolor{c}{rgb}{0,1,0.986667};
\draw [color=c, fill=c] (10.1194,4.10163) rectangle (10.1592,4.20748);
\draw [color=c, fill=c] (10.1592,4.10163) rectangle (10.199,4.20748);
\draw [color=c, fill=c] (10.199,4.10163) rectangle (10.2388,4.20748);
\draw [color=c, fill=c] (10.2388,4.10163) rectangle (10.2786,4.20748);
\draw [color=c, fill=c] (10.2786,4.10163) rectangle (10.3184,4.20748);
\draw [color=c, fill=c] (10.3184,4.10163) rectangle (10.3582,4.20748);
\draw [color=c, fill=c] (10.3582,4.10163) rectangle (10.398,4.20748);
\draw [color=c, fill=c] (10.398,4.10163) rectangle (10.4378,4.20748);
\draw [color=c, fill=c] (10.4378,4.10163) rectangle (10.4776,4.20748);
\draw [color=c, fill=c] (10.4776,4.10163) rectangle (10.5174,4.20748);
\draw [color=c, fill=c] (10.5174,4.10163) rectangle (10.5572,4.20748);
\draw [color=c, fill=c] (10.5572,4.10163) rectangle (10.597,4.20748);
\draw [color=c, fill=c] (10.597,4.10163) rectangle (10.6368,4.20748);
\draw [color=c, fill=c] (10.6368,4.10163) rectangle (10.6766,4.20748);
\draw [color=c, fill=c] (10.6766,4.10163) rectangle (10.7164,4.20748);
\draw [color=c, fill=c] (10.7164,4.10163) rectangle (10.7562,4.20748);
\draw [color=c, fill=c] (10.7562,4.10163) rectangle (10.796,4.20748);
\draw [color=c, fill=c] (10.796,4.10163) rectangle (10.8358,4.20748);
\draw [color=c, fill=c] (10.8358,4.10163) rectangle (10.8756,4.20748);
\draw [color=c, fill=c] (10.8756,4.10163) rectangle (10.9154,4.20748);
\draw [color=c, fill=c] (10.9154,4.10163) rectangle (10.9552,4.20748);
\draw [color=c, fill=c] (10.9552,4.10163) rectangle (10.995,4.20748);
\draw [color=c, fill=c] (10.995,4.10163) rectangle (11.0348,4.20748);
\draw [color=c, fill=c] (11.0348,4.10163) rectangle (11.0746,4.20748);
\draw [color=c, fill=c] (11.0746,4.10163) rectangle (11.1144,4.20748);
\draw [color=c, fill=c] (11.1144,4.10163) rectangle (11.1542,4.20748);
\draw [color=c, fill=c] (11.1542,4.10163) rectangle (11.194,4.20748);
\draw [color=c, fill=c] (11.194,4.10163) rectangle (11.2338,4.20748);
\draw [color=c, fill=c] (11.2338,4.10163) rectangle (11.2736,4.20748);
\draw [color=c, fill=c] (11.2736,4.10163) rectangle (11.3134,4.20748);
\draw [color=c, fill=c] (11.3134,4.10163) rectangle (11.3532,4.20748);
\draw [color=c, fill=c] (11.3532,4.10163) rectangle (11.393,4.20748);
\draw [color=c, fill=c] (11.393,4.10163) rectangle (11.4328,4.20748);
\draw [color=c, fill=c] (11.4328,4.10163) rectangle (11.4726,4.20748);
\draw [color=c, fill=c] (11.4726,4.10163) rectangle (11.5124,4.20748);
\draw [color=c, fill=c] (11.5124,4.10163) rectangle (11.5522,4.20748);
\draw [color=c, fill=c] (11.5522,4.10163) rectangle (11.592,4.20748);
\draw [color=c, fill=c] (11.592,4.10163) rectangle (11.6318,4.20748);
\draw [color=c, fill=c] (11.6318,4.10163) rectangle (11.6716,4.20748);
\draw [color=c, fill=c] (11.6716,4.10163) rectangle (11.7114,4.20748);
\draw [color=c, fill=c] (11.7114,4.10163) rectangle (11.7512,4.20748);
\definecolor{c}{rgb}{0,0.733333,1};
\draw [color=c, fill=c] (11.7512,4.10163) rectangle (11.791,4.20748);
\draw [color=c, fill=c] (11.791,4.10163) rectangle (11.8308,4.20748);
\draw [color=c, fill=c] (11.8308,4.10163) rectangle (11.8706,4.20748);
\draw [color=c, fill=c] (11.8706,4.10163) rectangle (11.9104,4.20748);
\draw [color=c, fill=c] (11.9104,4.10163) rectangle (11.9502,4.20748);
\draw [color=c, fill=c] (11.9502,4.10163) rectangle (11.99,4.20748);
\draw [color=c, fill=c] (11.99,4.10163) rectangle (12.0299,4.20748);
\draw [color=c, fill=c] (12.0299,4.10163) rectangle (12.0697,4.20748);
\draw [color=c, fill=c] (12.0697,4.10163) rectangle (12.1095,4.20748);
\draw [color=c, fill=c] (12.1095,4.10163) rectangle (12.1493,4.20748);
\draw [color=c, fill=c] (12.1493,4.10163) rectangle (12.1891,4.20748);
\draw [color=c, fill=c] (12.1891,4.10163) rectangle (12.2289,4.20748);
\draw [color=c, fill=c] (12.2289,4.10163) rectangle (12.2687,4.20748);
\draw [color=c, fill=c] (12.2687,4.10163) rectangle (12.3085,4.20748);
\draw [color=c, fill=c] (12.3085,4.10163) rectangle (12.3483,4.20748);
\draw [color=c, fill=c] (12.3483,4.10163) rectangle (12.3881,4.20748);
\draw [color=c, fill=c] (12.3881,4.10163) rectangle (12.4279,4.20748);
\draw [color=c, fill=c] (12.4279,4.10163) rectangle (12.4677,4.20748);
\draw [color=c, fill=c] (12.4677,4.10163) rectangle (12.5075,4.20748);
\draw [color=c, fill=c] (12.5075,4.10163) rectangle (12.5473,4.20748);
\draw [color=c, fill=c] (12.5473,4.10163) rectangle (12.5871,4.20748);
\draw [color=c, fill=c] (12.5871,4.10163) rectangle (12.6269,4.20748);
\draw [color=c, fill=c] (12.6269,4.10163) rectangle (12.6667,4.20748);
\draw [color=c, fill=c] (12.6667,4.10163) rectangle (12.7065,4.20748);
\draw [color=c, fill=c] (12.7065,4.10163) rectangle (12.7463,4.20748);
\draw [color=c, fill=c] (12.7463,4.10163) rectangle (12.7861,4.20748);
\draw [color=c, fill=c] (12.7861,4.10163) rectangle (12.8259,4.20748);
\draw [color=c, fill=c] (12.8259,4.10163) rectangle (12.8657,4.20748);
\draw [color=c, fill=c] (12.8657,4.10163) rectangle (12.9055,4.20748);
\draw [color=c, fill=c] (12.9055,4.10163) rectangle (12.9453,4.20748);
\draw [color=c, fill=c] (12.9453,4.10163) rectangle (12.9851,4.20748);
\draw [color=c, fill=c] (12.9851,4.10163) rectangle (13.0249,4.20748);
\draw [color=c, fill=c] (13.0249,4.10163) rectangle (13.0647,4.20748);
\draw [color=c, fill=c] (13.0647,4.10163) rectangle (13.1045,4.20748);
\draw [color=c, fill=c] (13.1045,4.10163) rectangle (13.1443,4.20748);
\draw [color=c, fill=c] (13.1443,4.10163) rectangle (13.1841,4.20748);
\draw [color=c, fill=c] (13.1841,4.10163) rectangle (13.2239,4.20748);
\draw [color=c, fill=c] (13.2239,4.10163) rectangle (13.2637,4.20748);
\draw [color=c, fill=c] (13.2637,4.10163) rectangle (13.3035,4.20748);
\draw [color=c, fill=c] (13.3035,4.10163) rectangle (13.3433,4.20748);
\draw [color=c, fill=c] (13.3433,4.10163) rectangle (13.3831,4.20748);
\draw [color=c, fill=c] (13.3831,4.10163) rectangle (13.4229,4.20748);
\draw [color=c, fill=c] (13.4229,4.10163) rectangle (13.4627,4.20748);
\draw [color=c, fill=c] (13.4627,4.10163) rectangle (13.5025,4.20748);
\draw [color=c, fill=c] (13.5025,4.10163) rectangle (13.5423,4.20748);
\draw [color=c, fill=c] (13.5423,4.10163) rectangle (13.5821,4.20748);
\draw [color=c, fill=c] (13.5821,4.10163) rectangle (13.6219,4.20748);
\draw [color=c, fill=c] (13.6219,4.10163) rectangle (13.6617,4.20748);
\draw [color=c, fill=c] (13.6617,4.10163) rectangle (13.7015,4.20748);
\draw [color=c, fill=c] (13.7015,4.10163) rectangle (13.7413,4.20748);
\draw [color=c, fill=c] (13.7413,4.10163) rectangle (13.7811,4.20748);
\draw [color=c, fill=c] (13.7811,4.10163) rectangle (13.8209,4.20748);
\draw [color=c, fill=c] (13.8209,4.10163) rectangle (13.8607,4.20748);
\draw [color=c, fill=c] (13.8607,4.10163) rectangle (13.9005,4.20748);
\draw [color=c, fill=c] (13.9005,4.10163) rectangle (13.9403,4.20748);
\draw [color=c, fill=c] (13.9403,4.10163) rectangle (13.9801,4.20748);
\draw [color=c, fill=c] (13.9801,4.10163) rectangle (14.0199,4.20748);
\draw [color=c, fill=c] (14.0199,4.10163) rectangle (14.0597,4.20748);
\draw [color=c, fill=c] (14.0597,4.10163) rectangle (14.0995,4.20748);
\draw [color=c, fill=c] (14.0995,4.10163) rectangle (14.1393,4.20748);
\draw [color=c, fill=c] (14.1393,4.10163) rectangle (14.1791,4.20748);
\draw [color=c, fill=c] (14.1791,4.10163) rectangle (14.2189,4.20748);
\draw [color=c, fill=c] (14.2189,4.10163) rectangle (14.2587,4.20748);
\draw [color=c, fill=c] (14.2587,4.10163) rectangle (14.2985,4.20748);
\draw [color=c, fill=c] (14.2985,4.10163) rectangle (14.3383,4.20748);
\draw [color=c, fill=c] (14.3383,4.10163) rectangle (14.3781,4.20748);
\draw [color=c, fill=c] (14.3781,4.10163) rectangle (14.4179,4.20748);
\draw [color=c, fill=c] (14.4179,4.10163) rectangle (14.4577,4.20748);
\draw [color=c, fill=c] (14.4577,4.10163) rectangle (14.4975,4.20748);
\draw [color=c, fill=c] (14.4975,4.10163) rectangle (14.5373,4.20748);
\draw [color=c, fill=c] (14.5373,4.10163) rectangle (14.5771,4.20748);
\draw [color=c, fill=c] (14.5771,4.10163) rectangle (14.6169,4.20748);
\draw [color=c, fill=c] (14.6169,4.10163) rectangle (14.6567,4.20748);
\draw [color=c, fill=c] (14.6567,4.10163) rectangle (14.6965,4.20748);
\draw [color=c, fill=c] (14.6965,4.10163) rectangle (14.7363,4.20748);
\draw [color=c, fill=c] (14.7363,4.10163) rectangle (14.7761,4.20748);
\draw [color=c, fill=c] (14.7761,4.10163) rectangle (14.8159,4.20748);
\draw [color=c, fill=c] (14.8159,4.10163) rectangle (14.8557,4.20748);
\draw [color=c, fill=c] (14.8557,4.10163) rectangle (14.8955,4.20748);
\draw [color=c, fill=c] (14.8955,4.10163) rectangle (14.9353,4.20748);
\draw [color=c, fill=c] (14.9353,4.10163) rectangle (14.9751,4.20748);
\draw [color=c, fill=c] (14.9751,4.10163) rectangle (15.0149,4.20748);
\draw [color=c, fill=c] (15.0149,4.10163) rectangle (15.0547,4.20748);
\draw [color=c, fill=c] (15.0547,4.10163) rectangle (15.0945,4.20748);
\draw [color=c, fill=c] (15.0945,4.10163) rectangle (15.1343,4.20748);
\draw [color=c, fill=c] (15.1343,4.10163) rectangle (15.1741,4.20748);
\draw [color=c, fill=c] (15.1741,4.10163) rectangle (15.2139,4.20748);
\draw [color=c, fill=c] (15.2139,4.10163) rectangle (15.2537,4.20748);
\draw [color=c, fill=c] (15.2537,4.10163) rectangle (15.2935,4.20748);
\draw [color=c, fill=c] (15.2935,4.10163) rectangle (15.3333,4.20748);
\draw [color=c, fill=c] (15.3333,4.10163) rectangle (15.3731,4.20748);
\draw [color=c, fill=c] (15.3731,4.10163) rectangle (15.4129,4.20748);
\draw [color=c, fill=c] (15.4129,4.10163) rectangle (15.4527,4.20748);
\draw [color=c, fill=c] (15.4527,4.10163) rectangle (15.4925,4.20748);
\draw [color=c, fill=c] (15.4925,4.10163) rectangle (15.5323,4.20748);
\draw [color=c, fill=c] (15.5323,4.10163) rectangle (15.5721,4.20748);
\draw [color=c, fill=c] (15.5721,4.10163) rectangle (15.6119,4.20748);
\draw [color=c, fill=c] (15.6119,4.10163) rectangle (15.6517,4.20748);
\draw [color=c, fill=c] (15.6517,4.10163) rectangle (15.6915,4.20748);
\draw [color=c, fill=c] (15.6915,4.10163) rectangle (15.7313,4.20748);
\draw [color=c, fill=c] (15.7313,4.10163) rectangle (15.7711,4.20748);
\draw [color=c, fill=c] (15.7711,4.10163) rectangle (15.8109,4.20748);
\draw [color=c, fill=c] (15.8109,4.10163) rectangle (15.8507,4.20748);
\draw [color=c, fill=c] (15.8507,4.10163) rectangle (15.8905,4.20748);
\draw [color=c, fill=c] (15.8905,4.10163) rectangle (15.9303,4.20748);
\draw [color=c, fill=c] (15.9303,4.10163) rectangle (15.9701,4.20748);
\draw [color=c, fill=c] (15.9701,4.10163) rectangle (16.01,4.20748);
\draw [color=c, fill=c] (16.01,4.10163) rectangle (16.0498,4.20748);
\draw [color=c, fill=c] (16.0498,4.10163) rectangle (16.0896,4.20748);
\draw [color=c, fill=c] (16.0896,4.10163) rectangle (16.1294,4.20748);
\draw [color=c, fill=c] (16.1294,4.10163) rectangle (16.1692,4.20748);
\draw [color=c, fill=c] (16.1692,4.10163) rectangle (16.209,4.20748);
\draw [color=c, fill=c] (16.209,4.10163) rectangle (16.2488,4.20748);
\draw [color=c, fill=c] (16.2488,4.10163) rectangle (16.2886,4.20748);
\draw [color=c, fill=c] (16.2886,4.10163) rectangle (16.3284,4.20748);
\draw [color=c, fill=c] (16.3284,4.10163) rectangle (16.3682,4.20748);
\draw [color=c, fill=c] (16.3682,4.10163) rectangle (16.408,4.20748);
\draw [color=c, fill=c] (16.408,4.10163) rectangle (16.4478,4.20748);
\draw [color=c, fill=c] (16.4478,4.10163) rectangle (16.4876,4.20748);
\draw [color=c, fill=c] (16.4876,4.10163) rectangle (16.5274,4.20748);
\draw [color=c, fill=c] (16.5274,4.10163) rectangle (16.5672,4.20748);
\draw [color=c, fill=c] (16.5672,4.10163) rectangle (16.607,4.20748);
\draw [color=c, fill=c] (16.607,4.10163) rectangle (16.6468,4.20748);
\draw [color=c, fill=c] (16.6468,4.10163) rectangle (16.6866,4.20748);
\draw [color=c, fill=c] (16.6866,4.10163) rectangle (16.7264,4.20748);
\draw [color=c, fill=c] (16.7264,4.10163) rectangle (16.7662,4.20748);
\draw [color=c, fill=c] (16.7662,4.10163) rectangle (16.806,4.20748);
\draw [color=c, fill=c] (16.806,4.10163) rectangle (16.8458,4.20748);
\draw [color=c, fill=c] (16.8458,4.10163) rectangle (16.8856,4.20748);
\draw [color=c, fill=c] (16.8856,4.10163) rectangle (16.9254,4.20748);
\draw [color=c, fill=c] (16.9254,4.10163) rectangle (16.9652,4.20748);
\draw [color=c, fill=c] (16.9652,4.10163) rectangle (17.005,4.20748);
\draw [color=c, fill=c] (17.005,4.10163) rectangle (17.0448,4.20748);
\draw [color=c, fill=c] (17.0448,4.10163) rectangle (17.0846,4.20748);
\draw [color=c, fill=c] (17.0846,4.10163) rectangle (17.1244,4.20748);
\draw [color=c, fill=c] (17.1244,4.10163) rectangle (17.1642,4.20748);
\draw [color=c, fill=c] (17.1642,4.10163) rectangle (17.204,4.20748);
\draw [color=c, fill=c] (17.204,4.10163) rectangle (17.2438,4.20748);
\draw [color=c, fill=c] (17.2438,4.10163) rectangle (17.2836,4.20748);
\draw [color=c, fill=c] (17.2836,4.10163) rectangle (17.3234,4.20748);
\draw [color=c, fill=c] (17.3234,4.10163) rectangle (17.3632,4.20748);
\draw [color=c, fill=c] (17.3632,4.10163) rectangle (17.403,4.20748);
\draw [color=c, fill=c] (17.403,4.10163) rectangle (17.4428,4.20748);
\draw [color=c, fill=c] (17.4428,4.10163) rectangle (17.4826,4.20748);
\draw [color=c, fill=c] (17.4826,4.10163) rectangle (17.5224,4.20748);
\draw [color=c, fill=c] (17.5224,4.10163) rectangle (17.5622,4.20748);
\draw [color=c, fill=c] (17.5622,4.10163) rectangle (17.602,4.20748);
\draw [color=c, fill=c] (17.602,4.10163) rectangle (17.6418,4.20748);
\draw [color=c, fill=c] (17.6418,4.10163) rectangle (17.6816,4.20748);
\draw [color=c, fill=c] (17.6816,4.10163) rectangle (17.7214,4.20748);
\draw [color=c, fill=c] (17.7214,4.10163) rectangle (17.7612,4.20748);
\draw [color=c, fill=c] (17.7612,4.10163) rectangle (17.801,4.20748);
\draw [color=c, fill=c] (17.801,4.10163) rectangle (17.8408,4.20748);
\draw [color=c, fill=c] (17.8408,4.10163) rectangle (17.8806,4.20748);
\draw [color=c, fill=c] (17.8806,4.10163) rectangle (17.9204,4.20748);
\draw [color=c, fill=c] (17.9204,4.10163) rectangle (17.9602,4.20748);
\draw [color=c, fill=c] (17.9602,4.10163) rectangle (18,4.20748);
\definecolor{c}{rgb}{0,0.0800001,1};
\draw [color=c, fill=c] (2,4.20748) rectangle (2.0398,4.31333);
\draw [color=c, fill=c] (2.0398,4.20748) rectangle (2.0796,4.31333);
\draw [color=c, fill=c] (2.0796,4.20748) rectangle (2.1194,4.31333);
\draw [color=c, fill=c] (2.1194,4.20748) rectangle (2.1592,4.31333);
\draw [color=c, fill=c] (2.1592,4.20748) rectangle (2.19901,4.31333);
\draw [color=c, fill=c] (2.19901,4.20748) rectangle (2.23881,4.31333);
\draw [color=c, fill=c] (2.23881,4.20748) rectangle (2.27861,4.31333);
\draw [color=c, fill=c] (2.27861,4.20748) rectangle (2.31841,4.31333);
\draw [color=c, fill=c] (2.31841,4.20748) rectangle (2.35821,4.31333);
\draw [color=c, fill=c] (2.35821,4.20748) rectangle (2.39801,4.31333);
\draw [color=c, fill=c] (2.39801,4.20748) rectangle (2.43781,4.31333);
\draw [color=c, fill=c] (2.43781,4.20748) rectangle (2.47761,4.31333);
\draw [color=c, fill=c] (2.47761,4.20748) rectangle (2.51741,4.31333);
\draw [color=c, fill=c] (2.51741,4.20748) rectangle (2.55721,4.31333);
\draw [color=c, fill=c] (2.55721,4.20748) rectangle (2.59702,4.31333);
\draw [color=c, fill=c] (2.59702,4.20748) rectangle (2.63682,4.31333);
\draw [color=c, fill=c] (2.63682,4.20748) rectangle (2.67662,4.31333);
\draw [color=c, fill=c] (2.67662,4.20748) rectangle (2.71642,4.31333);
\draw [color=c, fill=c] (2.71642,4.20748) rectangle (2.75622,4.31333);
\draw [color=c, fill=c] (2.75622,4.20748) rectangle (2.79602,4.31333);
\draw [color=c, fill=c] (2.79602,4.20748) rectangle (2.83582,4.31333);
\draw [color=c, fill=c] (2.83582,4.20748) rectangle (2.87562,4.31333);
\draw [color=c, fill=c] (2.87562,4.20748) rectangle (2.91542,4.31333);
\draw [color=c, fill=c] (2.91542,4.20748) rectangle (2.95522,4.31333);
\draw [color=c, fill=c] (2.95522,4.20748) rectangle (2.99502,4.31333);
\draw [color=c, fill=c] (2.99502,4.20748) rectangle (3.03483,4.31333);
\draw [color=c, fill=c] (3.03483,4.20748) rectangle (3.07463,4.31333);
\draw [color=c, fill=c] (3.07463,4.20748) rectangle (3.11443,4.31333);
\draw [color=c, fill=c] (3.11443,4.20748) rectangle (3.15423,4.31333);
\draw [color=c, fill=c] (3.15423,4.20748) rectangle (3.19403,4.31333);
\draw [color=c, fill=c] (3.19403,4.20748) rectangle (3.23383,4.31333);
\draw [color=c, fill=c] (3.23383,4.20748) rectangle (3.27363,4.31333);
\draw [color=c, fill=c] (3.27363,4.20748) rectangle (3.31343,4.31333);
\draw [color=c, fill=c] (3.31343,4.20748) rectangle (3.35323,4.31333);
\draw [color=c, fill=c] (3.35323,4.20748) rectangle (3.39303,4.31333);
\draw [color=c, fill=c] (3.39303,4.20748) rectangle (3.43284,4.31333);
\draw [color=c, fill=c] (3.43284,4.20748) rectangle (3.47264,4.31333);
\draw [color=c, fill=c] (3.47264,4.20748) rectangle (3.51244,4.31333);
\draw [color=c, fill=c] (3.51244,4.20748) rectangle (3.55224,4.31333);
\draw [color=c, fill=c] (3.55224,4.20748) rectangle (3.59204,4.31333);
\draw [color=c, fill=c] (3.59204,4.20748) rectangle (3.63184,4.31333);
\draw [color=c, fill=c] (3.63184,4.20748) rectangle (3.67164,4.31333);
\draw [color=c, fill=c] (3.67164,4.20748) rectangle (3.71144,4.31333);
\draw [color=c, fill=c] (3.71144,4.20748) rectangle (3.75124,4.31333);
\draw [color=c, fill=c] (3.75124,4.20748) rectangle (3.79104,4.31333);
\draw [color=c, fill=c] (3.79104,4.20748) rectangle (3.83085,4.31333);
\draw [color=c, fill=c] (3.83085,4.20748) rectangle (3.87065,4.31333);
\draw [color=c, fill=c] (3.87065,4.20748) rectangle (3.91045,4.31333);
\draw [color=c, fill=c] (3.91045,4.20748) rectangle (3.95025,4.31333);
\draw [color=c, fill=c] (3.95025,4.20748) rectangle (3.99005,4.31333);
\draw [color=c, fill=c] (3.99005,4.20748) rectangle (4.02985,4.31333);
\draw [color=c, fill=c] (4.02985,4.20748) rectangle (4.06965,4.31333);
\draw [color=c, fill=c] (4.06965,4.20748) rectangle (4.10945,4.31333);
\draw [color=c, fill=c] (4.10945,4.20748) rectangle (4.14925,4.31333);
\draw [color=c, fill=c] (4.14925,4.20748) rectangle (4.18905,4.31333);
\draw [color=c, fill=c] (4.18905,4.20748) rectangle (4.22886,4.31333);
\draw [color=c, fill=c] (4.22886,4.20748) rectangle (4.26866,4.31333);
\draw [color=c, fill=c] (4.26866,4.20748) rectangle (4.30846,4.31333);
\draw [color=c, fill=c] (4.30846,4.20748) rectangle (4.34826,4.31333);
\draw [color=c, fill=c] (4.34826,4.20748) rectangle (4.38806,4.31333);
\draw [color=c, fill=c] (4.38806,4.20748) rectangle (4.42786,4.31333);
\draw [color=c, fill=c] (4.42786,4.20748) rectangle (4.46766,4.31333);
\draw [color=c, fill=c] (4.46766,4.20748) rectangle (4.50746,4.31333);
\draw [color=c, fill=c] (4.50746,4.20748) rectangle (4.54726,4.31333);
\draw [color=c, fill=c] (4.54726,4.20748) rectangle (4.58706,4.31333);
\draw [color=c, fill=c] (4.58706,4.20748) rectangle (4.62687,4.31333);
\draw [color=c, fill=c] (4.62687,4.20748) rectangle (4.66667,4.31333);
\draw [color=c, fill=c] (4.66667,4.20748) rectangle (4.70647,4.31333);
\draw [color=c, fill=c] (4.70647,4.20748) rectangle (4.74627,4.31333);
\draw [color=c, fill=c] (4.74627,4.20748) rectangle (4.78607,4.31333);
\draw [color=c, fill=c] (4.78607,4.20748) rectangle (4.82587,4.31333);
\draw [color=c, fill=c] (4.82587,4.20748) rectangle (4.86567,4.31333);
\draw [color=c, fill=c] (4.86567,4.20748) rectangle (4.90547,4.31333);
\draw [color=c, fill=c] (4.90547,4.20748) rectangle (4.94527,4.31333);
\draw [color=c, fill=c] (4.94527,4.20748) rectangle (4.98507,4.31333);
\draw [color=c, fill=c] (4.98507,4.20748) rectangle (5.02488,4.31333);
\draw [color=c, fill=c] (5.02488,4.20748) rectangle (5.06468,4.31333);
\draw [color=c, fill=c] (5.06468,4.20748) rectangle (5.10448,4.31333);
\draw [color=c, fill=c] (5.10448,4.20748) rectangle (5.14428,4.31333);
\draw [color=c, fill=c] (5.14428,4.20748) rectangle (5.18408,4.31333);
\draw [color=c, fill=c] (5.18408,4.20748) rectangle (5.22388,4.31333);
\draw [color=c, fill=c] (5.22388,4.20748) rectangle (5.26368,4.31333);
\draw [color=c, fill=c] (5.26368,4.20748) rectangle (5.30348,4.31333);
\draw [color=c, fill=c] (5.30348,4.20748) rectangle (5.34328,4.31333);
\draw [color=c, fill=c] (5.34328,4.20748) rectangle (5.38308,4.31333);
\draw [color=c, fill=c] (5.38308,4.20748) rectangle (5.42289,4.31333);
\draw [color=c, fill=c] (5.42289,4.20748) rectangle (5.46269,4.31333);
\draw [color=c, fill=c] (5.46269,4.20748) rectangle (5.50249,4.31333);
\draw [color=c, fill=c] (5.50249,4.20748) rectangle (5.54229,4.31333);
\draw [color=c, fill=c] (5.54229,4.20748) rectangle (5.58209,4.31333);
\draw [color=c, fill=c] (5.58209,4.20748) rectangle (5.62189,4.31333);
\draw [color=c, fill=c] (5.62189,4.20748) rectangle (5.66169,4.31333);
\draw [color=c, fill=c] (5.66169,4.20748) rectangle (5.70149,4.31333);
\draw [color=c, fill=c] (5.70149,4.20748) rectangle (5.74129,4.31333);
\draw [color=c, fill=c] (5.74129,4.20748) rectangle (5.78109,4.31333);
\draw [color=c, fill=c] (5.78109,4.20748) rectangle (5.8209,4.31333);
\draw [color=c, fill=c] (5.8209,4.20748) rectangle (5.8607,4.31333);
\draw [color=c, fill=c] (5.8607,4.20748) rectangle (5.9005,4.31333);
\definecolor{c}{rgb}{0.2,0,1};
\draw [color=c, fill=c] (5.9005,4.20748) rectangle (5.9403,4.31333);
\draw [color=c, fill=c] (5.9403,4.20748) rectangle (5.9801,4.31333);
\draw [color=c, fill=c] (5.9801,4.20748) rectangle (6.0199,4.31333);
\draw [color=c, fill=c] (6.0199,4.20748) rectangle (6.0597,4.31333);
\draw [color=c, fill=c] (6.0597,4.20748) rectangle (6.0995,4.31333);
\draw [color=c, fill=c] (6.0995,4.20748) rectangle (6.1393,4.31333);
\draw [color=c, fill=c] (6.1393,4.20748) rectangle (6.1791,4.31333);
\draw [color=c, fill=c] (6.1791,4.20748) rectangle (6.21891,4.31333);
\draw [color=c, fill=c] (6.21891,4.20748) rectangle (6.25871,4.31333);
\draw [color=c, fill=c] (6.25871,4.20748) rectangle (6.29851,4.31333);
\draw [color=c, fill=c] (6.29851,4.20748) rectangle (6.33831,4.31333);
\draw [color=c, fill=c] (6.33831,4.20748) rectangle (6.37811,4.31333);
\draw [color=c, fill=c] (6.37811,4.20748) rectangle (6.41791,4.31333);
\draw [color=c, fill=c] (6.41791,4.20748) rectangle (6.45771,4.31333);
\draw [color=c, fill=c] (6.45771,4.20748) rectangle (6.49751,4.31333);
\draw [color=c, fill=c] (6.49751,4.20748) rectangle (6.53731,4.31333);
\draw [color=c, fill=c] (6.53731,4.20748) rectangle (6.57711,4.31333);
\draw [color=c, fill=c] (6.57711,4.20748) rectangle (6.61692,4.31333);
\draw [color=c, fill=c] (6.61692,4.20748) rectangle (6.65672,4.31333);
\draw [color=c, fill=c] (6.65672,4.20748) rectangle (6.69652,4.31333);
\draw [color=c, fill=c] (6.69652,4.20748) rectangle (6.73632,4.31333);
\draw [color=c, fill=c] (6.73632,4.20748) rectangle (6.77612,4.31333);
\draw [color=c, fill=c] (6.77612,4.20748) rectangle (6.81592,4.31333);
\draw [color=c, fill=c] (6.81592,4.20748) rectangle (6.85572,4.31333);
\draw [color=c, fill=c] (6.85572,4.20748) rectangle (6.89552,4.31333);
\draw [color=c, fill=c] (6.89552,4.20748) rectangle (6.93532,4.31333);
\draw [color=c, fill=c] (6.93532,4.20748) rectangle (6.97512,4.31333);
\draw [color=c, fill=c] (6.97512,4.20748) rectangle (7.01493,4.31333);
\draw [color=c, fill=c] (7.01493,4.20748) rectangle (7.05473,4.31333);
\draw [color=c, fill=c] (7.05473,4.20748) rectangle (7.09453,4.31333);
\draw [color=c, fill=c] (7.09453,4.20748) rectangle (7.13433,4.31333);
\draw [color=c, fill=c] (7.13433,4.20748) rectangle (7.17413,4.31333);
\draw [color=c, fill=c] (7.17413,4.20748) rectangle (7.21393,4.31333);
\draw [color=c, fill=c] (7.21393,4.20748) rectangle (7.25373,4.31333);
\draw [color=c, fill=c] (7.25373,4.20748) rectangle (7.29353,4.31333);
\draw [color=c, fill=c] (7.29353,4.20748) rectangle (7.33333,4.31333);
\draw [color=c, fill=c] (7.33333,4.20748) rectangle (7.37313,4.31333);
\draw [color=c, fill=c] (7.37313,4.20748) rectangle (7.41294,4.31333);
\draw [color=c, fill=c] (7.41294,4.20748) rectangle (7.45274,4.31333);
\draw [color=c, fill=c] (7.45274,4.20748) rectangle (7.49254,4.31333);
\draw [color=c, fill=c] (7.49254,4.20748) rectangle (7.53234,4.31333);
\draw [color=c, fill=c] (7.53234,4.20748) rectangle (7.57214,4.31333);
\draw [color=c, fill=c] (7.57214,4.20748) rectangle (7.61194,4.31333);
\draw [color=c, fill=c] (7.61194,4.20748) rectangle (7.65174,4.31333);
\draw [color=c, fill=c] (7.65174,4.20748) rectangle (7.69154,4.31333);
\draw [color=c, fill=c] (7.69154,4.20748) rectangle (7.73134,4.31333);
\draw [color=c, fill=c] (7.73134,4.20748) rectangle (7.77114,4.31333);
\draw [color=c, fill=c] (7.77114,4.20748) rectangle (7.81095,4.31333);
\draw [color=c, fill=c] (7.81095,4.20748) rectangle (7.85075,4.31333);
\draw [color=c, fill=c] (7.85075,4.20748) rectangle (7.89055,4.31333);
\draw [color=c, fill=c] (7.89055,4.20748) rectangle (7.93035,4.31333);
\draw [color=c, fill=c] (7.93035,4.20748) rectangle (7.97015,4.31333);
\draw [color=c, fill=c] (7.97015,4.20748) rectangle (8.00995,4.31333);
\draw [color=c, fill=c] (8.00995,4.20748) rectangle (8.04975,4.31333);
\draw [color=c, fill=c] (8.04975,4.20748) rectangle (8.08955,4.31333);
\draw [color=c, fill=c] (8.08955,4.20748) rectangle (8.12935,4.31333);
\draw [color=c, fill=c] (8.12935,4.20748) rectangle (8.16915,4.31333);
\draw [color=c, fill=c] (8.16915,4.20748) rectangle (8.20895,4.31333);
\draw [color=c, fill=c] (8.20895,4.20748) rectangle (8.24876,4.31333);
\draw [color=c, fill=c] (8.24876,4.20748) rectangle (8.28856,4.31333);
\draw [color=c, fill=c] (8.28856,4.20748) rectangle (8.32836,4.31333);
\draw [color=c, fill=c] (8.32836,4.20748) rectangle (8.36816,4.31333);
\draw [color=c, fill=c] (8.36816,4.20748) rectangle (8.40796,4.31333);
\draw [color=c, fill=c] (8.40796,4.20748) rectangle (8.44776,4.31333);
\draw [color=c, fill=c] (8.44776,4.20748) rectangle (8.48756,4.31333);
\draw [color=c, fill=c] (8.48756,4.20748) rectangle (8.52736,4.31333);
\draw [color=c, fill=c] (8.52736,4.20748) rectangle (8.56716,4.31333);
\draw [color=c, fill=c] (8.56716,4.20748) rectangle (8.60697,4.31333);
\draw [color=c, fill=c] (8.60697,4.20748) rectangle (8.64677,4.31333);
\draw [color=c, fill=c] (8.64677,4.20748) rectangle (8.68657,4.31333);
\draw [color=c, fill=c] (8.68657,4.20748) rectangle (8.72637,4.31333);
\definecolor{c}{rgb}{0.386667,0,1};
\draw [color=c, fill=c] (8.72637,4.20748) rectangle (8.76617,4.31333);
\draw [color=c, fill=c] (8.76617,4.20748) rectangle (8.80597,4.31333);
\draw [color=c, fill=c] (8.80597,4.20748) rectangle (8.84577,4.31333);
\draw [color=c, fill=c] (8.84577,4.20748) rectangle (8.88557,4.31333);
\draw [color=c, fill=c] (8.88557,4.20748) rectangle (8.92537,4.31333);
\draw [color=c, fill=c] (8.92537,4.20748) rectangle (8.96517,4.31333);
\draw [color=c, fill=c] (8.96517,4.20748) rectangle (9.00498,4.31333);
\draw [color=c, fill=c] (9.00498,4.20748) rectangle (9.04478,4.31333);
\draw [color=c, fill=c] (9.04478,4.20748) rectangle (9.08458,4.31333);
\draw [color=c, fill=c] (9.08458,4.20748) rectangle (9.12438,4.31333);
\draw [color=c, fill=c] (9.12438,4.20748) rectangle (9.16418,4.31333);
\draw [color=c, fill=c] (9.16418,4.20748) rectangle (9.20398,4.31333);
\draw [color=c, fill=c] (9.20398,4.20748) rectangle (9.24378,4.31333);
\draw [color=c, fill=c] (9.24378,4.20748) rectangle (9.28358,4.31333);
\draw [color=c, fill=c] (9.28358,4.20748) rectangle (9.32338,4.31333);
\draw [color=c, fill=c] (9.32338,4.20748) rectangle (9.36318,4.31333);
\draw [color=c, fill=c] (9.36318,4.20748) rectangle (9.40298,4.31333);
\draw [color=c, fill=c] (9.40298,4.20748) rectangle (9.44279,4.31333);
\draw [color=c, fill=c] (9.44279,4.20748) rectangle (9.48259,4.31333);
\draw [color=c, fill=c] (9.48259,4.20748) rectangle (9.52239,4.31333);
\draw [color=c, fill=c] (9.52239,4.20748) rectangle (9.56219,4.31333);
\draw [color=c, fill=c] (9.56219,4.20748) rectangle (9.60199,4.31333);
\draw [color=c, fill=c] (9.60199,4.20748) rectangle (9.64179,4.31333);
\draw [color=c, fill=c] (9.64179,4.20748) rectangle (9.68159,4.31333);
\definecolor{c}{rgb}{0.2,0,1};
\draw [color=c, fill=c] (9.68159,4.20748) rectangle (9.72139,4.31333);
\draw [color=c, fill=c] (9.72139,4.20748) rectangle (9.76119,4.31333);
\draw [color=c, fill=c] (9.76119,4.20748) rectangle (9.80099,4.31333);
\draw [color=c, fill=c] (9.80099,4.20748) rectangle (9.8408,4.31333);
\definecolor{c}{rgb}{0,0.0800001,1};
\draw [color=c, fill=c] (9.8408,4.20748) rectangle (9.8806,4.31333);
\draw [color=c, fill=c] (9.8806,4.20748) rectangle (9.9204,4.31333);
\definecolor{c}{rgb}{0,0.266667,1};
\draw [color=c, fill=c] (9.9204,4.20748) rectangle (9.9602,4.31333);
\draw [color=c, fill=c] (9.9602,4.20748) rectangle (10,4.31333);
\definecolor{c}{rgb}{0,0.546666,1};
\draw [color=c, fill=c] (10,4.20748) rectangle (10.0398,4.31333);
\definecolor{c}{rgb}{0,0.733333,1};
\draw [color=c, fill=c] (10.0398,4.20748) rectangle (10.0796,4.31333);
\draw [color=c, fill=c] (10.0796,4.20748) rectangle (10.1194,4.31333);
\draw [color=c, fill=c] (10.1194,4.20748) rectangle (10.1592,4.31333);
\definecolor{c}{rgb}{0,1,0.986667};
\draw [color=c, fill=c] (10.1592,4.20748) rectangle (10.199,4.31333);
\draw [color=c, fill=c] (10.199,4.20748) rectangle (10.2388,4.31333);
\draw [color=c, fill=c] (10.2388,4.20748) rectangle (10.2786,4.31333);
\draw [color=c, fill=c] (10.2786,4.20748) rectangle (10.3184,4.31333);
\draw [color=c, fill=c] (10.3184,4.20748) rectangle (10.3582,4.31333);
\draw [color=c, fill=c] (10.3582,4.20748) rectangle (10.398,4.31333);
\draw [color=c, fill=c] (10.398,4.20748) rectangle (10.4378,4.31333);
\draw [color=c, fill=c] (10.4378,4.20748) rectangle (10.4776,4.31333);
\draw [color=c, fill=c] (10.4776,4.20748) rectangle (10.5174,4.31333);
\draw [color=c, fill=c] (10.5174,4.20748) rectangle (10.5572,4.31333);
\draw [color=c, fill=c] (10.5572,4.20748) rectangle (10.597,4.31333);
\draw [color=c, fill=c] (10.597,4.20748) rectangle (10.6368,4.31333);
\draw [color=c, fill=c] (10.6368,4.20748) rectangle (10.6766,4.31333);
\draw [color=c, fill=c] (10.6766,4.20748) rectangle (10.7164,4.31333);
\draw [color=c, fill=c] (10.7164,4.20748) rectangle (10.7562,4.31333);
\draw [color=c, fill=c] (10.7562,4.20748) rectangle (10.796,4.31333);
\draw [color=c, fill=c] (10.796,4.20748) rectangle (10.8358,4.31333);
\draw [color=c, fill=c] (10.8358,4.20748) rectangle (10.8756,4.31333);
\draw [color=c, fill=c] (10.8756,4.20748) rectangle (10.9154,4.31333);
\draw [color=c, fill=c] (10.9154,4.20748) rectangle (10.9552,4.31333);
\draw [color=c, fill=c] (10.9552,4.20748) rectangle (10.995,4.31333);
\draw [color=c, fill=c] (10.995,4.20748) rectangle (11.0348,4.31333);
\draw [color=c, fill=c] (11.0348,4.20748) rectangle (11.0746,4.31333);
\draw [color=c, fill=c] (11.0746,4.20748) rectangle (11.1144,4.31333);
\draw [color=c, fill=c] (11.1144,4.20748) rectangle (11.1542,4.31333);
\draw [color=c, fill=c] (11.1542,4.20748) rectangle (11.194,4.31333);
\draw [color=c, fill=c] (11.194,4.20748) rectangle (11.2338,4.31333);
\draw [color=c, fill=c] (11.2338,4.20748) rectangle (11.2736,4.31333);
\draw [color=c, fill=c] (11.2736,4.20748) rectangle (11.3134,4.31333);
\draw [color=c, fill=c] (11.3134,4.20748) rectangle (11.3532,4.31333);
\draw [color=c, fill=c] (11.3532,4.20748) rectangle (11.393,4.31333);
\draw [color=c, fill=c] (11.393,4.20748) rectangle (11.4328,4.31333);
\draw [color=c, fill=c] (11.4328,4.20748) rectangle (11.4726,4.31333);
\draw [color=c, fill=c] (11.4726,4.20748) rectangle (11.5124,4.31333);
\draw [color=c, fill=c] (11.5124,4.20748) rectangle (11.5522,4.31333);
\draw [color=c, fill=c] (11.5522,4.20748) rectangle (11.592,4.31333);
\draw [color=c, fill=c] (11.592,4.20748) rectangle (11.6318,4.31333);
\draw [color=c, fill=c] (11.6318,4.20748) rectangle (11.6716,4.31333);
\draw [color=c, fill=c] (11.6716,4.20748) rectangle (11.7114,4.31333);
\definecolor{c}{rgb}{0,0.733333,1};
\draw [color=c, fill=c] (11.7114,4.20748) rectangle (11.7512,4.31333);
\draw [color=c, fill=c] (11.7512,4.20748) rectangle (11.791,4.31333);
\draw [color=c, fill=c] (11.791,4.20748) rectangle (11.8308,4.31333);
\draw [color=c, fill=c] (11.8308,4.20748) rectangle (11.8706,4.31333);
\draw [color=c, fill=c] (11.8706,4.20748) rectangle (11.9104,4.31333);
\draw [color=c, fill=c] (11.9104,4.20748) rectangle (11.9502,4.31333);
\draw [color=c, fill=c] (11.9502,4.20748) rectangle (11.99,4.31333);
\draw [color=c, fill=c] (11.99,4.20748) rectangle (12.0299,4.31333);
\draw [color=c, fill=c] (12.0299,4.20748) rectangle (12.0697,4.31333);
\draw [color=c, fill=c] (12.0697,4.20748) rectangle (12.1095,4.31333);
\draw [color=c, fill=c] (12.1095,4.20748) rectangle (12.1493,4.31333);
\draw [color=c, fill=c] (12.1493,4.20748) rectangle (12.1891,4.31333);
\draw [color=c, fill=c] (12.1891,4.20748) rectangle (12.2289,4.31333);
\draw [color=c, fill=c] (12.2289,4.20748) rectangle (12.2687,4.31333);
\draw [color=c, fill=c] (12.2687,4.20748) rectangle (12.3085,4.31333);
\draw [color=c, fill=c] (12.3085,4.20748) rectangle (12.3483,4.31333);
\draw [color=c, fill=c] (12.3483,4.20748) rectangle (12.3881,4.31333);
\draw [color=c, fill=c] (12.3881,4.20748) rectangle (12.4279,4.31333);
\draw [color=c, fill=c] (12.4279,4.20748) rectangle (12.4677,4.31333);
\draw [color=c, fill=c] (12.4677,4.20748) rectangle (12.5075,4.31333);
\draw [color=c, fill=c] (12.5075,4.20748) rectangle (12.5473,4.31333);
\draw [color=c, fill=c] (12.5473,4.20748) rectangle (12.5871,4.31333);
\draw [color=c, fill=c] (12.5871,4.20748) rectangle (12.6269,4.31333);
\draw [color=c, fill=c] (12.6269,4.20748) rectangle (12.6667,4.31333);
\draw [color=c, fill=c] (12.6667,4.20748) rectangle (12.7065,4.31333);
\draw [color=c, fill=c] (12.7065,4.20748) rectangle (12.7463,4.31333);
\draw [color=c, fill=c] (12.7463,4.20748) rectangle (12.7861,4.31333);
\draw [color=c, fill=c] (12.7861,4.20748) rectangle (12.8259,4.31333);
\draw [color=c, fill=c] (12.8259,4.20748) rectangle (12.8657,4.31333);
\draw [color=c, fill=c] (12.8657,4.20748) rectangle (12.9055,4.31333);
\draw [color=c, fill=c] (12.9055,4.20748) rectangle (12.9453,4.31333);
\draw [color=c, fill=c] (12.9453,4.20748) rectangle (12.9851,4.31333);
\draw [color=c, fill=c] (12.9851,4.20748) rectangle (13.0249,4.31333);
\draw [color=c, fill=c] (13.0249,4.20748) rectangle (13.0647,4.31333);
\draw [color=c, fill=c] (13.0647,4.20748) rectangle (13.1045,4.31333);
\draw [color=c, fill=c] (13.1045,4.20748) rectangle (13.1443,4.31333);
\draw [color=c, fill=c] (13.1443,4.20748) rectangle (13.1841,4.31333);
\draw [color=c, fill=c] (13.1841,4.20748) rectangle (13.2239,4.31333);
\draw [color=c, fill=c] (13.2239,4.20748) rectangle (13.2637,4.31333);
\draw [color=c, fill=c] (13.2637,4.20748) rectangle (13.3035,4.31333);
\draw [color=c, fill=c] (13.3035,4.20748) rectangle (13.3433,4.31333);
\draw [color=c, fill=c] (13.3433,4.20748) rectangle (13.3831,4.31333);
\draw [color=c, fill=c] (13.3831,4.20748) rectangle (13.4229,4.31333);
\draw [color=c, fill=c] (13.4229,4.20748) rectangle (13.4627,4.31333);
\draw [color=c, fill=c] (13.4627,4.20748) rectangle (13.5025,4.31333);
\draw [color=c, fill=c] (13.5025,4.20748) rectangle (13.5423,4.31333);
\draw [color=c, fill=c] (13.5423,4.20748) rectangle (13.5821,4.31333);
\draw [color=c, fill=c] (13.5821,4.20748) rectangle (13.6219,4.31333);
\draw [color=c, fill=c] (13.6219,4.20748) rectangle (13.6617,4.31333);
\draw [color=c, fill=c] (13.6617,4.20748) rectangle (13.7015,4.31333);
\draw [color=c, fill=c] (13.7015,4.20748) rectangle (13.7413,4.31333);
\draw [color=c, fill=c] (13.7413,4.20748) rectangle (13.7811,4.31333);
\draw [color=c, fill=c] (13.7811,4.20748) rectangle (13.8209,4.31333);
\draw [color=c, fill=c] (13.8209,4.20748) rectangle (13.8607,4.31333);
\draw [color=c, fill=c] (13.8607,4.20748) rectangle (13.9005,4.31333);
\draw [color=c, fill=c] (13.9005,4.20748) rectangle (13.9403,4.31333);
\draw [color=c, fill=c] (13.9403,4.20748) rectangle (13.9801,4.31333);
\draw [color=c, fill=c] (13.9801,4.20748) rectangle (14.0199,4.31333);
\draw [color=c, fill=c] (14.0199,4.20748) rectangle (14.0597,4.31333);
\draw [color=c, fill=c] (14.0597,4.20748) rectangle (14.0995,4.31333);
\draw [color=c, fill=c] (14.0995,4.20748) rectangle (14.1393,4.31333);
\draw [color=c, fill=c] (14.1393,4.20748) rectangle (14.1791,4.31333);
\draw [color=c, fill=c] (14.1791,4.20748) rectangle (14.2189,4.31333);
\draw [color=c, fill=c] (14.2189,4.20748) rectangle (14.2587,4.31333);
\draw [color=c, fill=c] (14.2587,4.20748) rectangle (14.2985,4.31333);
\draw [color=c, fill=c] (14.2985,4.20748) rectangle (14.3383,4.31333);
\draw [color=c, fill=c] (14.3383,4.20748) rectangle (14.3781,4.31333);
\draw [color=c, fill=c] (14.3781,4.20748) rectangle (14.4179,4.31333);
\draw [color=c, fill=c] (14.4179,4.20748) rectangle (14.4577,4.31333);
\draw [color=c, fill=c] (14.4577,4.20748) rectangle (14.4975,4.31333);
\draw [color=c, fill=c] (14.4975,4.20748) rectangle (14.5373,4.31333);
\draw [color=c, fill=c] (14.5373,4.20748) rectangle (14.5771,4.31333);
\draw [color=c, fill=c] (14.5771,4.20748) rectangle (14.6169,4.31333);
\draw [color=c, fill=c] (14.6169,4.20748) rectangle (14.6567,4.31333);
\draw [color=c, fill=c] (14.6567,4.20748) rectangle (14.6965,4.31333);
\draw [color=c, fill=c] (14.6965,4.20748) rectangle (14.7363,4.31333);
\draw [color=c, fill=c] (14.7363,4.20748) rectangle (14.7761,4.31333);
\draw [color=c, fill=c] (14.7761,4.20748) rectangle (14.8159,4.31333);
\draw [color=c, fill=c] (14.8159,4.20748) rectangle (14.8557,4.31333);
\draw [color=c, fill=c] (14.8557,4.20748) rectangle (14.8955,4.31333);
\draw [color=c, fill=c] (14.8955,4.20748) rectangle (14.9353,4.31333);
\draw [color=c, fill=c] (14.9353,4.20748) rectangle (14.9751,4.31333);
\draw [color=c, fill=c] (14.9751,4.20748) rectangle (15.0149,4.31333);
\draw [color=c, fill=c] (15.0149,4.20748) rectangle (15.0547,4.31333);
\draw [color=c, fill=c] (15.0547,4.20748) rectangle (15.0945,4.31333);
\draw [color=c, fill=c] (15.0945,4.20748) rectangle (15.1343,4.31333);
\draw [color=c, fill=c] (15.1343,4.20748) rectangle (15.1741,4.31333);
\draw [color=c, fill=c] (15.1741,4.20748) rectangle (15.2139,4.31333);
\draw [color=c, fill=c] (15.2139,4.20748) rectangle (15.2537,4.31333);
\draw [color=c, fill=c] (15.2537,4.20748) rectangle (15.2935,4.31333);
\draw [color=c, fill=c] (15.2935,4.20748) rectangle (15.3333,4.31333);
\draw [color=c, fill=c] (15.3333,4.20748) rectangle (15.3731,4.31333);
\draw [color=c, fill=c] (15.3731,4.20748) rectangle (15.4129,4.31333);
\draw [color=c, fill=c] (15.4129,4.20748) rectangle (15.4527,4.31333);
\draw [color=c, fill=c] (15.4527,4.20748) rectangle (15.4925,4.31333);
\draw [color=c, fill=c] (15.4925,4.20748) rectangle (15.5323,4.31333);
\draw [color=c, fill=c] (15.5323,4.20748) rectangle (15.5721,4.31333);
\draw [color=c, fill=c] (15.5721,4.20748) rectangle (15.6119,4.31333);
\draw [color=c, fill=c] (15.6119,4.20748) rectangle (15.6517,4.31333);
\draw [color=c, fill=c] (15.6517,4.20748) rectangle (15.6915,4.31333);
\draw [color=c, fill=c] (15.6915,4.20748) rectangle (15.7313,4.31333);
\draw [color=c, fill=c] (15.7313,4.20748) rectangle (15.7711,4.31333);
\draw [color=c, fill=c] (15.7711,4.20748) rectangle (15.8109,4.31333);
\draw [color=c, fill=c] (15.8109,4.20748) rectangle (15.8507,4.31333);
\draw [color=c, fill=c] (15.8507,4.20748) rectangle (15.8905,4.31333);
\draw [color=c, fill=c] (15.8905,4.20748) rectangle (15.9303,4.31333);
\draw [color=c, fill=c] (15.9303,4.20748) rectangle (15.9701,4.31333);
\draw [color=c, fill=c] (15.9701,4.20748) rectangle (16.01,4.31333);
\draw [color=c, fill=c] (16.01,4.20748) rectangle (16.0498,4.31333);
\draw [color=c, fill=c] (16.0498,4.20748) rectangle (16.0896,4.31333);
\draw [color=c, fill=c] (16.0896,4.20748) rectangle (16.1294,4.31333);
\draw [color=c, fill=c] (16.1294,4.20748) rectangle (16.1692,4.31333);
\draw [color=c, fill=c] (16.1692,4.20748) rectangle (16.209,4.31333);
\draw [color=c, fill=c] (16.209,4.20748) rectangle (16.2488,4.31333);
\draw [color=c, fill=c] (16.2488,4.20748) rectangle (16.2886,4.31333);
\draw [color=c, fill=c] (16.2886,4.20748) rectangle (16.3284,4.31333);
\draw [color=c, fill=c] (16.3284,4.20748) rectangle (16.3682,4.31333);
\draw [color=c, fill=c] (16.3682,4.20748) rectangle (16.408,4.31333);
\draw [color=c, fill=c] (16.408,4.20748) rectangle (16.4478,4.31333);
\draw [color=c, fill=c] (16.4478,4.20748) rectangle (16.4876,4.31333);
\draw [color=c, fill=c] (16.4876,4.20748) rectangle (16.5274,4.31333);
\draw [color=c, fill=c] (16.5274,4.20748) rectangle (16.5672,4.31333);
\draw [color=c, fill=c] (16.5672,4.20748) rectangle (16.607,4.31333);
\draw [color=c, fill=c] (16.607,4.20748) rectangle (16.6468,4.31333);
\draw [color=c, fill=c] (16.6468,4.20748) rectangle (16.6866,4.31333);
\draw [color=c, fill=c] (16.6866,4.20748) rectangle (16.7264,4.31333);
\draw [color=c, fill=c] (16.7264,4.20748) rectangle (16.7662,4.31333);
\draw [color=c, fill=c] (16.7662,4.20748) rectangle (16.806,4.31333);
\draw [color=c, fill=c] (16.806,4.20748) rectangle (16.8458,4.31333);
\draw [color=c, fill=c] (16.8458,4.20748) rectangle (16.8856,4.31333);
\draw [color=c, fill=c] (16.8856,4.20748) rectangle (16.9254,4.31333);
\draw [color=c, fill=c] (16.9254,4.20748) rectangle (16.9652,4.31333);
\draw [color=c, fill=c] (16.9652,4.20748) rectangle (17.005,4.31333);
\draw [color=c, fill=c] (17.005,4.20748) rectangle (17.0448,4.31333);
\draw [color=c, fill=c] (17.0448,4.20748) rectangle (17.0846,4.31333);
\draw [color=c, fill=c] (17.0846,4.20748) rectangle (17.1244,4.31333);
\draw [color=c, fill=c] (17.1244,4.20748) rectangle (17.1642,4.31333);
\draw [color=c, fill=c] (17.1642,4.20748) rectangle (17.204,4.31333);
\draw [color=c, fill=c] (17.204,4.20748) rectangle (17.2438,4.31333);
\draw [color=c, fill=c] (17.2438,4.20748) rectangle (17.2836,4.31333);
\draw [color=c, fill=c] (17.2836,4.20748) rectangle (17.3234,4.31333);
\draw [color=c, fill=c] (17.3234,4.20748) rectangle (17.3632,4.31333);
\draw [color=c, fill=c] (17.3632,4.20748) rectangle (17.403,4.31333);
\draw [color=c, fill=c] (17.403,4.20748) rectangle (17.4428,4.31333);
\draw [color=c, fill=c] (17.4428,4.20748) rectangle (17.4826,4.31333);
\draw [color=c, fill=c] (17.4826,4.20748) rectangle (17.5224,4.31333);
\draw [color=c, fill=c] (17.5224,4.20748) rectangle (17.5622,4.31333);
\draw [color=c, fill=c] (17.5622,4.20748) rectangle (17.602,4.31333);
\draw [color=c, fill=c] (17.602,4.20748) rectangle (17.6418,4.31333);
\draw [color=c, fill=c] (17.6418,4.20748) rectangle (17.6816,4.31333);
\draw [color=c, fill=c] (17.6816,4.20748) rectangle (17.7214,4.31333);
\draw [color=c, fill=c] (17.7214,4.20748) rectangle (17.7612,4.31333);
\draw [color=c, fill=c] (17.7612,4.20748) rectangle (17.801,4.31333);
\draw [color=c, fill=c] (17.801,4.20748) rectangle (17.8408,4.31333);
\draw [color=c, fill=c] (17.8408,4.20748) rectangle (17.8806,4.31333);
\draw [color=c, fill=c] (17.8806,4.20748) rectangle (17.9204,4.31333);
\draw [color=c, fill=c] (17.9204,4.20748) rectangle (17.9602,4.31333);
\draw [color=c, fill=c] (17.9602,4.20748) rectangle (18,4.31333);
\definecolor{c}{rgb}{0,0.0800001,1};
\draw [color=c, fill=c] (2,4.31333) rectangle (2.0398,4.41918);
\draw [color=c, fill=c] (2.0398,4.31333) rectangle (2.0796,4.41918);
\draw [color=c, fill=c] (2.0796,4.31333) rectangle (2.1194,4.41918);
\draw [color=c, fill=c] (2.1194,4.31333) rectangle (2.1592,4.41918);
\draw [color=c, fill=c] (2.1592,4.31333) rectangle (2.19901,4.41918);
\draw [color=c, fill=c] (2.19901,4.31333) rectangle (2.23881,4.41918);
\draw [color=c, fill=c] (2.23881,4.31333) rectangle (2.27861,4.41918);
\draw [color=c, fill=c] (2.27861,4.31333) rectangle (2.31841,4.41918);
\draw [color=c, fill=c] (2.31841,4.31333) rectangle (2.35821,4.41918);
\draw [color=c, fill=c] (2.35821,4.31333) rectangle (2.39801,4.41918);
\draw [color=c, fill=c] (2.39801,4.31333) rectangle (2.43781,4.41918);
\draw [color=c, fill=c] (2.43781,4.31333) rectangle (2.47761,4.41918);
\draw [color=c, fill=c] (2.47761,4.31333) rectangle (2.51741,4.41918);
\draw [color=c, fill=c] (2.51741,4.31333) rectangle (2.55721,4.41918);
\draw [color=c, fill=c] (2.55721,4.31333) rectangle (2.59702,4.41918);
\draw [color=c, fill=c] (2.59702,4.31333) rectangle (2.63682,4.41918);
\draw [color=c, fill=c] (2.63682,4.31333) rectangle (2.67662,4.41918);
\draw [color=c, fill=c] (2.67662,4.31333) rectangle (2.71642,4.41918);
\draw [color=c, fill=c] (2.71642,4.31333) rectangle (2.75622,4.41918);
\draw [color=c, fill=c] (2.75622,4.31333) rectangle (2.79602,4.41918);
\draw [color=c, fill=c] (2.79602,4.31333) rectangle (2.83582,4.41918);
\draw [color=c, fill=c] (2.83582,4.31333) rectangle (2.87562,4.41918);
\draw [color=c, fill=c] (2.87562,4.31333) rectangle (2.91542,4.41918);
\draw [color=c, fill=c] (2.91542,4.31333) rectangle (2.95522,4.41918);
\draw [color=c, fill=c] (2.95522,4.31333) rectangle (2.99502,4.41918);
\draw [color=c, fill=c] (2.99502,4.31333) rectangle (3.03483,4.41918);
\draw [color=c, fill=c] (3.03483,4.31333) rectangle (3.07463,4.41918);
\draw [color=c, fill=c] (3.07463,4.31333) rectangle (3.11443,4.41918);
\draw [color=c, fill=c] (3.11443,4.31333) rectangle (3.15423,4.41918);
\draw [color=c, fill=c] (3.15423,4.31333) rectangle (3.19403,4.41918);
\draw [color=c, fill=c] (3.19403,4.31333) rectangle (3.23383,4.41918);
\draw [color=c, fill=c] (3.23383,4.31333) rectangle (3.27363,4.41918);
\draw [color=c, fill=c] (3.27363,4.31333) rectangle (3.31343,4.41918);
\draw [color=c, fill=c] (3.31343,4.31333) rectangle (3.35323,4.41918);
\draw [color=c, fill=c] (3.35323,4.31333) rectangle (3.39303,4.41918);
\draw [color=c, fill=c] (3.39303,4.31333) rectangle (3.43284,4.41918);
\draw [color=c, fill=c] (3.43284,4.31333) rectangle (3.47264,4.41918);
\draw [color=c, fill=c] (3.47264,4.31333) rectangle (3.51244,4.41918);
\draw [color=c, fill=c] (3.51244,4.31333) rectangle (3.55224,4.41918);
\draw [color=c, fill=c] (3.55224,4.31333) rectangle (3.59204,4.41918);
\draw [color=c, fill=c] (3.59204,4.31333) rectangle (3.63184,4.41918);
\draw [color=c, fill=c] (3.63184,4.31333) rectangle (3.67164,4.41918);
\draw [color=c, fill=c] (3.67164,4.31333) rectangle (3.71144,4.41918);
\draw [color=c, fill=c] (3.71144,4.31333) rectangle (3.75124,4.41918);
\draw [color=c, fill=c] (3.75124,4.31333) rectangle (3.79104,4.41918);
\draw [color=c, fill=c] (3.79104,4.31333) rectangle (3.83085,4.41918);
\draw [color=c, fill=c] (3.83085,4.31333) rectangle (3.87065,4.41918);
\draw [color=c, fill=c] (3.87065,4.31333) rectangle (3.91045,4.41918);
\draw [color=c, fill=c] (3.91045,4.31333) rectangle (3.95025,4.41918);
\draw [color=c, fill=c] (3.95025,4.31333) rectangle (3.99005,4.41918);
\draw [color=c, fill=c] (3.99005,4.31333) rectangle (4.02985,4.41918);
\draw [color=c, fill=c] (4.02985,4.31333) rectangle (4.06965,4.41918);
\draw [color=c, fill=c] (4.06965,4.31333) rectangle (4.10945,4.41918);
\draw [color=c, fill=c] (4.10945,4.31333) rectangle (4.14925,4.41918);
\draw [color=c, fill=c] (4.14925,4.31333) rectangle (4.18905,4.41918);
\draw [color=c, fill=c] (4.18905,4.31333) rectangle (4.22886,4.41918);
\draw [color=c, fill=c] (4.22886,4.31333) rectangle (4.26866,4.41918);
\draw [color=c, fill=c] (4.26866,4.31333) rectangle (4.30846,4.41918);
\draw [color=c, fill=c] (4.30846,4.31333) rectangle (4.34826,4.41918);
\draw [color=c, fill=c] (4.34826,4.31333) rectangle (4.38806,4.41918);
\draw [color=c, fill=c] (4.38806,4.31333) rectangle (4.42786,4.41918);
\draw [color=c, fill=c] (4.42786,4.31333) rectangle (4.46766,4.41918);
\draw [color=c, fill=c] (4.46766,4.31333) rectangle (4.50746,4.41918);
\draw [color=c, fill=c] (4.50746,4.31333) rectangle (4.54726,4.41918);
\draw [color=c, fill=c] (4.54726,4.31333) rectangle (4.58706,4.41918);
\draw [color=c, fill=c] (4.58706,4.31333) rectangle (4.62687,4.41918);
\draw [color=c, fill=c] (4.62687,4.31333) rectangle (4.66667,4.41918);
\draw [color=c, fill=c] (4.66667,4.31333) rectangle (4.70647,4.41918);
\draw [color=c, fill=c] (4.70647,4.31333) rectangle (4.74627,4.41918);
\draw [color=c, fill=c] (4.74627,4.31333) rectangle (4.78607,4.41918);
\draw [color=c, fill=c] (4.78607,4.31333) rectangle (4.82587,4.41918);
\draw [color=c, fill=c] (4.82587,4.31333) rectangle (4.86567,4.41918);
\draw [color=c, fill=c] (4.86567,4.31333) rectangle (4.90547,4.41918);
\draw [color=c, fill=c] (4.90547,4.31333) rectangle (4.94527,4.41918);
\draw [color=c, fill=c] (4.94527,4.31333) rectangle (4.98507,4.41918);
\draw [color=c, fill=c] (4.98507,4.31333) rectangle (5.02488,4.41918);
\draw [color=c, fill=c] (5.02488,4.31333) rectangle (5.06468,4.41918);
\draw [color=c, fill=c] (5.06468,4.31333) rectangle (5.10448,4.41918);
\draw [color=c, fill=c] (5.10448,4.31333) rectangle (5.14428,4.41918);
\draw [color=c, fill=c] (5.14428,4.31333) rectangle (5.18408,4.41918);
\draw [color=c, fill=c] (5.18408,4.31333) rectangle (5.22388,4.41918);
\draw [color=c, fill=c] (5.22388,4.31333) rectangle (5.26368,4.41918);
\draw [color=c, fill=c] (5.26368,4.31333) rectangle (5.30348,4.41918);
\draw [color=c, fill=c] (5.30348,4.31333) rectangle (5.34328,4.41918);
\draw [color=c, fill=c] (5.34328,4.31333) rectangle (5.38308,4.41918);
\draw [color=c, fill=c] (5.38308,4.31333) rectangle (5.42289,4.41918);
\draw [color=c, fill=c] (5.42289,4.31333) rectangle (5.46269,4.41918);
\draw [color=c, fill=c] (5.46269,4.31333) rectangle (5.50249,4.41918);
\draw [color=c, fill=c] (5.50249,4.31333) rectangle (5.54229,4.41918);
\draw [color=c, fill=c] (5.54229,4.31333) rectangle (5.58209,4.41918);
\draw [color=c, fill=c] (5.58209,4.31333) rectangle (5.62189,4.41918);
\draw [color=c, fill=c] (5.62189,4.31333) rectangle (5.66169,4.41918);
\draw [color=c, fill=c] (5.66169,4.31333) rectangle (5.70149,4.41918);
\draw [color=c, fill=c] (5.70149,4.31333) rectangle (5.74129,4.41918);
\draw [color=c, fill=c] (5.74129,4.31333) rectangle (5.78109,4.41918);
\draw [color=c, fill=c] (5.78109,4.31333) rectangle (5.8209,4.41918);
\draw [color=c, fill=c] (5.8209,4.31333) rectangle (5.8607,4.41918);
\definecolor{c}{rgb}{0.2,0,1};
\draw [color=c, fill=c] (5.8607,4.31333) rectangle (5.9005,4.41918);
\draw [color=c, fill=c] (5.9005,4.31333) rectangle (5.9403,4.41918);
\draw [color=c, fill=c] (5.9403,4.31333) rectangle (5.9801,4.41918);
\draw [color=c, fill=c] (5.9801,4.31333) rectangle (6.0199,4.41918);
\draw [color=c, fill=c] (6.0199,4.31333) rectangle (6.0597,4.41918);
\draw [color=c, fill=c] (6.0597,4.31333) rectangle (6.0995,4.41918);
\draw [color=c, fill=c] (6.0995,4.31333) rectangle (6.1393,4.41918);
\draw [color=c, fill=c] (6.1393,4.31333) rectangle (6.1791,4.41918);
\draw [color=c, fill=c] (6.1791,4.31333) rectangle (6.21891,4.41918);
\draw [color=c, fill=c] (6.21891,4.31333) rectangle (6.25871,4.41918);
\draw [color=c, fill=c] (6.25871,4.31333) rectangle (6.29851,4.41918);
\draw [color=c, fill=c] (6.29851,4.31333) rectangle (6.33831,4.41918);
\draw [color=c, fill=c] (6.33831,4.31333) rectangle (6.37811,4.41918);
\draw [color=c, fill=c] (6.37811,4.31333) rectangle (6.41791,4.41918);
\draw [color=c, fill=c] (6.41791,4.31333) rectangle (6.45771,4.41918);
\draw [color=c, fill=c] (6.45771,4.31333) rectangle (6.49751,4.41918);
\draw [color=c, fill=c] (6.49751,4.31333) rectangle (6.53731,4.41918);
\draw [color=c, fill=c] (6.53731,4.31333) rectangle (6.57711,4.41918);
\draw [color=c, fill=c] (6.57711,4.31333) rectangle (6.61692,4.41918);
\draw [color=c, fill=c] (6.61692,4.31333) rectangle (6.65672,4.41918);
\draw [color=c, fill=c] (6.65672,4.31333) rectangle (6.69652,4.41918);
\draw [color=c, fill=c] (6.69652,4.31333) rectangle (6.73632,4.41918);
\draw [color=c, fill=c] (6.73632,4.31333) rectangle (6.77612,4.41918);
\draw [color=c, fill=c] (6.77612,4.31333) rectangle (6.81592,4.41918);
\draw [color=c, fill=c] (6.81592,4.31333) rectangle (6.85572,4.41918);
\draw [color=c, fill=c] (6.85572,4.31333) rectangle (6.89552,4.41918);
\draw [color=c, fill=c] (6.89552,4.31333) rectangle (6.93532,4.41918);
\draw [color=c, fill=c] (6.93532,4.31333) rectangle (6.97512,4.41918);
\draw [color=c, fill=c] (6.97512,4.31333) rectangle (7.01493,4.41918);
\draw [color=c, fill=c] (7.01493,4.31333) rectangle (7.05473,4.41918);
\draw [color=c, fill=c] (7.05473,4.31333) rectangle (7.09453,4.41918);
\draw [color=c, fill=c] (7.09453,4.31333) rectangle (7.13433,4.41918);
\draw [color=c, fill=c] (7.13433,4.31333) rectangle (7.17413,4.41918);
\draw [color=c, fill=c] (7.17413,4.31333) rectangle (7.21393,4.41918);
\draw [color=c, fill=c] (7.21393,4.31333) rectangle (7.25373,4.41918);
\draw [color=c, fill=c] (7.25373,4.31333) rectangle (7.29353,4.41918);
\draw [color=c, fill=c] (7.29353,4.31333) rectangle (7.33333,4.41918);
\draw [color=c, fill=c] (7.33333,4.31333) rectangle (7.37313,4.41918);
\draw [color=c, fill=c] (7.37313,4.31333) rectangle (7.41294,4.41918);
\draw [color=c, fill=c] (7.41294,4.31333) rectangle (7.45274,4.41918);
\draw [color=c, fill=c] (7.45274,4.31333) rectangle (7.49254,4.41918);
\draw [color=c, fill=c] (7.49254,4.31333) rectangle (7.53234,4.41918);
\draw [color=c, fill=c] (7.53234,4.31333) rectangle (7.57214,4.41918);
\draw [color=c, fill=c] (7.57214,4.31333) rectangle (7.61194,4.41918);
\draw [color=c, fill=c] (7.61194,4.31333) rectangle (7.65174,4.41918);
\draw [color=c, fill=c] (7.65174,4.31333) rectangle (7.69154,4.41918);
\draw [color=c, fill=c] (7.69154,4.31333) rectangle (7.73134,4.41918);
\draw [color=c, fill=c] (7.73134,4.31333) rectangle (7.77114,4.41918);
\draw [color=c, fill=c] (7.77114,4.31333) rectangle (7.81095,4.41918);
\draw [color=c, fill=c] (7.81095,4.31333) rectangle (7.85075,4.41918);
\draw [color=c, fill=c] (7.85075,4.31333) rectangle (7.89055,4.41918);
\draw [color=c, fill=c] (7.89055,4.31333) rectangle (7.93035,4.41918);
\draw [color=c, fill=c] (7.93035,4.31333) rectangle (7.97015,4.41918);
\draw [color=c, fill=c] (7.97015,4.31333) rectangle (8.00995,4.41918);
\draw [color=c, fill=c] (8.00995,4.31333) rectangle (8.04975,4.41918);
\draw [color=c, fill=c] (8.04975,4.31333) rectangle (8.08955,4.41918);
\draw [color=c, fill=c] (8.08955,4.31333) rectangle (8.12935,4.41918);
\draw [color=c, fill=c] (8.12935,4.31333) rectangle (8.16915,4.41918);
\draw [color=c, fill=c] (8.16915,4.31333) rectangle (8.20895,4.41918);
\draw [color=c, fill=c] (8.20895,4.31333) rectangle (8.24876,4.41918);
\draw [color=c, fill=c] (8.24876,4.31333) rectangle (8.28856,4.41918);
\draw [color=c, fill=c] (8.28856,4.31333) rectangle (8.32836,4.41918);
\draw [color=c, fill=c] (8.32836,4.31333) rectangle (8.36816,4.41918);
\draw [color=c, fill=c] (8.36816,4.31333) rectangle (8.40796,4.41918);
\draw [color=c, fill=c] (8.40796,4.31333) rectangle (8.44776,4.41918);
\draw [color=c, fill=c] (8.44776,4.31333) rectangle (8.48756,4.41918);
\draw [color=c, fill=c] (8.48756,4.31333) rectangle (8.52736,4.41918);
\draw [color=c, fill=c] (8.52736,4.31333) rectangle (8.56716,4.41918);
\draw [color=c, fill=c] (8.56716,4.31333) rectangle (8.60697,4.41918);
\draw [color=c, fill=c] (8.60697,4.31333) rectangle (8.64677,4.41918);
\draw [color=c, fill=c] (8.64677,4.31333) rectangle (8.68657,4.41918);
\draw [color=c, fill=c] (8.68657,4.31333) rectangle (8.72637,4.41918);
\draw [color=c, fill=c] (8.72637,4.31333) rectangle (8.76617,4.41918);
\draw [color=c, fill=c] (8.76617,4.31333) rectangle (8.80597,4.41918);
\draw [color=c, fill=c] (8.80597,4.31333) rectangle (8.84577,4.41918);
\definecolor{c}{rgb}{0.386667,0,1};
\draw [color=c, fill=c] (8.84577,4.31333) rectangle (8.88557,4.41918);
\draw [color=c, fill=c] (8.88557,4.31333) rectangle (8.92537,4.41918);
\draw [color=c, fill=c] (8.92537,4.31333) rectangle (8.96517,4.41918);
\draw [color=c, fill=c] (8.96517,4.31333) rectangle (9.00498,4.41918);
\draw [color=c, fill=c] (9.00498,4.31333) rectangle (9.04478,4.41918);
\draw [color=c, fill=c] (9.04478,4.31333) rectangle (9.08458,4.41918);
\draw [color=c, fill=c] (9.08458,4.31333) rectangle (9.12438,4.41918);
\draw [color=c, fill=c] (9.12438,4.31333) rectangle (9.16418,4.41918);
\draw [color=c, fill=c] (9.16418,4.31333) rectangle (9.20398,4.41918);
\draw [color=c, fill=c] (9.20398,4.31333) rectangle (9.24378,4.41918);
\draw [color=c, fill=c] (9.24378,4.31333) rectangle (9.28358,4.41918);
\draw [color=c, fill=c] (9.28358,4.31333) rectangle (9.32338,4.41918);
\draw [color=c, fill=c] (9.32338,4.31333) rectangle (9.36318,4.41918);
\draw [color=c, fill=c] (9.36318,4.31333) rectangle (9.40298,4.41918);
\draw [color=c, fill=c] (9.40298,4.31333) rectangle (9.44279,4.41918);
\draw [color=c, fill=c] (9.44279,4.31333) rectangle (9.48259,4.41918);
\draw [color=c, fill=c] (9.48259,4.31333) rectangle (9.52239,4.41918);
\draw [color=c, fill=c] (9.52239,4.31333) rectangle (9.56219,4.41918);
\definecolor{c}{rgb}{0.2,0,1};
\draw [color=c, fill=c] (9.56219,4.31333) rectangle (9.60199,4.41918);
\draw [color=c, fill=c] (9.60199,4.31333) rectangle (9.64179,4.41918);
\draw [color=c, fill=c] (9.64179,4.31333) rectangle (9.68159,4.41918);
\draw [color=c, fill=c] (9.68159,4.31333) rectangle (9.72139,4.41918);
\draw [color=c, fill=c] (9.72139,4.31333) rectangle (9.76119,4.41918);
\draw [color=c, fill=c] (9.76119,4.31333) rectangle (9.80099,4.41918);
\definecolor{c}{rgb}{0,0.0800001,1};
\draw [color=c, fill=c] (9.80099,4.31333) rectangle (9.8408,4.41918);
\draw [color=c, fill=c] (9.8408,4.31333) rectangle (9.8806,4.41918);
\draw [color=c, fill=c] (9.8806,4.31333) rectangle (9.9204,4.41918);
\definecolor{c}{rgb}{0,0.266667,1};
\draw [color=c, fill=c] (9.9204,4.31333) rectangle (9.9602,4.41918);
\draw [color=c, fill=c] (9.9602,4.31333) rectangle (10,4.41918);
\definecolor{c}{rgb}{0,0.546666,1};
\draw [color=c, fill=c] (10,4.31333) rectangle (10.0398,4.41918);
\draw [color=c, fill=c] (10.0398,4.31333) rectangle (10.0796,4.41918);
\definecolor{c}{rgb}{0,0.733333,1};
\draw [color=c, fill=c] (10.0796,4.31333) rectangle (10.1194,4.41918);
\draw [color=c, fill=c] (10.1194,4.31333) rectangle (10.1592,4.41918);
\draw [color=c, fill=c] (10.1592,4.31333) rectangle (10.199,4.41918);
\draw [color=c, fill=c] (10.199,4.31333) rectangle (10.2388,4.41918);
\definecolor{c}{rgb}{0,1,0.986667};
\draw [color=c, fill=c] (10.2388,4.31333) rectangle (10.2786,4.41918);
\draw [color=c, fill=c] (10.2786,4.31333) rectangle (10.3184,4.41918);
\draw [color=c, fill=c] (10.3184,4.31333) rectangle (10.3582,4.41918);
\draw [color=c, fill=c] (10.3582,4.31333) rectangle (10.398,4.41918);
\draw [color=c, fill=c] (10.398,4.31333) rectangle (10.4378,4.41918);
\draw [color=c, fill=c] (10.4378,4.31333) rectangle (10.4776,4.41918);
\draw [color=c, fill=c] (10.4776,4.31333) rectangle (10.5174,4.41918);
\draw [color=c, fill=c] (10.5174,4.31333) rectangle (10.5572,4.41918);
\draw [color=c, fill=c] (10.5572,4.31333) rectangle (10.597,4.41918);
\draw [color=c, fill=c] (10.597,4.31333) rectangle (10.6368,4.41918);
\draw [color=c, fill=c] (10.6368,4.31333) rectangle (10.6766,4.41918);
\draw [color=c, fill=c] (10.6766,4.31333) rectangle (10.7164,4.41918);
\draw [color=c, fill=c] (10.7164,4.31333) rectangle (10.7562,4.41918);
\draw [color=c, fill=c] (10.7562,4.31333) rectangle (10.796,4.41918);
\draw [color=c, fill=c] (10.796,4.31333) rectangle (10.8358,4.41918);
\draw [color=c, fill=c] (10.8358,4.31333) rectangle (10.8756,4.41918);
\draw [color=c, fill=c] (10.8756,4.31333) rectangle (10.9154,4.41918);
\draw [color=c, fill=c] (10.9154,4.31333) rectangle (10.9552,4.41918);
\draw [color=c, fill=c] (10.9552,4.31333) rectangle (10.995,4.41918);
\draw [color=c, fill=c] (10.995,4.31333) rectangle (11.0348,4.41918);
\draw [color=c, fill=c] (11.0348,4.31333) rectangle (11.0746,4.41918);
\draw [color=c, fill=c] (11.0746,4.31333) rectangle (11.1144,4.41918);
\draw [color=c, fill=c] (11.1144,4.31333) rectangle (11.1542,4.41918);
\draw [color=c, fill=c] (11.1542,4.31333) rectangle (11.194,4.41918);
\draw [color=c, fill=c] (11.194,4.31333) rectangle (11.2338,4.41918);
\draw [color=c, fill=c] (11.2338,4.31333) rectangle (11.2736,4.41918);
\draw [color=c, fill=c] (11.2736,4.31333) rectangle (11.3134,4.41918);
\draw [color=c, fill=c] (11.3134,4.31333) rectangle (11.3532,4.41918);
\draw [color=c, fill=c] (11.3532,4.31333) rectangle (11.393,4.41918);
\draw [color=c, fill=c] (11.393,4.31333) rectangle (11.4328,4.41918);
\draw [color=c, fill=c] (11.4328,4.31333) rectangle (11.4726,4.41918);
\draw [color=c, fill=c] (11.4726,4.31333) rectangle (11.5124,4.41918);
\draw [color=c, fill=c] (11.5124,4.31333) rectangle (11.5522,4.41918);
\draw [color=c, fill=c] (11.5522,4.31333) rectangle (11.592,4.41918);
\draw [color=c, fill=c] (11.592,4.31333) rectangle (11.6318,4.41918);
\definecolor{c}{rgb}{0,0.733333,1};
\draw [color=c, fill=c] (11.6318,4.31333) rectangle (11.6716,4.41918);
\draw [color=c, fill=c] (11.6716,4.31333) rectangle (11.7114,4.41918);
\draw [color=c, fill=c] (11.7114,4.31333) rectangle (11.7512,4.41918);
\draw [color=c, fill=c] (11.7512,4.31333) rectangle (11.791,4.41918);
\draw [color=c, fill=c] (11.791,4.31333) rectangle (11.8308,4.41918);
\draw [color=c, fill=c] (11.8308,4.31333) rectangle (11.8706,4.41918);
\draw [color=c, fill=c] (11.8706,4.31333) rectangle (11.9104,4.41918);
\draw [color=c, fill=c] (11.9104,4.31333) rectangle (11.9502,4.41918);
\draw [color=c, fill=c] (11.9502,4.31333) rectangle (11.99,4.41918);
\draw [color=c, fill=c] (11.99,4.31333) rectangle (12.0299,4.41918);
\draw [color=c, fill=c] (12.0299,4.31333) rectangle (12.0697,4.41918);
\draw [color=c, fill=c] (12.0697,4.31333) rectangle (12.1095,4.41918);
\draw [color=c, fill=c] (12.1095,4.31333) rectangle (12.1493,4.41918);
\draw [color=c, fill=c] (12.1493,4.31333) rectangle (12.1891,4.41918);
\draw [color=c, fill=c] (12.1891,4.31333) rectangle (12.2289,4.41918);
\draw [color=c, fill=c] (12.2289,4.31333) rectangle (12.2687,4.41918);
\draw [color=c, fill=c] (12.2687,4.31333) rectangle (12.3085,4.41918);
\draw [color=c, fill=c] (12.3085,4.31333) rectangle (12.3483,4.41918);
\draw [color=c, fill=c] (12.3483,4.31333) rectangle (12.3881,4.41918);
\draw [color=c, fill=c] (12.3881,4.31333) rectangle (12.4279,4.41918);
\draw [color=c, fill=c] (12.4279,4.31333) rectangle (12.4677,4.41918);
\draw [color=c, fill=c] (12.4677,4.31333) rectangle (12.5075,4.41918);
\draw [color=c, fill=c] (12.5075,4.31333) rectangle (12.5473,4.41918);
\draw [color=c, fill=c] (12.5473,4.31333) rectangle (12.5871,4.41918);
\draw [color=c, fill=c] (12.5871,4.31333) rectangle (12.6269,4.41918);
\draw [color=c, fill=c] (12.6269,4.31333) rectangle (12.6667,4.41918);
\draw [color=c, fill=c] (12.6667,4.31333) rectangle (12.7065,4.41918);
\draw [color=c, fill=c] (12.7065,4.31333) rectangle (12.7463,4.41918);
\draw [color=c, fill=c] (12.7463,4.31333) rectangle (12.7861,4.41918);
\draw [color=c, fill=c] (12.7861,4.31333) rectangle (12.8259,4.41918);
\draw [color=c, fill=c] (12.8259,4.31333) rectangle (12.8657,4.41918);
\draw [color=c, fill=c] (12.8657,4.31333) rectangle (12.9055,4.41918);
\draw [color=c, fill=c] (12.9055,4.31333) rectangle (12.9453,4.41918);
\draw [color=c, fill=c] (12.9453,4.31333) rectangle (12.9851,4.41918);
\draw [color=c, fill=c] (12.9851,4.31333) rectangle (13.0249,4.41918);
\draw [color=c, fill=c] (13.0249,4.31333) rectangle (13.0647,4.41918);
\draw [color=c, fill=c] (13.0647,4.31333) rectangle (13.1045,4.41918);
\draw [color=c, fill=c] (13.1045,4.31333) rectangle (13.1443,4.41918);
\draw [color=c, fill=c] (13.1443,4.31333) rectangle (13.1841,4.41918);
\draw [color=c, fill=c] (13.1841,4.31333) rectangle (13.2239,4.41918);
\draw [color=c, fill=c] (13.2239,4.31333) rectangle (13.2637,4.41918);
\draw [color=c, fill=c] (13.2637,4.31333) rectangle (13.3035,4.41918);
\draw [color=c, fill=c] (13.3035,4.31333) rectangle (13.3433,4.41918);
\draw [color=c, fill=c] (13.3433,4.31333) rectangle (13.3831,4.41918);
\draw [color=c, fill=c] (13.3831,4.31333) rectangle (13.4229,4.41918);
\draw [color=c, fill=c] (13.4229,4.31333) rectangle (13.4627,4.41918);
\draw [color=c, fill=c] (13.4627,4.31333) rectangle (13.5025,4.41918);
\draw [color=c, fill=c] (13.5025,4.31333) rectangle (13.5423,4.41918);
\draw [color=c, fill=c] (13.5423,4.31333) rectangle (13.5821,4.41918);
\draw [color=c, fill=c] (13.5821,4.31333) rectangle (13.6219,4.41918);
\draw [color=c, fill=c] (13.6219,4.31333) rectangle (13.6617,4.41918);
\draw [color=c, fill=c] (13.6617,4.31333) rectangle (13.7015,4.41918);
\draw [color=c, fill=c] (13.7015,4.31333) rectangle (13.7413,4.41918);
\draw [color=c, fill=c] (13.7413,4.31333) rectangle (13.7811,4.41918);
\draw [color=c, fill=c] (13.7811,4.31333) rectangle (13.8209,4.41918);
\draw [color=c, fill=c] (13.8209,4.31333) rectangle (13.8607,4.41918);
\draw [color=c, fill=c] (13.8607,4.31333) rectangle (13.9005,4.41918);
\draw [color=c, fill=c] (13.9005,4.31333) rectangle (13.9403,4.41918);
\draw [color=c, fill=c] (13.9403,4.31333) rectangle (13.9801,4.41918);
\draw [color=c, fill=c] (13.9801,4.31333) rectangle (14.0199,4.41918);
\draw [color=c, fill=c] (14.0199,4.31333) rectangle (14.0597,4.41918);
\draw [color=c, fill=c] (14.0597,4.31333) rectangle (14.0995,4.41918);
\draw [color=c, fill=c] (14.0995,4.31333) rectangle (14.1393,4.41918);
\draw [color=c, fill=c] (14.1393,4.31333) rectangle (14.1791,4.41918);
\draw [color=c, fill=c] (14.1791,4.31333) rectangle (14.2189,4.41918);
\draw [color=c, fill=c] (14.2189,4.31333) rectangle (14.2587,4.41918);
\draw [color=c, fill=c] (14.2587,4.31333) rectangle (14.2985,4.41918);
\draw [color=c, fill=c] (14.2985,4.31333) rectangle (14.3383,4.41918);
\draw [color=c, fill=c] (14.3383,4.31333) rectangle (14.3781,4.41918);
\draw [color=c, fill=c] (14.3781,4.31333) rectangle (14.4179,4.41918);
\draw [color=c, fill=c] (14.4179,4.31333) rectangle (14.4577,4.41918);
\draw [color=c, fill=c] (14.4577,4.31333) rectangle (14.4975,4.41918);
\draw [color=c, fill=c] (14.4975,4.31333) rectangle (14.5373,4.41918);
\draw [color=c, fill=c] (14.5373,4.31333) rectangle (14.5771,4.41918);
\draw [color=c, fill=c] (14.5771,4.31333) rectangle (14.6169,4.41918);
\draw [color=c, fill=c] (14.6169,4.31333) rectangle (14.6567,4.41918);
\draw [color=c, fill=c] (14.6567,4.31333) rectangle (14.6965,4.41918);
\draw [color=c, fill=c] (14.6965,4.31333) rectangle (14.7363,4.41918);
\draw [color=c, fill=c] (14.7363,4.31333) rectangle (14.7761,4.41918);
\draw [color=c, fill=c] (14.7761,4.31333) rectangle (14.8159,4.41918);
\draw [color=c, fill=c] (14.8159,4.31333) rectangle (14.8557,4.41918);
\draw [color=c, fill=c] (14.8557,4.31333) rectangle (14.8955,4.41918);
\draw [color=c, fill=c] (14.8955,4.31333) rectangle (14.9353,4.41918);
\draw [color=c, fill=c] (14.9353,4.31333) rectangle (14.9751,4.41918);
\draw [color=c, fill=c] (14.9751,4.31333) rectangle (15.0149,4.41918);
\draw [color=c, fill=c] (15.0149,4.31333) rectangle (15.0547,4.41918);
\draw [color=c, fill=c] (15.0547,4.31333) rectangle (15.0945,4.41918);
\draw [color=c, fill=c] (15.0945,4.31333) rectangle (15.1343,4.41918);
\draw [color=c, fill=c] (15.1343,4.31333) rectangle (15.1741,4.41918);
\draw [color=c, fill=c] (15.1741,4.31333) rectangle (15.2139,4.41918);
\draw [color=c, fill=c] (15.2139,4.31333) rectangle (15.2537,4.41918);
\draw [color=c, fill=c] (15.2537,4.31333) rectangle (15.2935,4.41918);
\draw [color=c, fill=c] (15.2935,4.31333) rectangle (15.3333,4.41918);
\draw [color=c, fill=c] (15.3333,4.31333) rectangle (15.3731,4.41918);
\draw [color=c, fill=c] (15.3731,4.31333) rectangle (15.4129,4.41918);
\draw [color=c, fill=c] (15.4129,4.31333) rectangle (15.4527,4.41918);
\draw [color=c, fill=c] (15.4527,4.31333) rectangle (15.4925,4.41918);
\draw [color=c, fill=c] (15.4925,4.31333) rectangle (15.5323,4.41918);
\draw [color=c, fill=c] (15.5323,4.31333) rectangle (15.5721,4.41918);
\draw [color=c, fill=c] (15.5721,4.31333) rectangle (15.6119,4.41918);
\draw [color=c, fill=c] (15.6119,4.31333) rectangle (15.6517,4.41918);
\draw [color=c, fill=c] (15.6517,4.31333) rectangle (15.6915,4.41918);
\draw [color=c, fill=c] (15.6915,4.31333) rectangle (15.7313,4.41918);
\draw [color=c, fill=c] (15.7313,4.31333) rectangle (15.7711,4.41918);
\draw [color=c, fill=c] (15.7711,4.31333) rectangle (15.8109,4.41918);
\draw [color=c, fill=c] (15.8109,4.31333) rectangle (15.8507,4.41918);
\draw [color=c, fill=c] (15.8507,4.31333) rectangle (15.8905,4.41918);
\draw [color=c, fill=c] (15.8905,4.31333) rectangle (15.9303,4.41918);
\draw [color=c, fill=c] (15.9303,4.31333) rectangle (15.9701,4.41918);
\draw [color=c, fill=c] (15.9701,4.31333) rectangle (16.01,4.41918);
\draw [color=c, fill=c] (16.01,4.31333) rectangle (16.0498,4.41918);
\draw [color=c, fill=c] (16.0498,4.31333) rectangle (16.0896,4.41918);
\draw [color=c, fill=c] (16.0896,4.31333) rectangle (16.1294,4.41918);
\draw [color=c, fill=c] (16.1294,4.31333) rectangle (16.1692,4.41918);
\draw [color=c, fill=c] (16.1692,4.31333) rectangle (16.209,4.41918);
\draw [color=c, fill=c] (16.209,4.31333) rectangle (16.2488,4.41918);
\draw [color=c, fill=c] (16.2488,4.31333) rectangle (16.2886,4.41918);
\draw [color=c, fill=c] (16.2886,4.31333) rectangle (16.3284,4.41918);
\draw [color=c, fill=c] (16.3284,4.31333) rectangle (16.3682,4.41918);
\draw [color=c, fill=c] (16.3682,4.31333) rectangle (16.408,4.41918);
\draw [color=c, fill=c] (16.408,4.31333) rectangle (16.4478,4.41918);
\draw [color=c, fill=c] (16.4478,4.31333) rectangle (16.4876,4.41918);
\draw [color=c, fill=c] (16.4876,4.31333) rectangle (16.5274,4.41918);
\draw [color=c, fill=c] (16.5274,4.31333) rectangle (16.5672,4.41918);
\draw [color=c, fill=c] (16.5672,4.31333) rectangle (16.607,4.41918);
\draw [color=c, fill=c] (16.607,4.31333) rectangle (16.6468,4.41918);
\draw [color=c, fill=c] (16.6468,4.31333) rectangle (16.6866,4.41918);
\draw [color=c, fill=c] (16.6866,4.31333) rectangle (16.7264,4.41918);
\draw [color=c, fill=c] (16.7264,4.31333) rectangle (16.7662,4.41918);
\draw [color=c, fill=c] (16.7662,4.31333) rectangle (16.806,4.41918);
\draw [color=c, fill=c] (16.806,4.31333) rectangle (16.8458,4.41918);
\draw [color=c, fill=c] (16.8458,4.31333) rectangle (16.8856,4.41918);
\draw [color=c, fill=c] (16.8856,4.31333) rectangle (16.9254,4.41918);
\draw [color=c, fill=c] (16.9254,4.31333) rectangle (16.9652,4.41918);
\draw [color=c, fill=c] (16.9652,4.31333) rectangle (17.005,4.41918);
\draw [color=c, fill=c] (17.005,4.31333) rectangle (17.0448,4.41918);
\draw [color=c, fill=c] (17.0448,4.31333) rectangle (17.0846,4.41918);
\draw [color=c, fill=c] (17.0846,4.31333) rectangle (17.1244,4.41918);
\draw [color=c, fill=c] (17.1244,4.31333) rectangle (17.1642,4.41918);
\draw [color=c, fill=c] (17.1642,4.31333) rectangle (17.204,4.41918);
\draw [color=c, fill=c] (17.204,4.31333) rectangle (17.2438,4.41918);
\draw [color=c, fill=c] (17.2438,4.31333) rectangle (17.2836,4.41918);
\draw [color=c, fill=c] (17.2836,4.31333) rectangle (17.3234,4.41918);
\draw [color=c, fill=c] (17.3234,4.31333) rectangle (17.3632,4.41918);
\draw [color=c, fill=c] (17.3632,4.31333) rectangle (17.403,4.41918);
\draw [color=c, fill=c] (17.403,4.31333) rectangle (17.4428,4.41918);
\draw [color=c, fill=c] (17.4428,4.31333) rectangle (17.4826,4.41918);
\draw [color=c, fill=c] (17.4826,4.31333) rectangle (17.5224,4.41918);
\draw [color=c, fill=c] (17.5224,4.31333) rectangle (17.5622,4.41918);
\draw [color=c, fill=c] (17.5622,4.31333) rectangle (17.602,4.41918);
\draw [color=c, fill=c] (17.602,4.31333) rectangle (17.6418,4.41918);
\draw [color=c, fill=c] (17.6418,4.31333) rectangle (17.6816,4.41918);
\draw [color=c, fill=c] (17.6816,4.31333) rectangle (17.7214,4.41918);
\draw [color=c, fill=c] (17.7214,4.31333) rectangle (17.7612,4.41918);
\draw [color=c, fill=c] (17.7612,4.31333) rectangle (17.801,4.41918);
\draw [color=c, fill=c] (17.801,4.31333) rectangle (17.8408,4.41918);
\draw [color=c, fill=c] (17.8408,4.31333) rectangle (17.8806,4.41918);
\draw [color=c, fill=c] (17.8806,4.31333) rectangle (17.9204,4.41918);
\draw [color=c, fill=c] (17.9204,4.31333) rectangle (17.9602,4.41918);
\draw [color=c, fill=c] (17.9602,4.31333) rectangle (18,4.41918);
\definecolor{c}{rgb}{0,0.0800001,1};
\draw [color=c, fill=c] (2,4.41918) rectangle (2.0398,4.52503);
\draw [color=c, fill=c] (2.0398,4.41918) rectangle (2.0796,4.52503);
\draw [color=c, fill=c] (2.0796,4.41918) rectangle (2.1194,4.52503);
\draw [color=c, fill=c] (2.1194,4.41918) rectangle (2.1592,4.52503);
\draw [color=c, fill=c] (2.1592,4.41918) rectangle (2.19901,4.52503);
\draw [color=c, fill=c] (2.19901,4.41918) rectangle (2.23881,4.52503);
\draw [color=c, fill=c] (2.23881,4.41918) rectangle (2.27861,4.52503);
\draw [color=c, fill=c] (2.27861,4.41918) rectangle (2.31841,4.52503);
\draw [color=c, fill=c] (2.31841,4.41918) rectangle (2.35821,4.52503);
\draw [color=c, fill=c] (2.35821,4.41918) rectangle (2.39801,4.52503);
\draw [color=c, fill=c] (2.39801,4.41918) rectangle (2.43781,4.52503);
\draw [color=c, fill=c] (2.43781,4.41918) rectangle (2.47761,4.52503);
\draw [color=c, fill=c] (2.47761,4.41918) rectangle (2.51741,4.52503);
\draw [color=c, fill=c] (2.51741,4.41918) rectangle (2.55721,4.52503);
\draw [color=c, fill=c] (2.55721,4.41918) rectangle (2.59702,4.52503);
\draw [color=c, fill=c] (2.59702,4.41918) rectangle (2.63682,4.52503);
\draw [color=c, fill=c] (2.63682,4.41918) rectangle (2.67662,4.52503);
\draw [color=c, fill=c] (2.67662,4.41918) rectangle (2.71642,4.52503);
\draw [color=c, fill=c] (2.71642,4.41918) rectangle (2.75622,4.52503);
\draw [color=c, fill=c] (2.75622,4.41918) rectangle (2.79602,4.52503);
\draw [color=c, fill=c] (2.79602,4.41918) rectangle (2.83582,4.52503);
\draw [color=c, fill=c] (2.83582,4.41918) rectangle (2.87562,4.52503);
\draw [color=c, fill=c] (2.87562,4.41918) rectangle (2.91542,4.52503);
\draw [color=c, fill=c] (2.91542,4.41918) rectangle (2.95522,4.52503);
\draw [color=c, fill=c] (2.95522,4.41918) rectangle (2.99502,4.52503);
\draw [color=c, fill=c] (2.99502,4.41918) rectangle (3.03483,4.52503);
\draw [color=c, fill=c] (3.03483,4.41918) rectangle (3.07463,4.52503);
\draw [color=c, fill=c] (3.07463,4.41918) rectangle (3.11443,4.52503);
\draw [color=c, fill=c] (3.11443,4.41918) rectangle (3.15423,4.52503);
\draw [color=c, fill=c] (3.15423,4.41918) rectangle (3.19403,4.52503);
\draw [color=c, fill=c] (3.19403,4.41918) rectangle (3.23383,4.52503);
\draw [color=c, fill=c] (3.23383,4.41918) rectangle (3.27363,4.52503);
\draw [color=c, fill=c] (3.27363,4.41918) rectangle (3.31343,4.52503);
\draw [color=c, fill=c] (3.31343,4.41918) rectangle (3.35323,4.52503);
\draw [color=c, fill=c] (3.35323,4.41918) rectangle (3.39303,4.52503);
\draw [color=c, fill=c] (3.39303,4.41918) rectangle (3.43284,4.52503);
\draw [color=c, fill=c] (3.43284,4.41918) rectangle (3.47264,4.52503);
\draw [color=c, fill=c] (3.47264,4.41918) rectangle (3.51244,4.52503);
\draw [color=c, fill=c] (3.51244,4.41918) rectangle (3.55224,4.52503);
\draw [color=c, fill=c] (3.55224,4.41918) rectangle (3.59204,4.52503);
\draw [color=c, fill=c] (3.59204,4.41918) rectangle (3.63184,4.52503);
\draw [color=c, fill=c] (3.63184,4.41918) rectangle (3.67164,4.52503);
\draw [color=c, fill=c] (3.67164,4.41918) rectangle (3.71144,4.52503);
\draw [color=c, fill=c] (3.71144,4.41918) rectangle (3.75124,4.52503);
\draw [color=c, fill=c] (3.75124,4.41918) rectangle (3.79104,4.52503);
\draw [color=c, fill=c] (3.79104,4.41918) rectangle (3.83085,4.52503);
\draw [color=c, fill=c] (3.83085,4.41918) rectangle (3.87065,4.52503);
\draw [color=c, fill=c] (3.87065,4.41918) rectangle (3.91045,4.52503);
\draw [color=c, fill=c] (3.91045,4.41918) rectangle (3.95025,4.52503);
\draw [color=c, fill=c] (3.95025,4.41918) rectangle (3.99005,4.52503);
\draw [color=c, fill=c] (3.99005,4.41918) rectangle (4.02985,4.52503);
\draw [color=c, fill=c] (4.02985,4.41918) rectangle (4.06965,4.52503);
\draw [color=c, fill=c] (4.06965,4.41918) rectangle (4.10945,4.52503);
\draw [color=c, fill=c] (4.10945,4.41918) rectangle (4.14925,4.52503);
\draw [color=c, fill=c] (4.14925,4.41918) rectangle (4.18905,4.52503);
\draw [color=c, fill=c] (4.18905,4.41918) rectangle (4.22886,4.52503);
\draw [color=c, fill=c] (4.22886,4.41918) rectangle (4.26866,4.52503);
\draw [color=c, fill=c] (4.26866,4.41918) rectangle (4.30846,4.52503);
\draw [color=c, fill=c] (4.30846,4.41918) rectangle (4.34826,4.52503);
\draw [color=c, fill=c] (4.34826,4.41918) rectangle (4.38806,4.52503);
\draw [color=c, fill=c] (4.38806,4.41918) rectangle (4.42786,4.52503);
\draw [color=c, fill=c] (4.42786,4.41918) rectangle (4.46766,4.52503);
\draw [color=c, fill=c] (4.46766,4.41918) rectangle (4.50746,4.52503);
\draw [color=c, fill=c] (4.50746,4.41918) rectangle (4.54726,4.52503);
\draw [color=c, fill=c] (4.54726,4.41918) rectangle (4.58706,4.52503);
\draw [color=c, fill=c] (4.58706,4.41918) rectangle (4.62687,4.52503);
\draw [color=c, fill=c] (4.62687,4.41918) rectangle (4.66667,4.52503);
\draw [color=c, fill=c] (4.66667,4.41918) rectangle (4.70647,4.52503);
\draw [color=c, fill=c] (4.70647,4.41918) rectangle (4.74627,4.52503);
\draw [color=c, fill=c] (4.74627,4.41918) rectangle (4.78607,4.52503);
\draw [color=c, fill=c] (4.78607,4.41918) rectangle (4.82587,4.52503);
\draw [color=c, fill=c] (4.82587,4.41918) rectangle (4.86567,4.52503);
\draw [color=c, fill=c] (4.86567,4.41918) rectangle (4.90547,4.52503);
\draw [color=c, fill=c] (4.90547,4.41918) rectangle (4.94527,4.52503);
\draw [color=c, fill=c] (4.94527,4.41918) rectangle (4.98507,4.52503);
\draw [color=c, fill=c] (4.98507,4.41918) rectangle (5.02488,4.52503);
\draw [color=c, fill=c] (5.02488,4.41918) rectangle (5.06468,4.52503);
\draw [color=c, fill=c] (5.06468,4.41918) rectangle (5.10448,4.52503);
\draw [color=c, fill=c] (5.10448,4.41918) rectangle (5.14428,4.52503);
\draw [color=c, fill=c] (5.14428,4.41918) rectangle (5.18408,4.52503);
\draw [color=c, fill=c] (5.18408,4.41918) rectangle (5.22388,4.52503);
\draw [color=c, fill=c] (5.22388,4.41918) rectangle (5.26368,4.52503);
\draw [color=c, fill=c] (5.26368,4.41918) rectangle (5.30348,4.52503);
\draw [color=c, fill=c] (5.30348,4.41918) rectangle (5.34328,4.52503);
\draw [color=c, fill=c] (5.34328,4.41918) rectangle (5.38308,4.52503);
\draw [color=c, fill=c] (5.38308,4.41918) rectangle (5.42289,4.52503);
\draw [color=c, fill=c] (5.42289,4.41918) rectangle (5.46269,4.52503);
\draw [color=c, fill=c] (5.46269,4.41918) rectangle (5.50249,4.52503);
\draw [color=c, fill=c] (5.50249,4.41918) rectangle (5.54229,4.52503);
\draw [color=c, fill=c] (5.54229,4.41918) rectangle (5.58209,4.52503);
\draw [color=c, fill=c] (5.58209,4.41918) rectangle (5.62189,4.52503);
\draw [color=c, fill=c] (5.62189,4.41918) rectangle (5.66169,4.52503);
\draw [color=c, fill=c] (5.66169,4.41918) rectangle (5.70149,4.52503);
\draw [color=c, fill=c] (5.70149,4.41918) rectangle (5.74129,4.52503);
\draw [color=c, fill=c] (5.74129,4.41918) rectangle (5.78109,4.52503);
\draw [color=c, fill=c] (5.78109,4.41918) rectangle (5.8209,4.52503);
\draw [color=c, fill=c] (5.8209,4.41918) rectangle (5.8607,4.52503);
\definecolor{c}{rgb}{0.2,0,1};
\draw [color=c, fill=c] (5.8607,4.41918) rectangle (5.9005,4.52503);
\draw [color=c, fill=c] (5.9005,4.41918) rectangle (5.9403,4.52503);
\draw [color=c, fill=c] (5.9403,4.41918) rectangle (5.9801,4.52503);
\draw [color=c, fill=c] (5.9801,4.41918) rectangle (6.0199,4.52503);
\draw [color=c, fill=c] (6.0199,4.41918) rectangle (6.0597,4.52503);
\draw [color=c, fill=c] (6.0597,4.41918) rectangle (6.0995,4.52503);
\draw [color=c, fill=c] (6.0995,4.41918) rectangle (6.1393,4.52503);
\draw [color=c, fill=c] (6.1393,4.41918) rectangle (6.1791,4.52503);
\draw [color=c, fill=c] (6.1791,4.41918) rectangle (6.21891,4.52503);
\draw [color=c, fill=c] (6.21891,4.41918) rectangle (6.25871,4.52503);
\draw [color=c, fill=c] (6.25871,4.41918) rectangle (6.29851,4.52503);
\draw [color=c, fill=c] (6.29851,4.41918) rectangle (6.33831,4.52503);
\draw [color=c, fill=c] (6.33831,4.41918) rectangle (6.37811,4.52503);
\draw [color=c, fill=c] (6.37811,4.41918) rectangle (6.41791,4.52503);
\draw [color=c, fill=c] (6.41791,4.41918) rectangle (6.45771,4.52503);
\draw [color=c, fill=c] (6.45771,4.41918) rectangle (6.49751,4.52503);
\draw [color=c, fill=c] (6.49751,4.41918) rectangle (6.53731,4.52503);
\draw [color=c, fill=c] (6.53731,4.41918) rectangle (6.57711,4.52503);
\draw [color=c, fill=c] (6.57711,4.41918) rectangle (6.61692,4.52503);
\draw [color=c, fill=c] (6.61692,4.41918) rectangle (6.65672,4.52503);
\draw [color=c, fill=c] (6.65672,4.41918) rectangle (6.69652,4.52503);
\draw [color=c, fill=c] (6.69652,4.41918) rectangle (6.73632,4.52503);
\draw [color=c, fill=c] (6.73632,4.41918) rectangle (6.77612,4.52503);
\draw [color=c, fill=c] (6.77612,4.41918) rectangle (6.81592,4.52503);
\draw [color=c, fill=c] (6.81592,4.41918) rectangle (6.85572,4.52503);
\draw [color=c, fill=c] (6.85572,4.41918) rectangle (6.89552,4.52503);
\draw [color=c, fill=c] (6.89552,4.41918) rectangle (6.93532,4.52503);
\draw [color=c, fill=c] (6.93532,4.41918) rectangle (6.97512,4.52503);
\draw [color=c, fill=c] (6.97512,4.41918) rectangle (7.01493,4.52503);
\draw [color=c, fill=c] (7.01493,4.41918) rectangle (7.05473,4.52503);
\draw [color=c, fill=c] (7.05473,4.41918) rectangle (7.09453,4.52503);
\draw [color=c, fill=c] (7.09453,4.41918) rectangle (7.13433,4.52503);
\draw [color=c, fill=c] (7.13433,4.41918) rectangle (7.17413,4.52503);
\draw [color=c, fill=c] (7.17413,4.41918) rectangle (7.21393,4.52503);
\draw [color=c, fill=c] (7.21393,4.41918) rectangle (7.25373,4.52503);
\draw [color=c, fill=c] (7.25373,4.41918) rectangle (7.29353,4.52503);
\draw [color=c, fill=c] (7.29353,4.41918) rectangle (7.33333,4.52503);
\draw [color=c, fill=c] (7.33333,4.41918) rectangle (7.37313,4.52503);
\draw [color=c, fill=c] (7.37313,4.41918) rectangle (7.41294,4.52503);
\draw [color=c, fill=c] (7.41294,4.41918) rectangle (7.45274,4.52503);
\draw [color=c, fill=c] (7.45274,4.41918) rectangle (7.49254,4.52503);
\draw [color=c, fill=c] (7.49254,4.41918) rectangle (7.53234,4.52503);
\draw [color=c, fill=c] (7.53234,4.41918) rectangle (7.57214,4.52503);
\draw [color=c, fill=c] (7.57214,4.41918) rectangle (7.61194,4.52503);
\draw [color=c, fill=c] (7.61194,4.41918) rectangle (7.65174,4.52503);
\draw [color=c, fill=c] (7.65174,4.41918) rectangle (7.69154,4.52503);
\draw [color=c, fill=c] (7.69154,4.41918) rectangle (7.73134,4.52503);
\draw [color=c, fill=c] (7.73134,4.41918) rectangle (7.77114,4.52503);
\draw [color=c, fill=c] (7.77114,4.41918) rectangle (7.81095,4.52503);
\draw [color=c, fill=c] (7.81095,4.41918) rectangle (7.85075,4.52503);
\draw [color=c, fill=c] (7.85075,4.41918) rectangle (7.89055,4.52503);
\draw [color=c, fill=c] (7.89055,4.41918) rectangle (7.93035,4.52503);
\draw [color=c, fill=c] (7.93035,4.41918) rectangle (7.97015,4.52503);
\draw [color=c, fill=c] (7.97015,4.41918) rectangle (8.00995,4.52503);
\draw [color=c, fill=c] (8.00995,4.41918) rectangle (8.04975,4.52503);
\draw [color=c, fill=c] (8.04975,4.41918) rectangle (8.08955,4.52503);
\draw [color=c, fill=c] (8.08955,4.41918) rectangle (8.12935,4.52503);
\draw [color=c, fill=c] (8.12935,4.41918) rectangle (8.16915,4.52503);
\draw [color=c, fill=c] (8.16915,4.41918) rectangle (8.20895,4.52503);
\draw [color=c, fill=c] (8.20895,4.41918) rectangle (8.24876,4.52503);
\draw [color=c, fill=c] (8.24876,4.41918) rectangle (8.28856,4.52503);
\draw [color=c, fill=c] (8.28856,4.41918) rectangle (8.32836,4.52503);
\draw [color=c, fill=c] (8.32836,4.41918) rectangle (8.36816,4.52503);
\draw [color=c, fill=c] (8.36816,4.41918) rectangle (8.40796,4.52503);
\draw [color=c, fill=c] (8.40796,4.41918) rectangle (8.44776,4.52503);
\draw [color=c, fill=c] (8.44776,4.41918) rectangle (8.48756,4.52503);
\draw [color=c, fill=c] (8.48756,4.41918) rectangle (8.52736,4.52503);
\draw [color=c, fill=c] (8.52736,4.41918) rectangle (8.56716,4.52503);
\draw [color=c, fill=c] (8.56716,4.41918) rectangle (8.60697,4.52503);
\draw [color=c, fill=c] (8.60697,4.41918) rectangle (8.64677,4.52503);
\draw [color=c, fill=c] (8.64677,4.41918) rectangle (8.68657,4.52503);
\draw [color=c, fill=c] (8.68657,4.41918) rectangle (8.72637,4.52503);
\draw [color=c, fill=c] (8.72637,4.41918) rectangle (8.76617,4.52503);
\draw [color=c, fill=c] (8.76617,4.41918) rectangle (8.80597,4.52503);
\draw [color=c, fill=c] (8.80597,4.41918) rectangle (8.84577,4.52503);
\draw [color=c, fill=c] (8.84577,4.41918) rectangle (8.88557,4.52503);
\draw [color=c, fill=c] (8.88557,4.41918) rectangle (8.92537,4.52503);
\draw [color=c, fill=c] (8.92537,4.41918) rectangle (8.96517,4.52503);
\draw [color=c, fill=c] (8.96517,4.41918) rectangle (9.00498,4.52503);
\draw [color=c, fill=c] (9.00498,4.41918) rectangle (9.04478,4.52503);
\definecolor{c}{rgb}{0.386667,0,1};
\draw [color=c, fill=c] (9.04478,4.41918) rectangle (9.08458,4.52503);
\draw [color=c, fill=c] (9.08458,4.41918) rectangle (9.12438,4.52503);
\draw [color=c, fill=c] (9.12438,4.41918) rectangle (9.16418,4.52503);
\draw [color=c, fill=c] (9.16418,4.41918) rectangle (9.20398,4.52503);
\draw [color=c, fill=c] (9.20398,4.41918) rectangle (9.24378,4.52503);
\draw [color=c, fill=c] (9.24378,4.41918) rectangle (9.28358,4.52503);
\draw [color=c, fill=c] (9.28358,4.41918) rectangle (9.32338,4.52503);
\definecolor{c}{rgb}{0.2,0,1};
\draw [color=c, fill=c] (9.32338,4.41918) rectangle (9.36318,4.52503);
\draw [color=c, fill=c] (9.36318,4.41918) rectangle (9.40298,4.52503);
\draw [color=c, fill=c] (9.40298,4.41918) rectangle (9.44279,4.52503);
\draw [color=c, fill=c] (9.44279,4.41918) rectangle (9.48259,4.52503);
\draw [color=c, fill=c] (9.48259,4.41918) rectangle (9.52239,4.52503);
\draw [color=c, fill=c] (9.52239,4.41918) rectangle (9.56219,4.52503);
\draw [color=c, fill=c] (9.56219,4.41918) rectangle (9.60199,4.52503);
\draw [color=c, fill=c] (9.60199,4.41918) rectangle (9.64179,4.52503);
\draw [color=c, fill=c] (9.64179,4.41918) rectangle (9.68159,4.52503);
\draw [color=c, fill=c] (9.68159,4.41918) rectangle (9.72139,4.52503);
\draw [color=c, fill=c] (9.72139,4.41918) rectangle (9.76119,4.52503);
\definecolor{c}{rgb}{0,0.0800001,1};
\draw [color=c, fill=c] (9.76119,4.41918) rectangle (9.80099,4.52503);
\draw [color=c, fill=c] (9.80099,4.41918) rectangle (9.8408,4.52503);
\draw [color=c, fill=c] (9.8408,4.41918) rectangle (9.8806,4.52503);
\definecolor{c}{rgb}{0,0.266667,1};
\draw [color=c, fill=c] (9.8806,4.41918) rectangle (9.9204,4.52503);
\draw [color=c, fill=c] (9.9204,4.41918) rectangle (9.9602,4.52503);
\draw [color=c, fill=c] (9.9602,4.41918) rectangle (10,4.52503);
\definecolor{c}{rgb}{0,0.546666,1};
\draw [color=c, fill=c] (10,4.41918) rectangle (10.0398,4.52503);
\draw [color=c, fill=c] (10.0398,4.41918) rectangle (10.0796,4.52503);
\draw [color=c, fill=c] (10.0796,4.41918) rectangle (10.1194,4.52503);
\definecolor{c}{rgb}{0,0.733333,1};
\draw [color=c, fill=c] (10.1194,4.41918) rectangle (10.1592,4.52503);
\draw [color=c, fill=c] (10.1592,4.41918) rectangle (10.199,4.52503);
\draw [color=c, fill=c] (10.199,4.41918) rectangle (10.2388,4.52503);
\draw [color=c, fill=c] (10.2388,4.41918) rectangle (10.2786,4.52503);
\draw [color=c, fill=c] (10.2786,4.41918) rectangle (10.3184,4.52503);
\definecolor{c}{rgb}{0,1,0.986667};
\draw [color=c, fill=c] (10.3184,4.41918) rectangle (10.3582,4.52503);
\draw [color=c, fill=c] (10.3582,4.41918) rectangle (10.398,4.52503);
\draw [color=c, fill=c] (10.398,4.41918) rectangle (10.4378,4.52503);
\draw [color=c, fill=c] (10.4378,4.41918) rectangle (10.4776,4.52503);
\draw [color=c, fill=c] (10.4776,4.41918) rectangle (10.5174,4.52503);
\draw [color=c, fill=c] (10.5174,4.41918) rectangle (10.5572,4.52503);
\draw [color=c, fill=c] (10.5572,4.41918) rectangle (10.597,4.52503);
\draw [color=c, fill=c] (10.597,4.41918) rectangle (10.6368,4.52503);
\draw [color=c, fill=c] (10.6368,4.41918) rectangle (10.6766,4.52503);
\draw [color=c, fill=c] (10.6766,4.41918) rectangle (10.7164,4.52503);
\draw [color=c, fill=c] (10.7164,4.41918) rectangle (10.7562,4.52503);
\draw [color=c, fill=c] (10.7562,4.41918) rectangle (10.796,4.52503);
\draw [color=c, fill=c] (10.796,4.41918) rectangle (10.8358,4.52503);
\draw [color=c, fill=c] (10.8358,4.41918) rectangle (10.8756,4.52503);
\draw [color=c, fill=c] (10.8756,4.41918) rectangle (10.9154,4.52503);
\draw [color=c, fill=c] (10.9154,4.41918) rectangle (10.9552,4.52503);
\draw [color=c, fill=c] (10.9552,4.41918) rectangle (10.995,4.52503);
\draw [color=c, fill=c] (10.995,4.41918) rectangle (11.0348,4.52503);
\draw [color=c, fill=c] (11.0348,4.41918) rectangle (11.0746,4.52503);
\draw [color=c, fill=c] (11.0746,4.41918) rectangle (11.1144,4.52503);
\draw [color=c, fill=c] (11.1144,4.41918) rectangle (11.1542,4.52503);
\draw [color=c, fill=c] (11.1542,4.41918) rectangle (11.194,4.52503);
\draw [color=c, fill=c] (11.194,4.41918) rectangle (11.2338,4.52503);
\draw [color=c, fill=c] (11.2338,4.41918) rectangle (11.2736,4.52503);
\draw [color=c, fill=c] (11.2736,4.41918) rectangle (11.3134,4.52503);
\draw [color=c, fill=c] (11.3134,4.41918) rectangle (11.3532,4.52503);
\draw [color=c, fill=c] (11.3532,4.41918) rectangle (11.393,4.52503);
\draw [color=c, fill=c] (11.393,4.41918) rectangle (11.4328,4.52503);
\draw [color=c, fill=c] (11.4328,4.41918) rectangle (11.4726,4.52503);
\draw [color=c, fill=c] (11.4726,4.41918) rectangle (11.5124,4.52503);
\draw [color=c, fill=c] (11.5124,4.41918) rectangle (11.5522,4.52503);
\definecolor{c}{rgb}{0,0.733333,1};
\draw [color=c, fill=c] (11.5522,4.41918) rectangle (11.592,4.52503);
\draw [color=c, fill=c] (11.592,4.41918) rectangle (11.6318,4.52503);
\draw [color=c, fill=c] (11.6318,4.41918) rectangle (11.6716,4.52503);
\draw [color=c, fill=c] (11.6716,4.41918) rectangle (11.7114,4.52503);
\draw [color=c, fill=c] (11.7114,4.41918) rectangle (11.7512,4.52503);
\draw [color=c, fill=c] (11.7512,4.41918) rectangle (11.791,4.52503);
\draw [color=c, fill=c] (11.791,4.41918) rectangle (11.8308,4.52503);
\draw [color=c, fill=c] (11.8308,4.41918) rectangle (11.8706,4.52503);
\draw [color=c, fill=c] (11.8706,4.41918) rectangle (11.9104,4.52503);
\draw [color=c, fill=c] (11.9104,4.41918) rectangle (11.9502,4.52503);
\draw [color=c, fill=c] (11.9502,4.41918) rectangle (11.99,4.52503);
\draw [color=c, fill=c] (11.99,4.41918) rectangle (12.0299,4.52503);
\draw [color=c, fill=c] (12.0299,4.41918) rectangle (12.0697,4.52503);
\draw [color=c, fill=c] (12.0697,4.41918) rectangle (12.1095,4.52503);
\draw [color=c, fill=c] (12.1095,4.41918) rectangle (12.1493,4.52503);
\draw [color=c, fill=c] (12.1493,4.41918) rectangle (12.1891,4.52503);
\draw [color=c, fill=c] (12.1891,4.41918) rectangle (12.2289,4.52503);
\draw [color=c, fill=c] (12.2289,4.41918) rectangle (12.2687,4.52503);
\draw [color=c, fill=c] (12.2687,4.41918) rectangle (12.3085,4.52503);
\draw [color=c, fill=c] (12.3085,4.41918) rectangle (12.3483,4.52503);
\draw [color=c, fill=c] (12.3483,4.41918) rectangle (12.3881,4.52503);
\draw [color=c, fill=c] (12.3881,4.41918) rectangle (12.4279,4.52503);
\draw [color=c, fill=c] (12.4279,4.41918) rectangle (12.4677,4.52503);
\draw [color=c, fill=c] (12.4677,4.41918) rectangle (12.5075,4.52503);
\draw [color=c, fill=c] (12.5075,4.41918) rectangle (12.5473,4.52503);
\draw [color=c, fill=c] (12.5473,4.41918) rectangle (12.5871,4.52503);
\draw [color=c, fill=c] (12.5871,4.41918) rectangle (12.6269,4.52503);
\draw [color=c, fill=c] (12.6269,4.41918) rectangle (12.6667,4.52503);
\draw [color=c, fill=c] (12.6667,4.41918) rectangle (12.7065,4.52503);
\draw [color=c, fill=c] (12.7065,4.41918) rectangle (12.7463,4.52503);
\draw [color=c, fill=c] (12.7463,4.41918) rectangle (12.7861,4.52503);
\draw [color=c, fill=c] (12.7861,4.41918) rectangle (12.8259,4.52503);
\draw [color=c, fill=c] (12.8259,4.41918) rectangle (12.8657,4.52503);
\draw [color=c, fill=c] (12.8657,4.41918) rectangle (12.9055,4.52503);
\draw [color=c, fill=c] (12.9055,4.41918) rectangle (12.9453,4.52503);
\draw [color=c, fill=c] (12.9453,4.41918) rectangle (12.9851,4.52503);
\draw [color=c, fill=c] (12.9851,4.41918) rectangle (13.0249,4.52503);
\draw [color=c, fill=c] (13.0249,4.41918) rectangle (13.0647,4.52503);
\draw [color=c, fill=c] (13.0647,4.41918) rectangle (13.1045,4.52503);
\draw [color=c, fill=c] (13.1045,4.41918) rectangle (13.1443,4.52503);
\draw [color=c, fill=c] (13.1443,4.41918) rectangle (13.1841,4.52503);
\draw [color=c, fill=c] (13.1841,4.41918) rectangle (13.2239,4.52503);
\draw [color=c, fill=c] (13.2239,4.41918) rectangle (13.2637,4.52503);
\draw [color=c, fill=c] (13.2637,4.41918) rectangle (13.3035,4.52503);
\draw [color=c, fill=c] (13.3035,4.41918) rectangle (13.3433,4.52503);
\draw [color=c, fill=c] (13.3433,4.41918) rectangle (13.3831,4.52503);
\draw [color=c, fill=c] (13.3831,4.41918) rectangle (13.4229,4.52503);
\draw [color=c, fill=c] (13.4229,4.41918) rectangle (13.4627,4.52503);
\draw [color=c, fill=c] (13.4627,4.41918) rectangle (13.5025,4.52503);
\draw [color=c, fill=c] (13.5025,4.41918) rectangle (13.5423,4.52503);
\draw [color=c, fill=c] (13.5423,4.41918) rectangle (13.5821,4.52503);
\draw [color=c, fill=c] (13.5821,4.41918) rectangle (13.6219,4.52503);
\draw [color=c, fill=c] (13.6219,4.41918) rectangle (13.6617,4.52503);
\draw [color=c, fill=c] (13.6617,4.41918) rectangle (13.7015,4.52503);
\draw [color=c, fill=c] (13.7015,4.41918) rectangle (13.7413,4.52503);
\draw [color=c, fill=c] (13.7413,4.41918) rectangle (13.7811,4.52503);
\draw [color=c, fill=c] (13.7811,4.41918) rectangle (13.8209,4.52503);
\draw [color=c, fill=c] (13.8209,4.41918) rectangle (13.8607,4.52503);
\draw [color=c, fill=c] (13.8607,4.41918) rectangle (13.9005,4.52503);
\draw [color=c, fill=c] (13.9005,4.41918) rectangle (13.9403,4.52503);
\draw [color=c, fill=c] (13.9403,4.41918) rectangle (13.9801,4.52503);
\draw [color=c, fill=c] (13.9801,4.41918) rectangle (14.0199,4.52503);
\draw [color=c, fill=c] (14.0199,4.41918) rectangle (14.0597,4.52503);
\draw [color=c, fill=c] (14.0597,4.41918) rectangle (14.0995,4.52503);
\draw [color=c, fill=c] (14.0995,4.41918) rectangle (14.1393,4.52503);
\draw [color=c, fill=c] (14.1393,4.41918) rectangle (14.1791,4.52503);
\draw [color=c, fill=c] (14.1791,4.41918) rectangle (14.2189,4.52503);
\draw [color=c, fill=c] (14.2189,4.41918) rectangle (14.2587,4.52503);
\draw [color=c, fill=c] (14.2587,4.41918) rectangle (14.2985,4.52503);
\draw [color=c, fill=c] (14.2985,4.41918) rectangle (14.3383,4.52503);
\draw [color=c, fill=c] (14.3383,4.41918) rectangle (14.3781,4.52503);
\draw [color=c, fill=c] (14.3781,4.41918) rectangle (14.4179,4.52503);
\draw [color=c, fill=c] (14.4179,4.41918) rectangle (14.4577,4.52503);
\draw [color=c, fill=c] (14.4577,4.41918) rectangle (14.4975,4.52503);
\draw [color=c, fill=c] (14.4975,4.41918) rectangle (14.5373,4.52503);
\draw [color=c, fill=c] (14.5373,4.41918) rectangle (14.5771,4.52503);
\draw [color=c, fill=c] (14.5771,4.41918) rectangle (14.6169,4.52503);
\draw [color=c, fill=c] (14.6169,4.41918) rectangle (14.6567,4.52503);
\draw [color=c, fill=c] (14.6567,4.41918) rectangle (14.6965,4.52503);
\draw [color=c, fill=c] (14.6965,4.41918) rectangle (14.7363,4.52503);
\draw [color=c, fill=c] (14.7363,4.41918) rectangle (14.7761,4.52503);
\draw [color=c, fill=c] (14.7761,4.41918) rectangle (14.8159,4.52503);
\draw [color=c, fill=c] (14.8159,4.41918) rectangle (14.8557,4.52503);
\draw [color=c, fill=c] (14.8557,4.41918) rectangle (14.8955,4.52503);
\draw [color=c, fill=c] (14.8955,4.41918) rectangle (14.9353,4.52503);
\draw [color=c, fill=c] (14.9353,4.41918) rectangle (14.9751,4.52503);
\draw [color=c, fill=c] (14.9751,4.41918) rectangle (15.0149,4.52503);
\draw [color=c, fill=c] (15.0149,4.41918) rectangle (15.0547,4.52503);
\draw [color=c, fill=c] (15.0547,4.41918) rectangle (15.0945,4.52503);
\draw [color=c, fill=c] (15.0945,4.41918) rectangle (15.1343,4.52503);
\draw [color=c, fill=c] (15.1343,4.41918) rectangle (15.1741,4.52503);
\draw [color=c, fill=c] (15.1741,4.41918) rectangle (15.2139,4.52503);
\draw [color=c, fill=c] (15.2139,4.41918) rectangle (15.2537,4.52503);
\draw [color=c, fill=c] (15.2537,4.41918) rectangle (15.2935,4.52503);
\draw [color=c, fill=c] (15.2935,4.41918) rectangle (15.3333,4.52503);
\draw [color=c, fill=c] (15.3333,4.41918) rectangle (15.3731,4.52503);
\draw [color=c, fill=c] (15.3731,4.41918) rectangle (15.4129,4.52503);
\draw [color=c, fill=c] (15.4129,4.41918) rectangle (15.4527,4.52503);
\draw [color=c, fill=c] (15.4527,4.41918) rectangle (15.4925,4.52503);
\draw [color=c, fill=c] (15.4925,4.41918) rectangle (15.5323,4.52503);
\draw [color=c, fill=c] (15.5323,4.41918) rectangle (15.5721,4.52503);
\draw [color=c, fill=c] (15.5721,4.41918) rectangle (15.6119,4.52503);
\draw [color=c, fill=c] (15.6119,4.41918) rectangle (15.6517,4.52503);
\draw [color=c, fill=c] (15.6517,4.41918) rectangle (15.6915,4.52503);
\draw [color=c, fill=c] (15.6915,4.41918) rectangle (15.7313,4.52503);
\draw [color=c, fill=c] (15.7313,4.41918) rectangle (15.7711,4.52503);
\draw [color=c, fill=c] (15.7711,4.41918) rectangle (15.8109,4.52503);
\draw [color=c, fill=c] (15.8109,4.41918) rectangle (15.8507,4.52503);
\draw [color=c, fill=c] (15.8507,4.41918) rectangle (15.8905,4.52503);
\draw [color=c, fill=c] (15.8905,4.41918) rectangle (15.9303,4.52503);
\draw [color=c, fill=c] (15.9303,4.41918) rectangle (15.9701,4.52503);
\draw [color=c, fill=c] (15.9701,4.41918) rectangle (16.01,4.52503);
\draw [color=c, fill=c] (16.01,4.41918) rectangle (16.0498,4.52503);
\draw [color=c, fill=c] (16.0498,4.41918) rectangle (16.0896,4.52503);
\draw [color=c, fill=c] (16.0896,4.41918) rectangle (16.1294,4.52503);
\draw [color=c, fill=c] (16.1294,4.41918) rectangle (16.1692,4.52503);
\draw [color=c, fill=c] (16.1692,4.41918) rectangle (16.209,4.52503);
\draw [color=c, fill=c] (16.209,4.41918) rectangle (16.2488,4.52503);
\draw [color=c, fill=c] (16.2488,4.41918) rectangle (16.2886,4.52503);
\draw [color=c, fill=c] (16.2886,4.41918) rectangle (16.3284,4.52503);
\draw [color=c, fill=c] (16.3284,4.41918) rectangle (16.3682,4.52503);
\draw [color=c, fill=c] (16.3682,4.41918) rectangle (16.408,4.52503);
\draw [color=c, fill=c] (16.408,4.41918) rectangle (16.4478,4.52503);
\draw [color=c, fill=c] (16.4478,4.41918) rectangle (16.4876,4.52503);
\draw [color=c, fill=c] (16.4876,4.41918) rectangle (16.5274,4.52503);
\draw [color=c, fill=c] (16.5274,4.41918) rectangle (16.5672,4.52503);
\draw [color=c, fill=c] (16.5672,4.41918) rectangle (16.607,4.52503);
\draw [color=c, fill=c] (16.607,4.41918) rectangle (16.6468,4.52503);
\draw [color=c, fill=c] (16.6468,4.41918) rectangle (16.6866,4.52503);
\draw [color=c, fill=c] (16.6866,4.41918) rectangle (16.7264,4.52503);
\draw [color=c, fill=c] (16.7264,4.41918) rectangle (16.7662,4.52503);
\draw [color=c, fill=c] (16.7662,4.41918) rectangle (16.806,4.52503);
\draw [color=c, fill=c] (16.806,4.41918) rectangle (16.8458,4.52503);
\draw [color=c, fill=c] (16.8458,4.41918) rectangle (16.8856,4.52503);
\draw [color=c, fill=c] (16.8856,4.41918) rectangle (16.9254,4.52503);
\draw [color=c, fill=c] (16.9254,4.41918) rectangle (16.9652,4.52503);
\draw [color=c, fill=c] (16.9652,4.41918) rectangle (17.005,4.52503);
\draw [color=c, fill=c] (17.005,4.41918) rectangle (17.0448,4.52503);
\draw [color=c, fill=c] (17.0448,4.41918) rectangle (17.0846,4.52503);
\draw [color=c, fill=c] (17.0846,4.41918) rectangle (17.1244,4.52503);
\draw [color=c, fill=c] (17.1244,4.41918) rectangle (17.1642,4.52503);
\draw [color=c, fill=c] (17.1642,4.41918) rectangle (17.204,4.52503);
\draw [color=c, fill=c] (17.204,4.41918) rectangle (17.2438,4.52503);
\draw [color=c, fill=c] (17.2438,4.41918) rectangle (17.2836,4.52503);
\draw [color=c, fill=c] (17.2836,4.41918) rectangle (17.3234,4.52503);
\draw [color=c, fill=c] (17.3234,4.41918) rectangle (17.3632,4.52503);
\draw [color=c, fill=c] (17.3632,4.41918) rectangle (17.403,4.52503);
\draw [color=c, fill=c] (17.403,4.41918) rectangle (17.4428,4.52503);
\draw [color=c, fill=c] (17.4428,4.41918) rectangle (17.4826,4.52503);
\draw [color=c, fill=c] (17.4826,4.41918) rectangle (17.5224,4.52503);
\draw [color=c, fill=c] (17.5224,4.41918) rectangle (17.5622,4.52503);
\draw [color=c, fill=c] (17.5622,4.41918) rectangle (17.602,4.52503);
\draw [color=c, fill=c] (17.602,4.41918) rectangle (17.6418,4.52503);
\draw [color=c, fill=c] (17.6418,4.41918) rectangle (17.6816,4.52503);
\draw [color=c, fill=c] (17.6816,4.41918) rectangle (17.7214,4.52503);
\draw [color=c, fill=c] (17.7214,4.41918) rectangle (17.7612,4.52503);
\draw [color=c, fill=c] (17.7612,4.41918) rectangle (17.801,4.52503);
\draw [color=c, fill=c] (17.801,4.41918) rectangle (17.8408,4.52503);
\draw [color=c, fill=c] (17.8408,4.41918) rectangle (17.8806,4.52503);
\draw [color=c, fill=c] (17.8806,4.41918) rectangle (17.9204,4.52503);
\draw [color=c, fill=c] (17.9204,4.41918) rectangle (17.9602,4.52503);
\draw [color=c, fill=c] (17.9602,4.41918) rectangle (18,4.52503);
\definecolor{c}{rgb}{0,0.0800001,1};
\draw [color=c, fill=c] (2,4.52503) rectangle (2.0398,4.63088);
\draw [color=c, fill=c] (2.0398,4.52503) rectangle (2.0796,4.63088);
\draw [color=c, fill=c] (2.0796,4.52503) rectangle (2.1194,4.63088);
\draw [color=c, fill=c] (2.1194,4.52503) rectangle (2.1592,4.63088);
\draw [color=c, fill=c] (2.1592,4.52503) rectangle (2.19901,4.63088);
\draw [color=c, fill=c] (2.19901,4.52503) rectangle (2.23881,4.63088);
\draw [color=c, fill=c] (2.23881,4.52503) rectangle (2.27861,4.63088);
\draw [color=c, fill=c] (2.27861,4.52503) rectangle (2.31841,4.63088);
\draw [color=c, fill=c] (2.31841,4.52503) rectangle (2.35821,4.63088);
\draw [color=c, fill=c] (2.35821,4.52503) rectangle (2.39801,4.63088);
\draw [color=c, fill=c] (2.39801,4.52503) rectangle (2.43781,4.63088);
\draw [color=c, fill=c] (2.43781,4.52503) rectangle (2.47761,4.63088);
\draw [color=c, fill=c] (2.47761,4.52503) rectangle (2.51741,4.63088);
\draw [color=c, fill=c] (2.51741,4.52503) rectangle (2.55721,4.63088);
\draw [color=c, fill=c] (2.55721,4.52503) rectangle (2.59702,4.63088);
\draw [color=c, fill=c] (2.59702,4.52503) rectangle (2.63682,4.63088);
\draw [color=c, fill=c] (2.63682,4.52503) rectangle (2.67662,4.63088);
\draw [color=c, fill=c] (2.67662,4.52503) rectangle (2.71642,4.63088);
\draw [color=c, fill=c] (2.71642,4.52503) rectangle (2.75622,4.63088);
\draw [color=c, fill=c] (2.75622,4.52503) rectangle (2.79602,4.63088);
\draw [color=c, fill=c] (2.79602,4.52503) rectangle (2.83582,4.63088);
\draw [color=c, fill=c] (2.83582,4.52503) rectangle (2.87562,4.63088);
\draw [color=c, fill=c] (2.87562,4.52503) rectangle (2.91542,4.63088);
\draw [color=c, fill=c] (2.91542,4.52503) rectangle (2.95522,4.63088);
\draw [color=c, fill=c] (2.95522,4.52503) rectangle (2.99502,4.63088);
\draw [color=c, fill=c] (2.99502,4.52503) rectangle (3.03483,4.63088);
\draw [color=c, fill=c] (3.03483,4.52503) rectangle (3.07463,4.63088);
\draw [color=c, fill=c] (3.07463,4.52503) rectangle (3.11443,4.63088);
\draw [color=c, fill=c] (3.11443,4.52503) rectangle (3.15423,4.63088);
\draw [color=c, fill=c] (3.15423,4.52503) rectangle (3.19403,4.63088);
\draw [color=c, fill=c] (3.19403,4.52503) rectangle (3.23383,4.63088);
\draw [color=c, fill=c] (3.23383,4.52503) rectangle (3.27363,4.63088);
\draw [color=c, fill=c] (3.27363,4.52503) rectangle (3.31343,4.63088);
\draw [color=c, fill=c] (3.31343,4.52503) rectangle (3.35323,4.63088);
\draw [color=c, fill=c] (3.35323,4.52503) rectangle (3.39303,4.63088);
\draw [color=c, fill=c] (3.39303,4.52503) rectangle (3.43284,4.63088);
\draw [color=c, fill=c] (3.43284,4.52503) rectangle (3.47264,4.63088);
\draw [color=c, fill=c] (3.47264,4.52503) rectangle (3.51244,4.63088);
\draw [color=c, fill=c] (3.51244,4.52503) rectangle (3.55224,4.63088);
\draw [color=c, fill=c] (3.55224,4.52503) rectangle (3.59204,4.63088);
\draw [color=c, fill=c] (3.59204,4.52503) rectangle (3.63184,4.63088);
\draw [color=c, fill=c] (3.63184,4.52503) rectangle (3.67164,4.63088);
\draw [color=c, fill=c] (3.67164,4.52503) rectangle (3.71144,4.63088);
\draw [color=c, fill=c] (3.71144,4.52503) rectangle (3.75124,4.63088);
\draw [color=c, fill=c] (3.75124,4.52503) rectangle (3.79104,4.63088);
\draw [color=c, fill=c] (3.79104,4.52503) rectangle (3.83085,4.63088);
\draw [color=c, fill=c] (3.83085,4.52503) rectangle (3.87065,4.63088);
\draw [color=c, fill=c] (3.87065,4.52503) rectangle (3.91045,4.63088);
\draw [color=c, fill=c] (3.91045,4.52503) rectangle (3.95025,4.63088);
\draw [color=c, fill=c] (3.95025,4.52503) rectangle (3.99005,4.63088);
\draw [color=c, fill=c] (3.99005,4.52503) rectangle (4.02985,4.63088);
\draw [color=c, fill=c] (4.02985,4.52503) rectangle (4.06965,4.63088);
\draw [color=c, fill=c] (4.06965,4.52503) rectangle (4.10945,4.63088);
\draw [color=c, fill=c] (4.10945,4.52503) rectangle (4.14925,4.63088);
\draw [color=c, fill=c] (4.14925,4.52503) rectangle (4.18905,4.63088);
\draw [color=c, fill=c] (4.18905,4.52503) rectangle (4.22886,4.63088);
\draw [color=c, fill=c] (4.22886,4.52503) rectangle (4.26866,4.63088);
\draw [color=c, fill=c] (4.26866,4.52503) rectangle (4.30846,4.63088);
\draw [color=c, fill=c] (4.30846,4.52503) rectangle (4.34826,4.63088);
\draw [color=c, fill=c] (4.34826,4.52503) rectangle (4.38806,4.63088);
\draw [color=c, fill=c] (4.38806,4.52503) rectangle (4.42786,4.63088);
\draw [color=c, fill=c] (4.42786,4.52503) rectangle (4.46766,4.63088);
\draw [color=c, fill=c] (4.46766,4.52503) rectangle (4.50746,4.63088);
\draw [color=c, fill=c] (4.50746,4.52503) rectangle (4.54726,4.63088);
\draw [color=c, fill=c] (4.54726,4.52503) rectangle (4.58706,4.63088);
\draw [color=c, fill=c] (4.58706,4.52503) rectangle (4.62687,4.63088);
\draw [color=c, fill=c] (4.62687,4.52503) rectangle (4.66667,4.63088);
\draw [color=c, fill=c] (4.66667,4.52503) rectangle (4.70647,4.63088);
\draw [color=c, fill=c] (4.70647,4.52503) rectangle (4.74627,4.63088);
\draw [color=c, fill=c] (4.74627,4.52503) rectangle (4.78607,4.63088);
\draw [color=c, fill=c] (4.78607,4.52503) rectangle (4.82587,4.63088);
\draw [color=c, fill=c] (4.82587,4.52503) rectangle (4.86567,4.63088);
\draw [color=c, fill=c] (4.86567,4.52503) rectangle (4.90547,4.63088);
\draw [color=c, fill=c] (4.90547,4.52503) rectangle (4.94527,4.63088);
\draw [color=c, fill=c] (4.94527,4.52503) rectangle (4.98507,4.63088);
\draw [color=c, fill=c] (4.98507,4.52503) rectangle (5.02488,4.63088);
\draw [color=c, fill=c] (5.02488,4.52503) rectangle (5.06468,4.63088);
\draw [color=c, fill=c] (5.06468,4.52503) rectangle (5.10448,4.63088);
\draw [color=c, fill=c] (5.10448,4.52503) rectangle (5.14428,4.63088);
\draw [color=c, fill=c] (5.14428,4.52503) rectangle (5.18408,4.63088);
\draw [color=c, fill=c] (5.18408,4.52503) rectangle (5.22388,4.63088);
\draw [color=c, fill=c] (5.22388,4.52503) rectangle (5.26368,4.63088);
\draw [color=c, fill=c] (5.26368,4.52503) rectangle (5.30348,4.63088);
\draw [color=c, fill=c] (5.30348,4.52503) rectangle (5.34328,4.63088);
\draw [color=c, fill=c] (5.34328,4.52503) rectangle (5.38308,4.63088);
\draw [color=c, fill=c] (5.38308,4.52503) rectangle (5.42289,4.63088);
\draw [color=c, fill=c] (5.42289,4.52503) rectangle (5.46269,4.63088);
\draw [color=c, fill=c] (5.46269,4.52503) rectangle (5.50249,4.63088);
\draw [color=c, fill=c] (5.50249,4.52503) rectangle (5.54229,4.63088);
\draw [color=c, fill=c] (5.54229,4.52503) rectangle (5.58209,4.63088);
\draw [color=c, fill=c] (5.58209,4.52503) rectangle (5.62189,4.63088);
\draw [color=c, fill=c] (5.62189,4.52503) rectangle (5.66169,4.63088);
\draw [color=c, fill=c] (5.66169,4.52503) rectangle (5.70149,4.63088);
\draw [color=c, fill=c] (5.70149,4.52503) rectangle (5.74129,4.63088);
\draw [color=c, fill=c] (5.74129,4.52503) rectangle (5.78109,4.63088);
\draw [color=c, fill=c] (5.78109,4.52503) rectangle (5.8209,4.63088);
\draw [color=c, fill=c] (5.8209,4.52503) rectangle (5.8607,4.63088);
\definecolor{c}{rgb}{0.2,0,1};
\draw [color=c, fill=c] (5.8607,4.52503) rectangle (5.9005,4.63088);
\draw [color=c, fill=c] (5.9005,4.52503) rectangle (5.9403,4.63088);
\draw [color=c, fill=c] (5.9403,4.52503) rectangle (5.9801,4.63088);
\draw [color=c, fill=c] (5.9801,4.52503) rectangle (6.0199,4.63088);
\draw [color=c, fill=c] (6.0199,4.52503) rectangle (6.0597,4.63088);
\draw [color=c, fill=c] (6.0597,4.52503) rectangle (6.0995,4.63088);
\draw [color=c, fill=c] (6.0995,4.52503) rectangle (6.1393,4.63088);
\draw [color=c, fill=c] (6.1393,4.52503) rectangle (6.1791,4.63088);
\draw [color=c, fill=c] (6.1791,4.52503) rectangle (6.21891,4.63088);
\draw [color=c, fill=c] (6.21891,4.52503) rectangle (6.25871,4.63088);
\draw [color=c, fill=c] (6.25871,4.52503) rectangle (6.29851,4.63088);
\draw [color=c, fill=c] (6.29851,4.52503) rectangle (6.33831,4.63088);
\draw [color=c, fill=c] (6.33831,4.52503) rectangle (6.37811,4.63088);
\draw [color=c, fill=c] (6.37811,4.52503) rectangle (6.41791,4.63088);
\draw [color=c, fill=c] (6.41791,4.52503) rectangle (6.45771,4.63088);
\draw [color=c, fill=c] (6.45771,4.52503) rectangle (6.49751,4.63088);
\draw [color=c, fill=c] (6.49751,4.52503) rectangle (6.53731,4.63088);
\draw [color=c, fill=c] (6.53731,4.52503) rectangle (6.57711,4.63088);
\draw [color=c, fill=c] (6.57711,4.52503) rectangle (6.61692,4.63088);
\draw [color=c, fill=c] (6.61692,4.52503) rectangle (6.65672,4.63088);
\draw [color=c, fill=c] (6.65672,4.52503) rectangle (6.69652,4.63088);
\draw [color=c, fill=c] (6.69652,4.52503) rectangle (6.73632,4.63088);
\draw [color=c, fill=c] (6.73632,4.52503) rectangle (6.77612,4.63088);
\draw [color=c, fill=c] (6.77612,4.52503) rectangle (6.81592,4.63088);
\draw [color=c, fill=c] (6.81592,4.52503) rectangle (6.85572,4.63088);
\draw [color=c, fill=c] (6.85572,4.52503) rectangle (6.89552,4.63088);
\draw [color=c, fill=c] (6.89552,4.52503) rectangle (6.93532,4.63088);
\draw [color=c, fill=c] (6.93532,4.52503) rectangle (6.97512,4.63088);
\draw [color=c, fill=c] (6.97512,4.52503) rectangle (7.01493,4.63088);
\draw [color=c, fill=c] (7.01493,4.52503) rectangle (7.05473,4.63088);
\draw [color=c, fill=c] (7.05473,4.52503) rectangle (7.09453,4.63088);
\draw [color=c, fill=c] (7.09453,4.52503) rectangle (7.13433,4.63088);
\draw [color=c, fill=c] (7.13433,4.52503) rectangle (7.17413,4.63088);
\draw [color=c, fill=c] (7.17413,4.52503) rectangle (7.21393,4.63088);
\draw [color=c, fill=c] (7.21393,4.52503) rectangle (7.25373,4.63088);
\draw [color=c, fill=c] (7.25373,4.52503) rectangle (7.29353,4.63088);
\draw [color=c, fill=c] (7.29353,4.52503) rectangle (7.33333,4.63088);
\draw [color=c, fill=c] (7.33333,4.52503) rectangle (7.37313,4.63088);
\draw [color=c, fill=c] (7.37313,4.52503) rectangle (7.41294,4.63088);
\draw [color=c, fill=c] (7.41294,4.52503) rectangle (7.45274,4.63088);
\draw [color=c, fill=c] (7.45274,4.52503) rectangle (7.49254,4.63088);
\draw [color=c, fill=c] (7.49254,4.52503) rectangle (7.53234,4.63088);
\draw [color=c, fill=c] (7.53234,4.52503) rectangle (7.57214,4.63088);
\draw [color=c, fill=c] (7.57214,4.52503) rectangle (7.61194,4.63088);
\draw [color=c, fill=c] (7.61194,4.52503) rectangle (7.65174,4.63088);
\draw [color=c, fill=c] (7.65174,4.52503) rectangle (7.69154,4.63088);
\draw [color=c, fill=c] (7.69154,4.52503) rectangle (7.73134,4.63088);
\draw [color=c, fill=c] (7.73134,4.52503) rectangle (7.77114,4.63088);
\draw [color=c, fill=c] (7.77114,4.52503) rectangle (7.81095,4.63088);
\draw [color=c, fill=c] (7.81095,4.52503) rectangle (7.85075,4.63088);
\draw [color=c, fill=c] (7.85075,4.52503) rectangle (7.89055,4.63088);
\draw [color=c, fill=c] (7.89055,4.52503) rectangle (7.93035,4.63088);
\draw [color=c, fill=c] (7.93035,4.52503) rectangle (7.97015,4.63088);
\draw [color=c, fill=c] (7.97015,4.52503) rectangle (8.00995,4.63088);
\draw [color=c, fill=c] (8.00995,4.52503) rectangle (8.04975,4.63088);
\draw [color=c, fill=c] (8.04975,4.52503) rectangle (8.08955,4.63088);
\draw [color=c, fill=c] (8.08955,4.52503) rectangle (8.12935,4.63088);
\draw [color=c, fill=c] (8.12935,4.52503) rectangle (8.16915,4.63088);
\draw [color=c, fill=c] (8.16915,4.52503) rectangle (8.20895,4.63088);
\draw [color=c, fill=c] (8.20895,4.52503) rectangle (8.24876,4.63088);
\draw [color=c, fill=c] (8.24876,4.52503) rectangle (8.28856,4.63088);
\draw [color=c, fill=c] (8.28856,4.52503) rectangle (8.32836,4.63088);
\draw [color=c, fill=c] (8.32836,4.52503) rectangle (8.36816,4.63088);
\draw [color=c, fill=c] (8.36816,4.52503) rectangle (8.40796,4.63088);
\draw [color=c, fill=c] (8.40796,4.52503) rectangle (8.44776,4.63088);
\draw [color=c, fill=c] (8.44776,4.52503) rectangle (8.48756,4.63088);
\draw [color=c, fill=c] (8.48756,4.52503) rectangle (8.52736,4.63088);
\draw [color=c, fill=c] (8.52736,4.52503) rectangle (8.56716,4.63088);
\draw [color=c, fill=c] (8.56716,4.52503) rectangle (8.60697,4.63088);
\draw [color=c, fill=c] (8.60697,4.52503) rectangle (8.64677,4.63088);
\draw [color=c, fill=c] (8.64677,4.52503) rectangle (8.68657,4.63088);
\draw [color=c, fill=c] (8.68657,4.52503) rectangle (8.72637,4.63088);
\draw [color=c, fill=c] (8.72637,4.52503) rectangle (8.76617,4.63088);
\draw [color=c, fill=c] (8.76617,4.52503) rectangle (8.80597,4.63088);
\draw [color=c, fill=c] (8.80597,4.52503) rectangle (8.84577,4.63088);
\draw [color=c, fill=c] (8.84577,4.52503) rectangle (8.88557,4.63088);
\draw [color=c, fill=c] (8.88557,4.52503) rectangle (8.92537,4.63088);
\draw [color=c, fill=c] (8.92537,4.52503) rectangle (8.96517,4.63088);
\draw [color=c, fill=c] (8.96517,4.52503) rectangle (9.00498,4.63088);
\draw [color=c, fill=c] (9.00498,4.52503) rectangle (9.04478,4.63088);
\draw [color=c, fill=c] (9.04478,4.52503) rectangle (9.08458,4.63088);
\draw [color=c, fill=c] (9.08458,4.52503) rectangle (9.12438,4.63088);
\draw [color=c, fill=c] (9.12438,4.52503) rectangle (9.16418,4.63088);
\draw [color=c, fill=c] (9.16418,4.52503) rectangle (9.20398,4.63088);
\draw [color=c, fill=c] (9.20398,4.52503) rectangle (9.24378,4.63088);
\draw [color=c, fill=c] (9.24378,4.52503) rectangle (9.28358,4.63088);
\draw [color=c, fill=c] (9.28358,4.52503) rectangle (9.32338,4.63088);
\draw [color=c, fill=c] (9.32338,4.52503) rectangle (9.36318,4.63088);
\draw [color=c, fill=c] (9.36318,4.52503) rectangle (9.40298,4.63088);
\draw [color=c, fill=c] (9.40298,4.52503) rectangle (9.44279,4.63088);
\draw [color=c, fill=c] (9.44279,4.52503) rectangle (9.48259,4.63088);
\draw [color=c, fill=c] (9.48259,4.52503) rectangle (9.52239,4.63088);
\draw [color=c, fill=c] (9.52239,4.52503) rectangle (9.56219,4.63088);
\draw [color=c, fill=c] (9.56219,4.52503) rectangle (9.60199,4.63088);
\draw [color=c, fill=c] (9.60199,4.52503) rectangle (9.64179,4.63088);
\draw [color=c, fill=c] (9.64179,4.52503) rectangle (9.68159,4.63088);
\draw [color=c, fill=c] (9.68159,4.52503) rectangle (9.72139,4.63088);
\definecolor{c}{rgb}{0,0.0800001,1};
\draw [color=c, fill=c] (9.72139,4.52503) rectangle (9.76119,4.63088);
\draw [color=c, fill=c] (9.76119,4.52503) rectangle (9.80099,4.63088);
\draw [color=c, fill=c] (9.80099,4.52503) rectangle (9.8408,4.63088);
\draw [color=c, fill=c] (9.8408,4.52503) rectangle (9.8806,4.63088);
\definecolor{c}{rgb}{0,0.266667,1};
\draw [color=c, fill=c] (9.8806,4.52503) rectangle (9.9204,4.63088);
\draw [color=c, fill=c] (9.9204,4.52503) rectangle (9.9602,4.63088);
\draw [color=c, fill=c] (9.9602,4.52503) rectangle (10,4.63088);
\definecolor{c}{rgb}{0,0.546666,1};
\draw [color=c, fill=c] (10,4.52503) rectangle (10.0398,4.63088);
\draw [color=c, fill=c] (10.0398,4.52503) rectangle (10.0796,4.63088);
\draw [color=c, fill=c] (10.0796,4.52503) rectangle (10.1194,4.63088);
\definecolor{c}{rgb}{0,0.733333,1};
\draw [color=c, fill=c] (10.1194,4.52503) rectangle (10.1592,4.63088);
\draw [color=c, fill=c] (10.1592,4.52503) rectangle (10.199,4.63088);
\draw [color=c, fill=c] (10.199,4.52503) rectangle (10.2388,4.63088);
\draw [color=c, fill=c] (10.2388,4.52503) rectangle (10.2786,4.63088);
\draw [color=c, fill=c] (10.2786,4.52503) rectangle (10.3184,4.63088);
\draw [color=c, fill=c] (10.3184,4.52503) rectangle (10.3582,4.63088);
\draw [color=c, fill=c] (10.3582,4.52503) rectangle (10.398,4.63088);
\draw [color=c, fill=c] (10.398,4.52503) rectangle (10.4378,4.63088);
\definecolor{c}{rgb}{0,1,0.986667};
\draw [color=c, fill=c] (10.4378,4.52503) rectangle (10.4776,4.63088);
\draw [color=c, fill=c] (10.4776,4.52503) rectangle (10.5174,4.63088);
\draw [color=c, fill=c] (10.5174,4.52503) rectangle (10.5572,4.63088);
\draw [color=c, fill=c] (10.5572,4.52503) rectangle (10.597,4.63088);
\draw [color=c, fill=c] (10.597,4.52503) rectangle (10.6368,4.63088);
\draw [color=c, fill=c] (10.6368,4.52503) rectangle (10.6766,4.63088);
\draw [color=c, fill=c] (10.6766,4.52503) rectangle (10.7164,4.63088);
\draw [color=c, fill=c] (10.7164,4.52503) rectangle (10.7562,4.63088);
\draw [color=c, fill=c] (10.7562,4.52503) rectangle (10.796,4.63088);
\draw [color=c, fill=c] (10.796,4.52503) rectangle (10.8358,4.63088);
\draw [color=c, fill=c] (10.8358,4.52503) rectangle (10.8756,4.63088);
\draw [color=c, fill=c] (10.8756,4.52503) rectangle (10.9154,4.63088);
\draw [color=c, fill=c] (10.9154,4.52503) rectangle (10.9552,4.63088);
\draw [color=c, fill=c] (10.9552,4.52503) rectangle (10.995,4.63088);
\draw [color=c, fill=c] (10.995,4.52503) rectangle (11.0348,4.63088);
\draw [color=c, fill=c] (11.0348,4.52503) rectangle (11.0746,4.63088);
\draw [color=c, fill=c] (11.0746,4.52503) rectangle (11.1144,4.63088);
\draw [color=c, fill=c] (11.1144,4.52503) rectangle (11.1542,4.63088);
\draw [color=c, fill=c] (11.1542,4.52503) rectangle (11.194,4.63088);
\draw [color=c, fill=c] (11.194,4.52503) rectangle (11.2338,4.63088);
\draw [color=c, fill=c] (11.2338,4.52503) rectangle (11.2736,4.63088);
\draw [color=c, fill=c] (11.2736,4.52503) rectangle (11.3134,4.63088);
\draw [color=c, fill=c] (11.3134,4.52503) rectangle (11.3532,4.63088);
\draw [color=c, fill=c] (11.3532,4.52503) rectangle (11.393,4.63088);
\draw [color=c, fill=c] (11.393,4.52503) rectangle (11.4328,4.63088);
\definecolor{c}{rgb}{0,0.733333,1};
\draw [color=c, fill=c] (11.4328,4.52503) rectangle (11.4726,4.63088);
\draw [color=c, fill=c] (11.4726,4.52503) rectangle (11.5124,4.63088);
\draw [color=c, fill=c] (11.5124,4.52503) rectangle (11.5522,4.63088);
\draw [color=c, fill=c] (11.5522,4.52503) rectangle (11.592,4.63088);
\draw [color=c, fill=c] (11.592,4.52503) rectangle (11.6318,4.63088);
\draw [color=c, fill=c] (11.6318,4.52503) rectangle (11.6716,4.63088);
\draw [color=c, fill=c] (11.6716,4.52503) rectangle (11.7114,4.63088);
\draw [color=c, fill=c] (11.7114,4.52503) rectangle (11.7512,4.63088);
\draw [color=c, fill=c] (11.7512,4.52503) rectangle (11.791,4.63088);
\draw [color=c, fill=c] (11.791,4.52503) rectangle (11.8308,4.63088);
\draw [color=c, fill=c] (11.8308,4.52503) rectangle (11.8706,4.63088);
\draw [color=c, fill=c] (11.8706,4.52503) rectangle (11.9104,4.63088);
\draw [color=c, fill=c] (11.9104,4.52503) rectangle (11.9502,4.63088);
\draw [color=c, fill=c] (11.9502,4.52503) rectangle (11.99,4.63088);
\draw [color=c, fill=c] (11.99,4.52503) rectangle (12.0299,4.63088);
\draw [color=c, fill=c] (12.0299,4.52503) rectangle (12.0697,4.63088);
\draw [color=c, fill=c] (12.0697,4.52503) rectangle (12.1095,4.63088);
\draw [color=c, fill=c] (12.1095,4.52503) rectangle (12.1493,4.63088);
\draw [color=c, fill=c] (12.1493,4.52503) rectangle (12.1891,4.63088);
\draw [color=c, fill=c] (12.1891,4.52503) rectangle (12.2289,4.63088);
\draw [color=c, fill=c] (12.2289,4.52503) rectangle (12.2687,4.63088);
\draw [color=c, fill=c] (12.2687,4.52503) rectangle (12.3085,4.63088);
\draw [color=c, fill=c] (12.3085,4.52503) rectangle (12.3483,4.63088);
\draw [color=c, fill=c] (12.3483,4.52503) rectangle (12.3881,4.63088);
\draw [color=c, fill=c] (12.3881,4.52503) rectangle (12.4279,4.63088);
\draw [color=c, fill=c] (12.4279,4.52503) rectangle (12.4677,4.63088);
\draw [color=c, fill=c] (12.4677,4.52503) rectangle (12.5075,4.63088);
\draw [color=c, fill=c] (12.5075,4.52503) rectangle (12.5473,4.63088);
\draw [color=c, fill=c] (12.5473,4.52503) rectangle (12.5871,4.63088);
\draw [color=c, fill=c] (12.5871,4.52503) rectangle (12.6269,4.63088);
\draw [color=c, fill=c] (12.6269,4.52503) rectangle (12.6667,4.63088);
\draw [color=c, fill=c] (12.6667,4.52503) rectangle (12.7065,4.63088);
\draw [color=c, fill=c] (12.7065,4.52503) rectangle (12.7463,4.63088);
\draw [color=c, fill=c] (12.7463,4.52503) rectangle (12.7861,4.63088);
\draw [color=c, fill=c] (12.7861,4.52503) rectangle (12.8259,4.63088);
\draw [color=c, fill=c] (12.8259,4.52503) rectangle (12.8657,4.63088);
\draw [color=c, fill=c] (12.8657,4.52503) rectangle (12.9055,4.63088);
\draw [color=c, fill=c] (12.9055,4.52503) rectangle (12.9453,4.63088);
\draw [color=c, fill=c] (12.9453,4.52503) rectangle (12.9851,4.63088);
\draw [color=c, fill=c] (12.9851,4.52503) rectangle (13.0249,4.63088);
\draw [color=c, fill=c] (13.0249,4.52503) rectangle (13.0647,4.63088);
\draw [color=c, fill=c] (13.0647,4.52503) rectangle (13.1045,4.63088);
\draw [color=c, fill=c] (13.1045,4.52503) rectangle (13.1443,4.63088);
\draw [color=c, fill=c] (13.1443,4.52503) rectangle (13.1841,4.63088);
\draw [color=c, fill=c] (13.1841,4.52503) rectangle (13.2239,4.63088);
\draw [color=c, fill=c] (13.2239,4.52503) rectangle (13.2637,4.63088);
\draw [color=c, fill=c] (13.2637,4.52503) rectangle (13.3035,4.63088);
\draw [color=c, fill=c] (13.3035,4.52503) rectangle (13.3433,4.63088);
\draw [color=c, fill=c] (13.3433,4.52503) rectangle (13.3831,4.63088);
\draw [color=c, fill=c] (13.3831,4.52503) rectangle (13.4229,4.63088);
\draw [color=c, fill=c] (13.4229,4.52503) rectangle (13.4627,4.63088);
\draw [color=c, fill=c] (13.4627,4.52503) rectangle (13.5025,4.63088);
\draw [color=c, fill=c] (13.5025,4.52503) rectangle (13.5423,4.63088);
\draw [color=c, fill=c] (13.5423,4.52503) rectangle (13.5821,4.63088);
\draw [color=c, fill=c] (13.5821,4.52503) rectangle (13.6219,4.63088);
\draw [color=c, fill=c] (13.6219,4.52503) rectangle (13.6617,4.63088);
\draw [color=c, fill=c] (13.6617,4.52503) rectangle (13.7015,4.63088);
\draw [color=c, fill=c] (13.7015,4.52503) rectangle (13.7413,4.63088);
\draw [color=c, fill=c] (13.7413,4.52503) rectangle (13.7811,4.63088);
\draw [color=c, fill=c] (13.7811,4.52503) rectangle (13.8209,4.63088);
\draw [color=c, fill=c] (13.8209,4.52503) rectangle (13.8607,4.63088);
\draw [color=c, fill=c] (13.8607,4.52503) rectangle (13.9005,4.63088);
\draw [color=c, fill=c] (13.9005,4.52503) rectangle (13.9403,4.63088);
\draw [color=c, fill=c] (13.9403,4.52503) rectangle (13.9801,4.63088);
\draw [color=c, fill=c] (13.9801,4.52503) rectangle (14.0199,4.63088);
\draw [color=c, fill=c] (14.0199,4.52503) rectangle (14.0597,4.63088);
\draw [color=c, fill=c] (14.0597,4.52503) rectangle (14.0995,4.63088);
\draw [color=c, fill=c] (14.0995,4.52503) rectangle (14.1393,4.63088);
\draw [color=c, fill=c] (14.1393,4.52503) rectangle (14.1791,4.63088);
\draw [color=c, fill=c] (14.1791,4.52503) rectangle (14.2189,4.63088);
\draw [color=c, fill=c] (14.2189,4.52503) rectangle (14.2587,4.63088);
\draw [color=c, fill=c] (14.2587,4.52503) rectangle (14.2985,4.63088);
\draw [color=c, fill=c] (14.2985,4.52503) rectangle (14.3383,4.63088);
\draw [color=c, fill=c] (14.3383,4.52503) rectangle (14.3781,4.63088);
\draw [color=c, fill=c] (14.3781,4.52503) rectangle (14.4179,4.63088);
\draw [color=c, fill=c] (14.4179,4.52503) rectangle (14.4577,4.63088);
\draw [color=c, fill=c] (14.4577,4.52503) rectangle (14.4975,4.63088);
\draw [color=c, fill=c] (14.4975,4.52503) rectangle (14.5373,4.63088);
\draw [color=c, fill=c] (14.5373,4.52503) rectangle (14.5771,4.63088);
\draw [color=c, fill=c] (14.5771,4.52503) rectangle (14.6169,4.63088);
\draw [color=c, fill=c] (14.6169,4.52503) rectangle (14.6567,4.63088);
\draw [color=c, fill=c] (14.6567,4.52503) rectangle (14.6965,4.63088);
\draw [color=c, fill=c] (14.6965,4.52503) rectangle (14.7363,4.63088);
\draw [color=c, fill=c] (14.7363,4.52503) rectangle (14.7761,4.63088);
\draw [color=c, fill=c] (14.7761,4.52503) rectangle (14.8159,4.63088);
\draw [color=c, fill=c] (14.8159,4.52503) rectangle (14.8557,4.63088);
\draw [color=c, fill=c] (14.8557,4.52503) rectangle (14.8955,4.63088);
\draw [color=c, fill=c] (14.8955,4.52503) rectangle (14.9353,4.63088);
\draw [color=c, fill=c] (14.9353,4.52503) rectangle (14.9751,4.63088);
\draw [color=c, fill=c] (14.9751,4.52503) rectangle (15.0149,4.63088);
\draw [color=c, fill=c] (15.0149,4.52503) rectangle (15.0547,4.63088);
\draw [color=c, fill=c] (15.0547,4.52503) rectangle (15.0945,4.63088);
\draw [color=c, fill=c] (15.0945,4.52503) rectangle (15.1343,4.63088);
\draw [color=c, fill=c] (15.1343,4.52503) rectangle (15.1741,4.63088);
\draw [color=c, fill=c] (15.1741,4.52503) rectangle (15.2139,4.63088);
\draw [color=c, fill=c] (15.2139,4.52503) rectangle (15.2537,4.63088);
\draw [color=c, fill=c] (15.2537,4.52503) rectangle (15.2935,4.63088);
\draw [color=c, fill=c] (15.2935,4.52503) rectangle (15.3333,4.63088);
\draw [color=c, fill=c] (15.3333,4.52503) rectangle (15.3731,4.63088);
\draw [color=c, fill=c] (15.3731,4.52503) rectangle (15.4129,4.63088);
\draw [color=c, fill=c] (15.4129,4.52503) rectangle (15.4527,4.63088);
\draw [color=c, fill=c] (15.4527,4.52503) rectangle (15.4925,4.63088);
\draw [color=c, fill=c] (15.4925,4.52503) rectangle (15.5323,4.63088);
\draw [color=c, fill=c] (15.5323,4.52503) rectangle (15.5721,4.63088);
\draw [color=c, fill=c] (15.5721,4.52503) rectangle (15.6119,4.63088);
\draw [color=c, fill=c] (15.6119,4.52503) rectangle (15.6517,4.63088);
\draw [color=c, fill=c] (15.6517,4.52503) rectangle (15.6915,4.63088);
\draw [color=c, fill=c] (15.6915,4.52503) rectangle (15.7313,4.63088);
\draw [color=c, fill=c] (15.7313,4.52503) rectangle (15.7711,4.63088);
\draw [color=c, fill=c] (15.7711,4.52503) rectangle (15.8109,4.63088);
\draw [color=c, fill=c] (15.8109,4.52503) rectangle (15.8507,4.63088);
\draw [color=c, fill=c] (15.8507,4.52503) rectangle (15.8905,4.63088);
\draw [color=c, fill=c] (15.8905,4.52503) rectangle (15.9303,4.63088);
\draw [color=c, fill=c] (15.9303,4.52503) rectangle (15.9701,4.63088);
\draw [color=c, fill=c] (15.9701,4.52503) rectangle (16.01,4.63088);
\draw [color=c, fill=c] (16.01,4.52503) rectangle (16.0498,4.63088);
\draw [color=c, fill=c] (16.0498,4.52503) rectangle (16.0896,4.63088);
\draw [color=c, fill=c] (16.0896,4.52503) rectangle (16.1294,4.63088);
\draw [color=c, fill=c] (16.1294,4.52503) rectangle (16.1692,4.63088);
\draw [color=c, fill=c] (16.1692,4.52503) rectangle (16.209,4.63088);
\draw [color=c, fill=c] (16.209,4.52503) rectangle (16.2488,4.63088);
\draw [color=c, fill=c] (16.2488,4.52503) rectangle (16.2886,4.63088);
\draw [color=c, fill=c] (16.2886,4.52503) rectangle (16.3284,4.63088);
\draw [color=c, fill=c] (16.3284,4.52503) rectangle (16.3682,4.63088);
\draw [color=c, fill=c] (16.3682,4.52503) rectangle (16.408,4.63088);
\draw [color=c, fill=c] (16.408,4.52503) rectangle (16.4478,4.63088);
\draw [color=c, fill=c] (16.4478,4.52503) rectangle (16.4876,4.63088);
\draw [color=c, fill=c] (16.4876,4.52503) rectangle (16.5274,4.63088);
\draw [color=c, fill=c] (16.5274,4.52503) rectangle (16.5672,4.63088);
\draw [color=c, fill=c] (16.5672,4.52503) rectangle (16.607,4.63088);
\draw [color=c, fill=c] (16.607,4.52503) rectangle (16.6468,4.63088);
\draw [color=c, fill=c] (16.6468,4.52503) rectangle (16.6866,4.63088);
\draw [color=c, fill=c] (16.6866,4.52503) rectangle (16.7264,4.63088);
\draw [color=c, fill=c] (16.7264,4.52503) rectangle (16.7662,4.63088);
\draw [color=c, fill=c] (16.7662,4.52503) rectangle (16.806,4.63088);
\draw [color=c, fill=c] (16.806,4.52503) rectangle (16.8458,4.63088);
\draw [color=c, fill=c] (16.8458,4.52503) rectangle (16.8856,4.63088);
\draw [color=c, fill=c] (16.8856,4.52503) rectangle (16.9254,4.63088);
\draw [color=c, fill=c] (16.9254,4.52503) rectangle (16.9652,4.63088);
\draw [color=c, fill=c] (16.9652,4.52503) rectangle (17.005,4.63088);
\draw [color=c, fill=c] (17.005,4.52503) rectangle (17.0448,4.63088);
\draw [color=c, fill=c] (17.0448,4.52503) rectangle (17.0846,4.63088);
\draw [color=c, fill=c] (17.0846,4.52503) rectangle (17.1244,4.63088);
\draw [color=c, fill=c] (17.1244,4.52503) rectangle (17.1642,4.63088);
\draw [color=c, fill=c] (17.1642,4.52503) rectangle (17.204,4.63088);
\draw [color=c, fill=c] (17.204,4.52503) rectangle (17.2438,4.63088);
\draw [color=c, fill=c] (17.2438,4.52503) rectangle (17.2836,4.63088);
\draw [color=c, fill=c] (17.2836,4.52503) rectangle (17.3234,4.63088);
\draw [color=c, fill=c] (17.3234,4.52503) rectangle (17.3632,4.63088);
\draw [color=c, fill=c] (17.3632,4.52503) rectangle (17.403,4.63088);
\draw [color=c, fill=c] (17.403,4.52503) rectangle (17.4428,4.63088);
\draw [color=c, fill=c] (17.4428,4.52503) rectangle (17.4826,4.63088);
\draw [color=c, fill=c] (17.4826,4.52503) rectangle (17.5224,4.63088);
\draw [color=c, fill=c] (17.5224,4.52503) rectangle (17.5622,4.63088);
\draw [color=c, fill=c] (17.5622,4.52503) rectangle (17.602,4.63088);
\draw [color=c, fill=c] (17.602,4.52503) rectangle (17.6418,4.63088);
\draw [color=c, fill=c] (17.6418,4.52503) rectangle (17.6816,4.63088);
\draw [color=c, fill=c] (17.6816,4.52503) rectangle (17.7214,4.63088);
\draw [color=c, fill=c] (17.7214,4.52503) rectangle (17.7612,4.63088);
\draw [color=c, fill=c] (17.7612,4.52503) rectangle (17.801,4.63088);
\draw [color=c, fill=c] (17.801,4.52503) rectangle (17.8408,4.63088);
\draw [color=c, fill=c] (17.8408,4.52503) rectangle (17.8806,4.63088);
\draw [color=c, fill=c] (17.8806,4.52503) rectangle (17.9204,4.63088);
\draw [color=c, fill=c] (17.9204,4.52503) rectangle (17.9602,4.63088);
\draw [color=c, fill=c] (17.9602,4.52503) rectangle (18,4.63088);
\definecolor{c}{rgb}{0,0.0800001,1};
\draw [color=c, fill=c] (2,4.63088) rectangle (2.0398,4.73673);
\draw [color=c, fill=c] (2.0398,4.63088) rectangle (2.0796,4.73673);
\draw [color=c, fill=c] (2.0796,4.63088) rectangle (2.1194,4.73673);
\draw [color=c, fill=c] (2.1194,4.63088) rectangle (2.1592,4.73673);
\draw [color=c, fill=c] (2.1592,4.63088) rectangle (2.19901,4.73673);
\draw [color=c, fill=c] (2.19901,4.63088) rectangle (2.23881,4.73673);
\draw [color=c, fill=c] (2.23881,4.63088) rectangle (2.27861,4.73673);
\draw [color=c, fill=c] (2.27861,4.63088) rectangle (2.31841,4.73673);
\draw [color=c, fill=c] (2.31841,4.63088) rectangle (2.35821,4.73673);
\draw [color=c, fill=c] (2.35821,4.63088) rectangle (2.39801,4.73673);
\draw [color=c, fill=c] (2.39801,4.63088) rectangle (2.43781,4.73673);
\draw [color=c, fill=c] (2.43781,4.63088) rectangle (2.47761,4.73673);
\draw [color=c, fill=c] (2.47761,4.63088) rectangle (2.51741,4.73673);
\draw [color=c, fill=c] (2.51741,4.63088) rectangle (2.55721,4.73673);
\draw [color=c, fill=c] (2.55721,4.63088) rectangle (2.59702,4.73673);
\draw [color=c, fill=c] (2.59702,4.63088) rectangle (2.63682,4.73673);
\draw [color=c, fill=c] (2.63682,4.63088) rectangle (2.67662,4.73673);
\draw [color=c, fill=c] (2.67662,4.63088) rectangle (2.71642,4.73673);
\draw [color=c, fill=c] (2.71642,4.63088) rectangle (2.75622,4.73673);
\draw [color=c, fill=c] (2.75622,4.63088) rectangle (2.79602,4.73673);
\draw [color=c, fill=c] (2.79602,4.63088) rectangle (2.83582,4.73673);
\draw [color=c, fill=c] (2.83582,4.63088) rectangle (2.87562,4.73673);
\draw [color=c, fill=c] (2.87562,4.63088) rectangle (2.91542,4.73673);
\draw [color=c, fill=c] (2.91542,4.63088) rectangle (2.95522,4.73673);
\draw [color=c, fill=c] (2.95522,4.63088) rectangle (2.99502,4.73673);
\draw [color=c, fill=c] (2.99502,4.63088) rectangle (3.03483,4.73673);
\draw [color=c, fill=c] (3.03483,4.63088) rectangle (3.07463,4.73673);
\draw [color=c, fill=c] (3.07463,4.63088) rectangle (3.11443,4.73673);
\draw [color=c, fill=c] (3.11443,4.63088) rectangle (3.15423,4.73673);
\draw [color=c, fill=c] (3.15423,4.63088) rectangle (3.19403,4.73673);
\draw [color=c, fill=c] (3.19403,4.63088) rectangle (3.23383,4.73673);
\draw [color=c, fill=c] (3.23383,4.63088) rectangle (3.27363,4.73673);
\draw [color=c, fill=c] (3.27363,4.63088) rectangle (3.31343,4.73673);
\draw [color=c, fill=c] (3.31343,4.63088) rectangle (3.35323,4.73673);
\draw [color=c, fill=c] (3.35323,4.63088) rectangle (3.39303,4.73673);
\draw [color=c, fill=c] (3.39303,4.63088) rectangle (3.43284,4.73673);
\draw [color=c, fill=c] (3.43284,4.63088) rectangle (3.47264,4.73673);
\draw [color=c, fill=c] (3.47264,4.63088) rectangle (3.51244,4.73673);
\draw [color=c, fill=c] (3.51244,4.63088) rectangle (3.55224,4.73673);
\draw [color=c, fill=c] (3.55224,4.63088) rectangle (3.59204,4.73673);
\draw [color=c, fill=c] (3.59204,4.63088) rectangle (3.63184,4.73673);
\draw [color=c, fill=c] (3.63184,4.63088) rectangle (3.67164,4.73673);
\draw [color=c, fill=c] (3.67164,4.63088) rectangle (3.71144,4.73673);
\draw [color=c, fill=c] (3.71144,4.63088) rectangle (3.75124,4.73673);
\draw [color=c, fill=c] (3.75124,4.63088) rectangle (3.79104,4.73673);
\draw [color=c, fill=c] (3.79104,4.63088) rectangle (3.83085,4.73673);
\draw [color=c, fill=c] (3.83085,4.63088) rectangle (3.87065,4.73673);
\draw [color=c, fill=c] (3.87065,4.63088) rectangle (3.91045,4.73673);
\draw [color=c, fill=c] (3.91045,4.63088) rectangle (3.95025,4.73673);
\draw [color=c, fill=c] (3.95025,4.63088) rectangle (3.99005,4.73673);
\draw [color=c, fill=c] (3.99005,4.63088) rectangle (4.02985,4.73673);
\draw [color=c, fill=c] (4.02985,4.63088) rectangle (4.06965,4.73673);
\draw [color=c, fill=c] (4.06965,4.63088) rectangle (4.10945,4.73673);
\draw [color=c, fill=c] (4.10945,4.63088) rectangle (4.14925,4.73673);
\draw [color=c, fill=c] (4.14925,4.63088) rectangle (4.18905,4.73673);
\draw [color=c, fill=c] (4.18905,4.63088) rectangle (4.22886,4.73673);
\draw [color=c, fill=c] (4.22886,4.63088) rectangle (4.26866,4.73673);
\draw [color=c, fill=c] (4.26866,4.63088) rectangle (4.30846,4.73673);
\draw [color=c, fill=c] (4.30846,4.63088) rectangle (4.34826,4.73673);
\draw [color=c, fill=c] (4.34826,4.63088) rectangle (4.38806,4.73673);
\draw [color=c, fill=c] (4.38806,4.63088) rectangle (4.42786,4.73673);
\draw [color=c, fill=c] (4.42786,4.63088) rectangle (4.46766,4.73673);
\draw [color=c, fill=c] (4.46766,4.63088) rectangle (4.50746,4.73673);
\draw [color=c, fill=c] (4.50746,4.63088) rectangle (4.54726,4.73673);
\draw [color=c, fill=c] (4.54726,4.63088) rectangle (4.58706,4.73673);
\draw [color=c, fill=c] (4.58706,4.63088) rectangle (4.62687,4.73673);
\draw [color=c, fill=c] (4.62687,4.63088) rectangle (4.66667,4.73673);
\draw [color=c, fill=c] (4.66667,4.63088) rectangle (4.70647,4.73673);
\draw [color=c, fill=c] (4.70647,4.63088) rectangle (4.74627,4.73673);
\draw [color=c, fill=c] (4.74627,4.63088) rectangle (4.78607,4.73673);
\draw [color=c, fill=c] (4.78607,4.63088) rectangle (4.82587,4.73673);
\draw [color=c, fill=c] (4.82587,4.63088) rectangle (4.86567,4.73673);
\draw [color=c, fill=c] (4.86567,4.63088) rectangle (4.90547,4.73673);
\draw [color=c, fill=c] (4.90547,4.63088) rectangle (4.94527,4.73673);
\draw [color=c, fill=c] (4.94527,4.63088) rectangle (4.98507,4.73673);
\draw [color=c, fill=c] (4.98507,4.63088) rectangle (5.02488,4.73673);
\draw [color=c, fill=c] (5.02488,4.63088) rectangle (5.06468,4.73673);
\draw [color=c, fill=c] (5.06468,4.63088) rectangle (5.10448,4.73673);
\draw [color=c, fill=c] (5.10448,4.63088) rectangle (5.14428,4.73673);
\draw [color=c, fill=c] (5.14428,4.63088) rectangle (5.18408,4.73673);
\draw [color=c, fill=c] (5.18408,4.63088) rectangle (5.22388,4.73673);
\draw [color=c, fill=c] (5.22388,4.63088) rectangle (5.26368,4.73673);
\draw [color=c, fill=c] (5.26368,4.63088) rectangle (5.30348,4.73673);
\draw [color=c, fill=c] (5.30348,4.63088) rectangle (5.34328,4.73673);
\draw [color=c, fill=c] (5.34328,4.63088) rectangle (5.38308,4.73673);
\draw [color=c, fill=c] (5.38308,4.63088) rectangle (5.42289,4.73673);
\draw [color=c, fill=c] (5.42289,4.63088) rectangle (5.46269,4.73673);
\draw [color=c, fill=c] (5.46269,4.63088) rectangle (5.50249,4.73673);
\draw [color=c, fill=c] (5.50249,4.63088) rectangle (5.54229,4.73673);
\draw [color=c, fill=c] (5.54229,4.63088) rectangle (5.58209,4.73673);
\draw [color=c, fill=c] (5.58209,4.63088) rectangle (5.62189,4.73673);
\draw [color=c, fill=c] (5.62189,4.63088) rectangle (5.66169,4.73673);
\draw [color=c, fill=c] (5.66169,4.63088) rectangle (5.70149,4.73673);
\draw [color=c, fill=c] (5.70149,4.63088) rectangle (5.74129,4.73673);
\draw [color=c, fill=c] (5.74129,4.63088) rectangle (5.78109,4.73673);
\draw [color=c, fill=c] (5.78109,4.63088) rectangle (5.8209,4.73673);
\draw [color=c, fill=c] (5.8209,4.63088) rectangle (5.8607,4.73673);
\definecolor{c}{rgb}{0.2,0,1};
\draw [color=c, fill=c] (5.8607,4.63088) rectangle (5.9005,4.73673);
\draw [color=c, fill=c] (5.9005,4.63088) rectangle (5.9403,4.73673);
\draw [color=c, fill=c] (5.9403,4.63088) rectangle (5.9801,4.73673);
\draw [color=c, fill=c] (5.9801,4.63088) rectangle (6.0199,4.73673);
\draw [color=c, fill=c] (6.0199,4.63088) rectangle (6.0597,4.73673);
\draw [color=c, fill=c] (6.0597,4.63088) rectangle (6.0995,4.73673);
\draw [color=c, fill=c] (6.0995,4.63088) rectangle (6.1393,4.73673);
\draw [color=c, fill=c] (6.1393,4.63088) rectangle (6.1791,4.73673);
\draw [color=c, fill=c] (6.1791,4.63088) rectangle (6.21891,4.73673);
\draw [color=c, fill=c] (6.21891,4.63088) rectangle (6.25871,4.73673);
\draw [color=c, fill=c] (6.25871,4.63088) rectangle (6.29851,4.73673);
\draw [color=c, fill=c] (6.29851,4.63088) rectangle (6.33831,4.73673);
\draw [color=c, fill=c] (6.33831,4.63088) rectangle (6.37811,4.73673);
\draw [color=c, fill=c] (6.37811,4.63088) rectangle (6.41791,4.73673);
\draw [color=c, fill=c] (6.41791,4.63088) rectangle (6.45771,4.73673);
\draw [color=c, fill=c] (6.45771,4.63088) rectangle (6.49751,4.73673);
\draw [color=c, fill=c] (6.49751,4.63088) rectangle (6.53731,4.73673);
\draw [color=c, fill=c] (6.53731,4.63088) rectangle (6.57711,4.73673);
\draw [color=c, fill=c] (6.57711,4.63088) rectangle (6.61692,4.73673);
\draw [color=c, fill=c] (6.61692,4.63088) rectangle (6.65672,4.73673);
\draw [color=c, fill=c] (6.65672,4.63088) rectangle (6.69652,4.73673);
\draw [color=c, fill=c] (6.69652,4.63088) rectangle (6.73632,4.73673);
\draw [color=c, fill=c] (6.73632,4.63088) rectangle (6.77612,4.73673);
\draw [color=c, fill=c] (6.77612,4.63088) rectangle (6.81592,4.73673);
\draw [color=c, fill=c] (6.81592,4.63088) rectangle (6.85572,4.73673);
\draw [color=c, fill=c] (6.85572,4.63088) rectangle (6.89552,4.73673);
\draw [color=c, fill=c] (6.89552,4.63088) rectangle (6.93532,4.73673);
\draw [color=c, fill=c] (6.93532,4.63088) rectangle (6.97512,4.73673);
\draw [color=c, fill=c] (6.97512,4.63088) rectangle (7.01493,4.73673);
\draw [color=c, fill=c] (7.01493,4.63088) rectangle (7.05473,4.73673);
\draw [color=c, fill=c] (7.05473,4.63088) rectangle (7.09453,4.73673);
\draw [color=c, fill=c] (7.09453,4.63088) rectangle (7.13433,4.73673);
\draw [color=c, fill=c] (7.13433,4.63088) rectangle (7.17413,4.73673);
\draw [color=c, fill=c] (7.17413,4.63088) rectangle (7.21393,4.73673);
\draw [color=c, fill=c] (7.21393,4.63088) rectangle (7.25373,4.73673);
\draw [color=c, fill=c] (7.25373,4.63088) rectangle (7.29353,4.73673);
\draw [color=c, fill=c] (7.29353,4.63088) rectangle (7.33333,4.73673);
\draw [color=c, fill=c] (7.33333,4.63088) rectangle (7.37313,4.73673);
\draw [color=c, fill=c] (7.37313,4.63088) rectangle (7.41294,4.73673);
\draw [color=c, fill=c] (7.41294,4.63088) rectangle (7.45274,4.73673);
\draw [color=c, fill=c] (7.45274,4.63088) rectangle (7.49254,4.73673);
\draw [color=c, fill=c] (7.49254,4.63088) rectangle (7.53234,4.73673);
\draw [color=c, fill=c] (7.53234,4.63088) rectangle (7.57214,4.73673);
\draw [color=c, fill=c] (7.57214,4.63088) rectangle (7.61194,4.73673);
\draw [color=c, fill=c] (7.61194,4.63088) rectangle (7.65174,4.73673);
\draw [color=c, fill=c] (7.65174,4.63088) rectangle (7.69154,4.73673);
\draw [color=c, fill=c] (7.69154,4.63088) rectangle (7.73134,4.73673);
\draw [color=c, fill=c] (7.73134,4.63088) rectangle (7.77114,4.73673);
\draw [color=c, fill=c] (7.77114,4.63088) rectangle (7.81095,4.73673);
\draw [color=c, fill=c] (7.81095,4.63088) rectangle (7.85075,4.73673);
\draw [color=c, fill=c] (7.85075,4.63088) rectangle (7.89055,4.73673);
\draw [color=c, fill=c] (7.89055,4.63088) rectangle (7.93035,4.73673);
\draw [color=c, fill=c] (7.93035,4.63088) rectangle (7.97015,4.73673);
\draw [color=c, fill=c] (7.97015,4.63088) rectangle (8.00995,4.73673);
\draw [color=c, fill=c] (8.00995,4.63088) rectangle (8.04975,4.73673);
\draw [color=c, fill=c] (8.04975,4.63088) rectangle (8.08955,4.73673);
\draw [color=c, fill=c] (8.08955,4.63088) rectangle (8.12935,4.73673);
\draw [color=c, fill=c] (8.12935,4.63088) rectangle (8.16915,4.73673);
\draw [color=c, fill=c] (8.16915,4.63088) rectangle (8.20895,4.73673);
\draw [color=c, fill=c] (8.20895,4.63088) rectangle (8.24876,4.73673);
\draw [color=c, fill=c] (8.24876,4.63088) rectangle (8.28856,4.73673);
\draw [color=c, fill=c] (8.28856,4.63088) rectangle (8.32836,4.73673);
\draw [color=c, fill=c] (8.32836,4.63088) rectangle (8.36816,4.73673);
\draw [color=c, fill=c] (8.36816,4.63088) rectangle (8.40796,4.73673);
\draw [color=c, fill=c] (8.40796,4.63088) rectangle (8.44776,4.73673);
\draw [color=c, fill=c] (8.44776,4.63088) rectangle (8.48756,4.73673);
\draw [color=c, fill=c] (8.48756,4.63088) rectangle (8.52736,4.73673);
\draw [color=c, fill=c] (8.52736,4.63088) rectangle (8.56716,4.73673);
\draw [color=c, fill=c] (8.56716,4.63088) rectangle (8.60697,4.73673);
\draw [color=c, fill=c] (8.60697,4.63088) rectangle (8.64677,4.73673);
\draw [color=c, fill=c] (8.64677,4.63088) rectangle (8.68657,4.73673);
\draw [color=c, fill=c] (8.68657,4.63088) rectangle (8.72637,4.73673);
\draw [color=c, fill=c] (8.72637,4.63088) rectangle (8.76617,4.73673);
\draw [color=c, fill=c] (8.76617,4.63088) rectangle (8.80597,4.73673);
\draw [color=c, fill=c] (8.80597,4.63088) rectangle (8.84577,4.73673);
\draw [color=c, fill=c] (8.84577,4.63088) rectangle (8.88557,4.73673);
\draw [color=c, fill=c] (8.88557,4.63088) rectangle (8.92537,4.73673);
\draw [color=c, fill=c] (8.92537,4.63088) rectangle (8.96517,4.73673);
\draw [color=c, fill=c] (8.96517,4.63088) rectangle (9.00498,4.73673);
\draw [color=c, fill=c] (9.00498,4.63088) rectangle (9.04478,4.73673);
\draw [color=c, fill=c] (9.04478,4.63088) rectangle (9.08458,4.73673);
\draw [color=c, fill=c] (9.08458,4.63088) rectangle (9.12438,4.73673);
\draw [color=c, fill=c] (9.12438,4.63088) rectangle (9.16418,4.73673);
\draw [color=c, fill=c] (9.16418,4.63088) rectangle (9.20398,4.73673);
\draw [color=c, fill=c] (9.20398,4.63088) rectangle (9.24378,4.73673);
\draw [color=c, fill=c] (9.24378,4.63088) rectangle (9.28358,4.73673);
\draw [color=c, fill=c] (9.28358,4.63088) rectangle (9.32338,4.73673);
\draw [color=c, fill=c] (9.32338,4.63088) rectangle (9.36318,4.73673);
\draw [color=c, fill=c] (9.36318,4.63088) rectangle (9.40298,4.73673);
\draw [color=c, fill=c] (9.40298,4.63088) rectangle (9.44279,4.73673);
\draw [color=c, fill=c] (9.44279,4.63088) rectangle (9.48259,4.73673);
\draw [color=c, fill=c] (9.48259,4.63088) rectangle (9.52239,4.73673);
\draw [color=c, fill=c] (9.52239,4.63088) rectangle (9.56219,4.73673);
\draw [color=c, fill=c] (9.56219,4.63088) rectangle (9.60199,4.73673);
\draw [color=c, fill=c] (9.60199,4.63088) rectangle (9.64179,4.73673);
\draw [color=c, fill=c] (9.64179,4.63088) rectangle (9.68159,4.73673);
\definecolor{c}{rgb}{0,0.0800001,1};
\draw [color=c, fill=c] (9.68159,4.63088) rectangle (9.72139,4.73673);
\draw [color=c, fill=c] (9.72139,4.63088) rectangle (9.76119,4.73673);
\draw [color=c, fill=c] (9.76119,4.63088) rectangle (9.80099,4.73673);
\draw [color=c, fill=c] (9.80099,4.63088) rectangle (9.8408,4.73673);
\draw [color=c, fill=c] (9.8408,4.63088) rectangle (9.8806,4.73673);
\definecolor{c}{rgb}{0,0.266667,1};
\draw [color=c, fill=c] (9.8806,4.63088) rectangle (9.9204,4.73673);
\draw [color=c, fill=c] (9.9204,4.63088) rectangle (9.9602,4.73673);
\draw [color=c, fill=c] (9.9602,4.63088) rectangle (10,4.73673);
\definecolor{c}{rgb}{0,0.546666,1};
\draw [color=c, fill=c] (10,4.63088) rectangle (10.0398,4.73673);
\draw [color=c, fill=c] (10.0398,4.63088) rectangle (10.0796,4.73673);
\draw [color=c, fill=c] (10.0796,4.63088) rectangle (10.1194,4.73673);
\draw [color=c, fill=c] (10.1194,4.63088) rectangle (10.1592,4.73673);
\definecolor{c}{rgb}{0,0.733333,1};
\draw [color=c, fill=c] (10.1592,4.63088) rectangle (10.199,4.73673);
\draw [color=c, fill=c] (10.199,4.63088) rectangle (10.2388,4.73673);
\draw [color=c, fill=c] (10.2388,4.63088) rectangle (10.2786,4.73673);
\draw [color=c, fill=c] (10.2786,4.63088) rectangle (10.3184,4.73673);
\draw [color=c, fill=c] (10.3184,4.63088) rectangle (10.3582,4.73673);
\draw [color=c, fill=c] (10.3582,4.63088) rectangle (10.398,4.73673);
\draw [color=c, fill=c] (10.398,4.63088) rectangle (10.4378,4.73673);
\draw [color=c, fill=c] (10.4378,4.63088) rectangle (10.4776,4.73673);
\draw [color=c, fill=c] (10.4776,4.63088) rectangle (10.5174,4.73673);
\draw [color=c, fill=c] (10.5174,4.63088) rectangle (10.5572,4.73673);
\definecolor{c}{rgb}{0,1,0.986667};
\draw [color=c, fill=c] (10.5572,4.63088) rectangle (10.597,4.73673);
\draw [color=c, fill=c] (10.597,4.63088) rectangle (10.6368,4.73673);
\draw [color=c, fill=c] (10.6368,4.63088) rectangle (10.6766,4.73673);
\draw [color=c, fill=c] (10.6766,4.63088) rectangle (10.7164,4.73673);
\draw [color=c, fill=c] (10.7164,4.63088) rectangle (10.7562,4.73673);
\draw [color=c, fill=c] (10.7562,4.63088) rectangle (10.796,4.73673);
\draw [color=c, fill=c] (10.796,4.63088) rectangle (10.8358,4.73673);
\draw [color=c, fill=c] (10.8358,4.63088) rectangle (10.8756,4.73673);
\draw [color=c, fill=c] (10.8756,4.63088) rectangle (10.9154,4.73673);
\draw [color=c, fill=c] (10.9154,4.63088) rectangle (10.9552,4.73673);
\draw [color=c, fill=c] (10.9552,4.63088) rectangle (10.995,4.73673);
\draw [color=c, fill=c] (10.995,4.63088) rectangle (11.0348,4.73673);
\draw [color=c, fill=c] (11.0348,4.63088) rectangle (11.0746,4.73673);
\draw [color=c, fill=c] (11.0746,4.63088) rectangle (11.1144,4.73673);
\draw [color=c, fill=c] (11.1144,4.63088) rectangle (11.1542,4.73673);
\draw [color=c, fill=c] (11.1542,4.63088) rectangle (11.194,4.73673);
\draw [color=c, fill=c] (11.194,4.63088) rectangle (11.2338,4.73673);
\draw [color=c, fill=c] (11.2338,4.63088) rectangle (11.2736,4.73673);
\draw [color=c, fill=c] (11.2736,4.63088) rectangle (11.3134,4.73673);
\definecolor{c}{rgb}{0,0.733333,1};
\draw [color=c, fill=c] (11.3134,4.63088) rectangle (11.3532,4.73673);
\draw [color=c, fill=c] (11.3532,4.63088) rectangle (11.393,4.73673);
\draw [color=c, fill=c] (11.393,4.63088) rectangle (11.4328,4.73673);
\draw [color=c, fill=c] (11.4328,4.63088) rectangle (11.4726,4.73673);
\draw [color=c, fill=c] (11.4726,4.63088) rectangle (11.5124,4.73673);
\draw [color=c, fill=c] (11.5124,4.63088) rectangle (11.5522,4.73673);
\draw [color=c, fill=c] (11.5522,4.63088) rectangle (11.592,4.73673);
\draw [color=c, fill=c] (11.592,4.63088) rectangle (11.6318,4.73673);
\draw [color=c, fill=c] (11.6318,4.63088) rectangle (11.6716,4.73673);
\draw [color=c, fill=c] (11.6716,4.63088) rectangle (11.7114,4.73673);
\draw [color=c, fill=c] (11.7114,4.63088) rectangle (11.7512,4.73673);
\draw [color=c, fill=c] (11.7512,4.63088) rectangle (11.791,4.73673);
\draw [color=c, fill=c] (11.791,4.63088) rectangle (11.8308,4.73673);
\draw [color=c, fill=c] (11.8308,4.63088) rectangle (11.8706,4.73673);
\draw [color=c, fill=c] (11.8706,4.63088) rectangle (11.9104,4.73673);
\draw [color=c, fill=c] (11.9104,4.63088) rectangle (11.9502,4.73673);
\draw [color=c, fill=c] (11.9502,4.63088) rectangle (11.99,4.73673);
\draw [color=c, fill=c] (11.99,4.63088) rectangle (12.0299,4.73673);
\draw [color=c, fill=c] (12.0299,4.63088) rectangle (12.0697,4.73673);
\draw [color=c, fill=c] (12.0697,4.63088) rectangle (12.1095,4.73673);
\draw [color=c, fill=c] (12.1095,4.63088) rectangle (12.1493,4.73673);
\draw [color=c, fill=c] (12.1493,4.63088) rectangle (12.1891,4.73673);
\draw [color=c, fill=c] (12.1891,4.63088) rectangle (12.2289,4.73673);
\draw [color=c, fill=c] (12.2289,4.63088) rectangle (12.2687,4.73673);
\draw [color=c, fill=c] (12.2687,4.63088) rectangle (12.3085,4.73673);
\draw [color=c, fill=c] (12.3085,4.63088) rectangle (12.3483,4.73673);
\draw [color=c, fill=c] (12.3483,4.63088) rectangle (12.3881,4.73673);
\draw [color=c, fill=c] (12.3881,4.63088) rectangle (12.4279,4.73673);
\draw [color=c, fill=c] (12.4279,4.63088) rectangle (12.4677,4.73673);
\draw [color=c, fill=c] (12.4677,4.63088) rectangle (12.5075,4.73673);
\draw [color=c, fill=c] (12.5075,4.63088) rectangle (12.5473,4.73673);
\draw [color=c, fill=c] (12.5473,4.63088) rectangle (12.5871,4.73673);
\draw [color=c, fill=c] (12.5871,4.63088) rectangle (12.6269,4.73673);
\draw [color=c, fill=c] (12.6269,4.63088) rectangle (12.6667,4.73673);
\draw [color=c, fill=c] (12.6667,4.63088) rectangle (12.7065,4.73673);
\draw [color=c, fill=c] (12.7065,4.63088) rectangle (12.7463,4.73673);
\draw [color=c, fill=c] (12.7463,4.63088) rectangle (12.7861,4.73673);
\draw [color=c, fill=c] (12.7861,4.63088) rectangle (12.8259,4.73673);
\draw [color=c, fill=c] (12.8259,4.63088) rectangle (12.8657,4.73673);
\draw [color=c, fill=c] (12.8657,4.63088) rectangle (12.9055,4.73673);
\draw [color=c, fill=c] (12.9055,4.63088) rectangle (12.9453,4.73673);
\draw [color=c, fill=c] (12.9453,4.63088) rectangle (12.9851,4.73673);
\draw [color=c, fill=c] (12.9851,4.63088) rectangle (13.0249,4.73673);
\draw [color=c, fill=c] (13.0249,4.63088) rectangle (13.0647,4.73673);
\draw [color=c, fill=c] (13.0647,4.63088) rectangle (13.1045,4.73673);
\draw [color=c, fill=c] (13.1045,4.63088) rectangle (13.1443,4.73673);
\draw [color=c, fill=c] (13.1443,4.63088) rectangle (13.1841,4.73673);
\draw [color=c, fill=c] (13.1841,4.63088) rectangle (13.2239,4.73673);
\draw [color=c, fill=c] (13.2239,4.63088) rectangle (13.2637,4.73673);
\draw [color=c, fill=c] (13.2637,4.63088) rectangle (13.3035,4.73673);
\draw [color=c, fill=c] (13.3035,4.63088) rectangle (13.3433,4.73673);
\draw [color=c, fill=c] (13.3433,4.63088) rectangle (13.3831,4.73673);
\draw [color=c, fill=c] (13.3831,4.63088) rectangle (13.4229,4.73673);
\draw [color=c, fill=c] (13.4229,4.63088) rectangle (13.4627,4.73673);
\draw [color=c, fill=c] (13.4627,4.63088) rectangle (13.5025,4.73673);
\draw [color=c, fill=c] (13.5025,4.63088) rectangle (13.5423,4.73673);
\draw [color=c, fill=c] (13.5423,4.63088) rectangle (13.5821,4.73673);
\draw [color=c, fill=c] (13.5821,4.63088) rectangle (13.6219,4.73673);
\draw [color=c, fill=c] (13.6219,4.63088) rectangle (13.6617,4.73673);
\draw [color=c, fill=c] (13.6617,4.63088) rectangle (13.7015,4.73673);
\draw [color=c, fill=c] (13.7015,4.63088) rectangle (13.7413,4.73673);
\draw [color=c, fill=c] (13.7413,4.63088) rectangle (13.7811,4.73673);
\draw [color=c, fill=c] (13.7811,4.63088) rectangle (13.8209,4.73673);
\draw [color=c, fill=c] (13.8209,4.63088) rectangle (13.8607,4.73673);
\draw [color=c, fill=c] (13.8607,4.63088) rectangle (13.9005,4.73673);
\draw [color=c, fill=c] (13.9005,4.63088) rectangle (13.9403,4.73673);
\draw [color=c, fill=c] (13.9403,4.63088) rectangle (13.9801,4.73673);
\draw [color=c, fill=c] (13.9801,4.63088) rectangle (14.0199,4.73673);
\draw [color=c, fill=c] (14.0199,4.63088) rectangle (14.0597,4.73673);
\draw [color=c, fill=c] (14.0597,4.63088) rectangle (14.0995,4.73673);
\draw [color=c, fill=c] (14.0995,4.63088) rectangle (14.1393,4.73673);
\draw [color=c, fill=c] (14.1393,4.63088) rectangle (14.1791,4.73673);
\draw [color=c, fill=c] (14.1791,4.63088) rectangle (14.2189,4.73673);
\draw [color=c, fill=c] (14.2189,4.63088) rectangle (14.2587,4.73673);
\draw [color=c, fill=c] (14.2587,4.63088) rectangle (14.2985,4.73673);
\draw [color=c, fill=c] (14.2985,4.63088) rectangle (14.3383,4.73673);
\draw [color=c, fill=c] (14.3383,4.63088) rectangle (14.3781,4.73673);
\draw [color=c, fill=c] (14.3781,4.63088) rectangle (14.4179,4.73673);
\draw [color=c, fill=c] (14.4179,4.63088) rectangle (14.4577,4.73673);
\draw [color=c, fill=c] (14.4577,4.63088) rectangle (14.4975,4.73673);
\draw [color=c, fill=c] (14.4975,4.63088) rectangle (14.5373,4.73673);
\draw [color=c, fill=c] (14.5373,4.63088) rectangle (14.5771,4.73673);
\draw [color=c, fill=c] (14.5771,4.63088) rectangle (14.6169,4.73673);
\draw [color=c, fill=c] (14.6169,4.63088) rectangle (14.6567,4.73673);
\draw [color=c, fill=c] (14.6567,4.63088) rectangle (14.6965,4.73673);
\draw [color=c, fill=c] (14.6965,4.63088) rectangle (14.7363,4.73673);
\draw [color=c, fill=c] (14.7363,4.63088) rectangle (14.7761,4.73673);
\draw [color=c, fill=c] (14.7761,4.63088) rectangle (14.8159,4.73673);
\draw [color=c, fill=c] (14.8159,4.63088) rectangle (14.8557,4.73673);
\draw [color=c, fill=c] (14.8557,4.63088) rectangle (14.8955,4.73673);
\draw [color=c, fill=c] (14.8955,4.63088) rectangle (14.9353,4.73673);
\draw [color=c, fill=c] (14.9353,4.63088) rectangle (14.9751,4.73673);
\draw [color=c, fill=c] (14.9751,4.63088) rectangle (15.0149,4.73673);
\draw [color=c, fill=c] (15.0149,4.63088) rectangle (15.0547,4.73673);
\draw [color=c, fill=c] (15.0547,4.63088) rectangle (15.0945,4.73673);
\draw [color=c, fill=c] (15.0945,4.63088) rectangle (15.1343,4.73673);
\draw [color=c, fill=c] (15.1343,4.63088) rectangle (15.1741,4.73673);
\draw [color=c, fill=c] (15.1741,4.63088) rectangle (15.2139,4.73673);
\draw [color=c, fill=c] (15.2139,4.63088) rectangle (15.2537,4.73673);
\draw [color=c, fill=c] (15.2537,4.63088) rectangle (15.2935,4.73673);
\draw [color=c, fill=c] (15.2935,4.63088) rectangle (15.3333,4.73673);
\draw [color=c, fill=c] (15.3333,4.63088) rectangle (15.3731,4.73673);
\draw [color=c, fill=c] (15.3731,4.63088) rectangle (15.4129,4.73673);
\draw [color=c, fill=c] (15.4129,4.63088) rectangle (15.4527,4.73673);
\draw [color=c, fill=c] (15.4527,4.63088) rectangle (15.4925,4.73673);
\draw [color=c, fill=c] (15.4925,4.63088) rectangle (15.5323,4.73673);
\draw [color=c, fill=c] (15.5323,4.63088) rectangle (15.5721,4.73673);
\draw [color=c, fill=c] (15.5721,4.63088) rectangle (15.6119,4.73673);
\draw [color=c, fill=c] (15.6119,4.63088) rectangle (15.6517,4.73673);
\draw [color=c, fill=c] (15.6517,4.63088) rectangle (15.6915,4.73673);
\draw [color=c, fill=c] (15.6915,4.63088) rectangle (15.7313,4.73673);
\draw [color=c, fill=c] (15.7313,4.63088) rectangle (15.7711,4.73673);
\draw [color=c, fill=c] (15.7711,4.63088) rectangle (15.8109,4.73673);
\draw [color=c, fill=c] (15.8109,4.63088) rectangle (15.8507,4.73673);
\draw [color=c, fill=c] (15.8507,4.63088) rectangle (15.8905,4.73673);
\draw [color=c, fill=c] (15.8905,4.63088) rectangle (15.9303,4.73673);
\draw [color=c, fill=c] (15.9303,4.63088) rectangle (15.9701,4.73673);
\draw [color=c, fill=c] (15.9701,4.63088) rectangle (16.01,4.73673);
\draw [color=c, fill=c] (16.01,4.63088) rectangle (16.0498,4.73673);
\draw [color=c, fill=c] (16.0498,4.63088) rectangle (16.0896,4.73673);
\draw [color=c, fill=c] (16.0896,4.63088) rectangle (16.1294,4.73673);
\draw [color=c, fill=c] (16.1294,4.63088) rectangle (16.1692,4.73673);
\draw [color=c, fill=c] (16.1692,4.63088) rectangle (16.209,4.73673);
\draw [color=c, fill=c] (16.209,4.63088) rectangle (16.2488,4.73673);
\draw [color=c, fill=c] (16.2488,4.63088) rectangle (16.2886,4.73673);
\draw [color=c, fill=c] (16.2886,4.63088) rectangle (16.3284,4.73673);
\draw [color=c, fill=c] (16.3284,4.63088) rectangle (16.3682,4.73673);
\draw [color=c, fill=c] (16.3682,4.63088) rectangle (16.408,4.73673);
\draw [color=c, fill=c] (16.408,4.63088) rectangle (16.4478,4.73673);
\draw [color=c, fill=c] (16.4478,4.63088) rectangle (16.4876,4.73673);
\draw [color=c, fill=c] (16.4876,4.63088) rectangle (16.5274,4.73673);
\draw [color=c, fill=c] (16.5274,4.63088) rectangle (16.5672,4.73673);
\draw [color=c, fill=c] (16.5672,4.63088) rectangle (16.607,4.73673);
\draw [color=c, fill=c] (16.607,4.63088) rectangle (16.6468,4.73673);
\draw [color=c, fill=c] (16.6468,4.63088) rectangle (16.6866,4.73673);
\draw [color=c, fill=c] (16.6866,4.63088) rectangle (16.7264,4.73673);
\draw [color=c, fill=c] (16.7264,4.63088) rectangle (16.7662,4.73673);
\draw [color=c, fill=c] (16.7662,4.63088) rectangle (16.806,4.73673);
\draw [color=c, fill=c] (16.806,4.63088) rectangle (16.8458,4.73673);
\draw [color=c, fill=c] (16.8458,4.63088) rectangle (16.8856,4.73673);
\draw [color=c, fill=c] (16.8856,4.63088) rectangle (16.9254,4.73673);
\draw [color=c, fill=c] (16.9254,4.63088) rectangle (16.9652,4.73673);
\draw [color=c, fill=c] (16.9652,4.63088) rectangle (17.005,4.73673);
\draw [color=c, fill=c] (17.005,4.63088) rectangle (17.0448,4.73673);
\draw [color=c, fill=c] (17.0448,4.63088) rectangle (17.0846,4.73673);
\draw [color=c, fill=c] (17.0846,4.63088) rectangle (17.1244,4.73673);
\draw [color=c, fill=c] (17.1244,4.63088) rectangle (17.1642,4.73673);
\draw [color=c, fill=c] (17.1642,4.63088) rectangle (17.204,4.73673);
\draw [color=c, fill=c] (17.204,4.63088) rectangle (17.2438,4.73673);
\draw [color=c, fill=c] (17.2438,4.63088) rectangle (17.2836,4.73673);
\draw [color=c, fill=c] (17.2836,4.63088) rectangle (17.3234,4.73673);
\draw [color=c, fill=c] (17.3234,4.63088) rectangle (17.3632,4.73673);
\draw [color=c, fill=c] (17.3632,4.63088) rectangle (17.403,4.73673);
\draw [color=c, fill=c] (17.403,4.63088) rectangle (17.4428,4.73673);
\draw [color=c, fill=c] (17.4428,4.63088) rectangle (17.4826,4.73673);
\draw [color=c, fill=c] (17.4826,4.63088) rectangle (17.5224,4.73673);
\draw [color=c, fill=c] (17.5224,4.63088) rectangle (17.5622,4.73673);
\draw [color=c, fill=c] (17.5622,4.63088) rectangle (17.602,4.73673);
\draw [color=c, fill=c] (17.602,4.63088) rectangle (17.6418,4.73673);
\draw [color=c, fill=c] (17.6418,4.63088) rectangle (17.6816,4.73673);
\draw [color=c, fill=c] (17.6816,4.63088) rectangle (17.7214,4.73673);
\draw [color=c, fill=c] (17.7214,4.63088) rectangle (17.7612,4.73673);
\draw [color=c, fill=c] (17.7612,4.63088) rectangle (17.801,4.73673);
\draw [color=c, fill=c] (17.801,4.63088) rectangle (17.8408,4.73673);
\draw [color=c, fill=c] (17.8408,4.63088) rectangle (17.8806,4.73673);
\draw [color=c, fill=c] (17.8806,4.63088) rectangle (17.9204,4.73673);
\draw [color=c, fill=c] (17.9204,4.63088) rectangle (17.9602,4.73673);
\draw [color=c, fill=c] (17.9602,4.63088) rectangle (18,4.73673);
\definecolor{c}{rgb}{0,0.0800001,1};
\draw [color=c, fill=c] (2,4.73673) rectangle (2.0398,4.84258);
\draw [color=c, fill=c] (2.0398,4.73673) rectangle (2.0796,4.84258);
\draw [color=c, fill=c] (2.0796,4.73673) rectangle (2.1194,4.84258);
\draw [color=c, fill=c] (2.1194,4.73673) rectangle (2.1592,4.84258);
\draw [color=c, fill=c] (2.1592,4.73673) rectangle (2.19901,4.84258);
\draw [color=c, fill=c] (2.19901,4.73673) rectangle (2.23881,4.84258);
\draw [color=c, fill=c] (2.23881,4.73673) rectangle (2.27861,4.84258);
\draw [color=c, fill=c] (2.27861,4.73673) rectangle (2.31841,4.84258);
\draw [color=c, fill=c] (2.31841,4.73673) rectangle (2.35821,4.84258);
\draw [color=c, fill=c] (2.35821,4.73673) rectangle (2.39801,4.84258);
\draw [color=c, fill=c] (2.39801,4.73673) rectangle (2.43781,4.84258);
\draw [color=c, fill=c] (2.43781,4.73673) rectangle (2.47761,4.84258);
\draw [color=c, fill=c] (2.47761,4.73673) rectangle (2.51741,4.84258);
\draw [color=c, fill=c] (2.51741,4.73673) rectangle (2.55721,4.84258);
\draw [color=c, fill=c] (2.55721,4.73673) rectangle (2.59702,4.84258);
\draw [color=c, fill=c] (2.59702,4.73673) rectangle (2.63682,4.84258);
\draw [color=c, fill=c] (2.63682,4.73673) rectangle (2.67662,4.84258);
\draw [color=c, fill=c] (2.67662,4.73673) rectangle (2.71642,4.84258);
\draw [color=c, fill=c] (2.71642,4.73673) rectangle (2.75622,4.84258);
\draw [color=c, fill=c] (2.75622,4.73673) rectangle (2.79602,4.84258);
\draw [color=c, fill=c] (2.79602,4.73673) rectangle (2.83582,4.84258);
\draw [color=c, fill=c] (2.83582,4.73673) rectangle (2.87562,4.84258);
\draw [color=c, fill=c] (2.87562,4.73673) rectangle (2.91542,4.84258);
\draw [color=c, fill=c] (2.91542,4.73673) rectangle (2.95522,4.84258);
\draw [color=c, fill=c] (2.95522,4.73673) rectangle (2.99502,4.84258);
\draw [color=c, fill=c] (2.99502,4.73673) rectangle (3.03483,4.84258);
\draw [color=c, fill=c] (3.03483,4.73673) rectangle (3.07463,4.84258);
\draw [color=c, fill=c] (3.07463,4.73673) rectangle (3.11443,4.84258);
\draw [color=c, fill=c] (3.11443,4.73673) rectangle (3.15423,4.84258);
\draw [color=c, fill=c] (3.15423,4.73673) rectangle (3.19403,4.84258);
\draw [color=c, fill=c] (3.19403,4.73673) rectangle (3.23383,4.84258);
\draw [color=c, fill=c] (3.23383,4.73673) rectangle (3.27363,4.84258);
\draw [color=c, fill=c] (3.27363,4.73673) rectangle (3.31343,4.84258);
\draw [color=c, fill=c] (3.31343,4.73673) rectangle (3.35323,4.84258);
\draw [color=c, fill=c] (3.35323,4.73673) rectangle (3.39303,4.84258);
\draw [color=c, fill=c] (3.39303,4.73673) rectangle (3.43284,4.84258);
\draw [color=c, fill=c] (3.43284,4.73673) rectangle (3.47264,4.84258);
\draw [color=c, fill=c] (3.47264,4.73673) rectangle (3.51244,4.84258);
\draw [color=c, fill=c] (3.51244,4.73673) rectangle (3.55224,4.84258);
\draw [color=c, fill=c] (3.55224,4.73673) rectangle (3.59204,4.84258);
\draw [color=c, fill=c] (3.59204,4.73673) rectangle (3.63184,4.84258);
\draw [color=c, fill=c] (3.63184,4.73673) rectangle (3.67164,4.84258);
\draw [color=c, fill=c] (3.67164,4.73673) rectangle (3.71144,4.84258);
\draw [color=c, fill=c] (3.71144,4.73673) rectangle (3.75124,4.84258);
\draw [color=c, fill=c] (3.75124,4.73673) rectangle (3.79104,4.84258);
\draw [color=c, fill=c] (3.79104,4.73673) rectangle (3.83085,4.84258);
\draw [color=c, fill=c] (3.83085,4.73673) rectangle (3.87065,4.84258);
\draw [color=c, fill=c] (3.87065,4.73673) rectangle (3.91045,4.84258);
\draw [color=c, fill=c] (3.91045,4.73673) rectangle (3.95025,4.84258);
\draw [color=c, fill=c] (3.95025,4.73673) rectangle (3.99005,4.84258);
\draw [color=c, fill=c] (3.99005,4.73673) rectangle (4.02985,4.84258);
\draw [color=c, fill=c] (4.02985,4.73673) rectangle (4.06965,4.84258);
\draw [color=c, fill=c] (4.06965,4.73673) rectangle (4.10945,4.84258);
\draw [color=c, fill=c] (4.10945,4.73673) rectangle (4.14925,4.84258);
\draw [color=c, fill=c] (4.14925,4.73673) rectangle (4.18905,4.84258);
\draw [color=c, fill=c] (4.18905,4.73673) rectangle (4.22886,4.84258);
\draw [color=c, fill=c] (4.22886,4.73673) rectangle (4.26866,4.84258);
\draw [color=c, fill=c] (4.26866,4.73673) rectangle (4.30846,4.84258);
\draw [color=c, fill=c] (4.30846,4.73673) rectangle (4.34826,4.84258);
\draw [color=c, fill=c] (4.34826,4.73673) rectangle (4.38806,4.84258);
\draw [color=c, fill=c] (4.38806,4.73673) rectangle (4.42786,4.84258);
\draw [color=c, fill=c] (4.42786,4.73673) rectangle (4.46766,4.84258);
\draw [color=c, fill=c] (4.46766,4.73673) rectangle (4.50746,4.84258);
\draw [color=c, fill=c] (4.50746,4.73673) rectangle (4.54726,4.84258);
\draw [color=c, fill=c] (4.54726,4.73673) rectangle (4.58706,4.84258);
\draw [color=c, fill=c] (4.58706,4.73673) rectangle (4.62687,4.84258);
\draw [color=c, fill=c] (4.62687,4.73673) rectangle (4.66667,4.84258);
\draw [color=c, fill=c] (4.66667,4.73673) rectangle (4.70647,4.84258);
\draw [color=c, fill=c] (4.70647,4.73673) rectangle (4.74627,4.84258);
\draw [color=c, fill=c] (4.74627,4.73673) rectangle (4.78607,4.84258);
\draw [color=c, fill=c] (4.78607,4.73673) rectangle (4.82587,4.84258);
\draw [color=c, fill=c] (4.82587,4.73673) rectangle (4.86567,4.84258);
\draw [color=c, fill=c] (4.86567,4.73673) rectangle (4.90547,4.84258);
\draw [color=c, fill=c] (4.90547,4.73673) rectangle (4.94527,4.84258);
\draw [color=c, fill=c] (4.94527,4.73673) rectangle (4.98507,4.84258);
\draw [color=c, fill=c] (4.98507,4.73673) rectangle (5.02488,4.84258);
\draw [color=c, fill=c] (5.02488,4.73673) rectangle (5.06468,4.84258);
\draw [color=c, fill=c] (5.06468,4.73673) rectangle (5.10448,4.84258);
\draw [color=c, fill=c] (5.10448,4.73673) rectangle (5.14428,4.84258);
\draw [color=c, fill=c] (5.14428,4.73673) rectangle (5.18408,4.84258);
\draw [color=c, fill=c] (5.18408,4.73673) rectangle (5.22388,4.84258);
\draw [color=c, fill=c] (5.22388,4.73673) rectangle (5.26368,4.84258);
\draw [color=c, fill=c] (5.26368,4.73673) rectangle (5.30348,4.84258);
\draw [color=c, fill=c] (5.30348,4.73673) rectangle (5.34328,4.84258);
\draw [color=c, fill=c] (5.34328,4.73673) rectangle (5.38308,4.84258);
\draw [color=c, fill=c] (5.38308,4.73673) rectangle (5.42289,4.84258);
\draw [color=c, fill=c] (5.42289,4.73673) rectangle (5.46269,4.84258);
\draw [color=c, fill=c] (5.46269,4.73673) rectangle (5.50249,4.84258);
\draw [color=c, fill=c] (5.50249,4.73673) rectangle (5.54229,4.84258);
\draw [color=c, fill=c] (5.54229,4.73673) rectangle (5.58209,4.84258);
\draw [color=c, fill=c] (5.58209,4.73673) rectangle (5.62189,4.84258);
\draw [color=c, fill=c] (5.62189,4.73673) rectangle (5.66169,4.84258);
\draw [color=c, fill=c] (5.66169,4.73673) rectangle (5.70149,4.84258);
\draw [color=c, fill=c] (5.70149,4.73673) rectangle (5.74129,4.84258);
\draw [color=c, fill=c] (5.74129,4.73673) rectangle (5.78109,4.84258);
\draw [color=c, fill=c] (5.78109,4.73673) rectangle (5.8209,4.84258);
\draw [color=c, fill=c] (5.8209,4.73673) rectangle (5.8607,4.84258);
\definecolor{c}{rgb}{0.2,0,1};
\draw [color=c, fill=c] (5.8607,4.73673) rectangle (5.9005,4.84258);
\draw [color=c, fill=c] (5.9005,4.73673) rectangle (5.9403,4.84258);
\draw [color=c, fill=c] (5.9403,4.73673) rectangle (5.9801,4.84258);
\draw [color=c, fill=c] (5.9801,4.73673) rectangle (6.0199,4.84258);
\draw [color=c, fill=c] (6.0199,4.73673) rectangle (6.0597,4.84258);
\draw [color=c, fill=c] (6.0597,4.73673) rectangle (6.0995,4.84258);
\draw [color=c, fill=c] (6.0995,4.73673) rectangle (6.1393,4.84258);
\draw [color=c, fill=c] (6.1393,4.73673) rectangle (6.1791,4.84258);
\draw [color=c, fill=c] (6.1791,4.73673) rectangle (6.21891,4.84258);
\draw [color=c, fill=c] (6.21891,4.73673) rectangle (6.25871,4.84258);
\draw [color=c, fill=c] (6.25871,4.73673) rectangle (6.29851,4.84258);
\draw [color=c, fill=c] (6.29851,4.73673) rectangle (6.33831,4.84258);
\draw [color=c, fill=c] (6.33831,4.73673) rectangle (6.37811,4.84258);
\draw [color=c, fill=c] (6.37811,4.73673) rectangle (6.41791,4.84258);
\draw [color=c, fill=c] (6.41791,4.73673) rectangle (6.45771,4.84258);
\draw [color=c, fill=c] (6.45771,4.73673) rectangle (6.49751,4.84258);
\draw [color=c, fill=c] (6.49751,4.73673) rectangle (6.53731,4.84258);
\draw [color=c, fill=c] (6.53731,4.73673) rectangle (6.57711,4.84258);
\draw [color=c, fill=c] (6.57711,4.73673) rectangle (6.61692,4.84258);
\draw [color=c, fill=c] (6.61692,4.73673) rectangle (6.65672,4.84258);
\draw [color=c, fill=c] (6.65672,4.73673) rectangle (6.69652,4.84258);
\draw [color=c, fill=c] (6.69652,4.73673) rectangle (6.73632,4.84258);
\draw [color=c, fill=c] (6.73632,4.73673) rectangle (6.77612,4.84258);
\draw [color=c, fill=c] (6.77612,4.73673) rectangle (6.81592,4.84258);
\draw [color=c, fill=c] (6.81592,4.73673) rectangle (6.85572,4.84258);
\draw [color=c, fill=c] (6.85572,4.73673) rectangle (6.89552,4.84258);
\draw [color=c, fill=c] (6.89552,4.73673) rectangle (6.93532,4.84258);
\draw [color=c, fill=c] (6.93532,4.73673) rectangle (6.97512,4.84258);
\draw [color=c, fill=c] (6.97512,4.73673) rectangle (7.01493,4.84258);
\draw [color=c, fill=c] (7.01493,4.73673) rectangle (7.05473,4.84258);
\draw [color=c, fill=c] (7.05473,4.73673) rectangle (7.09453,4.84258);
\draw [color=c, fill=c] (7.09453,4.73673) rectangle (7.13433,4.84258);
\draw [color=c, fill=c] (7.13433,4.73673) rectangle (7.17413,4.84258);
\draw [color=c, fill=c] (7.17413,4.73673) rectangle (7.21393,4.84258);
\draw [color=c, fill=c] (7.21393,4.73673) rectangle (7.25373,4.84258);
\draw [color=c, fill=c] (7.25373,4.73673) rectangle (7.29353,4.84258);
\draw [color=c, fill=c] (7.29353,4.73673) rectangle (7.33333,4.84258);
\draw [color=c, fill=c] (7.33333,4.73673) rectangle (7.37313,4.84258);
\draw [color=c, fill=c] (7.37313,4.73673) rectangle (7.41294,4.84258);
\draw [color=c, fill=c] (7.41294,4.73673) rectangle (7.45274,4.84258);
\draw [color=c, fill=c] (7.45274,4.73673) rectangle (7.49254,4.84258);
\draw [color=c, fill=c] (7.49254,4.73673) rectangle (7.53234,4.84258);
\draw [color=c, fill=c] (7.53234,4.73673) rectangle (7.57214,4.84258);
\draw [color=c, fill=c] (7.57214,4.73673) rectangle (7.61194,4.84258);
\draw [color=c, fill=c] (7.61194,4.73673) rectangle (7.65174,4.84258);
\draw [color=c, fill=c] (7.65174,4.73673) rectangle (7.69154,4.84258);
\draw [color=c, fill=c] (7.69154,4.73673) rectangle (7.73134,4.84258);
\draw [color=c, fill=c] (7.73134,4.73673) rectangle (7.77114,4.84258);
\draw [color=c, fill=c] (7.77114,4.73673) rectangle (7.81095,4.84258);
\draw [color=c, fill=c] (7.81095,4.73673) rectangle (7.85075,4.84258);
\draw [color=c, fill=c] (7.85075,4.73673) rectangle (7.89055,4.84258);
\draw [color=c, fill=c] (7.89055,4.73673) rectangle (7.93035,4.84258);
\draw [color=c, fill=c] (7.93035,4.73673) rectangle (7.97015,4.84258);
\draw [color=c, fill=c] (7.97015,4.73673) rectangle (8.00995,4.84258);
\draw [color=c, fill=c] (8.00995,4.73673) rectangle (8.04975,4.84258);
\draw [color=c, fill=c] (8.04975,4.73673) rectangle (8.08955,4.84258);
\draw [color=c, fill=c] (8.08955,4.73673) rectangle (8.12935,4.84258);
\draw [color=c, fill=c] (8.12935,4.73673) rectangle (8.16915,4.84258);
\draw [color=c, fill=c] (8.16915,4.73673) rectangle (8.20895,4.84258);
\draw [color=c, fill=c] (8.20895,4.73673) rectangle (8.24876,4.84258);
\draw [color=c, fill=c] (8.24876,4.73673) rectangle (8.28856,4.84258);
\draw [color=c, fill=c] (8.28856,4.73673) rectangle (8.32836,4.84258);
\draw [color=c, fill=c] (8.32836,4.73673) rectangle (8.36816,4.84258);
\draw [color=c, fill=c] (8.36816,4.73673) rectangle (8.40796,4.84258);
\draw [color=c, fill=c] (8.40796,4.73673) rectangle (8.44776,4.84258);
\draw [color=c, fill=c] (8.44776,4.73673) rectangle (8.48756,4.84258);
\draw [color=c, fill=c] (8.48756,4.73673) rectangle (8.52736,4.84258);
\draw [color=c, fill=c] (8.52736,4.73673) rectangle (8.56716,4.84258);
\draw [color=c, fill=c] (8.56716,4.73673) rectangle (8.60697,4.84258);
\draw [color=c, fill=c] (8.60697,4.73673) rectangle (8.64677,4.84258);
\draw [color=c, fill=c] (8.64677,4.73673) rectangle (8.68657,4.84258);
\draw [color=c, fill=c] (8.68657,4.73673) rectangle (8.72637,4.84258);
\draw [color=c, fill=c] (8.72637,4.73673) rectangle (8.76617,4.84258);
\draw [color=c, fill=c] (8.76617,4.73673) rectangle (8.80597,4.84258);
\draw [color=c, fill=c] (8.80597,4.73673) rectangle (8.84577,4.84258);
\draw [color=c, fill=c] (8.84577,4.73673) rectangle (8.88557,4.84258);
\draw [color=c, fill=c] (8.88557,4.73673) rectangle (8.92537,4.84258);
\draw [color=c, fill=c] (8.92537,4.73673) rectangle (8.96517,4.84258);
\draw [color=c, fill=c] (8.96517,4.73673) rectangle (9.00498,4.84258);
\draw [color=c, fill=c] (9.00498,4.73673) rectangle (9.04478,4.84258);
\draw [color=c, fill=c] (9.04478,4.73673) rectangle (9.08458,4.84258);
\draw [color=c, fill=c] (9.08458,4.73673) rectangle (9.12438,4.84258);
\draw [color=c, fill=c] (9.12438,4.73673) rectangle (9.16418,4.84258);
\draw [color=c, fill=c] (9.16418,4.73673) rectangle (9.20398,4.84258);
\draw [color=c, fill=c] (9.20398,4.73673) rectangle (9.24378,4.84258);
\draw [color=c, fill=c] (9.24378,4.73673) rectangle (9.28358,4.84258);
\draw [color=c, fill=c] (9.28358,4.73673) rectangle (9.32338,4.84258);
\draw [color=c, fill=c] (9.32338,4.73673) rectangle (9.36318,4.84258);
\draw [color=c, fill=c] (9.36318,4.73673) rectangle (9.40298,4.84258);
\draw [color=c, fill=c] (9.40298,4.73673) rectangle (9.44279,4.84258);
\draw [color=c, fill=c] (9.44279,4.73673) rectangle (9.48259,4.84258);
\draw [color=c, fill=c] (9.48259,4.73673) rectangle (9.52239,4.84258);
\draw [color=c, fill=c] (9.52239,4.73673) rectangle (9.56219,4.84258);
\draw [color=c, fill=c] (9.56219,4.73673) rectangle (9.60199,4.84258);
\draw [color=c, fill=c] (9.60199,4.73673) rectangle (9.64179,4.84258);
\definecolor{c}{rgb}{0,0.0800001,1};
\draw [color=c, fill=c] (9.64179,4.73673) rectangle (9.68159,4.84258);
\draw [color=c, fill=c] (9.68159,4.73673) rectangle (9.72139,4.84258);
\draw [color=c, fill=c] (9.72139,4.73673) rectangle (9.76119,4.84258);
\draw [color=c, fill=c] (9.76119,4.73673) rectangle (9.80099,4.84258);
\draw [color=c, fill=c] (9.80099,4.73673) rectangle (9.8408,4.84258);
\draw [color=c, fill=c] (9.8408,4.73673) rectangle (9.8806,4.84258);
\definecolor{c}{rgb}{0,0.266667,1};
\draw [color=c, fill=c] (9.8806,4.73673) rectangle (9.9204,4.84258);
\draw [color=c, fill=c] (9.9204,4.73673) rectangle (9.9602,4.84258);
\draw [color=c, fill=c] (9.9602,4.73673) rectangle (10,4.84258);
\definecolor{c}{rgb}{0,0.546666,1};
\draw [color=c, fill=c] (10,4.73673) rectangle (10.0398,4.84258);
\draw [color=c, fill=c] (10.0398,4.73673) rectangle (10.0796,4.84258);
\draw [color=c, fill=c] (10.0796,4.73673) rectangle (10.1194,4.84258);
\draw [color=c, fill=c] (10.1194,4.73673) rectangle (10.1592,4.84258);
\draw [color=c, fill=c] (10.1592,4.73673) rectangle (10.199,4.84258);
\definecolor{c}{rgb}{0,0.733333,1};
\draw [color=c, fill=c] (10.199,4.73673) rectangle (10.2388,4.84258);
\draw [color=c, fill=c] (10.2388,4.73673) rectangle (10.2786,4.84258);
\draw [color=c, fill=c] (10.2786,4.73673) rectangle (10.3184,4.84258);
\draw [color=c, fill=c] (10.3184,4.73673) rectangle (10.3582,4.84258);
\draw [color=c, fill=c] (10.3582,4.73673) rectangle (10.398,4.84258);
\draw [color=c, fill=c] (10.398,4.73673) rectangle (10.4378,4.84258);
\draw [color=c, fill=c] (10.4378,4.73673) rectangle (10.4776,4.84258);
\draw [color=c, fill=c] (10.4776,4.73673) rectangle (10.5174,4.84258);
\draw [color=c, fill=c] (10.5174,4.73673) rectangle (10.5572,4.84258);
\draw [color=c, fill=c] (10.5572,4.73673) rectangle (10.597,4.84258);
\draw [color=c, fill=c] (10.597,4.73673) rectangle (10.6368,4.84258);
\draw [color=c, fill=c] (10.6368,4.73673) rectangle (10.6766,4.84258);
\draw [color=c, fill=c] (10.6766,4.73673) rectangle (10.7164,4.84258);
\draw [color=c, fill=c] (10.7164,4.73673) rectangle (10.7562,4.84258);
\definecolor{c}{rgb}{0,1,0.986667};
\draw [color=c, fill=c] (10.7562,4.73673) rectangle (10.796,4.84258);
\draw [color=c, fill=c] (10.796,4.73673) rectangle (10.8358,4.84258);
\draw [color=c, fill=c] (10.8358,4.73673) rectangle (10.8756,4.84258);
\draw [color=c, fill=c] (10.8756,4.73673) rectangle (10.9154,4.84258);
\draw [color=c, fill=c] (10.9154,4.73673) rectangle (10.9552,4.84258);
\draw [color=c, fill=c] (10.9552,4.73673) rectangle (10.995,4.84258);
\draw [color=c, fill=c] (10.995,4.73673) rectangle (11.0348,4.84258);
\draw [color=c, fill=c] (11.0348,4.73673) rectangle (11.0746,4.84258);
\draw [color=c, fill=c] (11.0746,4.73673) rectangle (11.1144,4.84258);
\definecolor{c}{rgb}{0,0.733333,1};
\draw [color=c, fill=c] (11.1144,4.73673) rectangle (11.1542,4.84258);
\draw [color=c, fill=c] (11.1542,4.73673) rectangle (11.194,4.84258);
\draw [color=c, fill=c] (11.194,4.73673) rectangle (11.2338,4.84258);
\draw [color=c, fill=c] (11.2338,4.73673) rectangle (11.2736,4.84258);
\draw [color=c, fill=c] (11.2736,4.73673) rectangle (11.3134,4.84258);
\draw [color=c, fill=c] (11.3134,4.73673) rectangle (11.3532,4.84258);
\draw [color=c, fill=c] (11.3532,4.73673) rectangle (11.393,4.84258);
\draw [color=c, fill=c] (11.393,4.73673) rectangle (11.4328,4.84258);
\draw [color=c, fill=c] (11.4328,4.73673) rectangle (11.4726,4.84258);
\draw [color=c, fill=c] (11.4726,4.73673) rectangle (11.5124,4.84258);
\draw [color=c, fill=c] (11.5124,4.73673) rectangle (11.5522,4.84258);
\draw [color=c, fill=c] (11.5522,4.73673) rectangle (11.592,4.84258);
\draw [color=c, fill=c] (11.592,4.73673) rectangle (11.6318,4.84258);
\draw [color=c, fill=c] (11.6318,4.73673) rectangle (11.6716,4.84258);
\draw [color=c, fill=c] (11.6716,4.73673) rectangle (11.7114,4.84258);
\draw [color=c, fill=c] (11.7114,4.73673) rectangle (11.7512,4.84258);
\draw [color=c, fill=c] (11.7512,4.73673) rectangle (11.791,4.84258);
\draw [color=c, fill=c] (11.791,4.73673) rectangle (11.8308,4.84258);
\draw [color=c, fill=c] (11.8308,4.73673) rectangle (11.8706,4.84258);
\draw [color=c, fill=c] (11.8706,4.73673) rectangle (11.9104,4.84258);
\draw [color=c, fill=c] (11.9104,4.73673) rectangle (11.9502,4.84258);
\draw [color=c, fill=c] (11.9502,4.73673) rectangle (11.99,4.84258);
\draw [color=c, fill=c] (11.99,4.73673) rectangle (12.0299,4.84258);
\draw [color=c, fill=c] (12.0299,4.73673) rectangle (12.0697,4.84258);
\draw [color=c, fill=c] (12.0697,4.73673) rectangle (12.1095,4.84258);
\draw [color=c, fill=c] (12.1095,4.73673) rectangle (12.1493,4.84258);
\draw [color=c, fill=c] (12.1493,4.73673) rectangle (12.1891,4.84258);
\draw [color=c, fill=c] (12.1891,4.73673) rectangle (12.2289,4.84258);
\draw [color=c, fill=c] (12.2289,4.73673) rectangle (12.2687,4.84258);
\draw [color=c, fill=c] (12.2687,4.73673) rectangle (12.3085,4.84258);
\draw [color=c, fill=c] (12.3085,4.73673) rectangle (12.3483,4.84258);
\draw [color=c, fill=c] (12.3483,4.73673) rectangle (12.3881,4.84258);
\draw [color=c, fill=c] (12.3881,4.73673) rectangle (12.4279,4.84258);
\draw [color=c, fill=c] (12.4279,4.73673) rectangle (12.4677,4.84258);
\draw [color=c, fill=c] (12.4677,4.73673) rectangle (12.5075,4.84258);
\draw [color=c, fill=c] (12.5075,4.73673) rectangle (12.5473,4.84258);
\draw [color=c, fill=c] (12.5473,4.73673) rectangle (12.5871,4.84258);
\draw [color=c, fill=c] (12.5871,4.73673) rectangle (12.6269,4.84258);
\draw [color=c, fill=c] (12.6269,4.73673) rectangle (12.6667,4.84258);
\draw [color=c, fill=c] (12.6667,4.73673) rectangle (12.7065,4.84258);
\draw [color=c, fill=c] (12.7065,4.73673) rectangle (12.7463,4.84258);
\draw [color=c, fill=c] (12.7463,4.73673) rectangle (12.7861,4.84258);
\draw [color=c, fill=c] (12.7861,4.73673) rectangle (12.8259,4.84258);
\draw [color=c, fill=c] (12.8259,4.73673) rectangle (12.8657,4.84258);
\draw [color=c, fill=c] (12.8657,4.73673) rectangle (12.9055,4.84258);
\draw [color=c, fill=c] (12.9055,4.73673) rectangle (12.9453,4.84258);
\draw [color=c, fill=c] (12.9453,4.73673) rectangle (12.9851,4.84258);
\draw [color=c, fill=c] (12.9851,4.73673) rectangle (13.0249,4.84258);
\draw [color=c, fill=c] (13.0249,4.73673) rectangle (13.0647,4.84258);
\draw [color=c, fill=c] (13.0647,4.73673) rectangle (13.1045,4.84258);
\draw [color=c, fill=c] (13.1045,4.73673) rectangle (13.1443,4.84258);
\draw [color=c, fill=c] (13.1443,4.73673) rectangle (13.1841,4.84258);
\draw [color=c, fill=c] (13.1841,4.73673) rectangle (13.2239,4.84258);
\draw [color=c, fill=c] (13.2239,4.73673) rectangle (13.2637,4.84258);
\draw [color=c, fill=c] (13.2637,4.73673) rectangle (13.3035,4.84258);
\draw [color=c, fill=c] (13.3035,4.73673) rectangle (13.3433,4.84258);
\draw [color=c, fill=c] (13.3433,4.73673) rectangle (13.3831,4.84258);
\draw [color=c, fill=c] (13.3831,4.73673) rectangle (13.4229,4.84258);
\draw [color=c, fill=c] (13.4229,4.73673) rectangle (13.4627,4.84258);
\draw [color=c, fill=c] (13.4627,4.73673) rectangle (13.5025,4.84258);
\draw [color=c, fill=c] (13.5025,4.73673) rectangle (13.5423,4.84258);
\draw [color=c, fill=c] (13.5423,4.73673) rectangle (13.5821,4.84258);
\draw [color=c, fill=c] (13.5821,4.73673) rectangle (13.6219,4.84258);
\draw [color=c, fill=c] (13.6219,4.73673) rectangle (13.6617,4.84258);
\draw [color=c, fill=c] (13.6617,4.73673) rectangle (13.7015,4.84258);
\draw [color=c, fill=c] (13.7015,4.73673) rectangle (13.7413,4.84258);
\draw [color=c, fill=c] (13.7413,4.73673) rectangle (13.7811,4.84258);
\draw [color=c, fill=c] (13.7811,4.73673) rectangle (13.8209,4.84258);
\draw [color=c, fill=c] (13.8209,4.73673) rectangle (13.8607,4.84258);
\draw [color=c, fill=c] (13.8607,4.73673) rectangle (13.9005,4.84258);
\draw [color=c, fill=c] (13.9005,4.73673) rectangle (13.9403,4.84258);
\draw [color=c, fill=c] (13.9403,4.73673) rectangle (13.9801,4.84258);
\draw [color=c, fill=c] (13.9801,4.73673) rectangle (14.0199,4.84258);
\draw [color=c, fill=c] (14.0199,4.73673) rectangle (14.0597,4.84258);
\draw [color=c, fill=c] (14.0597,4.73673) rectangle (14.0995,4.84258);
\draw [color=c, fill=c] (14.0995,4.73673) rectangle (14.1393,4.84258);
\draw [color=c, fill=c] (14.1393,4.73673) rectangle (14.1791,4.84258);
\draw [color=c, fill=c] (14.1791,4.73673) rectangle (14.2189,4.84258);
\draw [color=c, fill=c] (14.2189,4.73673) rectangle (14.2587,4.84258);
\draw [color=c, fill=c] (14.2587,4.73673) rectangle (14.2985,4.84258);
\draw [color=c, fill=c] (14.2985,4.73673) rectangle (14.3383,4.84258);
\draw [color=c, fill=c] (14.3383,4.73673) rectangle (14.3781,4.84258);
\draw [color=c, fill=c] (14.3781,4.73673) rectangle (14.4179,4.84258);
\draw [color=c, fill=c] (14.4179,4.73673) rectangle (14.4577,4.84258);
\draw [color=c, fill=c] (14.4577,4.73673) rectangle (14.4975,4.84258);
\draw [color=c, fill=c] (14.4975,4.73673) rectangle (14.5373,4.84258);
\draw [color=c, fill=c] (14.5373,4.73673) rectangle (14.5771,4.84258);
\draw [color=c, fill=c] (14.5771,4.73673) rectangle (14.6169,4.84258);
\draw [color=c, fill=c] (14.6169,4.73673) rectangle (14.6567,4.84258);
\draw [color=c, fill=c] (14.6567,4.73673) rectangle (14.6965,4.84258);
\draw [color=c, fill=c] (14.6965,4.73673) rectangle (14.7363,4.84258);
\draw [color=c, fill=c] (14.7363,4.73673) rectangle (14.7761,4.84258);
\draw [color=c, fill=c] (14.7761,4.73673) rectangle (14.8159,4.84258);
\draw [color=c, fill=c] (14.8159,4.73673) rectangle (14.8557,4.84258);
\draw [color=c, fill=c] (14.8557,4.73673) rectangle (14.8955,4.84258);
\draw [color=c, fill=c] (14.8955,4.73673) rectangle (14.9353,4.84258);
\draw [color=c, fill=c] (14.9353,4.73673) rectangle (14.9751,4.84258);
\draw [color=c, fill=c] (14.9751,4.73673) rectangle (15.0149,4.84258);
\draw [color=c, fill=c] (15.0149,4.73673) rectangle (15.0547,4.84258);
\draw [color=c, fill=c] (15.0547,4.73673) rectangle (15.0945,4.84258);
\draw [color=c, fill=c] (15.0945,4.73673) rectangle (15.1343,4.84258);
\draw [color=c, fill=c] (15.1343,4.73673) rectangle (15.1741,4.84258);
\draw [color=c, fill=c] (15.1741,4.73673) rectangle (15.2139,4.84258);
\draw [color=c, fill=c] (15.2139,4.73673) rectangle (15.2537,4.84258);
\draw [color=c, fill=c] (15.2537,4.73673) rectangle (15.2935,4.84258);
\draw [color=c, fill=c] (15.2935,4.73673) rectangle (15.3333,4.84258);
\draw [color=c, fill=c] (15.3333,4.73673) rectangle (15.3731,4.84258);
\draw [color=c, fill=c] (15.3731,4.73673) rectangle (15.4129,4.84258);
\draw [color=c, fill=c] (15.4129,4.73673) rectangle (15.4527,4.84258);
\draw [color=c, fill=c] (15.4527,4.73673) rectangle (15.4925,4.84258);
\draw [color=c, fill=c] (15.4925,4.73673) rectangle (15.5323,4.84258);
\draw [color=c, fill=c] (15.5323,4.73673) rectangle (15.5721,4.84258);
\draw [color=c, fill=c] (15.5721,4.73673) rectangle (15.6119,4.84258);
\draw [color=c, fill=c] (15.6119,4.73673) rectangle (15.6517,4.84258);
\draw [color=c, fill=c] (15.6517,4.73673) rectangle (15.6915,4.84258);
\draw [color=c, fill=c] (15.6915,4.73673) rectangle (15.7313,4.84258);
\draw [color=c, fill=c] (15.7313,4.73673) rectangle (15.7711,4.84258);
\draw [color=c, fill=c] (15.7711,4.73673) rectangle (15.8109,4.84258);
\draw [color=c, fill=c] (15.8109,4.73673) rectangle (15.8507,4.84258);
\draw [color=c, fill=c] (15.8507,4.73673) rectangle (15.8905,4.84258);
\draw [color=c, fill=c] (15.8905,4.73673) rectangle (15.9303,4.84258);
\draw [color=c, fill=c] (15.9303,4.73673) rectangle (15.9701,4.84258);
\draw [color=c, fill=c] (15.9701,4.73673) rectangle (16.01,4.84258);
\draw [color=c, fill=c] (16.01,4.73673) rectangle (16.0498,4.84258);
\draw [color=c, fill=c] (16.0498,4.73673) rectangle (16.0896,4.84258);
\draw [color=c, fill=c] (16.0896,4.73673) rectangle (16.1294,4.84258);
\draw [color=c, fill=c] (16.1294,4.73673) rectangle (16.1692,4.84258);
\draw [color=c, fill=c] (16.1692,4.73673) rectangle (16.209,4.84258);
\draw [color=c, fill=c] (16.209,4.73673) rectangle (16.2488,4.84258);
\draw [color=c, fill=c] (16.2488,4.73673) rectangle (16.2886,4.84258);
\draw [color=c, fill=c] (16.2886,4.73673) rectangle (16.3284,4.84258);
\draw [color=c, fill=c] (16.3284,4.73673) rectangle (16.3682,4.84258);
\draw [color=c, fill=c] (16.3682,4.73673) rectangle (16.408,4.84258);
\draw [color=c, fill=c] (16.408,4.73673) rectangle (16.4478,4.84258);
\draw [color=c, fill=c] (16.4478,4.73673) rectangle (16.4876,4.84258);
\draw [color=c, fill=c] (16.4876,4.73673) rectangle (16.5274,4.84258);
\draw [color=c, fill=c] (16.5274,4.73673) rectangle (16.5672,4.84258);
\draw [color=c, fill=c] (16.5672,4.73673) rectangle (16.607,4.84258);
\draw [color=c, fill=c] (16.607,4.73673) rectangle (16.6468,4.84258);
\draw [color=c, fill=c] (16.6468,4.73673) rectangle (16.6866,4.84258);
\draw [color=c, fill=c] (16.6866,4.73673) rectangle (16.7264,4.84258);
\draw [color=c, fill=c] (16.7264,4.73673) rectangle (16.7662,4.84258);
\draw [color=c, fill=c] (16.7662,4.73673) rectangle (16.806,4.84258);
\draw [color=c, fill=c] (16.806,4.73673) rectangle (16.8458,4.84258);
\draw [color=c, fill=c] (16.8458,4.73673) rectangle (16.8856,4.84258);
\draw [color=c, fill=c] (16.8856,4.73673) rectangle (16.9254,4.84258);
\draw [color=c, fill=c] (16.9254,4.73673) rectangle (16.9652,4.84258);
\draw [color=c, fill=c] (16.9652,4.73673) rectangle (17.005,4.84258);
\draw [color=c, fill=c] (17.005,4.73673) rectangle (17.0448,4.84258);
\draw [color=c, fill=c] (17.0448,4.73673) rectangle (17.0846,4.84258);
\draw [color=c, fill=c] (17.0846,4.73673) rectangle (17.1244,4.84258);
\draw [color=c, fill=c] (17.1244,4.73673) rectangle (17.1642,4.84258);
\draw [color=c, fill=c] (17.1642,4.73673) rectangle (17.204,4.84258);
\draw [color=c, fill=c] (17.204,4.73673) rectangle (17.2438,4.84258);
\draw [color=c, fill=c] (17.2438,4.73673) rectangle (17.2836,4.84258);
\draw [color=c, fill=c] (17.2836,4.73673) rectangle (17.3234,4.84258);
\draw [color=c, fill=c] (17.3234,4.73673) rectangle (17.3632,4.84258);
\draw [color=c, fill=c] (17.3632,4.73673) rectangle (17.403,4.84258);
\draw [color=c, fill=c] (17.403,4.73673) rectangle (17.4428,4.84258);
\draw [color=c, fill=c] (17.4428,4.73673) rectangle (17.4826,4.84258);
\draw [color=c, fill=c] (17.4826,4.73673) rectangle (17.5224,4.84258);
\draw [color=c, fill=c] (17.5224,4.73673) rectangle (17.5622,4.84258);
\draw [color=c, fill=c] (17.5622,4.73673) rectangle (17.602,4.84258);
\draw [color=c, fill=c] (17.602,4.73673) rectangle (17.6418,4.84258);
\draw [color=c, fill=c] (17.6418,4.73673) rectangle (17.6816,4.84258);
\draw [color=c, fill=c] (17.6816,4.73673) rectangle (17.7214,4.84258);
\draw [color=c, fill=c] (17.7214,4.73673) rectangle (17.7612,4.84258);
\draw [color=c, fill=c] (17.7612,4.73673) rectangle (17.801,4.84258);
\draw [color=c, fill=c] (17.801,4.73673) rectangle (17.8408,4.84258);
\draw [color=c, fill=c] (17.8408,4.73673) rectangle (17.8806,4.84258);
\draw [color=c, fill=c] (17.8806,4.73673) rectangle (17.9204,4.84258);
\draw [color=c, fill=c] (17.9204,4.73673) rectangle (17.9602,4.84258);
\draw [color=c, fill=c] (17.9602,4.73673) rectangle (18,4.84258);
\definecolor{c}{rgb}{0,0.0800001,1};
\draw [color=c, fill=c] (2,4.84258) rectangle (2.0398,4.94842);
\draw [color=c, fill=c] (2.0398,4.84258) rectangle (2.0796,4.94842);
\draw [color=c, fill=c] (2.0796,4.84258) rectangle (2.1194,4.94842);
\draw [color=c, fill=c] (2.1194,4.84258) rectangle (2.1592,4.94842);
\draw [color=c, fill=c] (2.1592,4.84258) rectangle (2.19901,4.94842);
\draw [color=c, fill=c] (2.19901,4.84258) rectangle (2.23881,4.94842);
\draw [color=c, fill=c] (2.23881,4.84258) rectangle (2.27861,4.94842);
\draw [color=c, fill=c] (2.27861,4.84258) rectangle (2.31841,4.94842);
\draw [color=c, fill=c] (2.31841,4.84258) rectangle (2.35821,4.94842);
\draw [color=c, fill=c] (2.35821,4.84258) rectangle (2.39801,4.94842);
\draw [color=c, fill=c] (2.39801,4.84258) rectangle (2.43781,4.94842);
\draw [color=c, fill=c] (2.43781,4.84258) rectangle (2.47761,4.94842);
\draw [color=c, fill=c] (2.47761,4.84258) rectangle (2.51741,4.94842);
\draw [color=c, fill=c] (2.51741,4.84258) rectangle (2.55721,4.94842);
\draw [color=c, fill=c] (2.55721,4.84258) rectangle (2.59702,4.94842);
\draw [color=c, fill=c] (2.59702,4.84258) rectangle (2.63682,4.94842);
\draw [color=c, fill=c] (2.63682,4.84258) rectangle (2.67662,4.94842);
\draw [color=c, fill=c] (2.67662,4.84258) rectangle (2.71642,4.94842);
\draw [color=c, fill=c] (2.71642,4.84258) rectangle (2.75622,4.94842);
\draw [color=c, fill=c] (2.75622,4.84258) rectangle (2.79602,4.94842);
\draw [color=c, fill=c] (2.79602,4.84258) rectangle (2.83582,4.94842);
\draw [color=c, fill=c] (2.83582,4.84258) rectangle (2.87562,4.94842);
\draw [color=c, fill=c] (2.87562,4.84258) rectangle (2.91542,4.94842);
\draw [color=c, fill=c] (2.91542,4.84258) rectangle (2.95522,4.94842);
\draw [color=c, fill=c] (2.95522,4.84258) rectangle (2.99502,4.94842);
\draw [color=c, fill=c] (2.99502,4.84258) rectangle (3.03483,4.94842);
\draw [color=c, fill=c] (3.03483,4.84258) rectangle (3.07463,4.94842);
\draw [color=c, fill=c] (3.07463,4.84258) rectangle (3.11443,4.94842);
\draw [color=c, fill=c] (3.11443,4.84258) rectangle (3.15423,4.94842);
\draw [color=c, fill=c] (3.15423,4.84258) rectangle (3.19403,4.94842);
\draw [color=c, fill=c] (3.19403,4.84258) rectangle (3.23383,4.94842);
\draw [color=c, fill=c] (3.23383,4.84258) rectangle (3.27363,4.94842);
\draw [color=c, fill=c] (3.27363,4.84258) rectangle (3.31343,4.94842);
\draw [color=c, fill=c] (3.31343,4.84258) rectangle (3.35323,4.94842);
\draw [color=c, fill=c] (3.35323,4.84258) rectangle (3.39303,4.94842);
\draw [color=c, fill=c] (3.39303,4.84258) rectangle (3.43284,4.94842);
\draw [color=c, fill=c] (3.43284,4.84258) rectangle (3.47264,4.94842);
\draw [color=c, fill=c] (3.47264,4.84258) rectangle (3.51244,4.94842);
\draw [color=c, fill=c] (3.51244,4.84258) rectangle (3.55224,4.94842);
\draw [color=c, fill=c] (3.55224,4.84258) rectangle (3.59204,4.94842);
\draw [color=c, fill=c] (3.59204,4.84258) rectangle (3.63184,4.94842);
\draw [color=c, fill=c] (3.63184,4.84258) rectangle (3.67164,4.94842);
\draw [color=c, fill=c] (3.67164,4.84258) rectangle (3.71144,4.94842);
\draw [color=c, fill=c] (3.71144,4.84258) rectangle (3.75124,4.94842);
\draw [color=c, fill=c] (3.75124,4.84258) rectangle (3.79104,4.94842);
\draw [color=c, fill=c] (3.79104,4.84258) rectangle (3.83085,4.94842);
\draw [color=c, fill=c] (3.83085,4.84258) rectangle (3.87065,4.94842);
\draw [color=c, fill=c] (3.87065,4.84258) rectangle (3.91045,4.94842);
\draw [color=c, fill=c] (3.91045,4.84258) rectangle (3.95025,4.94842);
\draw [color=c, fill=c] (3.95025,4.84258) rectangle (3.99005,4.94842);
\draw [color=c, fill=c] (3.99005,4.84258) rectangle (4.02985,4.94842);
\draw [color=c, fill=c] (4.02985,4.84258) rectangle (4.06965,4.94842);
\draw [color=c, fill=c] (4.06965,4.84258) rectangle (4.10945,4.94842);
\draw [color=c, fill=c] (4.10945,4.84258) rectangle (4.14925,4.94842);
\draw [color=c, fill=c] (4.14925,4.84258) rectangle (4.18905,4.94842);
\draw [color=c, fill=c] (4.18905,4.84258) rectangle (4.22886,4.94842);
\draw [color=c, fill=c] (4.22886,4.84258) rectangle (4.26866,4.94842);
\draw [color=c, fill=c] (4.26866,4.84258) rectangle (4.30846,4.94842);
\draw [color=c, fill=c] (4.30846,4.84258) rectangle (4.34826,4.94842);
\draw [color=c, fill=c] (4.34826,4.84258) rectangle (4.38806,4.94842);
\draw [color=c, fill=c] (4.38806,4.84258) rectangle (4.42786,4.94842);
\draw [color=c, fill=c] (4.42786,4.84258) rectangle (4.46766,4.94842);
\draw [color=c, fill=c] (4.46766,4.84258) rectangle (4.50746,4.94842);
\draw [color=c, fill=c] (4.50746,4.84258) rectangle (4.54726,4.94842);
\draw [color=c, fill=c] (4.54726,4.84258) rectangle (4.58706,4.94842);
\draw [color=c, fill=c] (4.58706,4.84258) rectangle (4.62687,4.94842);
\draw [color=c, fill=c] (4.62687,4.84258) rectangle (4.66667,4.94842);
\draw [color=c, fill=c] (4.66667,4.84258) rectangle (4.70647,4.94842);
\draw [color=c, fill=c] (4.70647,4.84258) rectangle (4.74627,4.94842);
\draw [color=c, fill=c] (4.74627,4.84258) rectangle (4.78607,4.94842);
\draw [color=c, fill=c] (4.78607,4.84258) rectangle (4.82587,4.94842);
\draw [color=c, fill=c] (4.82587,4.84258) rectangle (4.86567,4.94842);
\draw [color=c, fill=c] (4.86567,4.84258) rectangle (4.90547,4.94842);
\draw [color=c, fill=c] (4.90547,4.84258) rectangle (4.94527,4.94842);
\draw [color=c, fill=c] (4.94527,4.84258) rectangle (4.98507,4.94842);
\draw [color=c, fill=c] (4.98507,4.84258) rectangle (5.02488,4.94842);
\draw [color=c, fill=c] (5.02488,4.84258) rectangle (5.06468,4.94842);
\draw [color=c, fill=c] (5.06468,4.84258) rectangle (5.10448,4.94842);
\draw [color=c, fill=c] (5.10448,4.84258) rectangle (5.14428,4.94842);
\draw [color=c, fill=c] (5.14428,4.84258) rectangle (5.18408,4.94842);
\draw [color=c, fill=c] (5.18408,4.84258) rectangle (5.22388,4.94842);
\draw [color=c, fill=c] (5.22388,4.84258) rectangle (5.26368,4.94842);
\draw [color=c, fill=c] (5.26368,4.84258) rectangle (5.30348,4.94842);
\draw [color=c, fill=c] (5.30348,4.84258) rectangle (5.34328,4.94842);
\draw [color=c, fill=c] (5.34328,4.84258) rectangle (5.38308,4.94842);
\draw [color=c, fill=c] (5.38308,4.84258) rectangle (5.42289,4.94842);
\draw [color=c, fill=c] (5.42289,4.84258) rectangle (5.46269,4.94842);
\draw [color=c, fill=c] (5.46269,4.84258) rectangle (5.50249,4.94842);
\draw [color=c, fill=c] (5.50249,4.84258) rectangle (5.54229,4.94842);
\draw [color=c, fill=c] (5.54229,4.84258) rectangle (5.58209,4.94842);
\draw [color=c, fill=c] (5.58209,4.84258) rectangle (5.62189,4.94842);
\draw [color=c, fill=c] (5.62189,4.84258) rectangle (5.66169,4.94842);
\draw [color=c, fill=c] (5.66169,4.84258) rectangle (5.70149,4.94842);
\draw [color=c, fill=c] (5.70149,4.84258) rectangle (5.74129,4.94842);
\draw [color=c, fill=c] (5.74129,4.84258) rectangle (5.78109,4.94842);
\draw [color=c, fill=c] (5.78109,4.84258) rectangle (5.8209,4.94842);
\definecolor{c}{rgb}{0.2,0,1};
\draw [color=c, fill=c] (5.8209,4.84258) rectangle (5.8607,4.94842);
\draw [color=c, fill=c] (5.8607,4.84258) rectangle (5.9005,4.94842);
\draw [color=c, fill=c] (5.9005,4.84258) rectangle (5.9403,4.94842);
\draw [color=c, fill=c] (5.9403,4.84258) rectangle (5.9801,4.94842);
\draw [color=c, fill=c] (5.9801,4.84258) rectangle (6.0199,4.94842);
\draw [color=c, fill=c] (6.0199,4.84258) rectangle (6.0597,4.94842);
\draw [color=c, fill=c] (6.0597,4.84258) rectangle (6.0995,4.94842);
\draw [color=c, fill=c] (6.0995,4.84258) rectangle (6.1393,4.94842);
\draw [color=c, fill=c] (6.1393,4.84258) rectangle (6.1791,4.94842);
\draw [color=c, fill=c] (6.1791,4.84258) rectangle (6.21891,4.94842);
\draw [color=c, fill=c] (6.21891,4.84258) rectangle (6.25871,4.94842);
\draw [color=c, fill=c] (6.25871,4.84258) rectangle (6.29851,4.94842);
\draw [color=c, fill=c] (6.29851,4.84258) rectangle (6.33831,4.94842);
\draw [color=c, fill=c] (6.33831,4.84258) rectangle (6.37811,4.94842);
\draw [color=c, fill=c] (6.37811,4.84258) rectangle (6.41791,4.94842);
\draw [color=c, fill=c] (6.41791,4.84258) rectangle (6.45771,4.94842);
\draw [color=c, fill=c] (6.45771,4.84258) rectangle (6.49751,4.94842);
\draw [color=c, fill=c] (6.49751,4.84258) rectangle (6.53731,4.94842);
\draw [color=c, fill=c] (6.53731,4.84258) rectangle (6.57711,4.94842);
\draw [color=c, fill=c] (6.57711,4.84258) rectangle (6.61692,4.94842);
\draw [color=c, fill=c] (6.61692,4.84258) rectangle (6.65672,4.94842);
\draw [color=c, fill=c] (6.65672,4.84258) rectangle (6.69652,4.94842);
\draw [color=c, fill=c] (6.69652,4.84258) rectangle (6.73632,4.94842);
\draw [color=c, fill=c] (6.73632,4.84258) rectangle (6.77612,4.94842);
\draw [color=c, fill=c] (6.77612,4.84258) rectangle (6.81592,4.94842);
\draw [color=c, fill=c] (6.81592,4.84258) rectangle (6.85572,4.94842);
\draw [color=c, fill=c] (6.85572,4.84258) rectangle (6.89552,4.94842);
\draw [color=c, fill=c] (6.89552,4.84258) rectangle (6.93532,4.94842);
\draw [color=c, fill=c] (6.93532,4.84258) rectangle (6.97512,4.94842);
\draw [color=c, fill=c] (6.97512,4.84258) rectangle (7.01493,4.94842);
\draw [color=c, fill=c] (7.01493,4.84258) rectangle (7.05473,4.94842);
\draw [color=c, fill=c] (7.05473,4.84258) rectangle (7.09453,4.94842);
\draw [color=c, fill=c] (7.09453,4.84258) rectangle (7.13433,4.94842);
\draw [color=c, fill=c] (7.13433,4.84258) rectangle (7.17413,4.94842);
\draw [color=c, fill=c] (7.17413,4.84258) rectangle (7.21393,4.94842);
\draw [color=c, fill=c] (7.21393,4.84258) rectangle (7.25373,4.94842);
\draw [color=c, fill=c] (7.25373,4.84258) rectangle (7.29353,4.94842);
\draw [color=c, fill=c] (7.29353,4.84258) rectangle (7.33333,4.94842);
\draw [color=c, fill=c] (7.33333,4.84258) rectangle (7.37313,4.94842);
\draw [color=c, fill=c] (7.37313,4.84258) rectangle (7.41294,4.94842);
\draw [color=c, fill=c] (7.41294,4.84258) rectangle (7.45274,4.94842);
\draw [color=c, fill=c] (7.45274,4.84258) rectangle (7.49254,4.94842);
\draw [color=c, fill=c] (7.49254,4.84258) rectangle (7.53234,4.94842);
\draw [color=c, fill=c] (7.53234,4.84258) rectangle (7.57214,4.94842);
\draw [color=c, fill=c] (7.57214,4.84258) rectangle (7.61194,4.94842);
\draw [color=c, fill=c] (7.61194,4.84258) rectangle (7.65174,4.94842);
\draw [color=c, fill=c] (7.65174,4.84258) rectangle (7.69154,4.94842);
\draw [color=c, fill=c] (7.69154,4.84258) rectangle (7.73134,4.94842);
\draw [color=c, fill=c] (7.73134,4.84258) rectangle (7.77114,4.94842);
\draw [color=c, fill=c] (7.77114,4.84258) rectangle (7.81095,4.94842);
\draw [color=c, fill=c] (7.81095,4.84258) rectangle (7.85075,4.94842);
\draw [color=c, fill=c] (7.85075,4.84258) rectangle (7.89055,4.94842);
\draw [color=c, fill=c] (7.89055,4.84258) rectangle (7.93035,4.94842);
\draw [color=c, fill=c] (7.93035,4.84258) rectangle (7.97015,4.94842);
\draw [color=c, fill=c] (7.97015,4.84258) rectangle (8.00995,4.94842);
\draw [color=c, fill=c] (8.00995,4.84258) rectangle (8.04975,4.94842);
\draw [color=c, fill=c] (8.04975,4.84258) rectangle (8.08955,4.94842);
\draw [color=c, fill=c] (8.08955,4.84258) rectangle (8.12935,4.94842);
\draw [color=c, fill=c] (8.12935,4.84258) rectangle (8.16915,4.94842);
\draw [color=c, fill=c] (8.16915,4.84258) rectangle (8.20895,4.94842);
\draw [color=c, fill=c] (8.20895,4.84258) rectangle (8.24876,4.94842);
\draw [color=c, fill=c] (8.24876,4.84258) rectangle (8.28856,4.94842);
\draw [color=c, fill=c] (8.28856,4.84258) rectangle (8.32836,4.94842);
\draw [color=c, fill=c] (8.32836,4.84258) rectangle (8.36816,4.94842);
\draw [color=c, fill=c] (8.36816,4.84258) rectangle (8.40796,4.94842);
\draw [color=c, fill=c] (8.40796,4.84258) rectangle (8.44776,4.94842);
\draw [color=c, fill=c] (8.44776,4.84258) rectangle (8.48756,4.94842);
\draw [color=c, fill=c] (8.48756,4.84258) rectangle (8.52736,4.94842);
\draw [color=c, fill=c] (8.52736,4.84258) rectangle (8.56716,4.94842);
\draw [color=c, fill=c] (8.56716,4.84258) rectangle (8.60697,4.94842);
\draw [color=c, fill=c] (8.60697,4.84258) rectangle (8.64677,4.94842);
\draw [color=c, fill=c] (8.64677,4.84258) rectangle (8.68657,4.94842);
\draw [color=c, fill=c] (8.68657,4.84258) rectangle (8.72637,4.94842);
\draw [color=c, fill=c] (8.72637,4.84258) rectangle (8.76617,4.94842);
\draw [color=c, fill=c] (8.76617,4.84258) rectangle (8.80597,4.94842);
\draw [color=c, fill=c] (8.80597,4.84258) rectangle (8.84577,4.94842);
\draw [color=c, fill=c] (8.84577,4.84258) rectangle (8.88557,4.94842);
\draw [color=c, fill=c] (8.88557,4.84258) rectangle (8.92537,4.94842);
\draw [color=c, fill=c] (8.92537,4.84258) rectangle (8.96517,4.94842);
\draw [color=c, fill=c] (8.96517,4.84258) rectangle (9.00498,4.94842);
\draw [color=c, fill=c] (9.00498,4.84258) rectangle (9.04478,4.94842);
\draw [color=c, fill=c] (9.04478,4.84258) rectangle (9.08458,4.94842);
\draw [color=c, fill=c] (9.08458,4.84258) rectangle (9.12438,4.94842);
\draw [color=c, fill=c] (9.12438,4.84258) rectangle (9.16418,4.94842);
\draw [color=c, fill=c] (9.16418,4.84258) rectangle (9.20398,4.94842);
\draw [color=c, fill=c] (9.20398,4.84258) rectangle (9.24378,4.94842);
\draw [color=c, fill=c] (9.24378,4.84258) rectangle (9.28358,4.94842);
\draw [color=c, fill=c] (9.28358,4.84258) rectangle (9.32338,4.94842);
\draw [color=c, fill=c] (9.32338,4.84258) rectangle (9.36318,4.94842);
\draw [color=c, fill=c] (9.36318,4.84258) rectangle (9.40298,4.94842);
\draw [color=c, fill=c] (9.40298,4.84258) rectangle (9.44279,4.94842);
\draw [color=c, fill=c] (9.44279,4.84258) rectangle (9.48259,4.94842);
\draw [color=c, fill=c] (9.48259,4.84258) rectangle (9.52239,4.94842);
\draw [color=c, fill=c] (9.52239,4.84258) rectangle (9.56219,4.94842);
\draw [color=c, fill=c] (9.56219,4.84258) rectangle (9.60199,4.94842);
\definecolor{c}{rgb}{0,0.0800001,1};
\draw [color=c, fill=c] (9.60199,4.84258) rectangle (9.64179,4.94842);
\draw [color=c, fill=c] (9.64179,4.84258) rectangle (9.68159,4.94842);
\draw [color=c, fill=c] (9.68159,4.84258) rectangle (9.72139,4.94842);
\draw [color=c, fill=c] (9.72139,4.84258) rectangle (9.76119,4.94842);
\draw [color=c, fill=c] (9.76119,4.84258) rectangle (9.80099,4.94842);
\draw [color=c, fill=c] (9.80099,4.84258) rectangle (9.8408,4.94842);
\definecolor{c}{rgb}{0,0.266667,1};
\draw [color=c, fill=c] (9.8408,4.84258) rectangle (9.8806,4.94842);
\draw [color=c, fill=c] (9.8806,4.84258) rectangle (9.9204,4.94842);
\draw [color=c, fill=c] (9.9204,4.84258) rectangle (9.9602,4.94842);
\draw [color=c, fill=c] (9.9602,4.84258) rectangle (10,4.94842);
\draw [color=c, fill=c] (10,4.84258) rectangle (10.0398,4.94842);
\definecolor{c}{rgb}{0,0.546666,1};
\draw [color=c, fill=c] (10.0398,4.84258) rectangle (10.0796,4.94842);
\draw [color=c, fill=c] (10.0796,4.84258) rectangle (10.1194,4.94842);
\draw [color=c, fill=c] (10.1194,4.84258) rectangle (10.1592,4.94842);
\draw [color=c, fill=c] (10.1592,4.84258) rectangle (10.199,4.94842);
\draw [color=c, fill=c] (10.199,4.84258) rectangle (10.2388,4.94842);
\definecolor{c}{rgb}{0,0.733333,1};
\draw [color=c, fill=c] (10.2388,4.84258) rectangle (10.2786,4.94842);
\draw [color=c, fill=c] (10.2786,4.84258) rectangle (10.3184,4.94842);
\draw [color=c, fill=c] (10.3184,4.84258) rectangle (10.3582,4.94842);
\draw [color=c, fill=c] (10.3582,4.84258) rectangle (10.398,4.94842);
\draw [color=c, fill=c] (10.398,4.84258) rectangle (10.4378,4.94842);
\draw [color=c, fill=c] (10.4378,4.84258) rectangle (10.4776,4.94842);
\draw [color=c, fill=c] (10.4776,4.84258) rectangle (10.5174,4.94842);
\draw [color=c, fill=c] (10.5174,4.84258) rectangle (10.5572,4.94842);
\draw [color=c, fill=c] (10.5572,4.84258) rectangle (10.597,4.94842);
\draw [color=c, fill=c] (10.597,4.84258) rectangle (10.6368,4.94842);
\draw [color=c, fill=c] (10.6368,4.84258) rectangle (10.6766,4.94842);
\draw [color=c, fill=c] (10.6766,4.84258) rectangle (10.7164,4.94842);
\draw [color=c, fill=c] (10.7164,4.84258) rectangle (10.7562,4.94842);
\draw [color=c, fill=c] (10.7562,4.84258) rectangle (10.796,4.94842);
\draw [color=c, fill=c] (10.796,4.84258) rectangle (10.8358,4.94842);
\draw [color=c, fill=c] (10.8358,4.84258) rectangle (10.8756,4.94842);
\draw [color=c, fill=c] (10.8756,4.84258) rectangle (10.9154,4.94842);
\draw [color=c, fill=c] (10.9154,4.84258) rectangle (10.9552,4.94842);
\draw [color=c, fill=c] (10.9552,4.84258) rectangle (10.995,4.94842);
\draw [color=c, fill=c] (10.995,4.84258) rectangle (11.0348,4.94842);
\draw [color=c, fill=c] (11.0348,4.84258) rectangle (11.0746,4.94842);
\draw [color=c, fill=c] (11.0746,4.84258) rectangle (11.1144,4.94842);
\draw [color=c, fill=c] (11.1144,4.84258) rectangle (11.1542,4.94842);
\draw [color=c, fill=c] (11.1542,4.84258) rectangle (11.194,4.94842);
\draw [color=c, fill=c] (11.194,4.84258) rectangle (11.2338,4.94842);
\draw [color=c, fill=c] (11.2338,4.84258) rectangle (11.2736,4.94842);
\draw [color=c, fill=c] (11.2736,4.84258) rectangle (11.3134,4.94842);
\draw [color=c, fill=c] (11.3134,4.84258) rectangle (11.3532,4.94842);
\draw [color=c, fill=c] (11.3532,4.84258) rectangle (11.393,4.94842);
\draw [color=c, fill=c] (11.393,4.84258) rectangle (11.4328,4.94842);
\draw [color=c, fill=c] (11.4328,4.84258) rectangle (11.4726,4.94842);
\draw [color=c, fill=c] (11.4726,4.84258) rectangle (11.5124,4.94842);
\draw [color=c, fill=c] (11.5124,4.84258) rectangle (11.5522,4.94842);
\draw [color=c, fill=c] (11.5522,4.84258) rectangle (11.592,4.94842);
\draw [color=c, fill=c] (11.592,4.84258) rectangle (11.6318,4.94842);
\draw [color=c, fill=c] (11.6318,4.84258) rectangle (11.6716,4.94842);
\draw [color=c, fill=c] (11.6716,4.84258) rectangle (11.7114,4.94842);
\draw [color=c, fill=c] (11.7114,4.84258) rectangle (11.7512,4.94842);
\draw [color=c, fill=c] (11.7512,4.84258) rectangle (11.791,4.94842);
\draw [color=c, fill=c] (11.791,4.84258) rectangle (11.8308,4.94842);
\draw [color=c, fill=c] (11.8308,4.84258) rectangle (11.8706,4.94842);
\draw [color=c, fill=c] (11.8706,4.84258) rectangle (11.9104,4.94842);
\draw [color=c, fill=c] (11.9104,4.84258) rectangle (11.9502,4.94842);
\draw [color=c, fill=c] (11.9502,4.84258) rectangle (11.99,4.94842);
\draw [color=c, fill=c] (11.99,4.84258) rectangle (12.0299,4.94842);
\draw [color=c, fill=c] (12.0299,4.84258) rectangle (12.0697,4.94842);
\draw [color=c, fill=c] (12.0697,4.84258) rectangle (12.1095,4.94842);
\draw [color=c, fill=c] (12.1095,4.84258) rectangle (12.1493,4.94842);
\draw [color=c, fill=c] (12.1493,4.84258) rectangle (12.1891,4.94842);
\draw [color=c, fill=c] (12.1891,4.84258) rectangle (12.2289,4.94842);
\draw [color=c, fill=c] (12.2289,4.84258) rectangle (12.2687,4.94842);
\draw [color=c, fill=c] (12.2687,4.84258) rectangle (12.3085,4.94842);
\draw [color=c, fill=c] (12.3085,4.84258) rectangle (12.3483,4.94842);
\draw [color=c, fill=c] (12.3483,4.84258) rectangle (12.3881,4.94842);
\draw [color=c, fill=c] (12.3881,4.84258) rectangle (12.4279,4.94842);
\draw [color=c, fill=c] (12.4279,4.84258) rectangle (12.4677,4.94842);
\draw [color=c, fill=c] (12.4677,4.84258) rectangle (12.5075,4.94842);
\draw [color=c, fill=c] (12.5075,4.84258) rectangle (12.5473,4.94842);
\draw [color=c, fill=c] (12.5473,4.84258) rectangle (12.5871,4.94842);
\draw [color=c, fill=c] (12.5871,4.84258) rectangle (12.6269,4.94842);
\draw [color=c, fill=c] (12.6269,4.84258) rectangle (12.6667,4.94842);
\draw [color=c, fill=c] (12.6667,4.84258) rectangle (12.7065,4.94842);
\draw [color=c, fill=c] (12.7065,4.84258) rectangle (12.7463,4.94842);
\draw [color=c, fill=c] (12.7463,4.84258) rectangle (12.7861,4.94842);
\draw [color=c, fill=c] (12.7861,4.84258) rectangle (12.8259,4.94842);
\draw [color=c, fill=c] (12.8259,4.84258) rectangle (12.8657,4.94842);
\draw [color=c, fill=c] (12.8657,4.84258) rectangle (12.9055,4.94842);
\draw [color=c, fill=c] (12.9055,4.84258) rectangle (12.9453,4.94842);
\draw [color=c, fill=c] (12.9453,4.84258) rectangle (12.9851,4.94842);
\draw [color=c, fill=c] (12.9851,4.84258) rectangle (13.0249,4.94842);
\draw [color=c, fill=c] (13.0249,4.84258) rectangle (13.0647,4.94842);
\draw [color=c, fill=c] (13.0647,4.84258) rectangle (13.1045,4.94842);
\draw [color=c, fill=c] (13.1045,4.84258) rectangle (13.1443,4.94842);
\draw [color=c, fill=c] (13.1443,4.84258) rectangle (13.1841,4.94842);
\draw [color=c, fill=c] (13.1841,4.84258) rectangle (13.2239,4.94842);
\draw [color=c, fill=c] (13.2239,4.84258) rectangle (13.2637,4.94842);
\draw [color=c, fill=c] (13.2637,4.84258) rectangle (13.3035,4.94842);
\draw [color=c, fill=c] (13.3035,4.84258) rectangle (13.3433,4.94842);
\draw [color=c, fill=c] (13.3433,4.84258) rectangle (13.3831,4.94842);
\draw [color=c, fill=c] (13.3831,4.84258) rectangle (13.4229,4.94842);
\draw [color=c, fill=c] (13.4229,4.84258) rectangle (13.4627,4.94842);
\draw [color=c, fill=c] (13.4627,4.84258) rectangle (13.5025,4.94842);
\draw [color=c, fill=c] (13.5025,4.84258) rectangle (13.5423,4.94842);
\draw [color=c, fill=c] (13.5423,4.84258) rectangle (13.5821,4.94842);
\draw [color=c, fill=c] (13.5821,4.84258) rectangle (13.6219,4.94842);
\draw [color=c, fill=c] (13.6219,4.84258) rectangle (13.6617,4.94842);
\draw [color=c, fill=c] (13.6617,4.84258) rectangle (13.7015,4.94842);
\draw [color=c, fill=c] (13.7015,4.84258) rectangle (13.7413,4.94842);
\draw [color=c, fill=c] (13.7413,4.84258) rectangle (13.7811,4.94842);
\draw [color=c, fill=c] (13.7811,4.84258) rectangle (13.8209,4.94842);
\draw [color=c, fill=c] (13.8209,4.84258) rectangle (13.8607,4.94842);
\draw [color=c, fill=c] (13.8607,4.84258) rectangle (13.9005,4.94842);
\draw [color=c, fill=c] (13.9005,4.84258) rectangle (13.9403,4.94842);
\draw [color=c, fill=c] (13.9403,4.84258) rectangle (13.9801,4.94842);
\draw [color=c, fill=c] (13.9801,4.84258) rectangle (14.0199,4.94842);
\draw [color=c, fill=c] (14.0199,4.84258) rectangle (14.0597,4.94842);
\draw [color=c, fill=c] (14.0597,4.84258) rectangle (14.0995,4.94842);
\draw [color=c, fill=c] (14.0995,4.84258) rectangle (14.1393,4.94842);
\draw [color=c, fill=c] (14.1393,4.84258) rectangle (14.1791,4.94842);
\draw [color=c, fill=c] (14.1791,4.84258) rectangle (14.2189,4.94842);
\draw [color=c, fill=c] (14.2189,4.84258) rectangle (14.2587,4.94842);
\draw [color=c, fill=c] (14.2587,4.84258) rectangle (14.2985,4.94842);
\draw [color=c, fill=c] (14.2985,4.84258) rectangle (14.3383,4.94842);
\draw [color=c, fill=c] (14.3383,4.84258) rectangle (14.3781,4.94842);
\draw [color=c, fill=c] (14.3781,4.84258) rectangle (14.4179,4.94842);
\draw [color=c, fill=c] (14.4179,4.84258) rectangle (14.4577,4.94842);
\draw [color=c, fill=c] (14.4577,4.84258) rectangle (14.4975,4.94842);
\draw [color=c, fill=c] (14.4975,4.84258) rectangle (14.5373,4.94842);
\draw [color=c, fill=c] (14.5373,4.84258) rectangle (14.5771,4.94842);
\draw [color=c, fill=c] (14.5771,4.84258) rectangle (14.6169,4.94842);
\draw [color=c, fill=c] (14.6169,4.84258) rectangle (14.6567,4.94842);
\draw [color=c, fill=c] (14.6567,4.84258) rectangle (14.6965,4.94842);
\draw [color=c, fill=c] (14.6965,4.84258) rectangle (14.7363,4.94842);
\draw [color=c, fill=c] (14.7363,4.84258) rectangle (14.7761,4.94842);
\draw [color=c, fill=c] (14.7761,4.84258) rectangle (14.8159,4.94842);
\draw [color=c, fill=c] (14.8159,4.84258) rectangle (14.8557,4.94842);
\draw [color=c, fill=c] (14.8557,4.84258) rectangle (14.8955,4.94842);
\draw [color=c, fill=c] (14.8955,4.84258) rectangle (14.9353,4.94842);
\draw [color=c, fill=c] (14.9353,4.84258) rectangle (14.9751,4.94842);
\draw [color=c, fill=c] (14.9751,4.84258) rectangle (15.0149,4.94842);
\draw [color=c, fill=c] (15.0149,4.84258) rectangle (15.0547,4.94842);
\draw [color=c, fill=c] (15.0547,4.84258) rectangle (15.0945,4.94842);
\draw [color=c, fill=c] (15.0945,4.84258) rectangle (15.1343,4.94842);
\draw [color=c, fill=c] (15.1343,4.84258) rectangle (15.1741,4.94842);
\draw [color=c, fill=c] (15.1741,4.84258) rectangle (15.2139,4.94842);
\draw [color=c, fill=c] (15.2139,4.84258) rectangle (15.2537,4.94842);
\draw [color=c, fill=c] (15.2537,4.84258) rectangle (15.2935,4.94842);
\draw [color=c, fill=c] (15.2935,4.84258) rectangle (15.3333,4.94842);
\draw [color=c, fill=c] (15.3333,4.84258) rectangle (15.3731,4.94842);
\draw [color=c, fill=c] (15.3731,4.84258) rectangle (15.4129,4.94842);
\draw [color=c, fill=c] (15.4129,4.84258) rectangle (15.4527,4.94842);
\draw [color=c, fill=c] (15.4527,4.84258) rectangle (15.4925,4.94842);
\draw [color=c, fill=c] (15.4925,4.84258) rectangle (15.5323,4.94842);
\draw [color=c, fill=c] (15.5323,4.84258) rectangle (15.5721,4.94842);
\draw [color=c, fill=c] (15.5721,4.84258) rectangle (15.6119,4.94842);
\draw [color=c, fill=c] (15.6119,4.84258) rectangle (15.6517,4.94842);
\draw [color=c, fill=c] (15.6517,4.84258) rectangle (15.6915,4.94842);
\draw [color=c, fill=c] (15.6915,4.84258) rectangle (15.7313,4.94842);
\draw [color=c, fill=c] (15.7313,4.84258) rectangle (15.7711,4.94842);
\draw [color=c, fill=c] (15.7711,4.84258) rectangle (15.8109,4.94842);
\draw [color=c, fill=c] (15.8109,4.84258) rectangle (15.8507,4.94842);
\draw [color=c, fill=c] (15.8507,4.84258) rectangle (15.8905,4.94842);
\draw [color=c, fill=c] (15.8905,4.84258) rectangle (15.9303,4.94842);
\draw [color=c, fill=c] (15.9303,4.84258) rectangle (15.9701,4.94842);
\draw [color=c, fill=c] (15.9701,4.84258) rectangle (16.01,4.94842);
\draw [color=c, fill=c] (16.01,4.84258) rectangle (16.0498,4.94842);
\draw [color=c, fill=c] (16.0498,4.84258) rectangle (16.0896,4.94842);
\draw [color=c, fill=c] (16.0896,4.84258) rectangle (16.1294,4.94842);
\draw [color=c, fill=c] (16.1294,4.84258) rectangle (16.1692,4.94842);
\draw [color=c, fill=c] (16.1692,4.84258) rectangle (16.209,4.94842);
\draw [color=c, fill=c] (16.209,4.84258) rectangle (16.2488,4.94842);
\draw [color=c, fill=c] (16.2488,4.84258) rectangle (16.2886,4.94842);
\draw [color=c, fill=c] (16.2886,4.84258) rectangle (16.3284,4.94842);
\draw [color=c, fill=c] (16.3284,4.84258) rectangle (16.3682,4.94842);
\draw [color=c, fill=c] (16.3682,4.84258) rectangle (16.408,4.94842);
\draw [color=c, fill=c] (16.408,4.84258) rectangle (16.4478,4.94842);
\draw [color=c, fill=c] (16.4478,4.84258) rectangle (16.4876,4.94842);
\draw [color=c, fill=c] (16.4876,4.84258) rectangle (16.5274,4.94842);
\draw [color=c, fill=c] (16.5274,4.84258) rectangle (16.5672,4.94842);
\draw [color=c, fill=c] (16.5672,4.84258) rectangle (16.607,4.94842);
\draw [color=c, fill=c] (16.607,4.84258) rectangle (16.6468,4.94842);
\draw [color=c, fill=c] (16.6468,4.84258) rectangle (16.6866,4.94842);
\draw [color=c, fill=c] (16.6866,4.84258) rectangle (16.7264,4.94842);
\draw [color=c, fill=c] (16.7264,4.84258) rectangle (16.7662,4.94842);
\draw [color=c, fill=c] (16.7662,4.84258) rectangle (16.806,4.94842);
\draw [color=c, fill=c] (16.806,4.84258) rectangle (16.8458,4.94842);
\draw [color=c, fill=c] (16.8458,4.84258) rectangle (16.8856,4.94842);
\draw [color=c, fill=c] (16.8856,4.84258) rectangle (16.9254,4.94842);
\draw [color=c, fill=c] (16.9254,4.84258) rectangle (16.9652,4.94842);
\draw [color=c, fill=c] (16.9652,4.84258) rectangle (17.005,4.94842);
\draw [color=c, fill=c] (17.005,4.84258) rectangle (17.0448,4.94842);
\draw [color=c, fill=c] (17.0448,4.84258) rectangle (17.0846,4.94842);
\draw [color=c, fill=c] (17.0846,4.84258) rectangle (17.1244,4.94842);
\draw [color=c, fill=c] (17.1244,4.84258) rectangle (17.1642,4.94842);
\draw [color=c, fill=c] (17.1642,4.84258) rectangle (17.204,4.94842);
\draw [color=c, fill=c] (17.204,4.84258) rectangle (17.2438,4.94842);
\draw [color=c, fill=c] (17.2438,4.84258) rectangle (17.2836,4.94842);
\draw [color=c, fill=c] (17.2836,4.84258) rectangle (17.3234,4.94842);
\draw [color=c, fill=c] (17.3234,4.84258) rectangle (17.3632,4.94842);
\draw [color=c, fill=c] (17.3632,4.84258) rectangle (17.403,4.94842);
\draw [color=c, fill=c] (17.403,4.84258) rectangle (17.4428,4.94842);
\draw [color=c, fill=c] (17.4428,4.84258) rectangle (17.4826,4.94842);
\draw [color=c, fill=c] (17.4826,4.84258) rectangle (17.5224,4.94842);
\draw [color=c, fill=c] (17.5224,4.84258) rectangle (17.5622,4.94842);
\draw [color=c, fill=c] (17.5622,4.84258) rectangle (17.602,4.94842);
\draw [color=c, fill=c] (17.602,4.84258) rectangle (17.6418,4.94842);
\draw [color=c, fill=c] (17.6418,4.84258) rectangle (17.6816,4.94842);
\draw [color=c, fill=c] (17.6816,4.84258) rectangle (17.7214,4.94842);
\draw [color=c, fill=c] (17.7214,4.84258) rectangle (17.7612,4.94842);
\draw [color=c, fill=c] (17.7612,4.84258) rectangle (17.801,4.94842);
\draw [color=c, fill=c] (17.801,4.84258) rectangle (17.8408,4.94842);
\draw [color=c, fill=c] (17.8408,4.84258) rectangle (17.8806,4.94842);
\draw [color=c, fill=c] (17.8806,4.84258) rectangle (17.9204,4.94842);
\draw [color=c, fill=c] (17.9204,4.84258) rectangle (17.9602,4.94842);
\draw [color=c, fill=c] (17.9602,4.84258) rectangle (18,4.94842);
\definecolor{c}{rgb}{0,0.0800001,1};
\draw [color=c, fill=c] (2,4.94842) rectangle (2.0398,5.05427);
\draw [color=c, fill=c] (2.0398,4.94842) rectangle (2.0796,5.05427);
\draw [color=c, fill=c] (2.0796,4.94842) rectangle (2.1194,5.05427);
\draw [color=c, fill=c] (2.1194,4.94842) rectangle (2.1592,5.05427);
\draw [color=c, fill=c] (2.1592,4.94842) rectangle (2.19901,5.05427);
\draw [color=c, fill=c] (2.19901,4.94842) rectangle (2.23881,5.05427);
\draw [color=c, fill=c] (2.23881,4.94842) rectangle (2.27861,5.05427);
\draw [color=c, fill=c] (2.27861,4.94842) rectangle (2.31841,5.05427);
\draw [color=c, fill=c] (2.31841,4.94842) rectangle (2.35821,5.05427);
\draw [color=c, fill=c] (2.35821,4.94842) rectangle (2.39801,5.05427);
\draw [color=c, fill=c] (2.39801,4.94842) rectangle (2.43781,5.05427);
\draw [color=c, fill=c] (2.43781,4.94842) rectangle (2.47761,5.05427);
\draw [color=c, fill=c] (2.47761,4.94842) rectangle (2.51741,5.05427);
\draw [color=c, fill=c] (2.51741,4.94842) rectangle (2.55721,5.05427);
\draw [color=c, fill=c] (2.55721,4.94842) rectangle (2.59702,5.05427);
\draw [color=c, fill=c] (2.59702,4.94842) rectangle (2.63682,5.05427);
\draw [color=c, fill=c] (2.63682,4.94842) rectangle (2.67662,5.05427);
\draw [color=c, fill=c] (2.67662,4.94842) rectangle (2.71642,5.05427);
\draw [color=c, fill=c] (2.71642,4.94842) rectangle (2.75622,5.05427);
\draw [color=c, fill=c] (2.75622,4.94842) rectangle (2.79602,5.05427);
\draw [color=c, fill=c] (2.79602,4.94842) rectangle (2.83582,5.05427);
\draw [color=c, fill=c] (2.83582,4.94842) rectangle (2.87562,5.05427);
\draw [color=c, fill=c] (2.87562,4.94842) rectangle (2.91542,5.05427);
\draw [color=c, fill=c] (2.91542,4.94842) rectangle (2.95522,5.05427);
\draw [color=c, fill=c] (2.95522,4.94842) rectangle (2.99502,5.05427);
\draw [color=c, fill=c] (2.99502,4.94842) rectangle (3.03483,5.05427);
\draw [color=c, fill=c] (3.03483,4.94842) rectangle (3.07463,5.05427);
\draw [color=c, fill=c] (3.07463,4.94842) rectangle (3.11443,5.05427);
\draw [color=c, fill=c] (3.11443,4.94842) rectangle (3.15423,5.05427);
\draw [color=c, fill=c] (3.15423,4.94842) rectangle (3.19403,5.05427);
\draw [color=c, fill=c] (3.19403,4.94842) rectangle (3.23383,5.05427);
\draw [color=c, fill=c] (3.23383,4.94842) rectangle (3.27363,5.05427);
\draw [color=c, fill=c] (3.27363,4.94842) rectangle (3.31343,5.05427);
\draw [color=c, fill=c] (3.31343,4.94842) rectangle (3.35323,5.05427);
\draw [color=c, fill=c] (3.35323,4.94842) rectangle (3.39303,5.05427);
\draw [color=c, fill=c] (3.39303,4.94842) rectangle (3.43284,5.05427);
\draw [color=c, fill=c] (3.43284,4.94842) rectangle (3.47264,5.05427);
\draw [color=c, fill=c] (3.47264,4.94842) rectangle (3.51244,5.05427);
\draw [color=c, fill=c] (3.51244,4.94842) rectangle (3.55224,5.05427);
\draw [color=c, fill=c] (3.55224,4.94842) rectangle (3.59204,5.05427);
\draw [color=c, fill=c] (3.59204,4.94842) rectangle (3.63184,5.05427);
\draw [color=c, fill=c] (3.63184,4.94842) rectangle (3.67164,5.05427);
\draw [color=c, fill=c] (3.67164,4.94842) rectangle (3.71144,5.05427);
\draw [color=c, fill=c] (3.71144,4.94842) rectangle (3.75124,5.05427);
\draw [color=c, fill=c] (3.75124,4.94842) rectangle (3.79104,5.05427);
\draw [color=c, fill=c] (3.79104,4.94842) rectangle (3.83085,5.05427);
\draw [color=c, fill=c] (3.83085,4.94842) rectangle (3.87065,5.05427);
\draw [color=c, fill=c] (3.87065,4.94842) rectangle (3.91045,5.05427);
\draw [color=c, fill=c] (3.91045,4.94842) rectangle (3.95025,5.05427);
\draw [color=c, fill=c] (3.95025,4.94842) rectangle (3.99005,5.05427);
\draw [color=c, fill=c] (3.99005,4.94842) rectangle (4.02985,5.05427);
\draw [color=c, fill=c] (4.02985,4.94842) rectangle (4.06965,5.05427);
\draw [color=c, fill=c] (4.06965,4.94842) rectangle (4.10945,5.05427);
\draw [color=c, fill=c] (4.10945,4.94842) rectangle (4.14925,5.05427);
\draw [color=c, fill=c] (4.14925,4.94842) rectangle (4.18905,5.05427);
\draw [color=c, fill=c] (4.18905,4.94842) rectangle (4.22886,5.05427);
\draw [color=c, fill=c] (4.22886,4.94842) rectangle (4.26866,5.05427);
\draw [color=c, fill=c] (4.26866,4.94842) rectangle (4.30846,5.05427);
\draw [color=c, fill=c] (4.30846,4.94842) rectangle (4.34826,5.05427);
\draw [color=c, fill=c] (4.34826,4.94842) rectangle (4.38806,5.05427);
\draw [color=c, fill=c] (4.38806,4.94842) rectangle (4.42786,5.05427);
\draw [color=c, fill=c] (4.42786,4.94842) rectangle (4.46766,5.05427);
\draw [color=c, fill=c] (4.46766,4.94842) rectangle (4.50746,5.05427);
\draw [color=c, fill=c] (4.50746,4.94842) rectangle (4.54726,5.05427);
\draw [color=c, fill=c] (4.54726,4.94842) rectangle (4.58706,5.05427);
\draw [color=c, fill=c] (4.58706,4.94842) rectangle (4.62687,5.05427);
\draw [color=c, fill=c] (4.62687,4.94842) rectangle (4.66667,5.05427);
\draw [color=c, fill=c] (4.66667,4.94842) rectangle (4.70647,5.05427);
\draw [color=c, fill=c] (4.70647,4.94842) rectangle (4.74627,5.05427);
\draw [color=c, fill=c] (4.74627,4.94842) rectangle (4.78607,5.05427);
\draw [color=c, fill=c] (4.78607,4.94842) rectangle (4.82587,5.05427);
\draw [color=c, fill=c] (4.82587,4.94842) rectangle (4.86567,5.05427);
\draw [color=c, fill=c] (4.86567,4.94842) rectangle (4.90547,5.05427);
\draw [color=c, fill=c] (4.90547,4.94842) rectangle (4.94527,5.05427);
\draw [color=c, fill=c] (4.94527,4.94842) rectangle (4.98507,5.05427);
\draw [color=c, fill=c] (4.98507,4.94842) rectangle (5.02488,5.05427);
\draw [color=c, fill=c] (5.02488,4.94842) rectangle (5.06468,5.05427);
\draw [color=c, fill=c] (5.06468,4.94842) rectangle (5.10448,5.05427);
\draw [color=c, fill=c] (5.10448,4.94842) rectangle (5.14428,5.05427);
\draw [color=c, fill=c] (5.14428,4.94842) rectangle (5.18408,5.05427);
\draw [color=c, fill=c] (5.18408,4.94842) rectangle (5.22388,5.05427);
\draw [color=c, fill=c] (5.22388,4.94842) rectangle (5.26368,5.05427);
\draw [color=c, fill=c] (5.26368,4.94842) rectangle (5.30348,5.05427);
\draw [color=c, fill=c] (5.30348,4.94842) rectangle (5.34328,5.05427);
\draw [color=c, fill=c] (5.34328,4.94842) rectangle (5.38308,5.05427);
\draw [color=c, fill=c] (5.38308,4.94842) rectangle (5.42289,5.05427);
\draw [color=c, fill=c] (5.42289,4.94842) rectangle (5.46269,5.05427);
\draw [color=c, fill=c] (5.46269,4.94842) rectangle (5.50249,5.05427);
\draw [color=c, fill=c] (5.50249,4.94842) rectangle (5.54229,5.05427);
\draw [color=c, fill=c] (5.54229,4.94842) rectangle (5.58209,5.05427);
\draw [color=c, fill=c] (5.58209,4.94842) rectangle (5.62189,5.05427);
\draw [color=c, fill=c] (5.62189,4.94842) rectangle (5.66169,5.05427);
\draw [color=c, fill=c] (5.66169,4.94842) rectangle (5.70149,5.05427);
\draw [color=c, fill=c] (5.70149,4.94842) rectangle (5.74129,5.05427);
\draw [color=c, fill=c] (5.74129,4.94842) rectangle (5.78109,5.05427);
\draw [color=c, fill=c] (5.78109,4.94842) rectangle (5.8209,5.05427);
\definecolor{c}{rgb}{0.2,0,1};
\draw [color=c, fill=c] (5.8209,4.94842) rectangle (5.8607,5.05427);
\draw [color=c, fill=c] (5.8607,4.94842) rectangle (5.9005,5.05427);
\draw [color=c, fill=c] (5.9005,4.94842) rectangle (5.9403,5.05427);
\draw [color=c, fill=c] (5.9403,4.94842) rectangle (5.9801,5.05427);
\draw [color=c, fill=c] (5.9801,4.94842) rectangle (6.0199,5.05427);
\draw [color=c, fill=c] (6.0199,4.94842) rectangle (6.0597,5.05427);
\draw [color=c, fill=c] (6.0597,4.94842) rectangle (6.0995,5.05427);
\draw [color=c, fill=c] (6.0995,4.94842) rectangle (6.1393,5.05427);
\draw [color=c, fill=c] (6.1393,4.94842) rectangle (6.1791,5.05427);
\draw [color=c, fill=c] (6.1791,4.94842) rectangle (6.21891,5.05427);
\draw [color=c, fill=c] (6.21891,4.94842) rectangle (6.25871,5.05427);
\draw [color=c, fill=c] (6.25871,4.94842) rectangle (6.29851,5.05427);
\draw [color=c, fill=c] (6.29851,4.94842) rectangle (6.33831,5.05427);
\draw [color=c, fill=c] (6.33831,4.94842) rectangle (6.37811,5.05427);
\draw [color=c, fill=c] (6.37811,4.94842) rectangle (6.41791,5.05427);
\draw [color=c, fill=c] (6.41791,4.94842) rectangle (6.45771,5.05427);
\draw [color=c, fill=c] (6.45771,4.94842) rectangle (6.49751,5.05427);
\draw [color=c, fill=c] (6.49751,4.94842) rectangle (6.53731,5.05427);
\draw [color=c, fill=c] (6.53731,4.94842) rectangle (6.57711,5.05427);
\draw [color=c, fill=c] (6.57711,4.94842) rectangle (6.61692,5.05427);
\draw [color=c, fill=c] (6.61692,4.94842) rectangle (6.65672,5.05427);
\draw [color=c, fill=c] (6.65672,4.94842) rectangle (6.69652,5.05427);
\draw [color=c, fill=c] (6.69652,4.94842) rectangle (6.73632,5.05427);
\draw [color=c, fill=c] (6.73632,4.94842) rectangle (6.77612,5.05427);
\draw [color=c, fill=c] (6.77612,4.94842) rectangle (6.81592,5.05427);
\draw [color=c, fill=c] (6.81592,4.94842) rectangle (6.85572,5.05427);
\draw [color=c, fill=c] (6.85572,4.94842) rectangle (6.89552,5.05427);
\draw [color=c, fill=c] (6.89552,4.94842) rectangle (6.93532,5.05427);
\draw [color=c, fill=c] (6.93532,4.94842) rectangle (6.97512,5.05427);
\draw [color=c, fill=c] (6.97512,4.94842) rectangle (7.01493,5.05427);
\draw [color=c, fill=c] (7.01493,4.94842) rectangle (7.05473,5.05427);
\draw [color=c, fill=c] (7.05473,4.94842) rectangle (7.09453,5.05427);
\draw [color=c, fill=c] (7.09453,4.94842) rectangle (7.13433,5.05427);
\draw [color=c, fill=c] (7.13433,4.94842) rectangle (7.17413,5.05427);
\draw [color=c, fill=c] (7.17413,4.94842) rectangle (7.21393,5.05427);
\draw [color=c, fill=c] (7.21393,4.94842) rectangle (7.25373,5.05427);
\draw [color=c, fill=c] (7.25373,4.94842) rectangle (7.29353,5.05427);
\draw [color=c, fill=c] (7.29353,4.94842) rectangle (7.33333,5.05427);
\draw [color=c, fill=c] (7.33333,4.94842) rectangle (7.37313,5.05427);
\draw [color=c, fill=c] (7.37313,4.94842) rectangle (7.41294,5.05427);
\draw [color=c, fill=c] (7.41294,4.94842) rectangle (7.45274,5.05427);
\draw [color=c, fill=c] (7.45274,4.94842) rectangle (7.49254,5.05427);
\draw [color=c, fill=c] (7.49254,4.94842) rectangle (7.53234,5.05427);
\draw [color=c, fill=c] (7.53234,4.94842) rectangle (7.57214,5.05427);
\draw [color=c, fill=c] (7.57214,4.94842) rectangle (7.61194,5.05427);
\draw [color=c, fill=c] (7.61194,4.94842) rectangle (7.65174,5.05427);
\draw [color=c, fill=c] (7.65174,4.94842) rectangle (7.69154,5.05427);
\draw [color=c, fill=c] (7.69154,4.94842) rectangle (7.73134,5.05427);
\draw [color=c, fill=c] (7.73134,4.94842) rectangle (7.77114,5.05427);
\draw [color=c, fill=c] (7.77114,4.94842) rectangle (7.81095,5.05427);
\draw [color=c, fill=c] (7.81095,4.94842) rectangle (7.85075,5.05427);
\draw [color=c, fill=c] (7.85075,4.94842) rectangle (7.89055,5.05427);
\draw [color=c, fill=c] (7.89055,4.94842) rectangle (7.93035,5.05427);
\draw [color=c, fill=c] (7.93035,4.94842) rectangle (7.97015,5.05427);
\draw [color=c, fill=c] (7.97015,4.94842) rectangle (8.00995,5.05427);
\draw [color=c, fill=c] (8.00995,4.94842) rectangle (8.04975,5.05427);
\draw [color=c, fill=c] (8.04975,4.94842) rectangle (8.08955,5.05427);
\draw [color=c, fill=c] (8.08955,4.94842) rectangle (8.12935,5.05427);
\draw [color=c, fill=c] (8.12935,4.94842) rectangle (8.16915,5.05427);
\draw [color=c, fill=c] (8.16915,4.94842) rectangle (8.20895,5.05427);
\draw [color=c, fill=c] (8.20895,4.94842) rectangle (8.24876,5.05427);
\draw [color=c, fill=c] (8.24876,4.94842) rectangle (8.28856,5.05427);
\draw [color=c, fill=c] (8.28856,4.94842) rectangle (8.32836,5.05427);
\draw [color=c, fill=c] (8.32836,4.94842) rectangle (8.36816,5.05427);
\draw [color=c, fill=c] (8.36816,4.94842) rectangle (8.40796,5.05427);
\draw [color=c, fill=c] (8.40796,4.94842) rectangle (8.44776,5.05427);
\draw [color=c, fill=c] (8.44776,4.94842) rectangle (8.48756,5.05427);
\draw [color=c, fill=c] (8.48756,4.94842) rectangle (8.52736,5.05427);
\draw [color=c, fill=c] (8.52736,4.94842) rectangle (8.56716,5.05427);
\draw [color=c, fill=c] (8.56716,4.94842) rectangle (8.60697,5.05427);
\draw [color=c, fill=c] (8.60697,4.94842) rectangle (8.64677,5.05427);
\draw [color=c, fill=c] (8.64677,4.94842) rectangle (8.68657,5.05427);
\draw [color=c, fill=c] (8.68657,4.94842) rectangle (8.72637,5.05427);
\draw [color=c, fill=c] (8.72637,4.94842) rectangle (8.76617,5.05427);
\draw [color=c, fill=c] (8.76617,4.94842) rectangle (8.80597,5.05427);
\draw [color=c, fill=c] (8.80597,4.94842) rectangle (8.84577,5.05427);
\draw [color=c, fill=c] (8.84577,4.94842) rectangle (8.88557,5.05427);
\draw [color=c, fill=c] (8.88557,4.94842) rectangle (8.92537,5.05427);
\draw [color=c, fill=c] (8.92537,4.94842) rectangle (8.96517,5.05427);
\draw [color=c, fill=c] (8.96517,4.94842) rectangle (9.00498,5.05427);
\draw [color=c, fill=c] (9.00498,4.94842) rectangle (9.04478,5.05427);
\draw [color=c, fill=c] (9.04478,4.94842) rectangle (9.08458,5.05427);
\draw [color=c, fill=c] (9.08458,4.94842) rectangle (9.12438,5.05427);
\draw [color=c, fill=c] (9.12438,4.94842) rectangle (9.16418,5.05427);
\draw [color=c, fill=c] (9.16418,4.94842) rectangle (9.20398,5.05427);
\draw [color=c, fill=c] (9.20398,4.94842) rectangle (9.24378,5.05427);
\draw [color=c, fill=c] (9.24378,4.94842) rectangle (9.28358,5.05427);
\draw [color=c, fill=c] (9.28358,4.94842) rectangle (9.32338,5.05427);
\draw [color=c, fill=c] (9.32338,4.94842) rectangle (9.36318,5.05427);
\draw [color=c, fill=c] (9.36318,4.94842) rectangle (9.40298,5.05427);
\draw [color=c, fill=c] (9.40298,4.94842) rectangle (9.44279,5.05427);
\draw [color=c, fill=c] (9.44279,4.94842) rectangle (9.48259,5.05427);
\draw [color=c, fill=c] (9.48259,4.94842) rectangle (9.52239,5.05427);
\draw [color=c, fill=c] (9.52239,4.94842) rectangle (9.56219,5.05427);
\definecolor{c}{rgb}{0,0.0800001,1};
\draw [color=c, fill=c] (9.56219,4.94842) rectangle (9.60199,5.05427);
\draw [color=c, fill=c] (9.60199,4.94842) rectangle (9.64179,5.05427);
\draw [color=c, fill=c] (9.64179,4.94842) rectangle (9.68159,5.05427);
\draw [color=c, fill=c] (9.68159,4.94842) rectangle (9.72139,5.05427);
\draw [color=c, fill=c] (9.72139,4.94842) rectangle (9.76119,5.05427);
\draw [color=c, fill=c] (9.76119,4.94842) rectangle (9.80099,5.05427);
\draw [color=c, fill=c] (9.80099,4.94842) rectangle (9.8408,5.05427);
\definecolor{c}{rgb}{0,0.266667,1};
\draw [color=c, fill=c] (9.8408,4.94842) rectangle (9.8806,5.05427);
\draw [color=c, fill=c] (9.8806,4.94842) rectangle (9.9204,5.05427);
\draw [color=c, fill=c] (9.9204,4.94842) rectangle (9.9602,5.05427);
\draw [color=c, fill=c] (9.9602,4.94842) rectangle (10,5.05427);
\draw [color=c, fill=c] (10,4.94842) rectangle (10.0398,5.05427);
\definecolor{c}{rgb}{0,0.546666,1};
\draw [color=c, fill=c] (10.0398,4.94842) rectangle (10.0796,5.05427);
\draw [color=c, fill=c] (10.0796,4.94842) rectangle (10.1194,5.05427);
\draw [color=c, fill=c] (10.1194,4.94842) rectangle (10.1592,5.05427);
\draw [color=c, fill=c] (10.1592,4.94842) rectangle (10.199,5.05427);
\draw [color=c, fill=c] (10.199,4.94842) rectangle (10.2388,5.05427);
\draw [color=c, fill=c] (10.2388,4.94842) rectangle (10.2786,5.05427);
\definecolor{c}{rgb}{0,0.733333,1};
\draw [color=c, fill=c] (10.2786,4.94842) rectangle (10.3184,5.05427);
\draw [color=c, fill=c] (10.3184,4.94842) rectangle (10.3582,5.05427);
\draw [color=c, fill=c] (10.3582,4.94842) rectangle (10.398,5.05427);
\draw [color=c, fill=c] (10.398,4.94842) rectangle (10.4378,5.05427);
\draw [color=c, fill=c] (10.4378,4.94842) rectangle (10.4776,5.05427);
\draw [color=c, fill=c] (10.4776,4.94842) rectangle (10.5174,5.05427);
\draw [color=c, fill=c] (10.5174,4.94842) rectangle (10.5572,5.05427);
\draw [color=c, fill=c] (10.5572,4.94842) rectangle (10.597,5.05427);
\draw [color=c, fill=c] (10.597,4.94842) rectangle (10.6368,5.05427);
\draw [color=c, fill=c] (10.6368,4.94842) rectangle (10.6766,5.05427);
\draw [color=c, fill=c] (10.6766,4.94842) rectangle (10.7164,5.05427);
\draw [color=c, fill=c] (10.7164,4.94842) rectangle (10.7562,5.05427);
\draw [color=c, fill=c] (10.7562,4.94842) rectangle (10.796,5.05427);
\draw [color=c, fill=c] (10.796,4.94842) rectangle (10.8358,5.05427);
\draw [color=c, fill=c] (10.8358,4.94842) rectangle (10.8756,5.05427);
\draw [color=c, fill=c] (10.8756,4.94842) rectangle (10.9154,5.05427);
\draw [color=c, fill=c] (10.9154,4.94842) rectangle (10.9552,5.05427);
\draw [color=c, fill=c] (10.9552,4.94842) rectangle (10.995,5.05427);
\draw [color=c, fill=c] (10.995,4.94842) rectangle (11.0348,5.05427);
\draw [color=c, fill=c] (11.0348,4.94842) rectangle (11.0746,5.05427);
\draw [color=c, fill=c] (11.0746,4.94842) rectangle (11.1144,5.05427);
\draw [color=c, fill=c] (11.1144,4.94842) rectangle (11.1542,5.05427);
\draw [color=c, fill=c] (11.1542,4.94842) rectangle (11.194,5.05427);
\draw [color=c, fill=c] (11.194,4.94842) rectangle (11.2338,5.05427);
\draw [color=c, fill=c] (11.2338,4.94842) rectangle (11.2736,5.05427);
\draw [color=c, fill=c] (11.2736,4.94842) rectangle (11.3134,5.05427);
\draw [color=c, fill=c] (11.3134,4.94842) rectangle (11.3532,5.05427);
\draw [color=c, fill=c] (11.3532,4.94842) rectangle (11.393,5.05427);
\draw [color=c, fill=c] (11.393,4.94842) rectangle (11.4328,5.05427);
\draw [color=c, fill=c] (11.4328,4.94842) rectangle (11.4726,5.05427);
\draw [color=c, fill=c] (11.4726,4.94842) rectangle (11.5124,5.05427);
\draw [color=c, fill=c] (11.5124,4.94842) rectangle (11.5522,5.05427);
\draw [color=c, fill=c] (11.5522,4.94842) rectangle (11.592,5.05427);
\draw [color=c, fill=c] (11.592,4.94842) rectangle (11.6318,5.05427);
\draw [color=c, fill=c] (11.6318,4.94842) rectangle (11.6716,5.05427);
\draw [color=c, fill=c] (11.6716,4.94842) rectangle (11.7114,5.05427);
\draw [color=c, fill=c] (11.7114,4.94842) rectangle (11.7512,5.05427);
\draw [color=c, fill=c] (11.7512,4.94842) rectangle (11.791,5.05427);
\draw [color=c, fill=c] (11.791,4.94842) rectangle (11.8308,5.05427);
\draw [color=c, fill=c] (11.8308,4.94842) rectangle (11.8706,5.05427);
\draw [color=c, fill=c] (11.8706,4.94842) rectangle (11.9104,5.05427);
\draw [color=c, fill=c] (11.9104,4.94842) rectangle (11.9502,5.05427);
\draw [color=c, fill=c] (11.9502,4.94842) rectangle (11.99,5.05427);
\draw [color=c, fill=c] (11.99,4.94842) rectangle (12.0299,5.05427);
\draw [color=c, fill=c] (12.0299,4.94842) rectangle (12.0697,5.05427);
\draw [color=c, fill=c] (12.0697,4.94842) rectangle (12.1095,5.05427);
\draw [color=c, fill=c] (12.1095,4.94842) rectangle (12.1493,5.05427);
\draw [color=c, fill=c] (12.1493,4.94842) rectangle (12.1891,5.05427);
\draw [color=c, fill=c] (12.1891,4.94842) rectangle (12.2289,5.05427);
\draw [color=c, fill=c] (12.2289,4.94842) rectangle (12.2687,5.05427);
\draw [color=c, fill=c] (12.2687,4.94842) rectangle (12.3085,5.05427);
\draw [color=c, fill=c] (12.3085,4.94842) rectangle (12.3483,5.05427);
\draw [color=c, fill=c] (12.3483,4.94842) rectangle (12.3881,5.05427);
\draw [color=c, fill=c] (12.3881,4.94842) rectangle (12.4279,5.05427);
\draw [color=c, fill=c] (12.4279,4.94842) rectangle (12.4677,5.05427);
\draw [color=c, fill=c] (12.4677,4.94842) rectangle (12.5075,5.05427);
\draw [color=c, fill=c] (12.5075,4.94842) rectangle (12.5473,5.05427);
\draw [color=c, fill=c] (12.5473,4.94842) rectangle (12.5871,5.05427);
\draw [color=c, fill=c] (12.5871,4.94842) rectangle (12.6269,5.05427);
\draw [color=c, fill=c] (12.6269,4.94842) rectangle (12.6667,5.05427);
\draw [color=c, fill=c] (12.6667,4.94842) rectangle (12.7065,5.05427);
\draw [color=c, fill=c] (12.7065,4.94842) rectangle (12.7463,5.05427);
\draw [color=c, fill=c] (12.7463,4.94842) rectangle (12.7861,5.05427);
\draw [color=c, fill=c] (12.7861,4.94842) rectangle (12.8259,5.05427);
\draw [color=c, fill=c] (12.8259,4.94842) rectangle (12.8657,5.05427);
\draw [color=c, fill=c] (12.8657,4.94842) rectangle (12.9055,5.05427);
\draw [color=c, fill=c] (12.9055,4.94842) rectangle (12.9453,5.05427);
\draw [color=c, fill=c] (12.9453,4.94842) rectangle (12.9851,5.05427);
\draw [color=c, fill=c] (12.9851,4.94842) rectangle (13.0249,5.05427);
\draw [color=c, fill=c] (13.0249,4.94842) rectangle (13.0647,5.05427);
\draw [color=c, fill=c] (13.0647,4.94842) rectangle (13.1045,5.05427);
\draw [color=c, fill=c] (13.1045,4.94842) rectangle (13.1443,5.05427);
\draw [color=c, fill=c] (13.1443,4.94842) rectangle (13.1841,5.05427);
\draw [color=c, fill=c] (13.1841,4.94842) rectangle (13.2239,5.05427);
\draw [color=c, fill=c] (13.2239,4.94842) rectangle (13.2637,5.05427);
\draw [color=c, fill=c] (13.2637,4.94842) rectangle (13.3035,5.05427);
\draw [color=c, fill=c] (13.3035,4.94842) rectangle (13.3433,5.05427);
\draw [color=c, fill=c] (13.3433,4.94842) rectangle (13.3831,5.05427);
\draw [color=c, fill=c] (13.3831,4.94842) rectangle (13.4229,5.05427);
\draw [color=c, fill=c] (13.4229,4.94842) rectangle (13.4627,5.05427);
\draw [color=c, fill=c] (13.4627,4.94842) rectangle (13.5025,5.05427);
\draw [color=c, fill=c] (13.5025,4.94842) rectangle (13.5423,5.05427);
\draw [color=c, fill=c] (13.5423,4.94842) rectangle (13.5821,5.05427);
\draw [color=c, fill=c] (13.5821,4.94842) rectangle (13.6219,5.05427);
\draw [color=c, fill=c] (13.6219,4.94842) rectangle (13.6617,5.05427);
\draw [color=c, fill=c] (13.6617,4.94842) rectangle (13.7015,5.05427);
\draw [color=c, fill=c] (13.7015,4.94842) rectangle (13.7413,5.05427);
\draw [color=c, fill=c] (13.7413,4.94842) rectangle (13.7811,5.05427);
\draw [color=c, fill=c] (13.7811,4.94842) rectangle (13.8209,5.05427);
\draw [color=c, fill=c] (13.8209,4.94842) rectangle (13.8607,5.05427);
\draw [color=c, fill=c] (13.8607,4.94842) rectangle (13.9005,5.05427);
\draw [color=c, fill=c] (13.9005,4.94842) rectangle (13.9403,5.05427);
\draw [color=c, fill=c] (13.9403,4.94842) rectangle (13.9801,5.05427);
\draw [color=c, fill=c] (13.9801,4.94842) rectangle (14.0199,5.05427);
\draw [color=c, fill=c] (14.0199,4.94842) rectangle (14.0597,5.05427);
\draw [color=c, fill=c] (14.0597,4.94842) rectangle (14.0995,5.05427);
\draw [color=c, fill=c] (14.0995,4.94842) rectangle (14.1393,5.05427);
\draw [color=c, fill=c] (14.1393,4.94842) rectangle (14.1791,5.05427);
\draw [color=c, fill=c] (14.1791,4.94842) rectangle (14.2189,5.05427);
\draw [color=c, fill=c] (14.2189,4.94842) rectangle (14.2587,5.05427);
\draw [color=c, fill=c] (14.2587,4.94842) rectangle (14.2985,5.05427);
\draw [color=c, fill=c] (14.2985,4.94842) rectangle (14.3383,5.05427);
\draw [color=c, fill=c] (14.3383,4.94842) rectangle (14.3781,5.05427);
\draw [color=c, fill=c] (14.3781,4.94842) rectangle (14.4179,5.05427);
\draw [color=c, fill=c] (14.4179,4.94842) rectangle (14.4577,5.05427);
\draw [color=c, fill=c] (14.4577,4.94842) rectangle (14.4975,5.05427);
\draw [color=c, fill=c] (14.4975,4.94842) rectangle (14.5373,5.05427);
\draw [color=c, fill=c] (14.5373,4.94842) rectangle (14.5771,5.05427);
\draw [color=c, fill=c] (14.5771,4.94842) rectangle (14.6169,5.05427);
\draw [color=c, fill=c] (14.6169,4.94842) rectangle (14.6567,5.05427);
\draw [color=c, fill=c] (14.6567,4.94842) rectangle (14.6965,5.05427);
\draw [color=c, fill=c] (14.6965,4.94842) rectangle (14.7363,5.05427);
\draw [color=c, fill=c] (14.7363,4.94842) rectangle (14.7761,5.05427);
\draw [color=c, fill=c] (14.7761,4.94842) rectangle (14.8159,5.05427);
\draw [color=c, fill=c] (14.8159,4.94842) rectangle (14.8557,5.05427);
\draw [color=c, fill=c] (14.8557,4.94842) rectangle (14.8955,5.05427);
\draw [color=c, fill=c] (14.8955,4.94842) rectangle (14.9353,5.05427);
\draw [color=c, fill=c] (14.9353,4.94842) rectangle (14.9751,5.05427);
\draw [color=c, fill=c] (14.9751,4.94842) rectangle (15.0149,5.05427);
\draw [color=c, fill=c] (15.0149,4.94842) rectangle (15.0547,5.05427);
\draw [color=c, fill=c] (15.0547,4.94842) rectangle (15.0945,5.05427);
\draw [color=c, fill=c] (15.0945,4.94842) rectangle (15.1343,5.05427);
\draw [color=c, fill=c] (15.1343,4.94842) rectangle (15.1741,5.05427);
\draw [color=c, fill=c] (15.1741,4.94842) rectangle (15.2139,5.05427);
\draw [color=c, fill=c] (15.2139,4.94842) rectangle (15.2537,5.05427);
\draw [color=c, fill=c] (15.2537,4.94842) rectangle (15.2935,5.05427);
\draw [color=c, fill=c] (15.2935,4.94842) rectangle (15.3333,5.05427);
\draw [color=c, fill=c] (15.3333,4.94842) rectangle (15.3731,5.05427);
\draw [color=c, fill=c] (15.3731,4.94842) rectangle (15.4129,5.05427);
\draw [color=c, fill=c] (15.4129,4.94842) rectangle (15.4527,5.05427);
\draw [color=c, fill=c] (15.4527,4.94842) rectangle (15.4925,5.05427);
\draw [color=c, fill=c] (15.4925,4.94842) rectangle (15.5323,5.05427);
\draw [color=c, fill=c] (15.5323,4.94842) rectangle (15.5721,5.05427);
\draw [color=c, fill=c] (15.5721,4.94842) rectangle (15.6119,5.05427);
\draw [color=c, fill=c] (15.6119,4.94842) rectangle (15.6517,5.05427);
\draw [color=c, fill=c] (15.6517,4.94842) rectangle (15.6915,5.05427);
\draw [color=c, fill=c] (15.6915,4.94842) rectangle (15.7313,5.05427);
\draw [color=c, fill=c] (15.7313,4.94842) rectangle (15.7711,5.05427);
\draw [color=c, fill=c] (15.7711,4.94842) rectangle (15.8109,5.05427);
\draw [color=c, fill=c] (15.8109,4.94842) rectangle (15.8507,5.05427);
\draw [color=c, fill=c] (15.8507,4.94842) rectangle (15.8905,5.05427);
\draw [color=c, fill=c] (15.8905,4.94842) rectangle (15.9303,5.05427);
\draw [color=c, fill=c] (15.9303,4.94842) rectangle (15.9701,5.05427);
\draw [color=c, fill=c] (15.9701,4.94842) rectangle (16.01,5.05427);
\draw [color=c, fill=c] (16.01,4.94842) rectangle (16.0498,5.05427);
\draw [color=c, fill=c] (16.0498,4.94842) rectangle (16.0896,5.05427);
\draw [color=c, fill=c] (16.0896,4.94842) rectangle (16.1294,5.05427);
\draw [color=c, fill=c] (16.1294,4.94842) rectangle (16.1692,5.05427);
\draw [color=c, fill=c] (16.1692,4.94842) rectangle (16.209,5.05427);
\draw [color=c, fill=c] (16.209,4.94842) rectangle (16.2488,5.05427);
\draw [color=c, fill=c] (16.2488,4.94842) rectangle (16.2886,5.05427);
\draw [color=c, fill=c] (16.2886,4.94842) rectangle (16.3284,5.05427);
\draw [color=c, fill=c] (16.3284,4.94842) rectangle (16.3682,5.05427);
\draw [color=c, fill=c] (16.3682,4.94842) rectangle (16.408,5.05427);
\draw [color=c, fill=c] (16.408,4.94842) rectangle (16.4478,5.05427);
\draw [color=c, fill=c] (16.4478,4.94842) rectangle (16.4876,5.05427);
\draw [color=c, fill=c] (16.4876,4.94842) rectangle (16.5274,5.05427);
\draw [color=c, fill=c] (16.5274,4.94842) rectangle (16.5672,5.05427);
\draw [color=c, fill=c] (16.5672,4.94842) rectangle (16.607,5.05427);
\draw [color=c, fill=c] (16.607,4.94842) rectangle (16.6468,5.05427);
\draw [color=c, fill=c] (16.6468,4.94842) rectangle (16.6866,5.05427);
\draw [color=c, fill=c] (16.6866,4.94842) rectangle (16.7264,5.05427);
\draw [color=c, fill=c] (16.7264,4.94842) rectangle (16.7662,5.05427);
\draw [color=c, fill=c] (16.7662,4.94842) rectangle (16.806,5.05427);
\draw [color=c, fill=c] (16.806,4.94842) rectangle (16.8458,5.05427);
\draw [color=c, fill=c] (16.8458,4.94842) rectangle (16.8856,5.05427);
\draw [color=c, fill=c] (16.8856,4.94842) rectangle (16.9254,5.05427);
\draw [color=c, fill=c] (16.9254,4.94842) rectangle (16.9652,5.05427);
\draw [color=c, fill=c] (16.9652,4.94842) rectangle (17.005,5.05427);
\draw [color=c, fill=c] (17.005,4.94842) rectangle (17.0448,5.05427);
\draw [color=c, fill=c] (17.0448,4.94842) rectangle (17.0846,5.05427);
\draw [color=c, fill=c] (17.0846,4.94842) rectangle (17.1244,5.05427);
\draw [color=c, fill=c] (17.1244,4.94842) rectangle (17.1642,5.05427);
\draw [color=c, fill=c] (17.1642,4.94842) rectangle (17.204,5.05427);
\draw [color=c, fill=c] (17.204,4.94842) rectangle (17.2438,5.05427);
\draw [color=c, fill=c] (17.2438,4.94842) rectangle (17.2836,5.05427);
\draw [color=c, fill=c] (17.2836,4.94842) rectangle (17.3234,5.05427);
\draw [color=c, fill=c] (17.3234,4.94842) rectangle (17.3632,5.05427);
\draw [color=c, fill=c] (17.3632,4.94842) rectangle (17.403,5.05427);
\draw [color=c, fill=c] (17.403,4.94842) rectangle (17.4428,5.05427);
\draw [color=c, fill=c] (17.4428,4.94842) rectangle (17.4826,5.05427);
\draw [color=c, fill=c] (17.4826,4.94842) rectangle (17.5224,5.05427);
\draw [color=c, fill=c] (17.5224,4.94842) rectangle (17.5622,5.05427);
\draw [color=c, fill=c] (17.5622,4.94842) rectangle (17.602,5.05427);
\draw [color=c, fill=c] (17.602,4.94842) rectangle (17.6418,5.05427);
\draw [color=c, fill=c] (17.6418,4.94842) rectangle (17.6816,5.05427);
\draw [color=c, fill=c] (17.6816,4.94842) rectangle (17.7214,5.05427);
\draw [color=c, fill=c] (17.7214,4.94842) rectangle (17.7612,5.05427);
\draw [color=c, fill=c] (17.7612,4.94842) rectangle (17.801,5.05427);
\draw [color=c, fill=c] (17.801,4.94842) rectangle (17.8408,5.05427);
\draw [color=c, fill=c] (17.8408,4.94842) rectangle (17.8806,5.05427);
\draw [color=c, fill=c] (17.8806,4.94842) rectangle (17.9204,5.05427);
\draw [color=c, fill=c] (17.9204,4.94842) rectangle (17.9602,5.05427);
\draw [color=c, fill=c] (17.9602,4.94842) rectangle (18,5.05427);
\definecolor{c}{rgb}{0,0.0800001,1};
\draw [color=c, fill=c] (2,5.05427) rectangle (2.0398,5.16012);
\draw [color=c, fill=c] (2.0398,5.05427) rectangle (2.0796,5.16012);
\draw [color=c, fill=c] (2.0796,5.05427) rectangle (2.1194,5.16012);
\draw [color=c, fill=c] (2.1194,5.05427) rectangle (2.1592,5.16012);
\draw [color=c, fill=c] (2.1592,5.05427) rectangle (2.19901,5.16012);
\draw [color=c, fill=c] (2.19901,5.05427) rectangle (2.23881,5.16012);
\draw [color=c, fill=c] (2.23881,5.05427) rectangle (2.27861,5.16012);
\draw [color=c, fill=c] (2.27861,5.05427) rectangle (2.31841,5.16012);
\draw [color=c, fill=c] (2.31841,5.05427) rectangle (2.35821,5.16012);
\draw [color=c, fill=c] (2.35821,5.05427) rectangle (2.39801,5.16012);
\draw [color=c, fill=c] (2.39801,5.05427) rectangle (2.43781,5.16012);
\draw [color=c, fill=c] (2.43781,5.05427) rectangle (2.47761,5.16012);
\draw [color=c, fill=c] (2.47761,5.05427) rectangle (2.51741,5.16012);
\draw [color=c, fill=c] (2.51741,5.05427) rectangle (2.55721,5.16012);
\draw [color=c, fill=c] (2.55721,5.05427) rectangle (2.59702,5.16012);
\draw [color=c, fill=c] (2.59702,5.05427) rectangle (2.63682,5.16012);
\draw [color=c, fill=c] (2.63682,5.05427) rectangle (2.67662,5.16012);
\draw [color=c, fill=c] (2.67662,5.05427) rectangle (2.71642,5.16012);
\draw [color=c, fill=c] (2.71642,5.05427) rectangle (2.75622,5.16012);
\draw [color=c, fill=c] (2.75622,5.05427) rectangle (2.79602,5.16012);
\draw [color=c, fill=c] (2.79602,5.05427) rectangle (2.83582,5.16012);
\draw [color=c, fill=c] (2.83582,5.05427) rectangle (2.87562,5.16012);
\draw [color=c, fill=c] (2.87562,5.05427) rectangle (2.91542,5.16012);
\draw [color=c, fill=c] (2.91542,5.05427) rectangle (2.95522,5.16012);
\draw [color=c, fill=c] (2.95522,5.05427) rectangle (2.99502,5.16012);
\draw [color=c, fill=c] (2.99502,5.05427) rectangle (3.03483,5.16012);
\draw [color=c, fill=c] (3.03483,5.05427) rectangle (3.07463,5.16012);
\draw [color=c, fill=c] (3.07463,5.05427) rectangle (3.11443,5.16012);
\draw [color=c, fill=c] (3.11443,5.05427) rectangle (3.15423,5.16012);
\draw [color=c, fill=c] (3.15423,5.05427) rectangle (3.19403,5.16012);
\draw [color=c, fill=c] (3.19403,5.05427) rectangle (3.23383,5.16012);
\draw [color=c, fill=c] (3.23383,5.05427) rectangle (3.27363,5.16012);
\draw [color=c, fill=c] (3.27363,5.05427) rectangle (3.31343,5.16012);
\draw [color=c, fill=c] (3.31343,5.05427) rectangle (3.35323,5.16012);
\draw [color=c, fill=c] (3.35323,5.05427) rectangle (3.39303,5.16012);
\draw [color=c, fill=c] (3.39303,5.05427) rectangle (3.43284,5.16012);
\draw [color=c, fill=c] (3.43284,5.05427) rectangle (3.47264,5.16012);
\draw [color=c, fill=c] (3.47264,5.05427) rectangle (3.51244,5.16012);
\draw [color=c, fill=c] (3.51244,5.05427) rectangle (3.55224,5.16012);
\draw [color=c, fill=c] (3.55224,5.05427) rectangle (3.59204,5.16012);
\draw [color=c, fill=c] (3.59204,5.05427) rectangle (3.63184,5.16012);
\draw [color=c, fill=c] (3.63184,5.05427) rectangle (3.67164,5.16012);
\draw [color=c, fill=c] (3.67164,5.05427) rectangle (3.71144,5.16012);
\draw [color=c, fill=c] (3.71144,5.05427) rectangle (3.75124,5.16012);
\draw [color=c, fill=c] (3.75124,5.05427) rectangle (3.79104,5.16012);
\draw [color=c, fill=c] (3.79104,5.05427) rectangle (3.83085,5.16012);
\draw [color=c, fill=c] (3.83085,5.05427) rectangle (3.87065,5.16012);
\draw [color=c, fill=c] (3.87065,5.05427) rectangle (3.91045,5.16012);
\draw [color=c, fill=c] (3.91045,5.05427) rectangle (3.95025,5.16012);
\draw [color=c, fill=c] (3.95025,5.05427) rectangle (3.99005,5.16012);
\draw [color=c, fill=c] (3.99005,5.05427) rectangle (4.02985,5.16012);
\draw [color=c, fill=c] (4.02985,5.05427) rectangle (4.06965,5.16012);
\draw [color=c, fill=c] (4.06965,5.05427) rectangle (4.10945,5.16012);
\draw [color=c, fill=c] (4.10945,5.05427) rectangle (4.14925,5.16012);
\draw [color=c, fill=c] (4.14925,5.05427) rectangle (4.18905,5.16012);
\draw [color=c, fill=c] (4.18905,5.05427) rectangle (4.22886,5.16012);
\draw [color=c, fill=c] (4.22886,5.05427) rectangle (4.26866,5.16012);
\draw [color=c, fill=c] (4.26866,5.05427) rectangle (4.30846,5.16012);
\draw [color=c, fill=c] (4.30846,5.05427) rectangle (4.34826,5.16012);
\draw [color=c, fill=c] (4.34826,5.05427) rectangle (4.38806,5.16012);
\draw [color=c, fill=c] (4.38806,5.05427) rectangle (4.42786,5.16012);
\draw [color=c, fill=c] (4.42786,5.05427) rectangle (4.46766,5.16012);
\draw [color=c, fill=c] (4.46766,5.05427) rectangle (4.50746,5.16012);
\draw [color=c, fill=c] (4.50746,5.05427) rectangle (4.54726,5.16012);
\draw [color=c, fill=c] (4.54726,5.05427) rectangle (4.58706,5.16012);
\draw [color=c, fill=c] (4.58706,5.05427) rectangle (4.62687,5.16012);
\draw [color=c, fill=c] (4.62687,5.05427) rectangle (4.66667,5.16012);
\draw [color=c, fill=c] (4.66667,5.05427) rectangle (4.70647,5.16012);
\draw [color=c, fill=c] (4.70647,5.05427) rectangle (4.74627,5.16012);
\draw [color=c, fill=c] (4.74627,5.05427) rectangle (4.78607,5.16012);
\draw [color=c, fill=c] (4.78607,5.05427) rectangle (4.82587,5.16012);
\draw [color=c, fill=c] (4.82587,5.05427) rectangle (4.86567,5.16012);
\draw [color=c, fill=c] (4.86567,5.05427) rectangle (4.90547,5.16012);
\draw [color=c, fill=c] (4.90547,5.05427) rectangle (4.94527,5.16012);
\draw [color=c, fill=c] (4.94527,5.05427) rectangle (4.98507,5.16012);
\draw [color=c, fill=c] (4.98507,5.05427) rectangle (5.02488,5.16012);
\draw [color=c, fill=c] (5.02488,5.05427) rectangle (5.06468,5.16012);
\draw [color=c, fill=c] (5.06468,5.05427) rectangle (5.10448,5.16012);
\draw [color=c, fill=c] (5.10448,5.05427) rectangle (5.14428,5.16012);
\draw [color=c, fill=c] (5.14428,5.05427) rectangle (5.18408,5.16012);
\draw [color=c, fill=c] (5.18408,5.05427) rectangle (5.22388,5.16012);
\draw [color=c, fill=c] (5.22388,5.05427) rectangle (5.26368,5.16012);
\draw [color=c, fill=c] (5.26368,5.05427) rectangle (5.30348,5.16012);
\draw [color=c, fill=c] (5.30348,5.05427) rectangle (5.34328,5.16012);
\draw [color=c, fill=c] (5.34328,5.05427) rectangle (5.38308,5.16012);
\draw [color=c, fill=c] (5.38308,5.05427) rectangle (5.42289,5.16012);
\draw [color=c, fill=c] (5.42289,5.05427) rectangle (5.46269,5.16012);
\draw [color=c, fill=c] (5.46269,5.05427) rectangle (5.50249,5.16012);
\draw [color=c, fill=c] (5.50249,5.05427) rectangle (5.54229,5.16012);
\draw [color=c, fill=c] (5.54229,5.05427) rectangle (5.58209,5.16012);
\draw [color=c, fill=c] (5.58209,5.05427) rectangle (5.62189,5.16012);
\draw [color=c, fill=c] (5.62189,5.05427) rectangle (5.66169,5.16012);
\draw [color=c, fill=c] (5.66169,5.05427) rectangle (5.70149,5.16012);
\draw [color=c, fill=c] (5.70149,5.05427) rectangle (5.74129,5.16012);
\draw [color=c, fill=c] (5.74129,5.05427) rectangle (5.78109,5.16012);
\definecolor{c}{rgb}{0.2,0,1};
\draw [color=c, fill=c] (5.78109,5.05427) rectangle (5.8209,5.16012);
\draw [color=c, fill=c] (5.8209,5.05427) rectangle (5.8607,5.16012);
\draw [color=c, fill=c] (5.8607,5.05427) rectangle (5.9005,5.16012);
\draw [color=c, fill=c] (5.9005,5.05427) rectangle (5.9403,5.16012);
\draw [color=c, fill=c] (5.9403,5.05427) rectangle (5.9801,5.16012);
\draw [color=c, fill=c] (5.9801,5.05427) rectangle (6.0199,5.16012);
\draw [color=c, fill=c] (6.0199,5.05427) rectangle (6.0597,5.16012);
\draw [color=c, fill=c] (6.0597,5.05427) rectangle (6.0995,5.16012);
\draw [color=c, fill=c] (6.0995,5.05427) rectangle (6.1393,5.16012);
\draw [color=c, fill=c] (6.1393,5.05427) rectangle (6.1791,5.16012);
\draw [color=c, fill=c] (6.1791,5.05427) rectangle (6.21891,5.16012);
\draw [color=c, fill=c] (6.21891,5.05427) rectangle (6.25871,5.16012);
\draw [color=c, fill=c] (6.25871,5.05427) rectangle (6.29851,5.16012);
\draw [color=c, fill=c] (6.29851,5.05427) rectangle (6.33831,5.16012);
\draw [color=c, fill=c] (6.33831,5.05427) rectangle (6.37811,5.16012);
\draw [color=c, fill=c] (6.37811,5.05427) rectangle (6.41791,5.16012);
\draw [color=c, fill=c] (6.41791,5.05427) rectangle (6.45771,5.16012);
\draw [color=c, fill=c] (6.45771,5.05427) rectangle (6.49751,5.16012);
\draw [color=c, fill=c] (6.49751,5.05427) rectangle (6.53731,5.16012);
\draw [color=c, fill=c] (6.53731,5.05427) rectangle (6.57711,5.16012);
\draw [color=c, fill=c] (6.57711,5.05427) rectangle (6.61692,5.16012);
\draw [color=c, fill=c] (6.61692,5.05427) rectangle (6.65672,5.16012);
\draw [color=c, fill=c] (6.65672,5.05427) rectangle (6.69652,5.16012);
\draw [color=c, fill=c] (6.69652,5.05427) rectangle (6.73632,5.16012);
\draw [color=c, fill=c] (6.73632,5.05427) rectangle (6.77612,5.16012);
\draw [color=c, fill=c] (6.77612,5.05427) rectangle (6.81592,5.16012);
\draw [color=c, fill=c] (6.81592,5.05427) rectangle (6.85572,5.16012);
\draw [color=c, fill=c] (6.85572,5.05427) rectangle (6.89552,5.16012);
\draw [color=c, fill=c] (6.89552,5.05427) rectangle (6.93532,5.16012);
\draw [color=c, fill=c] (6.93532,5.05427) rectangle (6.97512,5.16012);
\draw [color=c, fill=c] (6.97512,5.05427) rectangle (7.01493,5.16012);
\draw [color=c, fill=c] (7.01493,5.05427) rectangle (7.05473,5.16012);
\draw [color=c, fill=c] (7.05473,5.05427) rectangle (7.09453,5.16012);
\draw [color=c, fill=c] (7.09453,5.05427) rectangle (7.13433,5.16012);
\draw [color=c, fill=c] (7.13433,5.05427) rectangle (7.17413,5.16012);
\draw [color=c, fill=c] (7.17413,5.05427) rectangle (7.21393,5.16012);
\draw [color=c, fill=c] (7.21393,5.05427) rectangle (7.25373,5.16012);
\draw [color=c, fill=c] (7.25373,5.05427) rectangle (7.29353,5.16012);
\draw [color=c, fill=c] (7.29353,5.05427) rectangle (7.33333,5.16012);
\draw [color=c, fill=c] (7.33333,5.05427) rectangle (7.37313,5.16012);
\draw [color=c, fill=c] (7.37313,5.05427) rectangle (7.41294,5.16012);
\draw [color=c, fill=c] (7.41294,5.05427) rectangle (7.45274,5.16012);
\draw [color=c, fill=c] (7.45274,5.05427) rectangle (7.49254,5.16012);
\draw [color=c, fill=c] (7.49254,5.05427) rectangle (7.53234,5.16012);
\draw [color=c, fill=c] (7.53234,5.05427) rectangle (7.57214,5.16012);
\draw [color=c, fill=c] (7.57214,5.05427) rectangle (7.61194,5.16012);
\draw [color=c, fill=c] (7.61194,5.05427) rectangle (7.65174,5.16012);
\draw [color=c, fill=c] (7.65174,5.05427) rectangle (7.69154,5.16012);
\draw [color=c, fill=c] (7.69154,5.05427) rectangle (7.73134,5.16012);
\draw [color=c, fill=c] (7.73134,5.05427) rectangle (7.77114,5.16012);
\draw [color=c, fill=c] (7.77114,5.05427) rectangle (7.81095,5.16012);
\draw [color=c, fill=c] (7.81095,5.05427) rectangle (7.85075,5.16012);
\draw [color=c, fill=c] (7.85075,5.05427) rectangle (7.89055,5.16012);
\draw [color=c, fill=c] (7.89055,5.05427) rectangle (7.93035,5.16012);
\draw [color=c, fill=c] (7.93035,5.05427) rectangle (7.97015,5.16012);
\draw [color=c, fill=c] (7.97015,5.05427) rectangle (8.00995,5.16012);
\draw [color=c, fill=c] (8.00995,5.05427) rectangle (8.04975,5.16012);
\draw [color=c, fill=c] (8.04975,5.05427) rectangle (8.08955,5.16012);
\draw [color=c, fill=c] (8.08955,5.05427) rectangle (8.12935,5.16012);
\draw [color=c, fill=c] (8.12935,5.05427) rectangle (8.16915,5.16012);
\draw [color=c, fill=c] (8.16915,5.05427) rectangle (8.20895,5.16012);
\draw [color=c, fill=c] (8.20895,5.05427) rectangle (8.24876,5.16012);
\draw [color=c, fill=c] (8.24876,5.05427) rectangle (8.28856,5.16012);
\draw [color=c, fill=c] (8.28856,5.05427) rectangle (8.32836,5.16012);
\draw [color=c, fill=c] (8.32836,5.05427) rectangle (8.36816,5.16012);
\draw [color=c, fill=c] (8.36816,5.05427) rectangle (8.40796,5.16012);
\draw [color=c, fill=c] (8.40796,5.05427) rectangle (8.44776,5.16012);
\draw [color=c, fill=c] (8.44776,5.05427) rectangle (8.48756,5.16012);
\draw [color=c, fill=c] (8.48756,5.05427) rectangle (8.52736,5.16012);
\draw [color=c, fill=c] (8.52736,5.05427) rectangle (8.56716,5.16012);
\draw [color=c, fill=c] (8.56716,5.05427) rectangle (8.60697,5.16012);
\draw [color=c, fill=c] (8.60697,5.05427) rectangle (8.64677,5.16012);
\draw [color=c, fill=c] (8.64677,5.05427) rectangle (8.68657,5.16012);
\draw [color=c, fill=c] (8.68657,5.05427) rectangle (8.72637,5.16012);
\draw [color=c, fill=c] (8.72637,5.05427) rectangle (8.76617,5.16012);
\draw [color=c, fill=c] (8.76617,5.05427) rectangle (8.80597,5.16012);
\draw [color=c, fill=c] (8.80597,5.05427) rectangle (8.84577,5.16012);
\draw [color=c, fill=c] (8.84577,5.05427) rectangle (8.88557,5.16012);
\draw [color=c, fill=c] (8.88557,5.05427) rectangle (8.92537,5.16012);
\draw [color=c, fill=c] (8.92537,5.05427) rectangle (8.96517,5.16012);
\draw [color=c, fill=c] (8.96517,5.05427) rectangle (9.00498,5.16012);
\draw [color=c, fill=c] (9.00498,5.05427) rectangle (9.04478,5.16012);
\draw [color=c, fill=c] (9.04478,5.05427) rectangle (9.08458,5.16012);
\draw [color=c, fill=c] (9.08458,5.05427) rectangle (9.12438,5.16012);
\draw [color=c, fill=c] (9.12438,5.05427) rectangle (9.16418,5.16012);
\draw [color=c, fill=c] (9.16418,5.05427) rectangle (9.20398,5.16012);
\draw [color=c, fill=c] (9.20398,5.05427) rectangle (9.24378,5.16012);
\draw [color=c, fill=c] (9.24378,5.05427) rectangle (9.28358,5.16012);
\draw [color=c, fill=c] (9.28358,5.05427) rectangle (9.32338,5.16012);
\draw [color=c, fill=c] (9.32338,5.05427) rectangle (9.36318,5.16012);
\draw [color=c, fill=c] (9.36318,5.05427) rectangle (9.40298,5.16012);
\draw [color=c, fill=c] (9.40298,5.05427) rectangle (9.44279,5.16012);
\draw [color=c, fill=c] (9.44279,5.05427) rectangle (9.48259,5.16012);
\draw [color=c, fill=c] (9.48259,5.05427) rectangle (9.52239,5.16012);
\definecolor{c}{rgb}{0,0.0800001,1};
\draw [color=c, fill=c] (9.52239,5.05427) rectangle (9.56219,5.16012);
\draw [color=c, fill=c] (9.56219,5.05427) rectangle (9.60199,5.16012);
\draw [color=c, fill=c] (9.60199,5.05427) rectangle (9.64179,5.16012);
\draw [color=c, fill=c] (9.64179,5.05427) rectangle (9.68159,5.16012);
\draw [color=c, fill=c] (9.68159,5.05427) rectangle (9.72139,5.16012);
\draw [color=c, fill=c] (9.72139,5.05427) rectangle (9.76119,5.16012);
\draw [color=c, fill=c] (9.76119,5.05427) rectangle (9.80099,5.16012);
\draw [color=c, fill=c] (9.80099,5.05427) rectangle (9.8408,5.16012);
\definecolor{c}{rgb}{0,0.266667,1};
\draw [color=c, fill=c] (9.8408,5.05427) rectangle (9.8806,5.16012);
\draw [color=c, fill=c] (9.8806,5.05427) rectangle (9.9204,5.16012);
\draw [color=c, fill=c] (9.9204,5.05427) rectangle (9.9602,5.16012);
\draw [color=c, fill=c] (9.9602,5.05427) rectangle (10,5.16012);
\draw [color=c, fill=c] (10,5.05427) rectangle (10.0398,5.16012);
\definecolor{c}{rgb}{0,0.546666,1};
\draw [color=c, fill=c] (10.0398,5.05427) rectangle (10.0796,5.16012);
\draw [color=c, fill=c] (10.0796,5.05427) rectangle (10.1194,5.16012);
\draw [color=c, fill=c] (10.1194,5.05427) rectangle (10.1592,5.16012);
\draw [color=c, fill=c] (10.1592,5.05427) rectangle (10.199,5.16012);
\draw [color=c, fill=c] (10.199,5.05427) rectangle (10.2388,5.16012);
\draw [color=c, fill=c] (10.2388,5.05427) rectangle (10.2786,5.16012);
\definecolor{c}{rgb}{0,0.733333,1};
\draw [color=c, fill=c] (10.2786,5.05427) rectangle (10.3184,5.16012);
\draw [color=c, fill=c] (10.3184,5.05427) rectangle (10.3582,5.16012);
\draw [color=c, fill=c] (10.3582,5.05427) rectangle (10.398,5.16012);
\draw [color=c, fill=c] (10.398,5.05427) rectangle (10.4378,5.16012);
\draw [color=c, fill=c] (10.4378,5.05427) rectangle (10.4776,5.16012);
\draw [color=c, fill=c] (10.4776,5.05427) rectangle (10.5174,5.16012);
\draw [color=c, fill=c] (10.5174,5.05427) rectangle (10.5572,5.16012);
\draw [color=c, fill=c] (10.5572,5.05427) rectangle (10.597,5.16012);
\draw [color=c, fill=c] (10.597,5.05427) rectangle (10.6368,5.16012);
\draw [color=c, fill=c] (10.6368,5.05427) rectangle (10.6766,5.16012);
\draw [color=c, fill=c] (10.6766,5.05427) rectangle (10.7164,5.16012);
\draw [color=c, fill=c] (10.7164,5.05427) rectangle (10.7562,5.16012);
\draw [color=c, fill=c] (10.7562,5.05427) rectangle (10.796,5.16012);
\draw [color=c, fill=c] (10.796,5.05427) rectangle (10.8358,5.16012);
\draw [color=c, fill=c] (10.8358,5.05427) rectangle (10.8756,5.16012);
\draw [color=c, fill=c] (10.8756,5.05427) rectangle (10.9154,5.16012);
\draw [color=c, fill=c] (10.9154,5.05427) rectangle (10.9552,5.16012);
\draw [color=c, fill=c] (10.9552,5.05427) rectangle (10.995,5.16012);
\draw [color=c, fill=c] (10.995,5.05427) rectangle (11.0348,5.16012);
\draw [color=c, fill=c] (11.0348,5.05427) rectangle (11.0746,5.16012);
\draw [color=c, fill=c] (11.0746,5.05427) rectangle (11.1144,5.16012);
\draw [color=c, fill=c] (11.1144,5.05427) rectangle (11.1542,5.16012);
\draw [color=c, fill=c] (11.1542,5.05427) rectangle (11.194,5.16012);
\draw [color=c, fill=c] (11.194,5.05427) rectangle (11.2338,5.16012);
\draw [color=c, fill=c] (11.2338,5.05427) rectangle (11.2736,5.16012);
\draw [color=c, fill=c] (11.2736,5.05427) rectangle (11.3134,5.16012);
\draw [color=c, fill=c] (11.3134,5.05427) rectangle (11.3532,5.16012);
\draw [color=c, fill=c] (11.3532,5.05427) rectangle (11.393,5.16012);
\draw [color=c, fill=c] (11.393,5.05427) rectangle (11.4328,5.16012);
\draw [color=c, fill=c] (11.4328,5.05427) rectangle (11.4726,5.16012);
\draw [color=c, fill=c] (11.4726,5.05427) rectangle (11.5124,5.16012);
\draw [color=c, fill=c] (11.5124,5.05427) rectangle (11.5522,5.16012);
\draw [color=c, fill=c] (11.5522,5.05427) rectangle (11.592,5.16012);
\draw [color=c, fill=c] (11.592,5.05427) rectangle (11.6318,5.16012);
\draw [color=c, fill=c] (11.6318,5.05427) rectangle (11.6716,5.16012);
\draw [color=c, fill=c] (11.6716,5.05427) rectangle (11.7114,5.16012);
\draw [color=c, fill=c] (11.7114,5.05427) rectangle (11.7512,5.16012);
\draw [color=c, fill=c] (11.7512,5.05427) rectangle (11.791,5.16012);
\draw [color=c, fill=c] (11.791,5.05427) rectangle (11.8308,5.16012);
\draw [color=c, fill=c] (11.8308,5.05427) rectangle (11.8706,5.16012);
\draw [color=c, fill=c] (11.8706,5.05427) rectangle (11.9104,5.16012);
\draw [color=c, fill=c] (11.9104,5.05427) rectangle (11.9502,5.16012);
\draw [color=c, fill=c] (11.9502,5.05427) rectangle (11.99,5.16012);
\draw [color=c, fill=c] (11.99,5.05427) rectangle (12.0299,5.16012);
\draw [color=c, fill=c] (12.0299,5.05427) rectangle (12.0697,5.16012);
\draw [color=c, fill=c] (12.0697,5.05427) rectangle (12.1095,5.16012);
\draw [color=c, fill=c] (12.1095,5.05427) rectangle (12.1493,5.16012);
\draw [color=c, fill=c] (12.1493,5.05427) rectangle (12.1891,5.16012);
\draw [color=c, fill=c] (12.1891,5.05427) rectangle (12.2289,5.16012);
\draw [color=c, fill=c] (12.2289,5.05427) rectangle (12.2687,5.16012);
\draw [color=c, fill=c] (12.2687,5.05427) rectangle (12.3085,5.16012);
\draw [color=c, fill=c] (12.3085,5.05427) rectangle (12.3483,5.16012);
\draw [color=c, fill=c] (12.3483,5.05427) rectangle (12.3881,5.16012);
\draw [color=c, fill=c] (12.3881,5.05427) rectangle (12.4279,5.16012);
\draw [color=c, fill=c] (12.4279,5.05427) rectangle (12.4677,5.16012);
\draw [color=c, fill=c] (12.4677,5.05427) rectangle (12.5075,5.16012);
\draw [color=c, fill=c] (12.5075,5.05427) rectangle (12.5473,5.16012);
\draw [color=c, fill=c] (12.5473,5.05427) rectangle (12.5871,5.16012);
\draw [color=c, fill=c] (12.5871,5.05427) rectangle (12.6269,5.16012);
\draw [color=c, fill=c] (12.6269,5.05427) rectangle (12.6667,5.16012);
\draw [color=c, fill=c] (12.6667,5.05427) rectangle (12.7065,5.16012);
\draw [color=c, fill=c] (12.7065,5.05427) rectangle (12.7463,5.16012);
\draw [color=c, fill=c] (12.7463,5.05427) rectangle (12.7861,5.16012);
\draw [color=c, fill=c] (12.7861,5.05427) rectangle (12.8259,5.16012);
\draw [color=c, fill=c] (12.8259,5.05427) rectangle (12.8657,5.16012);
\draw [color=c, fill=c] (12.8657,5.05427) rectangle (12.9055,5.16012);
\draw [color=c, fill=c] (12.9055,5.05427) rectangle (12.9453,5.16012);
\draw [color=c, fill=c] (12.9453,5.05427) rectangle (12.9851,5.16012);
\draw [color=c, fill=c] (12.9851,5.05427) rectangle (13.0249,5.16012);
\draw [color=c, fill=c] (13.0249,5.05427) rectangle (13.0647,5.16012);
\draw [color=c, fill=c] (13.0647,5.05427) rectangle (13.1045,5.16012);
\draw [color=c, fill=c] (13.1045,5.05427) rectangle (13.1443,5.16012);
\draw [color=c, fill=c] (13.1443,5.05427) rectangle (13.1841,5.16012);
\draw [color=c, fill=c] (13.1841,5.05427) rectangle (13.2239,5.16012);
\draw [color=c, fill=c] (13.2239,5.05427) rectangle (13.2637,5.16012);
\draw [color=c, fill=c] (13.2637,5.05427) rectangle (13.3035,5.16012);
\draw [color=c, fill=c] (13.3035,5.05427) rectangle (13.3433,5.16012);
\draw [color=c, fill=c] (13.3433,5.05427) rectangle (13.3831,5.16012);
\draw [color=c, fill=c] (13.3831,5.05427) rectangle (13.4229,5.16012);
\draw [color=c, fill=c] (13.4229,5.05427) rectangle (13.4627,5.16012);
\draw [color=c, fill=c] (13.4627,5.05427) rectangle (13.5025,5.16012);
\draw [color=c, fill=c] (13.5025,5.05427) rectangle (13.5423,5.16012);
\draw [color=c, fill=c] (13.5423,5.05427) rectangle (13.5821,5.16012);
\draw [color=c, fill=c] (13.5821,5.05427) rectangle (13.6219,5.16012);
\draw [color=c, fill=c] (13.6219,5.05427) rectangle (13.6617,5.16012);
\draw [color=c, fill=c] (13.6617,5.05427) rectangle (13.7015,5.16012);
\draw [color=c, fill=c] (13.7015,5.05427) rectangle (13.7413,5.16012);
\draw [color=c, fill=c] (13.7413,5.05427) rectangle (13.7811,5.16012);
\draw [color=c, fill=c] (13.7811,5.05427) rectangle (13.8209,5.16012);
\draw [color=c, fill=c] (13.8209,5.05427) rectangle (13.8607,5.16012);
\draw [color=c, fill=c] (13.8607,5.05427) rectangle (13.9005,5.16012);
\draw [color=c, fill=c] (13.9005,5.05427) rectangle (13.9403,5.16012);
\draw [color=c, fill=c] (13.9403,5.05427) rectangle (13.9801,5.16012);
\draw [color=c, fill=c] (13.9801,5.05427) rectangle (14.0199,5.16012);
\draw [color=c, fill=c] (14.0199,5.05427) rectangle (14.0597,5.16012);
\draw [color=c, fill=c] (14.0597,5.05427) rectangle (14.0995,5.16012);
\draw [color=c, fill=c] (14.0995,5.05427) rectangle (14.1393,5.16012);
\draw [color=c, fill=c] (14.1393,5.05427) rectangle (14.1791,5.16012);
\draw [color=c, fill=c] (14.1791,5.05427) rectangle (14.2189,5.16012);
\draw [color=c, fill=c] (14.2189,5.05427) rectangle (14.2587,5.16012);
\draw [color=c, fill=c] (14.2587,5.05427) rectangle (14.2985,5.16012);
\draw [color=c, fill=c] (14.2985,5.05427) rectangle (14.3383,5.16012);
\draw [color=c, fill=c] (14.3383,5.05427) rectangle (14.3781,5.16012);
\draw [color=c, fill=c] (14.3781,5.05427) rectangle (14.4179,5.16012);
\draw [color=c, fill=c] (14.4179,5.05427) rectangle (14.4577,5.16012);
\draw [color=c, fill=c] (14.4577,5.05427) rectangle (14.4975,5.16012);
\draw [color=c, fill=c] (14.4975,5.05427) rectangle (14.5373,5.16012);
\draw [color=c, fill=c] (14.5373,5.05427) rectangle (14.5771,5.16012);
\draw [color=c, fill=c] (14.5771,5.05427) rectangle (14.6169,5.16012);
\draw [color=c, fill=c] (14.6169,5.05427) rectangle (14.6567,5.16012);
\draw [color=c, fill=c] (14.6567,5.05427) rectangle (14.6965,5.16012);
\draw [color=c, fill=c] (14.6965,5.05427) rectangle (14.7363,5.16012);
\draw [color=c, fill=c] (14.7363,5.05427) rectangle (14.7761,5.16012);
\draw [color=c, fill=c] (14.7761,5.05427) rectangle (14.8159,5.16012);
\draw [color=c, fill=c] (14.8159,5.05427) rectangle (14.8557,5.16012);
\draw [color=c, fill=c] (14.8557,5.05427) rectangle (14.8955,5.16012);
\draw [color=c, fill=c] (14.8955,5.05427) rectangle (14.9353,5.16012);
\draw [color=c, fill=c] (14.9353,5.05427) rectangle (14.9751,5.16012);
\draw [color=c, fill=c] (14.9751,5.05427) rectangle (15.0149,5.16012);
\draw [color=c, fill=c] (15.0149,5.05427) rectangle (15.0547,5.16012);
\draw [color=c, fill=c] (15.0547,5.05427) rectangle (15.0945,5.16012);
\draw [color=c, fill=c] (15.0945,5.05427) rectangle (15.1343,5.16012);
\draw [color=c, fill=c] (15.1343,5.05427) rectangle (15.1741,5.16012);
\draw [color=c, fill=c] (15.1741,5.05427) rectangle (15.2139,5.16012);
\draw [color=c, fill=c] (15.2139,5.05427) rectangle (15.2537,5.16012);
\draw [color=c, fill=c] (15.2537,5.05427) rectangle (15.2935,5.16012);
\draw [color=c, fill=c] (15.2935,5.05427) rectangle (15.3333,5.16012);
\draw [color=c, fill=c] (15.3333,5.05427) rectangle (15.3731,5.16012);
\draw [color=c, fill=c] (15.3731,5.05427) rectangle (15.4129,5.16012);
\draw [color=c, fill=c] (15.4129,5.05427) rectangle (15.4527,5.16012);
\draw [color=c, fill=c] (15.4527,5.05427) rectangle (15.4925,5.16012);
\draw [color=c, fill=c] (15.4925,5.05427) rectangle (15.5323,5.16012);
\draw [color=c, fill=c] (15.5323,5.05427) rectangle (15.5721,5.16012);
\draw [color=c, fill=c] (15.5721,5.05427) rectangle (15.6119,5.16012);
\draw [color=c, fill=c] (15.6119,5.05427) rectangle (15.6517,5.16012);
\draw [color=c, fill=c] (15.6517,5.05427) rectangle (15.6915,5.16012);
\draw [color=c, fill=c] (15.6915,5.05427) rectangle (15.7313,5.16012);
\draw [color=c, fill=c] (15.7313,5.05427) rectangle (15.7711,5.16012);
\draw [color=c, fill=c] (15.7711,5.05427) rectangle (15.8109,5.16012);
\draw [color=c, fill=c] (15.8109,5.05427) rectangle (15.8507,5.16012);
\draw [color=c, fill=c] (15.8507,5.05427) rectangle (15.8905,5.16012);
\draw [color=c, fill=c] (15.8905,5.05427) rectangle (15.9303,5.16012);
\draw [color=c, fill=c] (15.9303,5.05427) rectangle (15.9701,5.16012);
\draw [color=c, fill=c] (15.9701,5.05427) rectangle (16.01,5.16012);
\draw [color=c, fill=c] (16.01,5.05427) rectangle (16.0498,5.16012);
\draw [color=c, fill=c] (16.0498,5.05427) rectangle (16.0896,5.16012);
\draw [color=c, fill=c] (16.0896,5.05427) rectangle (16.1294,5.16012);
\draw [color=c, fill=c] (16.1294,5.05427) rectangle (16.1692,5.16012);
\draw [color=c, fill=c] (16.1692,5.05427) rectangle (16.209,5.16012);
\draw [color=c, fill=c] (16.209,5.05427) rectangle (16.2488,5.16012);
\draw [color=c, fill=c] (16.2488,5.05427) rectangle (16.2886,5.16012);
\draw [color=c, fill=c] (16.2886,5.05427) rectangle (16.3284,5.16012);
\draw [color=c, fill=c] (16.3284,5.05427) rectangle (16.3682,5.16012);
\draw [color=c, fill=c] (16.3682,5.05427) rectangle (16.408,5.16012);
\draw [color=c, fill=c] (16.408,5.05427) rectangle (16.4478,5.16012);
\draw [color=c, fill=c] (16.4478,5.05427) rectangle (16.4876,5.16012);
\draw [color=c, fill=c] (16.4876,5.05427) rectangle (16.5274,5.16012);
\draw [color=c, fill=c] (16.5274,5.05427) rectangle (16.5672,5.16012);
\draw [color=c, fill=c] (16.5672,5.05427) rectangle (16.607,5.16012);
\draw [color=c, fill=c] (16.607,5.05427) rectangle (16.6468,5.16012);
\draw [color=c, fill=c] (16.6468,5.05427) rectangle (16.6866,5.16012);
\draw [color=c, fill=c] (16.6866,5.05427) rectangle (16.7264,5.16012);
\draw [color=c, fill=c] (16.7264,5.05427) rectangle (16.7662,5.16012);
\draw [color=c, fill=c] (16.7662,5.05427) rectangle (16.806,5.16012);
\draw [color=c, fill=c] (16.806,5.05427) rectangle (16.8458,5.16012);
\draw [color=c, fill=c] (16.8458,5.05427) rectangle (16.8856,5.16012);
\draw [color=c, fill=c] (16.8856,5.05427) rectangle (16.9254,5.16012);
\draw [color=c, fill=c] (16.9254,5.05427) rectangle (16.9652,5.16012);
\draw [color=c, fill=c] (16.9652,5.05427) rectangle (17.005,5.16012);
\draw [color=c, fill=c] (17.005,5.05427) rectangle (17.0448,5.16012);
\draw [color=c, fill=c] (17.0448,5.05427) rectangle (17.0846,5.16012);
\draw [color=c, fill=c] (17.0846,5.05427) rectangle (17.1244,5.16012);
\draw [color=c, fill=c] (17.1244,5.05427) rectangle (17.1642,5.16012);
\draw [color=c, fill=c] (17.1642,5.05427) rectangle (17.204,5.16012);
\draw [color=c, fill=c] (17.204,5.05427) rectangle (17.2438,5.16012);
\draw [color=c, fill=c] (17.2438,5.05427) rectangle (17.2836,5.16012);
\draw [color=c, fill=c] (17.2836,5.05427) rectangle (17.3234,5.16012);
\draw [color=c, fill=c] (17.3234,5.05427) rectangle (17.3632,5.16012);
\draw [color=c, fill=c] (17.3632,5.05427) rectangle (17.403,5.16012);
\draw [color=c, fill=c] (17.403,5.05427) rectangle (17.4428,5.16012);
\draw [color=c, fill=c] (17.4428,5.05427) rectangle (17.4826,5.16012);
\draw [color=c, fill=c] (17.4826,5.05427) rectangle (17.5224,5.16012);
\draw [color=c, fill=c] (17.5224,5.05427) rectangle (17.5622,5.16012);
\draw [color=c, fill=c] (17.5622,5.05427) rectangle (17.602,5.16012);
\draw [color=c, fill=c] (17.602,5.05427) rectangle (17.6418,5.16012);
\draw [color=c, fill=c] (17.6418,5.05427) rectangle (17.6816,5.16012);
\draw [color=c, fill=c] (17.6816,5.05427) rectangle (17.7214,5.16012);
\draw [color=c, fill=c] (17.7214,5.05427) rectangle (17.7612,5.16012);
\draw [color=c, fill=c] (17.7612,5.05427) rectangle (17.801,5.16012);
\draw [color=c, fill=c] (17.801,5.05427) rectangle (17.8408,5.16012);
\draw [color=c, fill=c] (17.8408,5.05427) rectangle (17.8806,5.16012);
\draw [color=c, fill=c] (17.8806,5.05427) rectangle (17.9204,5.16012);
\draw [color=c, fill=c] (17.9204,5.05427) rectangle (17.9602,5.16012);
\draw [color=c, fill=c] (17.9602,5.05427) rectangle (18,5.16012);
\definecolor{c}{rgb}{0,0.0800001,1};
\draw [color=c, fill=c] (2,5.16012) rectangle (2.0398,5.26597);
\draw [color=c, fill=c] (2.0398,5.16012) rectangle (2.0796,5.26597);
\draw [color=c, fill=c] (2.0796,5.16012) rectangle (2.1194,5.26597);
\draw [color=c, fill=c] (2.1194,5.16012) rectangle (2.1592,5.26597);
\draw [color=c, fill=c] (2.1592,5.16012) rectangle (2.19901,5.26597);
\draw [color=c, fill=c] (2.19901,5.16012) rectangle (2.23881,5.26597);
\draw [color=c, fill=c] (2.23881,5.16012) rectangle (2.27861,5.26597);
\draw [color=c, fill=c] (2.27861,5.16012) rectangle (2.31841,5.26597);
\draw [color=c, fill=c] (2.31841,5.16012) rectangle (2.35821,5.26597);
\draw [color=c, fill=c] (2.35821,5.16012) rectangle (2.39801,5.26597);
\draw [color=c, fill=c] (2.39801,5.16012) rectangle (2.43781,5.26597);
\draw [color=c, fill=c] (2.43781,5.16012) rectangle (2.47761,5.26597);
\draw [color=c, fill=c] (2.47761,5.16012) rectangle (2.51741,5.26597);
\draw [color=c, fill=c] (2.51741,5.16012) rectangle (2.55721,5.26597);
\draw [color=c, fill=c] (2.55721,5.16012) rectangle (2.59702,5.26597);
\draw [color=c, fill=c] (2.59702,5.16012) rectangle (2.63682,5.26597);
\draw [color=c, fill=c] (2.63682,5.16012) rectangle (2.67662,5.26597);
\draw [color=c, fill=c] (2.67662,5.16012) rectangle (2.71642,5.26597);
\draw [color=c, fill=c] (2.71642,5.16012) rectangle (2.75622,5.26597);
\draw [color=c, fill=c] (2.75622,5.16012) rectangle (2.79602,5.26597);
\draw [color=c, fill=c] (2.79602,5.16012) rectangle (2.83582,5.26597);
\draw [color=c, fill=c] (2.83582,5.16012) rectangle (2.87562,5.26597);
\draw [color=c, fill=c] (2.87562,5.16012) rectangle (2.91542,5.26597);
\draw [color=c, fill=c] (2.91542,5.16012) rectangle (2.95522,5.26597);
\draw [color=c, fill=c] (2.95522,5.16012) rectangle (2.99502,5.26597);
\draw [color=c, fill=c] (2.99502,5.16012) rectangle (3.03483,5.26597);
\draw [color=c, fill=c] (3.03483,5.16012) rectangle (3.07463,5.26597);
\draw [color=c, fill=c] (3.07463,5.16012) rectangle (3.11443,5.26597);
\draw [color=c, fill=c] (3.11443,5.16012) rectangle (3.15423,5.26597);
\draw [color=c, fill=c] (3.15423,5.16012) rectangle (3.19403,5.26597);
\draw [color=c, fill=c] (3.19403,5.16012) rectangle (3.23383,5.26597);
\draw [color=c, fill=c] (3.23383,5.16012) rectangle (3.27363,5.26597);
\draw [color=c, fill=c] (3.27363,5.16012) rectangle (3.31343,5.26597);
\draw [color=c, fill=c] (3.31343,5.16012) rectangle (3.35323,5.26597);
\draw [color=c, fill=c] (3.35323,5.16012) rectangle (3.39303,5.26597);
\draw [color=c, fill=c] (3.39303,5.16012) rectangle (3.43284,5.26597);
\draw [color=c, fill=c] (3.43284,5.16012) rectangle (3.47264,5.26597);
\draw [color=c, fill=c] (3.47264,5.16012) rectangle (3.51244,5.26597);
\draw [color=c, fill=c] (3.51244,5.16012) rectangle (3.55224,5.26597);
\draw [color=c, fill=c] (3.55224,5.16012) rectangle (3.59204,5.26597);
\draw [color=c, fill=c] (3.59204,5.16012) rectangle (3.63184,5.26597);
\draw [color=c, fill=c] (3.63184,5.16012) rectangle (3.67164,5.26597);
\draw [color=c, fill=c] (3.67164,5.16012) rectangle (3.71144,5.26597);
\draw [color=c, fill=c] (3.71144,5.16012) rectangle (3.75124,5.26597);
\draw [color=c, fill=c] (3.75124,5.16012) rectangle (3.79104,5.26597);
\draw [color=c, fill=c] (3.79104,5.16012) rectangle (3.83085,5.26597);
\draw [color=c, fill=c] (3.83085,5.16012) rectangle (3.87065,5.26597);
\draw [color=c, fill=c] (3.87065,5.16012) rectangle (3.91045,5.26597);
\draw [color=c, fill=c] (3.91045,5.16012) rectangle (3.95025,5.26597);
\draw [color=c, fill=c] (3.95025,5.16012) rectangle (3.99005,5.26597);
\draw [color=c, fill=c] (3.99005,5.16012) rectangle (4.02985,5.26597);
\draw [color=c, fill=c] (4.02985,5.16012) rectangle (4.06965,5.26597);
\draw [color=c, fill=c] (4.06965,5.16012) rectangle (4.10945,5.26597);
\draw [color=c, fill=c] (4.10945,5.16012) rectangle (4.14925,5.26597);
\draw [color=c, fill=c] (4.14925,5.16012) rectangle (4.18905,5.26597);
\draw [color=c, fill=c] (4.18905,5.16012) rectangle (4.22886,5.26597);
\draw [color=c, fill=c] (4.22886,5.16012) rectangle (4.26866,5.26597);
\draw [color=c, fill=c] (4.26866,5.16012) rectangle (4.30846,5.26597);
\draw [color=c, fill=c] (4.30846,5.16012) rectangle (4.34826,5.26597);
\draw [color=c, fill=c] (4.34826,5.16012) rectangle (4.38806,5.26597);
\draw [color=c, fill=c] (4.38806,5.16012) rectangle (4.42786,5.26597);
\draw [color=c, fill=c] (4.42786,5.16012) rectangle (4.46766,5.26597);
\draw [color=c, fill=c] (4.46766,5.16012) rectangle (4.50746,5.26597);
\draw [color=c, fill=c] (4.50746,5.16012) rectangle (4.54726,5.26597);
\draw [color=c, fill=c] (4.54726,5.16012) rectangle (4.58706,5.26597);
\draw [color=c, fill=c] (4.58706,5.16012) rectangle (4.62687,5.26597);
\draw [color=c, fill=c] (4.62687,5.16012) rectangle (4.66667,5.26597);
\draw [color=c, fill=c] (4.66667,5.16012) rectangle (4.70647,5.26597);
\draw [color=c, fill=c] (4.70647,5.16012) rectangle (4.74627,5.26597);
\draw [color=c, fill=c] (4.74627,5.16012) rectangle (4.78607,5.26597);
\draw [color=c, fill=c] (4.78607,5.16012) rectangle (4.82587,5.26597);
\draw [color=c, fill=c] (4.82587,5.16012) rectangle (4.86567,5.26597);
\draw [color=c, fill=c] (4.86567,5.16012) rectangle (4.90547,5.26597);
\draw [color=c, fill=c] (4.90547,5.16012) rectangle (4.94527,5.26597);
\draw [color=c, fill=c] (4.94527,5.16012) rectangle (4.98507,5.26597);
\draw [color=c, fill=c] (4.98507,5.16012) rectangle (5.02488,5.26597);
\draw [color=c, fill=c] (5.02488,5.16012) rectangle (5.06468,5.26597);
\draw [color=c, fill=c] (5.06468,5.16012) rectangle (5.10448,5.26597);
\draw [color=c, fill=c] (5.10448,5.16012) rectangle (5.14428,5.26597);
\draw [color=c, fill=c] (5.14428,5.16012) rectangle (5.18408,5.26597);
\draw [color=c, fill=c] (5.18408,5.16012) rectangle (5.22388,5.26597);
\draw [color=c, fill=c] (5.22388,5.16012) rectangle (5.26368,5.26597);
\draw [color=c, fill=c] (5.26368,5.16012) rectangle (5.30348,5.26597);
\draw [color=c, fill=c] (5.30348,5.16012) rectangle (5.34328,5.26597);
\draw [color=c, fill=c] (5.34328,5.16012) rectangle (5.38308,5.26597);
\draw [color=c, fill=c] (5.38308,5.16012) rectangle (5.42289,5.26597);
\draw [color=c, fill=c] (5.42289,5.16012) rectangle (5.46269,5.26597);
\draw [color=c, fill=c] (5.46269,5.16012) rectangle (5.50249,5.26597);
\draw [color=c, fill=c] (5.50249,5.16012) rectangle (5.54229,5.26597);
\draw [color=c, fill=c] (5.54229,5.16012) rectangle (5.58209,5.26597);
\draw [color=c, fill=c] (5.58209,5.16012) rectangle (5.62189,5.26597);
\draw [color=c, fill=c] (5.62189,5.16012) rectangle (5.66169,5.26597);
\draw [color=c, fill=c] (5.66169,5.16012) rectangle (5.70149,5.26597);
\draw [color=c, fill=c] (5.70149,5.16012) rectangle (5.74129,5.26597);
\draw [color=c, fill=c] (5.74129,5.16012) rectangle (5.78109,5.26597);
\definecolor{c}{rgb}{0.2,0,1};
\draw [color=c, fill=c] (5.78109,5.16012) rectangle (5.8209,5.26597);
\draw [color=c, fill=c] (5.8209,5.16012) rectangle (5.8607,5.26597);
\draw [color=c, fill=c] (5.8607,5.16012) rectangle (5.9005,5.26597);
\draw [color=c, fill=c] (5.9005,5.16012) rectangle (5.9403,5.26597);
\draw [color=c, fill=c] (5.9403,5.16012) rectangle (5.9801,5.26597);
\draw [color=c, fill=c] (5.9801,5.16012) rectangle (6.0199,5.26597);
\draw [color=c, fill=c] (6.0199,5.16012) rectangle (6.0597,5.26597);
\draw [color=c, fill=c] (6.0597,5.16012) rectangle (6.0995,5.26597);
\draw [color=c, fill=c] (6.0995,5.16012) rectangle (6.1393,5.26597);
\draw [color=c, fill=c] (6.1393,5.16012) rectangle (6.1791,5.26597);
\draw [color=c, fill=c] (6.1791,5.16012) rectangle (6.21891,5.26597);
\draw [color=c, fill=c] (6.21891,5.16012) rectangle (6.25871,5.26597);
\draw [color=c, fill=c] (6.25871,5.16012) rectangle (6.29851,5.26597);
\draw [color=c, fill=c] (6.29851,5.16012) rectangle (6.33831,5.26597);
\draw [color=c, fill=c] (6.33831,5.16012) rectangle (6.37811,5.26597);
\draw [color=c, fill=c] (6.37811,5.16012) rectangle (6.41791,5.26597);
\draw [color=c, fill=c] (6.41791,5.16012) rectangle (6.45771,5.26597);
\draw [color=c, fill=c] (6.45771,5.16012) rectangle (6.49751,5.26597);
\draw [color=c, fill=c] (6.49751,5.16012) rectangle (6.53731,5.26597);
\draw [color=c, fill=c] (6.53731,5.16012) rectangle (6.57711,5.26597);
\draw [color=c, fill=c] (6.57711,5.16012) rectangle (6.61692,5.26597);
\draw [color=c, fill=c] (6.61692,5.16012) rectangle (6.65672,5.26597);
\draw [color=c, fill=c] (6.65672,5.16012) rectangle (6.69652,5.26597);
\draw [color=c, fill=c] (6.69652,5.16012) rectangle (6.73632,5.26597);
\draw [color=c, fill=c] (6.73632,5.16012) rectangle (6.77612,5.26597);
\draw [color=c, fill=c] (6.77612,5.16012) rectangle (6.81592,5.26597);
\draw [color=c, fill=c] (6.81592,5.16012) rectangle (6.85572,5.26597);
\draw [color=c, fill=c] (6.85572,5.16012) rectangle (6.89552,5.26597);
\draw [color=c, fill=c] (6.89552,5.16012) rectangle (6.93532,5.26597);
\draw [color=c, fill=c] (6.93532,5.16012) rectangle (6.97512,5.26597);
\draw [color=c, fill=c] (6.97512,5.16012) rectangle (7.01493,5.26597);
\draw [color=c, fill=c] (7.01493,5.16012) rectangle (7.05473,5.26597);
\draw [color=c, fill=c] (7.05473,5.16012) rectangle (7.09453,5.26597);
\draw [color=c, fill=c] (7.09453,5.16012) rectangle (7.13433,5.26597);
\draw [color=c, fill=c] (7.13433,5.16012) rectangle (7.17413,5.26597);
\draw [color=c, fill=c] (7.17413,5.16012) rectangle (7.21393,5.26597);
\draw [color=c, fill=c] (7.21393,5.16012) rectangle (7.25373,5.26597);
\draw [color=c, fill=c] (7.25373,5.16012) rectangle (7.29353,5.26597);
\draw [color=c, fill=c] (7.29353,5.16012) rectangle (7.33333,5.26597);
\draw [color=c, fill=c] (7.33333,5.16012) rectangle (7.37313,5.26597);
\draw [color=c, fill=c] (7.37313,5.16012) rectangle (7.41294,5.26597);
\draw [color=c, fill=c] (7.41294,5.16012) rectangle (7.45274,5.26597);
\draw [color=c, fill=c] (7.45274,5.16012) rectangle (7.49254,5.26597);
\draw [color=c, fill=c] (7.49254,5.16012) rectangle (7.53234,5.26597);
\draw [color=c, fill=c] (7.53234,5.16012) rectangle (7.57214,5.26597);
\draw [color=c, fill=c] (7.57214,5.16012) rectangle (7.61194,5.26597);
\draw [color=c, fill=c] (7.61194,5.16012) rectangle (7.65174,5.26597);
\draw [color=c, fill=c] (7.65174,5.16012) rectangle (7.69154,5.26597);
\draw [color=c, fill=c] (7.69154,5.16012) rectangle (7.73134,5.26597);
\draw [color=c, fill=c] (7.73134,5.16012) rectangle (7.77114,5.26597);
\draw [color=c, fill=c] (7.77114,5.16012) rectangle (7.81095,5.26597);
\draw [color=c, fill=c] (7.81095,5.16012) rectangle (7.85075,5.26597);
\draw [color=c, fill=c] (7.85075,5.16012) rectangle (7.89055,5.26597);
\draw [color=c, fill=c] (7.89055,5.16012) rectangle (7.93035,5.26597);
\draw [color=c, fill=c] (7.93035,5.16012) rectangle (7.97015,5.26597);
\draw [color=c, fill=c] (7.97015,5.16012) rectangle (8.00995,5.26597);
\draw [color=c, fill=c] (8.00995,5.16012) rectangle (8.04975,5.26597);
\draw [color=c, fill=c] (8.04975,5.16012) rectangle (8.08955,5.26597);
\draw [color=c, fill=c] (8.08955,5.16012) rectangle (8.12935,5.26597);
\draw [color=c, fill=c] (8.12935,5.16012) rectangle (8.16915,5.26597);
\draw [color=c, fill=c] (8.16915,5.16012) rectangle (8.20895,5.26597);
\draw [color=c, fill=c] (8.20895,5.16012) rectangle (8.24876,5.26597);
\draw [color=c, fill=c] (8.24876,5.16012) rectangle (8.28856,5.26597);
\draw [color=c, fill=c] (8.28856,5.16012) rectangle (8.32836,5.26597);
\draw [color=c, fill=c] (8.32836,5.16012) rectangle (8.36816,5.26597);
\draw [color=c, fill=c] (8.36816,5.16012) rectangle (8.40796,5.26597);
\draw [color=c, fill=c] (8.40796,5.16012) rectangle (8.44776,5.26597);
\draw [color=c, fill=c] (8.44776,5.16012) rectangle (8.48756,5.26597);
\draw [color=c, fill=c] (8.48756,5.16012) rectangle (8.52736,5.26597);
\draw [color=c, fill=c] (8.52736,5.16012) rectangle (8.56716,5.26597);
\draw [color=c, fill=c] (8.56716,5.16012) rectangle (8.60697,5.26597);
\draw [color=c, fill=c] (8.60697,5.16012) rectangle (8.64677,5.26597);
\draw [color=c, fill=c] (8.64677,5.16012) rectangle (8.68657,5.26597);
\draw [color=c, fill=c] (8.68657,5.16012) rectangle (8.72637,5.26597);
\draw [color=c, fill=c] (8.72637,5.16012) rectangle (8.76617,5.26597);
\draw [color=c, fill=c] (8.76617,5.16012) rectangle (8.80597,5.26597);
\draw [color=c, fill=c] (8.80597,5.16012) rectangle (8.84577,5.26597);
\draw [color=c, fill=c] (8.84577,5.16012) rectangle (8.88557,5.26597);
\draw [color=c, fill=c] (8.88557,5.16012) rectangle (8.92537,5.26597);
\draw [color=c, fill=c] (8.92537,5.16012) rectangle (8.96517,5.26597);
\draw [color=c, fill=c] (8.96517,5.16012) rectangle (9.00498,5.26597);
\draw [color=c, fill=c] (9.00498,5.16012) rectangle (9.04478,5.26597);
\draw [color=c, fill=c] (9.04478,5.16012) rectangle (9.08458,5.26597);
\draw [color=c, fill=c] (9.08458,5.16012) rectangle (9.12438,5.26597);
\draw [color=c, fill=c] (9.12438,5.16012) rectangle (9.16418,5.26597);
\draw [color=c, fill=c] (9.16418,5.16012) rectangle (9.20398,5.26597);
\draw [color=c, fill=c] (9.20398,5.16012) rectangle (9.24378,5.26597);
\draw [color=c, fill=c] (9.24378,5.16012) rectangle (9.28358,5.26597);
\draw [color=c, fill=c] (9.28358,5.16012) rectangle (9.32338,5.26597);
\draw [color=c, fill=c] (9.32338,5.16012) rectangle (9.36318,5.26597);
\draw [color=c, fill=c] (9.36318,5.16012) rectangle (9.40298,5.26597);
\draw [color=c, fill=c] (9.40298,5.16012) rectangle (9.44279,5.26597);
\draw [color=c, fill=c] (9.44279,5.16012) rectangle (9.48259,5.26597);
\definecolor{c}{rgb}{0,0.0800001,1};
\draw [color=c, fill=c] (9.48259,5.16012) rectangle (9.52239,5.26597);
\draw [color=c, fill=c] (9.52239,5.16012) rectangle (9.56219,5.26597);
\draw [color=c, fill=c] (9.56219,5.16012) rectangle (9.60199,5.26597);
\draw [color=c, fill=c] (9.60199,5.16012) rectangle (9.64179,5.26597);
\draw [color=c, fill=c] (9.64179,5.16012) rectangle (9.68159,5.26597);
\draw [color=c, fill=c] (9.68159,5.16012) rectangle (9.72139,5.26597);
\draw [color=c, fill=c] (9.72139,5.16012) rectangle (9.76119,5.26597);
\draw [color=c, fill=c] (9.76119,5.16012) rectangle (9.80099,5.26597);
\definecolor{c}{rgb}{0,0.266667,1};
\draw [color=c, fill=c] (9.80099,5.16012) rectangle (9.8408,5.26597);
\draw [color=c, fill=c] (9.8408,5.16012) rectangle (9.8806,5.26597);
\draw [color=c, fill=c] (9.8806,5.16012) rectangle (9.9204,5.26597);
\draw [color=c, fill=c] (9.9204,5.16012) rectangle (9.9602,5.26597);
\draw [color=c, fill=c] (9.9602,5.16012) rectangle (10,5.26597);
\draw [color=c, fill=c] (10,5.16012) rectangle (10.0398,5.26597);
\definecolor{c}{rgb}{0,0.546666,1};
\draw [color=c, fill=c] (10.0398,5.16012) rectangle (10.0796,5.26597);
\draw [color=c, fill=c] (10.0796,5.16012) rectangle (10.1194,5.26597);
\draw [color=c, fill=c] (10.1194,5.16012) rectangle (10.1592,5.26597);
\draw [color=c, fill=c] (10.1592,5.16012) rectangle (10.199,5.26597);
\draw [color=c, fill=c] (10.199,5.16012) rectangle (10.2388,5.26597);
\draw [color=c, fill=c] (10.2388,5.16012) rectangle (10.2786,5.26597);
\draw [color=c, fill=c] (10.2786,5.16012) rectangle (10.3184,5.26597);
\definecolor{c}{rgb}{0,0.733333,1};
\draw [color=c, fill=c] (10.3184,5.16012) rectangle (10.3582,5.26597);
\draw [color=c, fill=c] (10.3582,5.16012) rectangle (10.398,5.26597);
\draw [color=c, fill=c] (10.398,5.16012) rectangle (10.4378,5.26597);
\draw [color=c, fill=c] (10.4378,5.16012) rectangle (10.4776,5.26597);
\draw [color=c, fill=c] (10.4776,5.16012) rectangle (10.5174,5.26597);
\draw [color=c, fill=c] (10.5174,5.16012) rectangle (10.5572,5.26597);
\draw [color=c, fill=c] (10.5572,5.16012) rectangle (10.597,5.26597);
\draw [color=c, fill=c] (10.597,5.16012) rectangle (10.6368,5.26597);
\draw [color=c, fill=c] (10.6368,5.16012) rectangle (10.6766,5.26597);
\draw [color=c, fill=c] (10.6766,5.16012) rectangle (10.7164,5.26597);
\draw [color=c, fill=c] (10.7164,5.16012) rectangle (10.7562,5.26597);
\draw [color=c, fill=c] (10.7562,5.16012) rectangle (10.796,5.26597);
\draw [color=c, fill=c] (10.796,5.16012) rectangle (10.8358,5.26597);
\draw [color=c, fill=c] (10.8358,5.16012) rectangle (10.8756,5.26597);
\draw [color=c, fill=c] (10.8756,5.16012) rectangle (10.9154,5.26597);
\draw [color=c, fill=c] (10.9154,5.16012) rectangle (10.9552,5.26597);
\draw [color=c, fill=c] (10.9552,5.16012) rectangle (10.995,5.26597);
\draw [color=c, fill=c] (10.995,5.16012) rectangle (11.0348,5.26597);
\draw [color=c, fill=c] (11.0348,5.16012) rectangle (11.0746,5.26597);
\draw [color=c, fill=c] (11.0746,5.16012) rectangle (11.1144,5.26597);
\draw [color=c, fill=c] (11.1144,5.16012) rectangle (11.1542,5.26597);
\draw [color=c, fill=c] (11.1542,5.16012) rectangle (11.194,5.26597);
\draw [color=c, fill=c] (11.194,5.16012) rectangle (11.2338,5.26597);
\draw [color=c, fill=c] (11.2338,5.16012) rectangle (11.2736,5.26597);
\draw [color=c, fill=c] (11.2736,5.16012) rectangle (11.3134,5.26597);
\draw [color=c, fill=c] (11.3134,5.16012) rectangle (11.3532,5.26597);
\draw [color=c, fill=c] (11.3532,5.16012) rectangle (11.393,5.26597);
\draw [color=c, fill=c] (11.393,5.16012) rectangle (11.4328,5.26597);
\draw [color=c, fill=c] (11.4328,5.16012) rectangle (11.4726,5.26597);
\draw [color=c, fill=c] (11.4726,5.16012) rectangle (11.5124,5.26597);
\draw [color=c, fill=c] (11.5124,5.16012) rectangle (11.5522,5.26597);
\draw [color=c, fill=c] (11.5522,5.16012) rectangle (11.592,5.26597);
\draw [color=c, fill=c] (11.592,5.16012) rectangle (11.6318,5.26597);
\draw [color=c, fill=c] (11.6318,5.16012) rectangle (11.6716,5.26597);
\draw [color=c, fill=c] (11.6716,5.16012) rectangle (11.7114,5.26597);
\draw [color=c, fill=c] (11.7114,5.16012) rectangle (11.7512,5.26597);
\draw [color=c, fill=c] (11.7512,5.16012) rectangle (11.791,5.26597);
\draw [color=c, fill=c] (11.791,5.16012) rectangle (11.8308,5.26597);
\draw [color=c, fill=c] (11.8308,5.16012) rectangle (11.8706,5.26597);
\draw [color=c, fill=c] (11.8706,5.16012) rectangle (11.9104,5.26597);
\draw [color=c, fill=c] (11.9104,5.16012) rectangle (11.9502,5.26597);
\draw [color=c, fill=c] (11.9502,5.16012) rectangle (11.99,5.26597);
\draw [color=c, fill=c] (11.99,5.16012) rectangle (12.0299,5.26597);
\draw [color=c, fill=c] (12.0299,5.16012) rectangle (12.0697,5.26597);
\draw [color=c, fill=c] (12.0697,5.16012) rectangle (12.1095,5.26597);
\draw [color=c, fill=c] (12.1095,5.16012) rectangle (12.1493,5.26597);
\draw [color=c, fill=c] (12.1493,5.16012) rectangle (12.1891,5.26597);
\draw [color=c, fill=c] (12.1891,5.16012) rectangle (12.2289,5.26597);
\draw [color=c, fill=c] (12.2289,5.16012) rectangle (12.2687,5.26597);
\draw [color=c, fill=c] (12.2687,5.16012) rectangle (12.3085,5.26597);
\draw [color=c, fill=c] (12.3085,5.16012) rectangle (12.3483,5.26597);
\draw [color=c, fill=c] (12.3483,5.16012) rectangle (12.3881,5.26597);
\draw [color=c, fill=c] (12.3881,5.16012) rectangle (12.4279,5.26597);
\draw [color=c, fill=c] (12.4279,5.16012) rectangle (12.4677,5.26597);
\draw [color=c, fill=c] (12.4677,5.16012) rectangle (12.5075,5.26597);
\draw [color=c, fill=c] (12.5075,5.16012) rectangle (12.5473,5.26597);
\draw [color=c, fill=c] (12.5473,5.16012) rectangle (12.5871,5.26597);
\draw [color=c, fill=c] (12.5871,5.16012) rectangle (12.6269,5.26597);
\draw [color=c, fill=c] (12.6269,5.16012) rectangle (12.6667,5.26597);
\draw [color=c, fill=c] (12.6667,5.16012) rectangle (12.7065,5.26597);
\draw [color=c, fill=c] (12.7065,5.16012) rectangle (12.7463,5.26597);
\draw [color=c, fill=c] (12.7463,5.16012) rectangle (12.7861,5.26597);
\draw [color=c, fill=c] (12.7861,5.16012) rectangle (12.8259,5.26597);
\draw [color=c, fill=c] (12.8259,5.16012) rectangle (12.8657,5.26597);
\draw [color=c, fill=c] (12.8657,5.16012) rectangle (12.9055,5.26597);
\draw [color=c, fill=c] (12.9055,5.16012) rectangle (12.9453,5.26597);
\draw [color=c, fill=c] (12.9453,5.16012) rectangle (12.9851,5.26597);
\draw [color=c, fill=c] (12.9851,5.16012) rectangle (13.0249,5.26597);
\draw [color=c, fill=c] (13.0249,5.16012) rectangle (13.0647,5.26597);
\draw [color=c, fill=c] (13.0647,5.16012) rectangle (13.1045,5.26597);
\draw [color=c, fill=c] (13.1045,5.16012) rectangle (13.1443,5.26597);
\draw [color=c, fill=c] (13.1443,5.16012) rectangle (13.1841,5.26597);
\draw [color=c, fill=c] (13.1841,5.16012) rectangle (13.2239,5.26597);
\draw [color=c, fill=c] (13.2239,5.16012) rectangle (13.2637,5.26597);
\draw [color=c, fill=c] (13.2637,5.16012) rectangle (13.3035,5.26597);
\draw [color=c, fill=c] (13.3035,5.16012) rectangle (13.3433,5.26597);
\draw [color=c, fill=c] (13.3433,5.16012) rectangle (13.3831,5.26597);
\draw [color=c, fill=c] (13.3831,5.16012) rectangle (13.4229,5.26597);
\draw [color=c, fill=c] (13.4229,5.16012) rectangle (13.4627,5.26597);
\draw [color=c, fill=c] (13.4627,5.16012) rectangle (13.5025,5.26597);
\draw [color=c, fill=c] (13.5025,5.16012) rectangle (13.5423,5.26597);
\draw [color=c, fill=c] (13.5423,5.16012) rectangle (13.5821,5.26597);
\draw [color=c, fill=c] (13.5821,5.16012) rectangle (13.6219,5.26597);
\draw [color=c, fill=c] (13.6219,5.16012) rectangle (13.6617,5.26597);
\draw [color=c, fill=c] (13.6617,5.16012) rectangle (13.7015,5.26597);
\draw [color=c, fill=c] (13.7015,5.16012) rectangle (13.7413,5.26597);
\draw [color=c, fill=c] (13.7413,5.16012) rectangle (13.7811,5.26597);
\draw [color=c, fill=c] (13.7811,5.16012) rectangle (13.8209,5.26597);
\draw [color=c, fill=c] (13.8209,5.16012) rectangle (13.8607,5.26597);
\draw [color=c, fill=c] (13.8607,5.16012) rectangle (13.9005,5.26597);
\draw [color=c, fill=c] (13.9005,5.16012) rectangle (13.9403,5.26597);
\draw [color=c, fill=c] (13.9403,5.16012) rectangle (13.9801,5.26597);
\draw [color=c, fill=c] (13.9801,5.16012) rectangle (14.0199,5.26597);
\draw [color=c, fill=c] (14.0199,5.16012) rectangle (14.0597,5.26597);
\draw [color=c, fill=c] (14.0597,5.16012) rectangle (14.0995,5.26597);
\draw [color=c, fill=c] (14.0995,5.16012) rectangle (14.1393,5.26597);
\draw [color=c, fill=c] (14.1393,5.16012) rectangle (14.1791,5.26597);
\draw [color=c, fill=c] (14.1791,5.16012) rectangle (14.2189,5.26597);
\draw [color=c, fill=c] (14.2189,5.16012) rectangle (14.2587,5.26597);
\draw [color=c, fill=c] (14.2587,5.16012) rectangle (14.2985,5.26597);
\draw [color=c, fill=c] (14.2985,5.16012) rectangle (14.3383,5.26597);
\draw [color=c, fill=c] (14.3383,5.16012) rectangle (14.3781,5.26597);
\draw [color=c, fill=c] (14.3781,5.16012) rectangle (14.4179,5.26597);
\draw [color=c, fill=c] (14.4179,5.16012) rectangle (14.4577,5.26597);
\draw [color=c, fill=c] (14.4577,5.16012) rectangle (14.4975,5.26597);
\draw [color=c, fill=c] (14.4975,5.16012) rectangle (14.5373,5.26597);
\draw [color=c, fill=c] (14.5373,5.16012) rectangle (14.5771,5.26597);
\draw [color=c, fill=c] (14.5771,5.16012) rectangle (14.6169,5.26597);
\draw [color=c, fill=c] (14.6169,5.16012) rectangle (14.6567,5.26597);
\draw [color=c, fill=c] (14.6567,5.16012) rectangle (14.6965,5.26597);
\draw [color=c, fill=c] (14.6965,5.16012) rectangle (14.7363,5.26597);
\draw [color=c, fill=c] (14.7363,5.16012) rectangle (14.7761,5.26597);
\draw [color=c, fill=c] (14.7761,5.16012) rectangle (14.8159,5.26597);
\draw [color=c, fill=c] (14.8159,5.16012) rectangle (14.8557,5.26597);
\draw [color=c, fill=c] (14.8557,5.16012) rectangle (14.8955,5.26597);
\draw [color=c, fill=c] (14.8955,5.16012) rectangle (14.9353,5.26597);
\draw [color=c, fill=c] (14.9353,5.16012) rectangle (14.9751,5.26597);
\draw [color=c, fill=c] (14.9751,5.16012) rectangle (15.0149,5.26597);
\draw [color=c, fill=c] (15.0149,5.16012) rectangle (15.0547,5.26597);
\draw [color=c, fill=c] (15.0547,5.16012) rectangle (15.0945,5.26597);
\draw [color=c, fill=c] (15.0945,5.16012) rectangle (15.1343,5.26597);
\draw [color=c, fill=c] (15.1343,5.16012) rectangle (15.1741,5.26597);
\draw [color=c, fill=c] (15.1741,5.16012) rectangle (15.2139,5.26597);
\draw [color=c, fill=c] (15.2139,5.16012) rectangle (15.2537,5.26597);
\draw [color=c, fill=c] (15.2537,5.16012) rectangle (15.2935,5.26597);
\draw [color=c, fill=c] (15.2935,5.16012) rectangle (15.3333,5.26597);
\draw [color=c, fill=c] (15.3333,5.16012) rectangle (15.3731,5.26597);
\draw [color=c, fill=c] (15.3731,5.16012) rectangle (15.4129,5.26597);
\draw [color=c, fill=c] (15.4129,5.16012) rectangle (15.4527,5.26597);
\draw [color=c, fill=c] (15.4527,5.16012) rectangle (15.4925,5.26597);
\draw [color=c, fill=c] (15.4925,5.16012) rectangle (15.5323,5.26597);
\draw [color=c, fill=c] (15.5323,5.16012) rectangle (15.5721,5.26597);
\draw [color=c, fill=c] (15.5721,5.16012) rectangle (15.6119,5.26597);
\draw [color=c, fill=c] (15.6119,5.16012) rectangle (15.6517,5.26597);
\draw [color=c, fill=c] (15.6517,5.16012) rectangle (15.6915,5.26597);
\draw [color=c, fill=c] (15.6915,5.16012) rectangle (15.7313,5.26597);
\draw [color=c, fill=c] (15.7313,5.16012) rectangle (15.7711,5.26597);
\draw [color=c, fill=c] (15.7711,5.16012) rectangle (15.8109,5.26597);
\draw [color=c, fill=c] (15.8109,5.16012) rectangle (15.8507,5.26597);
\draw [color=c, fill=c] (15.8507,5.16012) rectangle (15.8905,5.26597);
\draw [color=c, fill=c] (15.8905,5.16012) rectangle (15.9303,5.26597);
\draw [color=c, fill=c] (15.9303,5.16012) rectangle (15.9701,5.26597);
\draw [color=c, fill=c] (15.9701,5.16012) rectangle (16.01,5.26597);
\draw [color=c, fill=c] (16.01,5.16012) rectangle (16.0498,5.26597);
\draw [color=c, fill=c] (16.0498,5.16012) rectangle (16.0896,5.26597);
\draw [color=c, fill=c] (16.0896,5.16012) rectangle (16.1294,5.26597);
\draw [color=c, fill=c] (16.1294,5.16012) rectangle (16.1692,5.26597);
\draw [color=c, fill=c] (16.1692,5.16012) rectangle (16.209,5.26597);
\draw [color=c, fill=c] (16.209,5.16012) rectangle (16.2488,5.26597);
\draw [color=c, fill=c] (16.2488,5.16012) rectangle (16.2886,5.26597);
\draw [color=c, fill=c] (16.2886,5.16012) rectangle (16.3284,5.26597);
\draw [color=c, fill=c] (16.3284,5.16012) rectangle (16.3682,5.26597);
\draw [color=c, fill=c] (16.3682,5.16012) rectangle (16.408,5.26597);
\draw [color=c, fill=c] (16.408,5.16012) rectangle (16.4478,5.26597);
\draw [color=c, fill=c] (16.4478,5.16012) rectangle (16.4876,5.26597);
\draw [color=c, fill=c] (16.4876,5.16012) rectangle (16.5274,5.26597);
\draw [color=c, fill=c] (16.5274,5.16012) rectangle (16.5672,5.26597);
\draw [color=c, fill=c] (16.5672,5.16012) rectangle (16.607,5.26597);
\draw [color=c, fill=c] (16.607,5.16012) rectangle (16.6468,5.26597);
\draw [color=c, fill=c] (16.6468,5.16012) rectangle (16.6866,5.26597);
\draw [color=c, fill=c] (16.6866,5.16012) rectangle (16.7264,5.26597);
\draw [color=c, fill=c] (16.7264,5.16012) rectangle (16.7662,5.26597);
\draw [color=c, fill=c] (16.7662,5.16012) rectangle (16.806,5.26597);
\draw [color=c, fill=c] (16.806,5.16012) rectangle (16.8458,5.26597);
\draw [color=c, fill=c] (16.8458,5.16012) rectangle (16.8856,5.26597);
\draw [color=c, fill=c] (16.8856,5.16012) rectangle (16.9254,5.26597);
\draw [color=c, fill=c] (16.9254,5.16012) rectangle (16.9652,5.26597);
\draw [color=c, fill=c] (16.9652,5.16012) rectangle (17.005,5.26597);
\draw [color=c, fill=c] (17.005,5.16012) rectangle (17.0448,5.26597);
\draw [color=c, fill=c] (17.0448,5.16012) rectangle (17.0846,5.26597);
\draw [color=c, fill=c] (17.0846,5.16012) rectangle (17.1244,5.26597);
\draw [color=c, fill=c] (17.1244,5.16012) rectangle (17.1642,5.26597);
\draw [color=c, fill=c] (17.1642,5.16012) rectangle (17.204,5.26597);
\draw [color=c, fill=c] (17.204,5.16012) rectangle (17.2438,5.26597);
\draw [color=c, fill=c] (17.2438,5.16012) rectangle (17.2836,5.26597);
\draw [color=c, fill=c] (17.2836,5.16012) rectangle (17.3234,5.26597);
\draw [color=c, fill=c] (17.3234,5.16012) rectangle (17.3632,5.26597);
\draw [color=c, fill=c] (17.3632,5.16012) rectangle (17.403,5.26597);
\draw [color=c, fill=c] (17.403,5.16012) rectangle (17.4428,5.26597);
\draw [color=c, fill=c] (17.4428,5.16012) rectangle (17.4826,5.26597);
\draw [color=c, fill=c] (17.4826,5.16012) rectangle (17.5224,5.26597);
\draw [color=c, fill=c] (17.5224,5.16012) rectangle (17.5622,5.26597);
\draw [color=c, fill=c] (17.5622,5.16012) rectangle (17.602,5.26597);
\draw [color=c, fill=c] (17.602,5.16012) rectangle (17.6418,5.26597);
\draw [color=c, fill=c] (17.6418,5.16012) rectangle (17.6816,5.26597);
\draw [color=c, fill=c] (17.6816,5.16012) rectangle (17.7214,5.26597);
\draw [color=c, fill=c] (17.7214,5.16012) rectangle (17.7612,5.26597);
\draw [color=c, fill=c] (17.7612,5.16012) rectangle (17.801,5.26597);
\draw [color=c, fill=c] (17.801,5.16012) rectangle (17.8408,5.26597);
\draw [color=c, fill=c] (17.8408,5.16012) rectangle (17.8806,5.26597);
\draw [color=c, fill=c] (17.8806,5.16012) rectangle (17.9204,5.26597);
\draw [color=c, fill=c] (17.9204,5.16012) rectangle (17.9602,5.26597);
\draw [color=c, fill=c] (17.9602,5.16012) rectangle (18,5.26597);
\definecolor{c}{rgb}{0,0.0800001,1};
\draw [color=c, fill=c] (2,5.26597) rectangle (2.0398,5.37182);
\draw [color=c, fill=c] (2.0398,5.26597) rectangle (2.0796,5.37182);
\draw [color=c, fill=c] (2.0796,5.26597) rectangle (2.1194,5.37182);
\draw [color=c, fill=c] (2.1194,5.26597) rectangle (2.1592,5.37182);
\draw [color=c, fill=c] (2.1592,5.26597) rectangle (2.19901,5.37182);
\draw [color=c, fill=c] (2.19901,5.26597) rectangle (2.23881,5.37182);
\draw [color=c, fill=c] (2.23881,5.26597) rectangle (2.27861,5.37182);
\draw [color=c, fill=c] (2.27861,5.26597) rectangle (2.31841,5.37182);
\draw [color=c, fill=c] (2.31841,5.26597) rectangle (2.35821,5.37182);
\draw [color=c, fill=c] (2.35821,5.26597) rectangle (2.39801,5.37182);
\draw [color=c, fill=c] (2.39801,5.26597) rectangle (2.43781,5.37182);
\draw [color=c, fill=c] (2.43781,5.26597) rectangle (2.47761,5.37182);
\draw [color=c, fill=c] (2.47761,5.26597) rectangle (2.51741,5.37182);
\draw [color=c, fill=c] (2.51741,5.26597) rectangle (2.55721,5.37182);
\draw [color=c, fill=c] (2.55721,5.26597) rectangle (2.59702,5.37182);
\draw [color=c, fill=c] (2.59702,5.26597) rectangle (2.63682,5.37182);
\draw [color=c, fill=c] (2.63682,5.26597) rectangle (2.67662,5.37182);
\draw [color=c, fill=c] (2.67662,5.26597) rectangle (2.71642,5.37182);
\draw [color=c, fill=c] (2.71642,5.26597) rectangle (2.75622,5.37182);
\draw [color=c, fill=c] (2.75622,5.26597) rectangle (2.79602,5.37182);
\draw [color=c, fill=c] (2.79602,5.26597) rectangle (2.83582,5.37182);
\draw [color=c, fill=c] (2.83582,5.26597) rectangle (2.87562,5.37182);
\draw [color=c, fill=c] (2.87562,5.26597) rectangle (2.91542,5.37182);
\draw [color=c, fill=c] (2.91542,5.26597) rectangle (2.95522,5.37182);
\draw [color=c, fill=c] (2.95522,5.26597) rectangle (2.99502,5.37182);
\draw [color=c, fill=c] (2.99502,5.26597) rectangle (3.03483,5.37182);
\draw [color=c, fill=c] (3.03483,5.26597) rectangle (3.07463,5.37182);
\draw [color=c, fill=c] (3.07463,5.26597) rectangle (3.11443,5.37182);
\draw [color=c, fill=c] (3.11443,5.26597) rectangle (3.15423,5.37182);
\draw [color=c, fill=c] (3.15423,5.26597) rectangle (3.19403,5.37182);
\draw [color=c, fill=c] (3.19403,5.26597) rectangle (3.23383,5.37182);
\draw [color=c, fill=c] (3.23383,5.26597) rectangle (3.27363,5.37182);
\draw [color=c, fill=c] (3.27363,5.26597) rectangle (3.31343,5.37182);
\draw [color=c, fill=c] (3.31343,5.26597) rectangle (3.35323,5.37182);
\draw [color=c, fill=c] (3.35323,5.26597) rectangle (3.39303,5.37182);
\draw [color=c, fill=c] (3.39303,5.26597) rectangle (3.43284,5.37182);
\draw [color=c, fill=c] (3.43284,5.26597) rectangle (3.47264,5.37182);
\draw [color=c, fill=c] (3.47264,5.26597) rectangle (3.51244,5.37182);
\draw [color=c, fill=c] (3.51244,5.26597) rectangle (3.55224,5.37182);
\draw [color=c, fill=c] (3.55224,5.26597) rectangle (3.59204,5.37182);
\draw [color=c, fill=c] (3.59204,5.26597) rectangle (3.63184,5.37182);
\draw [color=c, fill=c] (3.63184,5.26597) rectangle (3.67164,5.37182);
\draw [color=c, fill=c] (3.67164,5.26597) rectangle (3.71144,5.37182);
\draw [color=c, fill=c] (3.71144,5.26597) rectangle (3.75124,5.37182);
\draw [color=c, fill=c] (3.75124,5.26597) rectangle (3.79104,5.37182);
\draw [color=c, fill=c] (3.79104,5.26597) rectangle (3.83085,5.37182);
\draw [color=c, fill=c] (3.83085,5.26597) rectangle (3.87065,5.37182);
\draw [color=c, fill=c] (3.87065,5.26597) rectangle (3.91045,5.37182);
\draw [color=c, fill=c] (3.91045,5.26597) rectangle (3.95025,5.37182);
\draw [color=c, fill=c] (3.95025,5.26597) rectangle (3.99005,5.37182);
\draw [color=c, fill=c] (3.99005,5.26597) rectangle (4.02985,5.37182);
\draw [color=c, fill=c] (4.02985,5.26597) rectangle (4.06965,5.37182);
\draw [color=c, fill=c] (4.06965,5.26597) rectangle (4.10945,5.37182);
\draw [color=c, fill=c] (4.10945,5.26597) rectangle (4.14925,5.37182);
\draw [color=c, fill=c] (4.14925,5.26597) rectangle (4.18905,5.37182);
\draw [color=c, fill=c] (4.18905,5.26597) rectangle (4.22886,5.37182);
\draw [color=c, fill=c] (4.22886,5.26597) rectangle (4.26866,5.37182);
\draw [color=c, fill=c] (4.26866,5.26597) rectangle (4.30846,5.37182);
\draw [color=c, fill=c] (4.30846,5.26597) rectangle (4.34826,5.37182);
\draw [color=c, fill=c] (4.34826,5.26597) rectangle (4.38806,5.37182);
\draw [color=c, fill=c] (4.38806,5.26597) rectangle (4.42786,5.37182);
\draw [color=c, fill=c] (4.42786,5.26597) rectangle (4.46766,5.37182);
\draw [color=c, fill=c] (4.46766,5.26597) rectangle (4.50746,5.37182);
\draw [color=c, fill=c] (4.50746,5.26597) rectangle (4.54726,5.37182);
\draw [color=c, fill=c] (4.54726,5.26597) rectangle (4.58706,5.37182);
\draw [color=c, fill=c] (4.58706,5.26597) rectangle (4.62687,5.37182);
\draw [color=c, fill=c] (4.62687,5.26597) rectangle (4.66667,5.37182);
\draw [color=c, fill=c] (4.66667,5.26597) rectangle (4.70647,5.37182);
\draw [color=c, fill=c] (4.70647,5.26597) rectangle (4.74627,5.37182);
\draw [color=c, fill=c] (4.74627,5.26597) rectangle (4.78607,5.37182);
\draw [color=c, fill=c] (4.78607,5.26597) rectangle (4.82587,5.37182);
\draw [color=c, fill=c] (4.82587,5.26597) rectangle (4.86567,5.37182);
\draw [color=c, fill=c] (4.86567,5.26597) rectangle (4.90547,5.37182);
\draw [color=c, fill=c] (4.90547,5.26597) rectangle (4.94527,5.37182);
\draw [color=c, fill=c] (4.94527,5.26597) rectangle (4.98507,5.37182);
\draw [color=c, fill=c] (4.98507,5.26597) rectangle (5.02488,5.37182);
\draw [color=c, fill=c] (5.02488,5.26597) rectangle (5.06468,5.37182);
\draw [color=c, fill=c] (5.06468,5.26597) rectangle (5.10448,5.37182);
\draw [color=c, fill=c] (5.10448,5.26597) rectangle (5.14428,5.37182);
\draw [color=c, fill=c] (5.14428,5.26597) rectangle (5.18408,5.37182);
\draw [color=c, fill=c] (5.18408,5.26597) rectangle (5.22388,5.37182);
\draw [color=c, fill=c] (5.22388,5.26597) rectangle (5.26368,5.37182);
\draw [color=c, fill=c] (5.26368,5.26597) rectangle (5.30348,5.37182);
\draw [color=c, fill=c] (5.30348,5.26597) rectangle (5.34328,5.37182);
\draw [color=c, fill=c] (5.34328,5.26597) rectangle (5.38308,5.37182);
\draw [color=c, fill=c] (5.38308,5.26597) rectangle (5.42289,5.37182);
\draw [color=c, fill=c] (5.42289,5.26597) rectangle (5.46269,5.37182);
\draw [color=c, fill=c] (5.46269,5.26597) rectangle (5.50249,5.37182);
\draw [color=c, fill=c] (5.50249,5.26597) rectangle (5.54229,5.37182);
\draw [color=c, fill=c] (5.54229,5.26597) rectangle (5.58209,5.37182);
\draw [color=c, fill=c] (5.58209,5.26597) rectangle (5.62189,5.37182);
\draw [color=c, fill=c] (5.62189,5.26597) rectangle (5.66169,5.37182);
\draw [color=c, fill=c] (5.66169,5.26597) rectangle (5.70149,5.37182);
\draw [color=c, fill=c] (5.70149,5.26597) rectangle (5.74129,5.37182);
\definecolor{c}{rgb}{0.2,0,1};
\draw [color=c, fill=c] (5.74129,5.26597) rectangle (5.78109,5.37182);
\draw [color=c, fill=c] (5.78109,5.26597) rectangle (5.8209,5.37182);
\draw [color=c, fill=c] (5.8209,5.26597) rectangle (5.8607,5.37182);
\draw [color=c, fill=c] (5.8607,5.26597) rectangle (5.9005,5.37182);
\draw [color=c, fill=c] (5.9005,5.26597) rectangle (5.9403,5.37182);
\draw [color=c, fill=c] (5.9403,5.26597) rectangle (5.9801,5.37182);
\draw [color=c, fill=c] (5.9801,5.26597) rectangle (6.0199,5.37182);
\draw [color=c, fill=c] (6.0199,5.26597) rectangle (6.0597,5.37182);
\draw [color=c, fill=c] (6.0597,5.26597) rectangle (6.0995,5.37182);
\draw [color=c, fill=c] (6.0995,5.26597) rectangle (6.1393,5.37182);
\draw [color=c, fill=c] (6.1393,5.26597) rectangle (6.1791,5.37182);
\draw [color=c, fill=c] (6.1791,5.26597) rectangle (6.21891,5.37182);
\draw [color=c, fill=c] (6.21891,5.26597) rectangle (6.25871,5.37182);
\draw [color=c, fill=c] (6.25871,5.26597) rectangle (6.29851,5.37182);
\draw [color=c, fill=c] (6.29851,5.26597) rectangle (6.33831,5.37182);
\draw [color=c, fill=c] (6.33831,5.26597) rectangle (6.37811,5.37182);
\draw [color=c, fill=c] (6.37811,5.26597) rectangle (6.41791,5.37182);
\draw [color=c, fill=c] (6.41791,5.26597) rectangle (6.45771,5.37182);
\draw [color=c, fill=c] (6.45771,5.26597) rectangle (6.49751,5.37182);
\draw [color=c, fill=c] (6.49751,5.26597) rectangle (6.53731,5.37182);
\draw [color=c, fill=c] (6.53731,5.26597) rectangle (6.57711,5.37182);
\draw [color=c, fill=c] (6.57711,5.26597) rectangle (6.61692,5.37182);
\draw [color=c, fill=c] (6.61692,5.26597) rectangle (6.65672,5.37182);
\draw [color=c, fill=c] (6.65672,5.26597) rectangle (6.69652,5.37182);
\draw [color=c, fill=c] (6.69652,5.26597) rectangle (6.73632,5.37182);
\draw [color=c, fill=c] (6.73632,5.26597) rectangle (6.77612,5.37182);
\draw [color=c, fill=c] (6.77612,5.26597) rectangle (6.81592,5.37182);
\draw [color=c, fill=c] (6.81592,5.26597) rectangle (6.85572,5.37182);
\draw [color=c, fill=c] (6.85572,5.26597) rectangle (6.89552,5.37182);
\draw [color=c, fill=c] (6.89552,5.26597) rectangle (6.93532,5.37182);
\draw [color=c, fill=c] (6.93532,5.26597) rectangle (6.97512,5.37182);
\draw [color=c, fill=c] (6.97512,5.26597) rectangle (7.01493,5.37182);
\draw [color=c, fill=c] (7.01493,5.26597) rectangle (7.05473,5.37182);
\draw [color=c, fill=c] (7.05473,5.26597) rectangle (7.09453,5.37182);
\draw [color=c, fill=c] (7.09453,5.26597) rectangle (7.13433,5.37182);
\draw [color=c, fill=c] (7.13433,5.26597) rectangle (7.17413,5.37182);
\draw [color=c, fill=c] (7.17413,5.26597) rectangle (7.21393,5.37182);
\draw [color=c, fill=c] (7.21393,5.26597) rectangle (7.25373,5.37182);
\draw [color=c, fill=c] (7.25373,5.26597) rectangle (7.29353,5.37182);
\draw [color=c, fill=c] (7.29353,5.26597) rectangle (7.33333,5.37182);
\draw [color=c, fill=c] (7.33333,5.26597) rectangle (7.37313,5.37182);
\draw [color=c, fill=c] (7.37313,5.26597) rectangle (7.41294,5.37182);
\draw [color=c, fill=c] (7.41294,5.26597) rectangle (7.45274,5.37182);
\draw [color=c, fill=c] (7.45274,5.26597) rectangle (7.49254,5.37182);
\draw [color=c, fill=c] (7.49254,5.26597) rectangle (7.53234,5.37182);
\draw [color=c, fill=c] (7.53234,5.26597) rectangle (7.57214,5.37182);
\draw [color=c, fill=c] (7.57214,5.26597) rectangle (7.61194,5.37182);
\draw [color=c, fill=c] (7.61194,5.26597) rectangle (7.65174,5.37182);
\draw [color=c, fill=c] (7.65174,5.26597) rectangle (7.69154,5.37182);
\draw [color=c, fill=c] (7.69154,5.26597) rectangle (7.73134,5.37182);
\draw [color=c, fill=c] (7.73134,5.26597) rectangle (7.77114,5.37182);
\draw [color=c, fill=c] (7.77114,5.26597) rectangle (7.81095,5.37182);
\draw [color=c, fill=c] (7.81095,5.26597) rectangle (7.85075,5.37182);
\draw [color=c, fill=c] (7.85075,5.26597) rectangle (7.89055,5.37182);
\draw [color=c, fill=c] (7.89055,5.26597) rectangle (7.93035,5.37182);
\draw [color=c, fill=c] (7.93035,5.26597) rectangle (7.97015,5.37182);
\draw [color=c, fill=c] (7.97015,5.26597) rectangle (8.00995,5.37182);
\draw [color=c, fill=c] (8.00995,5.26597) rectangle (8.04975,5.37182);
\draw [color=c, fill=c] (8.04975,5.26597) rectangle (8.08955,5.37182);
\draw [color=c, fill=c] (8.08955,5.26597) rectangle (8.12935,5.37182);
\draw [color=c, fill=c] (8.12935,5.26597) rectangle (8.16915,5.37182);
\draw [color=c, fill=c] (8.16915,5.26597) rectangle (8.20895,5.37182);
\draw [color=c, fill=c] (8.20895,5.26597) rectangle (8.24876,5.37182);
\draw [color=c, fill=c] (8.24876,5.26597) rectangle (8.28856,5.37182);
\draw [color=c, fill=c] (8.28856,5.26597) rectangle (8.32836,5.37182);
\draw [color=c, fill=c] (8.32836,5.26597) rectangle (8.36816,5.37182);
\draw [color=c, fill=c] (8.36816,5.26597) rectangle (8.40796,5.37182);
\draw [color=c, fill=c] (8.40796,5.26597) rectangle (8.44776,5.37182);
\draw [color=c, fill=c] (8.44776,5.26597) rectangle (8.48756,5.37182);
\draw [color=c, fill=c] (8.48756,5.26597) rectangle (8.52736,5.37182);
\draw [color=c, fill=c] (8.52736,5.26597) rectangle (8.56716,5.37182);
\draw [color=c, fill=c] (8.56716,5.26597) rectangle (8.60697,5.37182);
\draw [color=c, fill=c] (8.60697,5.26597) rectangle (8.64677,5.37182);
\draw [color=c, fill=c] (8.64677,5.26597) rectangle (8.68657,5.37182);
\draw [color=c, fill=c] (8.68657,5.26597) rectangle (8.72637,5.37182);
\draw [color=c, fill=c] (8.72637,5.26597) rectangle (8.76617,5.37182);
\draw [color=c, fill=c] (8.76617,5.26597) rectangle (8.80597,5.37182);
\draw [color=c, fill=c] (8.80597,5.26597) rectangle (8.84577,5.37182);
\draw [color=c, fill=c] (8.84577,5.26597) rectangle (8.88557,5.37182);
\draw [color=c, fill=c] (8.88557,5.26597) rectangle (8.92537,5.37182);
\draw [color=c, fill=c] (8.92537,5.26597) rectangle (8.96517,5.37182);
\draw [color=c, fill=c] (8.96517,5.26597) rectangle (9.00498,5.37182);
\draw [color=c, fill=c] (9.00498,5.26597) rectangle (9.04478,5.37182);
\draw [color=c, fill=c] (9.04478,5.26597) rectangle (9.08458,5.37182);
\draw [color=c, fill=c] (9.08458,5.26597) rectangle (9.12438,5.37182);
\draw [color=c, fill=c] (9.12438,5.26597) rectangle (9.16418,5.37182);
\draw [color=c, fill=c] (9.16418,5.26597) rectangle (9.20398,5.37182);
\draw [color=c, fill=c] (9.20398,5.26597) rectangle (9.24378,5.37182);
\draw [color=c, fill=c] (9.24378,5.26597) rectangle (9.28358,5.37182);
\draw [color=c, fill=c] (9.28358,5.26597) rectangle (9.32338,5.37182);
\draw [color=c, fill=c] (9.32338,5.26597) rectangle (9.36318,5.37182);
\draw [color=c, fill=c] (9.36318,5.26597) rectangle (9.40298,5.37182);
\definecolor{c}{rgb}{0,0.0800001,1};
\draw [color=c, fill=c] (9.40298,5.26597) rectangle (9.44279,5.37182);
\draw [color=c, fill=c] (9.44279,5.26597) rectangle (9.48259,5.37182);
\draw [color=c, fill=c] (9.48259,5.26597) rectangle (9.52239,5.37182);
\draw [color=c, fill=c] (9.52239,5.26597) rectangle (9.56219,5.37182);
\draw [color=c, fill=c] (9.56219,5.26597) rectangle (9.60199,5.37182);
\draw [color=c, fill=c] (9.60199,5.26597) rectangle (9.64179,5.37182);
\draw [color=c, fill=c] (9.64179,5.26597) rectangle (9.68159,5.37182);
\draw [color=c, fill=c] (9.68159,5.26597) rectangle (9.72139,5.37182);
\draw [color=c, fill=c] (9.72139,5.26597) rectangle (9.76119,5.37182);
\draw [color=c, fill=c] (9.76119,5.26597) rectangle (9.80099,5.37182);
\definecolor{c}{rgb}{0,0.266667,1};
\draw [color=c, fill=c] (9.80099,5.26597) rectangle (9.8408,5.37182);
\draw [color=c, fill=c] (9.8408,5.26597) rectangle (9.8806,5.37182);
\draw [color=c, fill=c] (9.8806,5.26597) rectangle (9.9204,5.37182);
\draw [color=c, fill=c] (9.9204,5.26597) rectangle (9.9602,5.37182);
\draw [color=c, fill=c] (9.9602,5.26597) rectangle (10,5.37182);
\draw [color=c, fill=c] (10,5.26597) rectangle (10.0398,5.37182);
\definecolor{c}{rgb}{0,0.546666,1};
\draw [color=c, fill=c] (10.0398,5.26597) rectangle (10.0796,5.37182);
\draw [color=c, fill=c] (10.0796,5.26597) rectangle (10.1194,5.37182);
\draw [color=c, fill=c] (10.1194,5.26597) rectangle (10.1592,5.37182);
\draw [color=c, fill=c] (10.1592,5.26597) rectangle (10.199,5.37182);
\draw [color=c, fill=c] (10.199,5.26597) rectangle (10.2388,5.37182);
\draw [color=c, fill=c] (10.2388,5.26597) rectangle (10.2786,5.37182);
\draw [color=c, fill=c] (10.2786,5.26597) rectangle (10.3184,5.37182);
\draw [color=c, fill=c] (10.3184,5.26597) rectangle (10.3582,5.37182);
\definecolor{c}{rgb}{0,0.733333,1};
\draw [color=c, fill=c] (10.3582,5.26597) rectangle (10.398,5.37182);
\draw [color=c, fill=c] (10.398,5.26597) rectangle (10.4378,5.37182);
\draw [color=c, fill=c] (10.4378,5.26597) rectangle (10.4776,5.37182);
\draw [color=c, fill=c] (10.4776,5.26597) rectangle (10.5174,5.37182);
\draw [color=c, fill=c] (10.5174,5.26597) rectangle (10.5572,5.37182);
\draw [color=c, fill=c] (10.5572,5.26597) rectangle (10.597,5.37182);
\draw [color=c, fill=c] (10.597,5.26597) rectangle (10.6368,5.37182);
\draw [color=c, fill=c] (10.6368,5.26597) rectangle (10.6766,5.37182);
\draw [color=c, fill=c] (10.6766,5.26597) rectangle (10.7164,5.37182);
\draw [color=c, fill=c] (10.7164,5.26597) rectangle (10.7562,5.37182);
\draw [color=c, fill=c] (10.7562,5.26597) rectangle (10.796,5.37182);
\draw [color=c, fill=c] (10.796,5.26597) rectangle (10.8358,5.37182);
\draw [color=c, fill=c] (10.8358,5.26597) rectangle (10.8756,5.37182);
\draw [color=c, fill=c] (10.8756,5.26597) rectangle (10.9154,5.37182);
\draw [color=c, fill=c] (10.9154,5.26597) rectangle (10.9552,5.37182);
\draw [color=c, fill=c] (10.9552,5.26597) rectangle (10.995,5.37182);
\draw [color=c, fill=c] (10.995,5.26597) rectangle (11.0348,5.37182);
\draw [color=c, fill=c] (11.0348,5.26597) rectangle (11.0746,5.37182);
\draw [color=c, fill=c] (11.0746,5.26597) rectangle (11.1144,5.37182);
\draw [color=c, fill=c] (11.1144,5.26597) rectangle (11.1542,5.37182);
\draw [color=c, fill=c] (11.1542,5.26597) rectangle (11.194,5.37182);
\draw [color=c, fill=c] (11.194,5.26597) rectangle (11.2338,5.37182);
\draw [color=c, fill=c] (11.2338,5.26597) rectangle (11.2736,5.37182);
\draw [color=c, fill=c] (11.2736,5.26597) rectangle (11.3134,5.37182);
\draw [color=c, fill=c] (11.3134,5.26597) rectangle (11.3532,5.37182);
\draw [color=c, fill=c] (11.3532,5.26597) rectangle (11.393,5.37182);
\draw [color=c, fill=c] (11.393,5.26597) rectangle (11.4328,5.37182);
\draw [color=c, fill=c] (11.4328,5.26597) rectangle (11.4726,5.37182);
\draw [color=c, fill=c] (11.4726,5.26597) rectangle (11.5124,5.37182);
\draw [color=c, fill=c] (11.5124,5.26597) rectangle (11.5522,5.37182);
\draw [color=c, fill=c] (11.5522,5.26597) rectangle (11.592,5.37182);
\draw [color=c, fill=c] (11.592,5.26597) rectangle (11.6318,5.37182);
\draw [color=c, fill=c] (11.6318,5.26597) rectangle (11.6716,5.37182);
\draw [color=c, fill=c] (11.6716,5.26597) rectangle (11.7114,5.37182);
\draw [color=c, fill=c] (11.7114,5.26597) rectangle (11.7512,5.37182);
\draw [color=c, fill=c] (11.7512,5.26597) rectangle (11.791,5.37182);
\draw [color=c, fill=c] (11.791,5.26597) rectangle (11.8308,5.37182);
\draw [color=c, fill=c] (11.8308,5.26597) rectangle (11.8706,5.37182);
\draw [color=c, fill=c] (11.8706,5.26597) rectangle (11.9104,5.37182);
\draw [color=c, fill=c] (11.9104,5.26597) rectangle (11.9502,5.37182);
\draw [color=c, fill=c] (11.9502,5.26597) rectangle (11.99,5.37182);
\draw [color=c, fill=c] (11.99,5.26597) rectangle (12.0299,5.37182);
\draw [color=c, fill=c] (12.0299,5.26597) rectangle (12.0697,5.37182);
\draw [color=c, fill=c] (12.0697,5.26597) rectangle (12.1095,5.37182);
\draw [color=c, fill=c] (12.1095,5.26597) rectangle (12.1493,5.37182);
\draw [color=c, fill=c] (12.1493,5.26597) rectangle (12.1891,5.37182);
\draw [color=c, fill=c] (12.1891,5.26597) rectangle (12.2289,5.37182);
\draw [color=c, fill=c] (12.2289,5.26597) rectangle (12.2687,5.37182);
\draw [color=c, fill=c] (12.2687,5.26597) rectangle (12.3085,5.37182);
\draw [color=c, fill=c] (12.3085,5.26597) rectangle (12.3483,5.37182);
\draw [color=c, fill=c] (12.3483,5.26597) rectangle (12.3881,5.37182);
\draw [color=c, fill=c] (12.3881,5.26597) rectangle (12.4279,5.37182);
\draw [color=c, fill=c] (12.4279,5.26597) rectangle (12.4677,5.37182);
\draw [color=c, fill=c] (12.4677,5.26597) rectangle (12.5075,5.37182);
\draw [color=c, fill=c] (12.5075,5.26597) rectangle (12.5473,5.37182);
\draw [color=c, fill=c] (12.5473,5.26597) rectangle (12.5871,5.37182);
\draw [color=c, fill=c] (12.5871,5.26597) rectangle (12.6269,5.37182);
\draw [color=c, fill=c] (12.6269,5.26597) rectangle (12.6667,5.37182);
\draw [color=c, fill=c] (12.6667,5.26597) rectangle (12.7065,5.37182);
\draw [color=c, fill=c] (12.7065,5.26597) rectangle (12.7463,5.37182);
\draw [color=c, fill=c] (12.7463,5.26597) rectangle (12.7861,5.37182);
\draw [color=c, fill=c] (12.7861,5.26597) rectangle (12.8259,5.37182);
\draw [color=c, fill=c] (12.8259,5.26597) rectangle (12.8657,5.37182);
\draw [color=c, fill=c] (12.8657,5.26597) rectangle (12.9055,5.37182);
\draw [color=c, fill=c] (12.9055,5.26597) rectangle (12.9453,5.37182);
\draw [color=c, fill=c] (12.9453,5.26597) rectangle (12.9851,5.37182);
\draw [color=c, fill=c] (12.9851,5.26597) rectangle (13.0249,5.37182);
\draw [color=c, fill=c] (13.0249,5.26597) rectangle (13.0647,5.37182);
\draw [color=c, fill=c] (13.0647,5.26597) rectangle (13.1045,5.37182);
\draw [color=c, fill=c] (13.1045,5.26597) rectangle (13.1443,5.37182);
\draw [color=c, fill=c] (13.1443,5.26597) rectangle (13.1841,5.37182);
\draw [color=c, fill=c] (13.1841,5.26597) rectangle (13.2239,5.37182);
\draw [color=c, fill=c] (13.2239,5.26597) rectangle (13.2637,5.37182);
\draw [color=c, fill=c] (13.2637,5.26597) rectangle (13.3035,5.37182);
\draw [color=c, fill=c] (13.3035,5.26597) rectangle (13.3433,5.37182);
\draw [color=c, fill=c] (13.3433,5.26597) rectangle (13.3831,5.37182);
\draw [color=c, fill=c] (13.3831,5.26597) rectangle (13.4229,5.37182);
\draw [color=c, fill=c] (13.4229,5.26597) rectangle (13.4627,5.37182);
\draw [color=c, fill=c] (13.4627,5.26597) rectangle (13.5025,5.37182);
\draw [color=c, fill=c] (13.5025,5.26597) rectangle (13.5423,5.37182);
\draw [color=c, fill=c] (13.5423,5.26597) rectangle (13.5821,5.37182);
\draw [color=c, fill=c] (13.5821,5.26597) rectangle (13.6219,5.37182);
\draw [color=c, fill=c] (13.6219,5.26597) rectangle (13.6617,5.37182);
\draw [color=c, fill=c] (13.6617,5.26597) rectangle (13.7015,5.37182);
\draw [color=c, fill=c] (13.7015,5.26597) rectangle (13.7413,5.37182);
\draw [color=c, fill=c] (13.7413,5.26597) rectangle (13.7811,5.37182);
\draw [color=c, fill=c] (13.7811,5.26597) rectangle (13.8209,5.37182);
\draw [color=c, fill=c] (13.8209,5.26597) rectangle (13.8607,5.37182);
\draw [color=c, fill=c] (13.8607,5.26597) rectangle (13.9005,5.37182);
\draw [color=c, fill=c] (13.9005,5.26597) rectangle (13.9403,5.37182);
\draw [color=c, fill=c] (13.9403,5.26597) rectangle (13.9801,5.37182);
\draw [color=c, fill=c] (13.9801,5.26597) rectangle (14.0199,5.37182);
\draw [color=c, fill=c] (14.0199,5.26597) rectangle (14.0597,5.37182);
\draw [color=c, fill=c] (14.0597,5.26597) rectangle (14.0995,5.37182);
\draw [color=c, fill=c] (14.0995,5.26597) rectangle (14.1393,5.37182);
\draw [color=c, fill=c] (14.1393,5.26597) rectangle (14.1791,5.37182);
\draw [color=c, fill=c] (14.1791,5.26597) rectangle (14.2189,5.37182);
\draw [color=c, fill=c] (14.2189,5.26597) rectangle (14.2587,5.37182);
\draw [color=c, fill=c] (14.2587,5.26597) rectangle (14.2985,5.37182);
\draw [color=c, fill=c] (14.2985,5.26597) rectangle (14.3383,5.37182);
\draw [color=c, fill=c] (14.3383,5.26597) rectangle (14.3781,5.37182);
\draw [color=c, fill=c] (14.3781,5.26597) rectangle (14.4179,5.37182);
\draw [color=c, fill=c] (14.4179,5.26597) rectangle (14.4577,5.37182);
\draw [color=c, fill=c] (14.4577,5.26597) rectangle (14.4975,5.37182);
\draw [color=c, fill=c] (14.4975,5.26597) rectangle (14.5373,5.37182);
\draw [color=c, fill=c] (14.5373,5.26597) rectangle (14.5771,5.37182);
\draw [color=c, fill=c] (14.5771,5.26597) rectangle (14.6169,5.37182);
\draw [color=c, fill=c] (14.6169,5.26597) rectangle (14.6567,5.37182);
\draw [color=c, fill=c] (14.6567,5.26597) rectangle (14.6965,5.37182);
\draw [color=c, fill=c] (14.6965,5.26597) rectangle (14.7363,5.37182);
\draw [color=c, fill=c] (14.7363,5.26597) rectangle (14.7761,5.37182);
\draw [color=c, fill=c] (14.7761,5.26597) rectangle (14.8159,5.37182);
\draw [color=c, fill=c] (14.8159,5.26597) rectangle (14.8557,5.37182);
\draw [color=c, fill=c] (14.8557,5.26597) rectangle (14.8955,5.37182);
\draw [color=c, fill=c] (14.8955,5.26597) rectangle (14.9353,5.37182);
\draw [color=c, fill=c] (14.9353,5.26597) rectangle (14.9751,5.37182);
\draw [color=c, fill=c] (14.9751,5.26597) rectangle (15.0149,5.37182);
\draw [color=c, fill=c] (15.0149,5.26597) rectangle (15.0547,5.37182);
\draw [color=c, fill=c] (15.0547,5.26597) rectangle (15.0945,5.37182);
\draw [color=c, fill=c] (15.0945,5.26597) rectangle (15.1343,5.37182);
\draw [color=c, fill=c] (15.1343,5.26597) rectangle (15.1741,5.37182);
\draw [color=c, fill=c] (15.1741,5.26597) rectangle (15.2139,5.37182);
\draw [color=c, fill=c] (15.2139,5.26597) rectangle (15.2537,5.37182);
\draw [color=c, fill=c] (15.2537,5.26597) rectangle (15.2935,5.37182);
\draw [color=c, fill=c] (15.2935,5.26597) rectangle (15.3333,5.37182);
\draw [color=c, fill=c] (15.3333,5.26597) rectangle (15.3731,5.37182);
\draw [color=c, fill=c] (15.3731,5.26597) rectangle (15.4129,5.37182);
\draw [color=c, fill=c] (15.4129,5.26597) rectangle (15.4527,5.37182);
\draw [color=c, fill=c] (15.4527,5.26597) rectangle (15.4925,5.37182);
\draw [color=c, fill=c] (15.4925,5.26597) rectangle (15.5323,5.37182);
\draw [color=c, fill=c] (15.5323,5.26597) rectangle (15.5721,5.37182);
\draw [color=c, fill=c] (15.5721,5.26597) rectangle (15.6119,5.37182);
\draw [color=c, fill=c] (15.6119,5.26597) rectangle (15.6517,5.37182);
\draw [color=c, fill=c] (15.6517,5.26597) rectangle (15.6915,5.37182);
\draw [color=c, fill=c] (15.6915,5.26597) rectangle (15.7313,5.37182);
\draw [color=c, fill=c] (15.7313,5.26597) rectangle (15.7711,5.37182);
\draw [color=c, fill=c] (15.7711,5.26597) rectangle (15.8109,5.37182);
\draw [color=c, fill=c] (15.8109,5.26597) rectangle (15.8507,5.37182);
\draw [color=c, fill=c] (15.8507,5.26597) rectangle (15.8905,5.37182);
\draw [color=c, fill=c] (15.8905,5.26597) rectangle (15.9303,5.37182);
\draw [color=c, fill=c] (15.9303,5.26597) rectangle (15.9701,5.37182);
\draw [color=c, fill=c] (15.9701,5.26597) rectangle (16.01,5.37182);
\draw [color=c, fill=c] (16.01,5.26597) rectangle (16.0498,5.37182);
\draw [color=c, fill=c] (16.0498,5.26597) rectangle (16.0896,5.37182);
\draw [color=c, fill=c] (16.0896,5.26597) rectangle (16.1294,5.37182);
\draw [color=c, fill=c] (16.1294,5.26597) rectangle (16.1692,5.37182);
\draw [color=c, fill=c] (16.1692,5.26597) rectangle (16.209,5.37182);
\draw [color=c, fill=c] (16.209,5.26597) rectangle (16.2488,5.37182);
\draw [color=c, fill=c] (16.2488,5.26597) rectangle (16.2886,5.37182);
\draw [color=c, fill=c] (16.2886,5.26597) rectangle (16.3284,5.37182);
\draw [color=c, fill=c] (16.3284,5.26597) rectangle (16.3682,5.37182);
\draw [color=c, fill=c] (16.3682,5.26597) rectangle (16.408,5.37182);
\draw [color=c, fill=c] (16.408,5.26597) rectangle (16.4478,5.37182);
\draw [color=c, fill=c] (16.4478,5.26597) rectangle (16.4876,5.37182);
\draw [color=c, fill=c] (16.4876,5.26597) rectangle (16.5274,5.37182);
\draw [color=c, fill=c] (16.5274,5.26597) rectangle (16.5672,5.37182);
\draw [color=c, fill=c] (16.5672,5.26597) rectangle (16.607,5.37182);
\draw [color=c, fill=c] (16.607,5.26597) rectangle (16.6468,5.37182);
\draw [color=c, fill=c] (16.6468,5.26597) rectangle (16.6866,5.37182);
\draw [color=c, fill=c] (16.6866,5.26597) rectangle (16.7264,5.37182);
\draw [color=c, fill=c] (16.7264,5.26597) rectangle (16.7662,5.37182);
\draw [color=c, fill=c] (16.7662,5.26597) rectangle (16.806,5.37182);
\draw [color=c, fill=c] (16.806,5.26597) rectangle (16.8458,5.37182);
\draw [color=c, fill=c] (16.8458,5.26597) rectangle (16.8856,5.37182);
\draw [color=c, fill=c] (16.8856,5.26597) rectangle (16.9254,5.37182);
\draw [color=c, fill=c] (16.9254,5.26597) rectangle (16.9652,5.37182);
\draw [color=c, fill=c] (16.9652,5.26597) rectangle (17.005,5.37182);
\draw [color=c, fill=c] (17.005,5.26597) rectangle (17.0448,5.37182);
\draw [color=c, fill=c] (17.0448,5.26597) rectangle (17.0846,5.37182);
\draw [color=c, fill=c] (17.0846,5.26597) rectangle (17.1244,5.37182);
\draw [color=c, fill=c] (17.1244,5.26597) rectangle (17.1642,5.37182);
\draw [color=c, fill=c] (17.1642,5.26597) rectangle (17.204,5.37182);
\draw [color=c, fill=c] (17.204,5.26597) rectangle (17.2438,5.37182);
\draw [color=c, fill=c] (17.2438,5.26597) rectangle (17.2836,5.37182);
\draw [color=c, fill=c] (17.2836,5.26597) rectangle (17.3234,5.37182);
\draw [color=c, fill=c] (17.3234,5.26597) rectangle (17.3632,5.37182);
\draw [color=c, fill=c] (17.3632,5.26597) rectangle (17.403,5.37182);
\draw [color=c, fill=c] (17.403,5.26597) rectangle (17.4428,5.37182);
\draw [color=c, fill=c] (17.4428,5.26597) rectangle (17.4826,5.37182);
\draw [color=c, fill=c] (17.4826,5.26597) rectangle (17.5224,5.37182);
\draw [color=c, fill=c] (17.5224,5.26597) rectangle (17.5622,5.37182);
\draw [color=c, fill=c] (17.5622,5.26597) rectangle (17.602,5.37182);
\draw [color=c, fill=c] (17.602,5.26597) rectangle (17.6418,5.37182);
\draw [color=c, fill=c] (17.6418,5.26597) rectangle (17.6816,5.37182);
\draw [color=c, fill=c] (17.6816,5.26597) rectangle (17.7214,5.37182);
\draw [color=c, fill=c] (17.7214,5.26597) rectangle (17.7612,5.37182);
\draw [color=c, fill=c] (17.7612,5.26597) rectangle (17.801,5.37182);
\draw [color=c, fill=c] (17.801,5.26597) rectangle (17.8408,5.37182);
\draw [color=c, fill=c] (17.8408,5.26597) rectangle (17.8806,5.37182);
\draw [color=c, fill=c] (17.8806,5.26597) rectangle (17.9204,5.37182);
\draw [color=c, fill=c] (17.9204,5.26597) rectangle (17.9602,5.37182);
\draw [color=c, fill=c] (17.9602,5.26597) rectangle (18,5.37182);
\definecolor{c}{rgb}{0,0.0800001,1};
\draw [color=c, fill=c] (2,5.37182) rectangle (2.0398,5.47767);
\draw [color=c, fill=c] (2.0398,5.37182) rectangle (2.0796,5.47767);
\draw [color=c, fill=c] (2.0796,5.37182) rectangle (2.1194,5.47767);
\draw [color=c, fill=c] (2.1194,5.37182) rectangle (2.1592,5.47767);
\draw [color=c, fill=c] (2.1592,5.37182) rectangle (2.19901,5.47767);
\draw [color=c, fill=c] (2.19901,5.37182) rectangle (2.23881,5.47767);
\draw [color=c, fill=c] (2.23881,5.37182) rectangle (2.27861,5.47767);
\draw [color=c, fill=c] (2.27861,5.37182) rectangle (2.31841,5.47767);
\draw [color=c, fill=c] (2.31841,5.37182) rectangle (2.35821,5.47767);
\draw [color=c, fill=c] (2.35821,5.37182) rectangle (2.39801,5.47767);
\draw [color=c, fill=c] (2.39801,5.37182) rectangle (2.43781,5.47767);
\draw [color=c, fill=c] (2.43781,5.37182) rectangle (2.47761,5.47767);
\draw [color=c, fill=c] (2.47761,5.37182) rectangle (2.51741,5.47767);
\draw [color=c, fill=c] (2.51741,5.37182) rectangle (2.55721,5.47767);
\draw [color=c, fill=c] (2.55721,5.37182) rectangle (2.59702,5.47767);
\draw [color=c, fill=c] (2.59702,5.37182) rectangle (2.63682,5.47767);
\draw [color=c, fill=c] (2.63682,5.37182) rectangle (2.67662,5.47767);
\draw [color=c, fill=c] (2.67662,5.37182) rectangle (2.71642,5.47767);
\draw [color=c, fill=c] (2.71642,5.37182) rectangle (2.75622,5.47767);
\draw [color=c, fill=c] (2.75622,5.37182) rectangle (2.79602,5.47767);
\draw [color=c, fill=c] (2.79602,5.37182) rectangle (2.83582,5.47767);
\draw [color=c, fill=c] (2.83582,5.37182) rectangle (2.87562,5.47767);
\draw [color=c, fill=c] (2.87562,5.37182) rectangle (2.91542,5.47767);
\draw [color=c, fill=c] (2.91542,5.37182) rectangle (2.95522,5.47767);
\draw [color=c, fill=c] (2.95522,5.37182) rectangle (2.99502,5.47767);
\draw [color=c, fill=c] (2.99502,5.37182) rectangle (3.03483,5.47767);
\draw [color=c, fill=c] (3.03483,5.37182) rectangle (3.07463,5.47767);
\draw [color=c, fill=c] (3.07463,5.37182) rectangle (3.11443,5.47767);
\draw [color=c, fill=c] (3.11443,5.37182) rectangle (3.15423,5.47767);
\draw [color=c, fill=c] (3.15423,5.37182) rectangle (3.19403,5.47767);
\draw [color=c, fill=c] (3.19403,5.37182) rectangle (3.23383,5.47767);
\draw [color=c, fill=c] (3.23383,5.37182) rectangle (3.27363,5.47767);
\draw [color=c, fill=c] (3.27363,5.37182) rectangle (3.31343,5.47767);
\draw [color=c, fill=c] (3.31343,5.37182) rectangle (3.35323,5.47767);
\draw [color=c, fill=c] (3.35323,5.37182) rectangle (3.39303,5.47767);
\draw [color=c, fill=c] (3.39303,5.37182) rectangle (3.43284,5.47767);
\draw [color=c, fill=c] (3.43284,5.37182) rectangle (3.47264,5.47767);
\draw [color=c, fill=c] (3.47264,5.37182) rectangle (3.51244,5.47767);
\draw [color=c, fill=c] (3.51244,5.37182) rectangle (3.55224,5.47767);
\draw [color=c, fill=c] (3.55224,5.37182) rectangle (3.59204,5.47767);
\draw [color=c, fill=c] (3.59204,5.37182) rectangle (3.63184,5.47767);
\draw [color=c, fill=c] (3.63184,5.37182) rectangle (3.67164,5.47767);
\draw [color=c, fill=c] (3.67164,5.37182) rectangle (3.71144,5.47767);
\draw [color=c, fill=c] (3.71144,5.37182) rectangle (3.75124,5.47767);
\draw [color=c, fill=c] (3.75124,5.37182) rectangle (3.79104,5.47767);
\draw [color=c, fill=c] (3.79104,5.37182) rectangle (3.83085,5.47767);
\draw [color=c, fill=c] (3.83085,5.37182) rectangle (3.87065,5.47767);
\draw [color=c, fill=c] (3.87065,5.37182) rectangle (3.91045,5.47767);
\draw [color=c, fill=c] (3.91045,5.37182) rectangle (3.95025,5.47767);
\draw [color=c, fill=c] (3.95025,5.37182) rectangle (3.99005,5.47767);
\draw [color=c, fill=c] (3.99005,5.37182) rectangle (4.02985,5.47767);
\draw [color=c, fill=c] (4.02985,5.37182) rectangle (4.06965,5.47767);
\draw [color=c, fill=c] (4.06965,5.37182) rectangle (4.10945,5.47767);
\draw [color=c, fill=c] (4.10945,5.37182) rectangle (4.14925,5.47767);
\draw [color=c, fill=c] (4.14925,5.37182) rectangle (4.18905,5.47767);
\draw [color=c, fill=c] (4.18905,5.37182) rectangle (4.22886,5.47767);
\draw [color=c, fill=c] (4.22886,5.37182) rectangle (4.26866,5.47767);
\draw [color=c, fill=c] (4.26866,5.37182) rectangle (4.30846,5.47767);
\draw [color=c, fill=c] (4.30846,5.37182) rectangle (4.34826,5.47767);
\draw [color=c, fill=c] (4.34826,5.37182) rectangle (4.38806,5.47767);
\draw [color=c, fill=c] (4.38806,5.37182) rectangle (4.42786,5.47767);
\draw [color=c, fill=c] (4.42786,5.37182) rectangle (4.46766,5.47767);
\draw [color=c, fill=c] (4.46766,5.37182) rectangle (4.50746,5.47767);
\draw [color=c, fill=c] (4.50746,5.37182) rectangle (4.54726,5.47767);
\draw [color=c, fill=c] (4.54726,5.37182) rectangle (4.58706,5.47767);
\draw [color=c, fill=c] (4.58706,5.37182) rectangle (4.62687,5.47767);
\draw [color=c, fill=c] (4.62687,5.37182) rectangle (4.66667,5.47767);
\draw [color=c, fill=c] (4.66667,5.37182) rectangle (4.70647,5.47767);
\draw [color=c, fill=c] (4.70647,5.37182) rectangle (4.74627,5.47767);
\draw [color=c, fill=c] (4.74627,5.37182) rectangle (4.78607,5.47767);
\draw [color=c, fill=c] (4.78607,5.37182) rectangle (4.82587,5.47767);
\draw [color=c, fill=c] (4.82587,5.37182) rectangle (4.86567,5.47767);
\draw [color=c, fill=c] (4.86567,5.37182) rectangle (4.90547,5.47767);
\draw [color=c, fill=c] (4.90547,5.37182) rectangle (4.94527,5.47767);
\draw [color=c, fill=c] (4.94527,5.37182) rectangle (4.98507,5.47767);
\draw [color=c, fill=c] (4.98507,5.37182) rectangle (5.02488,5.47767);
\draw [color=c, fill=c] (5.02488,5.37182) rectangle (5.06468,5.47767);
\draw [color=c, fill=c] (5.06468,5.37182) rectangle (5.10448,5.47767);
\draw [color=c, fill=c] (5.10448,5.37182) rectangle (5.14428,5.47767);
\draw [color=c, fill=c] (5.14428,5.37182) rectangle (5.18408,5.47767);
\draw [color=c, fill=c] (5.18408,5.37182) rectangle (5.22388,5.47767);
\draw [color=c, fill=c] (5.22388,5.37182) rectangle (5.26368,5.47767);
\draw [color=c, fill=c] (5.26368,5.37182) rectangle (5.30348,5.47767);
\draw [color=c, fill=c] (5.30348,5.37182) rectangle (5.34328,5.47767);
\draw [color=c, fill=c] (5.34328,5.37182) rectangle (5.38308,5.47767);
\draw [color=c, fill=c] (5.38308,5.37182) rectangle (5.42289,5.47767);
\draw [color=c, fill=c] (5.42289,5.37182) rectangle (5.46269,5.47767);
\draw [color=c, fill=c] (5.46269,5.37182) rectangle (5.50249,5.47767);
\draw [color=c, fill=c] (5.50249,5.37182) rectangle (5.54229,5.47767);
\draw [color=c, fill=c] (5.54229,5.37182) rectangle (5.58209,5.47767);
\draw [color=c, fill=c] (5.58209,5.37182) rectangle (5.62189,5.47767);
\draw [color=c, fill=c] (5.62189,5.37182) rectangle (5.66169,5.47767);
\draw [color=c, fill=c] (5.66169,5.37182) rectangle (5.70149,5.47767);
\draw [color=c, fill=c] (5.70149,5.37182) rectangle (5.74129,5.47767);
\definecolor{c}{rgb}{0.2,0,1};
\draw [color=c, fill=c] (5.74129,5.37182) rectangle (5.78109,5.47767);
\draw [color=c, fill=c] (5.78109,5.37182) rectangle (5.8209,5.47767);
\draw [color=c, fill=c] (5.8209,5.37182) rectangle (5.8607,5.47767);
\draw [color=c, fill=c] (5.8607,5.37182) rectangle (5.9005,5.47767);
\draw [color=c, fill=c] (5.9005,5.37182) rectangle (5.9403,5.47767);
\draw [color=c, fill=c] (5.9403,5.37182) rectangle (5.9801,5.47767);
\draw [color=c, fill=c] (5.9801,5.37182) rectangle (6.0199,5.47767);
\draw [color=c, fill=c] (6.0199,5.37182) rectangle (6.0597,5.47767);
\draw [color=c, fill=c] (6.0597,5.37182) rectangle (6.0995,5.47767);
\draw [color=c, fill=c] (6.0995,5.37182) rectangle (6.1393,5.47767);
\draw [color=c, fill=c] (6.1393,5.37182) rectangle (6.1791,5.47767);
\draw [color=c, fill=c] (6.1791,5.37182) rectangle (6.21891,5.47767);
\draw [color=c, fill=c] (6.21891,5.37182) rectangle (6.25871,5.47767);
\draw [color=c, fill=c] (6.25871,5.37182) rectangle (6.29851,5.47767);
\draw [color=c, fill=c] (6.29851,5.37182) rectangle (6.33831,5.47767);
\draw [color=c, fill=c] (6.33831,5.37182) rectangle (6.37811,5.47767);
\draw [color=c, fill=c] (6.37811,5.37182) rectangle (6.41791,5.47767);
\draw [color=c, fill=c] (6.41791,5.37182) rectangle (6.45771,5.47767);
\draw [color=c, fill=c] (6.45771,5.37182) rectangle (6.49751,5.47767);
\draw [color=c, fill=c] (6.49751,5.37182) rectangle (6.53731,5.47767);
\draw [color=c, fill=c] (6.53731,5.37182) rectangle (6.57711,5.47767);
\draw [color=c, fill=c] (6.57711,5.37182) rectangle (6.61692,5.47767);
\draw [color=c, fill=c] (6.61692,5.37182) rectangle (6.65672,5.47767);
\draw [color=c, fill=c] (6.65672,5.37182) rectangle (6.69652,5.47767);
\draw [color=c, fill=c] (6.69652,5.37182) rectangle (6.73632,5.47767);
\draw [color=c, fill=c] (6.73632,5.37182) rectangle (6.77612,5.47767);
\draw [color=c, fill=c] (6.77612,5.37182) rectangle (6.81592,5.47767);
\draw [color=c, fill=c] (6.81592,5.37182) rectangle (6.85572,5.47767);
\draw [color=c, fill=c] (6.85572,5.37182) rectangle (6.89552,5.47767);
\draw [color=c, fill=c] (6.89552,5.37182) rectangle (6.93532,5.47767);
\draw [color=c, fill=c] (6.93532,5.37182) rectangle (6.97512,5.47767);
\draw [color=c, fill=c] (6.97512,5.37182) rectangle (7.01493,5.47767);
\draw [color=c, fill=c] (7.01493,5.37182) rectangle (7.05473,5.47767);
\draw [color=c, fill=c] (7.05473,5.37182) rectangle (7.09453,5.47767);
\draw [color=c, fill=c] (7.09453,5.37182) rectangle (7.13433,5.47767);
\draw [color=c, fill=c] (7.13433,5.37182) rectangle (7.17413,5.47767);
\draw [color=c, fill=c] (7.17413,5.37182) rectangle (7.21393,5.47767);
\draw [color=c, fill=c] (7.21393,5.37182) rectangle (7.25373,5.47767);
\draw [color=c, fill=c] (7.25373,5.37182) rectangle (7.29353,5.47767);
\draw [color=c, fill=c] (7.29353,5.37182) rectangle (7.33333,5.47767);
\draw [color=c, fill=c] (7.33333,5.37182) rectangle (7.37313,5.47767);
\draw [color=c, fill=c] (7.37313,5.37182) rectangle (7.41294,5.47767);
\draw [color=c, fill=c] (7.41294,5.37182) rectangle (7.45274,5.47767);
\draw [color=c, fill=c] (7.45274,5.37182) rectangle (7.49254,5.47767);
\draw [color=c, fill=c] (7.49254,5.37182) rectangle (7.53234,5.47767);
\draw [color=c, fill=c] (7.53234,5.37182) rectangle (7.57214,5.47767);
\draw [color=c, fill=c] (7.57214,5.37182) rectangle (7.61194,5.47767);
\draw [color=c, fill=c] (7.61194,5.37182) rectangle (7.65174,5.47767);
\draw [color=c, fill=c] (7.65174,5.37182) rectangle (7.69154,5.47767);
\draw [color=c, fill=c] (7.69154,5.37182) rectangle (7.73134,5.47767);
\draw [color=c, fill=c] (7.73134,5.37182) rectangle (7.77114,5.47767);
\draw [color=c, fill=c] (7.77114,5.37182) rectangle (7.81095,5.47767);
\draw [color=c, fill=c] (7.81095,5.37182) rectangle (7.85075,5.47767);
\draw [color=c, fill=c] (7.85075,5.37182) rectangle (7.89055,5.47767);
\draw [color=c, fill=c] (7.89055,5.37182) rectangle (7.93035,5.47767);
\draw [color=c, fill=c] (7.93035,5.37182) rectangle (7.97015,5.47767);
\draw [color=c, fill=c] (7.97015,5.37182) rectangle (8.00995,5.47767);
\draw [color=c, fill=c] (8.00995,5.37182) rectangle (8.04975,5.47767);
\draw [color=c, fill=c] (8.04975,5.37182) rectangle (8.08955,5.47767);
\draw [color=c, fill=c] (8.08955,5.37182) rectangle (8.12935,5.47767);
\draw [color=c, fill=c] (8.12935,5.37182) rectangle (8.16915,5.47767);
\draw [color=c, fill=c] (8.16915,5.37182) rectangle (8.20895,5.47767);
\draw [color=c, fill=c] (8.20895,5.37182) rectangle (8.24876,5.47767);
\draw [color=c, fill=c] (8.24876,5.37182) rectangle (8.28856,5.47767);
\draw [color=c, fill=c] (8.28856,5.37182) rectangle (8.32836,5.47767);
\draw [color=c, fill=c] (8.32836,5.37182) rectangle (8.36816,5.47767);
\draw [color=c, fill=c] (8.36816,5.37182) rectangle (8.40796,5.47767);
\draw [color=c, fill=c] (8.40796,5.37182) rectangle (8.44776,5.47767);
\draw [color=c, fill=c] (8.44776,5.37182) rectangle (8.48756,5.47767);
\draw [color=c, fill=c] (8.48756,5.37182) rectangle (8.52736,5.47767);
\draw [color=c, fill=c] (8.52736,5.37182) rectangle (8.56716,5.47767);
\draw [color=c, fill=c] (8.56716,5.37182) rectangle (8.60697,5.47767);
\draw [color=c, fill=c] (8.60697,5.37182) rectangle (8.64677,5.47767);
\draw [color=c, fill=c] (8.64677,5.37182) rectangle (8.68657,5.47767);
\draw [color=c, fill=c] (8.68657,5.37182) rectangle (8.72637,5.47767);
\draw [color=c, fill=c] (8.72637,5.37182) rectangle (8.76617,5.47767);
\draw [color=c, fill=c] (8.76617,5.37182) rectangle (8.80597,5.47767);
\draw [color=c, fill=c] (8.80597,5.37182) rectangle (8.84577,5.47767);
\draw [color=c, fill=c] (8.84577,5.37182) rectangle (8.88557,5.47767);
\draw [color=c, fill=c] (8.88557,5.37182) rectangle (8.92537,5.47767);
\draw [color=c, fill=c] (8.92537,5.37182) rectangle (8.96517,5.47767);
\draw [color=c, fill=c] (8.96517,5.37182) rectangle (9.00498,5.47767);
\draw [color=c, fill=c] (9.00498,5.37182) rectangle (9.04478,5.47767);
\draw [color=c, fill=c] (9.04478,5.37182) rectangle (9.08458,5.47767);
\draw [color=c, fill=c] (9.08458,5.37182) rectangle (9.12438,5.47767);
\draw [color=c, fill=c] (9.12438,5.37182) rectangle (9.16418,5.47767);
\draw [color=c, fill=c] (9.16418,5.37182) rectangle (9.20398,5.47767);
\draw [color=c, fill=c] (9.20398,5.37182) rectangle (9.24378,5.47767);
\draw [color=c, fill=c] (9.24378,5.37182) rectangle (9.28358,5.47767);
\draw [color=c, fill=c] (9.28358,5.37182) rectangle (9.32338,5.47767);
\draw [color=c, fill=c] (9.32338,5.37182) rectangle (9.36318,5.47767);
\definecolor{c}{rgb}{0,0.0800001,1};
\draw [color=c, fill=c] (9.36318,5.37182) rectangle (9.40298,5.47767);
\draw [color=c, fill=c] (9.40298,5.37182) rectangle (9.44279,5.47767);
\draw [color=c, fill=c] (9.44279,5.37182) rectangle (9.48259,5.47767);
\draw [color=c, fill=c] (9.48259,5.37182) rectangle (9.52239,5.47767);
\draw [color=c, fill=c] (9.52239,5.37182) rectangle (9.56219,5.47767);
\draw [color=c, fill=c] (9.56219,5.37182) rectangle (9.60199,5.47767);
\draw [color=c, fill=c] (9.60199,5.37182) rectangle (9.64179,5.47767);
\draw [color=c, fill=c] (9.64179,5.37182) rectangle (9.68159,5.47767);
\draw [color=c, fill=c] (9.68159,5.37182) rectangle (9.72139,5.47767);
\draw [color=c, fill=c] (9.72139,5.37182) rectangle (9.76119,5.47767);
\draw [color=c, fill=c] (9.76119,5.37182) rectangle (9.80099,5.47767);
\definecolor{c}{rgb}{0,0.266667,1};
\draw [color=c, fill=c] (9.80099,5.37182) rectangle (9.8408,5.47767);
\draw [color=c, fill=c] (9.8408,5.37182) rectangle (9.8806,5.47767);
\draw [color=c, fill=c] (9.8806,5.37182) rectangle (9.9204,5.47767);
\draw [color=c, fill=c] (9.9204,5.37182) rectangle (9.9602,5.47767);
\draw [color=c, fill=c] (9.9602,5.37182) rectangle (10,5.47767);
\draw [color=c, fill=c] (10,5.37182) rectangle (10.0398,5.47767);
\draw [color=c, fill=c] (10.0398,5.37182) rectangle (10.0796,5.47767);
\definecolor{c}{rgb}{0,0.546666,1};
\draw [color=c, fill=c] (10.0796,5.37182) rectangle (10.1194,5.47767);
\draw [color=c, fill=c] (10.1194,5.37182) rectangle (10.1592,5.47767);
\draw [color=c, fill=c] (10.1592,5.37182) rectangle (10.199,5.47767);
\draw [color=c, fill=c] (10.199,5.37182) rectangle (10.2388,5.47767);
\draw [color=c, fill=c] (10.2388,5.37182) rectangle (10.2786,5.47767);
\draw [color=c, fill=c] (10.2786,5.37182) rectangle (10.3184,5.47767);
\draw [color=c, fill=c] (10.3184,5.37182) rectangle (10.3582,5.47767);
\draw [color=c, fill=c] (10.3582,5.37182) rectangle (10.398,5.47767);
\definecolor{c}{rgb}{0,0.733333,1};
\draw [color=c, fill=c] (10.398,5.37182) rectangle (10.4378,5.47767);
\draw [color=c, fill=c] (10.4378,5.37182) rectangle (10.4776,5.47767);
\draw [color=c, fill=c] (10.4776,5.37182) rectangle (10.5174,5.47767);
\draw [color=c, fill=c] (10.5174,5.37182) rectangle (10.5572,5.47767);
\draw [color=c, fill=c] (10.5572,5.37182) rectangle (10.597,5.47767);
\draw [color=c, fill=c] (10.597,5.37182) rectangle (10.6368,5.47767);
\draw [color=c, fill=c] (10.6368,5.37182) rectangle (10.6766,5.47767);
\draw [color=c, fill=c] (10.6766,5.37182) rectangle (10.7164,5.47767);
\draw [color=c, fill=c] (10.7164,5.37182) rectangle (10.7562,5.47767);
\draw [color=c, fill=c] (10.7562,5.37182) rectangle (10.796,5.47767);
\draw [color=c, fill=c] (10.796,5.37182) rectangle (10.8358,5.47767);
\draw [color=c, fill=c] (10.8358,5.37182) rectangle (10.8756,5.47767);
\draw [color=c, fill=c] (10.8756,5.37182) rectangle (10.9154,5.47767);
\draw [color=c, fill=c] (10.9154,5.37182) rectangle (10.9552,5.47767);
\draw [color=c, fill=c] (10.9552,5.37182) rectangle (10.995,5.47767);
\draw [color=c, fill=c] (10.995,5.37182) rectangle (11.0348,5.47767);
\draw [color=c, fill=c] (11.0348,5.37182) rectangle (11.0746,5.47767);
\draw [color=c, fill=c] (11.0746,5.37182) rectangle (11.1144,5.47767);
\draw [color=c, fill=c] (11.1144,5.37182) rectangle (11.1542,5.47767);
\draw [color=c, fill=c] (11.1542,5.37182) rectangle (11.194,5.47767);
\draw [color=c, fill=c] (11.194,5.37182) rectangle (11.2338,5.47767);
\draw [color=c, fill=c] (11.2338,5.37182) rectangle (11.2736,5.47767);
\draw [color=c, fill=c] (11.2736,5.37182) rectangle (11.3134,5.47767);
\draw [color=c, fill=c] (11.3134,5.37182) rectangle (11.3532,5.47767);
\draw [color=c, fill=c] (11.3532,5.37182) rectangle (11.393,5.47767);
\draw [color=c, fill=c] (11.393,5.37182) rectangle (11.4328,5.47767);
\draw [color=c, fill=c] (11.4328,5.37182) rectangle (11.4726,5.47767);
\draw [color=c, fill=c] (11.4726,5.37182) rectangle (11.5124,5.47767);
\draw [color=c, fill=c] (11.5124,5.37182) rectangle (11.5522,5.47767);
\draw [color=c, fill=c] (11.5522,5.37182) rectangle (11.592,5.47767);
\draw [color=c, fill=c] (11.592,5.37182) rectangle (11.6318,5.47767);
\draw [color=c, fill=c] (11.6318,5.37182) rectangle (11.6716,5.47767);
\draw [color=c, fill=c] (11.6716,5.37182) rectangle (11.7114,5.47767);
\draw [color=c, fill=c] (11.7114,5.37182) rectangle (11.7512,5.47767);
\draw [color=c, fill=c] (11.7512,5.37182) rectangle (11.791,5.47767);
\draw [color=c, fill=c] (11.791,5.37182) rectangle (11.8308,5.47767);
\draw [color=c, fill=c] (11.8308,5.37182) rectangle (11.8706,5.47767);
\draw [color=c, fill=c] (11.8706,5.37182) rectangle (11.9104,5.47767);
\draw [color=c, fill=c] (11.9104,5.37182) rectangle (11.9502,5.47767);
\draw [color=c, fill=c] (11.9502,5.37182) rectangle (11.99,5.47767);
\draw [color=c, fill=c] (11.99,5.37182) rectangle (12.0299,5.47767);
\draw [color=c, fill=c] (12.0299,5.37182) rectangle (12.0697,5.47767);
\draw [color=c, fill=c] (12.0697,5.37182) rectangle (12.1095,5.47767);
\draw [color=c, fill=c] (12.1095,5.37182) rectangle (12.1493,5.47767);
\draw [color=c, fill=c] (12.1493,5.37182) rectangle (12.1891,5.47767);
\draw [color=c, fill=c] (12.1891,5.37182) rectangle (12.2289,5.47767);
\draw [color=c, fill=c] (12.2289,5.37182) rectangle (12.2687,5.47767);
\draw [color=c, fill=c] (12.2687,5.37182) rectangle (12.3085,5.47767);
\draw [color=c, fill=c] (12.3085,5.37182) rectangle (12.3483,5.47767);
\draw [color=c, fill=c] (12.3483,5.37182) rectangle (12.3881,5.47767);
\draw [color=c, fill=c] (12.3881,5.37182) rectangle (12.4279,5.47767);
\draw [color=c, fill=c] (12.4279,5.37182) rectangle (12.4677,5.47767);
\draw [color=c, fill=c] (12.4677,5.37182) rectangle (12.5075,5.47767);
\draw [color=c, fill=c] (12.5075,5.37182) rectangle (12.5473,5.47767);
\draw [color=c, fill=c] (12.5473,5.37182) rectangle (12.5871,5.47767);
\draw [color=c, fill=c] (12.5871,5.37182) rectangle (12.6269,5.47767);
\draw [color=c, fill=c] (12.6269,5.37182) rectangle (12.6667,5.47767);
\draw [color=c, fill=c] (12.6667,5.37182) rectangle (12.7065,5.47767);
\draw [color=c, fill=c] (12.7065,5.37182) rectangle (12.7463,5.47767);
\draw [color=c, fill=c] (12.7463,5.37182) rectangle (12.7861,5.47767);
\draw [color=c, fill=c] (12.7861,5.37182) rectangle (12.8259,5.47767);
\draw [color=c, fill=c] (12.8259,5.37182) rectangle (12.8657,5.47767);
\draw [color=c, fill=c] (12.8657,5.37182) rectangle (12.9055,5.47767);
\draw [color=c, fill=c] (12.9055,5.37182) rectangle (12.9453,5.47767);
\draw [color=c, fill=c] (12.9453,5.37182) rectangle (12.9851,5.47767);
\draw [color=c, fill=c] (12.9851,5.37182) rectangle (13.0249,5.47767);
\draw [color=c, fill=c] (13.0249,5.37182) rectangle (13.0647,5.47767);
\draw [color=c, fill=c] (13.0647,5.37182) rectangle (13.1045,5.47767);
\draw [color=c, fill=c] (13.1045,5.37182) rectangle (13.1443,5.47767);
\draw [color=c, fill=c] (13.1443,5.37182) rectangle (13.1841,5.47767);
\draw [color=c, fill=c] (13.1841,5.37182) rectangle (13.2239,5.47767);
\draw [color=c, fill=c] (13.2239,5.37182) rectangle (13.2637,5.47767);
\draw [color=c, fill=c] (13.2637,5.37182) rectangle (13.3035,5.47767);
\draw [color=c, fill=c] (13.3035,5.37182) rectangle (13.3433,5.47767);
\draw [color=c, fill=c] (13.3433,5.37182) rectangle (13.3831,5.47767);
\draw [color=c, fill=c] (13.3831,5.37182) rectangle (13.4229,5.47767);
\draw [color=c, fill=c] (13.4229,5.37182) rectangle (13.4627,5.47767);
\draw [color=c, fill=c] (13.4627,5.37182) rectangle (13.5025,5.47767);
\draw [color=c, fill=c] (13.5025,5.37182) rectangle (13.5423,5.47767);
\draw [color=c, fill=c] (13.5423,5.37182) rectangle (13.5821,5.47767);
\draw [color=c, fill=c] (13.5821,5.37182) rectangle (13.6219,5.47767);
\draw [color=c, fill=c] (13.6219,5.37182) rectangle (13.6617,5.47767);
\draw [color=c, fill=c] (13.6617,5.37182) rectangle (13.7015,5.47767);
\draw [color=c, fill=c] (13.7015,5.37182) rectangle (13.7413,5.47767);
\draw [color=c, fill=c] (13.7413,5.37182) rectangle (13.7811,5.47767);
\draw [color=c, fill=c] (13.7811,5.37182) rectangle (13.8209,5.47767);
\draw [color=c, fill=c] (13.8209,5.37182) rectangle (13.8607,5.47767);
\draw [color=c, fill=c] (13.8607,5.37182) rectangle (13.9005,5.47767);
\draw [color=c, fill=c] (13.9005,5.37182) rectangle (13.9403,5.47767);
\draw [color=c, fill=c] (13.9403,5.37182) rectangle (13.9801,5.47767);
\draw [color=c, fill=c] (13.9801,5.37182) rectangle (14.0199,5.47767);
\draw [color=c, fill=c] (14.0199,5.37182) rectangle (14.0597,5.47767);
\draw [color=c, fill=c] (14.0597,5.37182) rectangle (14.0995,5.47767);
\draw [color=c, fill=c] (14.0995,5.37182) rectangle (14.1393,5.47767);
\draw [color=c, fill=c] (14.1393,5.37182) rectangle (14.1791,5.47767);
\draw [color=c, fill=c] (14.1791,5.37182) rectangle (14.2189,5.47767);
\draw [color=c, fill=c] (14.2189,5.37182) rectangle (14.2587,5.47767);
\draw [color=c, fill=c] (14.2587,5.37182) rectangle (14.2985,5.47767);
\draw [color=c, fill=c] (14.2985,5.37182) rectangle (14.3383,5.47767);
\draw [color=c, fill=c] (14.3383,5.37182) rectangle (14.3781,5.47767);
\draw [color=c, fill=c] (14.3781,5.37182) rectangle (14.4179,5.47767);
\draw [color=c, fill=c] (14.4179,5.37182) rectangle (14.4577,5.47767);
\draw [color=c, fill=c] (14.4577,5.37182) rectangle (14.4975,5.47767);
\draw [color=c, fill=c] (14.4975,5.37182) rectangle (14.5373,5.47767);
\draw [color=c, fill=c] (14.5373,5.37182) rectangle (14.5771,5.47767);
\draw [color=c, fill=c] (14.5771,5.37182) rectangle (14.6169,5.47767);
\draw [color=c, fill=c] (14.6169,5.37182) rectangle (14.6567,5.47767);
\draw [color=c, fill=c] (14.6567,5.37182) rectangle (14.6965,5.47767);
\draw [color=c, fill=c] (14.6965,5.37182) rectangle (14.7363,5.47767);
\draw [color=c, fill=c] (14.7363,5.37182) rectangle (14.7761,5.47767);
\draw [color=c, fill=c] (14.7761,5.37182) rectangle (14.8159,5.47767);
\draw [color=c, fill=c] (14.8159,5.37182) rectangle (14.8557,5.47767);
\draw [color=c, fill=c] (14.8557,5.37182) rectangle (14.8955,5.47767);
\draw [color=c, fill=c] (14.8955,5.37182) rectangle (14.9353,5.47767);
\draw [color=c, fill=c] (14.9353,5.37182) rectangle (14.9751,5.47767);
\draw [color=c, fill=c] (14.9751,5.37182) rectangle (15.0149,5.47767);
\draw [color=c, fill=c] (15.0149,5.37182) rectangle (15.0547,5.47767);
\draw [color=c, fill=c] (15.0547,5.37182) rectangle (15.0945,5.47767);
\draw [color=c, fill=c] (15.0945,5.37182) rectangle (15.1343,5.47767);
\draw [color=c, fill=c] (15.1343,5.37182) rectangle (15.1741,5.47767);
\draw [color=c, fill=c] (15.1741,5.37182) rectangle (15.2139,5.47767);
\draw [color=c, fill=c] (15.2139,5.37182) rectangle (15.2537,5.47767);
\draw [color=c, fill=c] (15.2537,5.37182) rectangle (15.2935,5.47767);
\draw [color=c, fill=c] (15.2935,5.37182) rectangle (15.3333,5.47767);
\draw [color=c, fill=c] (15.3333,5.37182) rectangle (15.3731,5.47767);
\draw [color=c, fill=c] (15.3731,5.37182) rectangle (15.4129,5.47767);
\draw [color=c, fill=c] (15.4129,5.37182) rectangle (15.4527,5.47767);
\draw [color=c, fill=c] (15.4527,5.37182) rectangle (15.4925,5.47767);
\draw [color=c, fill=c] (15.4925,5.37182) rectangle (15.5323,5.47767);
\draw [color=c, fill=c] (15.5323,5.37182) rectangle (15.5721,5.47767);
\draw [color=c, fill=c] (15.5721,5.37182) rectangle (15.6119,5.47767);
\draw [color=c, fill=c] (15.6119,5.37182) rectangle (15.6517,5.47767);
\draw [color=c, fill=c] (15.6517,5.37182) rectangle (15.6915,5.47767);
\draw [color=c, fill=c] (15.6915,5.37182) rectangle (15.7313,5.47767);
\draw [color=c, fill=c] (15.7313,5.37182) rectangle (15.7711,5.47767);
\draw [color=c, fill=c] (15.7711,5.37182) rectangle (15.8109,5.47767);
\draw [color=c, fill=c] (15.8109,5.37182) rectangle (15.8507,5.47767);
\draw [color=c, fill=c] (15.8507,5.37182) rectangle (15.8905,5.47767);
\draw [color=c, fill=c] (15.8905,5.37182) rectangle (15.9303,5.47767);
\draw [color=c, fill=c] (15.9303,5.37182) rectangle (15.9701,5.47767);
\draw [color=c, fill=c] (15.9701,5.37182) rectangle (16.01,5.47767);
\draw [color=c, fill=c] (16.01,5.37182) rectangle (16.0498,5.47767);
\draw [color=c, fill=c] (16.0498,5.37182) rectangle (16.0896,5.47767);
\draw [color=c, fill=c] (16.0896,5.37182) rectangle (16.1294,5.47767);
\draw [color=c, fill=c] (16.1294,5.37182) rectangle (16.1692,5.47767);
\draw [color=c, fill=c] (16.1692,5.37182) rectangle (16.209,5.47767);
\draw [color=c, fill=c] (16.209,5.37182) rectangle (16.2488,5.47767);
\draw [color=c, fill=c] (16.2488,5.37182) rectangle (16.2886,5.47767);
\draw [color=c, fill=c] (16.2886,5.37182) rectangle (16.3284,5.47767);
\draw [color=c, fill=c] (16.3284,5.37182) rectangle (16.3682,5.47767);
\draw [color=c, fill=c] (16.3682,5.37182) rectangle (16.408,5.47767);
\draw [color=c, fill=c] (16.408,5.37182) rectangle (16.4478,5.47767);
\draw [color=c, fill=c] (16.4478,5.37182) rectangle (16.4876,5.47767);
\draw [color=c, fill=c] (16.4876,5.37182) rectangle (16.5274,5.47767);
\draw [color=c, fill=c] (16.5274,5.37182) rectangle (16.5672,5.47767);
\draw [color=c, fill=c] (16.5672,5.37182) rectangle (16.607,5.47767);
\draw [color=c, fill=c] (16.607,5.37182) rectangle (16.6468,5.47767);
\draw [color=c, fill=c] (16.6468,5.37182) rectangle (16.6866,5.47767);
\draw [color=c, fill=c] (16.6866,5.37182) rectangle (16.7264,5.47767);
\draw [color=c, fill=c] (16.7264,5.37182) rectangle (16.7662,5.47767);
\draw [color=c, fill=c] (16.7662,5.37182) rectangle (16.806,5.47767);
\draw [color=c, fill=c] (16.806,5.37182) rectangle (16.8458,5.47767);
\draw [color=c, fill=c] (16.8458,5.37182) rectangle (16.8856,5.47767);
\draw [color=c, fill=c] (16.8856,5.37182) rectangle (16.9254,5.47767);
\draw [color=c, fill=c] (16.9254,5.37182) rectangle (16.9652,5.47767);
\draw [color=c, fill=c] (16.9652,5.37182) rectangle (17.005,5.47767);
\draw [color=c, fill=c] (17.005,5.37182) rectangle (17.0448,5.47767);
\draw [color=c, fill=c] (17.0448,5.37182) rectangle (17.0846,5.47767);
\draw [color=c, fill=c] (17.0846,5.37182) rectangle (17.1244,5.47767);
\draw [color=c, fill=c] (17.1244,5.37182) rectangle (17.1642,5.47767);
\draw [color=c, fill=c] (17.1642,5.37182) rectangle (17.204,5.47767);
\draw [color=c, fill=c] (17.204,5.37182) rectangle (17.2438,5.47767);
\draw [color=c, fill=c] (17.2438,5.37182) rectangle (17.2836,5.47767);
\draw [color=c, fill=c] (17.2836,5.37182) rectangle (17.3234,5.47767);
\draw [color=c, fill=c] (17.3234,5.37182) rectangle (17.3632,5.47767);
\draw [color=c, fill=c] (17.3632,5.37182) rectangle (17.403,5.47767);
\draw [color=c, fill=c] (17.403,5.37182) rectangle (17.4428,5.47767);
\draw [color=c, fill=c] (17.4428,5.37182) rectangle (17.4826,5.47767);
\draw [color=c, fill=c] (17.4826,5.37182) rectangle (17.5224,5.47767);
\draw [color=c, fill=c] (17.5224,5.37182) rectangle (17.5622,5.47767);
\draw [color=c, fill=c] (17.5622,5.37182) rectangle (17.602,5.47767);
\draw [color=c, fill=c] (17.602,5.37182) rectangle (17.6418,5.47767);
\draw [color=c, fill=c] (17.6418,5.37182) rectangle (17.6816,5.47767);
\draw [color=c, fill=c] (17.6816,5.37182) rectangle (17.7214,5.47767);
\draw [color=c, fill=c] (17.7214,5.37182) rectangle (17.7612,5.47767);
\draw [color=c, fill=c] (17.7612,5.37182) rectangle (17.801,5.47767);
\draw [color=c, fill=c] (17.801,5.37182) rectangle (17.8408,5.47767);
\draw [color=c, fill=c] (17.8408,5.37182) rectangle (17.8806,5.47767);
\draw [color=c, fill=c] (17.8806,5.37182) rectangle (17.9204,5.47767);
\draw [color=c, fill=c] (17.9204,5.37182) rectangle (17.9602,5.47767);
\draw [color=c, fill=c] (17.9602,5.37182) rectangle (18,5.47767);
\definecolor{c}{rgb}{0,0.0800001,1};
\draw [color=c, fill=c] (2,5.47767) rectangle (2.0398,5.58352);
\draw [color=c, fill=c] (2.0398,5.47767) rectangle (2.0796,5.58352);
\draw [color=c, fill=c] (2.0796,5.47767) rectangle (2.1194,5.58352);
\draw [color=c, fill=c] (2.1194,5.47767) rectangle (2.1592,5.58352);
\draw [color=c, fill=c] (2.1592,5.47767) rectangle (2.19901,5.58352);
\draw [color=c, fill=c] (2.19901,5.47767) rectangle (2.23881,5.58352);
\draw [color=c, fill=c] (2.23881,5.47767) rectangle (2.27861,5.58352);
\draw [color=c, fill=c] (2.27861,5.47767) rectangle (2.31841,5.58352);
\draw [color=c, fill=c] (2.31841,5.47767) rectangle (2.35821,5.58352);
\draw [color=c, fill=c] (2.35821,5.47767) rectangle (2.39801,5.58352);
\draw [color=c, fill=c] (2.39801,5.47767) rectangle (2.43781,5.58352);
\draw [color=c, fill=c] (2.43781,5.47767) rectangle (2.47761,5.58352);
\draw [color=c, fill=c] (2.47761,5.47767) rectangle (2.51741,5.58352);
\draw [color=c, fill=c] (2.51741,5.47767) rectangle (2.55721,5.58352);
\draw [color=c, fill=c] (2.55721,5.47767) rectangle (2.59702,5.58352);
\draw [color=c, fill=c] (2.59702,5.47767) rectangle (2.63682,5.58352);
\draw [color=c, fill=c] (2.63682,5.47767) rectangle (2.67662,5.58352);
\draw [color=c, fill=c] (2.67662,5.47767) rectangle (2.71642,5.58352);
\draw [color=c, fill=c] (2.71642,5.47767) rectangle (2.75622,5.58352);
\draw [color=c, fill=c] (2.75622,5.47767) rectangle (2.79602,5.58352);
\draw [color=c, fill=c] (2.79602,5.47767) rectangle (2.83582,5.58352);
\draw [color=c, fill=c] (2.83582,5.47767) rectangle (2.87562,5.58352);
\draw [color=c, fill=c] (2.87562,5.47767) rectangle (2.91542,5.58352);
\draw [color=c, fill=c] (2.91542,5.47767) rectangle (2.95522,5.58352);
\draw [color=c, fill=c] (2.95522,5.47767) rectangle (2.99502,5.58352);
\draw [color=c, fill=c] (2.99502,5.47767) rectangle (3.03483,5.58352);
\draw [color=c, fill=c] (3.03483,5.47767) rectangle (3.07463,5.58352);
\draw [color=c, fill=c] (3.07463,5.47767) rectangle (3.11443,5.58352);
\draw [color=c, fill=c] (3.11443,5.47767) rectangle (3.15423,5.58352);
\draw [color=c, fill=c] (3.15423,5.47767) rectangle (3.19403,5.58352);
\draw [color=c, fill=c] (3.19403,5.47767) rectangle (3.23383,5.58352);
\draw [color=c, fill=c] (3.23383,5.47767) rectangle (3.27363,5.58352);
\draw [color=c, fill=c] (3.27363,5.47767) rectangle (3.31343,5.58352);
\draw [color=c, fill=c] (3.31343,5.47767) rectangle (3.35323,5.58352);
\draw [color=c, fill=c] (3.35323,5.47767) rectangle (3.39303,5.58352);
\draw [color=c, fill=c] (3.39303,5.47767) rectangle (3.43284,5.58352);
\draw [color=c, fill=c] (3.43284,5.47767) rectangle (3.47264,5.58352);
\draw [color=c, fill=c] (3.47264,5.47767) rectangle (3.51244,5.58352);
\draw [color=c, fill=c] (3.51244,5.47767) rectangle (3.55224,5.58352);
\draw [color=c, fill=c] (3.55224,5.47767) rectangle (3.59204,5.58352);
\draw [color=c, fill=c] (3.59204,5.47767) rectangle (3.63184,5.58352);
\draw [color=c, fill=c] (3.63184,5.47767) rectangle (3.67164,5.58352);
\draw [color=c, fill=c] (3.67164,5.47767) rectangle (3.71144,5.58352);
\draw [color=c, fill=c] (3.71144,5.47767) rectangle (3.75124,5.58352);
\draw [color=c, fill=c] (3.75124,5.47767) rectangle (3.79104,5.58352);
\draw [color=c, fill=c] (3.79104,5.47767) rectangle (3.83085,5.58352);
\draw [color=c, fill=c] (3.83085,5.47767) rectangle (3.87065,5.58352);
\draw [color=c, fill=c] (3.87065,5.47767) rectangle (3.91045,5.58352);
\draw [color=c, fill=c] (3.91045,5.47767) rectangle (3.95025,5.58352);
\draw [color=c, fill=c] (3.95025,5.47767) rectangle (3.99005,5.58352);
\draw [color=c, fill=c] (3.99005,5.47767) rectangle (4.02985,5.58352);
\draw [color=c, fill=c] (4.02985,5.47767) rectangle (4.06965,5.58352);
\draw [color=c, fill=c] (4.06965,5.47767) rectangle (4.10945,5.58352);
\draw [color=c, fill=c] (4.10945,5.47767) rectangle (4.14925,5.58352);
\draw [color=c, fill=c] (4.14925,5.47767) rectangle (4.18905,5.58352);
\draw [color=c, fill=c] (4.18905,5.47767) rectangle (4.22886,5.58352);
\draw [color=c, fill=c] (4.22886,5.47767) rectangle (4.26866,5.58352);
\draw [color=c, fill=c] (4.26866,5.47767) rectangle (4.30846,5.58352);
\draw [color=c, fill=c] (4.30846,5.47767) rectangle (4.34826,5.58352);
\draw [color=c, fill=c] (4.34826,5.47767) rectangle (4.38806,5.58352);
\draw [color=c, fill=c] (4.38806,5.47767) rectangle (4.42786,5.58352);
\draw [color=c, fill=c] (4.42786,5.47767) rectangle (4.46766,5.58352);
\draw [color=c, fill=c] (4.46766,5.47767) rectangle (4.50746,5.58352);
\draw [color=c, fill=c] (4.50746,5.47767) rectangle (4.54726,5.58352);
\draw [color=c, fill=c] (4.54726,5.47767) rectangle (4.58706,5.58352);
\draw [color=c, fill=c] (4.58706,5.47767) rectangle (4.62687,5.58352);
\draw [color=c, fill=c] (4.62687,5.47767) rectangle (4.66667,5.58352);
\draw [color=c, fill=c] (4.66667,5.47767) rectangle (4.70647,5.58352);
\draw [color=c, fill=c] (4.70647,5.47767) rectangle (4.74627,5.58352);
\draw [color=c, fill=c] (4.74627,5.47767) rectangle (4.78607,5.58352);
\draw [color=c, fill=c] (4.78607,5.47767) rectangle (4.82587,5.58352);
\draw [color=c, fill=c] (4.82587,5.47767) rectangle (4.86567,5.58352);
\draw [color=c, fill=c] (4.86567,5.47767) rectangle (4.90547,5.58352);
\draw [color=c, fill=c] (4.90547,5.47767) rectangle (4.94527,5.58352);
\draw [color=c, fill=c] (4.94527,5.47767) rectangle (4.98507,5.58352);
\draw [color=c, fill=c] (4.98507,5.47767) rectangle (5.02488,5.58352);
\draw [color=c, fill=c] (5.02488,5.47767) rectangle (5.06468,5.58352);
\draw [color=c, fill=c] (5.06468,5.47767) rectangle (5.10448,5.58352);
\draw [color=c, fill=c] (5.10448,5.47767) rectangle (5.14428,5.58352);
\draw [color=c, fill=c] (5.14428,5.47767) rectangle (5.18408,5.58352);
\draw [color=c, fill=c] (5.18408,5.47767) rectangle (5.22388,5.58352);
\draw [color=c, fill=c] (5.22388,5.47767) rectangle (5.26368,5.58352);
\draw [color=c, fill=c] (5.26368,5.47767) rectangle (5.30348,5.58352);
\draw [color=c, fill=c] (5.30348,5.47767) rectangle (5.34328,5.58352);
\draw [color=c, fill=c] (5.34328,5.47767) rectangle (5.38308,5.58352);
\draw [color=c, fill=c] (5.38308,5.47767) rectangle (5.42289,5.58352);
\draw [color=c, fill=c] (5.42289,5.47767) rectangle (5.46269,5.58352);
\draw [color=c, fill=c] (5.46269,5.47767) rectangle (5.50249,5.58352);
\draw [color=c, fill=c] (5.50249,5.47767) rectangle (5.54229,5.58352);
\draw [color=c, fill=c] (5.54229,5.47767) rectangle (5.58209,5.58352);
\draw [color=c, fill=c] (5.58209,5.47767) rectangle (5.62189,5.58352);
\draw [color=c, fill=c] (5.62189,5.47767) rectangle (5.66169,5.58352);
\draw [color=c, fill=c] (5.66169,5.47767) rectangle (5.70149,5.58352);
\definecolor{c}{rgb}{0.2,0,1};
\draw [color=c, fill=c] (5.70149,5.47767) rectangle (5.74129,5.58352);
\draw [color=c, fill=c] (5.74129,5.47767) rectangle (5.78109,5.58352);
\draw [color=c, fill=c] (5.78109,5.47767) rectangle (5.8209,5.58352);
\draw [color=c, fill=c] (5.8209,5.47767) rectangle (5.8607,5.58352);
\draw [color=c, fill=c] (5.8607,5.47767) rectangle (5.9005,5.58352);
\draw [color=c, fill=c] (5.9005,5.47767) rectangle (5.9403,5.58352);
\draw [color=c, fill=c] (5.9403,5.47767) rectangle (5.9801,5.58352);
\draw [color=c, fill=c] (5.9801,5.47767) rectangle (6.0199,5.58352);
\draw [color=c, fill=c] (6.0199,5.47767) rectangle (6.0597,5.58352);
\draw [color=c, fill=c] (6.0597,5.47767) rectangle (6.0995,5.58352);
\draw [color=c, fill=c] (6.0995,5.47767) rectangle (6.1393,5.58352);
\draw [color=c, fill=c] (6.1393,5.47767) rectangle (6.1791,5.58352);
\draw [color=c, fill=c] (6.1791,5.47767) rectangle (6.21891,5.58352);
\draw [color=c, fill=c] (6.21891,5.47767) rectangle (6.25871,5.58352);
\draw [color=c, fill=c] (6.25871,5.47767) rectangle (6.29851,5.58352);
\draw [color=c, fill=c] (6.29851,5.47767) rectangle (6.33831,5.58352);
\draw [color=c, fill=c] (6.33831,5.47767) rectangle (6.37811,5.58352);
\draw [color=c, fill=c] (6.37811,5.47767) rectangle (6.41791,5.58352);
\draw [color=c, fill=c] (6.41791,5.47767) rectangle (6.45771,5.58352);
\draw [color=c, fill=c] (6.45771,5.47767) rectangle (6.49751,5.58352);
\draw [color=c, fill=c] (6.49751,5.47767) rectangle (6.53731,5.58352);
\draw [color=c, fill=c] (6.53731,5.47767) rectangle (6.57711,5.58352);
\draw [color=c, fill=c] (6.57711,5.47767) rectangle (6.61692,5.58352);
\draw [color=c, fill=c] (6.61692,5.47767) rectangle (6.65672,5.58352);
\draw [color=c, fill=c] (6.65672,5.47767) rectangle (6.69652,5.58352);
\draw [color=c, fill=c] (6.69652,5.47767) rectangle (6.73632,5.58352);
\draw [color=c, fill=c] (6.73632,5.47767) rectangle (6.77612,5.58352);
\draw [color=c, fill=c] (6.77612,5.47767) rectangle (6.81592,5.58352);
\draw [color=c, fill=c] (6.81592,5.47767) rectangle (6.85572,5.58352);
\draw [color=c, fill=c] (6.85572,5.47767) rectangle (6.89552,5.58352);
\draw [color=c, fill=c] (6.89552,5.47767) rectangle (6.93532,5.58352);
\draw [color=c, fill=c] (6.93532,5.47767) rectangle (6.97512,5.58352);
\draw [color=c, fill=c] (6.97512,5.47767) rectangle (7.01493,5.58352);
\draw [color=c, fill=c] (7.01493,5.47767) rectangle (7.05473,5.58352);
\draw [color=c, fill=c] (7.05473,5.47767) rectangle (7.09453,5.58352);
\draw [color=c, fill=c] (7.09453,5.47767) rectangle (7.13433,5.58352);
\draw [color=c, fill=c] (7.13433,5.47767) rectangle (7.17413,5.58352);
\draw [color=c, fill=c] (7.17413,5.47767) rectangle (7.21393,5.58352);
\draw [color=c, fill=c] (7.21393,5.47767) rectangle (7.25373,5.58352);
\draw [color=c, fill=c] (7.25373,5.47767) rectangle (7.29353,5.58352);
\draw [color=c, fill=c] (7.29353,5.47767) rectangle (7.33333,5.58352);
\draw [color=c, fill=c] (7.33333,5.47767) rectangle (7.37313,5.58352);
\draw [color=c, fill=c] (7.37313,5.47767) rectangle (7.41294,5.58352);
\draw [color=c, fill=c] (7.41294,5.47767) rectangle (7.45274,5.58352);
\draw [color=c, fill=c] (7.45274,5.47767) rectangle (7.49254,5.58352);
\draw [color=c, fill=c] (7.49254,5.47767) rectangle (7.53234,5.58352);
\draw [color=c, fill=c] (7.53234,5.47767) rectangle (7.57214,5.58352);
\draw [color=c, fill=c] (7.57214,5.47767) rectangle (7.61194,5.58352);
\draw [color=c, fill=c] (7.61194,5.47767) rectangle (7.65174,5.58352);
\draw [color=c, fill=c] (7.65174,5.47767) rectangle (7.69154,5.58352);
\draw [color=c, fill=c] (7.69154,5.47767) rectangle (7.73134,5.58352);
\draw [color=c, fill=c] (7.73134,5.47767) rectangle (7.77114,5.58352);
\draw [color=c, fill=c] (7.77114,5.47767) rectangle (7.81095,5.58352);
\draw [color=c, fill=c] (7.81095,5.47767) rectangle (7.85075,5.58352);
\draw [color=c, fill=c] (7.85075,5.47767) rectangle (7.89055,5.58352);
\draw [color=c, fill=c] (7.89055,5.47767) rectangle (7.93035,5.58352);
\draw [color=c, fill=c] (7.93035,5.47767) rectangle (7.97015,5.58352);
\draw [color=c, fill=c] (7.97015,5.47767) rectangle (8.00995,5.58352);
\draw [color=c, fill=c] (8.00995,5.47767) rectangle (8.04975,5.58352);
\draw [color=c, fill=c] (8.04975,5.47767) rectangle (8.08955,5.58352);
\draw [color=c, fill=c] (8.08955,5.47767) rectangle (8.12935,5.58352);
\draw [color=c, fill=c] (8.12935,5.47767) rectangle (8.16915,5.58352);
\draw [color=c, fill=c] (8.16915,5.47767) rectangle (8.20895,5.58352);
\draw [color=c, fill=c] (8.20895,5.47767) rectangle (8.24876,5.58352);
\draw [color=c, fill=c] (8.24876,5.47767) rectangle (8.28856,5.58352);
\draw [color=c, fill=c] (8.28856,5.47767) rectangle (8.32836,5.58352);
\draw [color=c, fill=c] (8.32836,5.47767) rectangle (8.36816,5.58352);
\draw [color=c, fill=c] (8.36816,5.47767) rectangle (8.40796,5.58352);
\draw [color=c, fill=c] (8.40796,5.47767) rectangle (8.44776,5.58352);
\draw [color=c, fill=c] (8.44776,5.47767) rectangle (8.48756,5.58352);
\draw [color=c, fill=c] (8.48756,5.47767) rectangle (8.52736,5.58352);
\draw [color=c, fill=c] (8.52736,5.47767) rectangle (8.56716,5.58352);
\draw [color=c, fill=c] (8.56716,5.47767) rectangle (8.60697,5.58352);
\draw [color=c, fill=c] (8.60697,5.47767) rectangle (8.64677,5.58352);
\draw [color=c, fill=c] (8.64677,5.47767) rectangle (8.68657,5.58352);
\draw [color=c, fill=c] (8.68657,5.47767) rectangle (8.72637,5.58352);
\draw [color=c, fill=c] (8.72637,5.47767) rectangle (8.76617,5.58352);
\draw [color=c, fill=c] (8.76617,5.47767) rectangle (8.80597,5.58352);
\draw [color=c, fill=c] (8.80597,5.47767) rectangle (8.84577,5.58352);
\draw [color=c, fill=c] (8.84577,5.47767) rectangle (8.88557,5.58352);
\draw [color=c, fill=c] (8.88557,5.47767) rectangle (8.92537,5.58352);
\draw [color=c, fill=c] (8.92537,5.47767) rectangle (8.96517,5.58352);
\draw [color=c, fill=c] (8.96517,5.47767) rectangle (9.00498,5.58352);
\draw [color=c, fill=c] (9.00498,5.47767) rectangle (9.04478,5.58352);
\draw [color=c, fill=c] (9.04478,5.47767) rectangle (9.08458,5.58352);
\draw [color=c, fill=c] (9.08458,5.47767) rectangle (9.12438,5.58352);
\draw [color=c, fill=c] (9.12438,5.47767) rectangle (9.16418,5.58352);
\draw [color=c, fill=c] (9.16418,5.47767) rectangle (9.20398,5.58352);
\draw [color=c, fill=c] (9.20398,5.47767) rectangle (9.24378,5.58352);
\draw [color=c, fill=c] (9.24378,5.47767) rectangle (9.28358,5.58352);
\draw [color=c, fill=c] (9.28358,5.47767) rectangle (9.32338,5.58352);
\definecolor{c}{rgb}{0,0.0800001,1};
\draw [color=c, fill=c] (9.32338,5.47767) rectangle (9.36318,5.58352);
\draw [color=c, fill=c] (9.36318,5.47767) rectangle (9.40298,5.58352);
\draw [color=c, fill=c] (9.40298,5.47767) rectangle (9.44279,5.58352);
\draw [color=c, fill=c] (9.44279,5.47767) rectangle (9.48259,5.58352);
\draw [color=c, fill=c] (9.48259,5.47767) rectangle (9.52239,5.58352);
\draw [color=c, fill=c] (9.52239,5.47767) rectangle (9.56219,5.58352);
\draw [color=c, fill=c] (9.56219,5.47767) rectangle (9.60199,5.58352);
\draw [color=c, fill=c] (9.60199,5.47767) rectangle (9.64179,5.58352);
\draw [color=c, fill=c] (9.64179,5.47767) rectangle (9.68159,5.58352);
\draw [color=c, fill=c] (9.68159,5.47767) rectangle (9.72139,5.58352);
\draw [color=c, fill=c] (9.72139,5.47767) rectangle (9.76119,5.58352);
\definecolor{c}{rgb}{0,0.266667,1};
\draw [color=c, fill=c] (9.76119,5.47767) rectangle (9.80099,5.58352);
\draw [color=c, fill=c] (9.80099,5.47767) rectangle (9.8408,5.58352);
\draw [color=c, fill=c] (9.8408,5.47767) rectangle (9.8806,5.58352);
\draw [color=c, fill=c] (9.8806,5.47767) rectangle (9.9204,5.58352);
\draw [color=c, fill=c] (9.9204,5.47767) rectangle (9.9602,5.58352);
\draw [color=c, fill=c] (9.9602,5.47767) rectangle (10,5.58352);
\draw [color=c, fill=c] (10,5.47767) rectangle (10.0398,5.58352);
\draw [color=c, fill=c] (10.0398,5.47767) rectangle (10.0796,5.58352);
\definecolor{c}{rgb}{0,0.546666,1};
\draw [color=c, fill=c] (10.0796,5.47767) rectangle (10.1194,5.58352);
\draw [color=c, fill=c] (10.1194,5.47767) rectangle (10.1592,5.58352);
\draw [color=c, fill=c] (10.1592,5.47767) rectangle (10.199,5.58352);
\draw [color=c, fill=c] (10.199,5.47767) rectangle (10.2388,5.58352);
\draw [color=c, fill=c] (10.2388,5.47767) rectangle (10.2786,5.58352);
\draw [color=c, fill=c] (10.2786,5.47767) rectangle (10.3184,5.58352);
\draw [color=c, fill=c] (10.3184,5.47767) rectangle (10.3582,5.58352);
\draw [color=c, fill=c] (10.3582,5.47767) rectangle (10.398,5.58352);
\draw [color=c, fill=c] (10.398,5.47767) rectangle (10.4378,5.58352);
\draw [color=c, fill=c] (10.4378,5.47767) rectangle (10.4776,5.58352);
\definecolor{c}{rgb}{0,0.733333,1};
\draw [color=c, fill=c] (10.4776,5.47767) rectangle (10.5174,5.58352);
\draw [color=c, fill=c] (10.5174,5.47767) rectangle (10.5572,5.58352);
\draw [color=c, fill=c] (10.5572,5.47767) rectangle (10.597,5.58352);
\draw [color=c, fill=c] (10.597,5.47767) rectangle (10.6368,5.58352);
\draw [color=c, fill=c] (10.6368,5.47767) rectangle (10.6766,5.58352);
\draw [color=c, fill=c] (10.6766,5.47767) rectangle (10.7164,5.58352);
\draw [color=c, fill=c] (10.7164,5.47767) rectangle (10.7562,5.58352);
\draw [color=c, fill=c] (10.7562,5.47767) rectangle (10.796,5.58352);
\draw [color=c, fill=c] (10.796,5.47767) rectangle (10.8358,5.58352);
\draw [color=c, fill=c] (10.8358,5.47767) rectangle (10.8756,5.58352);
\draw [color=c, fill=c] (10.8756,5.47767) rectangle (10.9154,5.58352);
\draw [color=c, fill=c] (10.9154,5.47767) rectangle (10.9552,5.58352);
\draw [color=c, fill=c] (10.9552,5.47767) rectangle (10.995,5.58352);
\draw [color=c, fill=c] (10.995,5.47767) rectangle (11.0348,5.58352);
\draw [color=c, fill=c] (11.0348,5.47767) rectangle (11.0746,5.58352);
\draw [color=c, fill=c] (11.0746,5.47767) rectangle (11.1144,5.58352);
\draw [color=c, fill=c] (11.1144,5.47767) rectangle (11.1542,5.58352);
\draw [color=c, fill=c] (11.1542,5.47767) rectangle (11.194,5.58352);
\draw [color=c, fill=c] (11.194,5.47767) rectangle (11.2338,5.58352);
\draw [color=c, fill=c] (11.2338,5.47767) rectangle (11.2736,5.58352);
\draw [color=c, fill=c] (11.2736,5.47767) rectangle (11.3134,5.58352);
\draw [color=c, fill=c] (11.3134,5.47767) rectangle (11.3532,5.58352);
\draw [color=c, fill=c] (11.3532,5.47767) rectangle (11.393,5.58352);
\draw [color=c, fill=c] (11.393,5.47767) rectangle (11.4328,5.58352);
\draw [color=c, fill=c] (11.4328,5.47767) rectangle (11.4726,5.58352);
\draw [color=c, fill=c] (11.4726,5.47767) rectangle (11.5124,5.58352);
\draw [color=c, fill=c] (11.5124,5.47767) rectangle (11.5522,5.58352);
\draw [color=c, fill=c] (11.5522,5.47767) rectangle (11.592,5.58352);
\draw [color=c, fill=c] (11.592,5.47767) rectangle (11.6318,5.58352);
\draw [color=c, fill=c] (11.6318,5.47767) rectangle (11.6716,5.58352);
\draw [color=c, fill=c] (11.6716,5.47767) rectangle (11.7114,5.58352);
\draw [color=c, fill=c] (11.7114,5.47767) rectangle (11.7512,5.58352);
\draw [color=c, fill=c] (11.7512,5.47767) rectangle (11.791,5.58352);
\draw [color=c, fill=c] (11.791,5.47767) rectangle (11.8308,5.58352);
\draw [color=c, fill=c] (11.8308,5.47767) rectangle (11.8706,5.58352);
\draw [color=c, fill=c] (11.8706,5.47767) rectangle (11.9104,5.58352);
\draw [color=c, fill=c] (11.9104,5.47767) rectangle (11.9502,5.58352);
\draw [color=c, fill=c] (11.9502,5.47767) rectangle (11.99,5.58352);
\draw [color=c, fill=c] (11.99,5.47767) rectangle (12.0299,5.58352);
\draw [color=c, fill=c] (12.0299,5.47767) rectangle (12.0697,5.58352);
\draw [color=c, fill=c] (12.0697,5.47767) rectangle (12.1095,5.58352);
\draw [color=c, fill=c] (12.1095,5.47767) rectangle (12.1493,5.58352);
\draw [color=c, fill=c] (12.1493,5.47767) rectangle (12.1891,5.58352);
\draw [color=c, fill=c] (12.1891,5.47767) rectangle (12.2289,5.58352);
\draw [color=c, fill=c] (12.2289,5.47767) rectangle (12.2687,5.58352);
\draw [color=c, fill=c] (12.2687,5.47767) rectangle (12.3085,5.58352);
\draw [color=c, fill=c] (12.3085,5.47767) rectangle (12.3483,5.58352);
\draw [color=c, fill=c] (12.3483,5.47767) rectangle (12.3881,5.58352);
\draw [color=c, fill=c] (12.3881,5.47767) rectangle (12.4279,5.58352);
\draw [color=c, fill=c] (12.4279,5.47767) rectangle (12.4677,5.58352);
\draw [color=c, fill=c] (12.4677,5.47767) rectangle (12.5075,5.58352);
\draw [color=c, fill=c] (12.5075,5.47767) rectangle (12.5473,5.58352);
\draw [color=c, fill=c] (12.5473,5.47767) rectangle (12.5871,5.58352);
\draw [color=c, fill=c] (12.5871,5.47767) rectangle (12.6269,5.58352);
\draw [color=c, fill=c] (12.6269,5.47767) rectangle (12.6667,5.58352);
\draw [color=c, fill=c] (12.6667,5.47767) rectangle (12.7065,5.58352);
\draw [color=c, fill=c] (12.7065,5.47767) rectangle (12.7463,5.58352);
\draw [color=c, fill=c] (12.7463,5.47767) rectangle (12.7861,5.58352);
\draw [color=c, fill=c] (12.7861,5.47767) rectangle (12.8259,5.58352);
\draw [color=c, fill=c] (12.8259,5.47767) rectangle (12.8657,5.58352);
\draw [color=c, fill=c] (12.8657,5.47767) rectangle (12.9055,5.58352);
\draw [color=c, fill=c] (12.9055,5.47767) rectangle (12.9453,5.58352);
\draw [color=c, fill=c] (12.9453,5.47767) rectangle (12.9851,5.58352);
\draw [color=c, fill=c] (12.9851,5.47767) rectangle (13.0249,5.58352);
\draw [color=c, fill=c] (13.0249,5.47767) rectangle (13.0647,5.58352);
\draw [color=c, fill=c] (13.0647,5.47767) rectangle (13.1045,5.58352);
\draw [color=c, fill=c] (13.1045,5.47767) rectangle (13.1443,5.58352);
\draw [color=c, fill=c] (13.1443,5.47767) rectangle (13.1841,5.58352);
\draw [color=c, fill=c] (13.1841,5.47767) rectangle (13.2239,5.58352);
\draw [color=c, fill=c] (13.2239,5.47767) rectangle (13.2637,5.58352);
\draw [color=c, fill=c] (13.2637,5.47767) rectangle (13.3035,5.58352);
\draw [color=c, fill=c] (13.3035,5.47767) rectangle (13.3433,5.58352);
\draw [color=c, fill=c] (13.3433,5.47767) rectangle (13.3831,5.58352);
\draw [color=c, fill=c] (13.3831,5.47767) rectangle (13.4229,5.58352);
\draw [color=c, fill=c] (13.4229,5.47767) rectangle (13.4627,5.58352);
\draw [color=c, fill=c] (13.4627,5.47767) rectangle (13.5025,5.58352);
\draw [color=c, fill=c] (13.5025,5.47767) rectangle (13.5423,5.58352);
\draw [color=c, fill=c] (13.5423,5.47767) rectangle (13.5821,5.58352);
\draw [color=c, fill=c] (13.5821,5.47767) rectangle (13.6219,5.58352);
\draw [color=c, fill=c] (13.6219,5.47767) rectangle (13.6617,5.58352);
\draw [color=c, fill=c] (13.6617,5.47767) rectangle (13.7015,5.58352);
\draw [color=c, fill=c] (13.7015,5.47767) rectangle (13.7413,5.58352);
\draw [color=c, fill=c] (13.7413,5.47767) rectangle (13.7811,5.58352);
\draw [color=c, fill=c] (13.7811,5.47767) rectangle (13.8209,5.58352);
\draw [color=c, fill=c] (13.8209,5.47767) rectangle (13.8607,5.58352);
\draw [color=c, fill=c] (13.8607,5.47767) rectangle (13.9005,5.58352);
\draw [color=c, fill=c] (13.9005,5.47767) rectangle (13.9403,5.58352);
\draw [color=c, fill=c] (13.9403,5.47767) rectangle (13.9801,5.58352);
\draw [color=c, fill=c] (13.9801,5.47767) rectangle (14.0199,5.58352);
\draw [color=c, fill=c] (14.0199,5.47767) rectangle (14.0597,5.58352);
\draw [color=c, fill=c] (14.0597,5.47767) rectangle (14.0995,5.58352);
\draw [color=c, fill=c] (14.0995,5.47767) rectangle (14.1393,5.58352);
\draw [color=c, fill=c] (14.1393,5.47767) rectangle (14.1791,5.58352);
\draw [color=c, fill=c] (14.1791,5.47767) rectangle (14.2189,5.58352);
\draw [color=c, fill=c] (14.2189,5.47767) rectangle (14.2587,5.58352);
\draw [color=c, fill=c] (14.2587,5.47767) rectangle (14.2985,5.58352);
\draw [color=c, fill=c] (14.2985,5.47767) rectangle (14.3383,5.58352);
\draw [color=c, fill=c] (14.3383,5.47767) rectangle (14.3781,5.58352);
\draw [color=c, fill=c] (14.3781,5.47767) rectangle (14.4179,5.58352);
\draw [color=c, fill=c] (14.4179,5.47767) rectangle (14.4577,5.58352);
\draw [color=c, fill=c] (14.4577,5.47767) rectangle (14.4975,5.58352);
\draw [color=c, fill=c] (14.4975,5.47767) rectangle (14.5373,5.58352);
\draw [color=c, fill=c] (14.5373,5.47767) rectangle (14.5771,5.58352);
\draw [color=c, fill=c] (14.5771,5.47767) rectangle (14.6169,5.58352);
\draw [color=c, fill=c] (14.6169,5.47767) rectangle (14.6567,5.58352);
\draw [color=c, fill=c] (14.6567,5.47767) rectangle (14.6965,5.58352);
\draw [color=c, fill=c] (14.6965,5.47767) rectangle (14.7363,5.58352);
\draw [color=c, fill=c] (14.7363,5.47767) rectangle (14.7761,5.58352);
\draw [color=c, fill=c] (14.7761,5.47767) rectangle (14.8159,5.58352);
\draw [color=c, fill=c] (14.8159,5.47767) rectangle (14.8557,5.58352);
\draw [color=c, fill=c] (14.8557,5.47767) rectangle (14.8955,5.58352);
\draw [color=c, fill=c] (14.8955,5.47767) rectangle (14.9353,5.58352);
\draw [color=c, fill=c] (14.9353,5.47767) rectangle (14.9751,5.58352);
\draw [color=c, fill=c] (14.9751,5.47767) rectangle (15.0149,5.58352);
\draw [color=c, fill=c] (15.0149,5.47767) rectangle (15.0547,5.58352);
\draw [color=c, fill=c] (15.0547,5.47767) rectangle (15.0945,5.58352);
\draw [color=c, fill=c] (15.0945,5.47767) rectangle (15.1343,5.58352);
\draw [color=c, fill=c] (15.1343,5.47767) rectangle (15.1741,5.58352);
\draw [color=c, fill=c] (15.1741,5.47767) rectangle (15.2139,5.58352);
\draw [color=c, fill=c] (15.2139,5.47767) rectangle (15.2537,5.58352);
\draw [color=c, fill=c] (15.2537,5.47767) rectangle (15.2935,5.58352);
\draw [color=c, fill=c] (15.2935,5.47767) rectangle (15.3333,5.58352);
\draw [color=c, fill=c] (15.3333,5.47767) rectangle (15.3731,5.58352);
\draw [color=c, fill=c] (15.3731,5.47767) rectangle (15.4129,5.58352);
\draw [color=c, fill=c] (15.4129,5.47767) rectangle (15.4527,5.58352);
\draw [color=c, fill=c] (15.4527,5.47767) rectangle (15.4925,5.58352);
\draw [color=c, fill=c] (15.4925,5.47767) rectangle (15.5323,5.58352);
\draw [color=c, fill=c] (15.5323,5.47767) rectangle (15.5721,5.58352);
\draw [color=c, fill=c] (15.5721,5.47767) rectangle (15.6119,5.58352);
\draw [color=c, fill=c] (15.6119,5.47767) rectangle (15.6517,5.58352);
\draw [color=c, fill=c] (15.6517,5.47767) rectangle (15.6915,5.58352);
\draw [color=c, fill=c] (15.6915,5.47767) rectangle (15.7313,5.58352);
\draw [color=c, fill=c] (15.7313,5.47767) rectangle (15.7711,5.58352);
\draw [color=c, fill=c] (15.7711,5.47767) rectangle (15.8109,5.58352);
\draw [color=c, fill=c] (15.8109,5.47767) rectangle (15.8507,5.58352);
\draw [color=c, fill=c] (15.8507,5.47767) rectangle (15.8905,5.58352);
\draw [color=c, fill=c] (15.8905,5.47767) rectangle (15.9303,5.58352);
\draw [color=c, fill=c] (15.9303,5.47767) rectangle (15.9701,5.58352);
\draw [color=c, fill=c] (15.9701,5.47767) rectangle (16.01,5.58352);
\draw [color=c, fill=c] (16.01,5.47767) rectangle (16.0498,5.58352);
\draw [color=c, fill=c] (16.0498,5.47767) rectangle (16.0896,5.58352);
\draw [color=c, fill=c] (16.0896,5.47767) rectangle (16.1294,5.58352);
\draw [color=c, fill=c] (16.1294,5.47767) rectangle (16.1692,5.58352);
\draw [color=c, fill=c] (16.1692,5.47767) rectangle (16.209,5.58352);
\draw [color=c, fill=c] (16.209,5.47767) rectangle (16.2488,5.58352);
\draw [color=c, fill=c] (16.2488,5.47767) rectangle (16.2886,5.58352);
\draw [color=c, fill=c] (16.2886,5.47767) rectangle (16.3284,5.58352);
\draw [color=c, fill=c] (16.3284,5.47767) rectangle (16.3682,5.58352);
\draw [color=c, fill=c] (16.3682,5.47767) rectangle (16.408,5.58352);
\draw [color=c, fill=c] (16.408,5.47767) rectangle (16.4478,5.58352);
\draw [color=c, fill=c] (16.4478,5.47767) rectangle (16.4876,5.58352);
\draw [color=c, fill=c] (16.4876,5.47767) rectangle (16.5274,5.58352);
\draw [color=c, fill=c] (16.5274,5.47767) rectangle (16.5672,5.58352);
\draw [color=c, fill=c] (16.5672,5.47767) rectangle (16.607,5.58352);
\draw [color=c, fill=c] (16.607,5.47767) rectangle (16.6468,5.58352);
\draw [color=c, fill=c] (16.6468,5.47767) rectangle (16.6866,5.58352);
\draw [color=c, fill=c] (16.6866,5.47767) rectangle (16.7264,5.58352);
\draw [color=c, fill=c] (16.7264,5.47767) rectangle (16.7662,5.58352);
\draw [color=c, fill=c] (16.7662,5.47767) rectangle (16.806,5.58352);
\draw [color=c, fill=c] (16.806,5.47767) rectangle (16.8458,5.58352);
\draw [color=c, fill=c] (16.8458,5.47767) rectangle (16.8856,5.58352);
\draw [color=c, fill=c] (16.8856,5.47767) rectangle (16.9254,5.58352);
\draw [color=c, fill=c] (16.9254,5.47767) rectangle (16.9652,5.58352);
\draw [color=c, fill=c] (16.9652,5.47767) rectangle (17.005,5.58352);
\draw [color=c, fill=c] (17.005,5.47767) rectangle (17.0448,5.58352);
\draw [color=c, fill=c] (17.0448,5.47767) rectangle (17.0846,5.58352);
\draw [color=c, fill=c] (17.0846,5.47767) rectangle (17.1244,5.58352);
\draw [color=c, fill=c] (17.1244,5.47767) rectangle (17.1642,5.58352);
\draw [color=c, fill=c] (17.1642,5.47767) rectangle (17.204,5.58352);
\draw [color=c, fill=c] (17.204,5.47767) rectangle (17.2438,5.58352);
\draw [color=c, fill=c] (17.2438,5.47767) rectangle (17.2836,5.58352);
\draw [color=c, fill=c] (17.2836,5.47767) rectangle (17.3234,5.58352);
\draw [color=c, fill=c] (17.3234,5.47767) rectangle (17.3632,5.58352);
\draw [color=c, fill=c] (17.3632,5.47767) rectangle (17.403,5.58352);
\draw [color=c, fill=c] (17.403,5.47767) rectangle (17.4428,5.58352);
\draw [color=c, fill=c] (17.4428,5.47767) rectangle (17.4826,5.58352);
\draw [color=c, fill=c] (17.4826,5.47767) rectangle (17.5224,5.58352);
\draw [color=c, fill=c] (17.5224,5.47767) rectangle (17.5622,5.58352);
\draw [color=c, fill=c] (17.5622,5.47767) rectangle (17.602,5.58352);
\draw [color=c, fill=c] (17.602,5.47767) rectangle (17.6418,5.58352);
\draw [color=c, fill=c] (17.6418,5.47767) rectangle (17.6816,5.58352);
\draw [color=c, fill=c] (17.6816,5.47767) rectangle (17.7214,5.58352);
\draw [color=c, fill=c] (17.7214,5.47767) rectangle (17.7612,5.58352);
\draw [color=c, fill=c] (17.7612,5.47767) rectangle (17.801,5.58352);
\draw [color=c, fill=c] (17.801,5.47767) rectangle (17.8408,5.58352);
\draw [color=c, fill=c] (17.8408,5.47767) rectangle (17.8806,5.58352);
\draw [color=c, fill=c] (17.8806,5.47767) rectangle (17.9204,5.58352);
\draw [color=c, fill=c] (17.9204,5.47767) rectangle (17.9602,5.58352);
\draw [color=c, fill=c] (17.9602,5.47767) rectangle (18,5.58352);
\definecolor{c}{rgb}{0,0.0800001,1};
\draw [color=c, fill=c] (2,5.58352) rectangle (2.0398,5.68936);
\draw [color=c, fill=c] (2.0398,5.58352) rectangle (2.0796,5.68936);
\draw [color=c, fill=c] (2.0796,5.58352) rectangle (2.1194,5.68936);
\draw [color=c, fill=c] (2.1194,5.58352) rectangle (2.1592,5.68936);
\draw [color=c, fill=c] (2.1592,5.58352) rectangle (2.19901,5.68936);
\draw [color=c, fill=c] (2.19901,5.58352) rectangle (2.23881,5.68936);
\draw [color=c, fill=c] (2.23881,5.58352) rectangle (2.27861,5.68936);
\draw [color=c, fill=c] (2.27861,5.58352) rectangle (2.31841,5.68936);
\draw [color=c, fill=c] (2.31841,5.58352) rectangle (2.35821,5.68936);
\draw [color=c, fill=c] (2.35821,5.58352) rectangle (2.39801,5.68936);
\draw [color=c, fill=c] (2.39801,5.58352) rectangle (2.43781,5.68936);
\draw [color=c, fill=c] (2.43781,5.58352) rectangle (2.47761,5.68936);
\draw [color=c, fill=c] (2.47761,5.58352) rectangle (2.51741,5.68936);
\draw [color=c, fill=c] (2.51741,5.58352) rectangle (2.55721,5.68936);
\draw [color=c, fill=c] (2.55721,5.58352) rectangle (2.59702,5.68936);
\draw [color=c, fill=c] (2.59702,5.58352) rectangle (2.63682,5.68936);
\draw [color=c, fill=c] (2.63682,5.58352) rectangle (2.67662,5.68936);
\draw [color=c, fill=c] (2.67662,5.58352) rectangle (2.71642,5.68936);
\draw [color=c, fill=c] (2.71642,5.58352) rectangle (2.75622,5.68936);
\draw [color=c, fill=c] (2.75622,5.58352) rectangle (2.79602,5.68936);
\draw [color=c, fill=c] (2.79602,5.58352) rectangle (2.83582,5.68936);
\draw [color=c, fill=c] (2.83582,5.58352) rectangle (2.87562,5.68936);
\draw [color=c, fill=c] (2.87562,5.58352) rectangle (2.91542,5.68936);
\draw [color=c, fill=c] (2.91542,5.58352) rectangle (2.95522,5.68936);
\draw [color=c, fill=c] (2.95522,5.58352) rectangle (2.99502,5.68936);
\draw [color=c, fill=c] (2.99502,5.58352) rectangle (3.03483,5.68936);
\draw [color=c, fill=c] (3.03483,5.58352) rectangle (3.07463,5.68936);
\draw [color=c, fill=c] (3.07463,5.58352) rectangle (3.11443,5.68936);
\draw [color=c, fill=c] (3.11443,5.58352) rectangle (3.15423,5.68936);
\draw [color=c, fill=c] (3.15423,5.58352) rectangle (3.19403,5.68936);
\draw [color=c, fill=c] (3.19403,5.58352) rectangle (3.23383,5.68936);
\draw [color=c, fill=c] (3.23383,5.58352) rectangle (3.27363,5.68936);
\draw [color=c, fill=c] (3.27363,5.58352) rectangle (3.31343,5.68936);
\draw [color=c, fill=c] (3.31343,5.58352) rectangle (3.35323,5.68936);
\draw [color=c, fill=c] (3.35323,5.58352) rectangle (3.39303,5.68936);
\draw [color=c, fill=c] (3.39303,5.58352) rectangle (3.43284,5.68936);
\draw [color=c, fill=c] (3.43284,5.58352) rectangle (3.47264,5.68936);
\draw [color=c, fill=c] (3.47264,5.58352) rectangle (3.51244,5.68936);
\draw [color=c, fill=c] (3.51244,5.58352) rectangle (3.55224,5.68936);
\draw [color=c, fill=c] (3.55224,5.58352) rectangle (3.59204,5.68936);
\draw [color=c, fill=c] (3.59204,5.58352) rectangle (3.63184,5.68936);
\draw [color=c, fill=c] (3.63184,5.58352) rectangle (3.67164,5.68936);
\draw [color=c, fill=c] (3.67164,5.58352) rectangle (3.71144,5.68936);
\draw [color=c, fill=c] (3.71144,5.58352) rectangle (3.75124,5.68936);
\draw [color=c, fill=c] (3.75124,5.58352) rectangle (3.79104,5.68936);
\draw [color=c, fill=c] (3.79104,5.58352) rectangle (3.83085,5.68936);
\draw [color=c, fill=c] (3.83085,5.58352) rectangle (3.87065,5.68936);
\draw [color=c, fill=c] (3.87065,5.58352) rectangle (3.91045,5.68936);
\draw [color=c, fill=c] (3.91045,5.58352) rectangle (3.95025,5.68936);
\draw [color=c, fill=c] (3.95025,5.58352) rectangle (3.99005,5.68936);
\draw [color=c, fill=c] (3.99005,5.58352) rectangle (4.02985,5.68936);
\draw [color=c, fill=c] (4.02985,5.58352) rectangle (4.06965,5.68936);
\draw [color=c, fill=c] (4.06965,5.58352) rectangle (4.10945,5.68936);
\draw [color=c, fill=c] (4.10945,5.58352) rectangle (4.14925,5.68936);
\draw [color=c, fill=c] (4.14925,5.58352) rectangle (4.18905,5.68936);
\draw [color=c, fill=c] (4.18905,5.58352) rectangle (4.22886,5.68936);
\draw [color=c, fill=c] (4.22886,5.58352) rectangle (4.26866,5.68936);
\draw [color=c, fill=c] (4.26866,5.58352) rectangle (4.30846,5.68936);
\draw [color=c, fill=c] (4.30846,5.58352) rectangle (4.34826,5.68936);
\draw [color=c, fill=c] (4.34826,5.58352) rectangle (4.38806,5.68936);
\draw [color=c, fill=c] (4.38806,5.58352) rectangle (4.42786,5.68936);
\draw [color=c, fill=c] (4.42786,5.58352) rectangle (4.46766,5.68936);
\draw [color=c, fill=c] (4.46766,5.58352) rectangle (4.50746,5.68936);
\draw [color=c, fill=c] (4.50746,5.58352) rectangle (4.54726,5.68936);
\draw [color=c, fill=c] (4.54726,5.58352) rectangle (4.58706,5.68936);
\draw [color=c, fill=c] (4.58706,5.58352) rectangle (4.62687,5.68936);
\draw [color=c, fill=c] (4.62687,5.58352) rectangle (4.66667,5.68936);
\draw [color=c, fill=c] (4.66667,5.58352) rectangle (4.70647,5.68936);
\draw [color=c, fill=c] (4.70647,5.58352) rectangle (4.74627,5.68936);
\draw [color=c, fill=c] (4.74627,5.58352) rectangle (4.78607,5.68936);
\draw [color=c, fill=c] (4.78607,5.58352) rectangle (4.82587,5.68936);
\draw [color=c, fill=c] (4.82587,5.58352) rectangle (4.86567,5.68936);
\draw [color=c, fill=c] (4.86567,5.58352) rectangle (4.90547,5.68936);
\draw [color=c, fill=c] (4.90547,5.58352) rectangle (4.94527,5.68936);
\draw [color=c, fill=c] (4.94527,5.58352) rectangle (4.98507,5.68936);
\draw [color=c, fill=c] (4.98507,5.58352) rectangle (5.02488,5.68936);
\draw [color=c, fill=c] (5.02488,5.58352) rectangle (5.06468,5.68936);
\draw [color=c, fill=c] (5.06468,5.58352) rectangle (5.10448,5.68936);
\draw [color=c, fill=c] (5.10448,5.58352) rectangle (5.14428,5.68936);
\draw [color=c, fill=c] (5.14428,5.58352) rectangle (5.18408,5.68936);
\draw [color=c, fill=c] (5.18408,5.58352) rectangle (5.22388,5.68936);
\draw [color=c, fill=c] (5.22388,5.58352) rectangle (5.26368,5.68936);
\draw [color=c, fill=c] (5.26368,5.58352) rectangle (5.30348,5.68936);
\draw [color=c, fill=c] (5.30348,5.58352) rectangle (5.34328,5.68936);
\draw [color=c, fill=c] (5.34328,5.58352) rectangle (5.38308,5.68936);
\draw [color=c, fill=c] (5.38308,5.58352) rectangle (5.42289,5.68936);
\draw [color=c, fill=c] (5.42289,5.58352) rectangle (5.46269,5.68936);
\draw [color=c, fill=c] (5.46269,5.58352) rectangle (5.50249,5.68936);
\draw [color=c, fill=c] (5.50249,5.58352) rectangle (5.54229,5.68936);
\draw [color=c, fill=c] (5.54229,5.58352) rectangle (5.58209,5.68936);
\draw [color=c, fill=c] (5.58209,5.58352) rectangle (5.62189,5.68936);
\draw [color=c, fill=c] (5.62189,5.58352) rectangle (5.66169,5.68936);
\definecolor{c}{rgb}{0.2,0,1};
\draw [color=c, fill=c] (5.66169,5.58352) rectangle (5.70149,5.68936);
\draw [color=c, fill=c] (5.70149,5.58352) rectangle (5.74129,5.68936);
\draw [color=c, fill=c] (5.74129,5.58352) rectangle (5.78109,5.68936);
\draw [color=c, fill=c] (5.78109,5.58352) rectangle (5.8209,5.68936);
\draw [color=c, fill=c] (5.8209,5.58352) rectangle (5.8607,5.68936);
\draw [color=c, fill=c] (5.8607,5.58352) rectangle (5.9005,5.68936);
\draw [color=c, fill=c] (5.9005,5.58352) rectangle (5.9403,5.68936);
\draw [color=c, fill=c] (5.9403,5.58352) rectangle (5.9801,5.68936);
\draw [color=c, fill=c] (5.9801,5.58352) rectangle (6.0199,5.68936);
\draw [color=c, fill=c] (6.0199,5.58352) rectangle (6.0597,5.68936);
\draw [color=c, fill=c] (6.0597,5.58352) rectangle (6.0995,5.68936);
\draw [color=c, fill=c] (6.0995,5.58352) rectangle (6.1393,5.68936);
\draw [color=c, fill=c] (6.1393,5.58352) rectangle (6.1791,5.68936);
\draw [color=c, fill=c] (6.1791,5.58352) rectangle (6.21891,5.68936);
\draw [color=c, fill=c] (6.21891,5.58352) rectangle (6.25871,5.68936);
\draw [color=c, fill=c] (6.25871,5.58352) rectangle (6.29851,5.68936);
\draw [color=c, fill=c] (6.29851,5.58352) rectangle (6.33831,5.68936);
\draw [color=c, fill=c] (6.33831,5.58352) rectangle (6.37811,5.68936);
\draw [color=c, fill=c] (6.37811,5.58352) rectangle (6.41791,5.68936);
\draw [color=c, fill=c] (6.41791,5.58352) rectangle (6.45771,5.68936);
\draw [color=c, fill=c] (6.45771,5.58352) rectangle (6.49751,5.68936);
\draw [color=c, fill=c] (6.49751,5.58352) rectangle (6.53731,5.68936);
\draw [color=c, fill=c] (6.53731,5.58352) rectangle (6.57711,5.68936);
\draw [color=c, fill=c] (6.57711,5.58352) rectangle (6.61692,5.68936);
\draw [color=c, fill=c] (6.61692,5.58352) rectangle (6.65672,5.68936);
\draw [color=c, fill=c] (6.65672,5.58352) rectangle (6.69652,5.68936);
\draw [color=c, fill=c] (6.69652,5.58352) rectangle (6.73632,5.68936);
\draw [color=c, fill=c] (6.73632,5.58352) rectangle (6.77612,5.68936);
\draw [color=c, fill=c] (6.77612,5.58352) rectangle (6.81592,5.68936);
\draw [color=c, fill=c] (6.81592,5.58352) rectangle (6.85572,5.68936);
\draw [color=c, fill=c] (6.85572,5.58352) rectangle (6.89552,5.68936);
\draw [color=c, fill=c] (6.89552,5.58352) rectangle (6.93532,5.68936);
\draw [color=c, fill=c] (6.93532,5.58352) rectangle (6.97512,5.68936);
\draw [color=c, fill=c] (6.97512,5.58352) rectangle (7.01493,5.68936);
\draw [color=c, fill=c] (7.01493,5.58352) rectangle (7.05473,5.68936);
\draw [color=c, fill=c] (7.05473,5.58352) rectangle (7.09453,5.68936);
\draw [color=c, fill=c] (7.09453,5.58352) rectangle (7.13433,5.68936);
\draw [color=c, fill=c] (7.13433,5.58352) rectangle (7.17413,5.68936);
\draw [color=c, fill=c] (7.17413,5.58352) rectangle (7.21393,5.68936);
\draw [color=c, fill=c] (7.21393,5.58352) rectangle (7.25373,5.68936);
\draw [color=c, fill=c] (7.25373,5.58352) rectangle (7.29353,5.68936);
\draw [color=c, fill=c] (7.29353,5.58352) rectangle (7.33333,5.68936);
\draw [color=c, fill=c] (7.33333,5.58352) rectangle (7.37313,5.68936);
\draw [color=c, fill=c] (7.37313,5.58352) rectangle (7.41294,5.68936);
\draw [color=c, fill=c] (7.41294,5.58352) rectangle (7.45274,5.68936);
\draw [color=c, fill=c] (7.45274,5.58352) rectangle (7.49254,5.68936);
\draw [color=c, fill=c] (7.49254,5.58352) rectangle (7.53234,5.68936);
\draw [color=c, fill=c] (7.53234,5.58352) rectangle (7.57214,5.68936);
\draw [color=c, fill=c] (7.57214,5.58352) rectangle (7.61194,5.68936);
\draw [color=c, fill=c] (7.61194,5.58352) rectangle (7.65174,5.68936);
\draw [color=c, fill=c] (7.65174,5.58352) rectangle (7.69154,5.68936);
\draw [color=c, fill=c] (7.69154,5.58352) rectangle (7.73134,5.68936);
\draw [color=c, fill=c] (7.73134,5.58352) rectangle (7.77114,5.68936);
\draw [color=c, fill=c] (7.77114,5.58352) rectangle (7.81095,5.68936);
\draw [color=c, fill=c] (7.81095,5.58352) rectangle (7.85075,5.68936);
\draw [color=c, fill=c] (7.85075,5.58352) rectangle (7.89055,5.68936);
\draw [color=c, fill=c] (7.89055,5.58352) rectangle (7.93035,5.68936);
\draw [color=c, fill=c] (7.93035,5.58352) rectangle (7.97015,5.68936);
\draw [color=c, fill=c] (7.97015,5.58352) rectangle (8.00995,5.68936);
\draw [color=c, fill=c] (8.00995,5.58352) rectangle (8.04975,5.68936);
\draw [color=c, fill=c] (8.04975,5.58352) rectangle (8.08955,5.68936);
\draw [color=c, fill=c] (8.08955,5.58352) rectangle (8.12935,5.68936);
\draw [color=c, fill=c] (8.12935,5.58352) rectangle (8.16915,5.68936);
\draw [color=c, fill=c] (8.16915,5.58352) rectangle (8.20895,5.68936);
\draw [color=c, fill=c] (8.20895,5.58352) rectangle (8.24876,5.68936);
\draw [color=c, fill=c] (8.24876,5.58352) rectangle (8.28856,5.68936);
\draw [color=c, fill=c] (8.28856,5.58352) rectangle (8.32836,5.68936);
\draw [color=c, fill=c] (8.32836,5.58352) rectangle (8.36816,5.68936);
\draw [color=c, fill=c] (8.36816,5.58352) rectangle (8.40796,5.68936);
\draw [color=c, fill=c] (8.40796,5.58352) rectangle (8.44776,5.68936);
\draw [color=c, fill=c] (8.44776,5.58352) rectangle (8.48756,5.68936);
\draw [color=c, fill=c] (8.48756,5.58352) rectangle (8.52736,5.68936);
\draw [color=c, fill=c] (8.52736,5.58352) rectangle (8.56716,5.68936);
\draw [color=c, fill=c] (8.56716,5.58352) rectangle (8.60697,5.68936);
\draw [color=c, fill=c] (8.60697,5.58352) rectangle (8.64677,5.68936);
\draw [color=c, fill=c] (8.64677,5.58352) rectangle (8.68657,5.68936);
\draw [color=c, fill=c] (8.68657,5.58352) rectangle (8.72637,5.68936);
\draw [color=c, fill=c] (8.72637,5.58352) rectangle (8.76617,5.68936);
\draw [color=c, fill=c] (8.76617,5.58352) rectangle (8.80597,5.68936);
\draw [color=c, fill=c] (8.80597,5.58352) rectangle (8.84577,5.68936);
\draw [color=c, fill=c] (8.84577,5.58352) rectangle (8.88557,5.68936);
\draw [color=c, fill=c] (8.88557,5.58352) rectangle (8.92537,5.68936);
\draw [color=c, fill=c] (8.92537,5.58352) rectangle (8.96517,5.68936);
\draw [color=c, fill=c] (8.96517,5.58352) rectangle (9.00498,5.68936);
\draw [color=c, fill=c] (9.00498,5.58352) rectangle (9.04478,5.68936);
\draw [color=c, fill=c] (9.04478,5.58352) rectangle (9.08458,5.68936);
\draw [color=c, fill=c] (9.08458,5.58352) rectangle (9.12438,5.68936);
\draw [color=c, fill=c] (9.12438,5.58352) rectangle (9.16418,5.68936);
\draw [color=c, fill=c] (9.16418,5.58352) rectangle (9.20398,5.68936);
\draw [color=c, fill=c] (9.20398,5.58352) rectangle (9.24378,5.68936);
\definecolor{c}{rgb}{0,0.0800001,1};
\draw [color=c, fill=c] (9.24378,5.58352) rectangle (9.28358,5.68936);
\draw [color=c, fill=c] (9.28358,5.58352) rectangle (9.32338,5.68936);
\draw [color=c, fill=c] (9.32338,5.58352) rectangle (9.36318,5.68936);
\draw [color=c, fill=c] (9.36318,5.58352) rectangle (9.40298,5.68936);
\draw [color=c, fill=c] (9.40298,5.58352) rectangle (9.44279,5.68936);
\draw [color=c, fill=c] (9.44279,5.58352) rectangle (9.48259,5.68936);
\draw [color=c, fill=c] (9.48259,5.58352) rectangle (9.52239,5.68936);
\draw [color=c, fill=c] (9.52239,5.58352) rectangle (9.56219,5.68936);
\draw [color=c, fill=c] (9.56219,5.58352) rectangle (9.60199,5.68936);
\draw [color=c, fill=c] (9.60199,5.58352) rectangle (9.64179,5.68936);
\draw [color=c, fill=c] (9.64179,5.58352) rectangle (9.68159,5.68936);
\draw [color=c, fill=c] (9.68159,5.58352) rectangle (9.72139,5.68936);
\draw [color=c, fill=c] (9.72139,5.58352) rectangle (9.76119,5.68936);
\definecolor{c}{rgb}{0,0.266667,1};
\draw [color=c, fill=c] (9.76119,5.58352) rectangle (9.80099,5.68936);
\draw [color=c, fill=c] (9.80099,5.58352) rectangle (9.8408,5.68936);
\draw [color=c, fill=c] (9.8408,5.58352) rectangle (9.8806,5.68936);
\draw [color=c, fill=c] (9.8806,5.58352) rectangle (9.9204,5.68936);
\draw [color=c, fill=c] (9.9204,5.58352) rectangle (9.9602,5.68936);
\draw [color=c, fill=c] (9.9602,5.58352) rectangle (10,5.68936);
\draw [color=c, fill=c] (10,5.58352) rectangle (10.0398,5.68936);
\draw [color=c, fill=c] (10.0398,5.58352) rectangle (10.0796,5.68936);
\definecolor{c}{rgb}{0,0.546666,1};
\draw [color=c, fill=c] (10.0796,5.58352) rectangle (10.1194,5.68936);
\draw [color=c, fill=c] (10.1194,5.58352) rectangle (10.1592,5.68936);
\draw [color=c, fill=c] (10.1592,5.58352) rectangle (10.199,5.68936);
\draw [color=c, fill=c] (10.199,5.58352) rectangle (10.2388,5.68936);
\draw [color=c, fill=c] (10.2388,5.58352) rectangle (10.2786,5.68936);
\draw [color=c, fill=c] (10.2786,5.58352) rectangle (10.3184,5.68936);
\draw [color=c, fill=c] (10.3184,5.58352) rectangle (10.3582,5.68936);
\draw [color=c, fill=c] (10.3582,5.58352) rectangle (10.398,5.68936);
\draw [color=c, fill=c] (10.398,5.58352) rectangle (10.4378,5.68936);
\draw [color=c, fill=c] (10.4378,5.58352) rectangle (10.4776,5.68936);
\draw [color=c, fill=c] (10.4776,5.58352) rectangle (10.5174,5.68936);
\definecolor{c}{rgb}{0,0.733333,1};
\draw [color=c, fill=c] (10.5174,5.58352) rectangle (10.5572,5.68936);
\draw [color=c, fill=c] (10.5572,5.58352) rectangle (10.597,5.68936);
\draw [color=c, fill=c] (10.597,5.58352) rectangle (10.6368,5.68936);
\draw [color=c, fill=c] (10.6368,5.58352) rectangle (10.6766,5.68936);
\draw [color=c, fill=c] (10.6766,5.58352) rectangle (10.7164,5.68936);
\draw [color=c, fill=c] (10.7164,5.58352) rectangle (10.7562,5.68936);
\draw [color=c, fill=c] (10.7562,5.58352) rectangle (10.796,5.68936);
\draw [color=c, fill=c] (10.796,5.58352) rectangle (10.8358,5.68936);
\draw [color=c, fill=c] (10.8358,5.58352) rectangle (10.8756,5.68936);
\draw [color=c, fill=c] (10.8756,5.58352) rectangle (10.9154,5.68936);
\draw [color=c, fill=c] (10.9154,5.58352) rectangle (10.9552,5.68936);
\draw [color=c, fill=c] (10.9552,5.58352) rectangle (10.995,5.68936);
\draw [color=c, fill=c] (10.995,5.58352) rectangle (11.0348,5.68936);
\draw [color=c, fill=c] (11.0348,5.58352) rectangle (11.0746,5.68936);
\draw [color=c, fill=c] (11.0746,5.58352) rectangle (11.1144,5.68936);
\draw [color=c, fill=c] (11.1144,5.58352) rectangle (11.1542,5.68936);
\draw [color=c, fill=c] (11.1542,5.58352) rectangle (11.194,5.68936);
\draw [color=c, fill=c] (11.194,5.58352) rectangle (11.2338,5.68936);
\draw [color=c, fill=c] (11.2338,5.58352) rectangle (11.2736,5.68936);
\draw [color=c, fill=c] (11.2736,5.58352) rectangle (11.3134,5.68936);
\draw [color=c, fill=c] (11.3134,5.58352) rectangle (11.3532,5.68936);
\draw [color=c, fill=c] (11.3532,5.58352) rectangle (11.393,5.68936);
\draw [color=c, fill=c] (11.393,5.58352) rectangle (11.4328,5.68936);
\draw [color=c, fill=c] (11.4328,5.58352) rectangle (11.4726,5.68936);
\draw [color=c, fill=c] (11.4726,5.58352) rectangle (11.5124,5.68936);
\draw [color=c, fill=c] (11.5124,5.58352) rectangle (11.5522,5.68936);
\draw [color=c, fill=c] (11.5522,5.58352) rectangle (11.592,5.68936);
\draw [color=c, fill=c] (11.592,5.58352) rectangle (11.6318,5.68936);
\draw [color=c, fill=c] (11.6318,5.58352) rectangle (11.6716,5.68936);
\draw [color=c, fill=c] (11.6716,5.58352) rectangle (11.7114,5.68936);
\draw [color=c, fill=c] (11.7114,5.58352) rectangle (11.7512,5.68936);
\draw [color=c, fill=c] (11.7512,5.58352) rectangle (11.791,5.68936);
\draw [color=c, fill=c] (11.791,5.58352) rectangle (11.8308,5.68936);
\draw [color=c, fill=c] (11.8308,5.58352) rectangle (11.8706,5.68936);
\draw [color=c, fill=c] (11.8706,5.58352) rectangle (11.9104,5.68936);
\draw [color=c, fill=c] (11.9104,5.58352) rectangle (11.9502,5.68936);
\draw [color=c, fill=c] (11.9502,5.58352) rectangle (11.99,5.68936);
\draw [color=c, fill=c] (11.99,5.58352) rectangle (12.0299,5.68936);
\draw [color=c, fill=c] (12.0299,5.58352) rectangle (12.0697,5.68936);
\draw [color=c, fill=c] (12.0697,5.58352) rectangle (12.1095,5.68936);
\draw [color=c, fill=c] (12.1095,5.58352) rectangle (12.1493,5.68936);
\draw [color=c, fill=c] (12.1493,5.58352) rectangle (12.1891,5.68936);
\draw [color=c, fill=c] (12.1891,5.58352) rectangle (12.2289,5.68936);
\draw [color=c, fill=c] (12.2289,5.58352) rectangle (12.2687,5.68936);
\draw [color=c, fill=c] (12.2687,5.58352) rectangle (12.3085,5.68936);
\draw [color=c, fill=c] (12.3085,5.58352) rectangle (12.3483,5.68936);
\draw [color=c, fill=c] (12.3483,5.58352) rectangle (12.3881,5.68936);
\draw [color=c, fill=c] (12.3881,5.58352) rectangle (12.4279,5.68936);
\draw [color=c, fill=c] (12.4279,5.58352) rectangle (12.4677,5.68936);
\draw [color=c, fill=c] (12.4677,5.58352) rectangle (12.5075,5.68936);
\draw [color=c, fill=c] (12.5075,5.58352) rectangle (12.5473,5.68936);
\draw [color=c, fill=c] (12.5473,5.58352) rectangle (12.5871,5.68936);
\draw [color=c, fill=c] (12.5871,5.58352) rectangle (12.6269,5.68936);
\draw [color=c, fill=c] (12.6269,5.58352) rectangle (12.6667,5.68936);
\draw [color=c, fill=c] (12.6667,5.58352) rectangle (12.7065,5.68936);
\draw [color=c, fill=c] (12.7065,5.58352) rectangle (12.7463,5.68936);
\draw [color=c, fill=c] (12.7463,5.58352) rectangle (12.7861,5.68936);
\draw [color=c, fill=c] (12.7861,5.58352) rectangle (12.8259,5.68936);
\draw [color=c, fill=c] (12.8259,5.58352) rectangle (12.8657,5.68936);
\draw [color=c, fill=c] (12.8657,5.58352) rectangle (12.9055,5.68936);
\draw [color=c, fill=c] (12.9055,5.58352) rectangle (12.9453,5.68936);
\draw [color=c, fill=c] (12.9453,5.58352) rectangle (12.9851,5.68936);
\draw [color=c, fill=c] (12.9851,5.58352) rectangle (13.0249,5.68936);
\draw [color=c, fill=c] (13.0249,5.58352) rectangle (13.0647,5.68936);
\draw [color=c, fill=c] (13.0647,5.58352) rectangle (13.1045,5.68936);
\draw [color=c, fill=c] (13.1045,5.58352) rectangle (13.1443,5.68936);
\draw [color=c, fill=c] (13.1443,5.58352) rectangle (13.1841,5.68936);
\draw [color=c, fill=c] (13.1841,5.58352) rectangle (13.2239,5.68936);
\draw [color=c, fill=c] (13.2239,5.58352) rectangle (13.2637,5.68936);
\draw [color=c, fill=c] (13.2637,5.58352) rectangle (13.3035,5.68936);
\draw [color=c, fill=c] (13.3035,5.58352) rectangle (13.3433,5.68936);
\draw [color=c, fill=c] (13.3433,5.58352) rectangle (13.3831,5.68936);
\draw [color=c, fill=c] (13.3831,5.58352) rectangle (13.4229,5.68936);
\draw [color=c, fill=c] (13.4229,5.58352) rectangle (13.4627,5.68936);
\draw [color=c, fill=c] (13.4627,5.58352) rectangle (13.5025,5.68936);
\draw [color=c, fill=c] (13.5025,5.58352) rectangle (13.5423,5.68936);
\draw [color=c, fill=c] (13.5423,5.58352) rectangle (13.5821,5.68936);
\draw [color=c, fill=c] (13.5821,5.58352) rectangle (13.6219,5.68936);
\draw [color=c, fill=c] (13.6219,5.58352) rectangle (13.6617,5.68936);
\draw [color=c, fill=c] (13.6617,5.58352) rectangle (13.7015,5.68936);
\draw [color=c, fill=c] (13.7015,5.58352) rectangle (13.7413,5.68936);
\draw [color=c, fill=c] (13.7413,5.58352) rectangle (13.7811,5.68936);
\draw [color=c, fill=c] (13.7811,5.58352) rectangle (13.8209,5.68936);
\draw [color=c, fill=c] (13.8209,5.58352) rectangle (13.8607,5.68936);
\draw [color=c, fill=c] (13.8607,5.58352) rectangle (13.9005,5.68936);
\draw [color=c, fill=c] (13.9005,5.58352) rectangle (13.9403,5.68936);
\draw [color=c, fill=c] (13.9403,5.58352) rectangle (13.9801,5.68936);
\draw [color=c, fill=c] (13.9801,5.58352) rectangle (14.0199,5.68936);
\draw [color=c, fill=c] (14.0199,5.58352) rectangle (14.0597,5.68936);
\draw [color=c, fill=c] (14.0597,5.58352) rectangle (14.0995,5.68936);
\draw [color=c, fill=c] (14.0995,5.58352) rectangle (14.1393,5.68936);
\draw [color=c, fill=c] (14.1393,5.58352) rectangle (14.1791,5.68936);
\draw [color=c, fill=c] (14.1791,5.58352) rectangle (14.2189,5.68936);
\draw [color=c, fill=c] (14.2189,5.58352) rectangle (14.2587,5.68936);
\draw [color=c, fill=c] (14.2587,5.58352) rectangle (14.2985,5.68936);
\draw [color=c, fill=c] (14.2985,5.58352) rectangle (14.3383,5.68936);
\draw [color=c, fill=c] (14.3383,5.58352) rectangle (14.3781,5.68936);
\draw [color=c, fill=c] (14.3781,5.58352) rectangle (14.4179,5.68936);
\draw [color=c, fill=c] (14.4179,5.58352) rectangle (14.4577,5.68936);
\draw [color=c, fill=c] (14.4577,5.58352) rectangle (14.4975,5.68936);
\draw [color=c, fill=c] (14.4975,5.58352) rectangle (14.5373,5.68936);
\draw [color=c, fill=c] (14.5373,5.58352) rectangle (14.5771,5.68936);
\draw [color=c, fill=c] (14.5771,5.58352) rectangle (14.6169,5.68936);
\draw [color=c, fill=c] (14.6169,5.58352) rectangle (14.6567,5.68936);
\draw [color=c, fill=c] (14.6567,5.58352) rectangle (14.6965,5.68936);
\draw [color=c, fill=c] (14.6965,5.58352) rectangle (14.7363,5.68936);
\draw [color=c, fill=c] (14.7363,5.58352) rectangle (14.7761,5.68936);
\draw [color=c, fill=c] (14.7761,5.58352) rectangle (14.8159,5.68936);
\draw [color=c, fill=c] (14.8159,5.58352) rectangle (14.8557,5.68936);
\draw [color=c, fill=c] (14.8557,5.58352) rectangle (14.8955,5.68936);
\draw [color=c, fill=c] (14.8955,5.58352) rectangle (14.9353,5.68936);
\draw [color=c, fill=c] (14.9353,5.58352) rectangle (14.9751,5.68936);
\draw [color=c, fill=c] (14.9751,5.58352) rectangle (15.0149,5.68936);
\draw [color=c, fill=c] (15.0149,5.58352) rectangle (15.0547,5.68936);
\draw [color=c, fill=c] (15.0547,5.58352) rectangle (15.0945,5.68936);
\draw [color=c, fill=c] (15.0945,5.58352) rectangle (15.1343,5.68936);
\draw [color=c, fill=c] (15.1343,5.58352) rectangle (15.1741,5.68936);
\draw [color=c, fill=c] (15.1741,5.58352) rectangle (15.2139,5.68936);
\draw [color=c, fill=c] (15.2139,5.58352) rectangle (15.2537,5.68936);
\draw [color=c, fill=c] (15.2537,5.58352) rectangle (15.2935,5.68936);
\draw [color=c, fill=c] (15.2935,5.58352) rectangle (15.3333,5.68936);
\draw [color=c, fill=c] (15.3333,5.58352) rectangle (15.3731,5.68936);
\draw [color=c, fill=c] (15.3731,5.58352) rectangle (15.4129,5.68936);
\draw [color=c, fill=c] (15.4129,5.58352) rectangle (15.4527,5.68936);
\draw [color=c, fill=c] (15.4527,5.58352) rectangle (15.4925,5.68936);
\draw [color=c, fill=c] (15.4925,5.58352) rectangle (15.5323,5.68936);
\draw [color=c, fill=c] (15.5323,5.58352) rectangle (15.5721,5.68936);
\draw [color=c, fill=c] (15.5721,5.58352) rectangle (15.6119,5.68936);
\draw [color=c, fill=c] (15.6119,5.58352) rectangle (15.6517,5.68936);
\draw [color=c, fill=c] (15.6517,5.58352) rectangle (15.6915,5.68936);
\draw [color=c, fill=c] (15.6915,5.58352) rectangle (15.7313,5.68936);
\draw [color=c, fill=c] (15.7313,5.58352) rectangle (15.7711,5.68936);
\draw [color=c, fill=c] (15.7711,5.58352) rectangle (15.8109,5.68936);
\draw [color=c, fill=c] (15.8109,5.58352) rectangle (15.8507,5.68936);
\draw [color=c, fill=c] (15.8507,5.58352) rectangle (15.8905,5.68936);
\draw [color=c, fill=c] (15.8905,5.58352) rectangle (15.9303,5.68936);
\draw [color=c, fill=c] (15.9303,5.58352) rectangle (15.9701,5.68936);
\draw [color=c, fill=c] (15.9701,5.58352) rectangle (16.01,5.68936);
\draw [color=c, fill=c] (16.01,5.58352) rectangle (16.0498,5.68936);
\draw [color=c, fill=c] (16.0498,5.58352) rectangle (16.0896,5.68936);
\draw [color=c, fill=c] (16.0896,5.58352) rectangle (16.1294,5.68936);
\draw [color=c, fill=c] (16.1294,5.58352) rectangle (16.1692,5.68936);
\draw [color=c, fill=c] (16.1692,5.58352) rectangle (16.209,5.68936);
\draw [color=c, fill=c] (16.209,5.58352) rectangle (16.2488,5.68936);
\draw [color=c, fill=c] (16.2488,5.58352) rectangle (16.2886,5.68936);
\draw [color=c, fill=c] (16.2886,5.58352) rectangle (16.3284,5.68936);
\draw [color=c, fill=c] (16.3284,5.58352) rectangle (16.3682,5.68936);
\draw [color=c, fill=c] (16.3682,5.58352) rectangle (16.408,5.68936);
\draw [color=c, fill=c] (16.408,5.58352) rectangle (16.4478,5.68936);
\draw [color=c, fill=c] (16.4478,5.58352) rectangle (16.4876,5.68936);
\draw [color=c, fill=c] (16.4876,5.58352) rectangle (16.5274,5.68936);
\draw [color=c, fill=c] (16.5274,5.58352) rectangle (16.5672,5.68936);
\draw [color=c, fill=c] (16.5672,5.58352) rectangle (16.607,5.68936);
\draw [color=c, fill=c] (16.607,5.58352) rectangle (16.6468,5.68936);
\draw [color=c, fill=c] (16.6468,5.58352) rectangle (16.6866,5.68936);
\draw [color=c, fill=c] (16.6866,5.58352) rectangle (16.7264,5.68936);
\draw [color=c, fill=c] (16.7264,5.58352) rectangle (16.7662,5.68936);
\draw [color=c, fill=c] (16.7662,5.58352) rectangle (16.806,5.68936);
\draw [color=c, fill=c] (16.806,5.58352) rectangle (16.8458,5.68936);
\draw [color=c, fill=c] (16.8458,5.58352) rectangle (16.8856,5.68936);
\draw [color=c, fill=c] (16.8856,5.58352) rectangle (16.9254,5.68936);
\draw [color=c, fill=c] (16.9254,5.58352) rectangle (16.9652,5.68936);
\draw [color=c, fill=c] (16.9652,5.58352) rectangle (17.005,5.68936);
\draw [color=c, fill=c] (17.005,5.58352) rectangle (17.0448,5.68936);
\draw [color=c, fill=c] (17.0448,5.58352) rectangle (17.0846,5.68936);
\draw [color=c, fill=c] (17.0846,5.58352) rectangle (17.1244,5.68936);
\draw [color=c, fill=c] (17.1244,5.58352) rectangle (17.1642,5.68936);
\draw [color=c, fill=c] (17.1642,5.58352) rectangle (17.204,5.68936);
\draw [color=c, fill=c] (17.204,5.58352) rectangle (17.2438,5.68936);
\draw [color=c, fill=c] (17.2438,5.58352) rectangle (17.2836,5.68936);
\draw [color=c, fill=c] (17.2836,5.58352) rectangle (17.3234,5.68936);
\draw [color=c, fill=c] (17.3234,5.58352) rectangle (17.3632,5.68936);
\draw [color=c, fill=c] (17.3632,5.58352) rectangle (17.403,5.68936);
\draw [color=c, fill=c] (17.403,5.58352) rectangle (17.4428,5.68936);
\draw [color=c, fill=c] (17.4428,5.58352) rectangle (17.4826,5.68936);
\draw [color=c, fill=c] (17.4826,5.58352) rectangle (17.5224,5.68936);
\draw [color=c, fill=c] (17.5224,5.58352) rectangle (17.5622,5.68936);
\draw [color=c, fill=c] (17.5622,5.58352) rectangle (17.602,5.68936);
\draw [color=c, fill=c] (17.602,5.58352) rectangle (17.6418,5.68936);
\draw [color=c, fill=c] (17.6418,5.58352) rectangle (17.6816,5.68936);
\draw [color=c, fill=c] (17.6816,5.58352) rectangle (17.7214,5.68936);
\draw [color=c, fill=c] (17.7214,5.58352) rectangle (17.7612,5.68936);
\draw [color=c, fill=c] (17.7612,5.58352) rectangle (17.801,5.68936);
\draw [color=c, fill=c] (17.801,5.58352) rectangle (17.8408,5.68936);
\draw [color=c, fill=c] (17.8408,5.58352) rectangle (17.8806,5.68936);
\draw [color=c, fill=c] (17.8806,5.58352) rectangle (17.9204,5.68936);
\draw [color=c, fill=c] (17.9204,5.58352) rectangle (17.9602,5.68936);
\draw [color=c, fill=c] (17.9602,5.58352) rectangle (18,5.68936);
\definecolor{c}{rgb}{0,0.0800001,1};
\draw [color=c, fill=c] (2,5.68936) rectangle (2.0398,5.79521);
\draw [color=c, fill=c] (2.0398,5.68936) rectangle (2.0796,5.79521);
\draw [color=c, fill=c] (2.0796,5.68936) rectangle (2.1194,5.79521);
\draw [color=c, fill=c] (2.1194,5.68936) rectangle (2.1592,5.79521);
\draw [color=c, fill=c] (2.1592,5.68936) rectangle (2.19901,5.79521);
\draw [color=c, fill=c] (2.19901,5.68936) rectangle (2.23881,5.79521);
\draw [color=c, fill=c] (2.23881,5.68936) rectangle (2.27861,5.79521);
\draw [color=c, fill=c] (2.27861,5.68936) rectangle (2.31841,5.79521);
\draw [color=c, fill=c] (2.31841,5.68936) rectangle (2.35821,5.79521);
\draw [color=c, fill=c] (2.35821,5.68936) rectangle (2.39801,5.79521);
\draw [color=c, fill=c] (2.39801,5.68936) rectangle (2.43781,5.79521);
\draw [color=c, fill=c] (2.43781,5.68936) rectangle (2.47761,5.79521);
\draw [color=c, fill=c] (2.47761,5.68936) rectangle (2.51741,5.79521);
\draw [color=c, fill=c] (2.51741,5.68936) rectangle (2.55721,5.79521);
\draw [color=c, fill=c] (2.55721,5.68936) rectangle (2.59702,5.79521);
\draw [color=c, fill=c] (2.59702,5.68936) rectangle (2.63682,5.79521);
\draw [color=c, fill=c] (2.63682,5.68936) rectangle (2.67662,5.79521);
\draw [color=c, fill=c] (2.67662,5.68936) rectangle (2.71642,5.79521);
\draw [color=c, fill=c] (2.71642,5.68936) rectangle (2.75622,5.79521);
\draw [color=c, fill=c] (2.75622,5.68936) rectangle (2.79602,5.79521);
\draw [color=c, fill=c] (2.79602,5.68936) rectangle (2.83582,5.79521);
\draw [color=c, fill=c] (2.83582,5.68936) rectangle (2.87562,5.79521);
\draw [color=c, fill=c] (2.87562,5.68936) rectangle (2.91542,5.79521);
\draw [color=c, fill=c] (2.91542,5.68936) rectangle (2.95522,5.79521);
\draw [color=c, fill=c] (2.95522,5.68936) rectangle (2.99502,5.79521);
\draw [color=c, fill=c] (2.99502,5.68936) rectangle (3.03483,5.79521);
\draw [color=c, fill=c] (3.03483,5.68936) rectangle (3.07463,5.79521);
\draw [color=c, fill=c] (3.07463,5.68936) rectangle (3.11443,5.79521);
\draw [color=c, fill=c] (3.11443,5.68936) rectangle (3.15423,5.79521);
\draw [color=c, fill=c] (3.15423,5.68936) rectangle (3.19403,5.79521);
\draw [color=c, fill=c] (3.19403,5.68936) rectangle (3.23383,5.79521);
\draw [color=c, fill=c] (3.23383,5.68936) rectangle (3.27363,5.79521);
\draw [color=c, fill=c] (3.27363,5.68936) rectangle (3.31343,5.79521);
\draw [color=c, fill=c] (3.31343,5.68936) rectangle (3.35323,5.79521);
\draw [color=c, fill=c] (3.35323,5.68936) rectangle (3.39303,5.79521);
\draw [color=c, fill=c] (3.39303,5.68936) rectangle (3.43284,5.79521);
\draw [color=c, fill=c] (3.43284,5.68936) rectangle (3.47264,5.79521);
\draw [color=c, fill=c] (3.47264,5.68936) rectangle (3.51244,5.79521);
\draw [color=c, fill=c] (3.51244,5.68936) rectangle (3.55224,5.79521);
\draw [color=c, fill=c] (3.55224,5.68936) rectangle (3.59204,5.79521);
\draw [color=c, fill=c] (3.59204,5.68936) rectangle (3.63184,5.79521);
\draw [color=c, fill=c] (3.63184,5.68936) rectangle (3.67164,5.79521);
\draw [color=c, fill=c] (3.67164,5.68936) rectangle (3.71144,5.79521);
\draw [color=c, fill=c] (3.71144,5.68936) rectangle (3.75124,5.79521);
\draw [color=c, fill=c] (3.75124,5.68936) rectangle (3.79104,5.79521);
\draw [color=c, fill=c] (3.79104,5.68936) rectangle (3.83085,5.79521);
\draw [color=c, fill=c] (3.83085,5.68936) rectangle (3.87065,5.79521);
\draw [color=c, fill=c] (3.87065,5.68936) rectangle (3.91045,5.79521);
\draw [color=c, fill=c] (3.91045,5.68936) rectangle (3.95025,5.79521);
\draw [color=c, fill=c] (3.95025,5.68936) rectangle (3.99005,5.79521);
\draw [color=c, fill=c] (3.99005,5.68936) rectangle (4.02985,5.79521);
\draw [color=c, fill=c] (4.02985,5.68936) rectangle (4.06965,5.79521);
\draw [color=c, fill=c] (4.06965,5.68936) rectangle (4.10945,5.79521);
\draw [color=c, fill=c] (4.10945,5.68936) rectangle (4.14925,5.79521);
\draw [color=c, fill=c] (4.14925,5.68936) rectangle (4.18905,5.79521);
\draw [color=c, fill=c] (4.18905,5.68936) rectangle (4.22886,5.79521);
\draw [color=c, fill=c] (4.22886,5.68936) rectangle (4.26866,5.79521);
\draw [color=c, fill=c] (4.26866,5.68936) rectangle (4.30846,5.79521);
\draw [color=c, fill=c] (4.30846,5.68936) rectangle (4.34826,5.79521);
\draw [color=c, fill=c] (4.34826,5.68936) rectangle (4.38806,5.79521);
\draw [color=c, fill=c] (4.38806,5.68936) rectangle (4.42786,5.79521);
\draw [color=c, fill=c] (4.42786,5.68936) rectangle (4.46766,5.79521);
\draw [color=c, fill=c] (4.46766,5.68936) rectangle (4.50746,5.79521);
\draw [color=c, fill=c] (4.50746,5.68936) rectangle (4.54726,5.79521);
\draw [color=c, fill=c] (4.54726,5.68936) rectangle (4.58706,5.79521);
\draw [color=c, fill=c] (4.58706,5.68936) rectangle (4.62687,5.79521);
\draw [color=c, fill=c] (4.62687,5.68936) rectangle (4.66667,5.79521);
\draw [color=c, fill=c] (4.66667,5.68936) rectangle (4.70647,5.79521);
\draw [color=c, fill=c] (4.70647,5.68936) rectangle (4.74627,5.79521);
\draw [color=c, fill=c] (4.74627,5.68936) rectangle (4.78607,5.79521);
\draw [color=c, fill=c] (4.78607,5.68936) rectangle (4.82587,5.79521);
\draw [color=c, fill=c] (4.82587,5.68936) rectangle (4.86567,5.79521);
\draw [color=c, fill=c] (4.86567,5.68936) rectangle (4.90547,5.79521);
\draw [color=c, fill=c] (4.90547,5.68936) rectangle (4.94527,5.79521);
\draw [color=c, fill=c] (4.94527,5.68936) rectangle (4.98507,5.79521);
\draw [color=c, fill=c] (4.98507,5.68936) rectangle (5.02488,5.79521);
\draw [color=c, fill=c] (5.02488,5.68936) rectangle (5.06468,5.79521);
\draw [color=c, fill=c] (5.06468,5.68936) rectangle (5.10448,5.79521);
\draw [color=c, fill=c] (5.10448,5.68936) rectangle (5.14428,5.79521);
\draw [color=c, fill=c] (5.14428,5.68936) rectangle (5.18408,5.79521);
\draw [color=c, fill=c] (5.18408,5.68936) rectangle (5.22388,5.79521);
\draw [color=c, fill=c] (5.22388,5.68936) rectangle (5.26368,5.79521);
\draw [color=c, fill=c] (5.26368,5.68936) rectangle (5.30348,5.79521);
\draw [color=c, fill=c] (5.30348,5.68936) rectangle (5.34328,5.79521);
\draw [color=c, fill=c] (5.34328,5.68936) rectangle (5.38308,5.79521);
\draw [color=c, fill=c] (5.38308,5.68936) rectangle (5.42289,5.79521);
\draw [color=c, fill=c] (5.42289,5.68936) rectangle (5.46269,5.79521);
\draw [color=c, fill=c] (5.46269,5.68936) rectangle (5.50249,5.79521);
\draw [color=c, fill=c] (5.50249,5.68936) rectangle (5.54229,5.79521);
\draw [color=c, fill=c] (5.54229,5.68936) rectangle (5.58209,5.79521);
\draw [color=c, fill=c] (5.58209,5.68936) rectangle (5.62189,5.79521);
\definecolor{c}{rgb}{0.2,0,1};
\draw [color=c, fill=c] (5.62189,5.68936) rectangle (5.66169,5.79521);
\draw [color=c, fill=c] (5.66169,5.68936) rectangle (5.70149,5.79521);
\draw [color=c, fill=c] (5.70149,5.68936) rectangle (5.74129,5.79521);
\draw [color=c, fill=c] (5.74129,5.68936) rectangle (5.78109,5.79521);
\draw [color=c, fill=c] (5.78109,5.68936) rectangle (5.8209,5.79521);
\draw [color=c, fill=c] (5.8209,5.68936) rectangle (5.8607,5.79521);
\draw [color=c, fill=c] (5.8607,5.68936) rectangle (5.9005,5.79521);
\draw [color=c, fill=c] (5.9005,5.68936) rectangle (5.9403,5.79521);
\draw [color=c, fill=c] (5.9403,5.68936) rectangle (5.9801,5.79521);
\draw [color=c, fill=c] (5.9801,5.68936) rectangle (6.0199,5.79521);
\draw [color=c, fill=c] (6.0199,5.68936) rectangle (6.0597,5.79521);
\draw [color=c, fill=c] (6.0597,5.68936) rectangle (6.0995,5.79521);
\draw [color=c, fill=c] (6.0995,5.68936) rectangle (6.1393,5.79521);
\draw [color=c, fill=c] (6.1393,5.68936) rectangle (6.1791,5.79521);
\draw [color=c, fill=c] (6.1791,5.68936) rectangle (6.21891,5.79521);
\draw [color=c, fill=c] (6.21891,5.68936) rectangle (6.25871,5.79521);
\draw [color=c, fill=c] (6.25871,5.68936) rectangle (6.29851,5.79521);
\draw [color=c, fill=c] (6.29851,5.68936) rectangle (6.33831,5.79521);
\draw [color=c, fill=c] (6.33831,5.68936) rectangle (6.37811,5.79521);
\draw [color=c, fill=c] (6.37811,5.68936) rectangle (6.41791,5.79521);
\draw [color=c, fill=c] (6.41791,5.68936) rectangle (6.45771,5.79521);
\draw [color=c, fill=c] (6.45771,5.68936) rectangle (6.49751,5.79521);
\draw [color=c, fill=c] (6.49751,5.68936) rectangle (6.53731,5.79521);
\draw [color=c, fill=c] (6.53731,5.68936) rectangle (6.57711,5.79521);
\draw [color=c, fill=c] (6.57711,5.68936) rectangle (6.61692,5.79521);
\draw [color=c, fill=c] (6.61692,5.68936) rectangle (6.65672,5.79521);
\draw [color=c, fill=c] (6.65672,5.68936) rectangle (6.69652,5.79521);
\draw [color=c, fill=c] (6.69652,5.68936) rectangle (6.73632,5.79521);
\draw [color=c, fill=c] (6.73632,5.68936) rectangle (6.77612,5.79521);
\draw [color=c, fill=c] (6.77612,5.68936) rectangle (6.81592,5.79521);
\draw [color=c, fill=c] (6.81592,5.68936) rectangle (6.85572,5.79521);
\draw [color=c, fill=c] (6.85572,5.68936) rectangle (6.89552,5.79521);
\draw [color=c, fill=c] (6.89552,5.68936) rectangle (6.93532,5.79521);
\draw [color=c, fill=c] (6.93532,5.68936) rectangle (6.97512,5.79521);
\draw [color=c, fill=c] (6.97512,5.68936) rectangle (7.01493,5.79521);
\draw [color=c, fill=c] (7.01493,5.68936) rectangle (7.05473,5.79521);
\draw [color=c, fill=c] (7.05473,5.68936) rectangle (7.09453,5.79521);
\draw [color=c, fill=c] (7.09453,5.68936) rectangle (7.13433,5.79521);
\draw [color=c, fill=c] (7.13433,5.68936) rectangle (7.17413,5.79521);
\draw [color=c, fill=c] (7.17413,5.68936) rectangle (7.21393,5.79521);
\draw [color=c, fill=c] (7.21393,5.68936) rectangle (7.25373,5.79521);
\draw [color=c, fill=c] (7.25373,5.68936) rectangle (7.29353,5.79521);
\draw [color=c, fill=c] (7.29353,5.68936) rectangle (7.33333,5.79521);
\draw [color=c, fill=c] (7.33333,5.68936) rectangle (7.37313,5.79521);
\draw [color=c, fill=c] (7.37313,5.68936) rectangle (7.41294,5.79521);
\draw [color=c, fill=c] (7.41294,5.68936) rectangle (7.45274,5.79521);
\draw [color=c, fill=c] (7.45274,5.68936) rectangle (7.49254,5.79521);
\draw [color=c, fill=c] (7.49254,5.68936) rectangle (7.53234,5.79521);
\draw [color=c, fill=c] (7.53234,5.68936) rectangle (7.57214,5.79521);
\draw [color=c, fill=c] (7.57214,5.68936) rectangle (7.61194,5.79521);
\draw [color=c, fill=c] (7.61194,5.68936) rectangle (7.65174,5.79521);
\draw [color=c, fill=c] (7.65174,5.68936) rectangle (7.69154,5.79521);
\draw [color=c, fill=c] (7.69154,5.68936) rectangle (7.73134,5.79521);
\draw [color=c, fill=c] (7.73134,5.68936) rectangle (7.77114,5.79521);
\draw [color=c, fill=c] (7.77114,5.68936) rectangle (7.81095,5.79521);
\draw [color=c, fill=c] (7.81095,5.68936) rectangle (7.85075,5.79521);
\draw [color=c, fill=c] (7.85075,5.68936) rectangle (7.89055,5.79521);
\draw [color=c, fill=c] (7.89055,5.68936) rectangle (7.93035,5.79521);
\draw [color=c, fill=c] (7.93035,5.68936) rectangle (7.97015,5.79521);
\draw [color=c, fill=c] (7.97015,5.68936) rectangle (8.00995,5.79521);
\draw [color=c, fill=c] (8.00995,5.68936) rectangle (8.04975,5.79521);
\draw [color=c, fill=c] (8.04975,5.68936) rectangle (8.08955,5.79521);
\draw [color=c, fill=c] (8.08955,5.68936) rectangle (8.12935,5.79521);
\draw [color=c, fill=c] (8.12935,5.68936) rectangle (8.16915,5.79521);
\draw [color=c, fill=c] (8.16915,5.68936) rectangle (8.20895,5.79521);
\draw [color=c, fill=c] (8.20895,5.68936) rectangle (8.24876,5.79521);
\draw [color=c, fill=c] (8.24876,5.68936) rectangle (8.28856,5.79521);
\draw [color=c, fill=c] (8.28856,5.68936) rectangle (8.32836,5.79521);
\draw [color=c, fill=c] (8.32836,5.68936) rectangle (8.36816,5.79521);
\draw [color=c, fill=c] (8.36816,5.68936) rectangle (8.40796,5.79521);
\draw [color=c, fill=c] (8.40796,5.68936) rectangle (8.44776,5.79521);
\draw [color=c, fill=c] (8.44776,5.68936) rectangle (8.48756,5.79521);
\draw [color=c, fill=c] (8.48756,5.68936) rectangle (8.52736,5.79521);
\draw [color=c, fill=c] (8.52736,5.68936) rectangle (8.56716,5.79521);
\draw [color=c, fill=c] (8.56716,5.68936) rectangle (8.60697,5.79521);
\draw [color=c, fill=c] (8.60697,5.68936) rectangle (8.64677,5.79521);
\draw [color=c, fill=c] (8.64677,5.68936) rectangle (8.68657,5.79521);
\draw [color=c, fill=c] (8.68657,5.68936) rectangle (8.72637,5.79521);
\draw [color=c, fill=c] (8.72637,5.68936) rectangle (8.76617,5.79521);
\draw [color=c, fill=c] (8.76617,5.68936) rectangle (8.80597,5.79521);
\draw [color=c, fill=c] (8.80597,5.68936) rectangle (8.84577,5.79521);
\draw [color=c, fill=c] (8.84577,5.68936) rectangle (8.88557,5.79521);
\draw [color=c, fill=c] (8.88557,5.68936) rectangle (8.92537,5.79521);
\draw [color=c, fill=c] (8.92537,5.68936) rectangle (8.96517,5.79521);
\draw [color=c, fill=c] (8.96517,5.68936) rectangle (9.00498,5.79521);
\draw [color=c, fill=c] (9.00498,5.68936) rectangle (9.04478,5.79521);
\draw [color=c, fill=c] (9.04478,5.68936) rectangle (9.08458,5.79521);
\draw [color=c, fill=c] (9.08458,5.68936) rectangle (9.12438,5.79521);
\draw [color=c, fill=c] (9.12438,5.68936) rectangle (9.16418,5.79521);
\definecolor{c}{rgb}{0,0.0800001,1};
\draw [color=c, fill=c] (9.16418,5.68936) rectangle (9.20398,5.79521);
\draw [color=c, fill=c] (9.20398,5.68936) rectangle (9.24378,5.79521);
\draw [color=c, fill=c] (9.24378,5.68936) rectangle (9.28358,5.79521);
\draw [color=c, fill=c] (9.28358,5.68936) rectangle (9.32338,5.79521);
\draw [color=c, fill=c] (9.32338,5.68936) rectangle (9.36318,5.79521);
\draw [color=c, fill=c] (9.36318,5.68936) rectangle (9.40298,5.79521);
\draw [color=c, fill=c] (9.40298,5.68936) rectangle (9.44279,5.79521);
\draw [color=c, fill=c] (9.44279,5.68936) rectangle (9.48259,5.79521);
\draw [color=c, fill=c] (9.48259,5.68936) rectangle (9.52239,5.79521);
\draw [color=c, fill=c] (9.52239,5.68936) rectangle (9.56219,5.79521);
\draw [color=c, fill=c] (9.56219,5.68936) rectangle (9.60199,5.79521);
\draw [color=c, fill=c] (9.60199,5.68936) rectangle (9.64179,5.79521);
\draw [color=c, fill=c] (9.64179,5.68936) rectangle (9.68159,5.79521);
\draw [color=c, fill=c] (9.68159,5.68936) rectangle (9.72139,5.79521);
\draw [color=c, fill=c] (9.72139,5.68936) rectangle (9.76119,5.79521);
\definecolor{c}{rgb}{0,0.266667,1};
\draw [color=c, fill=c] (9.76119,5.68936) rectangle (9.80099,5.79521);
\draw [color=c, fill=c] (9.80099,5.68936) rectangle (9.8408,5.79521);
\draw [color=c, fill=c] (9.8408,5.68936) rectangle (9.8806,5.79521);
\draw [color=c, fill=c] (9.8806,5.68936) rectangle (9.9204,5.79521);
\draw [color=c, fill=c] (9.9204,5.68936) rectangle (9.9602,5.79521);
\draw [color=c, fill=c] (9.9602,5.68936) rectangle (10,5.79521);
\draw [color=c, fill=c] (10,5.68936) rectangle (10.0398,5.79521);
\draw [color=c, fill=c] (10.0398,5.68936) rectangle (10.0796,5.79521);
\definecolor{c}{rgb}{0,0.546666,1};
\draw [color=c, fill=c] (10.0796,5.68936) rectangle (10.1194,5.79521);
\draw [color=c, fill=c] (10.1194,5.68936) rectangle (10.1592,5.79521);
\draw [color=c, fill=c] (10.1592,5.68936) rectangle (10.199,5.79521);
\draw [color=c, fill=c] (10.199,5.68936) rectangle (10.2388,5.79521);
\draw [color=c, fill=c] (10.2388,5.68936) rectangle (10.2786,5.79521);
\draw [color=c, fill=c] (10.2786,5.68936) rectangle (10.3184,5.79521);
\draw [color=c, fill=c] (10.3184,5.68936) rectangle (10.3582,5.79521);
\draw [color=c, fill=c] (10.3582,5.68936) rectangle (10.398,5.79521);
\draw [color=c, fill=c] (10.398,5.68936) rectangle (10.4378,5.79521);
\draw [color=c, fill=c] (10.4378,5.68936) rectangle (10.4776,5.79521);
\draw [color=c, fill=c] (10.4776,5.68936) rectangle (10.5174,5.79521);
\draw [color=c, fill=c] (10.5174,5.68936) rectangle (10.5572,5.79521);
\definecolor{c}{rgb}{0,0.733333,1};
\draw [color=c, fill=c] (10.5572,5.68936) rectangle (10.597,5.79521);
\draw [color=c, fill=c] (10.597,5.68936) rectangle (10.6368,5.79521);
\draw [color=c, fill=c] (10.6368,5.68936) rectangle (10.6766,5.79521);
\draw [color=c, fill=c] (10.6766,5.68936) rectangle (10.7164,5.79521);
\draw [color=c, fill=c] (10.7164,5.68936) rectangle (10.7562,5.79521);
\draw [color=c, fill=c] (10.7562,5.68936) rectangle (10.796,5.79521);
\draw [color=c, fill=c] (10.796,5.68936) rectangle (10.8358,5.79521);
\draw [color=c, fill=c] (10.8358,5.68936) rectangle (10.8756,5.79521);
\draw [color=c, fill=c] (10.8756,5.68936) rectangle (10.9154,5.79521);
\draw [color=c, fill=c] (10.9154,5.68936) rectangle (10.9552,5.79521);
\draw [color=c, fill=c] (10.9552,5.68936) rectangle (10.995,5.79521);
\draw [color=c, fill=c] (10.995,5.68936) rectangle (11.0348,5.79521);
\draw [color=c, fill=c] (11.0348,5.68936) rectangle (11.0746,5.79521);
\draw [color=c, fill=c] (11.0746,5.68936) rectangle (11.1144,5.79521);
\draw [color=c, fill=c] (11.1144,5.68936) rectangle (11.1542,5.79521);
\draw [color=c, fill=c] (11.1542,5.68936) rectangle (11.194,5.79521);
\draw [color=c, fill=c] (11.194,5.68936) rectangle (11.2338,5.79521);
\draw [color=c, fill=c] (11.2338,5.68936) rectangle (11.2736,5.79521);
\draw [color=c, fill=c] (11.2736,5.68936) rectangle (11.3134,5.79521);
\draw [color=c, fill=c] (11.3134,5.68936) rectangle (11.3532,5.79521);
\draw [color=c, fill=c] (11.3532,5.68936) rectangle (11.393,5.79521);
\draw [color=c, fill=c] (11.393,5.68936) rectangle (11.4328,5.79521);
\draw [color=c, fill=c] (11.4328,5.68936) rectangle (11.4726,5.79521);
\draw [color=c, fill=c] (11.4726,5.68936) rectangle (11.5124,5.79521);
\draw [color=c, fill=c] (11.5124,5.68936) rectangle (11.5522,5.79521);
\draw [color=c, fill=c] (11.5522,5.68936) rectangle (11.592,5.79521);
\draw [color=c, fill=c] (11.592,5.68936) rectangle (11.6318,5.79521);
\draw [color=c, fill=c] (11.6318,5.68936) rectangle (11.6716,5.79521);
\draw [color=c, fill=c] (11.6716,5.68936) rectangle (11.7114,5.79521);
\draw [color=c, fill=c] (11.7114,5.68936) rectangle (11.7512,5.79521);
\draw [color=c, fill=c] (11.7512,5.68936) rectangle (11.791,5.79521);
\draw [color=c, fill=c] (11.791,5.68936) rectangle (11.8308,5.79521);
\draw [color=c, fill=c] (11.8308,5.68936) rectangle (11.8706,5.79521);
\draw [color=c, fill=c] (11.8706,5.68936) rectangle (11.9104,5.79521);
\draw [color=c, fill=c] (11.9104,5.68936) rectangle (11.9502,5.79521);
\draw [color=c, fill=c] (11.9502,5.68936) rectangle (11.99,5.79521);
\draw [color=c, fill=c] (11.99,5.68936) rectangle (12.0299,5.79521);
\draw [color=c, fill=c] (12.0299,5.68936) rectangle (12.0697,5.79521);
\draw [color=c, fill=c] (12.0697,5.68936) rectangle (12.1095,5.79521);
\draw [color=c, fill=c] (12.1095,5.68936) rectangle (12.1493,5.79521);
\draw [color=c, fill=c] (12.1493,5.68936) rectangle (12.1891,5.79521);
\draw [color=c, fill=c] (12.1891,5.68936) rectangle (12.2289,5.79521);
\draw [color=c, fill=c] (12.2289,5.68936) rectangle (12.2687,5.79521);
\draw [color=c, fill=c] (12.2687,5.68936) rectangle (12.3085,5.79521);
\draw [color=c, fill=c] (12.3085,5.68936) rectangle (12.3483,5.79521);
\draw [color=c, fill=c] (12.3483,5.68936) rectangle (12.3881,5.79521);
\draw [color=c, fill=c] (12.3881,5.68936) rectangle (12.4279,5.79521);
\draw [color=c, fill=c] (12.4279,5.68936) rectangle (12.4677,5.79521);
\draw [color=c, fill=c] (12.4677,5.68936) rectangle (12.5075,5.79521);
\draw [color=c, fill=c] (12.5075,5.68936) rectangle (12.5473,5.79521);
\draw [color=c, fill=c] (12.5473,5.68936) rectangle (12.5871,5.79521);
\draw [color=c, fill=c] (12.5871,5.68936) rectangle (12.6269,5.79521);
\draw [color=c, fill=c] (12.6269,5.68936) rectangle (12.6667,5.79521);
\draw [color=c, fill=c] (12.6667,5.68936) rectangle (12.7065,5.79521);
\draw [color=c, fill=c] (12.7065,5.68936) rectangle (12.7463,5.79521);
\draw [color=c, fill=c] (12.7463,5.68936) rectangle (12.7861,5.79521);
\draw [color=c, fill=c] (12.7861,5.68936) rectangle (12.8259,5.79521);
\draw [color=c, fill=c] (12.8259,5.68936) rectangle (12.8657,5.79521);
\draw [color=c, fill=c] (12.8657,5.68936) rectangle (12.9055,5.79521);
\draw [color=c, fill=c] (12.9055,5.68936) rectangle (12.9453,5.79521);
\draw [color=c, fill=c] (12.9453,5.68936) rectangle (12.9851,5.79521);
\draw [color=c, fill=c] (12.9851,5.68936) rectangle (13.0249,5.79521);
\draw [color=c, fill=c] (13.0249,5.68936) rectangle (13.0647,5.79521);
\draw [color=c, fill=c] (13.0647,5.68936) rectangle (13.1045,5.79521);
\draw [color=c, fill=c] (13.1045,5.68936) rectangle (13.1443,5.79521);
\draw [color=c, fill=c] (13.1443,5.68936) rectangle (13.1841,5.79521);
\draw [color=c, fill=c] (13.1841,5.68936) rectangle (13.2239,5.79521);
\draw [color=c, fill=c] (13.2239,5.68936) rectangle (13.2637,5.79521);
\draw [color=c, fill=c] (13.2637,5.68936) rectangle (13.3035,5.79521);
\draw [color=c, fill=c] (13.3035,5.68936) rectangle (13.3433,5.79521);
\draw [color=c, fill=c] (13.3433,5.68936) rectangle (13.3831,5.79521);
\draw [color=c, fill=c] (13.3831,5.68936) rectangle (13.4229,5.79521);
\draw [color=c, fill=c] (13.4229,5.68936) rectangle (13.4627,5.79521);
\draw [color=c, fill=c] (13.4627,5.68936) rectangle (13.5025,5.79521);
\draw [color=c, fill=c] (13.5025,5.68936) rectangle (13.5423,5.79521);
\draw [color=c, fill=c] (13.5423,5.68936) rectangle (13.5821,5.79521);
\draw [color=c, fill=c] (13.5821,5.68936) rectangle (13.6219,5.79521);
\draw [color=c, fill=c] (13.6219,5.68936) rectangle (13.6617,5.79521);
\draw [color=c, fill=c] (13.6617,5.68936) rectangle (13.7015,5.79521);
\draw [color=c, fill=c] (13.7015,5.68936) rectangle (13.7413,5.79521);
\draw [color=c, fill=c] (13.7413,5.68936) rectangle (13.7811,5.79521);
\draw [color=c, fill=c] (13.7811,5.68936) rectangle (13.8209,5.79521);
\draw [color=c, fill=c] (13.8209,5.68936) rectangle (13.8607,5.79521);
\draw [color=c, fill=c] (13.8607,5.68936) rectangle (13.9005,5.79521);
\draw [color=c, fill=c] (13.9005,5.68936) rectangle (13.9403,5.79521);
\draw [color=c, fill=c] (13.9403,5.68936) rectangle (13.9801,5.79521);
\draw [color=c, fill=c] (13.9801,5.68936) rectangle (14.0199,5.79521);
\draw [color=c, fill=c] (14.0199,5.68936) rectangle (14.0597,5.79521);
\draw [color=c, fill=c] (14.0597,5.68936) rectangle (14.0995,5.79521);
\draw [color=c, fill=c] (14.0995,5.68936) rectangle (14.1393,5.79521);
\draw [color=c, fill=c] (14.1393,5.68936) rectangle (14.1791,5.79521);
\draw [color=c, fill=c] (14.1791,5.68936) rectangle (14.2189,5.79521);
\draw [color=c, fill=c] (14.2189,5.68936) rectangle (14.2587,5.79521);
\draw [color=c, fill=c] (14.2587,5.68936) rectangle (14.2985,5.79521);
\draw [color=c, fill=c] (14.2985,5.68936) rectangle (14.3383,5.79521);
\draw [color=c, fill=c] (14.3383,5.68936) rectangle (14.3781,5.79521);
\draw [color=c, fill=c] (14.3781,5.68936) rectangle (14.4179,5.79521);
\draw [color=c, fill=c] (14.4179,5.68936) rectangle (14.4577,5.79521);
\draw [color=c, fill=c] (14.4577,5.68936) rectangle (14.4975,5.79521);
\draw [color=c, fill=c] (14.4975,5.68936) rectangle (14.5373,5.79521);
\draw [color=c, fill=c] (14.5373,5.68936) rectangle (14.5771,5.79521);
\draw [color=c, fill=c] (14.5771,5.68936) rectangle (14.6169,5.79521);
\draw [color=c, fill=c] (14.6169,5.68936) rectangle (14.6567,5.79521);
\draw [color=c, fill=c] (14.6567,5.68936) rectangle (14.6965,5.79521);
\draw [color=c, fill=c] (14.6965,5.68936) rectangle (14.7363,5.79521);
\draw [color=c, fill=c] (14.7363,5.68936) rectangle (14.7761,5.79521);
\draw [color=c, fill=c] (14.7761,5.68936) rectangle (14.8159,5.79521);
\draw [color=c, fill=c] (14.8159,5.68936) rectangle (14.8557,5.79521);
\draw [color=c, fill=c] (14.8557,5.68936) rectangle (14.8955,5.79521);
\draw [color=c, fill=c] (14.8955,5.68936) rectangle (14.9353,5.79521);
\draw [color=c, fill=c] (14.9353,5.68936) rectangle (14.9751,5.79521);
\draw [color=c, fill=c] (14.9751,5.68936) rectangle (15.0149,5.79521);
\draw [color=c, fill=c] (15.0149,5.68936) rectangle (15.0547,5.79521);
\draw [color=c, fill=c] (15.0547,5.68936) rectangle (15.0945,5.79521);
\draw [color=c, fill=c] (15.0945,5.68936) rectangle (15.1343,5.79521);
\draw [color=c, fill=c] (15.1343,5.68936) rectangle (15.1741,5.79521);
\draw [color=c, fill=c] (15.1741,5.68936) rectangle (15.2139,5.79521);
\draw [color=c, fill=c] (15.2139,5.68936) rectangle (15.2537,5.79521);
\draw [color=c, fill=c] (15.2537,5.68936) rectangle (15.2935,5.79521);
\draw [color=c, fill=c] (15.2935,5.68936) rectangle (15.3333,5.79521);
\draw [color=c, fill=c] (15.3333,5.68936) rectangle (15.3731,5.79521);
\draw [color=c, fill=c] (15.3731,5.68936) rectangle (15.4129,5.79521);
\draw [color=c, fill=c] (15.4129,5.68936) rectangle (15.4527,5.79521);
\draw [color=c, fill=c] (15.4527,5.68936) rectangle (15.4925,5.79521);
\draw [color=c, fill=c] (15.4925,5.68936) rectangle (15.5323,5.79521);
\draw [color=c, fill=c] (15.5323,5.68936) rectangle (15.5721,5.79521);
\draw [color=c, fill=c] (15.5721,5.68936) rectangle (15.6119,5.79521);
\draw [color=c, fill=c] (15.6119,5.68936) rectangle (15.6517,5.79521);
\draw [color=c, fill=c] (15.6517,5.68936) rectangle (15.6915,5.79521);
\draw [color=c, fill=c] (15.6915,5.68936) rectangle (15.7313,5.79521);
\draw [color=c, fill=c] (15.7313,5.68936) rectangle (15.7711,5.79521);
\draw [color=c, fill=c] (15.7711,5.68936) rectangle (15.8109,5.79521);
\draw [color=c, fill=c] (15.8109,5.68936) rectangle (15.8507,5.79521);
\draw [color=c, fill=c] (15.8507,5.68936) rectangle (15.8905,5.79521);
\draw [color=c, fill=c] (15.8905,5.68936) rectangle (15.9303,5.79521);
\draw [color=c, fill=c] (15.9303,5.68936) rectangle (15.9701,5.79521);
\draw [color=c, fill=c] (15.9701,5.68936) rectangle (16.01,5.79521);
\draw [color=c, fill=c] (16.01,5.68936) rectangle (16.0498,5.79521);
\draw [color=c, fill=c] (16.0498,5.68936) rectangle (16.0896,5.79521);
\draw [color=c, fill=c] (16.0896,5.68936) rectangle (16.1294,5.79521);
\draw [color=c, fill=c] (16.1294,5.68936) rectangle (16.1692,5.79521);
\draw [color=c, fill=c] (16.1692,5.68936) rectangle (16.209,5.79521);
\draw [color=c, fill=c] (16.209,5.68936) rectangle (16.2488,5.79521);
\draw [color=c, fill=c] (16.2488,5.68936) rectangle (16.2886,5.79521);
\draw [color=c, fill=c] (16.2886,5.68936) rectangle (16.3284,5.79521);
\draw [color=c, fill=c] (16.3284,5.68936) rectangle (16.3682,5.79521);
\draw [color=c, fill=c] (16.3682,5.68936) rectangle (16.408,5.79521);
\draw [color=c, fill=c] (16.408,5.68936) rectangle (16.4478,5.79521);
\draw [color=c, fill=c] (16.4478,5.68936) rectangle (16.4876,5.79521);
\draw [color=c, fill=c] (16.4876,5.68936) rectangle (16.5274,5.79521);
\draw [color=c, fill=c] (16.5274,5.68936) rectangle (16.5672,5.79521);
\draw [color=c, fill=c] (16.5672,5.68936) rectangle (16.607,5.79521);
\draw [color=c, fill=c] (16.607,5.68936) rectangle (16.6468,5.79521);
\draw [color=c, fill=c] (16.6468,5.68936) rectangle (16.6866,5.79521);
\draw [color=c, fill=c] (16.6866,5.68936) rectangle (16.7264,5.79521);
\draw [color=c, fill=c] (16.7264,5.68936) rectangle (16.7662,5.79521);
\draw [color=c, fill=c] (16.7662,5.68936) rectangle (16.806,5.79521);
\draw [color=c, fill=c] (16.806,5.68936) rectangle (16.8458,5.79521);
\draw [color=c, fill=c] (16.8458,5.68936) rectangle (16.8856,5.79521);
\draw [color=c, fill=c] (16.8856,5.68936) rectangle (16.9254,5.79521);
\draw [color=c, fill=c] (16.9254,5.68936) rectangle (16.9652,5.79521);
\draw [color=c, fill=c] (16.9652,5.68936) rectangle (17.005,5.79521);
\draw [color=c, fill=c] (17.005,5.68936) rectangle (17.0448,5.79521);
\draw [color=c, fill=c] (17.0448,5.68936) rectangle (17.0846,5.79521);
\draw [color=c, fill=c] (17.0846,5.68936) rectangle (17.1244,5.79521);
\draw [color=c, fill=c] (17.1244,5.68936) rectangle (17.1642,5.79521);
\draw [color=c, fill=c] (17.1642,5.68936) rectangle (17.204,5.79521);
\draw [color=c, fill=c] (17.204,5.68936) rectangle (17.2438,5.79521);
\draw [color=c, fill=c] (17.2438,5.68936) rectangle (17.2836,5.79521);
\draw [color=c, fill=c] (17.2836,5.68936) rectangle (17.3234,5.79521);
\draw [color=c, fill=c] (17.3234,5.68936) rectangle (17.3632,5.79521);
\draw [color=c, fill=c] (17.3632,5.68936) rectangle (17.403,5.79521);
\draw [color=c, fill=c] (17.403,5.68936) rectangle (17.4428,5.79521);
\draw [color=c, fill=c] (17.4428,5.68936) rectangle (17.4826,5.79521);
\draw [color=c, fill=c] (17.4826,5.68936) rectangle (17.5224,5.79521);
\draw [color=c, fill=c] (17.5224,5.68936) rectangle (17.5622,5.79521);
\draw [color=c, fill=c] (17.5622,5.68936) rectangle (17.602,5.79521);
\draw [color=c, fill=c] (17.602,5.68936) rectangle (17.6418,5.79521);
\draw [color=c, fill=c] (17.6418,5.68936) rectangle (17.6816,5.79521);
\draw [color=c, fill=c] (17.6816,5.68936) rectangle (17.7214,5.79521);
\draw [color=c, fill=c] (17.7214,5.68936) rectangle (17.7612,5.79521);
\draw [color=c, fill=c] (17.7612,5.68936) rectangle (17.801,5.79521);
\draw [color=c, fill=c] (17.801,5.68936) rectangle (17.8408,5.79521);
\draw [color=c, fill=c] (17.8408,5.68936) rectangle (17.8806,5.79521);
\draw [color=c, fill=c] (17.8806,5.68936) rectangle (17.9204,5.79521);
\draw [color=c, fill=c] (17.9204,5.68936) rectangle (17.9602,5.79521);
\draw [color=c, fill=c] (17.9602,5.68936) rectangle (18,5.79521);
\definecolor{c}{rgb}{0,0.0800001,1};
\draw [color=c, fill=c] (2,5.79521) rectangle (2.0398,5.90106);
\draw [color=c, fill=c] (2.0398,5.79521) rectangle (2.0796,5.90106);
\draw [color=c, fill=c] (2.0796,5.79521) rectangle (2.1194,5.90106);
\draw [color=c, fill=c] (2.1194,5.79521) rectangle (2.1592,5.90106);
\draw [color=c, fill=c] (2.1592,5.79521) rectangle (2.19901,5.90106);
\draw [color=c, fill=c] (2.19901,5.79521) rectangle (2.23881,5.90106);
\draw [color=c, fill=c] (2.23881,5.79521) rectangle (2.27861,5.90106);
\draw [color=c, fill=c] (2.27861,5.79521) rectangle (2.31841,5.90106);
\draw [color=c, fill=c] (2.31841,5.79521) rectangle (2.35821,5.90106);
\draw [color=c, fill=c] (2.35821,5.79521) rectangle (2.39801,5.90106);
\draw [color=c, fill=c] (2.39801,5.79521) rectangle (2.43781,5.90106);
\draw [color=c, fill=c] (2.43781,5.79521) rectangle (2.47761,5.90106);
\draw [color=c, fill=c] (2.47761,5.79521) rectangle (2.51741,5.90106);
\draw [color=c, fill=c] (2.51741,5.79521) rectangle (2.55721,5.90106);
\draw [color=c, fill=c] (2.55721,5.79521) rectangle (2.59702,5.90106);
\draw [color=c, fill=c] (2.59702,5.79521) rectangle (2.63682,5.90106);
\draw [color=c, fill=c] (2.63682,5.79521) rectangle (2.67662,5.90106);
\draw [color=c, fill=c] (2.67662,5.79521) rectangle (2.71642,5.90106);
\draw [color=c, fill=c] (2.71642,5.79521) rectangle (2.75622,5.90106);
\draw [color=c, fill=c] (2.75622,5.79521) rectangle (2.79602,5.90106);
\draw [color=c, fill=c] (2.79602,5.79521) rectangle (2.83582,5.90106);
\draw [color=c, fill=c] (2.83582,5.79521) rectangle (2.87562,5.90106);
\draw [color=c, fill=c] (2.87562,5.79521) rectangle (2.91542,5.90106);
\draw [color=c, fill=c] (2.91542,5.79521) rectangle (2.95522,5.90106);
\draw [color=c, fill=c] (2.95522,5.79521) rectangle (2.99502,5.90106);
\draw [color=c, fill=c] (2.99502,5.79521) rectangle (3.03483,5.90106);
\draw [color=c, fill=c] (3.03483,5.79521) rectangle (3.07463,5.90106);
\draw [color=c, fill=c] (3.07463,5.79521) rectangle (3.11443,5.90106);
\draw [color=c, fill=c] (3.11443,5.79521) rectangle (3.15423,5.90106);
\draw [color=c, fill=c] (3.15423,5.79521) rectangle (3.19403,5.90106);
\draw [color=c, fill=c] (3.19403,5.79521) rectangle (3.23383,5.90106);
\draw [color=c, fill=c] (3.23383,5.79521) rectangle (3.27363,5.90106);
\draw [color=c, fill=c] (3.27363,5.79521) rectangle (3.31343,5.90106);
\draw [color=c, fill=c] (3.31343,5.79521) rectangle (3.35323,5.90106);
\draw [color=c, fill=c] (3.35323,5.79521) rectangle (3.39303,5.90106);
\draw [color=c, fill=c] (3.39303,5.79521) rectangle (3.43284,5.90106);
\draw [color=c, fill=c] (3.43284,5.79521) rectangle (3.47264,5.90106);
\draw [color=c, fill=c] (3.47264,5.79521) rectangle (3.51244,5.90106);
\draw [color=c, fill=c] (3.51244,5.79521) rectangle (3.55224,5.90106);
\draw [color=c, fill=c] (3.55224,5.79521) rectangle (3.59204,5.90106);
\draw [color=c, fill=c] (3.59204,5.79521) rectangle (3.63184,5.90106);
\draw [color=c, fill=c] (3.63184,5.79521) rectangle (3.67164,5.90106);
\draw [color=c, fill=c] (3.67164,5.79521) rectangle (3.71144,5.90106);
\draw [color=c, fill=c] (3.71144,5.79521) rectangle (3.75124,5.90106);
\draw [color=c, fill=c] (3.75124,5.79521) rectangle (3.79104,5.90106);
\draw [color=c, fill=c] (3.79104,5.79521) rectangle (3.83085,5.90106);
\draw [color=c, fill=c] (3.83085,5.79521) rectangle (3.87065,5.90106);
\draw [color=c, fill=c] (3.87065,5.79521) rectangle (3.91045,5.90106);
\draw [color=c, fill=c] (3.91045,5.79521) rectangle (3.95025,5.90106);
\draw [color=c, fill=c] (3.95025,5.79521) rectangle (3.99005,5.90106);
\draw [color=c, fill=c] (3.99005,5.79521) rectangle (4.02985,5.90106);
\draw [color=c, fill=c] (4.02985,5.79521) rectangle (4.06965,5.90106);
\draw [color=c, fill=c] (4.06965,5.79521) rectangle (4.10945,5.90106);
\draw [color=c, fill=c] (4.10945,5.79521) rectangle (4.14925,5.90106);
\draw [color=c, fill=c] (4.14925,5.79521) rectangle (4.18905,5.90106);
\draw [color=c, fill=c] (4.18905,5.79521) rectangle (4.22886,5.90106);
\draw [color=c, fill=c] (4.22886,5.79521) rectangle (4.26866,5.90106);
\draw [color=c, fill=c] (4.26866,5.79521) rectangle (4.30846,5.90106);
\draw [color=c, fill=c] (4.30846,5.79521) rectangle (4.34826,5.90106);
\draw [color=c, fill=c] (4.34826,5.79521) rectangle (4.38806,5.90106);
\draw [color=c, fill=c] (4.38806,5.79521) rectangle (4.42786,5.90106);
\draw [color=c, fill=c] (4.42786,5.79521) rectangle (4.46766,5.90106);
\draw [color=c, fill=c] (4.46766,5.79521) rectangle (4.50746,5.90106);
\draw [color=c, fill=c] (4.50746,5.79521) rectangle (4.54726,5.90106);
\draw [color=c, fill=c] (4.54726,5.79521) rectangle (4.58706,5.90106);
\draw [color=c, fill=c] (4.58706,5.79521) rectangle (4.62687,5.90106);
\draw [color=c, fill=c] (4.62687,5.79521) rectangle (4.66667,5.90106);
\draw [color=c, fill=c] (4.66667,5.79521) rectangle (4.70647,5.90106);
\draw [color=c, fill=c] (4.70647,5.79521) rectangle (4.74627,5.90106);
\draw [color=c, fill=c] (4.74627,5.79521) rectangle (4.78607,5.90106);
\draw [color=c, fill=c] (4.78607,5.79521) rectangle (4.82587,5.90106);
\draw [color=c, fill=c] (4.82587,5.79521) rectangle (4.86567,5.90106);
\draw [color=c, fill=c] (4.86567,5.79521) rectangle (4.90547,5.90106);
\draw [color=c, fill=c] (4.90547,5.79521) rectangle (4.94527,5.90106);
\draw [color=c, fill=c] (4.94527,5.79521) rectangle (4.98507,5.90106);
\draw [color=c, fill=c] (4.98507,5.79521) rectangle (5.02488,5.90106);
\draw [color=c, fill=c] (5.02488,5.79521) rectangle (5.06468,5.90106);
\draw [color=c, fill=c] (5.06468,5.79521) rectangle (5.10448,5.90106);
\draw [color=c, fill=c] (5.10448,5.79521) rectangle (5.14428,5.90106);
\draw [color=c, fill=c] (5.14428,5.79521) rectangle (5.18408,5.90106);
\draw [color=c, fill=c] (5.18408,5.79521) rectangle (5.22388,5.90106);
\draw [color=c, fill=c] (5.22388,5.79521) rectangle (5.26368,5.90106);
\draw [color=c, fill=c] (5.26368,5.79521) rectangle (5.30348,5.90106);
\draw [color=c, fill=c] (5.30348,5.79521) rectangle (5.34328,5.90106);
\draw [color=c, fill=c] (5.34328,5.79521) rectangle (5.38308,5.90106);
\draw [color=c, fill=c] (5.38308,5.79521) rectangle (5.42289,5.90106);
\draw [color=c, fill=c] (5.42289,5.79521) rectangle (5.46269,5.90106);
\draw [color=c, fill=c] (5.46269,5.79521) rectangle (5.50249,5.90106);
\draw [color=c, fill=c] (5.50249,5.79521) rectangle (5.54229,5.90106);
\definecolor{c}{rgb}{0.2,0,1};
\draw [color=c, fill=c] (5.54229,5.79521) rectangle (5.58209,5.90106);
\draw [color=c, fill=c] (5.58209,5.79521) rectangle (5.62189,5.90106);
\draw [color=c, fill=c] (5.62189,5.79521) rectangle (5.66169,5.90106);
\draw [color=c, fill=c] (5.66169,5.79521) rectangle (5.70149,5.90106);
\draw [color=c, fill=c] (5.70149,5.79521) rectangle (5.74129,5.90106);
\draw [color=c, fill=c] (5.74129,5.79521) rectangle (5.78109,5.90106);
\draw [color=c, fill=c] (5.78109,5.79521) rectangle (5.8209,5.90106);
\draw [color=c, fill=c] (5.8209,5.79521) rectangle (5.8607,5.90106);
\draw [color=c, fill=c] (5.8607,5.79521) rectangle (5.9005,5.90106);
\draw [color=c, fill=c] (5.9005,5.79521) rectangle (5.9403,5.90106);
\draw [color=c, fill=c] (5.9403,5.79521) rectangle (5.9801,5.90106);
\draw [color=c, fill=c] (5.9801,5.79521) rectangle (6.0199,5.90106);
\draw [color=c, fill=c] (6.0199,5.79521) rectangle (6.0597,5.90106);
\draw [color=c, fill=c] (6.0597,5.79521) rectangle (6.0995,5.90106);
\draw [color=c, fill=c] (6.0995,5.79521) rectangle (6.1393,5.90106);
\draw [color=c, fill=c] (6.1393,5.79521) rectangle (6.1791,5.90106);
\draw [color=c, fill=c] (6.1791,5.79521) rectangle (6.21891,5.90106);
\draw [color=c, fill=c] (6.21891,5.79521) rectangle (6.25871,5.90106);
\draw [color=c, fill=c] (6.25871,5.79521) rectangle (6.29851,5.90106);
\draw [color=c, fill=c] (6.29851,5.79521) rectangle (6.33831,5.90106);
\draw [color=c, fill=c] (6.33831,5.79521) rectangle (6.37811,5.90106);
\draw [color=c, fill=c] (6.37811,5.79521) rectangle (6.41791,5.90106);
\draw [color=c, fill=c] (6.41791,5.79521) rectangle (6.45771,5.90106);
\draw [color=c, fill=c] (6.45771,5.79521) rectangle (6.49751,5.90106);
\draw [color=c, fill=c] (6.49751,5.79521) rectangle (6.53731,5.90106);
\draw [color=c, fill=c] (6.53731,5.79521) rectangle (6.57711,5.90106);
\draw [color=c, fill=c] (6.57711,5.79521) rectangle (6.61692,5.90106);
\draw [color=c, fill=c] (6.61692,5.79521) rectangle (6.65672,5.90106);
\draw [color=c, fill=c] (6.65672,5.79521) rectangle (6.69652,5.90106);
\draw [color=c, fill=c] (6.69652,5.79521) rectangle (6.73632,5.90106);
\draw [color=c, fill=c] (6.73632,5.79521) rectangle (6.77612,5.90106);
\draw [color=c, fill=c] (6.77612,5.79521) rectangle (6.81592,5.90106);
\draw [color=c, fill=c] (6.81592,5.79521) rectangle (6.85572,5.90106);
\draw [color=c, fill=c] (6.85572,5.79521) rectangle (6.89552,5.90106);
\draw [color=c, fill=c] (6.89552,5.79521) rectangle (6.93532,5.90106);
\draw [color=c, fill=c] (6.93532,5.79521) rectangle (6.97512,5.90106);
\draw [color=c, fill=c] (6.97512,5.79521) rectangle (7.01493,5.90106);
\draw [color=c, fill=c] (7.01493,5.79521) rectangle (7.05473,5.90106);
\draw [color=c, fill=c] (7.05473,5.79521) rectangle (7.09453,5.90106);
\draw [color=c, fill=c] (7.09453,5.79521) rectangle (7.13433,5.90106);
\draw [color=c, fill=c] (7.13433,5.79521) rectangle (7.17413,5.90106);
\draw [color=c, fill=c] (7.17413,5.79521) rectangle (7.21393,5.90106);
\draw [color=c, fill=c] (7.21393,5.79521) rectangle (7.25373,5.90106);
\draw [color=c, fill=c] (7.25373,5.79521) rectangle (7.29353,5.90106);
\draw [color=c, fill=c] (7.29353,5.79521) rectangle (7.33333,5.90106);
\draw [color=c, fill=c] (7.33333,5.79521) rectangle (7.37313,5.90106);
\draw [color=c, fill=c] (7.37313,5.79521) rectangle (7.41294,5.90106);
\draw [color=c, fill=c] (7.41294,5.79521) rectangle (7.45274,5.90106);
\draw [color=c, fill=c] (7.45274,5.79521) rectangle (7.49254,5.90106);
\draw [color=c, fill=c] (7.49254,5.79521) rectangle (7.53234,5.90106);
\draw [color=c, fill=c] (7.53234,5.79521) rectangle (7.57214,5.90106);
\draw [color=c, fill=c] (7.57214,5.79521) rectangle (7.61194,5.90106);
\draw [color=c, fill=c] (7.61194,5.79521) rectangle (7.65174,5.90106);
\draw [color=c, fill=c] (7.65174,5.79521) rectangle (7.69154,5.90106);
\draw [color=c, fill=c] (7.69154,5.79521) rectangle (7.73134,5.90106);
\draw [color=c, fill=c] (7.73134,5.79521) rectangle (7.77114,5.90106);
\draw [color=c, fill=c] (7.77114,5.79521) rectangle (7.81095,5.90106);
\draw [color=c, fill=c] (7.81095,5.79521) rectangle (7.85075,5.90106);
\draw [color=c, fill=c] (7.85075,5.79521) rectangle (7.89055,5.90106);
\draw [color=c, fill=c] (7.89055,5.79521) rectangle (7.93035,5.90106);
\draw [color=c, fill=c] (7.93035,5.79521) rectangle (7.97015,5.90106);
\draw [color=c, fill=c] (7.97015,5.79521) rectangle (8.00995,5.90106);
\draw [color=c, fill=c] (8.00995,5.79521) rectangle (8.04975,5.90106);
\draw [color=c, fill=c] (8.04975,5.79521) rectangle (8.08955,5.90106);
\draw [color=c, fill=c] (8.08955,5.79521) rectangle (8.12935,5.90106);
\draw [color=c, fill=c] (8.12935,5.79521) rectangle (8.16915,5.90106);
\draw [color=c, fill=c] (8.16915,5.79521) rectangle (8.20895,5.90106);
\draw [color=c, fill=c] (8.20895,5.79521) rectangle (8.24876,5.90106);
\draw [color=c, fill=c] (8.24876,5.79521) rectangle (8.28856,5.90106);
\draw [color=c, fill=c] (8.28856,5.79521) rectangle (8.32836,5.90106);
\draw [color=c, fill=c] (8.32836,5.79521) rectangle (8.36816,5.90106);
\draw [color=c, fill=c] (8.36816,5.79521) rectangle (8.40796,5.90106);
\draw [color=c, fill=c] (8.40796,5.79521) rectangle (8.44776,5.90106);
\draw [color=c, fill=c] (8.44776,5.79521) rectangle (8.48756,5.90106);
\draw [color=c, fill=c] (8.48756,5.79521) rectangle (8.52736,5.90106);
\draw [color=c, fill=c] (8.52736,5.79521) rectangle (8.56716,5.90106);
\draw [color=c, fill=c] (8.56716,5.79521) rectangle (8.60697,5.90106);
\draw [color=c, fill=c] (8.60697,5.79521) rectangle (8.64677,5.90106);
\draw [color=c, fill=c] (8.64677,5.79521) rectangle (8.68657,5.90106);
\draw [color=c, fill=c] (8.68657,5.79521) rectangle (8.72637,5.90106);
\draw [color=c, fill=c] (8.72637,5.79521) rectangle (8.76617,5.90106);
\draw [color=c, fill=c] (8.76617,5.79521) rectangle (8.80597,5.90106);
\draw [color=c, fill=c] (8.80597,5.79521) rectangle (8.84577,5.90106);
\draw [color=c, fill=c] (8.84577,5.79521) rectangle (8.88557,5.90106);
\draw [color=c, fill=c] (8.88557,5.79521) rectangle (8.92537,5.90106);
\draw [color=c, fill=c] (8.92537,5.79521) rectangle (8.96517,5.90106);
\draw [color=c, fill=c] (8.96517,5.79521) rectangle (9.00498,5.90106);
\draw [color=c, fill=c] (9.00498,5.79521) rectangle (9.04478,5.90106);
\draw [color=c, fill=c] (9.04478,5.79521) rectangle (9.08458,5.90106);
\draw [color=c, fill=c] (9.08458,5.79521) rectangle (9.12438,5.90106);
\definecolor{c}{rgb}{0,0.0800001,1};
\draw [color=c, fill=c] (9.12438,5.79521) rectangle (9.16418,5.90106);
\draw [color=c, fill=c] (9.16418,5.79521) rectangle (9.20398,5.90106);
\draw [color=c, fill=c] (9.20398,5.79521) rectangle (9.24378,5.90106);
\draw [color=c, fill=c] (9.24378,5.79521) rectangle (9.28358,5.90106);
\draw [color=c, fill=c] (9.28358,5.79521) rectangle (9.32338,5.90106);
\draw [color=c, fill=c] (9.32338,5.79521) rectangle (9.36318,5.90106);
\draw [color=c, fill=c] (9.36318,5.79521) rectangle (9.40298,5.90106);
\draw [color=c, fill=c] (9.40298,5.79521) rectangle (9.44279,5.90106);
\draw [color=c, fill=c] (9.44279,5.79521) rectangle (9.48259,5.90106);
\draw [color=c, fill=c] (9.48259,5.79521) rectangle (9.52239,5.90106);
\draw [color=c, fill=c] (9.52239,5.79521) rectangle (9.56219,5.90106);
\draw [color=c, fill=c] (9.56219,5.79521) rectangle (9.60199,5.90106);
\draw [color=c, fill=c] (9.60199,5.79521) rectangle (9.64179,5.90106);
\draw [color=c, fill=c] (9.64179,5.79521) rectangle (9.68159,5.90106);
\draw [color=c, fill=c] (9.68159,5.79521) rectangle (9.72139,5.90106);
\definecolor{c}{rgb}{0,0.266667,1};
\draw [color=c, fill=c] (9.72139,5.79521) rectangle (9.76119,5.90106);
\draw [color=c, fill=c] (9.76119,5.79521) rectangle (9.80099,5.90106);
\draw [color=c, fill=c] (9.80099,5.79521) rectangle (9.8408,5.90106);
\draw [color=c, fill=c] (9.8408,5.79521) rectangle (9.8806,5.90106);
\draw [color=c, fill=c] (9.8806,5.79521) rectangle (9.9204,5.90106);
\draw [color=c, fill=c] (9.9204,5.79521) rectangle (9.9602,5.90106);
\draw [color=c, fill=c] (9.9602,5.79521) rectangle (10,5.90106);
\draw [color=c, fill=c] (10,5.79521) rectangle (10.0398,5.90106);
\draw [color=c, fill=c] (10.0398,5.79521) rectangle (10.0796,5.90106);
\draw [color=c, fill=c] (10.0796,5.79521) rectangle (10.1194,5.90106);
\definecolor{c}{rgb}{0,0.546666,1};
\draw [color=c, fill=c] (10.1194,5.79521) rectangle (10.1592,5.90106);
\draw [color=c, fill=c] (10.1592,5.79521) rectangle (10.199,5.90106);
\draw [color=c, fill=c] (10.199,5.79521) rectangle (10.2388,5.90106);
\draw [color=c, fill=c] (10.2388,5.79521) rectangle (10.2786,5.90106);
\draw [color=c, fill=c] (10.2786,5.79521) rectangle (10.3184,5.90106);
\draw [color=c, fill=c] (10.3184,5.79521) rectangle (10.3582,5.90106);
\draw [color=c, fill=c] (10.3582,5.79521) rectangle (10.398,5.90106);
\draw [color=c, fill=c] (10.398,5.79521) rectangle (10.4378,5.90106);
\draw [color=c, fill=c] (10.4378,5.79521) rectangle (10.4776,5.90106);
\draw [color=c, fill=c] (10.4776,5.79521) rectangle (10.5174,5.90106);
\draw [color=c, fill=c] (10.5174,5.79521) rectangle (10.5572,5.90106);
\draw [color=c, fill=c] (10.5572,5.79521) rectangle (10.597,5.90106);
\definecolor{c}{rgb}{0,0.733333,1};
\draw [color=c, fill=c] (10.597,5.79521) rectangle (10.6368,5.90106);
\draw [color=c, fill=c] (10.6368,5.79521) rectangle (10.6766,5.90106);
\draw [color=c, fill=c] (10.6766,5.79521) rectangle (10.7164,5.90106);
\draw [color=c, fill=c] (10.7164,5.79521) rectangle (10.7562,5.90106);
\draw [color=c, fill=c] (10.7562,5.79521) rectangle (10.796,5.90106);
\draw [color=c, fill=c] (10.796,5.79521) rectangle (10.8358,5.90106);
\draw [color=c, fill=c] (10.8358,5.79521) rectangle (10.8756,5.90106);
\draw [color=c, fill=c] (10.8756,5.79521) rectangle (10.9154,5.90106);
\draw [color=c, fill=c] (10.9154,5.79521) rectangle (10.9552,5.90106);
\draw [color=c, fill=c] (10.9552,5.79521) rectangle (10.995,5.90106);
\draw [color=c, fill=c] (10.995,5.79521) rectangle (11.0348,5.90106);
\draw [color=c, fill=c] (11.0348,5.79521) rectangle (11.0746,5.90106);
\draw [color=c, fill=c] (11.0746,5.79521) rectangle (11.1144,5.90106);
\draw [color=c, fill=c] (11.1144,5.79521) rectangle (11.1542,5.90106);
\draw [color=c, fill=c] (11.1542,5.79521) rectangle (11.194,5.90106);
\draw [color=c, fill=c] (11.194,5.79521) rectangle (11.2338,5.90106);
\draw [color=c, fill=c] (11.2338,5.79521) rectangle (11.2736,5.90106);
\draw [color=c, fill=c] (11.2736,5.79521) rectangle (11.3134,5.90106);
\draw [color=c, fill=c] (11.3134,5.79521) rectangle (11.3532,5.90106);
\draw [color=c, fill=c] (11.3532,5.79521) rectangle (11.393,5.90106);
\draw [color=c, fill=c] (11.393,5.79521) rectangle (11.4328,5.90106);
\draw [color=c, fill=c] (11.4328,5.79521) rectangle (11.4726,5.90106);
\draw [color=c, fill=c] (11.4726,5.79521) rectangle (11.5124,5.90106);
\draw [color=c, fill=c] (11.5124,5.79521) rectangle (11.5522,5.90106);
\draw [color=c, fill=c] (11.5522,5.79521) rectangle (11.592,5.90106);
\draw [color=c, fill=c] (11.592,5.79521) rectangle (11.6318,5.90106);
\draw [color=c, fill=c] (11.6318,5.79521) rectangle (11.6716,5.90106);
\draw [color=c, fill=c] (11.6716,5.79521) rectangle (11.7114,5.90106);
\draw [color=c, fill=c] (11.7114,5.79521) rectangle (11.7512,5.90106);
\draw [color=c, fill=c] (11.7512,5.79521) rectangle (11.791,5.90106);
\draw [color=c, fill=c] (11.791,5.79521) rectangle (11.8308,5.90106);
\draw [color=c, fill=c] (11.8308,5.79521) rectangle (11.8706,5.90106);
\draw [color=c, fill=c] (11.8706,5.79521) rectangle (11.9104,5.90106);
\draw [color=c, fill=c] (11.9104,5.79521) rectangle (11.9502,5.90106);
\draw [color=c, fill=c] (11.9502,5.79521) rectangle (11.99,5.90106);
\draw [color=c, fill=c] (11.99,5.79521) rectangle (12.0299,5.90106);
\draw [color=c, fill=c] (12.0299,5.79521) rectangle (12.0697,5.90106);
\draw [color=c, fill=c] (12.0697,5.79521) rectangle (12.1095,5.90106);
\draw [color=c, fill=c] (12.1095,5.79521) rectangle (12.1493,5.90106);
\draw [color=c, fill=c] (12.1493,5.79521) rectangle (12.1891,5.90106);
\draw [color=c, fill=c] (12.1891,5.79521) rectangle (12.2289,5.90106);
\draw [color=c, fill=c] (12.2289,5.79521) rectangle (12.2687,5.90106);
\draw [color=c, fill=c] (12.2687,5.79521) rectangle (12.3085,5.90106);
\draw [color=c, fill=c] (12.3085,5.79521) rectangle (12.3483,5.90106);
\draw [color=c, fill=c] (12.3483,5.79521) rectangle (12.3881,5.90106);
\draw [color=c, fill=c] (12.3881,5.79521) rectangle (12.4279,5.90106);
\draw [color=c, fill=c] (12.4279,5.79521) rectangle (12.4677,5.90106);
\draw [color=c, fill=c] (12.4677,5.79521) rectangle (12.5075,5.90106);
\draw [color=c, fill=c] (12.5075,5.79521) rectangle (12.5473,5.90106);
\draw [color=c, fill=c] (12.5473,5.79521) rectangle (12.5871,5.90106);
\draw [color=c, fill=c] (12.5871,5.79521) rectangle (12.6269,5.90106);
\draw [color=c, fill=c] (12.6269,5.79521) rectangle (12.6667,5.90106);
\draw [color=c, fill=c] (12.6667,5.79521) rectangle (12.7065,5.90106);
\draw [color=c, fill=c] (12.7065,5.79521) rectangle (12.7463,5.90106);
\draw [color=c, fill=c] (12.7463,5.79521) rectangle (12.7861,5.90106);
\draw [color=c, fill=c] (12.7861,5.79521) rectangle (12.8259,5.90106);
\draw [color=c, fill=c] (12.8259,5.79521) rectangle (12.8657,5.90106);
\draw [color=c, fill=c] (12.8657,5.79521) rectangle (12.9055,5.90106);
\draw [color=c, fill=c] (12.9055,5.79521) rectangle (12.9453,5.90106);
\draw [color=c, fill=c] (12.9453,5.79521) rectangle (12.9851,5.90106);
\draw [color=c, fill=c] (12.9851,5.79521) rectangle (13.0249,5.90106);
\draw [color=c, fill=c] (13.0249,5.79521) rectangle (13.0647,5.90106);
\draw [color=c, fill=c] (13.0647,5.79521) rectangle (13.1045,5.90106);
\draw [color=c, fill=c] (13.1045,5.79521) rectangle (13.1443,5.90106);
\draw [color=c, fill=c] (13.1443,5.79521) rectangle (13.1841,5.90106);
\draw [color=c, fill=c] (13.1841,5.79521) rectangle (13.2239,5.90106);
\draw [color=c, fill=c] (13.2239,5.79521) rectangle (13.2637,5.90106);
\draw [color=c, fill=c] (13.2637,5.79521) rectangle (13.3035,5.90106);
\draw [color=c, fill=c] (13.3035,5.79521) rectangle (13.3433,5.90106);
\draw [color=c, fill=c] (13.3433,5.79521) rectangle (13.3831,5.90106);
\draw [color=c, fill=c] (13.3831,5.79521) rectangle (13.4229,5.90106);
\draw [color=c, fill=c] (13.4229,5.79521) rectangle (13.4627,5.90106);
\draw [color=c, fill=c] (13.4627,5.79521) rectangle (13.5025,5.90106);
\draw [color=c, fill=c] (13.5025,5.79521) rectangle (13.5423,5.90106);
\draw [color=c, fill=c] (13.5423,5.79521) rectangle (13.5821,5.90106);
\draw [color=c, fill=c] (13.5821,5.79521) rectangle (13.6219,5.90106);
\draw [color=c, fill=c] (13.6219,5.79521) rectangle (13.6617,5.90106);
\draw [color=c, fill=c] (13.6617,5.79521) rectangle (13.7015,5.90106);
\draw [color=c, fill=c] (13.7015,5.79521) rectangle (13.7413,5.90106);
\draw [color=c, fill=c] (13.7413,5.79521) rectangle (13.7811,5.90106);
\draw [color=c, fill=c] (13.7811,5.79521) rectangle (13.8209,5.90106);
\draw [color=c, fill=c] (13.8209,5.79521) rectangle (13.8607,5.90106);
\draw [color=c, fill=c] (13.8607,5.79521) rectangle (13.9005,5.90106);
\draw [color=c, fill=c] (13.9005,5.79521) rectangle (13.9403,5.90106);
\draw [color=c, fill=c] (13.9403,5.79521) rectangle (13.9801,5.90106);
\draw [color=c, fill=c] (13.9801,5.79521) rectangle (14.0199,5.90106);
\draw [color=c, fill=c] (14.0199,5.79521) rectangle (14.0597,5.90106);
\draw [color=c, fill=c] (14.0597,5.79521) rectangle (14.0995,5.90106);
\draw [color=c, fill=c] (14.0995,5.79521) rectangle (14.1393,5.90106);
\draw [color=c, fill=c] (14.1393,5.79521) rectangle (14.1791,5.90106);
\draw [color=c, fill=c] (14.1791,5.79521) rectangle (14.2189,5.90106);
\draw [color=c, fill=c] (14.2189,5.79521) rectangle (14.2587,5.90106);
\draw [color=c, fill=c] (14.2587,5.79521) rectangle (14.2985,5.90106);
\draw [color=c, fill=c] (14.2985,5.79521) rectangle (14.3383,5.90106);
\draw [color=c, fill=c] (14.3383,5.79521) rectangle (14.3781,5.90106);
\draw [color=c, fill=c] (14.3781,5.79521) rectangle (14.4179,5.90106);
\draw [color=c, fill=c] (14.4179,5.79521) rectangle (14.4577,5.90106);
\draw [color=c, fill=c] (14.4577,5.79521) rectangle (14.4975,5.90106);
\draw [color=c, fill=c] (14.4975,5.79521) rectangle (14.5373,5.90106);
\draw [color=c, fill=c] (14.5373,5.79521) rectangle (14.5771,5.90106);
\draw [color=c, fill=c] (14.5771,5.79521) rectangle (14.6169,5.90106);
\draw [color=c, fill=c] (14.6169,5.79521) rectangle (14.6567,5.90106);
\draw [color=c, fill=c] (14.6567,5.79521) rectangle (14.6965,5.90106);
\draw [color=c, fill=c] (14.6965,5.79521) rectangle (14.7363,5.90106);
\draw [color=c, fill=c] (14.7363,5.79521) rectangle (14.7761,5.90106);
\draw [color=c, fill=c] (14.7761,5.79521) rectangle (14.8159,5.90106);
\draw [color=c, fill=c] (14.8159,5.79521) rectangle (14.8557,5.90106);
\draw [color=c, fill=c] (14.8557,5.79521) rectangle (14.8955,5.90106);
\draw [color=c, fill=c] (14.8955,5.79521) rectangle (14.9353,5.90106);
\draw [color=c, fill=c] (14.9353,5.79521) rectangle (14.9751,5.90106);
\draw [color=c, fill=c] (14.9751,5.79521) rectangle (15.0149,5.90106);
\draw [color=c, fill=c] (15.0149,5.79521) rectangle (15.0547,5.90106);
\draw [color=c, fill=c] (15.0547,5.79521) rectangle (15.0945,5.90106);
\draw [color=c, fill=c] (15.0945,5.79521) rectangle (15.1343,5.90106);
\draw [color=c, fill=c] (15.1343,5.79521) rectangle (15.1741,5.90106);
\draw [color=c, fill=c] (15.1741,5.79521) rectangle (15.2139,5.90106);
\draw [color=c, fill=c] (15.2139,5.79521) rectangle (15.2537,5.90106);
\draw [color=c, fill=c] (15.2537,5.79521) rectangle (15.2935,5.90106);
\draw [color=c, fill=c] (15.2935,5.79521) rectangle (15.3333,5.90106);
\draw [color=c, fill=c] (15.3333,5.79521) rectangle (15.3731,5.90106);
\draw [color=c, fill=c] (15.3731,5.79521) rectangle (15.4129,5.90106);
\draw [color=c, fill=c] (15.4129,5.79521) rectangle (15.4527,5.90106);
\draw [color=c, fill=c] (15.4527,5.79521) rectangle (15.4925,5.90106);
\draw [color=c, fill=c] (15.4925,5.79521) rectangle (15.5323,5.90106);
\draw [color=c, fill=c] (15.5323,5.79521) rectangle (15.5721,5.90106);
\draw [color=c, fill=c] (15.5721,5.79521) rectangle (15.6119,5.90106);
\draw [color=c, fill=c] (15.6119,5.79521) rectangle (15.6517,5.90106);
\draw [color=c, fill=c] (15.6517,5.79521) rectangle (15.6915,5.90106);
\draw [color=c, fill=c] (15.6915,5.79521) rectangle (15.7313,5.90106);
\draw [color=c, fill=c] (15.7313,5.79521) rectangle (15.7711,5.90106);
\draw [color=c, fill=c] (15.7711,5.79521) rectangle (15.8109,5.90106);
\draw [color=c, fill=c] (15.8109,5.79521) rectangle (15.8507,5.90106);
\draw [color=c, fill=c] (15.8507,5.79521) rectangle (15.8905,5.90106);
\draw [color=c, fill=c] (15.8905,5.79521) rectangle (15.9303,5.90106);
\draw [color=c, fill=c] (15.9303,5.79521) rectangle (15.9701,5.90106);
\draw [color=c, fill=c] (15.9701,5.79521) rectangle (16.01,5.90106);
\draw [color=c, fill=c] (16.01,5.79521) rectangle (16.0498,5.90106);
\draw [color=c, fill=c] (16.0498,5.79521) rectangle (16.0896,5.90106);
\draw [color=c, fill=c] (16.0896,5.79521) rectangle (16.1294,5.90106);
\draw [color=c, fill=c] (16.1294,5.79521) rectangle (16.1692,5.90106);
\draw [color=c, fill=c] (16.1692,5.79521) rectangle (16.209,5.90106);
\draw [color=c, fill=c] (16.209,5.79521) rectangle (16.2488,5.90106);
\draw [color=c, fill=c] (16.2488,5.79521) rectangle (16.2886,5.90106);
\draw [color=c, fill=c] (16.2886,5.79521) rectangle (16.3284,5.90106);
\draw [color=c, fill=c] (16.3284,5.79521) rectangle (16.3682,5.90106);
\draw [color=c, fill=c] (16.3682,5.79521) rectangle (16.408,5.90106);
\draw [color=c, fill=c] (16.408,5.79521) rectangle (16.4478,5.90106);
\draw [color=c, fill=c] (16.4478,5.79521) rectangle (16.4876,5.90106);
\draw [color=c, fill=c] (16.4876,5.79521) rectangle (16.5274,5.90106);
\draw [color=c, fill=c] (16.5274,5.79521) rectangle (16.5672,5.90106);
\draw [color=c, fill=c] (16.5672,5.79521) rectangle (16.607,5.90106);
\draw [color=c, fill=c] (16.607,5.79521) rectangle (16.6468,5.90106);
\draw [color=c, fill=c] (16.6468,5.79521) rectangle (16.6866,5.90106);
\draw [color=c, fill=c] (16.6866,5.79521) rectangle (16.7264,5.90106);
\draw [color=c, fill=c] (16.7264,5.79521) rectangle (16.7662,5.90106);
\draw [color=c, fill=c] (16.7662,5.79521) rectangle (16.806,5.90106);
\draw [color=c, fill=c] (16.806,5.79521) rectangle (16.8458,5.90106);
\draw [color=c, fill=c] (16.8458,5.79521) rectangle (16.8856,5.90106);
\draw [color=c, fill=c] (16.8856,5.79521) rectangle (16.9254,5.90106);
\draw [color=c, fill=c] (16.9254,5.79521) rectangle (16.9652,5.90106);
\draw [color=c, fill=c] (16.9652,5.79521) rectangle (17.005,5.90106);
\draw [color=c, fill=c] (17.005,5.79521) rectangle (17.0448,5.90106);
\draw [color=c, fill=c] (17.0448,5.79521) rectangle (17.0846,5.90106);
\draw [color=c, fill=c] (17.0846,5.79521) rectangle (17.1244,5.90106);
\draw [color=c, fill=c] (17.1244,5.79521) rectangle (17.1642,5.90106);
\draw [color=c, fill=c] (17.1642,5.79521) rectangle (17.204,5.90106);
\draw [color=c, fill=c] (17.204,5.79521) rectangle (17.2438,5.90106);
\draw [color=c, fill=c] (17.2438,5.79521) rectangle (17.2836,5.90106);
\draw [color=c, fill=c] (17.2836,5.79521) rectangle (17.3234,5.90106);
\draw [color=c, fill=c] (17.3234,5.79521) rectangle (17.3632,5.90106);
\draw [color=c, fill=c] (17.3632,5.79521) rectangle (17.403,5.90106);
\draw [color=c, fill=c] (17.403,5.79521) rectangle (17.4428,5.90106);
\draw [color=c, fill=c] (17.4428,5.79521) rectangle (17.4826,5.90106);
\draw [color=c, fill=c] (17.4826,5.79521) rectangle (17.5224,5.90106);
\draw [color=c, fill=c] (17.5224,5.79521) rectangle (17.5622,5.90106);
\draw [color=c, fill=c] (17.5622,5.79521) rectangle (17.602,5.90106);
\draw [color=c, fill=c] (17.602,5.79521) rectangle (17.6418,5.90106);
\draw [color=c, fill=c] (17.6418,5.79521) rectangle (17.6816,5.90106);
\draw [color=c, fill=c] (17.6816,5.79521) rectangle (17.7214,5.90106);
\draw [color=c, fill=c] (17.7214,5.79521) rectangle (17.7612,5.90106);
\draw [color=c, fill=c] (17.7612,5.79521) rectangle (17.801,5.90106);
\draw [color=c, fill=c] (17.801,5.79521) rectangle (17.8408,5.90106);
\draw [color=c, fill=c] (17.8408,5.79521) rectangle (17.8806,5.90106);
\draw [color=c, fill=c] (17.8806,5.79521) rectangle (17.9204,5.90106);
\draw [color=c, fill=c] (17.9204,5.79521) rectangle (17.9602,5.90106);
\draw [color=c, fill=c] (17.9602,5.79521) rectangle (18,5.90106);
\definecolor{c}{rgb}{0,0.0800001,1};
\draw [color=c, fill=c] (2,5.90106) rectangle (2.0398,6.00691);
\draw [color=c, fill=c] (2.0398,5.90106) rectangle (2.0796,6.00691);
\draw [color=c, fill=c] (2.0796,5.90106) rectangle (2.1194,6.00691);
\draw [color=c, fill=c] (2.1194,5.90106) rectangle (2.1592,6.00691);
\draw [color=c, fill=c] (2.1592,5.90106) rectangle (2.19901,6.00691);
\draw [color=c, fill=c] (2.19901,5.90106) rectangle (2.23881,6.00691);
\draw [color=c, fill=c] (2.23881,5.90106) rectangle (2.27861,6.00691);
\draw [color=c, fill=c] (2.27861,5.90106) rectangle (2.31841,6.00691);
\draw [color=c, fill=c] (2.31841,5.90106) rectangle (2.35821,6.00691);
\draw [color=c, fill=c] (2.35821,5.90106) rectangle (2.39801,6.00691);
\draw [color=c, fill=c] (2.39801,5.90106) rectangle (2.43781,6.00691);
\draw [color=c, fill=c] (2.43781,5.90106) rectangle (2.47761,6.00691);
\draw [color=c, fill=c] (2.47761,5.90106) rectangle (2.51741,6.00691);
\draw [color=c, fill=c] (2.51741,5.90106) rectangle (2.55721,6.00691);
\draw [color=c, fill=c] (2.55721,5.90106) rectangle (2.59702,6.00691);
\draw [color=c, fill=c] (2.59702,5.90106) rectangle (2.63682,6.00691);
\draw [color=c, fill=c] (2.63682,5.90106) rectangle (2.67662,6.00691);
\draw [color=c, fill=c] (2.67662,5.90106) rectangle (2.71642,6.00691);
\draw [color=c, fill=c] (2.71642,5.90106) rectangle (2.75622,6.00691);
\draw [color=c, fill=c] (2.75622,5.90106) rectangle (2.79602,6.00691);
\draw [color=c, fill=c] (2.79602,5.90106) rectangle (2.83582,6.00691);
\draw [color=c, fill=c] (2.83582,5.90106) rectangle (2.87562,6.00691);
\draw [color=c, fill=c] (2.87562,5.90106) rectangle (2.91542,6.00691);
\draw [color=c, fill=c] (2.91542,5.90106) rectangle (2.95522,6.00691);
\draw [color=c, fill=c] (2.95522,5.90106) rectangle (2.99502,6.00691);
\draw [color=c, fill=c] (2.99502,5.90106) rectangle (3.03483,6.00691);
\draw [color=c, fill=c] (3.03483,5.90106) rectangle (3.07463,6.00691);
\draw [color=c, fill=c] (3.07463,5.90106) rectangle (3.11443,6.00691);
\draw [color=c, fill=c] (3.11443,5.90106) rectangle (3.15423,6.00691);
\draw [color=c, fill=c] (3.15423,5.90106) rectangle (3.19403,6.00691);
\draw [color=c, fill=c] (3.19403,5.90106) rectangle (3.23383,6.00691);
\draw [color=c, fill=c] (3.23383,5.90106) rectangle (3.27363,6.00691);
\draw [color=c, fill=c] (3.27363,5.90106) rectangle (3.31343,6.00691);
\draw [color=c, fill=c] (3.31343,5.90106) rectangle (3.35323,6.00691);
\draw [color=c, fill=c] (3.35323,5.90106) rectangle (3.39303,6.00691);
\draw [color=c, fill=c] (3.39303,5.90106) rectangle (3.43284,6.00691);
\draw [color=c, fill=c] (3.43284,5.90106) rectangle (3.47264,6.00691);
\draw [color=c, fill=c] (3.47264,5.90106) rectangle (3.51244,6.00691);
\draw [color=c, fill=c] (3.51244,5.90106) rectangle (3.55224,6.00691);
\draw [color=c, fill=c] (3.55224,5.90106) rectangle (3.59204,6.00691);
\draw [color=c, fill=c] (3.59204,5.90106) rectangle (3.63184,6.00691);
\draw [color=c, fill=c] (3.63184,5.90106) rectangle (3.67164,6.00691);
\draw [color=c, fill=c] (3.67164,5.90106) rectangle (3.71144,6.00691);
\draw [color=c, fill=c] (3.71144,5.90106) rectangle (3.75124,6.00691);
\draw [color=c, fill=c] (3.75124,5.90106) rectangle (3.79104,6.00691);
\draw [color=c, fill=c] (3.79104,5.90106) rectangle (3.83085,6.00691);
\draw [color=c, fill=c] (3.83085,5.90106) rectangle (3.87065,6.00691);
\draw [color=c, fill=c] (3.87065,5.90106) rectangle (3.91045,6.00691);
\draw [color=c, fill=c] (3.91045,5.90106) rectangle (3.95025,6.00691);
\draw [color=c, fill=c] (3.95025,5.90106) rectangle (3.99005,6.00691);
\draw [color=c, fill=c] (3.99005,5.90106) rectangle (4.02985,6.00691);
\draw [color=c, fill=c] (4.02985,5.90106) rectangle (4.06965,6.00691);
\draw [color=c, fill=c] (4.06965,5.90106) rectangle (4.10945,6.00691);
\draw [color=c, fill=c] (4.10945,5.90106) rectangle (4.14925,6.00691);
\draw [color=c, fill=c] (4.14925,5.90106) rectangle (4.18905,6.00691);
\draw [color=c, fill=c] (4.18905,5.90106) rectangle (4.22886,6.00691);
\draw [color=c, fill=c] (4.22886,5.90106) rectangle (4.26866,6.00691);
\draw [color=c, fill=c] (4.26866,5.90106) rectangle (4.30846,6.00691);
\draw [color=c, fill=c] (4.30846,5.90106) rectangle (4.34826,6.00691);
\draw [color=c, fill=c] (4.34826,5.90106) rectangle (4.38806,6.00691);
\draw [color=c, fill=c] (4.38806,5.90106) rectangle (4.42786,6.00691);
\draw [color=c, fill=c] (4.42786,5.90106) rectangle (4.46766,6.00691);
\draw [color=c, fill=c] (4.46766,5.90106) rectangle (4.50746,6.00691);
\draw [color=c, fill=c] (4.50746,5.90106) rectangle (4.54726,6.00691);
\draw [color=c, fill=c] (4.54726,5.90106) rectangle (4.58706,6.00691);
\draw [color=c, fill=c] (4.58706,5.90106) rectangle (4.62687,6.00691);
\draw [color=c, fill=c] (4.62687,5.90106) rectangle (4.66667,6.00691);
\draw [color=c, fill=c] (4.66667,5.90106) rectangle (4.70647,6.00691);
\draw [color=c, fill=c] (4.70647,5.90106) rectangle (4.74627,6.00691);
\draw [color=c, fill=c] (4.74627,5.90106) rectangle (4.78607,6.00691);
\draw [color=c, fill=c] (4.78607,5.90106) rectangle (4.82587,6.00691);
\draw [color=c, fill=c] (4.82587,5.90106) rectangle (4.86567,6.00691);
\draw [color=c, fill=c] (4.86567,5.90106) rectangle (4.90547,6.00691);
\draw [color=c, fill=c] (4.90547,5.90106) rectangle (4.94527,6.00691);
\draw [color=c, fill=c] (4.94527,5.90106) rectangle (4.98507,6.00691);
\draw [color=c, fill=c] (4.98507,5.90106) rectangle (5.02488,6.00691);
\draw [color=c, fill=c] (5.02488,5.90106) rectangle (5.06468,6.00691);
\draw [color=c, fill=c] (5.06468,5.90106) rectangle (5.10448,6.00691);
\draw [color=c, fill=c] (5.10448,5.90106) rectangle (5.14428,6.00691);
\draw [color=c, fill=c] (5.14428,5.90106) rectangle (5.18408,6.00691);
\draw [color=c, fill=c] (5.18408,5.90106) rectangle (5.22388,6.00691);
\draw [color=c, fill=c] (5.22388,5.90106) rectangle (5.26368,6.00691);
\draw [color=c, fill=c] (5.26368,5.90106) rectangle (5.30348,6.00691);
\draw [color=c, fill=c] (5.30348,5.90106) rectangle (5.34328,6.00691);
\draw [color=c, fill=c] (5.34328,5.90106) rectangle (5.38308,6.00691);
\draw [color=c, fill=c] (5.38308,5.90106) rectangle (5.42289,6.00691);
\draw [color=c, fill=c] (5.42289,5.90106) rectangle (5.46269,6.00691);
\definecolor{c}{rgb}{0.2,0,1};
\draw [color=c, fill=c] (5.46269,5.90106) rectangle (5.50249,6.00691);
\draw [color=c, fill=c] (5.50249,5.90106) rectangle (5.54229,6.00691);
\draw [color=c, fill=c] (5.54229,5.90106) rectangle (5.58209,6.00691);
\draw [color=c, fill=c] (5.58209,5.90106) rectangle (5.62189,6.00691);
\draw [color=c, fill=c] (5.62189,5.90106) rectangle (5.66169,6.00691);
\draw [color=c, fill=c] (5.66169,5.90106) rectangle (5.70149,6.00691);
\draw [color=c, fill=c] (5.70149,5.90106) rectangle (5.74129,6.00691);
\draw [color=c, fill=c] (5.74129,5.90106) rectangle (5.78109,6.00691);
\draw [color=c, fill=c] (5.78109,5.90106) rectangle (5.8209,6.00691);
\draw [color=c, fill=c] (5.8209,5.90106) rectangle (5.8607,6.00691);
\draw [color=c, fill=c] (5.8607,5.90106) rectangle (5.9005,6.00691);
\draw [color=c, fill=c] (5.9005,5.90106) rectangle (5.9403,6.00691);
\draw [color=c, fill=c] (5.9403,5.90106) rectangle (5.9801,6.00691);
\draw [color=c, fill=c] (5.9801,5.90106) rectangle (6.0199,6.00691);
\draw [color=c, fill=c] (6.0199,5.90106) rectangle (6.0597,6.00691);
\draw [color=c, fill=c] (6.0597,5.90106) rectangle (6.0995,6.00691);
\draw [color=c, fill=c] (6.0995,5.90106) rectangle (6.1393,6.00691);
\draw [color=c, fill=c] (6.1393,5.90106) rectangle (6.1791,6.00691);
\draw [color=c, fill=c] (6.1791,5.90106) rectangle (6.21891,6.00691);
\draw [color=c, fill=c] (6.21891,5.90106) rectangle (6.25871,6.00691);
\draw [color=c, fill=c] (6.25871,5.90106) rectangle (6.29851,6.00691);
\draw [color=c, fill=c] (6.29851,5.90106) rectangle (6.33831,6.00691);
\draw [color=c, fill=c] (6.33831,5.90106) rectangle (6.37811,6.00691);
\draw [color=c, fill=c] (6.37811,5.90106) rectangle (6.41791,6.00691);
\draw [color=c, fill=c] (6.41791,5.90106) rectangle (6.45771,6.00691);
\draw [color=c, fill=c] (6.45771,5.90106) rectangle (6.49751,6.00691);
\draw [color=c, fill=c] (6.49751,5.90106) rectangle (6.53731,6.00691);
\draw [color=c, fill=c] (6.53731,5.90106) rectangle (6.57711,6.00691);
\draw [color=c, fill=c] (6.57711,5.90106) rectangle (6.61692,6.00691);
\draw [color=c, fill=c] (6.61692,5.90106) rectangle (6.65672,6.00691);
\draw [color=c, fill=c] (6.65672,5.90106) rectangle (6.69652,6.00691);
\draw [color=c, fill=c] (6.69652,5.90106) rectangle (6.73632,6.00691);
\draw [color=c, fill=c] (6.73632,5.90106) rectangle (6.77612,6.00691);
\draw [color=c, fill=c] (6.77612,5.90106) rectangle (6.81592,6.00691);
\draw [color=c, fill=c] (6.81592,5.90106) rectangle (6.85572,6.00691);
\draw [color=c, fill=c] (6.85572,5.90106) rectangle (6.89552,6.00691);
\draw [color=c, fill=c] (6.89552,5.90106) rectangle (6.93532,6.00691);
\draw [color=c, fill=c] (6.93532,5.90106) rectangle (6.97512,6.00691);
\draw [color=c, fill=c] (6.97512,5.90106) rectangle (7.01493,6.00691);
\draw [color=c, fill=c] (7.01493,5.90106) rectangle (7.05473,6.00691);
\draw [color=c, fill=c] (7.05473,5.90106) rectangle (7.09453,6.00691);
\draw [color=c, fill=c] (7.09453,5.90106) rectangle (7.13433,6.00691);
\draw [color=c, fill=c] (7.13433,5.90106) rectangle (7.17413,6.00691);
\draw [color=c, fill=c] (7.17413,5.90106) rectangle (7.21393,6.00691);
\draw [color=c, fill=c] (7.21393,5.90106) rectangle (7.25373,6.00691);
\draw [color=c, fill=c] (7.25373,5.90106) rectangle (7.29353,6.00691);
\draw [color=c, fill=c] (7.29353,5.90106) rectangle (7.33333,6.00691);
\draw [color=c, fill=c] (7.33333,5.90106) rectangle (7.37313,6.00691);
\draw [color=c, fill=c] (7.37313,5.90106) rectangle (7.41294,6.00691);
\draw [color=c, fill=c] (7.41294,5.90106) rectangle (7.45274,6.00691);
\draw [color=c, fill=c] (7.45274,5.90106) rectangle (7.49254,6.00691);
\draw [color=c, fill=c] (7.49254,5.90106) rectangle (7.53234,6.00691);
\draw [color=c, fill=c] (7.53234,5.90106) rectangle (7.57214,6.00691);
\draw [color=c, fill=c] (7.57214,5.90106) rectangle (7.61194,6.00691);
\draw [color=c, fill=c] (7.61194,5.90106) rectangle (7.65174,6.00691);
\draw [color=c, fill=c] (7.65174,5.90106) rectangle (7.69154,6.00691);
\draw [color=c, fill=c] (7.69154,5.90106) rectangle (7.73134,6.00691);
\draw [color=c, fill=c] (7.73134,5.90106) rectangle (7.77114,6.00691);
\draw [color=c, fill=c] (7.77114,5.90106) rectangle (7.81095,6.00691);
\draw [color=c, fill=c] (7.81095,5.90106) rectangle (7.85075,6.00691);
\draw [color=c, fill=c] (7.85075,5.90106) rectangle (7.89055,6.00691);
\draw [color=c, fill=c] (7.89055,5.90106) rectangle (7.93035,6.00691);
\draw [color=c, fill=c] (7.93035,5.90106) rectangle (7.97015,6.00691);
\draw [color=c, fill=c] (7.97015,5.90106) rectangle (8.00995,6.00691);
\draw [color=c, fill=c] (8.00995,5.90106) rectangle (8.04975,6.00691);
\draw [color=c, fill=c] (8.04975,5.90106) rectangle (8.08955,6.00691);
\draw [color=c, fill=c] (8.08955,5.90106) rectangle (8.12935,6.00691);
\draw [color=c, fill=c] (8.12935,5.90106) rectangle (8.16915,6.00691);
\draw [color=c, fill=c] (8.16915,5.90106) rectangle (8.20895,6.00691);
\draw [color=c, fill=c] (8.20895,5.90106) rectangle (8.24876,6.00691);
\draw [color=c, fill=c] (8.24876,5.90106) rectangle (8.28856,6.00691);
\draw [color=c, fill=c] (8.28856,5.90106) rectangle (8.32836,6.00691);
\draw [color=c, fill=c] (8.32836,5.90106) rectangle (8.36816,6.00691);
\draw [color=c, fill=c] (8.36816,5.90106) rectangle (8.40796,6.00691);
\draw [color=c, fill=c] (8.40796,5.90106) rectangle (8.44776,6.00691);
\draw [color=c, fill=c] (8.44776,5.90106) rectangle (8.48756,6.00691);
\draw [color=c, fill=c] (8.48756,5.90106) rectangle (8.52736,6.00691);
\draw [color=c, fill=c] (8.52736,5.90106) rectangle (8.56716,6.00691);
\draw [color=c, fill=c] (8.56716,5.90106) rectangle (8.60697,6.00691);
\draw [color=c, fill=c] (8.60697,5.90106) rectangle (8.64677,6.00691);
\draw [color=c, fill=c] (8.64677,5.90106) rectangle (8.68657,6.00691);
\draw [color=c, fill=c] (8.68657,5.90106) rectangle (8.72637,6.00691);
\draw [color=c, fill=c] (8.72637,5.90106) rectangle (8.76617,6.00691);
\draw [color=c, fill=c] (8.76617,5.90106) rectangle (8.80597,6.00691);
\draw [color=c, fill=c] (8.80597,5.90106) rectangle (8.84577,6.00691);
\draw [color=c, fill=c] (8.84577,5.90106) rectangle (8.88557,6.00691);
\draw [color=c, fill=c] (8.88557,5.90106) rectangle (8.92537,6.00691);
\draw [color=c, fill=c] (8.92537,5.90106) rectangle (8.96517,6.00691);
\draw [color=c, fill=c] (8.96517,5.90106) rectangle (9.00498,6.00691);
\draw [color=c, fill=c] (9.00498,5.90106) rectangle (9.04478,6.00691);
\definecolor{c}{rgb}{0,0.0800001,1};
\draw [color=c, fill=c] (9.04478,5.90106) rectangle (9.08458,6.00691);
\draw [color=c, fill=c] (9.08458,5.90106) rectangle (9.12438,6.00691);
\draw [color=c, fill=c] (9.12438,5.90106) rectangle (9.16418,6.00691);
\draw [color=c, fill=c] (9.16418,5.90106) rectangle (9.20398,6.00691);
\draw [color=c, fill=c] (9.20398,5.90106) rectangle (9.24378,6.00691);
\draw [color=c, fill=c] (9.24378,5.90106) rectangle (9.28358,6.00691);
\draw [color=c, fill=c] (9.28358,5.90106) rectangle (9.32338,6.00691);
\draw [color=c, fill=c] (9.32338,5.90106) rectangle (9.36318,6.00691);
\draw [color=c, fill=c] (9.36318,5.90106) rectangle (9.40298,6.00691);
\draw [color=c, fill=c] (9.40298,5.90106) rectangle (9.44279,6.00691);
\draw [color=c, fill=c] (9.44279,5.90106) rectangle (9.48259,6.00691);
\draw [color=c, fill=c] (9.48259,5.90106) rectangle (9.52239,6.00691);
\draw [color=c, fill=c] (9.52239,5.90106) rectangle (9.56219,6.00691);
\draw [color=c, fill=c] (9.56219,5.90106) rectangle (9.60199,6.00691);
\draw [color=c, fill=c] (9.60199,5.90106) rectangle (9.64179,6.00691);
\draw [color=c, fill=c] (9.64179,5.90106) rectangle (9.68159,6.00691);
\draw [color=c, fill=c] (9.68159,5.90106) rectangle (9.72139,6.00691);
\definecolor{c}{rgb}{0,0.266667,1};
\draw [color=c, fill=c] (9.72139,5.90106) rectangle (9.76119,6.00691);
\draw [color=c, fill=c] (9.76119,5.90106) rectangle (9.80099,6.00691);
\draw [color=c, fill=c] (9.80099,5.90106) rectangle (9.8408,6.00691);
\draw [color=c, fill=c] (9.8408,5.90106) rectangle (9.8806,6.00691);
\draw [color=c, fill=c] (9.8806,5.90106) rectangle (9.9204,6.00691);
\draw [color=c, fill=c] (9.9204,5.90106) rectangle (9.9602,6.00691);
\draw [color=c, fill=c] (9.9602,5.90106) rectangle (10,6.00691);
\draw [color=c, fill=c] (10,5.90106) rectangle (10.0398,6.00691);
\draw [color=c, fill=c] (10.0398,5.90106) rectangle (10.0796,6.00691);
\draw [color=c, fill=c] (10.0796,5.90106) rectangle (10.1194,6.00691);
\definecolor{c}{rgb}{0,0.546666,1};
\draw [color=c, fill=c] (10.1194,5.90106) rectangle (10.1592,6.00691);
\draw [color=c, fill=c] (10.1592,5.90106) rectangle (10.199,6.00691);
\draw [color=c, fill=c] (10.199,5.90106) rectangle (10.2388,6.00691);
\draw [color=c, fill=c] (10.2388,5.90106) rectangle (10.2786,6.00691);
\draw [color=c, fill=c] (10.2786,5.90106) rectangle (10.3184,6.00691);
\draw [color=c, fill=c] (10.3184,5.90106) rectangle (10.3582,6.00691);
\draw [color=c, fill=c] (10.3582,5.90106) rectangle (10.398,6.00691);
\draw [color=c, fill=c] (10.398,5.90106) rectangle (10.4378,6.00691);
\draw [color=c, fill=c] (10.4378,5.90106) rectangle (10.4776,6.00691);
\draw [color=c, fill=c] (10.4776,5.90106) rectangle (10.5174,6.00691);
\draw [color=c, fill=c] (10.5174,5.90106) rectangle (10.5572,6.00691);
\draw [color=c, fill=c] (10.5572,5.90106) rectangle (10.597,6.00691);
\draw [color=c, fill=c] (10.597,5.90106) rectangle (10.6368,6.00691);
\definecolor{c}{rgb}{0,0.733333,1};
\draw [color=c, fill=c] (10.6368,5.90106) rectangle (10.6766,6.00691);
\draw [color=c, fill=c] (10.6766,5.90106) rectangle (10.7164,6.00691);
\draw [color=c, fill=c] (10.7164,5.90106) rectangle (10.7562,6.00691);
\draw [color=c, fill=c] (10.7562,5.90106) rectangle (10.796,6.00691);
\draw [color=c, fill=c] (10.796,5.90106) rectangle (10.8358,6.00691);
\draw [color=c, fill=c] (10.8358,5.90106) rectangle (10.8756,6.00691);
\draw [color=c, fill=c] (10.8756,5.90106) rectangle (10.9154,6.00691);
\draw [color=c, fill=c] (10.9154,5.90106) rectangle (10.9552,6.00691);
\draw [color=c, fill=c] (10.9552,5.90106) rectangle (10.995,6.00691);
\draw [color=c, fill=c] (10.995,5.90106) rectangle (11.0348,6.00691);
\draw [color=c, fill=c] (11.0348,5.90106) rectangle (11.0746,6.00691);
\draw [color=c, fill=c] (11.0746,5.90106) rectangle (11.1144,6.00691);
\draw [color=c, fill=c] (11.1144,5.90106) rectangle (11.1542,6.00691);
\draw [color=c, fill=c] (11.1542,5.90106) rectangle (11.194,6.00691);
\draw [color=c, fill=c] (11.194,5.90106) rectangle (11.2338,6.00691);
\draw [color=c, fill=c] (11.2338,5.90106) rectangle (11.2736,6.00691);
\draw [color=c, fill=c] (11.2736,5.90106) rectangle (11.3134,6.00691);
\draw [color=c, fill=c] (11.3134,5.90106) rectangle (11.3532,6.00691);
\draw [color=c, fill=c] (11.3532,5.90106) rectangle (11.393,6.00691);
\draw [color=c, fill=c] (11.393,5.90106) rectangle (11.4328,6.00691);
\draw [color=c, fill=c] (11.4328,5.90106) rectangle (11.4726,6.00691);
\draw [color=c, fill=c] (11.4726,5.90106) rectangle (11.5124,6.00691);
\draw [color=c, fill=c] (11.5124,5.90106) rectangle (11.5522,6.00691);
\draw [color=c, fill=c] (11.5522,5.90106) rectangle (11.592,6.00691);
\draw [color=c, fill=c] (11.592,5.90106) rectangle (11.6318,6.00691);
\draw [color=c, fill=c] (11.6318,5.90106) rectangle (11.6716,6.00691);
\draw [color=c, fill=c] (11.6716,5.90106) rectangle (11.7114,6.00691);
\draw [color=c, fill=c] (11.7114,5.90106) rectangle (11.7512,6.00691);
\draw [color=c, fill=c] (11.7512,5.90106) rectangle (11.791,6.00691);
\draw [color=c, fill=c] (11.791,5.90106) rectangle (11.8308,6.00691);
\draw [color=c, fill=c] (11.8308,5.90106) rectangle (11.8706,6.00691);
\draw [color=c, fill=c] (11.8706,5.90106) rectangle (11.9104,6.00691);
\draw [color=c, fill=c] (11.9104,5.90106) rectangle (11.9502,6.00691);
\draw [color=c, fill=c] (11.9502,5.90106) rectangle (11.99,6.00691);
\draw [color=c, fill=c] (11.99,5.90106) rectangle (12.0299,6.00691);
\draw [color=c, fill=c] (12.0299,5.90106) rectangle (12.0697,6.00691);
\draw [color=c, fill=c] (12.0697,5.90106) rectangle (12.1095,6.00691);
\draw [color=c, fill=c] (12.1095,5.90106) rectangle (12.1493,6.00691);
\draw [color=c, fill=c] (12.1493,5.90106) rectangle (12.1891,6.00691);
\draw [color=c, fill=c] (12.1891,5.90106) rectangle (12.2289,6.00691);
\draw [color=c, fill=c] (12.2289,5.90106) rectangle (12.2687,6.00691);
\draw [color=c, fill=c] (12.2687,5.90106) rectangle (12.3085,6.00691);
\draw [color=c, fill=c] (12.3085,5.90106) rectangle (12.3483,6.00691);
\draw [color=c, fill=c] (12.3483,5.90106) rectangle (12.3881,6.00691);
\draw [color=c, fill=c] (12.3881,5.90106) rectangle (12.4279,6.00691);
\draw [color=c, fill=c] (12.4279,5.90106) rectangle (12.4677,6.00691);
\draw [color=c, fill=c] (12.4677,5.90106) rectangle (12.5075,6.00691);
\draw [color=c, fill=c] (12.5075,5.90106) rectangle (12.5473,6.00691);
\draw [color=c, fill=c] (12.5473,5.90106) rectangle (12.5871,6.00691);
\draw [color=c, fill=c] (12.5871,5.90106) rectangle (12.6269,6.00691);
\draw [color=c, fill=c] (12.6269,5.90106) rectangle (12.6667,6.00691);
\draw [color=c, fill=c] (12.6667,5.90106) rectangle (12.7065,6.00691);
\draw [color=c, fill=c] (12.7065,5.90106) rectangle (12.7463,6.00691);
\draw [color=c, fill=c] (12.7463,5.90106) rectangle (12.7861,6.00691);
\draw [color=c, fill=c] (12.7861,5.90106) rectangle (12.8259,6.00691);
\draw [color=c, fill=c] (12.8259,5.90106) rectangle (12.8657,6.00691);
\draw [color=c, fill=c] (12.8657,5.90106) rectangle (12.9055,6.00691);
\draw [color=c, fill=c] (12.9055,5.90106) rectangle (12.9453,6.00691);
\draw [color=c, fill=c] (12.9453,5.90106) rectangle (12.9851,6.00691);
\draw [color=c, fill=c] (12.9851,5.90106) rectangle (13.0249,6.00691);
\draw [color=c, fill=c] (13.0249,5.90106) rectangle (13.0647,6.00691);
\draw [color=c, fill=c] (13.0647,5.90106) rectangle (13.1045,6.00691);
\draw [color=c, fill=c] (13.1045,5.90106) rectangle (13.1443,6.00691);
\draw [color=c, fill=c] (13.1443,5.90106) rectangle (13.1841,6.00691);
\draw [color=c, fill=c] (13.1841,5.90106) rectangle (13.2239,6.00691);
\draw [color=c, fill=c] (13.2239,5.90106) rectangle (13.2637,6.00691);
\draw [color=c, fill=c] (13.2637,5.90106) rectangle (13.3035,6.00691);
\draw [color=c, fill=c] (13.3035,5.90106) rectangle (13.3433,6.00691);
\draw [color=c, fill=c] (13.3433,5.90106) rectangle (13.3831,6.00691);
\draw [color=c, fill=c] (13.3831,5.90106) rectangle (13.4229,6.00691);
\draw [color=c, fill=c] (13.4229,5.90106) rectangle (13.4627,6.00691);
\draw [color=c, fill=c] (13.4627,5.90106) rectangle (13.5025,6.00691);
\draw [color=c, fill=c] (13.5025,5.90106) rectangle (13.5423,6.00691);
\draw [color=c, fill=c] (13.5423,5.90106) rectangle (13.5821,6.00691);
\draw [color=c, fill=c] (13.5821,5.90106) rectangle (13.6219,6.00691);
\draw [color=c, fill=c] (13.6219,5.90106) rectangle (13.6617,6.00691);
\draw [color=c, fill=c] (13.6617,5.90106) rectangle (13.7015,6.00691);
\draw [color=c, fill=c] (13.7015,5.90106) rectangle (13.7413,6.00691);
\draw [color=c, fill=c] (13.7413,5.90106) rectangle (13.7811,6.00691);
\draw [color=c, fill=c] (13.7811,5.90106) rectangle (13.8209,6.00691);
\draw [color=c, fill=c] (13.8209,5.90106) rectangle (13.8607,6.00691);
\draw [color=c, fill=c] (13.8607,5.90106) rectangle (13.9005,6.00691);
\draw [color=c, fill=c] (13.9005,5.90106) rectangle (13.9403,6.00691);
\draw [color=c, fill=c] (13.9403,5.90106) rectangle (13.9801,6.00691);
\draw [color=c, fill=c] (13.9801,5.90106) rectangle (14.0199,6.00691);
\draw [color=c, fill=c] (14.0199,5.90106) rectangle (14.0597,6.00691);
\draw [color=c, fill=c] (14.0597,5.90106) rectangle (14.0995,6.00691);
\draw [color=c, fill=c] (14.0995,5.90106) rectangle (14.1393,6.00691);
\draw [color=c, fill=c] (14.1393,5.90106) rectangle (14.1791,6.00691);
\draw [color=c, fill=c] (14.1791,5.90106) rectangle (14.2189,6.00691);
\draw [color=c, fill=c] (14.2189,5.90106) rectangle (14.2587,6.00691);
\draw [color=c, fill=c] (14.2587,5.90106) rectangle (14.2985,6.00691);
\draw [color=c, fill=c] (14.2985,5.90106) rectangle (14.3383,6.00691);
\draw [color=c, fill=c] (14.3383,5.90106) rectangle (14.3781,6.00691);
\draw [color=c, fill=c] (14.3781,5.90106) rectangle (14.4179,6.00691);
\draw [color=c, fill=c] (14.4179,5.90106) rectangle (14.4577,6.00691);
\draw [color=c, fill=c] (14.4577,5.90106) rectangle (14.4975,6.00691);
\draw [color=c, fill=c] (14.4975,5.90106) rectangle (14.5373,6.00691);
\draw [color=c, fill=c] (14.5373,5.90106) rectangle (14.5771,6.00691);
\draw [color=c, fill=c] (14.5771,5.90106) rectangle (14.6169,6.00691);
\draw [color=c, fill=c] (14.6169,5.90106) rectangle (14.6567,6.00691);
\draw [color=c, fill=c] (14.6567,5.90106) rectangle (14.6965,6.00691);
\draw [color=c, fill=c] (14.6965,5.90106) rectangle (14.7363,6.00691);
\draw [color=c, fill=c] (14.7363,5.90106) rectangle (14.7761,6.00691);
\draw [color=c, fill=c] (14.7761,5.90106) rectangle (14.8159,6.00691);
\draw [color=c, fill=c] (14.8159,5.90106) rectangle (14.8557,6.00691);
\draw [color=c, fill=c] (14.8557,5.90106) rectangle (14.8955,6.00691);
\draw [color=c, fill=c] (14.8955,5.90106) rectangle (14.9353,6.00691);
\draw [color=c, fill=c] (14.9353,5.90106) rectangle (14.9751,6.00691);
\draw [color=c, fill=c] (14.9751,5.90106) rectangle (15.0149,6.00691);
\draw [color=c, fill=c] (15.0149,5.90106) rectangle (15.0547,6.00691);
\draw [color=c, fill=c] (15.0547,5.90106) rectangle (15.0945,6.00691);
\draw [color=c, fill=c] (15.0945,5.90106) rectangle (15.1343,6.00691);
\draw [color=c, fill=c] (15.1343,5.90106) rectangle (15.1741,6.00691);
\draw [color=c, fill=c] (15.1741,5.90106) rectangle (15.2139,6.00691);
\draw [color=c, fill=c] (15.2139,5.90106) rectangle (15.2537,6.00691);
\draw [color=c, fill=c] (15.2537,5.90106) rectangle (15.2935,6.00691);
\draw [color=c, fill=c] (15.2935,5.90106) rectangle (15.3333,6.00691);
\draw [color=c, fill=c] (15.3333,5.90106) rectangle (15.3731,6.00691);
\draw [color=c, fill=c] (15.3731,5.90106) rectangle (15.4129,6.00691);
\draw [color=c, fill=c] (15.4129,5.90106) rectangle (15.4527,6.00691);
\draw [color=c, fill=c] (15.4527,5.90106) rectangle (15.4925,6.00691);
\draw [color=c, fill=c] (15.4925,5.90106) rectangle (15.5323,6.00691);
\draw [color=c, fill=c] (15.5323,5.90106) rectangle (15.5721,6.00691);
\draw [color=c, fill=c] (15.5721,5.90106) rectangle (15.6119,6.00691);
\draw [color=c, fill=c] (15.6119,5.90106) rectangle (15.6517,6.00691);
\draw [color=c, fill=c] (15.6517,5.90106) rectangle (15.6915,6.00691);
\draw [color=c, fill=c] (15.6915,5.90106) rectangle (15.7313,6.00691);
\draw [color=c, fill=c] (15.7313,5.90106) rectangle (15.7711,6.00691);
\draw [color=c, fill=c] (15.7711,5.90106) rectangle (15.8109,6.00691);
\draw [color=c, fill=c] (15.8109,5.90106) rectangle (15.8507,6.00691);
\draw [color=c, fill=c] (15.8507,5.90106) rectangle (15.8905,6.00691);
\draw [color=c, fill=c] (15.8905,5.90106) rectangle (15.9303,6.00691);
\draw [color=c, fill=c] (15.9303,5.90106) rectangle (15.9701,6.00691);
\draw [color=c, fill=c] (15.9701,5.90106) rectangle (16.01,6.00691);
\draw [color=c, fill=c] (16.01,5.90106) rectangle (16.0498,6.00691);
\draw [color=c, fill=c] (16.0498,5.90106) rectangle (16.0896,6.00691);
\draw [color=c, fill=c] (16.0896,5.90106) rectangle (16.1294,6.00691);
\draw [color=c, fill=c] (16.1294,5.90106) rectangle (16.1692,6.00691);
\draw [color=c, fill=c] (16.1692,5.90106) rectangle (16.209,6.00691);
\draw [color=c, fill=c] (16.209,5.90106) rectangle (16.2488,6.00691);
\draw [color=c, fill=c] (16.2488,5.90106) rectangle (16.2886,6.00691);
\draw [color=c, fill=c] (16.2886,5.90106) rectangle (16.3284,6.00691);
\draw [color=c, fill=c] (16.3284,5.90106) rectangle (16.3682,6.00691);
\draw [color=c, fill=c] (16.3682,5.90106) rectangle (16.408,6.00691);
\draw [color=c, fill=c] (16.408,5.90106) rectangle (16.4478,6.00691);
\draw [color=c, fill=c] (16.4478,5.90106) rectangle (16.4876,6.00691);
\draw [color=c, fill=c] (16.4876,5.90106) rectangle (16.5274,6.00691);
\draw [color=c, fill=c] (16.5274,5.90106) rectangle (16.5672,6.00691);
\draw [color=c, fill=c] (16.5672,5.90106) rectangle (16.607,6.00691);
\draw [color=c, fill=c] (16.607,5.90106) rectangle (16.6468,6.00691);
\draw [color=c, fill=c] (16.6468,5.90106) rectangle (16.6866,6.00691);
\draw [color=c, fill=c] (16.6866,5.90106) rectangle (16.7264,6.00691);
\draw [color=c, fill=c] (16.7264,5.90106) rectangle (16.7662,6.00691);
\draw [color=c, fill=c] (16.7662,5.90106) rectangle (16.806,6.00691);
\draw [color=c, fill=c] (16.806,5.90106) rectangle (16.8458,6.00691);
\draw [color=c, fill=c] (16.8458,5.90106) rectangle (16.8856,6.00691);
\draw [color=c, fill=c] (16.8856,5.90106) rectangle (16.9254,6.00691);
\draw [color=c, fill=c] (16.9254,5.90106) rectangle (16.9652,6.00691);
\draw [color=c, fill=c] (16.9652,5.90106) rectangle (17.005,6.00691);
\draw [color=c, fill=c] (17.005,5.90106) rectangle (17.0448,6.00691);
\draw [color=c, fill=c] (17.0448,5.90106) rectangle (17.0846,6.00691);
\draw [color=c, fill=c] (17.0846,5.90106) rectangle (17.1244,6.00691);
\draw [color=c, fill=c] (17.1244,5.90106) rectangle (17.1642,6.00691);
\draw [color=c, fill=c] (17.1642,5.90106) rectangle (17.204,6.00691);
\draw [color=c, fill=c] (17.204,5.90106) rectangle (17.2438,6.00691);
\draw [color=c, fill=c] (17.2438,5.90106) rectangle (17.2836,6.00691);
\draw [color=c, fill=c] (17.2836,5.90106) rectangle (17.3234,6.00691);
\draw [color=c, fill=c] (17.3234,5.90106) rectangle (17.3632,6.00691);
\draw [color=c, fill=c] (17.3632,5.90106) rectangle (17.403,6.00691);
\draw [color=c, fill=c] (17.403,5.90106) rectangle (17.4428,6.00691);
\draw [color=c, fill=c] (17.4428,5.90106) rectangle (17.4826,6.00691);
\draw [color=c, fill=c] (17.4826,5.90106) rectangle (17.5224,6.00691);
\draw [color=c, fill=c] (17.5224,5.90106) rectangle (17.5622,6.00691);
\draw [color=c, fill=c] (17.5622,5.90106) rectangle (17.602,6.00691);
\draw [color=c, fill=c] (17.602,5.90106) rectangle (17.6418,6.00691);
\draw [color=c, fill=c] (17.6418,5.90106) rectangle (17.6816,6.00691);
\draw [color=c, fill=c] (17.6816,5.90106) rectangle (17.7214,6.00691);
\draw [color=c, fill=c] (17.7214,5.90106) rectangle (17.7612,6.00691);
\draw [color=c, fill=c] (17.7612,5.90106) rectangle (17.801,6.00691);
\draw [color=c, fill=c] (17.801,5.90106) rectangle (17.8408,6.00691);
\draw [color=c, fill=c] (17.8408,5.90106) rectangle (17.8806,6.00691);
\draw [color=c, fill=c] (17.8806,5.90106) rectangle (17.9204,6.00691);
\draw [color=c, fill=c] (17.9204,5.90106) rectangle (17.9602,6.00691);
\draw [color=c, fill=c] (17.9602,5.90106) rectangle (18,6.00691);
\definecolor{c}{rgb}{0,0.0800001,1};
\draw [color=c, fill=c] (2,6.00691) rectangle (2.0398,6.11276);
\draw [color=c, fill=c] (2.0398,6.00691) rectangle (2.0796,6.11276);
\draw [color=c, fill=c] (2.0796,6.00691) rectangle (2.1194,6.11276);
\draw [color=c, fill=c] (2.1194,6.00691) rectangle (2.1592,6.11276);
\draw [color=c, fill=c] (2.1592,6.00691) rectangle (2.19901,6.11276);
\draw [color=c, fill=c] (2.19901,6.00691) rectangle (2.23881,6.11276);
\draw [color=c, fill=c] (2.23881,6.00691) rectangle (2.27861,6.11276);
\draw [color=c, fill=c] (2.27861,6.00691) rectangle (2.31841,6.11276);
\draw [color=c, fill=c] (2.31841,6.00691) rectangle (2.35821,6.11276);
\draw [color=c, fill=c] (2.35821,6.00691) rectangle (2.39801,6.11276);
\draw [color=c, fill=c] (2.39801,6.00691) rectangle (2.43781,6.11276);
\draw [color=c, fill=c] (2.43781,6.00691) rectangle (2.47761,6.11276);
\draw [color=c, fill=c] (2.47761,6.00691) rectangle (2.51741,6.11276);
\draw [color=c, fill=c] (2.51741,6.00691) rectangle (2.55721,6.11276);
\draw [color=c, fill=c] (2.55721,6.00691) rectangle (2.59702,6.11276);
\draw [color=c, fill=c] (2.59702,6.00691) rectangle (2.63682,6.11276);
\draw [color=c, fill=c] (2.63682,6.00691) rectangle (2.67662,6.11276);
\draw [color=c, fill=c] (2.67662,6.00691) rectangle (2.71642,6.11276);
\draw [color=c, fill=c] (2.71642,6.00691) rectangle (2.75622,6.11276);
\draw [color=c, fill=c] (2.75622,6.00691) rectangle (2.79602,6.11276);
\draw [color=c, fill=c] (2.79602,6.00691) rectangle (2.83582,6.11276);
\draw [color=c, fill=c] (2.83582,6.00691) rectangle (2.87562,6.11276);
\draw [color=c, fill=c] (2.87562,6.00691) rectangle (2.91542,6.11276);
\draw [color=c, fill=c] (2.91542,6.00691) rectangle (2.95522,6.11276);
\draw [color=c, fill=c] (2.95522,6.00691) rectangle (2.99502,6.11276);
\draw [color=c, fill=c] (2.99502,6.00691) rectangle (3.03483,6.11276);
\draw [color=c, fill=c] (3.03483,6.00691) rectangle (3.07463,6.11276);
\draw [color=c, fill=c] (3.07463,6.00691) rectangle (3.11443,6.11276);
\draw [color=c, fill=c] (3.11443,6.00691) rectangle (3.15423,6.11276);
\draw [color=c, fill=c] (3.15423,6.00691) rectangle (3.19403,6.11276);
\draw [color=c, fill=c] (3.19403,6.00691) rectangle (3.23383,6.11276);
\draw [color=c, fill=c] (3.23383,6.00691) rectangle (3.27363,6.11276);
\draw [color=c, fill=c] (3.27363,6.00691) rectangle (3.31343,6.11276);
\draw [color=c, fill=c] (3.31343,6.00691) rectangle (3.35323,6.11276);
\draw [color=c, fill=c] (3.35323,6.00691) rectangle (3.39303,6.11276);
\draw [color=c, fill=c] (3.39303,6.00691) rectangle (3.43284,6.11276);
\draw [color=c, fill=c] (3.43284,6.00691) rectangle (3.47264,6.11276);
\draw [color=c, fill=c] (3.47264,6.00691) rectangle (3.51244,6.11276);
\draw [color=c, fill=c] (3.51244,6.00691) rectangle (3.55224,6.11276);
\draw [color=c, fill=c] (3.55224,6.00691) rectangle (3.59204,6.11276);
\draw [color=c, fill=c] (3.59204,6.00691) rectangle (3.63184,6.11276);
\draw [color=c, fill=c] (3.63184,6.00691) rectangle (3.67164,6.11276);
\draw [color=c, fill=c] (3.67164,6.00691) rectangle (3.71144,6.11276);
\draw [color=c, fill=c] (3.71144,6.00691) rectangle (3.75124,6.11276);
\draw [color=c, fill=c] (3.75124,6.00691) rectangle (3.79104,6.11276);
\draw [color=c, fill=c] (3.79104,6.00691) rectangle (3.83085,6.11276);
\draw [color=c, fill=c] (3.83085,6.00691) rectangle (3.87065,6.11276);
\draw [color=c, fill=c] (3.87065,6.00691) rectangle (3.91045,6.11276);
\draw [color=c, fill=c] (3.91045,6.00691) rectangle (3.95025,6.11276);
\draw [color=c, fill=c] (3.95025,6.00691) rectangle (3.99005,6.11276);
\draw [color=c, fill=c] (3.99005,6.00691) rectangle (4.02985,6.11276);
\draw [color=c, fill=c] (4.02985,6.00691) rectangle (4.06965,6.11276);
\draw [color=c, fill=c] (4.06965,6.00691) rectangle (4.10945,6.11276);
\draw [color=c, fill=c] (4.10945,6.00691) rectangle (4.14925,6.11276);
\draw [color=c, fill=c] (4.14925,6.00691) rectangle (4.18905,6.11276);
\draw [color=c, fill=c] (4.18905,6.00691) rectangle (4.22886,6.11276);
\draw [color=c, fill=c] (4.22886,6.00691) rectangle (4.26866,6.11276);
\draw [color=c, fill=c] (4.26866,6.00691) rectangle (4.30846,6.11276);
\draw [color=c, fill=c] (4.30846,6.00691) rectangle (4.34826,6.11276);
\draw [color=c, fill=c] (4.34826,6.00691) rectangle (4.38806,6.11276);
\draw [color=c, fill=c] (4.38806,6.00691) rectangle (4.42786,6.11276);
\draw [color=c, fill=c] (4.42786,6.00691) rectangle (4.46766,6.11276);
\draw [color=c, fill=c] (4.46766,6.00691) rectangle (4.50746,6.11276);
\draw [color=c, fill=c] (4.50746,6.00691) rectangle (4.54726,6.11276);
\draw [color=c, fill=c] (4.54726,6.00691) rectangle (4.58706,6.11276);
\draw [color=c, fill=c] (4.58706,6.00691) rectangle (4.62687,6.11276);
\draw [color=c, fill=c] (4.62687,6.00691) rectangle (4.66667,6.11276);
\draw [color=c, fill=c] (4.66667,6.00691) rectangle (4.70647,6.11276);
\draw [color=c, fill=c] (4.70647,6.00691) rectangle (4.74627,6.11276);
\draw [color=c, fill=c] (4.74627,6.00691) rectangle (4.78607,6.11276);
\draw [color=c, fill=c] (4.78607,6.00691) rectangle (4.82587,6.11276);
\draw [color=c, fill=c] (4.82587,6.00691) rectangle (4.86567,6.11276);
\draw [color=c, fill=c] (4.86567,6.00691) rectangle (4.90547,6.11276);
\draw [color=c, fill=c] (4.90547,6.00691) rectangle (4.94527,6.11276);
\draw [color=c, fill=c] (4.94527,6.00691) rectangle (4.98507,6.11276);
\draw [color=c, fill=c] (4.98507,6.00691) rectangle (5.02488,6.11276);
\draw [color=c, fill=c] (5.02488,6.00691) rectangle (5.06468,6.11276);
\draw [color=c, fill=c] (5.06468,6.00691) rectangle (5.10448,6.11276);
\draw [color=c, fill=c] (5.10448,6.00691) rectangle (5.14428,6.11276);
\draw [color=c, fill=c] (5.14428,6.00691) rectangle (5.18408,6.11276);
\draw [color=c, fill=c] (5.18408,6.00691) rectangle (5.22388,6.11276);
\draw [color=c, fill=c] (5.22388,6.00691) rectangle (5.26368,6.11276);
\draw [color=c, fill=c] (5.26368,6.00691) rectangle (5.30348,6.11276);
\draw [color=c, fill=c] (5.30348,6.00691) rectangle (5.34328,6.11276);
\draw [color=c, fill=c] (5.34328,6.00691) rectangle (5.38308,6.11276);
\definecolor{c}{rgb}{0.2,0,1};
\draw [color=c, fill=c] (5.38308,6.00691) rectangle (5.42289,6.11276);
\draw [color=c, fill=c] (5.42289,6.00691) rectangle (5.46269,6.11276);
\draw [color=c, fill=c] (5.46269,6.00691) rectangle (5.50249,6.11276);
\draw [color=c, fill=c] (5.50249,6.00691) rectangle (5.54229,6.11276);
\draw [color=c, fill=c] (5.54229,6.00691) rectangle (5.58209,6.11276);
\draw [color=c, fill=c] (5.58209,6.00691) rectangle (5.62189,6.11276);
\draw [color=c, fill=c] (5.62189,6.00691) rectangle (5.66169,6.11276);
\draw [color=c, fill=c] (5.66169,6.00691) rectangle (5.70149,6.11276);
\draw [color=c, fill=c] (5.70149,6.00691) rectangle (5.74129,6.11276);
\draw [color=c, fill=c] (5.74129,6.00691) rectangle (5.78109,6.11276);
\draw [color=c, fill=c] (5.78109,6.00691) rectangle (5.8209,6.11276);
\draw [color=c, fill=c] (5.8209,6.00691) rectangle (5.8607,6.11276);
\draw [color=c, fill=c] (5.8607,6.00691) rectangle (5.9005,6.11276);
\draw [color=c, fill=c] (5.9005,6.00691) rectangle (5.9403,6.11276);
\draw [color=c, fill=c] (5.9403,6.00691) rectangle (5.9801,6.11276);
\draw [color=c, fill=c] (5.9801,6.00691) rectangle (6.0199,6.11276);
\draw [color=c, fill=c] (6.0199,6.00691) rectangle (6.0597,6.11276);
\draw [color=c, fill=c] (6.0597,6.00691) rectangle (6.0995,6.11276);
\draw [color=c, fill=c] (6.0995,6.00691) rectangle (6.1393,6.11276);
\draw [color=c, fill=c] (6.1393,6.00691) rectangle (6.1791,6.11276);
\draw [color=c, fill=c] (6.1791,6.00691) rectangle (6.21891,6.11276);
\draw [color=c, fill=c] (6.21891,6.00691) rectangle (6.25871,6.11276);
\draw [color=c, fill=c] (6.25871,6.00691) rectangle (6.29851,6.11276);
\draw [color=c, fill=c] (6.29851,6.00691) rectangle (6.33831,6.11276);
\draw [color=c, fill=c] (6.33831,6.00691) rectangle (6.37811,6.11276);
\draw [color=c, fill=c] (6.37811,6.00691) rectangle (6.41791,6.11276);
\draw [color=c, fill=c] (6.41791,6.00691) rectangle (6.45771,6.11276);
\draw [color=c, fill=c] (6.45771,6.00691) rectangle (6.49751,6.11276);
\draw [color=c, fill=c] (6.49751,6.00691) rectangle (6.53731,6.11276);
\draw [color=c, fill=c] (6.53731,6.00691) rectangle (6.57711,6.11276);
\draw [color=c, fill=c] (6.57711,6.00691) rectangle (6.61692,6.11276);
\draw [color=c, fill=c] (6.61692,6.00691) rectangle (6.65672,6.11276);
\draw [color=c, fill=c] (6.65672,6.00691) rectangle (6.69652,6.11276);
\draw [color=c, fill=c] (6.69652,6.00691) rectangle (6.73632,6.11276);
\draw [color=c, fill=c] (6.73632,6.00691) rectangle (6.77612,6.11276);
\draw [color=c, fill=c] (6.77612,6.00691) rectangle (6.81592,6.11276);
\draw [color=c, fill=c] (6.81592,6.00691) rectangle (6.85572,6.11276);
\draw [color=c, fill=c] (6.85572,6.00691) rectangle (6.89552,6.11276);
\draw [color=c, fill=c] (6.89552,6.00691) rectangle (6.93532,6.11276);
\draw [color=c, fill=c] (6.93532,6.00691) rectangle (6.97512,6.11276);
\draw [color=c, fill=c] (6.97512,6.00691) rectangle (7.01493,6.11276);
\draw [color=c, fill=c] (7.01493,6.00691) rectangle (7.05473,6.11276);
\draw [color=c, fill=c] (7.05473,6.00691) rectangle (7.09453,6.11276);
\draw [color=c, fill=c] (7.09453,6.00691) rectangle (7.13433,6.11276);
\draw [color=c, fill=c] (7.13433,6.00691) rectangle (7.17413,6.11276);
\draw [color=c, fill=c] (7.17413,6.00691) rectangle (7.21393,6.11276);
\draw [color=c, fill=c] (7.21393,6.00691) rectangle (7.25373,6.11276);
\draw [color=c, fill=c] (7.25373,6.00691) rectangle (7.29353,6.11276);
\draw [color=c, fill=c] (7.29353,6.00691) rectangle (7.33333,6.11276);
\draw [color=c, fill=c] (7.33333,6.00691) rectangle (7.37313,6.11276);
\draw [color=c, fill=c] (7.37313,6.00691) rectangle (7.41294,6.11276);
\draw [color=c, fill=c] (7.41294,6.00691) rectangle (7.45274,6.11276);
\draw [color=c, fill=c] (7.45274,6.00691) rectangle (7.49254,6.11276);
\draw [color=c, fill=c] (7.49254,6.00691) rectangle (7.53234,6.11276);
\draw [color=c, fill=c] (7.53234,6.00691) rectangle (7.57214,6.11276);
\draw [color=c, fill=c] (7.57214,6.00691) rectangle (7.61194,6.11276);
\draw [color=c, fill=c] (7.61194,6.00691) rectangle (7.65174,6.11276);
\draw [color=c, fill=c] (7.65174,6.00691) rectangle (7.69154,6.11276);
\draw [color=c, fill=c] (7.69154,6.00691) rectangle (7.73134,6.11276);
\draw [color=c, fill=c] (7.73134,6.00691) rectangle (7.77114,6.11276);
\draw [color=c, fill=c] (7.77114,6.00691) rectangle (7.81095,6.11276);
\draw [color=c, fill=c] (7.81095,6.00691) rectangle (7.85075,6.11276);
\draw [color=c, fill=c] (7.85075,6.00691) rectangle (7.89055,6.11276);
\draw [color=c, fill=c] (7.89055,6.00691) rectangle (7.93035,6.11276);
\draw [color=c, fill=c] (7.93035,6.00691) rectangle (7.97015,6.11276);
\draw [color=c, fill=c] (7.97015,6.00691) rectangle (8.00995,6.11276);
\draw [color=c, fill=c] (8.00995,6.00691) rectangle (8.04975,6.11276);
\draw [color=c, fill=c] (8.04975,6.00691) rectangle (8.08955,6.11276);
\draw [color=c, fill=c] (8.08955,6.00691) rectangle (8.12935,6.11276);
\draw [color=c, fill=c] (8.12935,6.00691) rectangle (8.16915,6.11276);
\draw [color=c, fill=c] (8.16915,6.00691) rectangle (8.20895,6.11276);
\draw [color=c, fill=c] (8.20895,6.00691) rectangle (8.24876,6.11276);
\draw [color=c, fill=c] (8.24876,6.00691) rectangle (8.28856,6.11276);
\draw [color=c, fill=c] (8.28856,6.00691) rectangle (8.32836,6.11276);
\draw [color=c, fill=c] (8.32836,6.00691) rectangle (8.36816,6.11276);
\draw [color=c, fill=c] (8.36816,6.00691) rectangle (8.40796,6.11276);
\draw [color=c, fill=c] (8.40796,6.00691) rectangle (8.44776,6.11276);
\draw [color=c, fill=c] (8.44776,6.00691) rectangle (8.48756,6.11276);
\draw [color=c, fill=c] (8.48756,6.00691) rectangle (8.52736,6.11276);
\draw [color=c, fill=c] (8.52736,6.00691) rectangle (8.56716,6.11276);
\draw [color=c, fill=c] (8.56716,6.00691) rectangle (8.60697,6.11276);
\draw [color=c, fill=c] (8.60697,6.00691) rectangle (8.64677,6.11276);
\draw [color=c, fill=c] (8.64677,6.00691) rectangle (8.68657,6.11276);
\draw [color=c, fill=c] (8.68657,6.00691) rectangle (8.72637,6.11276);
\draw [color=c, fill=c] (8.72637,6.00691) rectangle (8.76617,6.11276);
\draw [color=c, fill=c] (8.76617,6.00691) rectangle (8.80597,6.11276);
\draw [color=c, fill=c] (8.80597,6.00691) rectangle (8.84577,6.11276);
\draw [color=c, fill=c] (8.84577,6.00691) rectangle (8.88557,6.11276);
\draw [color=c, fill=c] (8.88557,6.00691) rectangle (8.92537,6.11276);
\draw [color=c, fill=c] (8.92537,6.00691) rectangle (8.96517,6.11276);
\definecolor{c}{rgb}{0,0.0800001,1};
\draw [color=c, fill=c] (8.96517,6.00691) rectangle (9.00498,6.11276);
\draw [color=c, fill=c] (9.00498,6.00691) rectangle (9.04478,6.11276);
\draw [color=c, fill=c] (9.04478,6.00691) rectangle (9.08458,6.11276);
\draw [color=c, fill=c] (9.08458,6.00691) rectangle (9.12438,6.11276);
\draw [color=c, fill=c] (9.12438,6.00691) rectangle (9.16418,6.11276);
\draw [color=c, fill=c] (9.16418,6.00691) rectangle (9.20398,6.11276);
\draw [color=c, fill=c] (9.20398,6.00691) rectangle (9.24378,6.11276);
\draw [color=c, fill=c] (9.24378,6.00691) rectangle (9.28358,6.11276);
\draw [color=c, fill=c] (9.28358,6.00691) rectangle (9.32338,6.11276);
\draw [color=c, fill=c] (9.32338,6.00691) rectangle (9.36318,6.11276);
\draw [color=c, fill=c] (9.36318,6.00691) rectangle (9.40298,6.11276);
\draw [color=c, fill=c] (9.40298,6.00691) rectangle (9.44279,6.11276);
\draw [color=c, fill=c] (9.44279,6.00691) rectangle (9.48259,6.11276);
\draw [color=c, fill=c] (9.48259,6.00691) rectangle (9.52239,6.11276);
\draw [color=c, fill=c] (9.52239,6.00691) rectangle (9.56219,6.11276);
\draw [color=c, fill=c] (9.56219,6.00691) rectangle (9.60199,6.11276);
\draw [color=c, fill=c] (9.60199,6.00691) rectangle (9.64179,6.11276);
\draw [color=c, fill=c] (9.64179,6.00691) rectangle (9.68159,6.11276);
\draw [color=c, fill=c] (9.68159,6.00691) rectangle (9.72139,6.11276);
\definecolor{c}{rgb}{0,0.266667,1};
\draw [color=c, fill=c] (9.72139,6.00691) rectangle (9.76119,6.11276);
\draw [color=c, fill=c] (9.76119,6.00691) rectangle (9.80099,6.11276);
\draw [color=c, fill=c] (9.80099,6.00691) rectangle (9.8408,6.11276);
\draw [color=c, fill=c] (9.8408,6.00691) rectangle (9.8806,6.11276);
\draw [color=c, fill=c] (9.8806,6.00691) rectangle (9.9204,6.11276);
\draw [color=c, fill=c] (9.9204,6.00691) rectangle (9.9602,6.11276);
\draw [color=c, fill=c] (9.9602,6.00691) rectangle (10,6.11276);
\draw [color=c, fill=c] (10,6.00691) rectangle (10.0398,6.11276);
\draw [color=c, fill=c] (10.0398,6.00691) rectangle (10.0796,6.11276);
\draw [color=c, fill=c] (10.0796,6.00691) rectangle (10.1194,6.11276);
\definecolor{c}{rgb}{0,0.546666,1};
\draw [color=c, fill=c] (10.1194,6.00691) rectangle (10.1592,6.11276);
\draw [color=c, fill=c] (10.1592,6.00691) rectangle (10.199,6.11276);
\draw [color=c, fill=c] (10.199,6.00691) rectangle (10.2388,6.11276);
\draw [color=c, fill=c] (10.2388,6.00691) rectangle (10.2786,6.11276);
\draw [color=c, fill=c] (10.2786,6.00691) rectangle (10.3184,6.11276);
\draw [color=c, fill=c] (10.3184,6.00691) rectangle (10.3582,6.11276);
\draw [color=c, fill=c] (10.3582,6.00691) rectangle (10.398,6.11276);
\draw [color=c, fill=c] (10.398,6.00691) rectangle (10.4378,6.11276);
\draw [color=c, fill=c] (10.4378,6.00691) rectangle (10.4776,6.11276);
\draw [color=c, fill=c] (10.4776,6.00691) rectangle (10.5174,6.11276);
\draw [color=c, fill=c] (10.5174,6.00691) rectangle (10.5572,6.11276);
\draw [color=c, fill=c] (10.5572,6.00691) rectangle (10.597,6.11276);
\draw [color=c, fill=c] (10.597,6.00691) rectangle (10.6368,6.11276);
\draw [color=c, fill=c] (10.6368,6.00691) rectangle (10.6766,6.11276);
\draw [color=c, fill=c] (10.6766,6.00691) rectangle (10.7164,6.11276);
\definecolor{c}{rgb}{0,0.733333,1};
\draw [color=c, fill=c] (10.7164,6.00691) rectangle (10.7562,6.11276);
\draw [color=c, fill=c] (10.7562,6.00691) rectangle (10.796,6.11276);
\draw [color=c, fill=c] (10.796,6.00691) rectangle (10.8358,6.11276);
\draw [color=c, fill=c] (10.8358,6.00691) rectangle (10.8756,6.11276);
\draw [color=c, fill=c] (10.8756,6.00691) rectangle (10.9154,6.11276);
\draw [color=c, fill=c] (10.9154,6.00691) rectangle (10.9552,6.11276);
\draw [color=c, fill=c] (10.9552,6.00691) rectangle (10.995,6.11276);
\draw [color=c, fill=c] (10.995,6.00691) rectangle (11.0348,6.11276);
\draw [color=c, fill=c] (11.0348,6.00691) rectangle (11.0746,6.11276);
\draw [color=c, fill=c] (11.0746,6.00691) rectangle (11.1144,6.11276);
\draw [color=c, fill=c] (11.1144,6.00691) rectangle (11.1542,6.11276);
\draw [color=c, fill=c] (11.1542,6.00691) rectangle (11.194,6.11276);
\draw [color=c, fill=c] (11.194,6.00691) rectangle (11.2338,6.11276);
\draw [color=c, fill=c] (11.2338,6.00691) rectangle (11.2736,6.11276);
\draw [color=c, fill=c] (11.2736,6.00691) rectangle (11.3134,6.11276);
\draw [color=c, fill=c] (11.3134,6.00691) rectangle (11.3532,6.11276);
\draw [color=c, fill=c] (11.3532,6.00691) rectangle (11.393,6.11276);
\draw [color=c, fill=c] (11.393,6.00691) rectangle (11.4328,6.11276);
\draw [color=c, fill=c] (11.4328,6.00691) rectangle (11.4726,6.11276);
\draw [color=c, fill=c] (11.4726,6.00691) rectangle (11.5124,6.11276);
\draw [color=c, fill=c] (11.5124,6.00691) rectangle (11.5522,6.11276);
\draw [color=c, fill=c] (11.5522,6.00691) rectangle (11.592,6.11276);
\draw [color=c, fill=c] (11.592,6.00691) rectangle (11.6318,6.11276);
\draw [color=c, fill=c] (11.6318,6.00691) rectangle (11.6716,6.11276);
\draw [color=c, fill=c] (11.6716,6.00691) rectangle (11.7114,6.11276);
\draw [color=c, fill=c] (11.7114,6.00691) rectangle (11.7512,6.11276);
\draw [color=c, fill=c] (11.7512,6.00691) rectangle (11.791,6.11276);
\draw [color=c, fill=c] (11.791,6.00691) rectangle (11.8308,6.11276);
\draw [color=c, fill=c] (11.8308,6.00691) rectangle (11.8706,6.11276);
\draw [color=c, fill=c] (11.8706,6.00691) rectangle (11.9104,6.11276);
\draw [color=c, fill=c] (11.9104,6.00691) rectangle (11.9502,6.11276);
\draw [color=c, fill=c] (11.9502,6.00691) rectangle (11.99,6.11276);
\draw [color=c, fill=c] (11.99,6.00691) rectangle (12.0299,6.11276);
\draw [color=c, fill=c] (12.0299,6.00691) rectangle (12.0697,6.11276);
\draw [color=c, fill=c] (12.0697,6.00691) rectangle (12.1095,6.11276);
\draw [color=c, fill=c] (12.1095,6.00691) rectangle (12.1493,6.11276);
\draw [color=c, fill=c] (12.1493,6.00691) rectangle (12.1891,6.11276);
\draw [color=c, fill=c] (12.1891,6.00691) rectangle (12.2289,6.11276);
\draw [color=c, fill=c] (12.2289,6.00691) rectangle (12.2687,6.11276);
\draw [color=c, fill=c] (12.2687,6.00691) rectangle (12.3085,6.11276);
\draw [color=c, fill=c] (12.3085,6.00691) rectangle (12.3483,6.11276);
\draw [color=c, fill=c] (12.3483,6.00691) rectangle (12.3881,6.11276);
\draw [color=c, fill=c] (12.3881,6.00691) rectangle (12.4279,6.11276);
\draw [color=c, fill=c] (12.4279,6.00691) rectangle (12.4677,6.11276);
\draw [color=c, fill=c] (12.4677,6.00691) rectangle (12.5075,6.11276);
\draw [color=c, fill=c] (12.5075,6.00691) rectangle (12.5473,6.11276);
\draw [color=c, fill=c] (12.5473,6.00691) rectangle (12.5871,6.11276);
\draw [color=c, fill=c] (12.5871,6.00691) rectangle (12.6269,6.11276);
\draw [color=c, fill=c] (12.6269,6.00691) rectangle (12.6667,6.11276);
\draw [color=c, fill=c] (12.6667,6.00691) rectangle (12.7065,6.11276);
\draw [color=c, fill=c] (12.7065,6.00691) rectangle (12.7463,6.11276);
\draw [color=c, fill=c] (12.7463,6.00691) rectangle (12.7861,6.11276);
\draw [color=c, fill=c] (12.7861,6.00691) rectangle (12.8259,6.11276);
\draw [color=c, fill=c] (12.8259,6.00691) rectangle (12.8657,6.11276);
\draw [color=c, fill=c] (12.8657,6.00691) rectangle (12.9055,6.11276);
\draw [color=c, fill=c] (12.9055,6.00691) rectangle (12.9453,6.11276);
\draw [color=c, fill=c] (12.9453,6.00691) rectangle (12.9851,6.11276);
\draw [color=c, fill=c] (12.9851,6.00691) rectangle (13.0249,6.11276);
\draw [color=c, fill=c] (13.0249,6.00691) rectangle (13.0647,6.11276);
\draw [color=c, fill=c] (13.0647,6.00691) rectangle (13.1045,6.11276);
\draw [color=c, fill=c] (13.1045,6.00691) rectangle (13.1443,6.11276);
\draw [color=c, fill=c] (13.1443,6.00691) rectangle (13.1841,6.11276);
\draw [color=c, fill=c] (13.1841,6.00691) rectangle (13.2239,6.11276);
\draw [color=c, fill=c] (13.2239,6.00691) rectangle (13.2637,6.11276);
\draw [color=c, fill=c] (13.2637,6.00691) rectangle (13.3035,6.11276);
\draw [color=c, fill=c] (13.3035,6.00691) rectangle (13.3433,6.11276);
\draw [color=c, fill=c] (13.3433,6.00691) rectangle (13.3831,6.11276);
\draw [color=c, fill=c] (13.3831,6.00691) rectangle (13.4229,6.11276);
\draw [color=c, fill=c] (13.4229,6.00691) rectangle (13.4627,6.11276);
\draw [color=c, fill=c] (13.4627,6.00691) rectangle (13.5025,6.11276);
\draw [color=c, fill=c] (13.5025,6.00691) rectangle (13.5423,6.11276);
\draw [color=c, fill=c] (13.5423,6.00691) rectangle (13.5821,6.11276);
\draw [color=c, fill=c] (13.5821,6.00691) rectangle (13.6219,6.11276);
\draw [color=c, fill=c] (13.6219,6.00691) rectangle (13.6617,6.11276);
\draw [color=c, fill=c] (13.6617,6.00691) rectangle (13.7015,6.11276);
\draw [color=c, fill=c] (13.7015,6.00691) rectangle (13.7413,6.11276);
\draw [color=c, fill=c] (13.7413,6.00691) rectangle (13.7811,6.11276);
\draw [color=c, fill=c] (13.7811,6.00691) rectangle (13.8209,6.11276);
\draw [color=c, fill=c] (13.8209,6.00691) rectangle (13.8607,6.11276);
\draw [color=c, fill=c] (13.8607,6.00691) rectangle (13.9005,6.11276);
\draw [color=c, fill=c] (13.9005,6.00691) rectangle (13.9403,6.11276);
\draw [color=c, fill=c] (13.9403,6.00691) rectangle (13.9801,6.11276);
\draw [color=c, fill=c] (13.9801,6.00691) rectangle (14.0199,6.11276);
\draw [color=c, fill=c] (14.0199,6.00691) rectangle (14.0597,6.11276);
\draw [color=c, fill=c] (14.0597,6.00691) rectangle (14.0995,6.11276);
\draw [color=c, fill=c] (14.0995,6.00691) rectangle (14.1393,6.11276);
\draw [color=c, fill=c] (14.1393,6.00691) rectangle (14.1791,6.11276);
\draw [color=c, fill=c] (14.1791,6.00691) rectangle (14.2189,6.11276);
\draw [color=c, fill=c] (14.2189,6.00691) rectangle (14.2587,6.11276);
\draw [color=c, fill=c] (14.2587,6.00691) rectangle (14.2985,6.11276);
\draw [color=c, fill=c] (14.2985,6.00691) rectangle (14.3383,6.11276);
\draw [color=c, fill=c] (14.3383,6.00691) rectangle (14.3781,6.11276);
\draw [color=c, fill=c] (14.3781,6.00691) rectangle (14.4179,6.11276);
\draw [color=c, fill=c] (14.4179,6.00691) rectangle (14.4577,6.11276);
\draw [color=c, fill=c] (14.4577,6.00691) rectangle (14.4975,6.11276);
\draw [color=c, fill=c] (14.4975,6.00691) rectangle (14.5373,6.11276);
\draw [color=c, fill=c] (14.5373,6.00691) rectangle (14.5771,6.11276);
\draw [color=c, fill=c] (14.5771,6.00691) rectangle (14.6169,6.11276);
\draw [color=c, fill=c] (14.6169,6.00691) rectangle (14.6567,6.11276);
\draw [color=c, fill=c] (14.6567,6.00691) rectangle (14.6965,6.11276);
\draw [color=c, fill=c] (14.6965,6.00691) rectangle (14.7363,6.11276);
\draw [color=c, fill=c] (14.7363,6.00691) rectangle (14.7761,6.11276);
\draw [color=c, fill=c] (14.7761,6.00691) rectangle (14.8159,6.11276);
\draw [color=c, fill=c] (14.8159,6.00691) rectangle (14.8557,6.11276);
\draw [color=c, fill=c] (14.8557,6.00691) rectangle (14.8955,6.11276);
\draw [color=c, fill=c] (14.8955,6.00691) rectangle (14.9353,6.11276);
\draw [color=c, fill=c] (14.9353,6.00691) rectangle (14.9751,6.11276);
\draw [color=c, fill=c] (14.9751,6.00691) rectangle (15.0149,6.11276);
\draw [color=c, fill=c] (15.0149,6.00691) rectangle (15.0547,6.11276);
\draw [color=c, fill=c] (15.0547,6.00691) rectangle (15.0945,6.11276);
\draw [color=c, fill=c] (15.0945,6.00691) rectangle (15.1343,6.11276);
\draw [color=c, fill=c] (15.1343,6.00691) rectangle (15.1741,6.11276);
\draw [color=c, fill=c] (15.1741,6.00691) rectangle (15.2139,6.11276);
\draw [color=c, fill=c] (15.2139,6.00691) rectangle (15.2537,6.11276);
\draw [color=c, fill=c] (15.2537,6.00691) rectangle (15.2935,6.11276);
\draw [color=c, fill=c] (15.2935,6.00691) rectangle (15.3333,6.11276);
\draw [color=c, fill=c] (15.3333,6.00691) rectangle (15.3731,6.11276);
\draw [color=c, fill=c] (15.3731,6.00691) rectangle (15.4129,6.11276);
\draw [color=c, fill=c] (15.4129,6.00691) rectangle (15.4527,6.11276);
\draw [color=c, fill=c] (15.4527,6.00691) rectangle (15.4925,6.11276);
\draw [color=c, fill=c] (15.4925,6.00691) rectangle (15.5323,6.11276);
\draw [color=c, fill=c] (15.5323,6.00691) rectangle (15.5721,6.11276);
\draw [color=c, fill=c] (15.5721,6.00691) rectangle (15.6119,6.11276);
\draw [color=c, fill=c] (15.6119,6.00691) rectangle (15.6517,6.11276);
\draw [color=c, fill=c] (15.6517,6.00691) rectangle (15.6915,6.11276);
\draw [color=c, fill=c] (15.6915,6.00691) rectangle (15.7313,6.11276);
\draw [color=c, fill=c] (15.7313,6.00691) rectangle (15.7711,6.11276);
\draw [color=c, fill=c] (15.7711,6.00691) rectangle (15.8109,6.11276);
\draw [color=c, fill=c] (15.8109,6.00691) rectangle (15.8507,6.11276);
\draw [color=c, fill=c] (15.8507,6.00691) rectangle (15.8905,6.11276);
\draw [color=c, fill=c] (15.8905,6.00691) rectangle (15.9303,6.11276);
\draw [color=c, fill=c] (15.9303,6.00691) rectangle (15.9701,6.11276);
\draw [color=c, fill=c] (15.9701,6.00691) rectangle (16.01,6.11276);
\draw [color=c, fill=c] (16.01,6.00691) rectangle (16.0498,6.11276);
\draw [color=c, fill=c] (16.0498,6.00691) rectangle (16.0896,6.11276);
\draw [color=c, fill=c] (16.0896,6.00691) rectangle (16.1294,6.11276);
\draw [color=c, fill=c] (16.1294,6.00691) rectangle (16.1692,6.11276);
\draw [color=c, fill=c] (16.1692,6.00691) rectangle (16.209,6.11276);
\draw [color=c, fill=c] (16.209,6.00691) rectangle (16.2488,6.11276);
\draw [color=c, fill=c] (16.2488,6.00691) rectangle (16.2886,6.11276);
\draw [color=c, fill=c] (16.2886,6.00691) rectangle (16.3284,6.11276);
\draw [color=c, fill=c] (16.3284,6.00691) rectangle (16.3682,6.11276);
\draw [color=c, fill=c] (16.3682,6.00691) rectangle (16.408,6.11276);
\draw [color=c, fill=c] (16.408,6.00691) rectangle (16.4478,6.11276);
\draw [color=c, fill=c] (16.4478,6.00691) rectangle (16.4876,6.11276);
\draw [color=c, fill=c] (16.4876,6.00691) rectangle (16.5274,6.11276);
\draw [color=c, fill=c] (16.5274,6.00691) rectangle (16.5672,6.11276);
\draw [color=c, fill=c] (16.5672,6.00691) rectangle (16.607,6.11276);
\draw [color=c, fill=c] (16.607,6.00691) rectangle (16.6468,6.11276);
\draw [color=c, fill=c] (16.6468,6.00691) rectangle (16.6866,6.11276);
\draw [color=c, fill=c] (16.6866,6.00691) rectangle (16.7264,6.11276);
\draw [color=c, fill=c] (16.7264,6.00691) rectangle (16.7662,6.11276);
\draw [color=c, fill=c] (16.7662,6.00691) rectangle (16.806,6.11276);
\draw [color=c, fill=c] (16.806,6.00691) rectangle (16.8458,6.11276);
\draw [color=c, fill=c] (16.8458,6.00691) rectangle (16.8856,6.11276);
\draw [color=c, fill=c] (16.8856,6.00691) rectangle (16.9254,6.11276);
\draw [color=c, fill=c] (16.9254,6.00691) rectangle (16.9652,6.11276);
\draw [color=c, fill=c] (16.9652,6.00691) rectangle (17.005,6.11276);
\draw [color=c, fill=c] (17.005,6.00691) rectangle (17.0448,6.11276);
\draw [color=c, fill=c] (17.0448,6.00691) rectangle (17.0846,6.11276);
\draw [color=c, fill=c] (17.0846,6.00691) rectangle (17.1244,6.11276);
\draw [color=c, fill=c] (17.1244,6.00691) rectangle (17.1642,6.11276);
\draw [color=c, fill=c] (17.1642,6.00691) rectangle (17.204,6.11276);
\draw [color=c, fill=c] (17.204,6.00691) rectangle (17.2438,6.11276);
\draw [color=c, fill=c] (17.2438,6.00691) rectangle (17.2836,6.11276);
\draw [color=c, fill=c] (17.2836,6.00691) rectangle (17.3234,6.11276);
\draw [color=c, fill=c] (17.3234,6.00691) rectangle (17.3632,6.11276);
\draw [color=c, fill=c] (17.3632,6.00691) rectangle (17.403,6.11276);
\draw [color=c, fill=c] (17.403,6.00691) rectangle (17.4428,6.11276);
\draw [color=c, fill=c] (17.4428,6.00691) rectangle (17.4826,6.11276);
\draw [color=c, fill=c] (17.4826,6.00691) rectangle (17.5224,6.11276);
\draw [color=c, fill=c] (17.5224,6.00691) rectangle (17.5622,6.11276);
\draw [color=c, fill=c] (17.5622,6.00691) rectangle (17.602,6.11276);
\draw [color=c, fill=c] (17.602,6.00691) rectangle (17.6418,6.11276);
\draw [color=c, fill=c] (17.6418,6.00691) rectangle (17.6816,6.11276);
\draw [color=c, fill=c] (17.6816,6.00691) rectangle (17.7214,6.11276);
\draw [color=c, fill=c] (17.7214,6.00691) rectangle (17.7612,6.11276);
\draw [color=c, fill=c] (17.7612,6.00691) rectangle (17.801,6.11276);
\draw [color=c, fill=c] (17.801,6.00691) rectangle (17.8408,6.11276);
\draw [color=c, fill=c] (17.8408,6.00691) rectangle (17.8806,6.11276);
\draw [color=c, fill=c] (17.8806,6.00691) rectangle (17.9204,6.11276);
\draw [color=c, fill=c] (17.9204,6.00691) rectangle (17.9602,6.11276);
\draw [color=c, fill=c] (17.9602,6.00691) rectangle (18,6.11276);
\definecolor{c}{rgb}{0,0.0800001,1};
\draw [color=c, fill=c] (2,6.11276) rectangle (2.0398,6.21861);
\draw [color=c, fill=c] (2.0398,6.11276) rectangle (2.0796,6.21861);
\draw [color=c, fill=c] (2.0796,6.11276) rectangle (2.1194,6.21861);
\draw [color=c, fill=c] (2.1194,6.11276) rectangle (2.1592,6.21861);
\draw [color=c, fill=c] (2.1592,6.11276) rectangle (2.19901,6.21861);
\draw [color=c, fill=c] (2.19901,6.11276) rectangle (2.23881,6.21861);
\draw [color=c, fill=c] (2.23881,6.11276) rectangle (2.27861,6.21861);
\draw [color=c, fill=c] (2.27861,6.11276) rectangle (2.31841,6.21861);
\draw [color=c, fill=c] (2.31841,6.11276) rectangle (2.35821,6.21861);
\draw [color=c, fill=c] (2.35821,6.11276) rectangle (2.39801,6.21861);
\draw [color=c, fill=c] (2.39801,6.11276) rectangle (2.43781,6.21861);
\draw [color=c, fill=c] (2.43781,6.11276) rectangle (2.47761,6.21861);
\draw [color=c, fill=c] (2.47761,6.11276) rectangle (2.51741,6.21861);
\draw [color=c, fill=c] (2.51741,6.11276) rectangle (2.55721,6.21861);
\draw [color=c, fill=c] (2.55721,6.11276) rectangle (2.59702,6.21861);
\draw [color=c, fill=c] (2.59702,6.11276) rectangle (2.63682,6.21861);
\draw [color=c, fill=c] (2.63682,6.11276) rectangle (2.67662,6.21861);
\draw [color=c, fill=c] (2.67662,6.11276) rectangle (2.71642,6.21861);
\draw [color=c, fill=c] (2.71642,6.11276) rectangle (2.75622,6.21861);
\draw [color=c, fill=c] (2.75622,6.11276) rectangle (2.79602,6.21861);
\draw [color=c, fill=c] (2.79602,6.11276) rectangle (2.83582,6.21861);
\draw [color=c, fill=c] (2.83582,6.11276) rectangle (2.87562,6.21861);
\draw [color=c, fill=c] (2.87562,6.11276) rectangle (2.91542,6.21861);
\draw [color=c, fill=c] (2.91542,6.11276) rectangle (2.95522,6.21861);
\draw [color=c, fill=c] (2.95522,6.11276) rectangle (2.99502,6.21861);
\draw [color=c, fill=c] (2.99502,6.11276) rectangle (3.03483,6.21861);
\draw [color=c, fill=c] (3.03483,6.11276) rectangle (3.07463,6.21861);
\draw [color=c, fill=c] (3.07463,6.11276) rectangle (3.11443,6.21861);
\draw [color=c, fill=c] (3.11443,6.11276) rectangle (3.15423,6.21861);
\draw [color=c, fill=c] (3.15423,6.11276) rectangle (3.19403,6.21861);
\draw [color=c, fill=c] (3.19403,6.11276) rectangle (3.23383,6.21861);
\draw [color=c, fill=c] (3.23383,6.11276) rectangle (3.27363,6.21861);
\draw [color=c, fill=c] (3.27363,6.11276) rectangle (3.31343,6.21861);
\draw [color=c, fill=c] (3.31343,6.11276) rectangle (3.35323,6.21861);
\draw [color=c, fill=c] (3.35323,6.11276) rectangle (3.39303,6.21861);
\draw [color=c, fill=c] (3.39303,6.11276) rectangle (3.43284,6.21861);
\draw [color=c, fill=c] (3.43284,6.11276) rectangle (3.47264,6.21861);
\draw [color=c, fill=c] (3.47264,6.11276) rectangle (3.51244,6.21861);
\draw [color=c, fill=c] (3.51244,6.11276) rectangle (3.55224,6.21861);
\draw [color=c, fill=c] (3.55224,6.11276) rectangle (3.59204,6.21861);
\draw [color=c, fill=c] (3.59204,6.11276) rectangle (3.63184,6.21861);
\draw [color=c, fill=c] (3.63184,6.11276) rectangle (3.67164,6.21861);
\draw [color=c, fill=c] (3.67164,6.11276) rectangle (3.71144,6.21861);
\draw [color=c, fill=c] (3.71144,6.11276) rectangle (3.75124,6.21861);
\draw [color=c, fill=c] (3.75124,6.11276) rectangle (3.79104,6.21861);
\draw [color=c, fill=c] (3.79104,6.11276) rectangle (3.83085,6.21861);
\draw [color=c, fill=c] (3.83085,6.11276) rectangle (3.87065,6.21861);
\draw [color=c, fill=c] (3.87065,6.11276) rectangle (3.91045,6.21861);
\draw [color=c, fill=c] (3.91045,6.11276) rectangle (3.95025,6.21861);
\draw [color=c, fill=c] (3.95025,6.11276) rectangle (3.99005,6.21861);
\draw [color=c, fill=c] (3.99005,6.11276) rectangle (4.02985,6.21861);
\draw [color=c, fill=c] (4.02985,6.11276) rectangle (4.06965,6.21861);
\draw [color=c, fill=c] (4.06965,6.11276) rectangle (4.10945,6.21861);
\draw [color=c, fill=c] (4.10945,6.11276) rectangle (4.14925,6.21861);
\draw [color=c, fill=c] (4.14925,6.11276) rectangle (4.18905,6.21861);
\draw [color=c, fill=c] (4.18905,6.11276) rectangle (4.22886,6.21861);
\draw [color=c, fill=c] (4.22886,6.11276) rectangle (4.26866,6.21861);
\draw [color=c, fill=c] (4.26866,6.11276) rectangle (4.30846,6.21861);
\draw [color=c, fill=c] (4.30846,6.11276) rectangle (4.34826,6.21861);
\draw [color=c, fill=c] (4.34826,6.11276) rectangle (4.38806,6.21861);
\draw [color=c, fill=c] (4.38806,6.11276) rectangle (4.42786,6.21861);
\draw [color=c, fill=c] (4.42786,6.11276) rectangle (4.46766,6.21861);
\draw [color=c, fill=c] (4.46766,6.11276) rectangle (4.50746,6.21861);
\draw [color=c, fill=c] (4.50746,6.11276) rectangle (4.54726,6.21861);
\draw [color=c, fill=c] (4.54726,6.11276) rectangle (4.58706,6.21861);
\draw [color=c, fill=c] (4.58706,6.11276) rectangle (4.62687,6.21861);
\draw [color=c, fill=c] (4.62687,6.11276) rectangle (4.66667,6.21861);
\draw [color=c, fill=c] (4.66667,6.11276) rectangle (4.70647,6.21861);
\draw [color=c, fill=c] (4.70647,6.11276) rectangle (4.74627,6.21861);
\draw [color=c, fill=c] (4.74627,6.11276) rectangle (4.78607,6.21861);
\draw [color=c, fill=c] (4.78607,6.11276) rectangle (4.82587,6.21861);
\draw [color=c, fill=c] (4.82587,6.11276) rectangle (4.86567,6.21861);
\draw [color=c, fill=c] (4.86567,6.11276) rectangle (4.90547,6.21861);
\draw [color=c, fill=c] (4.90547,6.11276) rectangle (4.94527,6.21861);
\draw [color=c, fill=c] (4.94527,6.11276) rectangle (4.98507,6.21861);
\draw [color=c, fill=c] (4.98507,6.11276) rectangle (5.02488,6.21861);
\draw [color=c, fill=c] (5.02488,6.11276) rectangle (5.06468,6.21861);
\draw [color=c, fill=c] (5.06468,6.11276) rectangle (5.10448,6.21861);
\draw [color=c, fill=c] (5.10448,6.11276) rectangle (5.14428,6.21861);
\draw [color=c, fill=c] (5.14428,6.11276) rectangle (5.18408,6.21861);
\draw [color=c, fill=c] (5.18408,6.11276) rectangle (5.22388,6.21861);
\draw [color=c, fill=c] (5.22388,6.11276) rectangle (5.26368,6.21861);
\draw [color=c, fill=c] (5.26368,6.11276) rectangle (5.30348,6.21861);
\definecolor{c}{rgb}{0.2,0,1};
\draw [color=c, fill=c] (5.30348,6.11276) rectangle (5.34328,6.21861);
\draw [color=c, fill=c] (5.34328,6.11276) rectangle (5.38308,6.21861);
\draw [color=c, fill=c] (5.38308,6.11276) rectangle (5.42289,6.21861);
\draw [color=c, fill=c] (5.42289,6.11276) rectangle (5.46269,6.21861);
\draw [color=c, fill=c] (5.46269,6.11276) rectangle (5.50249,6.21861);
\draw [color=c, fill=c] (5.50249,6.11276) rectangle (5.54229,6.21861);
\draw [color=c, fill=c] (5.54229,6.11276) rectangle (5.58209,6.21861);
\draw [color=c, fill=c] (5.58209,6.11276) rectangle (5.62189,6.21861);
\draw [color=c, fill=c] (5.62189,6.11276) rectangle (5.66169,6.21861);
\draw [color=c, fill=c] (5.66169,6.11276) rectangle (5.70149,6.21861);
\draw [color=c, fill=c] (5.70149,6.11276) rectangle (5.74129,6.21861);
\draw [color=c, fill=c] (5.74129,6.11276) rectangle (5.78109,6.21861);
\draw [color=c, fill=c] (5.78109,6.11276) rectangle (5.8209,6.21861);
\draw [color=c, fill=c] (5.8209,6.11276) rectangle (5.8607,6.21861);
\draw [color=c, fill=c] (5.8607,6.11276) rectangle (5.9005,6.21861);
\draw [color=c, fill=c] (5.9005,6.11276) rectangle (5.9403,6.21861);
\draw [color=c, fill=c] (5.9403,6.11276) rectangle (5.9801,6.21861);
\draw [color=c, fill=c] (5.9801,6.11276) rectangle (6.0199,6.21861);
\draw [color=c, fill=c] (6.0199,6.11276) rectangle (6.0597,6.21861);
\draw [color=c, fill=c] (6.0597,6.11276) rectangle (6.0995,6.21861);
\draw [color=c, fill=c] (6.0995,6.11276) rectangle (6.1393,6.21861);
\draw [color=c, fill=c] (6.1393,6.11276) rectangle (6.1791,6.21861);
\draw [color=c, fill=c] (6.1791,6.11276) rectangle (6.21891,6.21861);
\draw [color=c, fill=c] (6.21891,6.11276) rectangle (6.25871,6.21861);
\draw [color=c, fill=c] (6.25871,6.11276) rectangle (6.29851,6.21861);
\draw [color=c, fill=c] (6.29851,6.11276) rectangle (6.33831,6.21861);
\draw [color=c, fill=c] (6.33831,6.11276) rectangle (6.37811,6.21861);
\draw [color=c, fill=c] (6.37811,6.11276) rectangle (6.41791,6.21861);
\draw [color=c, fill=c] (6.41791,6.11276) rectangle (6.45771,6.21861);
\draw [color=c, fill=c] (6.45771,6.11276) rectangle (6.49751,6.21861);
\draw [color=c, fill=c] (6.49751,6.11276) rectangle (6.53731,6.21861);
\draw [color=c, fill=c] (6.53731,6.11276) rectangle (6.57711,6.21861);
\draw [color=c, fill=c] (6.57711,6.11276) rectangle (6.61692,6.21861);
\draw [color=c, fill=c] (6.61692,6.11276) rectangle (6.65672,6.21861);
\draw [color=c, fill=c] (6.65672,6.11276) rectangle (6.69652,6.21861);
\draw [color=c, fill=c] (6.69652,6.11276) rectangle (6.73632,6.21861);
\draw [color=c, fill=c] (6.73632,6.11276) rectangle (6.77612,6.21861);
\draw [color=c, fill=c] (6.77612,6.11276) rectangle (6.81592,6.21861);
\draw [color=c, fill=c] (6.81592,6.11276) rectangle (6.85572,6.21861);
\draw [color=c, fill=c] (6.85572,6.11276) rectangle (6.89552,6.21861);
\draw [color=c, fill=c] (6.89552,6.11276) rectangle (6.93532,6.21861);
\draw [color=c, fill=c] (6.93532,6.11276) rectangle (6.97512,6.21861);
\draw [color=c, fill=c] (6.97512,6.11276) rectangle (7.01493,6.21861);
\draw [color=c, fill=c] (7.01493,6.11276) rectangle (7.05473,6.21861);
\draw [color=c, fill=c] (7.05473,6.11276) rectangle (7.09453,6.21861);
\draw [color=c, fill=c] (7.09453,6.11276) rectangle (7.13433,6.21861);
\draw [color=c, fill=c] (7.13433,6.11276) rectangle (7.17413,6.21861);
\draw [color=c, fill=c] (7.17413,6.11276) rectangle (7.21393,6.21861);
\draw [color=c, fill=c] (7.21393,6.11276) rectangle (7.25373,6.21861);
\draw [color=c, fill=c] (7.25373,6.11276) rectangle (7.29353,6.21861);
\draw [color=c, fill=c] (7.29353,6.11276) rectangle (7.33333,6.21861);
\draw [color=c, fill=c] (7.33333,6.11276) rectangle (7.37313,6.21861);
\draw [color=c, fill=c] (7.37313,6.11276) rectangle (7.41294,6.21861);
\draw [color=c, fill=c] (7.41294,6.11276) rectangle (7.45274,6.21861);
\draw [color=c, fill=c] (7.45274,6.11276) rectangle (7.49254,6.21861);
\draw [color=c, fill=c] (7.49254,6.11276) rectangle (7.53234,6.21861);
\draw [color=c, fill=c] (7.53234,6.11276) rectangle (7.57214,6.21861);
\draw [color=c, fill=c] (7.57214,6.11276) rectangle (7.61194,6.21861);
\draw [color=c, fill=c] (7.61194,6.11276) rectangle (7.65174,6.21861);
\draw [color=c, fill=c] (7.65174,6.11276) rectangle (7.69154,6.21861);
\draw [color=c, fill=c] (7.69154,6.11276) rectangle (7.73134,6.21861);
\draw [color=c, fill=c] (7.73134,6.11276) rectangle (7.77114,6.21861);
\draw [color=c, fill=c] (7.77114,6.11276) rectangle (7.81095,6.21861);
\draw [color=c, fill=c] (7.81095,6.11276) rectangle (7.85075,6.21861);
\draw [color=c, fill=c] (7.85075,6.11276) rectangle (7.89055,6.21861);
\draw [color=c, fill=c] (7.89055,6.11276) rectangle (7.93035,6.21861);
\draw [color=c, fill=c] (7.93035,6.11276) rectangle (7.97015,6.21861);
\draw [color=c, fill=c] (7.97015,6.11276) rectangle (8.00995,6.21861);
\draw [color=c, fill=c] (8.00995,6.11276) rectangle (8.04975,6.21861);
\draw [color=c, fill=c] (8.04975,6.11276) rectangle (8.08955,6.21861);
\draw [color=c, fill=c] (8.08955,6.11276) rectangle (8.12935,6.21861);
\draw [color=c, fill=c] (8.12935,6.11276) rectangle (8.16915,6.21861);
\draw [color=c, fill=c] (8.16915,6.11276) rectangle (8.20895,6.21861);
\draw [color=c, fill=c] (8.20895,6.11276) rectangle (8.24876,6.21861);
\draw [color=c, fill=c] (8.24876,6.11276) rectangle (8.28856,6.21861);
\draw [color=c, fill=c] (8.28856,6.11276) rectangle (8.32836,6.21861);
\draw [color=c, fill=c] (8.32836,6.11276) rectangle (8.36816,6.21861);
\draw [color=c, fill=c] (8.36816,6.11276) rectangle (8.40796,6.21861);
\draw [color=c, fill=c] (8.40796,6.11276) rectangle (8.44776,6.21861);
\draw [color=c, fill=c] (8.44776,6.11276) rectangle (8.48756,6.21861);
\draw [color=c, fill=c] (8.48756,6.11276) rectangle (8.52736,6.21861);
\draw [color=c, fill=c] (8.52736,6.11276) rectangle (8.56716,6.21861);
\draw [color=c, fill=c] (8.56716,6.11276) rectangle (8.60697,6.21861);
\draw [color=c, fill=c] (8.60697,6.11276) rectangle (8.64677,6.21861);
\draw [color=c, fill=c] (8.64677,6.11276) rectangle (8.68657,6.21861);
\draw [color=c, fill=c] (8.68657,6.11276) rectangle (8.72637,6.21861);
\draw [color=c, fill=c] (8.72637,6.11276) rectangle (8.76617,6.21861);
\draw [color=c, fill=c] (8.76617,6.11276) rectangle (8.80597,6.21861);
\draw [color=c, fill=c] (8.80597,6.11276) rectangle (8.84577,6.21861);
\draw [color=c, fill=c] (8.84577,6.11276) rectangle (8.88557,6.21861);
\definecolor{c}{rgb}{0,0.0800001,1};
\draw [color=c, fill=c] (8.88557,6.11276) rectangle (8.92537,6.21861);
\draw [color=c, fill=c] (8.92537,6.11276) rectangle (8.96517,6.21861);
\draw [color=c, fill=c] (8.96517,6.11276) rectangle (9.00498,6.21861);
\draw [color=c, fill=c] (9.00498,6.11276) rectangle (9.04478,6.21861);
\draw [color=c, fill=c] (9.04478,6.11276) rectangle (9.08458,6.21861);
\draw [color=c, fill=c] (9.08458,6.11276) rectangle (9.12438,6.21861);
\draw [color=c, fill=c] (9.12438,6.11276) rectangle (9.16418,6.21861);
\draw [color=c, fill=c] (9.16418,6.11276) rectangle (9.20398,6.21861);
\draw [color=c, fill=c] (9.20398,6.11276) rectangle (9.24378,6.21861);
\draw [color=c, fill=c] (9.24378,6.11276) rectangle (9.28358,6.21861);
\draw [color=c, fill=c] (9.28358,6.11276) rectangle (9.32338,6.21861);
\draw [color=c, fill=c] (9.32338,6.11276) rectangle (9.36318,6.21861);
\draw [color=c, fill=c] (9.36318,6.11276) rectangle (9.40298,6.21861);
\draw [color=c, fill=c] (9.40298,6.11276) rectangle (9.44279,6.21861);
\draw [color=c, fill=c] (9.44279,6.11276) rectangle (9.48259,6.21861);
\draw [color=c, fill=c] (9.48259,6.11276) rectangle (9.52239,6.21861);
\draw [color=c, fill=c] (9.52239,6.11276) rectangle (9.56219,6.21861);
\draw [color=c, fill=c] (9.56219,6.11276) rectangle (9.60199,6.21861);
\draw [color=c, fill=c] (9.60199,6.11276) rectangle (9.64179,6.21861);
\draw [color=c, fill=c] (9.64179,6.11276) rectangle (9.68159,6.21861);
\definecolor{c}{rgb}{0,0.266667,1};
\draw [color=c, fill=c] (9.68159,6.11276) rectangle (9.72139,6.21861);
\draw [color=c, fill=c] (9.72139,6.11276) rectangle (9.76119,6.21861);
\draw [color=c, fill=c] (9.76119,6.11276) rectangle (9.80099,6.21861);
\draw [color=c, fill=c] (9.80099,6.11276) rectangle (9.8408,6.21861);
\draw [color=c, fill=c] (9.8408,6.11276) rectangle (9.8806,6.21861);
\draw [color=c, fill=c] (9.8806,6.11276) rectangle (9.9204,6.21861);
\draw [color=c, fill=c] (9.9204,6.11276) rectangle (9.9602,6.21861);
\draw [color=c, fill=c] (9.9602,6.11276) rectangle (10,6.21861);
\draw [color=c, fill=c] (10,6.11276) rectangle (10.0398,6.21861);
\draw [color=c, fill=c] (10.0398,6.11276) rectangle (10.0796,6.21861);
\draw [color=c, fill=c] (10.0796,6.11276) rectangle (10.1194,6.21861);
\definecolor{c}{rgb}{0,0.546666,1};
\draw [color=c, fill=c] (10.1194,6.11276) rectangle (10.1592,6.21861);
\draw [color=c, fill=c] (10.1592,6.11276) rectangle (10.199,6.21861);
\draw [color=c, fill=c] (10.199,6.11276) rectangle (10.2388,6.21861);
\draw [color=c, fill=c] (10.2388,6.11276) rectangle (10.2786,6.21861);
\draw [color=c, fill=c] (10.2786,6.11276) rectangle (10.3184,6.21861);
\draw [color=c, fill=c] (10.3184,6.11276) rectangle (10.3582,6.21861);
\draw [color=c, fill=c] (10.3582,6.11276) rectangle (10.398,6.21861);
\draw [color=c, fill=c] (10.398,6.11276) rectangle (10.4378,6.21861);
\draw [color=c, fill=c] (10.4378,6.11276) rectangle (10.4776,6.21861);
\draw [color=c, fill=c] (10.4776,6.11276) rectangle (10.5174,6.21861);
\draw [color=c, fill=c] (10.5174,6.11276) rectangle (10.5572,6.21861);
\draw [color=c, fill=c] (10.5572,6.11276) rectangle (10.597,6.21861);
\draw [color=c, fill=c] (10.597,6.11276) rectangle (10.6368,6.21861);
\draw [color=c, fill=c] (10.6368,6.11276) rectangle (10.6766,6.21861);
\draw [color=c, fill=c] (10.6766,6.11276) rectangle (10.7164,6.21861);
\draw [color=c, fill=c] (10.7164,6.11276) rectangle (10.7562,6.21861);
\definecolor{c}{rgb}{0,0.733333,1};
\draw [color=c, fill=c] (10.7562,6.11276) rectangle (10.796,6.21861);
\draw [color=c, fill=c] (10.796,6.11276) rectangle (10.8358,6.21861);
\draw [color=c, fill=c] (10.8358,6.11276) rectangle (10.8756,6.21861);
\draw [color=c, fill=c] (10.8756,6.11276) rectangle (10.9154,6.21861);
\draw [color=c, fill=c] (10.9154,6.11276) rectangle (10.9552,6.21861);
\draw [color=c, fill=c] (10.9552,6.11276) rectangle (10.995,6.21861);
\draw [color=c, fill=c] (10.995,6.11276) rectangle (11.0348,6.21861);
\draw [color=c, fill=c] (11.0348,6.11276) rectangle (11.0746,6.21861);
\draw [color=c, fill=c] (11.0746,6.11276) rectangle (11.1144,6.21861);
\draw [color=c, fill=c] (11.1144,6.11276) rectangle (11.1542,6.21861);
\draw [color=c, fill=c] (11.1542,6.11276) rectangle (11.194,6.21861);
\draw [color=c, fill=c] (11.194,6.11276) rectangle (11.2338,6.21861);
\draw [color=c, fill=c] (11.2338,6.11276) rectangle (11.2736,6.21861);
\draw [color=c, fill=c] (11.2736,6.11276) rectangle (11.3134,6.21861);
\draw [color=c, fill=c] (11.3134,6.11276) rectangle (11.3532,6.21861);
\draw [color=c, fill=c] (11.3532,6.11276) rectangle (11.393,6.21861);
\draw [color=c, fill=c] (11.393,6.11276) rectangle (11.4328,6.21861);
\draw [color=c, fill=c] (11.4328,6.11276) rectangle (11.4726,6.21861);
\draw [color=c, fill=c] (11.4726,6.11276) rectangle (11.5124,6.21861);
\draw [color=c, fill=c] (11.5124,6.11276) rectangle (11.5522,6.21861);
\draw [color=c, fill=c] (11.5522,6.11276) rectangle (11.592,6.21861);
\draw [color=c, fill=c] (11.592,6.11276) rectangle (11.6318,6.21861);
\draw [color=c, fill=c] (11.6318,6.11276) rectangle (11.6716,6.21861);
\draw [color=c, fill=c] (11.6716,6.11276) rectangle (11.7114,6.21861);
\draw [color=c, fill=c] (11.7114,6.11276) rectangle (11.7512,6.21861);
\draw [color=c, fill=c] (11.7512,6.11276) rectangle (11.791,6.21861);
\draw [color=c, fill=c] (11.791,6.11276) rectangle (11.8308,6.21861);
\draw [color=c, fill=c] (11.8308,6.11276) rectangle (11.8706,6.21861);
\draw [color=c, fill=c] (11.8706,6.11276) rectangle (11.9104,6.21861);
\draw [color=c, fill=c] (11.9104,6.11276) rectangle (11.9502,6.21861);
\draw [color=c, fill=c] (11.9502,6.11276) rectangle (11.99,6.21861);
\draw [color=c, fill=c] (11.99,6.11276) rectangle (12.0299,6.21861);
\draw [color=c, fill=c] (12.0299,6.11276) rectangle (12.0697,6.21861);
\draw [color=c, fill=c] (12.0697,6.11276) rectangle (12.1095,6.21861);
\draw [color=c, fill=c] (12.1095,6.11276) rectangle (12.1493,6.21861);
\draw [color=c, fill=c] (12.1493,6.11276) rectangle (12.1891,6.21861);
\draw [color=c, fill=c] (12.1891,6.11276) rectangle (12.2289,6.21861);
\draw [color=c, fill=c] (12.2289,6.11276) rectangle (12.2687,6.21861);
\draw [color=c, fill=c] (12.2687,6.11276) rectangle (12.3085,6.21861);
\draw [color=c, fill=c] (12.3085,6.11276) rectangle (12.3483,6.21861);
\draw [color=c, fill=c] (12.3483,6.11276) rectangle (12.3881,6.21861);
\draw [color=c, fill=c] (12.3881,6.11276) rectangle (12.4279,6.21861);
\draw [color=c, fill=c] (12.4279,6.11276) rectangle (12.4677,6.21861);
\draw [color=c, fill=c] (12.4677,6.11276) rectangle (12.5075,6.21861);
\draw [color=c, fill=c] (12.5075,6.11276) rectangle (12.5473,6.21861);
\draw [color=c, fill=c] (12.5473,6.11276) rectangle (12.5871,6.21861);
\draw [color=c, fill=c] (12.5871,6.11276) rectangle (12.6269,6.21861);
\draw [color=c, fill=c] (12.6269,6.11276) rectangle (12.6667,6.21861);
\draw [color=c, fill=c] (12.6667,6.11276) rectangle (12.7065,6.21861);
\draw [color=c, fill=c] (12.7065,6.11276) rectangle (12.7463,6.21861);
\draw [color=c, fill=c] (12.7463,6.11276) rectangle (12.7861,6.21861);
\draw [color=c, fill=c] (12.7861,6.11276) rectangle (12.8259,6.21861);
\draw [color=c, fill=c] (12.8259,6.11276) rectangle (12.8657,6.21861);
\draw [color=c, fill=c] (12.8657,6.11276) rectangle (12.9055,6.21861);
\draw [color=c, fill=c] (12.9055,6.11276) rectangle (12.9453,6.21861);
\draw [color=c, fill=c] (12.9453,6.11276) rectangle (12.9851,6.21861);
\draw [color=c, fill=c] (12.9851,6.11276) rectangle (13.0249,6.21861);
\draw [color=c, fill=c] (13.0249,6.11276) rectangle (13.0647,6.21861);
\draw [color=c, fill=c] (13.0647,6.11276) rectangle (13.1045,6.21861);
\draw [color=c, fill=c] (13.1045,6.11276) rectangle (13.1443,6.21861);
\draw [color=c, fill=c] (13.1443,6.11276) rectangle (13.1841,6.21861);
\draw [color=c, fill=c] (13.1841,6.11276) rectangle (13.2239,6.21861);
\draw [color=c, fill=c] (13.2239,6.11276) rectangle (13.2637,6.21861);
\draw [color=c, fill=c] (13.2637,6.11276) rectangle (13.3035,6.21861);
\draw [color=c, fill=c] (13.3035,6.11276) rectangle (13.3433,6.21861);
\draw [color=c, fill=c] (13.3433,6.11276) rectangle (13.3831,6.21861);
\draw [color=c, fill=c] (13.3831,6.11276) rectangle (13.4229,6.21861);
\draw [color=c, fill=c] (13.4229,6.11276) rectangle (13.4627,6.21861);
\draw [color=c, fill=c] (13.4627,6.11276) rectangle (13.5025,6.21861);
\draw [color=c, fill=c] (13.5025,6.11276) rectangle (13.5423,6.21861);
\draw [color=c, fill=c] (13.5423,6.11276) rectangle (13.5821,6.21861);
\draw [color=c, fill=c] (13.5821,6.11276) rectangle (13.6219,6.21861);
\draw [color=c, fill=c] (13.6219,6.11276) rectangle (13.6617,6.21861);
\draw [color=c, fill=c] (13.6617,6.11276) rectangle (13.7015,6.21861);
\draw [color=c, fill=c] (13.7015,6.11276) rectangle (13.7413,6.21861);
\draw [color=c, fill=c] (13.7413,6.11276) rectangle (13.7811,6.21861);
\draw [color=c, fill=c] (13.7811,6.11276) rectangle (13.8209,6.21861);
\draw [color=c, fill=c] (13.8209,6.11276) rectangle (13.8607,6.21861);
\draw [color=c, fill=c] (13.8607,6.11276) rectangle (13.9005,6.21861);
\draw [color=c, fill=c] (13.9005,6.11276) rectangle (13.9403,6.21861);
\draw [color=c, fill=c] (13.9403,6.11276) rectangle (13.9801,6.21861);
\draw [color=c, fill=c] (13.9801,6.11276) rectangle (14.0199,6.21861);
\draw [color=c, fill=c] (14.0199,6.11276) rectangle (14.0597,6.21861);
\draw [color=c, fill=c] (14.0597,6.11276) rectangle (14.0995,6.21861);
\draw [color=c, fill=c] (14.0995,6.11276) rectangle (14.1393,6.21861);
\draw [color=c, fill=c] (14.1393,6.11276) rectangle (14.1791,6.21861);
\draw [color=c, fill=c] (14.1791,6.11276) rectangle (14.2189,6.21861);
\draw [color=c, fill=c] (14.2189,6.11276) rectangle (14.2587,6.21861);
\draw [color=c, fill=c] (14.2587,6.11276) rectangle (14.2985,6.21861);
\draw [color=c, fill=c] (14.2985,6.11276) rectangle (14.3383,6.21861);
\draw [color=c, fill=c] (14.3383,6.11276) rectangle (14.3781,6.21861);
\draw [color=c, fill=c] (14.3781,6.11276) rectangle (14.4179,6.21861);
\draw [color=c, fill=c] (14.4179,6.11276) rectangle (14.4577,6.21861);
\draw [color=c, fill=c] (14.4577,6.11276) rectangle (14.4975,6.21861);
\draw [color=c, fill=c] (14.4975,6.11276) rectangle (14.5373,6.21861);
\draw [color=c, fill=c] (14.5373,6.11276) rectangle (14.5771,6.21861);
\draw [color=c, fill=c] (14.5771,6.11276) rectangle (14.6169,6.21861);
\draw [color=c, fill=c] (14.6169,6.11276) rectangle (14.6567,6.21861);
\draw [color=c, fill=c] (14.6567,6.11276) rectangle (14.6965,6.21861);
\draw [color=c, fill=c] (14.6965,6.11276) rectangle (14.7363,6.21861);
\draw [color=c, fill=c] (14.7363,6.11276) rectangle (14.7761,6.21861);
\draw [color=c, fill=c] (14.7761,6.11276) rectangle (14.8159,6.21861);
\draw [color=c, fill=c] (14.8159,6.11276) rectangle (14.8557,6.21861);
\draw [color=c, fill=c] (14.8557,6.11276) rectangle (14.8955,6.21861);
\draw [color=c, fill=c] (14.8955,6.11276) rectangle (14.9353,6.21861);
\draw [color=c, fill=c] (14.9353,6.11276) rectangle (14.9751,6.21861);
\draw [color=c, fill=c] (14.9751,6.11276) rectangle (15.0149,6.21861);
\draw [color=c, fill=c] (15.0149,6.11276) rectangle (15.0547,6.21861);
\draw [color=c, fill=c] (15.0547,6.11276) rectangle (15.0945,6.21861);
\draw [color=c, fill=c] (15.0945,6.11276) rectangle (15.1343,6.21861);
\draw [color=c, fill=c] (15.1343,6.11276) rectangle (15.1741,6.21861);
\draw [color=c, fill=c] (15.1741,6.11276) rectangle (15.2139,6.21861);
\draw [color=c, fill=c] (15.2139,6.11276) rectangle (15.2537,6.21861);
\draw [color=c, fill=c] (15.2537,6.11276) rectangle (15.2935,6.21861);
\draw [color=c, fill=c] (15.2935,6.11276) rectangle (15.3333,6.21861);
\draw [color=c, fill=c] (15.3333,6.11276) rectangle (15.3731,6.21861);
\draw [color=c, fill=c] (15.3731,6.11276) rectangle (15.4129,6.21861);
\draw [color=c, fill=c] (15.4129,6.11276) rectangle (15.4527,6.21861);
\draw [color=c, fill=c] (15.4527,6.11276) rectangle (15.4925,6.21861);
\draw [color=c, fill=c] (15.4925,6.11276) rectangle (15.5323,6.21861);
\draw [color=c, fill=c] (15.5323,6.11276) rectangle (15.5721,6.21861);
\draw [color=c, fill=c] (15.5721,6.11276) rectangle (15.6119,6.21861);
\draw [color=c, fill=c] (15.6119,6.11276) rectangle (15.6517,6.21861);
\draw [color=c, fill=c] (15.6517,6.11276) rectangle (15.6915,6.21861);
\draw [color=c, fill=c] (15.6915,6.11276) rectangle (15.7313,6.21861);
\draw [color=c, fill=c] (15.7313,6.11276) rectangle (15.7711,6.21861);
\draw [color=c, fill=c] (15.7711,6.11276) rectangle (15.8109,6.21861);
\draw [color=c, fill=c] (15.8109,6.11276) rectangle (15.8507,6.21861);
\draw [color=c, fill=c] (15.8507,6.11276) rectangle (15.8905,6.21861);
\draw [color=c, fill=c] (15.8905,6.11276) rectangle (15.9303,6.21861);
\draw [color=c, fill=c] (15.9303,6.11276) rectangle (15.9701,6.21861);
\draw [color=c, fill=c] (15.9701,6.11276) rectangle (16.01,6.21861);
\draw [color=c, fill=c] (16.01,6.11276) rectangle (16.0498,6.21861);
\draw [color=c, fill=c] (16.0498,6.11276) rectangle (16.0896,6.21861);
\draw [color=c, fill=c] (16.0896,6.11276) rectangle (16.1294,6.21861);
\draw [color=c, fill=c] (16.1294,6.11276) rectangle (16.1692,6.21861);
\draw [color=c, fill=c] (16.1692,6.11276) rectangle (16.209,6.21861);
\draw [color=c, fill=c] (16.209,6.11276) rectangle (16.2488,6.21861);
\draw [color=c, fill=c] (16.2488,6.11276) rectangle (16.2886,6.21861);
\draw [color=c, fill=c] (16.2886,6.11276) rectangle (16.3284,6.21861);
\draw [color=c, fill=c] (16.3284,6.11276) rectangle (16.3682,6.21861);
\draw [color=c, fill=c] (16.3682,6.11276) rectangle (16.408,6.21861);
\draw [color=c, fill=c] (16.408,6.11276) rectangle (16.4478,6.21861);
\draw [color=c, fill=c] (16.4478,6.11276) rectangle (16.4876,6.21861);
\draw [color=c, fill=c] (16.4876,6.11276) rectangle (16.5274,6.21861);
\draw [color=c, fill=c] (16.5274,6.11276) rectangle (16.5672,6.21861);
\draw [color=c, fill=c] (16.5672,6.11276) rectangle (16.607,6.21861);
\draw [color=c, fill=c] (16.607,6.11276) rectangle (16.6468,6.21861);
\draw [color=c, fill=c] (16.6468,6.11276) rectangle (16.6866,6.21861);
\draw [color=c, fill=c] (16.6866,6.11276) rectangle (16.7264,6.21861);
\draw [color=c, fill=c] (16.7264,6.11276) rectangle (16.7662,6.21861);
\draw [color=c, fill=c] (16.7662,6.11276) rectangle (16.806,6.21861);
\draw [color=c, fill=c] (16.806,6.11276) rectangle (16.8458,6.21861);
\draw [color=c, fill=c] (16.8458,6.11276) rectangle (16.8856,6.21861);
\draw [color=c, fill=c] (16.8856,6.11276) rectangle (16.9254,6.21861);
\draw [color=c, fill=c] (16.9254,6.11276) rectangle (16.9652,6.21861);
\draw [color=c, fill=c] (16.9652,6.11276) rectangle (17.005,6.21861);
\draw [color=c, fill=c] (17.005,6.11276) rectangle (17.0448,6.21861);
\draw [color=c, fill=c] (17.0448,6.11276) rectangle (17.0846,6.21861);
\draw [color=c, fill=c] (17.0846,6.11276) rectangle (17.1244,6.21861);
\draw [color=c, fill=c] (17.1244,6.11276) rectangle (17.1642,6.21861);
\draw [color=c, fill=c] (17.1642,6.11276) rectangle (17.204,6.21861);
\draw [color=c, fill=c] (17.204,6.11276) rectangle (17.2438,6.21861);
\draw [color=c, fill=c] (17.2438,6.11276) rectangle (17.2836,6.21861);
\draw [color=c, fill=c] (17.2836,6.11276) rectangle (17.3234,6.21861);
\draw [color=c, fill=c] (17.3234,6.11276) rectangle (17.3632,6.21861);
\draw [color=c, fill=c] (17.3632,6.11276) rectangle (17.403,6.21861);
\draw [color=c, fill=c] (17.403,6.11276) rectangle (17.4428,6.21861);
\draw [color=c, fill=c] (17.4428,6.11276) rectangle (17.4826,6.21861);
\draw [color=c, fill=c] (17.4826,6.11276) rectangle (17.5224,6.21861);
\draw [color=c, fill=c] (17.5224,6.11276) rectangle (17.5622,6.21861);
\draw [color=c, fill=c] (17.5622,6.11276) rectangle (17.602,6.21861);
\draw [color=c, fill=c] (17.602,6.11276) rectangle (17.6418,6.21861);
\draw [color=c, fill=c] (17.6418,6.11276) rectangle (17.6816,6.21861);
\draw [color=c, fill=c] (17.6816,6.11276) rectangle (17.7214,6.21861);
\draw [color=c, fill=c] (17.7214,6.11276) rectangle (17.7612,6.21861);
\draw [color=c, fill=c] (17.7612,6.11276) rectangle (17.801,6.21861);
\draw [color=c, fill=c] (17.801,6.11276) rectangle (17.8408,6.21861);
\draw [color=c, fill=c] (17.8408,6.11276) rectangle (17.8806,6.21861);
\draw [color=c, fill=c] (17.8806,6.11276) rectangle (17.9204,6.21861);
\draw [color=c, fill=c] (17.9204,6.11276) rectangle (17.9602,6.21861);
\draw [color=c, fill=c] (17.9602,6.11276) rectangle (18,6.21861);
\definecolor{c}{rgb}{0,0.0800001,1};
\draw [color=c, fill=c] (2,6.21861) rectangle (2.0398,6.32446);
\draw [color=c, fill=c] (2.0398,6.21861) rectangle (2.0796,6.32446);
\draw [color=c, fill=c] (2.0796,6.21861) rectangle (2.1194,6.32446);
\draw [color=c, fill=c] (2.1194,6.21861) rectangle (2.1592,6.32446);
\draw [color=c, fill=c] (2.1592,6.21861) rectangle (2.19901,6.32446);
\draw [color=c, fill=c] (2.19901,6.21861) rectangle (2.23881,6.32446);
\draw [color=c, fill=c] (2.23881,6.21861) rectangle (2.27861,6.32446);
\draw [color=c, fill=c] (2.27861,6.21861) rectangle (2.31841,6.32446);
\draw [color=c, fill=c] (2.31841,6.21861) rectangle (2.35821,6.32446);
\draw [color=c, fill=c] (2.35821,6.21861) rectangle (2.39801,6.32446);
\draw [color=c, fill=c] (2.39801,6.21861) rectangle (2.43781,6.32446);
\draw [color=c, fill=c] (2.43781,6.21861) rectangle (2.47761,6.32446);
\draw [color=c, fill=c] (2.47761,6.21861) rectangle (2.51741,6.32446);
\draw [color=c, fill=c] (2.51741,6.21861) rectangle (2.55721,6.32446);
\draw [color=c, fill=c] (2.55721,6.21861) rectangle (2.59702,6.32446);
\draw [color=c, fill=c] (2.59702,6.21861) rectangle (2.63682,6.32446);
\draw [color=c, fill=c] (2.63682,6.21861) rectangle (2.67662,6.32446);
\draw [color=c, fill=c] (2.67662,6.21861) rectangle (2.71642,6.32446);
\draw [color=c, fill=c] (2.71642,6.21861) rectangle (2.75622,6.32446);
\draw [color=c, fill=c] (2.75622,6.21861) rectangle (2.79602,6.32446);
\draw [color=c, fill=c] (2.79602,6.21861) rectangle (2.83582,6.32446);
\draw [color=c, fill=c] (2.83582,6.21861) rectangle (2.87562,6.32446);
\draw [color=c, fill=c] (2.87562,6.21861) rectangle (2.91542,6.32446);
\draw [color=c, fill=c] (2.91542,6.21861) rectangle (2.95522,6.32446);
\draw [color=c, fill=c] (2.95522,6.21861) rectangle (2.99502,6.32446);
\draw [color=c, fill=c] (2.99502,6.21861) rectangle (3.03483,6.32446);
\draw [color=c, fill=c] (3.03483,6.21861) rectangle (3.07463,6.32446);
\draw [color=c, fill=c] (3.07463,6.21861) rectangle (3.11443,6.32446);
\draw [color=c, fill=c] (3.11443,6.21861) rectangle (3.15423,6.32446);
\draw [color=c, fill=c] (3.15423,6.21861) rectangle (3.19403,6.32446);
\draw [color=c, fill=c] (3.19403,6.21861) rectangle (3.23383,6.32446);
\draw [color=c, fill=c] (3.23383,6.21861) rectangle (3.27363,6.32446);
\draw [color=c, fill=c] (3.27363,6.21861) rectangle (3.31343,6.32446);
\draw [color=c, fill=c] (3.31343,6.21861) rectangle (3.35323,6.32446);
\draw [color=c, fill=c] (3.35323,6.21861) rectangle (3.39303,6.32446);
\draw [color=c, fill=c] (3.39303,6.21861) rectangle (3.43284,6.32446);
\draw [color=c, fill=c] (3.43284,6.21861) rectangle (3.47264,6.32446);
\draw [color=c, fill=c] (3.47264,6.21861) rectangle (3.51244,6.32446);
\draw [color=c, fill=c] (3.51244,6.21861) rectangle (3.55224,6.32446);
\draw [color=c, fill=c] (3.55224,6.21861) rectangle (3.59204,6.32446);
\draw [color=c, fill=c] (3.59204,6.21861) rectangle (3.63184,6.32446);
\draw [color=c, fill=c] (3.63184,6.21861) rectangle (3.67164,6.32446);
\draw [color=c, fill=c] (3.67164,6.21861) rectangle (3.71144,6.32446);
\draw [color=c, fill=c] (3.71144,6.21861) rectangle (3.75124,6.32446);
\draw [color=c, fill=c] (3.75124,6.21861) rectangle (3.79104,6.32446);
\draw [color=c, fill=c] (3.79104,6.21861) rectangle (3.83085,6.32446);
\draw [color=c, fill=c] (3.83085,6.21861) rectangle (3.87065,6.32446);
\draw [color=c, fill=c] (3.87065,6.21861) rectangle (3.91045,6.32446);
\draw [color=c, fill=c] (3.91045,6.21861) rectangle (3.95025,6.32446);
\draw [color=c, fill=c] (3.95025,6.21861) rectangle (3.99005,6.32446);
\draw [color=c, fill=c] (3.99005,6.21861) rectangle (4.02985,6.32446);
\draw [color=c, fill=c] (4.02985,6.21861) rectangle (4.06965,6.32446);
\draw [color=c, fill=c] (4.06965,6.21861) rectangle (4.10945,6.32446);
\draw [color=c, fill=c] (4.10945,6.21861) rectangle (4.14925,6.32446);
\draw [color=c, fill=c] (4.14925,6.21861) rectangle (4.18905,6.32446);
\draw [color=c, fill=c] (4.18905,6.21861) rectangle (4.22886,6.32446);
\draw [color=c, fill=c] (4.22886,6.21861) rectangle (4.26866,6.32446);
\draw [color=c, fill=c] (4.26866,6.21861) rectangle (4.30846,6.32446);
\draw [color=c, fill=c] (4.30846,6.21861) rectangle (4.34826,6.32446);
\draw [color=c, fill=c] (4.34826,6.21861) rectangle (4.38806,6.32446);
\draw [color=c, fill=c] (4.38806,6.21861) rectangle (4.42786,6.32446);
\draw [color=c, fill=c] (4.42786,6.21861) rectangle (4.46766,6.32446);
\draw [color=c, fill=c] (4.46766,6.21861) rectangle (4.50746,6.32446);
\draw [color=c, fill=c] (4.50746,6.21861) rectangle (4.54726,6.32446);
\draw [color=c, fill=c] (4.54726,6.21861) rectangle (4.58706,6.32446);
\draw [color=c, fill=c] (4.58706,6.21861) rectangle (4.62687,6.32446);
\draw [color=c, fill=c] (4.62687,6.21861) rectangle (4.66667,6.32446);
\draw [color=c, fill=c] (4.66667,6.21861) rectangle (4.70647,6.32446);
\draw [color=c, fill=c] (4.70647,6.21861) rectangle (4.74627,6.32446);
\draw [color=c, fill=c] (4.74627,6.21861) rectangle (4.78607,6.32446);
\draw [color=c, fill=c] (4.78607,6.21861) rectangle (4.82587,6.32446);
\draw [color=c, fill=c] (4.82587,6.21861) rectangle (4.86567,6.32446);
\draw [color=c, fill=c] (4.86567,6.21861) rectangle (4.90547,6.32446);
\draw [color=c, fill=c] (4.90547,6.21861) rectangle (4.94527,6.32446);
\draw [color=c, fill=c] (4.94527,6.21861) rectangle (4.98507,6.32446);
\draw [color=c, fill=c] (4.98507,6.21861) rectangle (5.02488,6.32446);
\draw [color=c, fill=c] (5.02488,6.21861) rectangle (5.06468,6.32446);
\draw [color=c, fill=c] (5.06468,6.21861) rectangle (5.10448,6.32446);
\draw [color=c, fill=c] (5.10448,6.21861) rectangle (5.14428,6.32446);
\draw [color=c, fill=c] (5.14428,6.21861) rectangle (5.18408,6.32446);
\definecolor{c}{rgb}{0.2,0,1};
\draw [color=c, fill=c] (5.18408,6.21861) rectangle (5.22388,6.32446);
\draw [color=c, fill=c] (5.22388,6.21861) rectangle (5.26368,6.32446);
\draw [color=c, fill=c] (5.26368,6.21861) rectangle (5.30348,6.32446);
\draw [color=c, fill=c] (5.30348,6.21861) rectangle (5.34328,6.32446);
\draw [color=c, fill=c] (5.34328,6.21861) rectangle (5.38308,6.32446);
\draw [color=c, fill=c] (5.38308,6.21861) rectangle (5.42289,6.32446);
\draw [color=c, fill=c] (5.42289,6.21861) rectangle (5.46269,6.32446);
\draw [color=c, fill=c] (5.46269,6.21861) rectangle (5.50249,6.32446);
\draw [color=c, fill=c] (5.50249,6.21861) rectangle (5.54229,6.32446);
\draw [color=c, fill=c] (5.54229,6.21861) rectangle (5.58209,6.32446);
\draw [color=c, fill=c] (5.58209,6.21861) rectangle (5.62189,6.32446);
\draw [color=c, fill=c] (5.62189,6.21861) rectangle (5.66169,6.32446);
\draw [color=c, fill=c] (5.66169,6.21861) rectangle (5.70149,6.32446);
\draw [color=c, fill=c] (5.70149,6.21861) rectangle (5.74129,6.32446);
\draw [color=c, fill=c] (5.74129,6.21861) rectangle (5.78109,6.32446);
\draw [color=c, fill=c] (5.78109,6.21861) rectangle (5.8209,6.32446);
\draw [color=c, fill=c] (5.8209,6.21861) rectangle (5.8607,6.32446);
\draw [color=c, fill=c] (5.8607,6.21861) rectangle (5.9005,6.32446);
\draw [color=c, fill=c] (5.9005,6.21861) rectangle (5.9403,6.32446);
\draw [color=c, fill=c] (5.9403,6.21861) rectangle (5.9801,6.32446);
\draw [color=c, fill=c] (5.9801,6.21861) rectangle (6.0199,6.32446);
\draw [color=c, fill=c] (6.0199,6.21861) rectangle (6.0597,6.32446);
\draw [color=c, fill=c] (6.0597,6.21861) rectangle (6.0995,6.32446);
\draw [color=c, fill=c] (6.0995,6.21861) rectangle (6.1393,6.32446);
\draw [color=c, fill=c] (6.1393,6.21861) rectangle (6.1791,6.32446);
\draw [color=c, fill=c] (6.1791,6.21861) rectangle (6.21891,6.32446);
\draw [color=c, fill=c] (6.21891,6.21861) rectangle (6.25871,6.32446);
\draw [color=c, fill=c] (6.25871,6.21861) rectangle (6.29851,6.32446);
\draw [color=c, fill=c] (6.29851,6.21861) rectangle (6.33831,6.32446);
\draw [color=c, fill=c] (6.33831,6.21861) rectangle (6.37811,6.32446);
\draw [color=c, fill=c] (6.37811,6.21861) rectangle (6.41791,6.32446);
\draw [color=c, fill=c] (6.41791,6.21861) rectangle (6.45771,6.32446);
\draw [color=c, fill=c] (6.45771,6.21861) rectangle (6.49751,6.32446);
\draw [color=c, fill=c] (6.49751,6.21861) rectangle (6.53731,6.32446);
\draw [color=c, fill=c] (6.53731,6.21861) rectangle (6.57711,6.32446);
\draw [color=c, fill=c] (6.57711,6.21861) rectangle (6.61692,6.32446);
\draw [color=c, fill=c] (6.61692,6.21861) rectangle (6.65672,6.32446);
\draw [color=c, fill=c] (6.65672,6.21861) rectangle (6.69652,6.32446);
\draw [color=c, fill=c] (6.69652,6.21861) rectangle (6.73632,6.32446);
\draw [color=c, fill=c] (6.73632,6.21861) rectangle (6.77612,6.32446);
\draw [color=c, fill=c] (6.77612,6.21861) rectangle (6.81592,6.32446);
\draw [color=c, fill=c] (6.81592,6.21861) rectangle (6.85572,6.32446);
\draw [color=c, fill=c] (6.85572,6.21861) rectangle (6.89552,6.32446);
\draw [color=c, fill=c] (6.89552,6.21861) rectangle (6.93532,6.32446);
\draw [color=c, fill=c] (6.93532,6.21861) rectangle (6.97512,6.32446);
\draw [color=c, fill=c] (6.97512,6.21861) rectangle (7.01493,6.32446);
\draw [color=c, fill=c] (7.01493,6.21861) rectangle (7.05473,6.32446);
\draw [color=c, fill=c] (7.05473,6.21861) rectangle (7.09453,6.32446);
\draw [color=c, fill=c] (7.09453,6.21861) rectangle (7.13433,6.32446);
\draw [color=c, fill=c] (7.13433,6.21861) rectangle (7.17413,6.32446);
\draw [color=c, fill=c] (7.17413,6.21861) rectangle (7.21393,6.32446);
\draw [color=c, fill=c] (7.21393,6.21861) rectangle (7.25373,6.32446);
\draw [color=c, fill=c] (7.25373,6.21861) rectangle (7.29353,6.32446);
\draw [color=c, fill=c] (7.29353,6.21861) rectangle (7.33333,6.32446);
\draw [color=c, fill=c] (7.33333,6.21861) rectangle (7.37313,6.32446);
\draw [color=c, fill=c] (7.37313,6.21861) rectangle (7.41294,6.32446);
\draw [color=c, fill=c] (7.41294,6.21861) rectangle (7.45274,6.32446);
\draw [color=c, fill=c] (7.45274,6.21861) rectangle (7.49254,6.32446);
\draw [color=c, fill=c] (7.49254,6.21861) rectangle (7.53234,6.32446);
\draw [color=c, fill=c] (7.53234,6.21861) rectangle (7.57214,6.32446);
\draw [color=c, fill=c] (7.57214,6.21861) rectangle (7.61194,6.32446);
\draw [color=c, fill=c] (7.61194,6.21861) rectangle (7.65174,6.32446);
\draw [color=c, fill=c] (7.65174,6.21861) rectangle (7.69154,6.32446);
\draw [color=c, fill=c] (7.69154,6.21861) rectangle (7.73134,6.32446);
\draw [color=c, fill=c] (7.73134,6.21861) rectangle (7.77114,6.32446);
\draw [color=c, fill=c] (7.77114,6.21861) rectangle (7.81095,6.32446);
\draw [color=c, fill=c] (7.81095,6.21861) rectangle (7.85075,6.32446);
\draw [color=c, fill=c] (7.85075,6.21861) rectangle (7.89055,6.32446);
\draw [color=c, fill=c] (7.89055,6.21861) rectangle (7.93035,6.32446);
\draw [color=c, fill=c] (7.93035,6.21861) rectangle (7.97015,6.32446);
\draw [color=c, fill=c] (7.97015,6.21861) rectangle (8.00995,6.32446);
\draw [color=c, fill=c] (8.00995,6.21861) rectangle (8.04975,6.32446);
\draw [color=c, fill=c] (8.04975,6.21861) rectangle (8.08955,6.32446);
\draw [color=c, fill=c] (8.08955,6.21861) rectangle (8.12935,6.32446);
\draw [color=c, fill=c] (8.12935,6.21861) rectangle (8.16915,6.32446);
\draw [color=c, fill=c] (8.16915,6.21861) rectangle (8.20895,6.32446);
\draw [color=c, fill=c] (8.20895,6.21861) rectangle (8.24876,6.32446);
\draw [color=c, fill=c] (8.24876,6.21861) rectangle (8.28856,6.32446);
\draw [color=c, fill=c] (8.28856,6.21861) rectangle (8.32836,6.32446);
\draw [color=c, fill=c] (8.32836,6.21861) rectangle (8.36816,6.32446);
\draw [color=c, fill=c] (8.36816,6.21861) rectangle (8.40796,6.32446);
\draw [color=c, fill=c] (8.40796,6.21861) rectangle (8.44776,6.32446);
\draw [color=c, fill=c] (8.44776,6.21861) rectangle (8.48756,6.32446);
\draw [color=c, fill=c] (8.48756,6.21861) rectangle (8.52736,6.32446);
\draw [color=c, fill=c] (8.52736,6.21861) rectangle (8.56716,6.32446);
\draw [color=c, fill=c] (8.56716,6.21861) rectangle (8.60697,6.32446);
\draw [color=c, fill=c] (8.60697,6.21861) rectangle (8.64677,6.32446);
\draw [color=c, fill=c] (8.64677,6.21861) rectangle (8.68657,6.32446);
\draw [color=c, fill=c] (8.68657,6.21861) rectangle (8.72637,6.32446);
\draw [color=c, fill=c] (8.72637,6.21861) rectangle (8.76617,6.32446);
\draw [color=c, fill=c] (8.76617,6.21861) rectangle (8.80597,6.32446);
\definecolor{c}{rgb}{0,0.0800001,1};
\draw [color=c, fill=c] (8.80597,6.21861) rectangle (8.84577,6.32446);
\draw [color=c, fill=c] (8.84577,6.21861) rectangle (8.88557,6.32446);
\draw [color=c, fill=c] (8.88557,6.21861) rectangle (8.92537,6.32446);
\draw [color=c, fill=c] (8.92537,6.21861) rectangle (8.96517,6.32446);
\draw [color=c, fill=c] (8.96517,6.21861) rectangle (9.00498,6.32446);
\draw [color=c, fill=c] (9.00498,6.21861) rectangle (9.04478,6.32446);
\draw [color=c, fill=c] (9.04478,6.21861) rectangle (9.08458,6.32446);
\draw [color=c, fill=c] (9.08458,6.21861) rectangle (9.12438,6.32446);
\draw [color=c, fill=c] (9.12438,6.21861) rectangle (9.16418,6.32446);
\draw [color=c, fill=c] (9.16418,6.21861) rectangle (9.20398,6.32446);
\draw [color=c, fill=c] (9.20398,6.21861) rectangle (9.24378,6.32446);
\draw [color=c, fill=c] (9.24378,6.21861) rectangle (9.28358,6.32446);
\draw [color=c, fill=c] (9.28358,6.21861) rectangle (9.32338,6.32446);
\draw [color=c, fill=c] (9.32338,6.21861) rectangle (9.36318,6.32446);
\draw [color=c, fill=c] (9.36318,6.21861) rectangle (9.40298,6.32446);
\draw [color=c, fill=c] (9.40298,6.21861) rectangle (9.44279,6.32446);
\draw [color=c, fill=c] (9.44279,6.21861) rectangle (9.48259,6.32446);
\draw [color=c, fill=c] (9.48259,6.21861) rectangle (9.52239,6.32446);
\draw [color=c, fill=c] (9.52239,6.21861) rectangle (9.56219,6.32446);
\draw [color=c, fill=c] (9.56219,6.21861) rectangle (9.60199,6.32446);
\draw [color=c, fill=c] (9.60199,6.21861) rectangle (9.64179,6.32446);
\draw [color=c, fill=c] (9.64179,6.21861) rectangle (9.68159,6.32446);
\definecolor{c}{rgb}{0,0.266667,1};
\draw [color=c, fill=c] (9.68159,6.21861) rectangle (9.72139,6.32446);
\draw [color=c, fill=c] (9.72139,6.21861) rectangle (9.76119,6.32446);
\draw [color=c, fill=c] (9.76119,6.21861) rectangle (9.80099,6.32446);
\draw [color=c, fill=c] (9.80099,6.21861) rectangle (9.8408,6.32446);
\draw [color=c, fill=c] (9.8408,6.21861) rectangle (9.8806,6.32446);
\draw [color=c, fill=c] (9.8806,6.21861) rectangle (9.9204,6.32446);
\draw [color=c, fill=c] (9.9204,6.21861) rectangle (9.9602,6.32446);
\draw [color=c, fill=c] (9.9602,6.21861) rectangle (10,6.32446);
\draw [color=c, fill=c] (10,6.21861) rectangle (10.0398,6.32446);
\draw [color=c, fill=c] (10.0398,6.21861) rectangle (10.0796,6.32446);
\draw [color=c, fill=c] (10.0796,6.21861) rectangle (10.1194,6.32446);
\draw [color=c, fill=c] (10.1194,6.21861) rectangle (10.1592,6.32446);
\definecolor{c}{rgb}{0,0.546666,1};
\draw [color=c, fill=c] (10.1592,6.21861) rectangle (10.199,6.32446);
\draw [color=c, fill=c] (10.199,6.21861) rectangle (10.2388,6.32446);
\draw [color=c, fill=c] (10.2388,6.21861) rectangle (10.2786,6.32446);
\draw [color=c, fill=c] (10.2786,6.21861) rectangle (10.3184,6.32446);
\draw [color=c, fill=c] (10.3184,6.21861) rectangle (10.3582,6.32446);
\draw [color=c, fill=c] (10.3582,6.21861) rectangle (10.398,6.32446);
\draw [color=c, fill=c] (10.398,6.21861) rectangle (10.4378,6.32446);
\draw [color=c, fill=c] (10.4378,6.21861) rectangle (10.4776,6.32446);
\draw [color=c, fill=c] (10.4776,6.21861) rectangle (10.5174,6.32446);
\draw [color=c, fill=c] (10.5174,6.21861) rectangle (10.5572,6.32446);
\draw [color=c, fill=c] (10.5572,6.21861) rectangle (10.597,6.32446);
\draw [color=c, fill=c] (10.597,6.21861) rectangle (10.6368,6.32446);
\draw [color=c, fill=c] (10.6368,6.21861) rectangle (10.6766,6.32446);
\draw [color=c, fill=c] (10.6766,6.21861) rectangle (10.7164,6.32446);
\draw [color=c, fill=c] (10.7164,6.21861) rectangle (10.7562,6.32446);
\draw [color=c, fill=c] (10.7562,6.21861) rectangle (10.796,6.32446);
\draw [color=c, fill=c] (10.796,6.21861) rectangle (10.8358,6.32446);
\definecolor{c}{rgb}{0,0.733333,1};
\draw [color=c, fill=c] (10.8358,6.21861) rectangle (10.8756,6.32446);
\draw [color=c, fill=c] (10.8756,6.21861) rectangle (10.9154,6.32446);
\draw [color=c, fill=c] (10.9154,6.21861) rectangle (10.9552,6.32446);
\draw [color=c, fill=c] (10.9552,6.21861) rectangle (10.995,6.32446);
\draw [color=c, fill=c] (10.995,6.21861) rectangle (11.0348,6.32446);
\draw [color=c, fill=c] (11.0348,6.21861) rectangle (11.0746,6.32446);
\draw [color=c, fill=c] (11.0746,6.21861) rectangle (11.1144,6.32446);
\draw [color=c, fill=c] (11.1144,6.21861) rectangle (11.1542,6.32446);
\draw [color=c, fill=c] (11.1542,6.21861) rectangle (11.194,6.32446);
\draw [color=c, fill=c] (11.194,6.21861) rectangle (11.2338,6.32446);
\draw [color=c, fill=c] (11.2338,6.21861) rectangle (11.2736,6.32446);
\draw [color=c, fill=c] (11.2736,6.21861) rectangle (11.3134,6.32446);
\draw [color=c, fill=c] (11.3134,6.21861) rectangle (11.3532,6.32446);
\draw [color=c, fill=c] (11.3532,6.21861) rectangle (11.393,6.32446);
\draw [color=c, fill=c] (11.393,6.21861) rectangle (11.4328,6.32446);
\draw [color=c, fill=c] (11.4328,6.21861) rectangle (11.4726,6.32446);
\draw [color=c, fill=c] (11.4726,6.21861) rectangle (11.5124,6.32446);
\draw [color=c, fill=c] (11.5124,6.21861) rectangle (11.5522,6.32446);
\draw [color=c, fill=c] (11.5522,6.21861) rectangle (11.592,6.32446);
\draw [color=c, fill=c] (11.592,6.21861) rectangle (11.6318,6.32446);
\draw [color=c, fill=c] (11.6318,6.21861) rectangle (11.6716,6.32446);
\draw [color=c, fill=c] (11.6716,6.21861) rectangle (11.7114,6.32446);
\draw [color=c, fill=c] (11.7114,6.21861) rectangle (11.7512,6.32446);
\draw [color=c, fill=c] (11.7512,6.21861) rectangle (11.791,6.32446);
\draw [color=c, fill=c] (11.791,6.21861) rectangle (11.8308,6.32446);
\draw [color=c, fill=c] (11.8308,6.21861) rectangle (11.8706,6.32446);
\draw [color=c, fill=c] (11.8706,6.21861) rectangle (11.9104,6.32446);
\draw [color=c, fill=c] (11.9104,6.21861) rectangle (11.9502,6.32446);
\draw [color=c, fill=c] (11.9502,6.21861) rectangle (11.99,6.32446);
\draw [color=c, fill=c] (11.99,6.21861) rectangle (12.0299,6.32446);
\draw [color=c, fill=c] (12.0299,6.21861) rectangle (12.0697,6.32446);
\draw [color=c, fill=c] (12.0697,6.21861) rectangle (12.1095,6.32446);
\draw [color=c, fill=c] (12.1095,6.21861) rectangle (12.1493,6.32446);
\draw [color=c, fill=c] (12.1493,6.21861) rectangle (12.1891,6.32446);
\draw [color=c, fill=c] (12.1891,6.21861) rectangle (12.2289,6.32446);
\draw [color=c, fill=c] (12.2289,6.21861) rectangle (12.2687,6.32446);
\draw [color=c, fill=c] (12.2687,6.21861) rectangle (12.3085,6.32446);
\draw [color=c, fill=c] (12.3085,6.21861) rectangle (12.3483,6.32446);
\draw [color=c, fill=c] (12.3483,6.21861) rectangle (12.3881,6.32446);
\draw [color=c, fill=c] (12.3881,6.21861) rectangle (12.4279,6.32446);
\draw [color=c, fill=c] (12.4279,6.21861) rectangle (12.4677,6.32446);
\draw [color=c, fill=c] (12.4677,6.21861) rectangle (12.5075,6.32446);
\draw [color=c, fill=c] (12.5075,6.21861) rectangle (12.5473,6.32446);
\draw [color=c, fill=c] (12.5473,6.21861) rectangle (12.5871,6.32446);
\draw [color=c, fill=c] (12.5871,6.21861) rectangle (12.6269,6.32446);
\draw [color=c, fill=c] (12.6269,6.21861) rectangle (12.6667,6.32446);
\draw [color=c, fill=c] (12.6667,6.21861) rectangle (12.7065,6.32446);
\draw [color=c, fill=c] (12.7065,6.21861) rectangle (12.7463,6.32446);
\draw [color=c, fill=c] (12.7463,6.21861) rectangle (12.7861,6.32446);
\draw [color=c, fill=c] (12.7861,6.21861) rectangle (12.8259,6.32446);
\draw [color=c, fill=c] (12.8259,6.21861) rectangle (12.8657,6.32446);
\draw [color=c, fill=c] (12.8657,6.21861) rectangle (12.9055,6.32446);
\draw [color=c, fill=c] (12.9055,6.21861) rectangle (12.9453,6.32446);
\draw [color=c, fill=c] (12.9453,6.21861) rectangle (12.9851,6.32446);
\draw [color=c, fill=c] (12.9851,6.21861) rectangle (13.0249,6.32446);
\draw [color=c, fill=c] (13.0249,6.21861) rectangle (13.0647,6.32446);
\draw [color=c, fill=c] (13.0647,6.21861) rectangle (13.1045,6.32446);
\draw [color=c, fill=c] (13.1045,6.21861) rectangle (13.1443,6.32446);
\draw [color=c, fill=c] (13.1443,6.21861) rectangle (13.1841,6.32446);
\draw [color=c, fill=c] (13.1841,6.21861) rectangle (13.2239,6.32446);
\draw [color=c, fill=c] (13.2239,6.21861) rectangle (13.2637,6.32446);
\draw [color=c, fill=c] (13.2637,6.21861) rectangle (13.3035,6.32446);
\draw [color=c, fill=c] (13.3035,6.21861) rectangle (13.3433,6.32446);
\draw [color=c, fill=c] (13.3433,6.21861) rectangle (13.3831,6.32446);
\draw [color=c, fill=c] (13.3831,6.21861) rectangle (13.4229,6.32446);
\draw [color=c, fill=c] (13.4229,6.21861) rectangle (13.4627,6.32446);
\draw [color=c, fill=c] (13.4627,6.21861) rectangle (13.5025,6.32446);
\draw [color=c, fill=c] (13.5025,6.21861) rectangle (13.5423,6.32446);
\draw [color=c, fill=c] (13.5423,6.21861) rectangle (13.5821,6.32446);
\draw [color=c, fill=c] (13.5821,6.21861) rectangle (13.6219,6.32446);
\draw [color=c, fill=c] (13.6219,6.21861) rectangle (13.6617,6.32446);
\draw [color=c, fill=c] (13.6617,6.21861) rectangle (13.7015,6.32446);
\draw [color=c, fill=c] (13.7015,6.21861) rectangle (13.7413,6.32446);
\draw [color=c, fill=c] (13.7413,6.21861) rectangle (13.7811,6.32446);
\draw [color=c, fill=c] (13.7811,6.21861) rectangle (13.8209,6.32446);
\draw [color=c, fill=c] (13.8209,6.21861) rectangle (13.8607,6.32446);
\draw [color=c, fill=c] (13.8607,6.21861) rectangle (13.9005,6.32446);
\draw [color=c, fill=c] (13.9005,6.21861) rectangle (13.9403,6.32446);
\draw [color=c, fill=c] (13.9403,6.21861) rectangle (13.9801,6.32446);
\draw [color=c, fill=c] (13.9801,6.21861) rectangle (14.0199,6.32446);
\draw [color=c, fill=c] (14.0199,6.21861) rectangle (14.0597,6.32446);
\draw [color=c, fill=c] (14.0597,6.21861) rectangle (14.0995,6.32446);
\draw [color=c, fill=c] (14.0995,6.21861) rectangle (14.1393,6.32446);
\draw [color=c, fill=c] (14.1393,6.21861) rectangle (14.1791,6.32446);
\draw [color=c, fill=c] (14.1791,6.21861) rectangle (14.2189,6.32446);
\draw [color=c, fill=c] (14.2189,6.21861) rectangle (14.2587,6.32446);
\draw [color=c, fill=c] (14.2587,6.21861) rectangle (14.2985,6.32446);
\draw [color=c, fill=c] (14.2985,6.21861) rectangle (14.3383,6.32446);
\draw [color=c, fill=c] (14.3383,6.21861) rectangle (14.3781,6.32446);
\draw [color=c, fill=c] (14.3781,6.21861) rectangle (14.4179,6.32446);
\draw [color=c, fill=c] (14.4179,6.21861) rectangle (14.4577,6.32446);
\draw [color=c, fill=c] (14.4577,6.21861) rectangle (14.4975,6.32446);
\draw [color=c, fill=c] (14.4975,6.21861) rectangle (14.5373,6.32446);
\draw [color=c, fill=c] (14.5373,6.21861) rectangle (14.5771,6.32446);
\draw [color=c, fill=c] (14.5771,6.21861) rectangle (14.6169,6.32446);
\draw [color=c, fill=c] (14.6169,6.21861) rectangle (14.6567,6.32446);
\draw [color=c, fill=c] (14.6567,6.21861) rectangle (14.6965,6.32446);
\draw [color=c, fill=c] (14.6965,6.21861) rectangle (14.7363,6.32446);
\draw [color=c, fill=c] (14.7363,6.21861) rectangle (14.7761,6.32446);
\draw [color=c, fill=c] (14.7761,6.21861) rectangle (14.8159,6.32446);
\draw [color=c, fill=c] (14.8159,6.21861) rectangle (14.8557,6.32446);
\draw [color=c, fill=c] (14.8557,6.21861) rectangle (14.8955,6.32446);
\draw [color=c, fill=c] (14.8955,6.21861) rectangle (14.9353,6.32446);
\draw [color=c, fill=c] (14.9353,6.21861) rectangle (14.9751,6.32446);
\draw [color=c, fill=c] (14.9751,6.21861) rectangle (15.0149,6.32446);
\draw [color=c, fill=c] (15.0149,6.21861) rectangle (15.0547,6.32446);
\draw [color=c, fill=c] (15.0547,6.21861) rectangle (15.0945,6.32446);
\draw [color=c, fill=c] (15.0945,6.21861) rectangle (15.1343,6.32446);
\draw [color=c, fill=c] (15.1343,6.21861) rectangle (15.1741,6.32446);
\draw [color=c, fill=c] (15.1741,6.21861) rectangle (15.2139,6.32446);
\draw [color=c, fill=c] (15.2139,6.21861) rectangle (15.2537,6.32446);
\draw [color=c, fill=c] (15.2537,6.21861) rectangle (15.2935,6.32446);
\draw [color=c, fill=c] (15.2935,6.21861) rectangle (15.3333,6.32446);
\draw [color=c, fill=c] (15.3333,6.21861) rectangle (15.3731,6.32446);
\draw [color=c, fill=c] (15.3731,6.21861) rectangle (15.4129,6.32446);
\draw [color=c, fill=c] (15.4129,6.21861) rectangle (15.4527,6.32446);
\draw [color=c, fill=c] (15.4527,6.21861) rectangle (15.4925,6.32446);
\draw [color=c, fill=c] (15.4925,6.21861) rectangle (15.5323,6.32446);
\draw [color=c, fill=c] (15.5323,6.21861) rectangle (15.5721,6.32446);
\draw [color=c, fill=c] (15.5721,6.21861) rectangle (15.6119,6.32446);
\draw [color=c, fill=c] (15.6119,6.21861) rectangle (15.6517,6.32446);
\draw [color=c, fill=c] (15.6517,6.21861) rectangle (15.6915,6.32446);
\draw [color=c, fill=c] (15.6915,6.21861) rectangle (15.7313,6.32446);
\draw [color=c, fill=c] (15.7313,6.21861) rectangle (15.7711,6.32446);
\draw [color=c, fill=c] (15.7711,6.21861) rectangle (15.8109,6.32446);
\draw [color=c, fill=c] (15.8109,6.21861) rectangle (15.8507,6.32446);
\draw [color=c, fill=c] (15.8507,6.21861) rectangle (15.8905,6.32446);
\draw [color=c, fill=c] (15.8905,6.21861) rectangle (15.9303,6.32446);
\draw [color=c, fill=c] (15.9303,6.21861) rectangle (15.9701,6.32446);
\draw [color=c, fill=c] (15.9701,6.21861) rectangle (16.01,6.32446);
\draw [color=c, fill=c] (16.01,6.21861) rectangle (16.0498,6.32446);
\draw [color=c, fill=c] (16.0498,6.21861) rectangle (16.0896,6.32446);
\draw [color=c, fill=c] (16.0896,6.21861) rectangle (16.1294,6.32446);
\draw [color=c, fill=c] (16.1294,6.21861) rectangle (16.1692,6.32446);
\draw [color=c, fill=c] (16.1692,6.21861) rectangle (16.209,6.32446);
\draw [color=c, fill=c] (16.209,6.21861) rectangle (16.2488,6.32446);
\draw [color=c, fill=c] (16.2488,6.21861) rectangle (16.2886,6.32446);
\draw [color=c, fill=c] (16.2886,6.21861) rectangle (16.3284,6.32446);
\draw [color=c, fill=c] (16.3284,6.21861) rectangle (16.3682,6.32446);
\draw [color=c, fill=c] (16.3682,6.21861) rectangle (16.408,6.32446);
\draw [color=c, fill=c] (16.408,6.21861) rectangle (16.4478,6.32446);
\draw [color=c, fill=c] (16.4478,6.21861) rectangle (16.4876,6.32446);
\draw [color=c, fill=c] (16.4876,6.21861) rectangle (16.5274,6.32446);
\draw [color=c, fill=c] (16.5274,6.21861) rectangle (16.5672,6.32446);
\draw [color=c, fill=c] (16.5672,6.21861) rectangle (16.607,6.32446);
\draw [color=c, fill=c] (16.607,6.21861) rectangle (16.6468,6.32446);
\draw [color=c, fill=c] (16.6468,6.21861) rectangle (16.6866,6.32446);
\draw [color=c, fill=c] (16.6866,6.21861) rectangle (16.7264,6.32446);
\draw [color=c, fill=c] (16.7264,6.21861) rectangle (16.7662,6.32446);
\draw [color=c, fill=c] (16.7662,6.21861) rectangle (16.806,6.32446);
\draw [color=c, fill=c] (16.806,6.21861) rectangle (16.8458,6.32446);
\draw [color=c, fill=c] (16.8458,6.21861) rectangle (16.8856,6.32446);
\draw [color=c, fill=c] (16.8856,6.21861) rectangle (16.9254,6.32446);
\draw [color=c, fill=c] (16.9254,6.21861) rectangle (16.9652,6.32446);
\draw [color=c, fill=c] (16.9652,6.21861) rectangle (17.005,6.32446);
\draw [color=c, fill=c] (17.005,6.21861) rectangle (17.0448,6.32446);
\draw [color=c, fill=c] (17.0448,6.21861) rectangle (17.0846,6.32446);
\draw [color=c, fill=c] (17.0846,6.21861) rectangle (17.1244,6.32446);
\draw [color=c, fill=c] (17.1244,6.21861) rectangle (17.1642,6.32446);
\draw [color=c, fill=c] (17.1642,6.21861) rectangle (17.204,6.32446);
\draw [color=c, fill=c] (17.204,6.21861) rectangle (17.2438,6.32446);
\draw [color=c, fill=c] (17.2438,6.21861) rectangle (17.2836,6.32446);
\draw [color=c, fill=c] (17.2836,6.21861) rectangle (17.3234,6.32446);
\draw [color=c, fill=c] (17.3234,6.21861) rectangle (17.3632,6.32446);
\draw [color=c, fill=c] (17.3632,6.21861) rectangle (17.403,6.32446);
\draw [color=c, fill=c] (17.403,6.21861) rectangle (17.4428,6.32446);
\draw [color=c, fill=c] (17.4428,6.21861) rectangle (17.4826,6.32446);
\draw [color=c, fill=c] (17.4826,6.21861) rectangle (17.5224,6.32446);
\draw [color=c, fill=c] (17.5224,6.21861) rectangle (17.5622,6.32446);
\draw [color=c, fill=c] (17.5622,6.21861) rectangle (17.602,6.32446);
\draw [color=c, fill=c] (17.602,6.21861) rectangle (17.6418,6.32446);
\draw [color=c, fill=c] (17.6418,6.21861) rectangle (17.6816,6.32446);
\draw [color=c, fill=c] (17.6816,6.21861) rectangle (17.7214,6.32446);
\draw [color=c, fill=c] (17.7214,6.21861) rectangle (17.7612,6.32446);
\draw [color=c, fill=c] (17.7612,6.21861) rectangle (17.801,6.32446);
\draw [color=c, fill=c] (17.801,6.21861) rectangle (17.8408,6.32446);
\draw [color=c, fill=c] (17.8408,6.21861) rectangle (17.8806,6.32446);
\draw [color=c, fill=c] (17.8806,6.21861) rectangle (17.9204,6.32446);
\draw [color=c, fill=c] (17.9204,6.21861) rectangle (17.9602,6.32446);
\draw [color=c, fill=c] (17.9602,6.21861) rectangle (18,6.32446);
\definecolor{c}{rgb}{0,0.0800001,1};
\draw [color=c, fill=c] (2,6.32446) rectangle (2.0398,6.43031);
\draw [color=c, fill=c] (2.0398,6.32446) rectangle (2.0796,6.43031);
\draw [color=c, fill=c] (2.0796,6.32446) rectangle (2.1194,6.43031);
\draw [color=c, fill=c] (2.1194,6.32446) rectangle (2.1592,6.43031);
\draw [color=c, fill=c] (2.1592,6.32446) rectangle (2.19901,6.43031);
\draw [color=c, fill=c] (2.19901,6.32446) rectangle (2.23881,6.43031);
\draw [color=c, fill=c] (2.23881,6.32446) rectangle (2.27861,6.43031);
\draw [color=c, fill=c] (2.27861,6.32446) rectangle (2.31841,6.43031);
\draw [color=c, fill=c] (2.31841,6.32446) rectangle (2.35821,6.43031);
\draw [color=c, fill=c] (2.35821,6.32446) rectangle (2.39801,6.43031);
\draw [color=c, fill=c] (2.39801,6.32446) rectangle (2.43781,6.43031);
\draw [color=c, fill=c] (2.43781,6.32446) rectangle (2.47761,6.43031);
\draw [color=c, fill=c] (2.47761,6.32446) rectangle (2.51741,6.43031);
\draw [color=c, fill=c] (2.51741,6.32446) rectangle (2.55721,6.43031);
\draw [color=c, fill=c] (2.55721,6.32446) rectangle (2.59702,6.43031);
\draw [color=c, fill=c] (2.59702,6.32446) rectangle (2.63682,6.43031);
\draw [color=c, fill=c] (2.63682,6.32446) rectangle (2.67662,6.43031);
\draw [color=c, fill=c] (2.67662,6.32446) rectangle (2.71642,6.43031);
\draw [color=c, fill=c] (2.71642,6.32446) rectangle (2.75622,6.43031);
\draw [color=c, fill=c] (2.75622,6.32446) rectangle (2.79602,6.43031);
\draw [color=c, fill=c] (2.79602,6.32446) rectangle (2.83582,6.43031);
\draw [color=c, fill=c] (2.83582,6.32446) rectangle (2.87562,6.43031);
\draw [color=c, fill=c] (2.87562,6.32446) rectangle (2.91542,6.43031);
\draw [color=c, fill=c] (2.91542,6.32446) rectangle (2.95522,6.43031);
\draw [color=c, fill=c] (2.95522,6.32446) rectangle (2.99502,6.43031);
\draw [color=c, fill=c] (2.99502,6.32446) rectangle (3.03483,6.43031);
\draw [color=c, fill=c] (3.03483,6.32446) rectangle (3.07463,6.43031);
\draw [color=c, fill=c] (3.07463,6.32446) rectangle (3.11443,6.43031);
\draw [color=c, fill=c] (3.11443,6.32446) rectangle (3.15423,6.43031);
\draw [color=c, fill=c] (3.15423,6.32446) rectangle (3.19403,6.43031);
\draw [color=c, fill=c] (3.19403,6.32446) rectangle (3.23383,6.43031);
\draw [color=c, fill=c] (3.23383,6.32446) rectangle (3.27363,6.43031);
\draw [color=c, fill=c] (3.27363,6.32446) rectangle (3.31343,6.43031);
\draw [color=c, fill=c] (3.31343,6.32446) rectangle (3.35323,6.43031);
\draw [color=c, fill=c] (3.35323,6.32446) rectangle (3.39303,6.43031);
\draw [color=c, fill=c] (3.39303,6.32446) rectangle (3.43284,6.43031);
\draw [color=c, fill=c] (3.43284,6.32446) rectangle (3.47264,6.43031);
\draw [color=c, fill=c] (3.47264,6.32446) rectangle (3.51244,6.43031);
\draw [color=c, fill=c] (3.51244,6.32446) rectangle (3.55224,6.43031);
\draw [color=c, fill=c] (3.55224,6.32446) rectangle (3.59204,6.43031);
\draw [color=c, fill=c] (3.59204,6.32446) rectangle (3.63184,6.43031);
\draw [color=c, fill=c] (3.63184,6.32446) rectangle (3.67164,6.43031);
\draw [color=c, fill=c] (3.67164,6.32446) rectangle (3.71144,6.43031);
\draw [color=c, fill=c] (3.71144,6.32446) rectangle (3.75124,6.43031);
\draw [color=c, fill=c] (3.75124,6.32446) rectangle (3.79104,6.43031);
\draw [color=c, fill=c] (3.79104,6.32446) rectangle (3.83085,6.43031);
\draw [color=c, fill=c] (3.83085,6.32446) rectangle (3.87065,6.43031);
\draw [color=c, fill=c] (3.87065,6.32446) rectangle (3.91045,6.43031);
\draw [color=c, fill=c] (3.91045,6.32446) rectangle (3.95025,6.43031);
\draw [color=c, fill=c] (3.95025,6.32446) rectangle (3.99005,6.43031);
\draw [color=c, fill=c] (3.99005,6.32446) rectangle (4.02985,6.43031);
\draw [color=c, fill=c] (4.02985,6.32446) rectangle (4.06965,6.43031);
\draw [color=c, fill=c] (4.06965,6.32446) rectangle (4.10945,6.43031);
\draw [color=c, fill=c] (4.10945,6.32446) rectangle (4.14925,6.43031);
\draw [color=c, fill=c] (4.14925,6.32446) rectangle (4.18905,6.43031);
\draw [color=c, fill=c] (4.18905,6.32446) rectangle (4.22886,6.43031);
\draw [color=c, fill=c] (4.22886,6.32446) rectangle (4.26866,6.43031);
\draw [color=c, fill=c] (4.26866,6.32446) rectangle (4.30846,6.43031);
\draw [color=c, fill=c] (4.30846,6.32446) rectangle (4.34826,6.43031);
\draw [color=c, fill=c] (4.34826,6.32446) rectangle (4.38806,6.43031);
\draw [color=c, fill=c] (4.38806,6.32446) rectangle (4.42786,6.43031);
\draw [color=c, fill=c] (4.42786,6.32446) rectangle (4.46766,6.43031);
\draw [color=c, fill=c] (4.46766,6.32446) rectangle (4.50746,6.43031);
\draw [color=c, fill=c] (4.50746,6.32446) rectangle (4.54726,6.43031);
\draw [color=c, fill=c] (4.54726,6.32446) rectangle (4.58706,6.43031);
\draw [color=c, fill=c] (4.58706,6.32446) rectangle (4.62687,6.43031);
\draw [color=c, fill=c] (4.62687,6.32446) rectangle (4.66667,6.43031);
\draw [color=c, fill=c] (4.66667,6.32446) rectangle (4.70647,6.43031);
\draw [color=c, fill=c] (4.70647,6.32446) rectangle (4.74627,6.43031);
\draw [color=c, fill=c] (4.74627,6.32446) rectangle (4.78607,6.43031);
\draw [color=c, fill=c] (4.78607,6.32446) rectangle (4.82587,6.43031);
\draw [color=c, fill=c] (4.82587,6.32446) rectangle (4.86567,6.43031);
\draw [color=c, fill=c] (4.86567,6.32446) rectangle (4.90547,6.43031);
\draw [color=c, fill=c] (4.90547,6.32446) rectangle (4.94527,6.43031);
\draw [color=c, fill=c] (4.94527,6.32446) rectangle (4.98507,6.43031);
\draw [color=c, fill=c] (4.98507,6.32446) rectangle (5.02488,6.43031);
\definecolor{c}{rgb}{0.2,0,1};
\draw [color=c, fill=c] (5.02488,6.32446) rectangle (5.06468,6.43031);
\draw [color=c, fill=c] (5.06468,6.32446) rectangle (5.10448,6.43031);
\draw [color=c, fill=c] (5.10448,6.32446) rectangle (5.14428,6.43031);
\draw [color=c, fill=c] (5.14428,6.32446) rectangle (5.18408,6.43031);
\draw [color=c, fill=c] (5.18408,6.32446) rectangle (5.22388,6.43031);
\draw [color=c, fill=c] (5.22388,6.32446) rectangle (5.26368,6.43031);
\draw [color=c, fill=c] (5.26368,6.32446) rectangle (5.30348,6.43031);
\draw [color=c, fill=c] (5.30348,6.32446) rectangle (5.34328,6.43031);
\draw [color=c, fill=c] (5.34328,6.32446) rectangle (5.38308,6.43031);
\draw [color=c, fill=c] (5.38308,6.32446) rectangle (5.42289,6.43031);
\draw [color=c, fill=c] (5.42289,6.32446) rectangle (5.46269,6.43031);
\draw [color=c, fill=c] (5.46269,6.32446) rectangle (5.50249,6.43031);
\draw [color=c, fill=c] (5.50249,6.32446) rectangle (5.54229,6.43031);
\draw [color=c, fill=c] (5.54229,6.32446) rectangle (5.58209,6.43031);
\draw [color=c, fill=c] (5.58209,6.32446) rectangle (5.62189,6.43031);
\draw [color=c, fill=c] (5.62189,6.32446) rectangle (5.66169,6.43031);
\draw [color=c, fill=c] (5.66169,6.32446) rectangle (5.70149,6.43031);
\draw [color=c, fill=c] (5.70149,6.32446) rectangle (5.74129,6.43031);
\draw [color=c, fill=c] (5.74129,6.32446) rectangle (5.78109,6.43031);
\draw [color=c, fill=c] (5.78109,6.32446) rectangle (5.8209,6.43031);
\draw [color=c, fill=c] (5.8209,6.32446) rectangle (5.8607,6.43031);
\draw [color=c, fill=c] (5.8607,6.32446) rectangle (5.9005,6.43031);
\draw [color=c, fill=c] (5.9005,6.32446) rectangle (5.9403,6.43031);
\draw [color=c, fill=c] (5.9403,6.32446) rectangle (5.9801,6.43031);
\draw [color=c, fill=c] (5.9801,6.32446) rectangle (6.0199,6.43031);
\draw [color=c, fill=c] (6.0199,6.32446) rectangle (6.0597,6.43031);
\draw [color=c, fill=c] (6.0597,6.32446) rectangle (6.0995,6.43031);
\draw [color=c, fill=c] (6.0995,6.32446) rectangle (6.1393,6.43031);
\draw [color=c, fill=c] (6.1393,6.32446) rectangle (6.1791,6.43031);
\draw [color=c, fill=c] (6.1791,6.32446) rectangle (6.21891,6.43031);
\draw [color=c, fill=c] (6.21891,6.32446) rectangle (6.25871,6.43031);
\draw [color=c, fill=c] (6.25871,6.32446) rectangle (6.29851,6.43031);
\draw [color=c, fill=c] (6.29851,6.32446) rectangle (6.33831,6.43031);
\draw [color=c, fill=c] (6.33831,6.32446) rectangle (6.37811,6.43031);
\draw [color=c, fill=c] (6.37811,6.32446) rectangle (6.41791,6.43031);
\draw [color=c, fill=c] (6.41791,6.32446) rectangle (6.45771,6.43031);
\draw [color=c, fill=c] (6.45771,6.32446) rectangle (6.49751,6.43031);
\draw [color=c, fill=c] (6.49751,6.32446) rectangle (6.53731,6.43031);
\draw [color=c, fill=c] (6.53731,6.32446) rectangle (6.57711,6.43031);
\draw [color=c, fill=c] (6.57711,6.32446) rectangle (6.61692,6.43031);
\draw [color=c, fill=c] (6.61692,6.32446) rectangle (6.65672,6.43031);
\draw [color=c, fill=c] (6.65672,6.32446) rectangle (6.69652,6.43031);
\draw [color=c, fill=c] (6.69652,6.32446) rectangle (6.73632,6.43031);
\draw [color=c, fill=c] (6.73632,6.32446) rectangle (6.77612,6.43031);
\draw [color=c, fill=c] (6.77612,6.32446) rectangle (6.81592,6.43031);
\draw [color=c, fill=c] (6.81592,6.32446) rectangle (6.85572,6.43031);
\draw [color=c, fill=c] (6.85572,6.32446) rectangle (6.89552,6.43031);
\draw [color=c, fill=c] (6.89552,6.32446) rectangle (6.93532,6.43031);
\draw [color=c, fill=c] (6.93532,6.32446) rectangle (6.97512,6.43031);
\draw [color=c, fill=c] (6.97512,6.32446) rectangle (7.01493,6.43031);
\draw [color=c, fill=c] (7.01493,6.32446) rectangle (7.05473,6.43031);
\draw [color=c, fill=c] (7.05473,6.32446) rectangle (7.09453,6.43031);
\draw [color=c, fill=c] (7.09453,6.32446) rectangle (7.13433,6.43031);
\draw [color=c, fill=c] (7.13433,6.32446) rectangle (7.17413,6.43031);
\draw [color=c, fill=c] (7.17413,6.32446) rectangle (7.21393,6.43031);
\draw [color=c, fill=c] (7.21393,6.32446) rectangle (7.25373,6.43031);
\draw [color=c, fill=c] (7.25373,6.32446) rectangle (7.29353,6.43031);
\draw [color=c, fill=c] (7.29353,6.32446) rectangle (7.33333,6.43031);
\draw [color=c, fill=c] (7.33333,6.32446) rectangle (7.37313,6.43031);
\draw [color=c, fill=c] (7.37313,6.32446) rectangle (7.41294,6.43031);
\draw [color=c, fill=c] (7.41294,6.32446) rectangle (7.45274,6.43031);
\draw [color=c, fill=c] (7.45274,6.32446) rectangle (7.49254,6.43031);
\draw [color=c, fill=c] (7.49254,6.32446) rectangle (7.53234,6.43031);
\draw [color=c, fill=c] (7.53234,6.32446) rectangle (7.57214,6.43031);
\draw [color=c, fill=c] (7.57214,6.32446) rectangle (7.61194,6.43031);
\draw [color=c, fill=c] (7.61194,6.32446) rectangle (7.65174,6.43031);
\draw [color=c, fill=c] (7.65174,6.32446) rectangle (7.69154,6.43031);
\draw [color=c, fill=c] (7.69154,6.32446) rectangle (7.73134,6.43031);
\draw [color=c, fill=c] (7.73134,6.32446) rectangle (7.77114,6.43031);
\draw [color=c, fill=c] (7.77114,6.32446) rectangle (7.81095,6.43031);
\draw [color=c, fill=c] (7.81095,6.32446) rectangle (7.85075,6.43031);
\draw [color=c, fill=c] (7.85075,6.32446) rectangle (7.89055,6.43031);
\draw [color=c, fill=c] (7.89055,6.32446) rectangle (7.93035,6.43031);
\draw [color=c, fill=c] (7.93035,6.32446) rectangle (7.97015,6.43031);
\draw [color=c, fill=c] (7.97015,6.32446) rectangle (8.00995,6.43031);
\draw [color=c, fill=c] (8.00995,6.32446) rectangle (8.04975,6.43031);
\draw [color=c, fill=c] (8.04975,6.32446) rectangle (8.08955,6.43031);
\draw [color=c, fill=c] (8.08955,6.32446) rectangle (8.12935,6.43031);
\draw [color=c, fill=c] (8.12935,6.32446) rectangle (8.16915,6.43031);
\draw [color=c, fill=c] (8.16915,6.32446) rectangle (8.20895,6.43031);
\draw [color=c, fill=c] (8.20895,6.32446) rectangle (8.24876,6.43031);
\draw [color=c, fill=c] (8.24876,6.32446) rectangle (8.28856,6.43031);
\draw [color=c, fill=c] (8.28856,6.32446) rectangle (8.32836,6.43031);
\draw [color=c, fill=c] (8.32836,6.32446) rectangle (8.36816,6.43031);
\draw [color=c, fill=c] (8.36816,6.32446) rectangle (8.40796,6.43031);
\draw [color=c, fill=c] (8.40796,6.32446) rectangle (8.44776,6.43031);
\draw [color=c, fill=c] (8.44776,6.32446) rectangle (8.48756,6.43031);
\draw [color=c, fill=c] (8.48756,6.32446) rectangle (8.52736,6.43031);
\draw [color=c, fill=c] (8.52736,6.32446) rectangle (8.56716,6.43031);
\draw [color=c, fill=c] (8.56716,6.32446) rectangle (8.60697,6.43031);
\draw [color=c, fill=c] (8.60697,6.32446) rectangle (8.64677,6.43031);
\draw [color=c, fill=c] (8.64677,6.32446) rectangle (8.68657,6.43031);
\draw [color=c, fill=c] (8.68657,6.32446) rectangle (8.72637,6.43031);
\definecolor{c}{rgb}{0,0.0800001,1};
\draw [color=c, fill=c] (8.72637,6.32446) rectangle (8.76617,6.43031);
\draw [color=c, fill=c] (8.76617,6.32446) rectangle (8.80597,6.43031);
\draw [color=c, fill=c] (8.80597,6.32446) rectangle (8.84577,6.43031);
\draw [color=c, fill=c] (8.84577,6.32446) rectangle (8.88557,6.43031);
\draw [color=c, fill=c] (8.88557,6.32446) rectangle (8.92537,6.43031);
\draw [color=c, fill=c] (8.92537,6.32446) rectangle (8.96517,6.43031);
\draw [color=c, fill=c] (8.96517,6.32446) rectangle (9.00498,6.43031);
\draw [color=c, fill=c] (9.00498,6.32446) rectangle (9.04478,6.43031);
\draw [color=c, fill=c] (9.04478,6.32446) rectangle (9.08458,6.43031);
\draw [color=c, fill=c] (9.08458,6.32446) rectangle (9.12438,6.43031);
\draw [color=c, fill=c] (9.12438,6.32446) rectangle (9.16418,6.43031);
\draw [color=c, fill=c] (9.16418,6.32446) rectangle (9.20398,6.43031);
\draw [color=c, fill=c] (9.20398,6.32446) rectangle (9.24378,6.43031);
\draw [color=c, fill=c] (9.24378,6.32446) rectangle (9.28358,6.43031);
\draw [color=c, fill=c] (9.28358,6.32446) rectangle (9.32338,6.43031);
\draw [color=c, fill=c] (9.32338,6.32446) rectangle (9.36318,6.43031);
\draw [color=c, fill=c] (9.36318,6.32446) rectangle (9.40298,6.43031);
\draw [color=c, fill=c] (9.40298,6.32446) rectangle (9.44279,6.43031);
\draw [color=c, fill=c] (9.44279,6.32446) rectangle (9.48259,6.43031);
\draw [color=c, fill=c] (9.48259,6.32446) rectangle (9.52239,6.43031);
\draw [color=c, fill=c] (9.52239,6.32446) rectangle (9.56219,6.43031);
\draw [color=c, fill=c] (9.56219,6.32446) rectangle (9.60199,6.43031);
\draw [color=c, fill=c] (9.60199,6.32446) rectangle (9.64179,6.43031);
\draw [color=c, fill=c] (9.64179,6.32446) rectangle (9.68159,6.43031);
\definecolor{c}{rgb}{0,0.266667,1};
\draw [color=c, fill=c] (9.68159,6.32446) rectangle (9.72139,6.43031);
\draw [color=c, fill=c] (9.72139,6.32446) rectangle (9.76119,6.43031);
\draw [color=c, fill=c] (9.76119,6.32446) rectangle (9.80099,6.43031);
\draw [color=c, fill=c] (9.80099,6.32446) rectangle (9.8408,6.43031);
\draw [color=c, fill=c] (9.8408,6.32446) rectangle (9.8806,6.43031);
\draw [color=c, fill=c] (9.8806,6.32446) rectangle (9.9204,6.43031);
\draw [color=c, fill=c] (9.9204,6.32446) rectangle (9.9602,6.43031);
\draw [color=c, fill=c] (9.9602,6.32446) rectangle (10,6.43031);
\draw [color=c, fill=c] (10,6.32446) rectangle (10.0398,6.43031);
\draw [color=c, fill=c] (10.0398,6.32446) rectangle (10.0796,6.43031);
\draw [color=c, fill=c] (10.0796,6.32446) rectangle (10.1194,6.43031);
\draw [color=c, fill=c] (10.1194,6.32446) rectangle (10.1592,6.43031);
\definecolor{c}{rgb}{0,0.546666,1};
\draw [color=c, fill=c] (10.1592,6.32446) rectangle (10.199,6.43031);
\draw [color=c, fill=c] (10.199,6.32446) rectangle (10.2388,6.43031);
\draw [color=c, fill=c] (10.2388,6.32446) rectangle (10.2786,6.43031);
\draw [color=c, fill=c] (10.2786,6.32446) rectangle (10.3184,6.43031);
\draw [color=c, fill=c] (10.3184,6.32446) rectangle (10.3582,6.43031);
\draw [color=c, fill=c] (10.3582,6.32446) rectangle (10.398,6.43031);
\draw [color=c, fill=c] (10.398,6.32446) rectangle (10.4378,6.43031);
\draw [color=c, fill=c] (10.4378,6.32446) rectangle (10.4776,6.43031);
\draw [color=c, fill=c] (10.4776,6.32446) rectangle (10.5174,6.43031);
\draw [color=c, fill=c] (10.5174,6.32446) rectangle (10.5572,6.43031);
\draw [color=c, fill=c] (10.5572,6.32446) rectangle (10.597,6.43031);
\draw [color=c, fill=c] (10.597,6.32446) rectangle (10.6368,6.43031);
\draw [color=c, fill=c] (10.6368,6.32446) rectangle (10.6766,6.43031);
\draw [color=c, fill=c] (10.6766,6.32446) rectangle (10.7164,6.43031);
\draw [color=c, fill=c] (10.7164,6.32446) rectangle (10.7562,6.43031);
\draw [color=c, fill=c] (10.7562,6.32446) rectangle (10.796,6.43031);
\draw [color=c, fill=c] (10.796,6.32446) rectangle (10.8358,6.43031);
\draw [color=c, fill=c] (10.8358,6.32446) rectangle (10.8756,6.43031);
\definecolor{c}{rgb}{0,0.733333,1};
\draw [color=c, fill=c] (10.8756,6.32446) rectangle (10.9154,6.43031);
\draw [color=c, fill=c] (10.9154,6.32446) rectangle (10.9552,6.43031);
\draw [color=c, fill=c] (10.9552,6.32446) rectangle (10.995,6.43031);
\draw [color=c, fill=c] (10.995,6.32446) rectangle (11.0348,6.43031);
\draw [color=c, fill=c] (11.0348,6.32446) rectangle (11.0746,6.43031);
\draw [color=c, fill=c] (11.0746,6.32446) rectangle (11.1144,6.43031);
\draw [color=c, fill=c] (11.1144,6.32446) rectangle (11.1542,6.43031);
\draw [color=c, fill=c] (11.1542,6.32446) rectangle (11.194,6.43031);
\draw [color=c, fill=c] (11.194,6.32446) rectangle (11.2338,6.43031);
\draw [color=c, fill=c] (11.2338,6.32446) rectangle (11.2736,6.43031);
\draw [color=c, fill=c] (11.2736,6.32446) rectangle (11.3134,6.43031);
\draw [color=c, fill=c] (11.3134,6.32446) rectangle (11.3532,6.43031);
\draw [color=c, fill=c] (11.3532,6.32446) rectangle (11.393,6.43031);
\draw [color=c, fill=c] (11.393,6.32446) rectangle (11.4328,6.43031);
\draw [color=c, fill=c] (11.4328,6.32446) rectangle (11.4726,6.43031);
\draw [color=c, fill=c] (11.4726,6.32446) rectangle (11.5124,6.43031);
\draw [color=c, fill=c] (11.5124,6.32446) rectangle (11.5522,6.43031);
\draw [color=c, fill=c] (11.5522,6.32446) rectangle (11.592,6.43031);
\draw [color=c, fill=c] (11.592,6.32446) rectangle (11.6318,6.43031);
\draw [color=c, fill=c] (11.6318,6.32446) rectangle (11.6716,6.43031);
\draw [color=c, fill=c] (11.6716,6.32446) rectangle (11.7114,6.43031);
\draw [color=c, fill=c] (11.7114,6.32446) rectangle (11.7512,6.43031);
\draw [color=c, fill=c] (11.7512,6.32446) rectangle (11.791,6.43031);
\draw [color=c, fill=c] (11.791,6.32446) rectangle (11.8308,6.43031);
\draw [color=c, fill=c] (11.8308,6.32446) rectangle (11.8706,6.43031);
\draw [color=c, fill=c] (11.8706,6.32446) rectangle (11.9104,6.43031);
\draw [color=c, fill=c] (11.9104,6.32446) rectangle (11.9502,6.43031);
\draw [color=c, fill=c] (11.9502,6.32446) rectangle (11.99,6.43031);
\draw [color=c, fill=c] (11.99,6.32446) rectangle (12.0299,6.43031);
\draw [color=c, fill=c] (12.0299,6.32446) rectangle (12.0697,6.43031);
\draw [color=c, fill=c] (12.0697,6.32446) rectangle (12.1095,6.43031);
\draw [color=c, fill=c] (12.1095,6.32446) rectangle (12.1493,6.43031);
\draw [color=c, fill=c] (12.1493,6.32446) rectangle (12.1891,6.43031);
\draw [color=c, fill=c] (12.1891,6.32446) rectangle (12.2289,6.43031);
\draw [color=c, fill=c] (12.2289,6.32446) rectangle (12.2687,6.43031);
\draw [color=c, fill=c] (12.2687,6.32446) rectangle (12.3085,6.43031);
\draw [color=c, fill=c] (12.3085,6.32446) rectangle (12.3483,6.43031);
\draw [color=c, fill=c] (12.3483,6.32446) rectangle (12.3881,6.43031);
\draw [color=c, fill=c] (12.3881,6.32446) rectangle (12.4279,6.43031);
\draw [color=c, fill=c] (12.4279,6.32446) rectangle (12.4677,6.43031);
\draw [color=c, fill=c] (12.4677,6.32446) rectangle (12.5075,6.43031);
\draw [color=c, fill=c] (12.5075,6.32446) rectangle (12.5473,6.43031);
\draw [color=c, fill=c] (12.5473,6.32446) rectangle (12.5871,6.43031);
\draw [color=c, fill=c] (12.5871,6.32446) rectangle (12.6269,6.43031);
\draw [color=c, fill=c] (12.6269,6.32446) rectangle (12.6667,6.43031);
\draw [color=c, fill=c] (12.6667,6.32446) rectangle (12.7065,6.43031);
\draw [color=c, fill=c] (12.7065,6.32446) rectangle (12.7463,6.43031);
\draw [color=c, fill=c] (12.7463,6.32446) rectangle (12.7861,6.43031);
\draw [color=c, fill=c] (12.7861,6.32446) rectangle (12.8259,6.43031);
\draw [color=c, fill=c] (12.8259,6.32446) rectangle (12.8657,6.43031);
\draw [color=c, fill=c] (12.8657,6.32446) rectangle (12.9055,6.43031);
\draw [color=c, fill=c] (12.9055,6.32446) rectangle (12.9453,6.43031);
\draw [color=c, fill=c] (12.9453,6.32446) rectangle (12.9851,6.43031);
\draw [color=c, fill=c] (12.9851,6.32446) rectangle (13.0249,6.43031);
\draw [color=c, fill=c] (13.0249,6.32446) rectangle (13.0647,6.43031);
\draw [color=c, fill=c] (13.0647,6.32446) rectangle (13.1045,6.43031);
\draw [color=c, fill=c] (13.1045,6.32446) rectangle (13.1443,6.43031);
\draw [color=c, fill=c] (13.1443,6.32446) rectangle (13.1841,6.43031);
\draw [color=c, fill=c] (13.1841,6.32446) rectangle (13.2239,6.43031);
\draw [color=c, fill=c] (13.2239,6.32446) rectangle (13.2637,6.43031);
\draw [color=c, fill=c] (13.2637,6.32446) rectangle (13.3035,6.43031);
\draw [color=c, fill=c] (13.3035,6.32446) rectangle (13.3433,6.43031);
\draw [color=c, fill=c] (13.3433,6.32446) rectangle (13.3831,6.43031);
\draw [color=c, fill=c] (13.3831,6.32446) rectangle (13.4229,6.43031);
\draw [color=c, fill=c] (13.4229,6.32446) rectangle (13.4627,6.43031);
\draw [color=c, fill=c] (13.4627,6.32446) rectangle (13.5025,6.43031);
\draw [color=c, fill=c] (13.5025,6.32446) rectangle (13.5423,6.43031);
\draw [color=c, fill=c] (13.5423,6.32446) rectangle (13.5821,6.43031);
\draw [color=c, fill=c] (13.5821,6.32446) rectangle (13.6219,6.43031);
\draw [color=c, fill=c] (13.6219,6.32446) rectangle (13.6617,6.43031);
\draw [color=c, fill=c] (13.6617,6.32446) rectangle (13.7015,6.43031);
\draw [color=c, fill=c] (13.7015,6.32446) rectangle (13.7413,6.43031);
\draw [color=c, fill=c] (13.7413,6.32446) rectangle (13.7811,6.43031);
\draw [color=c, fill=c] (13.7811,6.32446) rectangle (13.8209,6.43031);
\draw [color=c, fill=c] (13.8209,6.32446) rectangle (13.8607,6.43031);
\draw [color=c, fill=c] (13.8607,6.32446) rectangle (13.9005,6.43031);
\draw [color=c, fill=c] (13.9005,6.32446) rectangle (13.9403,6.43031);
\draw [color=c, fill=c] (13.9403,6.32446) rectangle (13.9801,6.43031);
\draw [color=c, fill=c] (13.9801,6.32446) rectangle (14.0199,6.43031);
\draw [color=c, fill=c] (14.0199,6.32446) rectangle (14.0597,6.43031);
\draw [color=c, fill=c] (14.0597,6.32446) rectangle (14.0995,6.43031);
\draw [color=c, fill=c] (14.0995,6.32446) rectangle (14.1393,6.43031);
\draw [color=c, fill=c] (14.1393,6.32446) rectangle (14.1791,6.43031);
\draw [color=c, fill=c] (14.1791,6.32446) rectangle (14.2189,6.43031);
\draw [color=c, fill=c] (14.2189,6.32446) rectangle (14.2587,6.43031);
\draw [color=c, fill=c] (14.2587,6.32446) rectangle (14.2985,6.43031);
\draw [color=c, fill=c] (14.2985,6.32446) rectangle (14.3383,6.43031);
\draw [color=c, fill=c] (14.3383,6.32446) rectangle (14.3781,6.43031);
\draw [color=c, fill=c] (14.3781,6.32446) rectangle (14.4179,6.43031);
\draw [color=c, fill=c] (14.4179,6.32446) rectangle (14.4577,6.43031);
\draw [color=c, fill=c] (14.4577,6.32446) rectangle (14.4975,6.43031);
\draw [color=c, fill=c] (14.4975,6.32446) rectangle (14.5373,6.43031);
\draw [color=c, fill=c] (14.5373,6.32446) rectangle (14.5771,6.43031);
\draw [color=c, fill=c] (14.5771,6.32446) rectangle (14.6169,6.43031);
\draw [color=c, fill=c] (14.6169,6.32446) rectangle (14.6567,6.43031);
\draw [color=c, fill=c] (14.6567,6.32446) rectangle (14.6965,6.43031);
\draw [color=c, fill=c] (14.6965,6.32446) rectangle (14.7363,6.43031);
\draw [color=c, fill=c] (14.7363,6.32446) rectangle (14.7761,6.43031);
\draw [color=c, fill=c] (14.7761,6.32446) rectangle (14.8159,6.43031);
\draw [color=c, fill=c] (14.8159,6.32446) rectangle (14.8557,6.43031);
\draw [color=c, fill=c] (14.8557,6.32446) rectangle (14.8955,6.43031);
\draw [color=c, fill=c] (14.8955,6.32446) rectangle (14.9353,6.43031);
\draw [color=c, fill=c] (14.9353,6.32446) rectangle (14.9751,6.43031);
\draw [color=c, fill=c] (14.9751,6.32446) rectangle (15.0149,6.43031);
\draw [color=c, fill=c] (15.0149,6.32446) rectangle (15.0547,6.43031);
\draw [color=c, fill=c] (15.0547,6.32446) rectangle (15.0945,6.43031);
\draw [color=c, fill=c] (15.0945,6.32446) rectangle (15.1343,6.43031);
\draw [color=c, fill=c] (15.1343,6.32446) rectangle (15.1741,6.43031);
\draw [color=c, fill=c] (15.1741,6.32446) rectangle (15.2139,6.43031);
\draw [color=c, fill=c] (15.2139,6.32446) rectangle (15.2537,6.43031);
\draw [color=c, fill=c] (15.2537,6.32446) rectangle (15.2935,6.43031);
\draw [color=c, fill=c] (15.2935,6.32446) rectangle (15.3333,6.43031);
\draw [color=c, fill=c] (15.3333,6.32446) rectangle (15.3731,6.43031);
\draw [color=c, fill=c] (15.3731,6.32446) rectangle (15.4129,6.43031);
\draw [color=c, fill=c] (15.4129,6.32446) rectangle (15.4527,6.43031);
\draw [color=c, fill=c] (15.4527,6.32446) rectangle (15.4925,6.43031);
\draw [color=c, fill=c] (15.4925,6.32446) rectangle (15.5323,6.43031);
\draw [color=c, fill=c] (15.5323,6.32446) rectangle (15.5721,6.43031);
\draw [color=c, fill=c] (15.5721,6.32446) rectangle (15.6119,6.43031);
\draw [color=c, fill=c] (15.6119,6.32446) rectangle (15.6517,6.43031);
\draw [color=c, fill=c] (15.6517,6.32446) rectangle (15.6915,6.43031);
\draw [color=c, fill=c] (15.6915,6.32446) rectangle (15.7313,6.43031);
\draw [color=c, fill=c] (15.7313,6.32446) rectangle (15.7711,6.43031);
\draw [color=c, fill=c] (15.7711,6.32446) rectangle (15.8109,6.43031);
\draw [color=c, fill=c] (15.8109,6.32446) rectangle (15.8507,6.43031);
\draw [color=c, fill=c] (15.8507,6.32446) rectangle (15.8905,6.43031);
\draw [color=c, fill=c] (15.8905,6.32446) rectangle (15.9303,6.43031);
\draw [color=c, fill=c] (15.9303,6.32446) rectangle (15.9701,6.43031);
\draw [color=c, fill=c] (15.9701,6.32446) rectangle (16.01,6.43031);
\draw [color=c, fill=c] (16.01,6.32446) rectangle (16.0498,6.43031);
\draw [color=c, fill=c] (16.0498,6.32446) rectangle (16.0896,6.43031);
\draw [color=c, fill=c] (16.0896,6.32446) rectangle (16.1294,6.43031);
\draw [color=c, fill=c] (16.1294,6.32446) rectangle (16.1692,6.43031);
\draw [color=c, fill=c] (16.1692,6.32446) rectangle (16.209,6.43031);
\draw [color=c, fill=c] (16.209,6.32446) rectangle (16.2488,6.43031);
\draw [color=c, fill=c] (16.2488,6.32446) rectangle (16.2886,6.43031);
\draw [color=c, fill=c] (16.2886,6.32446) rectangle (16.3284,6.43031);
\draw [color=c, fill=c] (16.3284,6.32446) rectangle (16.3682,6.43031);
\draw [color=c, fill=c] (16.3682,6.32446) rectangle (16.408,6.43031);
\draw [color=c, fill=c] (16.408,6.32446) rectangle (16.4478,6.43031);
\draw [color=c, fill=c] (16.4478,6.32446) rectangle (16.4876,6.43031);
\draw [color=c, fill=c] (16.4876,6.32446) rectangle (16.5274,6.43031);
\draw [color=c, fill=c] (16.5274,6.32446) rectangle (16.5672,6.43031);
\draw [color=c, fill=c] (16.5672,6.32446) rectangle (16.607,6.43031);
\draw [color=c, fill=c] (16.607,6.32446) rectangle (16.6468,6.43031);
\draw [color=c, fill=c] (16.6468,6.32446) rectangle (16.6866,6.43031);
\draw [color=c, fill=c] (16.6866,6.32446) rectangle (16.7264,6.43031);
\draw [color=c, fill=c] (16.7264,6.32446) rectangle (16.7662,6.43031);
\draw [color=c, fill=c] (16.7662,6.32446) rectangle (16.806,6.43031);
\draw [color=c, fill=c] (16.806,6.32446) rectangle (16.8458,6.43031);
\draw [color=c, fill=c] (16.8458,6.32446) rectangle (16.8856,6.43031);
\draw [color=c, fill=c] (16.8856,6.32446) rectangle (16.9254,6.43031);
\draw [color=c, fill=c] (16.9254,6.32446) rectangle (16.9652,6.43031);
\draw [color=c, fill=c] (16.9652,6.32446) rectangle (17.005,6.43031);
\draw [color=c, fill=c] (17.005,6.32446) rectangle (17.0448,6.43031);
\draw [color=c, fill=c] (17.0448,6.32446) rectangle (17.0846,6.43031);
\draw [color=c, fill=c] (17.0846,6.32446) rectangle (17.1244,6.43031);
\draw [color=c, fill=c] (17.1244,6.32446) rectangle (17.1642,6.43031);
\draw [color=c, fill=c] (17.1642,6.32446) rectangle (17.204,6.43031);
\draw [color=c, fill=c] (17.204,6.32446) rectangle (17.2438,6.43031);
\draw [color=c, fill=c] (17.2438,6.32446) rectangle (17.2836,6.43031);
\draw [color=c, fill=c] (17.2836,6.32446) rectangle (17.3234,6.43031);
\draw [color=c, fill=c] (17.3234,6.32446) rectangle (17.3632,6.43031);
\draw [color=c, fill=c] (17.3632,6.32446) rectangle (17.403,6.43031);
\draw [color=c, fill=c] (17.403,6.32446) rectangle (17.4428,6.43031);
\draw [color=c, fill=c] (17.4428,6.32446) rectangle (17.4826,6.43031);
\draw [color=c, fill=c] (17.4826,6.32446) rectangle (17.5224,6.43031);
\draw [color=c, fill=c] (17.5224,6.32446) rectangle (17.5622,6.43031);
\draw [color=c, fill=c] (17.5622,6.32446) rectangle (17.602,6.43031);
\draw [color=c, fill=c] (17.602,6.32446) rectangle (17.6418,6.43031);
\draw [color=c, fill=c] (17.6418,6.32446) rectangle (17.6816,6.43031);
\draw [color=c, fill=c] (17.6816,6.32446) rectangle (17.7214,6.43031);
\draw [color=c, fill=c] (17.7214,6.32446) rectangle (17.7612,6.43031);
\draw [color=c, fill=c] (17.7612,6.32446) rectangle (17.801,6.43031);
\draw [color=c, fill=c] (17.801,6.32446) rectangle (17.8408,6.43031);
\draw [color=c, fill=c] (17.8408,6.32446) rectangle (17.8806,6.43031);
\draw [color=c, fill=c] (17.8806,6.32446) rectangle (17.9204,6.43031);
\draw [color=c, fill=c] (17.9204,6.32446) rectangle (17.9602,6.43031);
\draw [color=c, fill=c] (17.9602,6.32446) rectangle (18,6.43031);
\definecolor{c}{rgb}{0,0.0800001,1};
\draw [color=c, fill=c] (2,6.43031) rectangle (2.0398,6.53615);
\draw [color=c, fill=c] (2.0398,6.43031) rectangle (2.0796,6.53615);
\draw [color=c, fill=c] (2.0796,6.43031) rectangle (2.1194,6.53615);
\draw [color=c, fill=c] (2.1194,6.43031) rectangle (2.1592,6.53615);
\draw [color=c, fill=c] (2.1592,6.43031) rectangle (2.19901,6.53615);
\draw [color=c, fill=c] (2.19901,6.43031) rectangle (2.23881,6.53615);
\draw [color=c, fill=c] (2.23881,6.43031) rectangle (2.27861,6.53615);
\draw [color=c, fill=c] (2.27861,6.43031) rectangle (2.31841,6.53615);
\draw [color=c, fill=c] (2.31841,6.43031) rectangle (2.35821,6.53615);
\draw [color=c, fill=c] (2.35821,6.43031) rectangle (2.39801,6.53615);
\draw [color=c, fill=c] (2.39801,6.43031) rectangle (2.43781,6.53615);
\draw [color=c, fill=c] (2.43781,6.43031) rectangle (2.47761,6.53615);
\draw [color=c, fill=c] (2.47761,6.43031) rectangle (2.51741,6.53615);
\draw [color=c, fill=c] (2.51741,6.43031) rectangle (2.55721,6.53615);
\draw [color=c, fill=c] (2.55721,6.43031) rectangle (2.59702,6.53615);
\draw [color=c, fill=c] (2.59702,6.43031) rectangle (2.63682,6.53615);
\draw [color=c, fill=c] (2.63682,6.43031) rectangle (2.67662,6.53615);
\draw [color=c, fill=c] (2.67662,6.43031) rectangle (2.71642,6.53615);
\draw [color=c, fill=c] (2.71642,6.43031) rectangle (2.75622,6.53615);
\draw [color=c, fill=c] (2.75622,6.43031) rectangle (2.79602,6.53615);
\draw [color=c, fill=c] (2.79602,6.43031) rectangle (2.83582,6.53615);
\draw [color=c, fill=c] (2.83582,6.43031) rectangle (2.87562,6.53615);
\draw [color=c, fill=c] (2.87562,6.43031) rectangle (2.91542,6.53615);
\draw [color=c, fill=c] (2.91542,6.43031) rectangle (2.95522,6.53615);
\draw [color=c, fill=c] (2.95522,6.43031) rectangle (2.99502,6.53615);
\draw [color=c, fill=c] (2.99502,6.43031) rectangle (3.03483,6.53615);
\draw [color=c, fill=c] (3.03483,6.43031) rectangle (3.07463,6.53615);
\draw [color=c, fill=c] (3.07463,6.43031) rectangle (3.11443,6.53615);
\draw [color=c, fill=c] (3.11443,6.43031) rectangle (3.15423,6.53615);
\draw [color=c, fill=c] (3.15423,6.43031) rectangle (3.19403,6.53615);
\draw [color=c, fill=c] (3.19403,6.43031) rectangle (3.23383,6.53615);
\draw [color=c, fill=c] (3.23383,6.43031) rectangle (3.27363,6.53615);
\draw [color=c, fill=c] (3.27363,6.43031) rectangle (3.31343,6.53615);
\draw [color=c, fill=c] (3.31343,6.43031) rectangle (3.35323,6.53615);
\draw [color=c, fill=c] (3.35323,6.43031) rectangle (3.39303,6.53615);
\draw [color=c, fill=c] (3.39303,6.43031) rectangle (3.43284,6.53615);
\draw [color=c, fill=c] (3.43284,6.43031) rectangle (3.47264,6.53615);
\draw [color=c, fill=c] (3.47264,6.43031) rectangle (3.51244,6.53615);
\draw [color=c, fill=c] (3.51244,6.43031) rectangle (3.55224,6.53615);
\draw [color=c, fill=c] (3.55224,6.43031) rectangle (3.59204,6.53615);
\draw [color=c, fill=c] (3.59204,6.43031) rectangle (3.63184,6.53615);
\draw [color=c, fill=c] (3.63184,6.43031) rectangle (3.67164,6.53615);
\draw [color=c, fill=c] (3.67164,6.43031) rectangle (3.71144,6.53615);
\draw [color=c, fill=c] (3.71144,6.43031) rectangle (3.75124,6.53615);
\draw [color=c, fill=c] (3.75124,6.43031) rectangle (3.79104,6.53615);
\draw [color=c, fill=c] (3.79104,6.43031) rectangle (3.83085,6.53615);
\draw [color=c, fill=c] (3.83085,6.43031) rectangle (3.87065,6.53615);
\draw [color=c, fill=c] (3.87065,6.43031) rectangle (3.91045,6.53615);
\draw [color=c, fill=c] (3.91045,6.43031) rectangle (3.95025,6.53615);
\draw [color=c, fill=c] (3.95025,6.43031) rectangle (3.99005,6.53615);
\draw [color=c, fill=c] (3.99005,6.43031) rectangle (4.02985,6.53615);
\draw [color=c, fill=c] (4.02985,6.43031) rectangle (4.06965,6.53615);
\draw [color=c, fill=c] (4.06965,6.43031) rectangle (4.10945,6.53615);
\draw [color=c, fill=c] (4.10945,6.43031) rectangle (4.14925,6.53615);
\draw [color=c, fill=c] (4.14925,6.43031) rectangle (4.18905,6.53615);
\draw [color=c, fill=c] (4.18905,6.43031) rectangle (4.22886,6.53615);
\draw [color=c, fill=c] (4.22886,6.43031) rectangle (4.26866,6.53615);
\draw [color=c, fill=c] (4.26866,6.43031) rectangle (4.30846,6.53615);
\draw [color=c, fill=c] (4.30846,6.43031) rectangle (4.34826,6.53615);
\draw [color=c, fill=c] (4.34826,6.43031) rectangle (4.38806,6.53615);
\draw [color=c, fill=c] (4.38806,6.43031) rectangle (4.42786,6.53615);
\draw [color=c, fill=c] (4.42786,6.43031) rectangle (4.46766,6.53615);
\draw [color=c, fill=c] (4.46766,6.43031) rectangle (4.50746,6.53615);
\draw [color=c, fill=c] (4.50746,6.43031) rectangle (4.54726,6.53615);
\draw [color=c, fill=c] (4.54726,6.43031) rectangle (4.58706,6.53615);
\draw [color=c, fill=c] (4.58706,6.43031) rectangle (4.62687,6.53615);
\draw [color=c, fill=c] (4.62687,6.43031) rectangle (4.66667,6.53615);
\draw [color=c, fill=c] (4.66667,6.43031) rectangle (4.70647,6.53615);
\draw [color=c, fill=c] (4.70647,6.43031) rectangle (4.74627,6.53615);
\draw [color=c, fill=c] (4.74627,6.43031) rectangle (4.78607,6.53615);
\draw [color=c, fill=c] (4.78607,6.43031) rectangle (4.82587,6.53615);
\draw [color=c, fill=c] (4.82587,6.43031) rectangle (4.86567,6.53615);
\definecolor{c}{rgb}{0.2,0,1};
\draw [color=c, fill=c] (4.86567,6.43031) rectangle (4.90547,6.53615);
\draw [color=c, fill=c] (4.90547,6.43031) rectangle (4.94527,6.53615);
\draw [color=c, fill=c] (4.94527,6.43031) rectangle (4.98507,6.53615);
\draw [color=c, fill=c] (4.98507,6.43031) rectangle (5.02488,6.53615);
\draw [color=c, fill=c] (5.02488,6.43031) rectangle (5.06468,6.53615);
\draw [color=c, fill=c] (5.06468,6.43031) rectangle (5.10448,6.53615);
\draw [color=c, fill=c] (5.10448,6.43031) rectangle (5.14428,6.53615);
\draw [color=c, fill=c] (5.14428,6.43031) rectangle (5.18408,6.53615);
\draw [color=c, fill=c] (5.18408,6.43031) rectangle (5.22388,6.53615);
\draw [color=c, fill=c] (5.22388,6.43031) rectangle (5.26368,6.53615);
\draw [color=c, fill=c] (5.26368,6.43031) rectangle (5.30348,6.53615);
\draw [color=c, fill=c] (5.30348,6.43031) rectangle (5.34328,6.53615);
\draw [color=c, fill=c] (5.34328,6.43031) rectangle (5.38308,6.53615);
\draw [color=c, fill=c] (5.38308,6.43031) rectangle (5.42289,6.53615);
\draw [color=c, fill=c] (5.42289,6.43031) rectangle (5.46269,6.53615);
\draw [color=c, fill=c] (5.46269,6.43031) rectangle (5.50249,6.53615);
\draw [color=c, fill=c] (5.50249,6.43031) rectangle (5.54229,6.53615);
\draw [color=c, fill=c] (5.54229,6.43031) rectangle (5.58209,6.53615);
\draw [color=c, fill=c] (5.58209,6.43031) rectangle (5.62189,6.53615);
\draw [color=c, fill=c] (5.62189,6.43031) rectangle (5.66169,6.53615);
\draw [color=c, fill=c] (5.66169,6.43031) rectangle (5.70149,6.53615);
\draw [color=c, fill=c] (5.70149,6.43031) rectangle (5.74129,6.53615);
\draw [color=c, fill=c] (5.74129,6.43031) rectangle (5.78109,6.53615);
\draw [color=c, fill=c] (5.78109,6.43031) rectangle (5.8209,6.53615);
\draw [color=c, fill=c] (5.8209,6.43031) rectangle (5.8607,6.53615);
\draw [color=c, fill=c] (5.8607,6.43031) rectangle (5.9005,6.53615);
\draw [color=c, fill=c] (5.9005,6.43031) rectangle (5.9403,6.53615);
\draw [color=c, fill=c] (5.9403,6.43031) rectangle (5.9801,6.53615);
\draw [color=c, fill=c] (5.9801,6.43031) rectangle (6.0199,6.53615);
\draw [color=c, fill=c] (6.0199,6.43031) rectangle (6.0597,6.53615);
\draw [color=c, fill=c] (6.0597,6.43031) rectangle (6.0995,6.53615);
\draw [color=c, fill=c] (6.0995,6.43031) rectangle (6.1393,6.53615);
\draw [color=c, fill=c] (6.1393,6.43031) rectangle (6.1791,6.53615);
\draw [color=c, fill=c] (6.1791,6.43031) rectangle (6.21891,6.53615);
\draw [color=c, fill=c] (6.21891,6.43031) rectangle (6.25871,6.53615);
\draw [color=c, fill=c] (6.25871,6.43031) rectangle (6.29851,6.53615);
\draw [color=c, fill=c] (6.29851,6.43031) rectangle (6.33831,6.53615);
\draw [color=c, fill=c] (6.33831,6.43031) rectangle (6.37811,6.53615);
\draw [color=c, fill=c] (6.37811,6.43031) rectangle (6.41791,6.53615);
\draw [color=c, fill=c] (6.41791,6.43031) rectangle (6.45771,6.53615);
\draw [color=c, fill=c] (6.45771,6.43031) rectangle (6.49751,6.53615);
\draw [color=c, fill=c] (6.49751,6.43031) rectangle (6.53731,6.53615);
\draw [color=c, fill=c] (6.53731,6.43031) rectangle (6.57711,6.53615);
\draw [color=c, fill=c] (6.57711,6.43031) rectangle (6.61692,6.53615);
\draw [color=c, fill=c] (6.61692,6.43031) rectangle (6.65672,6.53615);
\draw [color=c, fill=c] (6.65672,6.43031) rectangle (6.69652,6.53615);
\draw [color=c, fill=c] (6.69652,6.43031) rectangle (6.73632,6.53615);
\draw [color=c, fill=c] (6.73632,6.43031) rectangle (6.77612,6.53615);
\draw [color=c, fill=c] (6.77612,6.43031) rectangle (6.81592,6.53615);
\draw [color=c, fill=c] (6.81592,6.43031) rectangle (6.85572,6.53615);
\draw [color=c, fill=c] (6.85572,6.43031) rectangle (6.89552,6.53615);
\draw [color=c, fill=c] (6.89552,6.43031) rectangle (6.93532,6.53615);
\draw [color=c, fill=c] (6.93532,6.43031) rectangle (6.97512,6.53615);
\draw [color=c, fill=c] (6.97512,6.43031) rectangle (7.01493,6.53615);
\draw [color=c, fill=c] (7.01493,6.43031) rectangle (7.05473,6.53615);
\draw [color=c, fill=c] (7.05473,6.43031) rectangle (7.09453,6.53615);
\draw [color=c, fill=c] (7.09453,6.43031) rectangle (7.13433,6.53615);
\draw [color=c, fill=c] (7.13433,6.43031) rectangle (7.17413,6.53615);
\draw [color=c, fill=c] (7.17413,6.43031) rectangle (7.21393,6.53615);
\draw [color=c, fill=c] (7.21393,6.43031) rectangle (7.25373,6.53615);
\draw [color=c, fill=c] (7.25373,6.43031) rectangle (7.29353,6.53615);
\draw [color=c, fill=c] (7.29353,6.43031) rectangle (7.33333,6.53615);
\draw [color=c, fill=c] (7.33333,6.43031) rectangle (7.37313,6.53615);
\draw [color=c, fill=c] (7.37313,6.43031) rectangle (7.41294,6.53615);
\draw [color=c, fill=c] (7.41294,6.43031) rectangle (7.45274,6.53615);
\draw [color=c, fill=c] (7.45274,6.43031) rectangle (7.49254,6.53615);
\draw [color=c, fill=c] (7.49254,6.43031) rectangle (7.53234,6.53615);
\draw [color=c, fill=c] (7.53234,6.43031) rectangle (7.57214,6.53615);
\draw [color=c, fill=c] (7.57214,6.43031) rectangle (7.61194,6.53615);
\draw [color=c, fill=c] (7.61194,6.43031) rectangle (7.65174,6.53615);
\draw [color=c, fill=c] (7.65174,6.43031) rectangle (7.69154,6.53615);
\draw [color=c, fill=c] (7.69154,6.43031) rectangle (7.73134,6.53615);
\draw [color=c, fill=c] (7.73134,6.43031) rectangle (7.77114,6.53615);
\draw [color=c, fill=c] (7.77114,6.43031) rectangle (7.81095,6.53615);
\draw [color=c, fill=c] (7.81095,6.43031) rectangle (7.85075,6.53615);
\draw [color=c, fill=c] (7.85075,6.43031) rectangle (7.89055,6.53615);
\draw [color=c, fill=c] (7.89055,6.43031) rectangle (7.93035,6.53615);
\draw [color=c, fill=c] (7.93035,6.43031) rectangle (7.97015,6.53615);
\draw [color=c, fill=c] (7.97015,6.43031) rectangle (8.00995,6.53615);
\draw [color=c, fill=c] (8.00995,6.43031) rectangle (8.04975,6.53615);
\draw [color=c, fill=c] (8.04975,6.43031) rectangle (8.08955,6.53615);
\draw [color=c, fill=c] (8.08955,6.43031) rectangle (8.12935,6.53615);
\draw [color=c, fill=c] (8.12935,6.43031) rectangle (8.16915,6.53615);
\draw [color=c, fill=c] (8.16915,6.43031) rectangle (8.20895,6.53615);
\draw [color=c, fill=c] (8.20895,6.43031) rectangle (8.24876,6.53615);
\draw [color=c, fill=c] (8.24876,6.43031) rectangle (8.28856,6.53615);
\draw [color=c, fill=c] (8.28856,6.43031) rectangle (8.32836,6.53615);
\draw [color=c, fill=c] (8.32836,6.43031) rectangle (8.36816,6.53615);
\draw [color=c, fill=c] (8.36816,6.43031) rectangle (8.40796,6.53615);
\draw [color=c, fill=c] (8.40796,6.43031) rectangle (8.44776,6.53615);
\draw [color=c, fill=c] (8.44776,6.43031) rectangle (8.48756,6.53615);
\draw [color=c, fill=c] (8.48756,6.43031) rectangle (8.52736,6.53615);
\draw [color=c, fill=c] (8.52736,6.43031) rectangle (8.56716,6.53615);
\draw [color=c, fill=c] (8.56716,6.43031) rectangle (8.60697,6.53615);
\draw [color=c, fill=c] (8.60697,6.43031) rectangle (8.64677,6.53615);
\definecolor{c}{rgb}{0,0.0800001,1};
\draw [color=c, fill=c] (8.64677,6.43031) rectangle (8.68657,6.53615);
\draw [color=c, fill=c] (8.68657,6.43031) rectangle (8.72637,6.53615);
\draw [color=c, fill=c] (8.72637,6.43031) rectangle (8.76617,6.53615);
\draw [color=c, fill=c] (8.76617,6.43031) rectangle (8.80597,6.53615);
\draw [color=c, fill=c] (8.80597,6.43031) rectangle (8.84577,6.53615);
\draw [color=c, fill=c] (8.84577,6.43031) rectangle (8.88557,6.53615);
\draw [color=c, fill=c] (8.88557,6.43031) rectangle (8.92537,6.53615);
\draw [color=c, fill=c] (8.92537,6.43031) rectangle (8.96517,6.53615);
\draw [color=c, fill=c] (8.96517,6.43031) rectangle (9.00498,6.53615);
\draw [color=c, fill=c] (9.00498,6.43031) rectangle (9.04478,6.53615);
\draw [color=c, fill=c] (9.04478,6.43031) rectangle (9.08458,6.53615);
\draw [color=c, fill=c] (9.08458,6.43031) rectangle (9.12438,6.53615);
\draw [color=c, fill=c] (9.12438,6.43031) rectangle (9.16418,6.53615);
\draw [color=c, fill=c] (9.16418,6.43031) rectangle (9.20398,6.53615);
\draw [color=c, fill=c] (9.20398,6.43031) rectangle (9.24378,6.53615);
\draw [color=c, fill=c] (9.24378,6.43031) rectangle (9.28358,6.53615);
\draw [color=c, fill=c] (9.28358,6.43031) rectangle (9.32338,6.53615);
\draw [color=c, fill=c] (9.32338,6.43031) rectangle (9.36318,6.53615);
\draw [color=c, fill=c] (9.36318,6.43031) rectangle (9.40298,6.53615);
\draw [color=c, fill=c] (9.40298,6.43031) rectangle (9.44279,6.53615);
\draw [color=c, fill=c] (9.44279,6.43031) rectangle (9.48259,6.53615);
\draw [color=c, fill=c] (9.48259,6.43031) rectangle (9.52239,6.53615);
\draw [color=c, fill=c] (9.52239,6.43031) rectangle (9.56219,6.53615);
\draw [color=c, fill=c] (9.56219,6.43031) rectangle (9.60199,6.53615);
\draw [color=c, fill=c] (9.60199,6.43031) rectangle (9.64179,6.53615);
\definecolor{c}{rgb}{0,0.266667,1};
\draw [color=c, fill=c] (9.64179,6.43031) rectangle (9.68159,6.53615);
\draw [color=c, fill=c] (9.68159,6.43031) rectangle (9.72139,6.53615);
\draw [color=c, fill=c] (9.72139,6.43031) rectangle (9.76119,6.53615);
\draw [color=c, fill=c] (9.76119,6.43031) rectangle (9.80099,6.53615);
\draw [color=c, fill=c] (9.80099,6.43031) rectangle (9.8408,6.53615);
\draw [color=c, fill=c] (9.8408,6.43031) rectangle (9.8806,6.53615);
\draw [color=c, fill=c] (9.8806,6.43031) rectangle (9.9204,6.53615);
\draw [color=c, fill=c] (9.9204,6.43031) rectangle (9.9602,6.53615);
\draw [color=c, fill=c] (9.9602,6.43031) rectangle (10,6.53615);
\draw [color=c, fill=c] (10,6.43031) rectangle (10.0398,6.53615);
\draw [color=c, fill=c] (10.0398,6.43031) rectangle (10.0796,6.53615);
\draw [color=c, fill=c] (10.0796,6.43031) rectangle (10.1194,6.53615);
\draw [color=c, fill=c] (10.1194,6.43031) rectangle (10.1592,6.53615);
\definecolor{c}{rgb}{0,0.546666,1};
\draw [color=c, fill=c] (10.1592,6.43031) rectangle (10.199,6.53615);
\draw [color=c, fill=c] (10.199,6.43031) rectangle (10.2388,6.53615);
\draw [color=c, fill=c] (10.2388,6.43031) rectangle (10.2786,6.53615);
\draw [color=c, fill=c] (10.2786,6.43031) rectangle (10.3184,6.53615);
\draw [color=c, fill=c] (10.3184,6.43031) rectangle (10.3582,6.53615);
\draw [color=c, fill=c] (10.3582,6.43031) rectangle (10.398,6.53615);
\draw [color=c, fill=c] (10.398,6.43031) rectangle (10.4378,6.53615);
\draw [color=c, fill=c] (10.4378,6.43031) rectangle (10.4776,6.53615);
\draw [color=c, fill=c] (10.4776,6.43031) rectangle (10.5174,6.53615);
\draw [color=c, fill=c] (10.5174,6.43031) rectangle (10.5572,6.53615);
\draw [color=c, fill=c] (10.5572,6.43031) rectangle (10.597,6.53615);
\draw [color=c, fill=c] (10.597,6.43031) rectangle (10.6368,6.53615);
\draw [color=c, fill=c] (10.6368,6.43031) rectangle (10.6766,6.53615);
\draw [color=c, fill=c] (10.6766,6.43031) rectangle (10.7164,6.53615);
\draw [color=c, fill=c] (10.7164,6.43031) rectangle (10.7562,6.53615);
\draw [color=c, fill=c] (10.7562,6.43031) rectangle (10.796,6.53615);
\draw [color=c, fill=c] (10.796,6.43031) rectangle (10.8358,6.53615);
\draw [color=c, fill=c] (10.8358,6.43031) rectangle (10.8756,6.53615);
\draw [color=c, fill=c] (10.8756,6.43031) rectangle (10.9154,6.53615);
\draw [color=c, fill=c] (10.9154,6.43031) rectangle (10.9552,6.53615);
\definecolor{c}{rgb}{0,0.733333,1};
\draw [color=c, fill=c] (10.9552,6.43031) rectangle (10.995,6.53615);
\draw [color=c, fill=c] (10.995,6.43031) rectangle (11.0348,6.53615);
\draw [color=c, fill=c] (11.0348,6.43031) rectangle (11.0746,6.53615);
\draw [color=c, fill=c] (11.0746,6.43031) rectangle (11.1144,6.53615);
\draw [color=c, fill=c] (11.1144,6.43031) rectangle (11.1542,6.53615);
\draw [color=c, fill=c] (11.1542,6.43031) rectangle (11.194,6.53615);
\draw [color=c, fill=c] (11.194,6.43031) rectangle (11.2338,6.53615);
\draw [color=c, fill=c] (11.2338,6.43031) rectangle (11.2736,6.53615);
\draw [color=c, fill=c] (11.2736,6.43031) rectangle (11.3134,6.53615);
\draw [color=c, fill=c] (11.3134,6.43031) rectangle (11.3532,6.53615);
\draw [color=c, fill=c] (11.3532,6.43031) rectangle (11.393,6.53615);
\draw [color=c, fill=c] (11.393,6.43031) rectangle (11.4328,6.53615);
\draw [color=c, fill=c] (11.4328,6.43031) rectangle (11.4726,6.53615);
\draw [color=c, fill=c] (11.4726,6.43031) rectangle (11.5124,6.53615);
\draw [color=c, fill=c] (11.5124,6.43031) rectangle (11.5522,6.53615);
\draw [color=c, fill=c] (11.5522,6.43031) rectangle (11.592,6.53615);
\draw [color=c, fill=c] (11.592,6.43031) rectangle (11.6318,6.53615);
\draw [color=c, fill=c] (11.6318,6.43031) rectangle (11.6716,6.53615);
\draw [color=c, fill=c] (11.6716,6.43031) rectangle (11.7114,6.53615);
\draw [color=c, fill=c] (11.7114,6.43031) rectangle (11.7512,6.53615);
\draw [color=c, fill=c] (11.7512,6.43031) rectangle (11.791,6.53615);
\draw [color=c, fill=c] (11.791,6.43031) rectangle (11.8308,6.53615);
\draw [color=c, fill=c] (11.8308,6.43031) rectangle (11.8706,6.53615);
\draw [color=c, fill=c] (11.8706,6.43031) rectangle (11.9104,6.53615);
\draw [color=c, fill=c] (11.9104,6.43031) rectangle (11.9502,6.53615);
\draw [color=c, fill=c] (11.9502,6.43031) rectangle (11.99,6.53615);
\draw [color=c, fill=c] (11.99,6.43031) rectangle (12.0299,6.53615);
\draw [color=c, fill=c] (12.0299,6.43031) rectangle (12.0697,6.53615);
\draw [color=c, fill=c] (12.0697,6.43031) rectangle (12.1095,6.53615);
\draw [color=c, fill=c] (12.1095,6.43031) rectangle (12.1493,6.53615);
\draw [color=c, fill=c] (12.1493,6.43031) rectangle (12.1891,6.53615);
\draw [color=c, fill=c] (12.1891,6.43031) rectangle (12.2289,6.53615);
\draw [color=c, fill=c] (12.2289,6.43031) rectangle (12.2687,6.53615);
\draw [color=c, fill=c] (12.2687,6.43031) rectangle (12.3085,6.53615);
\draw [color=c, fill=c] (12.3085,6.43031) rectangle (12.3483,6.53615);
\draw [color=c, fill=c] (12.3483,6.43031) rectangle (12.3881,6.53615);
\draw [color=c, fill=c] (12.3881,6.43031) rectangle (12.4279,6.53615);
\draw [color=c, fill=c] (12.4279,6.43031) rectangle (12.4677,6.53615);
\draw [color=c, fill=c] (12.4677,6.43031) rectangle (12.5075,6.53615);
\draw [color=c, fill=c] (12.5075,6.43031) rectangle (12.5473,6.53615);
\draw [color=c, fill=c] (12.5473,6.43031) rectangle (12.5871,6.53615);
\draw [color=c, fill=c] (12.5871,6.43031) rectangle (12.6269,6.53615);
\draw [color=c, fill=c] (12.6269,6.43031) rectangle (12.6667,6.53615);
\draw [color=c, fill=c] (12.6667,6.43031) rectangle (12.7065,6.53615);
\draw [color=c, fill=c] (12.7065,6.43031) rectangle (12.7463,6.53615);
\draw [color=c, fill=c] (12.7463,6.43031) rectangle (12.7861,6.53615);
\draw [color=c, fill=c] (12.7861,6.43031) rectangle (12.8259,6.53615);
\draw [color=c, fill=c] (12.8259,6.43031) rectangle (12.8657,6.53615);
\draw [color=c, fill=c] (12.8657,6.43031) rectangle (12.9055,6.53615);
\draw [color=c, fill=c] (12.9055,6.43031) rectangle (12.9453,6.53615);
\draw [color=c, fill=c] (12.9453,6.43031) rectangle (12.9851,6.53615);
\draw [color=c, fill=c] (12.9851,6.43031) rectangle (13.0249,6.53615);
\draw [color=c, fill=c] (13.0249,6.43031) rectangle (13.0647,6.53615);
\draw [color=c, fill=c] (13.0647,6.43031) rectangle (13.1045,6.53615);
\draw [color=c, fill=c] (13.1045,6.43031) rectangle (13.1443,6.53615);
\draw [color=c, fill=c] (13.1443,6.43031) rectangle (13.1841,6.53615);
\draw [color=c, fill=c] (13.1841,6.43031) rectangle (13.2239,6.53615);
\draw [color=c, fill=c] (13.2239,6.43031) rectangle (13.2637,6.53615);
\draw [color=c, fill=c] (13.2637,6.43031) rectangle (13.3035,6.53615);
\draw [color=c, fill=c] (13.3035,6.43031) rectangle (13.3433,6.53615);
\draw [color=c, fill=c] (13.3433,6.43031) rectangle (13.3831,6.53615);
\draw [color=c, fill=c] (13.3831,6.43031) rectangle (13.4229,6.53615);
\draw [color=c, fill=c] (13.4229,6.43031) rectangle (13.4627,6.53615);
\draw [color=c, fill=c] (13.4627,6.43031) rectangle (13.5025,6.53615);
\draw [color=c, fill=c] (13.5025,6.43031) rectangle (13.5423,6.53615);
\draw [color=c, fill=c] (13.5423,6.43031) rectangle (13.5821,6.53615);
\draw [color=c, fill=c] (13.5821,6.43031) rectangle (13.6219,6.53615);
\draw [color=c, fill=c] (13.6219,6.43031) rectangle (13.6617,6.53615);
\draw [color=c, fill=c] (13.6617,6.43031) rectangle (13.7015,6.53615);
\draw [color=c, fill=c] (13.7015,6.43031) rectangle (13.7413,6.53615);
\draw [color=c, fill=c] (13.7413,6.43031) rectangle (13.7811,6.53615);
\draw [color=c, fill=c] (13.7811,6.43031) rectangle (13.8209,6.53615);
\draw [color=c, fill=c] (13.8209,6.43031) rectangle (13.8607,6.53615);
\draw [color=c, fill=c] (13.8607,6.43031) rectangle (13.9005,6.53615);
\draw [color=c, fill=c] (13.9005,6.43031) rectangle (13.9403,6.53615);
\draw [color=c, fill=c] (13.9403,6.43031) rectangle (13.9801,6.53615);
\draw [color=c, fill=c] (13.9801,6.43031) rectangle (14.0199,6.53615);
\draw [color=c, fill=c] (14.0199,6.43031) rectangle (14.0597,6.53615);
\draw [color=c, fill=c] (14.0597,6.43031) rectangle (14.0995,6.53615);
\draw [color=c, fill=c] (14.0995,6.43031) rectangle (14.1393,6.53615);
\draw [color=c, fill=c] (14.1393,6.43031) rectangle (14.1791,6.53615);
\draw [color=c, fill=c] (14.1791,6.43031) rectangle (14.2189,6.53615);
\draw [color=c, fill=c] (14.2189,6.43031) rectangle (14.2587,6.53615);
\draw [color=c, fill=c] (14.2587,6.43031) rectangle (14.2985,6.53615);
\draw [color=c, fill=c] (14.2985,6.43031) rectangle (14.3383,6.53615);
\draw [color=c, fill=c] (14.3383,6.43031) rectangle (14.3781,6.53615);
\draw [color=c, fill=c] (14.3781,6.43031) rectangle (14.4179,6.53615);
\draw [color=c, fill=c] (14.4179,6.43031) rectangle (14.4577,6.53615);
\draw [color=c, fill=c] (14.4577,6.43031) rectangle (14.4975,6.53615);
\draw [color=c, fill=c] (14.4975,6.43031) rectangle (14.5373,6.53615);
\draw [color=c, fill=c] (14.5373,6.43031) rectangle (14.5771,6.53615);
\draw [color=c, fill=c] (14.5771,6.43031) rectangle (14.6169,6.53615);
\draw [color=c, fill=c] (14.6169,6.43031) rectangle (14.6567,6.53615);
\draw [color=c, fill=c] (14.6567,6.43031) rectangle (14.6965,6.53615);
\draw [color=c, fill=c] (14.6965,6.43031) rectangle (14.7363,6.53615);
\draw [color=c, fill=c] (14.7363,6.43031) rectangle (14.7761,6.53615);
\draw [color=c, fill=c] (14.7761,6.43031) rectangle (14.8159,6.53615);
\draw [color=c, fill=c] (14.8159,6.43031) rectangle (14.8557,6.53615);
\draw [color=c, fill=c] (14.8557,6.43031) rectangle (14.8955,6.53615);
\draw [color=c, fill=c] (14.8955,6.43031) rectangle (14.9353,6.53615);
\draw [color=c, fill=c] (14.9353,6.43031) rectangle (14.9751,6.53615);
\draw [color=c, fill=c] (14.9751,6.43031) rectangle (15.0149,6.53615);
\draw [color=c, fill=c] (15.0149,6.43031) rectangle (15.0547,6.53615);
\draw [color=c, fill=c] (15.0547,6.43031) rectangle (15.0945,6.53615);
\draw [color=c, fill=c] (15.0945,6.43031) rectangle (15.1343,6.53615);
\draw [color=c, fill=c] (15.1343,6.43031) rectangle (15.1741,6.53615);
\draw [color=c, fill=c] (15.1741,6.43031) rectangle (15.2139,6.53615);
\draw [color=c, fill=c] (15.2139,6.43031) rectangle (15.2537,6.53615);
\draw [color=c, fill=c] (15.2537,6.43031) rectangle (15.2935,6.53615);
\draw [color=c, fill=c] (15.2935,6.43031) rectangle (15.3333,6.53615);
\draw [color=c, fill=c] (15.3333,6.43031) rectangle (15.3731,6.53615);
\draw [color=c, fill=c] (15.3731,6.43031) rectangle (15.4129,6.53615);
\draw [color=c, fill=c] (15.4129,6.43031) rectangle (15.4527,6.53615);
\draw [color=c, fill=c] (15.4527,6.43031) rectangle (15.4925,6.53615);
\draw [color=c, fill=c] (15.4925,6.43031) rectangle (15.5323,6.53615);
\draw [color=c, fill=c] (15.5323,6.43031) rectangle (15.5721,6.53615);
\draw [color=c, fill=c] (15.5721,6.43031) rectangle (15.6119,6.53615);
\draw [color=c, fill=c] (15.6119,6.43031) rectangle (15.6517,6.53615);
\draw [color=c, fill=c] (15.6517,6.43031) rectangle (15.6915,6.53615);
\draw [color=c, fill=c] (15.6915,6.43031) rectangle (15.7313,6.53615);
\draw [color=c, fill=c] (15.7313,6.43031) rectangle (15.7711,6.53615);
\draw [color=c, fill=c] (15.7711,6.43031) rectangle (15.8109,6.53615);
\draw [color=c, fill=c] (15.8109,6.43031) rectangle (15.8507,6.53615);
\draw [color=c, fill=c] (15.8507,6.43031) rectangle (15.8905,6.53615);
\draw [color=c, fill=c] (15.8905,6.43031) rectangle (15.9303,6.53615);
\draw [color=c, fill=c] (15.9303,6.43031) rectangle (15.9701,6.53615);
\draw [color=c, fill=c] (15.9701,6.43031) rectangle (16.01,6.53615);
\draw [color=c, fill=c] (16.01,6.43031) rectangle (16.0498,6.53615);
\draw [color=c, fill=c] (16.0498,6.43031) rectangle (16.0896,6.53615);
\draw [color=c, fill=c] (16.0896,6.43031) rectangle (16.1294,6.53615);
\draw [color=c, fill=c] (16.1294,6.43031) rectangle (16.1692,6.53615);
\draw [color=c, fill=c] (16.1692,6.43031) rectangle (16.209,6.53615);
\draw [color=c, fill=c] (16.209,6.43031) rectangle (16.2488,6.53615);
\draw [color=c, fill=c] (16.2488,6.43031) rectangle (16.2886,6.53615);
\draw [color=c, fill=c] (16.2886,6.43031) rectangle (16.3284,6.53615);
\draw [color=c, fill=c] (16.3284,6.43031) rectangle (16.3682,6.53615);
\draw [color=c, fill=c] (16.3682,6.43031) rectangle (16.408,6.53615);
\draw [color=c, fill=c] (16.408,6.43031) rectangle (16.4478,6.53615);
\draw [color=c, fill=c] (16.4478,6.43031) rectangle (16.4876,6.53615);
\draw [color=c, fill=c] (16.4876,6.43031) rectangle (16.5274,6.53615);
\draw [color=c, fill=c] (16.5274,6.43031) rectangle (16.5672,6.53615);
\draw [color=c, fill=c] (16.5672,6.43031) rectangle (16.607,6.53615);
\draw [color=c, fill=c] (16.607,6.43031) rectangle (16.6468,6.53615);
\draw [color=c, fill=c] (16.6468,6.43031) rectangle (16.6866,6.53615);
\draw [color=c, fill=c] (16.6866,6.43031) rectangle (16.7264,6.53615);
\draw [color=c, fill=c] (16.7264,6.43031) rectangle (16.7662,6.53615);
\draw [color=c, fill=c] (16.7662,6.43031) rectangle (16.806,6.53615);
\draw [color=c, fill=c] (16.806,6.43031) rectangle (16.8458,6.53615);
\draw [color=c, fill=c] (16.8458,6.43031) rectangle (16.8856,6.53615);
\draw [color=c, fill=c] (16.8856,6.43031) rectangle (16.9254,6.53615);
\draw [color=c, fill=c] (16.9254,6.43031) rectangle (16.9652,6.53615);
\draw [color=c, fill=c] (16.9652,6.43031) rectangle (17.005,6.53615);
\draw [color=c, fill=c] (17.005,6.43031) rectangle (17.0448,6.53615);
\draw [color=c, fill=c] (17.0448,6.43031) rectangle (17.0846,6.53615);
\draw [color=c, fill=c] (17.0846,6.43031) rectangle (17.1244,6.53615);
\draw [color=c, fill=c] (17.1244,6.43031) rectangle (17.1642,6.53615);
\draw [color=c, fill=c] (17.1642,6.43031) rectangle (17.204,6.53615);
\draw [color=c, fill=c] (17.204,6.43031) rectangle (17.2438,6.53615);
\draw [color=c, fill=c] (17.2438,6.43031) rectangle (17.2836,6.53615);
\draw [color=c, fill=c] (17.2836,6.43031) rectangle (17.3234,6.53615);
\draw [color=c, fill=c] (17.3234,6.43031) rectangle (17.3632,6.53615);
\draw [color=c, fill=c] (17.3632,6.43031) rectangle (17.403,6.53615);
\draw [color=c, fill=c] (17.403,6.43031) rectangle (17.4428,6.53615);
\draw [color=c, fill=c] (17.4428,6.43031) rectangle (17.4826,6.53615);
\draw [color=c, fill=c] (17.4826,6.43031) rectangle (17.5224,6.53615);
\draw [color=c, fill=c] (17.5224,6.43031) rectangle (17.5622,6.53615);
\draw [color=c, fill=c] (17.5622,6.43031) rectangle (17.602,6.53615);
\draw [color=c, fill=c] (17.602,6.43031) rectangle (17.6418,6.53615);
\draw [color=c, fill=c] (17.6418,6.43031) rectangle (17.6816,6.53615);
\draw [color=c, fill=c] (17.6816,6.43031) rectangle (17.7214,6.53615);
\draw [color=c, fill=c] (17.7214,6.43031) rectangle (17.7612,6.53615);
\draw [color=c, fill=c] (17.7612,6.43031) rectangle (17.801,6.53615);
\draw [color=c, fill=c] (17.801,6.43031) rectangle (17.8408,6.53615);
\draw [color=c, fill=c] (17.8408,6.43031) rectangle (17.8806,6.53615);
\draw [color=c, fill=c] (17.8806,6.43031) rectangle (17.9204,6.53615);
\draw [color=c, fill=c] (17.9204,6.43031) rectangle (17.9602,6.53615);
\draw [color=c, fill=c] (17.9602,6.43031) rectangle (18,6.53615);
\definecolor{c}{rgb}{0,0.0800001,1};
\draw [color=c, fill=c] (2,6.53615) rectangle (2.0398,6.642);
\draw [color=c, fill=c] (2.0398,6.53615) rectangle (2.0796,6.642);
\draw [color=c, fill=c] (2.0796,6.53615) rectangle (2.1194,6.642);
\draw [color=c, fill=c] (2.1194,6.53615) rectangle (2.1592,6.642);
\draw [color=c, fill=c] (2.1592,6.53615) rectangle (2.19901,6.642);
\draw [color=c, fill=c] (2.19901,6.53615) rectangle (2.23881,6.642);
\draw [color=c, fill=c] (2.23881,6.53615) rectangle (2.27861,6.642);
\draw [color=c, fill=c] (2.27861,6.53615) rectangle (2.31841,6.642);
\draw [color=c, fill=c] (2.31841,6.53615) rectangle (2.35821,6.642);
\draw [color=c, fill=c] (2.35821,6.53615) rectangle (2.39801,6.642);
\draw [color=c, fill=c] (2.39801,6.53615) rectangle (2.43781,6.642);
\draw [color=c, fill=c] (2.43781,6.53615) rectangle (2.47761,6.642);
\draw [color=c, fill=c] (2.47761,6.53615) rectangle (2.51741,6.642);
\draw [color=c, fill=c] (2.51741,6.53615) rectangle (2.55721,6.642);
\draw [color=c, fill=c] (2.55721,6.53615) rectangle (2.59702,6.642);
\draw [color=c, fill=c] (2.59702,6.53615) rectangle (2.63682,6.642);
\draw [color=c, fill=c] (2.63682,6.53615) rectangle (2.67662,6.642);
\draw [color=c, fill=c] (2.67662,6.53615) rectangle (2.71642,6.642);
\draw [color=c, fill=c] (2.71642,6.53615) rectangle (2.75622,6.642);
\draw [color=c, fill=c] (2.75622,6.53615) rectangle (2.79602,6.642);
\draw [color=c, fill=c] (2.79602,6.53615) rectangle (2.83582,6.642);
\draw [color=c, fill=c] (2.83582,6.53615) rectangle (2.87562,6.642);
\draw [color=c, fill=c] (2.87562,6.53615) rectangle (2.91542,6.642);
\draw [color=c, fill=c] (2.91542,6.53615) rectangle (2.95522,6.642);
\draw [color=c, fill=c] (2.95522,6.53615) rectangle (2.99502,6.642);
\draw [color=c, fill=c] (2.99502,6.53615) rectangle (3.03483,6.642);
\draw [color=c, fill=c] (3.03483,6.53615) rectangle (3.07463,6.642);
\draw [color=c, fill=c] (3.07463,6.53615) rectangle (3.11443,6.642);
\draw [color=c, fill=c] (3.11443,6.53615) rectangle (3.15423,6.642);
\draw [color=c, fill=c] (3.15423,6.53615) rectangle (3.19403,6.642);
\draw [color=c, fill=c] (3.19403,6.53615) rectangle (3.23383,6.642);
\draw [color=c, fill=c] (3.23383,6.53615) rectangle (3.27363,6.642);
\draw [color=c, fill=c] (3.27363,6.53615) rectangle (3.31343,6.642);
\draw [color=c, fill=c] (3.31343,6.53615) rectangle (3.35323,6.642);
\draw [color=c, fill=c] (3.35323,6.53615) rectangle (3.39303,6.642);
\draw [color=c, fill=c] (3.39303,6.53615) rectangle (3.43284,6.642);
\draw [color=c, fill=c] (3.43284,6.53615) rectangle (3.47264,6.642);
\draw [color=c, fill=c] (3.47264,6.53615) rectangle (3.51244,6.642);
\draw [color=c, fill=c] (3.51244,6.53615) rectangle (3.55224,6.642);
\draw [color=c, fill=c] (3.55224,6.53615) rectangle (3.59204,6.642);
\draw [color=c, fill=c] (3.59204,6.53615) rectangle (3.63184,6.642);
\draw [color=c, fill=c] (3.63184,6.53615) rectangle (3.67164,6.642);
\draw [color=c, fill=c] (3.67164,6.53615) rectangle (3.71144,6.642);
\draw [color=c, fill=c] (3.71144,6.53615) rectangle (3.75124,6.642);
\draw [color=c, fill=c] (3.75124,6.53615) rectangle (3.79104,6.642);
\draw [color=c, fill=c] (3.79104,6.53615) rectangle (3.83085,6.642);
\draw [color=c, fill=c] (3.83085,6.53615) rectangle (3.87065,6.642);
\draw [color=c, fill=c] (3.87065,6.53615) rectangle (3.91045,6.642);
\draw [color=c, fill=c] (3.91045,6.53615) rectangle (3.95025,6.642);
\draw [color=c, fill=c] (3.95025,6.53615) rectangle (3.99005,6.642);
\draw [color=c, fill=c] (3.99005,6.53615) rectangle (4.02985,6.642);
\draw [color=c, fill=c] (4.02985,6.53615) rectangle (4.06965,6.642);
\draw [color=c, fill=c] (4.06965,6.53615) rectangle (4.10945,6.642);
\draw [color=c, fill=c] (4.10945,6.53615) rectangle (4.14925,6.642);
\draw [color=c, fill=c] (4.14925,6.53615) rectangle (4.18905,6.642);
\draw [color=c, fill=c] (4.18905,6.53615) rectangle (4.22886,6.642);
\draw [color=c, fill=c] (4.22886,6.53615) rectangle (4.26866,6.642);
\draw [color=c, fill=c] (4.26866,6.53615) rectangle (4.30846,6.642);
\draw [color=c, fill=c] (4.30846,6.53615) rectangle (4.34826,6.642);
\draw [color=c, fill=c] (4.34826,6.53615) rectangle (4.38806,6.642);
\draw [color=c, fill=c] (4.38806,6.53615) rectangle (4.42786,6.642);
\draw [color=c, fill=c] (4.42786,6.53615) rectangle (4.46766,6.642);
\draw [color=c, fill=c] (4.46766,6.53615) rectangle (4.50746,6.642);
\draw [color=c, fill=c] (4.50746,6.53615) rectangle (4.54726,6.642);
\draw [color=c, fill=c] (4.54726,6.53615) rectangle (4.58706,6.642);
\definecolor{c}{rgb}{0.2,0,1};
\draw [color=c, fill=c] (4.58706,6.53615) rectangle (4.62687,6.642);
\draw [color=c, fill=c] (4.62687,6.53615) rectangle (4.66667,6.642);
\draw [color=c, fill=c] (4.66667,6.53615) rectangle (4.70647,6.642);
\draw [color=c, fill=c] (4.70647,6.53615) rectangle (4.74627,6.642);
\draw [color=c, fill=c] (4.74627,6.53615) rectangle (4.78607,6.642);
\draw [color=c, fill=c] (4.78607,6.53615) rectangle (4.82587,6.642);
\draw [color=c, fill=c] (4.82587,6.53615) rectangle (4.86567,6.642);
\draw [color=c, fill=c] (4.86567,6.53615) rectangle (4.90547,6.642);
\draw [color=c, fill=c] (4.90547,6.53615) rectangle (4.94527,6.642);
\draw [color=c, fill=c] (4.94527,6.53615) rectangle (4.98507,6.642);
\draw [color=c, fill=c] (4.98507,6.53615) rectangle (5.02488,6.642);
\draw [color=c, fill=c] (5.02488,6.53615) rectangle (5.06468,6.642);
\draw [color=c, fill=c] (5.06468,6.53615) rectangle (5.10448,6.642);
\draw [color=c, fill=c] (5.10448,6.53615) rectangle (5.14428,6.642);
\draw [color=c, fill=c] (5.14428,6.53615) rectangle (5.18408,6.642);
\draw [color=c, fill=c] (5.18408,6.53615) rectangle (5.22388,6.642);
\draw [color=c, fill=c] (5.22388,6.53615) rectangle (5.26368,6.642);
\draw [color=c, fill=c] (5.26368,6.53615) rectangle (5.30348,6.642);
\draw [color=c, fill=c] (5.30348,6.53615) rectangle (5.34328,6.642);
\draw [color=c, fill=c] (5.34328,6.53615) rectangle (5.38308,6.642);
\draw [color=c, fill=c] (5.38308,6.53615) rectangle (5.42289,6.642);
\draw [color=c, fill=c] (5.42289,6.53615) rectangle (5.46269,6.642);
\draw [color=c, fill=c] (5.46269,6.53615) rectangle (5.50249,6.642);
\draw [color=c, fill=c] (5.50249,6.53615) rectangle (5.54229,6.642);
\draw [color=c, fill=c] (5.54229,6.53615) rectangle (5.58209,6.642);
\draw [color=c, fill=c] (5.58209,6.53615) rectangle (5.62189,6.642);
\draw [color=c, fill=c] (5.62189,6.53615) rectangle (5.66169,6.642);
\draw [color=c, fill=c] (5.66169,6.53615) rectangle (5.70149,6.642);
\draw [color=c, fill=c] (5.70149,6.53615) rectangle (5.74129,6.642);
\draw [color=c, fill=c] (5.74129,6.53615) rectangle (5.78109,6.642);
\draw [color=c, fill=c] (5.78109,6.53615) rectangle (5.8209,6.642);
\draw [color=c, fill=c] (5.8209,6.53615) rectangle (5.8607,6.642);
\draw [color=c, fill=c] (5.8607,6.53615) rectangle (5.9005,6.642);
\draw [color=c, fill=c] (5.9005,6.53615) rectangle (5.9403,6.642);
\draw [color=c, fill=c] (5.9403,6.53615) rectangle (5.9801,6.642);
\draw [color=c, fill=c] (5.9801,6.53615) rectangle (6.0199,6.642);
\draw [color=c, fill=c] (6.0199,6.53615) rectangle (6.0597,6.642);
\draw [color=c, fill=c] (6.0597,6.53615) rectangle (6.0995,6.642);
\draw [color=c, fill=c] (6.0995,6.53615) rectangle (6.1393,6.642);
\draw [color=c, fill=c] (6.1393,6.53615) rectangle (6.1791,6.642);
\draw [color=c, fill=c] (6.1791,6.53615) rectangle (6.21891,6.642);
\draw [color=c, fill=c] (6.21891,6.53615) rectangle (6.25871,6.642);
\draw [color=c, fill=c] (6.25871,6.53615) rectangle (6.29851,6.642);
\draw [color=c, fill=c] (6.29851,6.53615) rectangle (6.33831,6.642);
\draw [color=c, fill=c] (6.33831,6.53615) rectangle (6.37811,6.642);
\draw [color=c, fill=c] (6.37811,6.53615) rectangle (6.41791,6.642);
\draw [color=c, fill=c] (6.41791,6.53615) rectangle (6.45771,6.642);
\draw [color=c, fill=c] (6.45771,6.53615) rectangle (6.49751,6.642);
\draw [color=c, fill=c] (6.49751,6.53615) rectangle (6.53731,6.642);
\draw [color=c, fill=c] (6.53731,6.53615) rectangle (6.57711,6.642);
\draw [color=c, fill=c] (6.57711,6.53615) rectangle (6.61692,6.642);
\draw [color=c, fill=c] (6.61692,6.53615) rectangle (6.65672,6.642);
\draw [color=c, fill=c] (6.65672,6.53615) rectangle (6.69652,6.642);
\draw [color=c, fill=c] (6.69652,6.53615) rectangle (6.73632,6.642);
\draw [color=c, fill=c] (6.73632,6.53615) rectangle (6.77612,6.642);
\draw [color=c, fill=c] (6.77612,6.53615) rectangle (6.81592,6.642);
\draw [color=c, fill=c] (6.81592,6.53615) rectangle (6.85572,6.642);
\draw [color=c, fill=c] (6.85572,6.53615) rectangle (6.89552,6.642);
\draw [color=c, fill=c] (6.89552,6.53615) rectangle (6.93532,6.642);
\draw [color=c, fill=c] (6.93532,6.53615) rectangle (6.97512,6.642);
\draw [color=c, fill=c] (6.97512,6.53615) rectangle (7.01493,6.642);
\draw [color=c, fill=c] (7.01493,6.53615) rectangle (7.05473,6.642);
\draw [color=c, fill=c] (7.05473,6.53615) rectangle (7.09453,6.642);
\draw [color=c, fill=c] (7.09453,6.53615) rectangle (7.13433,6.642);
\draw [color=c, fill=c] (7.13433,6.53615) rectangle (7.17413,6.642);
\draw [color=c, fill=c] (7.17413,6.53615) rectangle (7.21393,6.642);
\draw [color=c, fill=c] (7.21393,6.53615) rectangle (7.25373,6.642);
\draw [color=c, fill=c] (7.25373,6.53615) rectangle (7.29353,6.642);
\draw [color=c, fill=c] (7.29353,6.53615) rectangle (7.33333,6.642);
\draw [color=c, fill=c] (7.33333,6.53615) rectangle (7.37313,6.642);
\draw [color=c, fill=c] (7.37313,6.53615) rectangle (7.41294,6.642);
\draw [color=c, fill=c] (7.41294,6.53615) rectangle (7.45274,6.642);
\draw [color=c, fill=c] (7.45274,6.53615) rectangle (7.49254,6.642);
\draw [color=c, fill=c] (7.49254,6.53615) rectangle (7.53234,6.642);
\draw [color=c, fill=c] (7.53234,6.53615) rectangle (7.57214,6.642);
\draw [color=c, fill=c] (7.57214,6.53615) rectangle (7.61194,6.642);
\draw [color=c, fill=c] (7.61194,6.53615) rectangle (7.65174,6.642);
\draw [color=c, fill=c] (7.65174,6.53615) rectangle (7.69154,6.642);
\draw [color=c, fill=c] (7.69154,6.53615) rectangle (7.73134,6.642);
\draw [color=c, fill=c] (7.73134,6.53615) rectangle (7.77114,6.642);
\draw [color=c, fill=c] (7.77114,6.53615) rectangle (7.81095,6.642);
\draw [color=c, fill=c] (7.81095,6.53615) rectangle (7.85075,6.642);
\draw [color=c, fill=c] (7.85075,6.53615) rectangle (7.89055,6.642);
\draw [color=c, fill=c] (7.89055,6.53615) rectangle (7.93035,6.642);
\draw [color=c, fill=c] (7.93035,6.53615) rectangle (7.97015,6.642);
\draw [color=c, fill=c] (7.97015,6.53615) rectangle (8.00995,6.642);
\draw [color=c, fill=c] (8.00995,6.53615) rectangle (8.04975,6.642);
\draw [color=c, fill=c] (8.04975,6.53615) rectangle (8.08955,6.642);
\draw [color=c, fill=c] (8.08955,6.53615) rectangle (8.12935,6.642);
\draw [color=c, fill=c] (8.12935,6.53615) rectangle (8.16915,6.642);
\draw [color=c, fill=c] (8.16915,6.53615) rectangle (8.20895,6.642);
\draw [color=c, fill=c] (8.20895,6.53615) rectangle (8.24876,6.642);
\draw [color=c, fill=c] (8.24876,6.53615) rectangle (8.28856,6.642);
\draw [color=c, fill=c] (8.28856,6.53615) rectangle (8.32836,6.642);
\draw [color=c, fill=c] (8.32836,6.53615) rectangle (8.36816,6.642);
\draw [color=c, fill=c] (8.36816,6.53615) rectangle (8.40796,6.642);
\draw [color=c, fill=c] (8.40796,6.53615) rectangle (8.44776,6.642);
\draw [color=c, fill=c] (8.44776,6.53615) rectangle (8.48756,6.642);
\draw [color=c, fill=c] (8.48756,6.53615) rectangle (8.52736,6.642);
\draw [color=c, fill=c] (8.52736,6.53615) rectangle (8.56716,6.642);
\definecolor{c}{rgb}{0,0.0800001,1};
\draw [color=c, fill=c] (8.56716,6.53615) rectangle (8.60697,6.642);
\draw [color=c, fill=c] (8.60697,6.53615) rectangle (8.64677,6.642);
\draw [color=c, fill=c] (8.64677,6.53615) rectangle (8.68657,6.642);
\draw [color=c, fill=c] (8.68657,6.53615) rectangle (8.72637,6.642);
\draw [color=c, fill=c] (8.72637,6.53615) rectangle (8.76617,6.642);
\draw [color=c, fill=c] (8.76617,6.53615) rectangle (8.80597,6.642);
\draw [color=c, fill=c] (8.80597,6.53615) rectangle (8.84577,6.642);
\draw [color=c, fill=c] (8.84577,6.53615) rectangle (8.88557,6.642);
\draw [color=c, fill=c] (8.88557,6.53615) rectangle (8.92537,6.642);
\draw [color=c, fill=c] (8.92537,6.53615) rectangle (8.96517,6.642);
\draw [color=c, fill=c] (8.96517,6.53615) rectangle (9.00498,6.642);
\draw [color=c, fill=c] (9.00498,6.53615) rectangle (9.04478,6.642);
\draw [color=c, fill=c] (9.04478,6.53615) rectangle (9.08458,6.642);
\draw [color=c, fill=c] (9.08458,6.53615) rectangle (9.12438,6.642);
\draw [color=c, fill=c] (9.12438,6.53615) rectangle (9.16418,6.642);
\draw [color=c, fill=c] (9.16418,6.53615) rectangle (9.20398,6.642);
\draw [color=c, fill=c] (9.20398,6.53615) rectangle (9.24378,6.642);
\draw [color=c, fill=c] (9.24378,6.53615) rectangle (9.28358,6.642);
\draw [color=c, fill=c] (9.28358,6.53615) rectangle (9.32338,6.642);
\draw [color=c, fill=c] (9.32338,6.53615) rectangle (9.36318,6.642);
\draw [color=c, fill=c] (9.36318,6.53615) rectangle (9.40298,6.642);
\draw [color=c, fill=c] (9.40298,6.53615) rectangle (9.44279,6.642);
\draw [color=c, fill=c] (9.44279,6.53615) rectangle (9.48259,6.642);
\draw [color=c, fill=c] (9.48259,6.53615) rectangle (9.52239,6.642);
\draw [color=c, fill=c] (9.52239,6.53615) rectangle (9.56219,6.642);
\draw [color=c, fill=c] (9.56219,6.53615) rectangle (9.60199,6.642);
\draw [color=c, fill=c] (9.60199,6.53615) rectangle (9.64179,6.642);
\definecolor{c}{rgb}{0,0.266667,1};
\draw [color=c, fill=c] (9.64179,6.53615) rectangle (9.68159,6.642);
\draw [color=c, fill=c] (9.68159,6.53615) rectangle (9.72139,6.642);
\draw [color=c, fill=c] (9.72139,6.53615) rectangle (9.76119,6.642);
\draw [color=c, fill=c] (9.76119,6.53615) rectangle (9.80099,6.642);
\draw [color=c, fill=c] (9.80099,6.53615) rectangle (9.8408,6.642);
\draw [color=c, fill=c] (9.8408,6.53615) rectangle (9.8806,6.642);
\draw [color=c, fill=c] (9.8806,6.53615) rectangle (9.9204,6.642);
\draw [color=c, fill=c] (9.9204,6.53615) rectangle (9.9602,6.642);
\draw [color=c, fill=c] (9.9602,6.53615) rectangle (10,6.642);
\draw [color=c, fill=c] (10,6.53615) rectangle (10.0398,6.642);
\draw [color=c, fill=c] (10.0398,6.53615) rectangle (10.0796,6.642);
\draw [color=c, fill=c] (10.0796,6.53615) rectangle (10.1194,6.642);
\draw [color=c, fill=c] (10.1194,6.53615) rectangle (10.1592,6.642);
\draw [color=c, fill=c] (10.1592,6.53615) rectangle (10.199,6.642);
\definecolor{c}{rgb}{0,0.546666,1};
\draw [color=c, fill=c] (10.199,6.53615) rectangle (10.2388,6.642);
\draw [color=c, fill=c] (10.2388,6.53615) rectangle (10.2786,6.642);
\draw [color=c, fill=c] (10.2786,6.53615) rectangle (10.3184,6.642);
\draw [color=c, fill=c] (10.3184,6.53615) rectangle (10.3582,6.642);
\draw [color=c, fill=c] (10.3582,6.53615) rectangle (10.398,6.642);
\draw [color=c, fill=c] (10.398,6.53615) rectangle (10.4378,6.642);
\draw [color=c, fill=c] (10.4378,6.53615) rectangle (10.4776,6.642);
\draw [color=c, fill=c] (10.4776,6.53615) rectangle (10.5174,6.642);
\draw [color=c, fill=c] (10.5174,6.53615) rectangle (10.5572,6.642);
\draw [color=c, fill=c] (10.5572,6.53615) rectangle (10.597,6.642);
\draw [color=c, fill=c] (10.597,6.53615) rectangle (10.6368,6.642);
\draw [color=c, fill=c] (10.6368,6.53615) rectangle (10.6766,6.642);
\draw [color=c, fill=c] (10.6766,6.53615) rectangle (10.7164,6.642);
\draw [color=c, fill=c] (10.7164,6.53615) rectangle (10.7562,6.642);
\draw [color=c, fill=c] (10.7562,6.53615) rectangle (10.796,6.642);
\draw [color=c, fill=c] (10.796,6.53615) rectangle (10.8358,6.642);
\draw [color=c, fill=c] (10.8358,6.53615) rectangle (10.8756,6.642);
\draw [color=c, fill=c] (10.8756,6.53615) rectangle (10.9154,6.642);
\draw [color=c, fill=c] (10.9154,6.53615) rectangle (10.9552,6.642);
\draw [color=c, fill=c] (10.9552,6.53615) rectangle (10.995,6.642);
\draw [color=c, fill=c] (10.995,6.53615) rectangle (11.0348,6.642);
\definecolor{c}{rgb}{0,0.733333,1};
\draw [color=c, fill=c] (11.0348,6.53615) rectangle (11.0746,6.642);
\draw [color=c, fill=c] (11.0746,6.53615) rectangle (11.1144,6.642);
\draw [color=c, fill=c] (11.1144,6.53615) rectangle (11.1542,6.642);
\draw [color=c, fill=c] (11.1542,6.53615) rectangle (11.194,6.642);
\draw [color=c, fill=c] (11.194,6.53615) rectangle (11.2338,6.642);
\draw [color=c, fill=c] (11.2338,6.53615) rectangle (11.2736,6.642);
\draw [color=c, fill=c] (11.2736,6.53615) rectangle (11.3134,6.642);
\draw [color=c, fill=c] (11.3134,6.53615) rectangle (11.3532,6.642);
\draw [color=c, fill=c] (11.3532,6.53615) rectangle (11.393,6.642);
\draw [color=c, fill=c] (11.393,6.53615) rectangle (11.4328,6.642);
\draw [color=c, fill=c] (11.4328,6.53615) rectangle (11.4726,6.642);
\draw [color=c, fill=c] (11.4726,6.53615) rectangle (11.5124,6.642);
\draw [color=c, fill=c] (11.5124,6.53615) rectangle (11.5522,6.642);
\draw [color=c, fill=c] (11.5522,6.53615) rectangle (11.592,6.642);
\draw [color=c, fill=c] (11.592,6.53615) rectangle (11.6318,6.642);
\draw [color=c, fill=c] (11.6318,6.53615) rectangle (11.6716,6.642);
\draw [color=c, fill=c] (11.6716,6.53615) rectangle (11.7114,6.642);
\draw [color=c, fill=c] (11.7114,6.53615) rectangle (11.7512,6.642);
\draw [color=c, fill=c] (11.7512,6.53615) rectangle (11.791,6.642);
\draw [color=c, fill=c] (11.791,6.53615) rectangle (11.8308,6.642);
\draw [color=c, fill=c] (11.8308,6.53615) rectangle (11.8706,6.642);
\draw [color=c, fill=c] (11.8706,6.53615) rectangle (11.9104,6.642);
\draw [color=c, fill=c] (11.9104,6.53615) rectangle (11.9502,6.642);
\draw [color=c, fill=c] (11.9502,6.53615) rectangle (11.99,6.642);
\draw [color=c, fill=c] (11.99,6.53615) rectangle (12.0299,6.642);
\draw [color=c, fill=c] (12.0299,6.53615) rectangle (12.0697,6.642);
\draw [color=c, fill=c] (12.0697,6.53615) rectangle (12.1095,6.642);
\draw [color=c, fill=c] (12.1095,6.53615) rectangle (12.1493,6.642);
\draw [color=c, fill=c] (12.1493,6.53615) rectangle (12.1891,6.642);
\draw [color=c, fill=c] (12.1891,6.53615) rectangle (12.2289,6.642);
\draw [color=c, fill=c] (12.2289,6.53615) rectangle (12.2687,6.642);
\draw [color=c, fill=c] (12.2687,6.53615) rectangle (12.3085,6.642);
\draw [color=c, fill=c] (12.3085,6.53615) rectangle (12.3483,6.642);
\draw [color=c, fill=c] (12.3483,6.53615) rectangle (12.3881,6.642);
\draw [color=c, fill=c] (12.3881,6.53615) rectangle (12.4279,6.642);
\draw [color=c, fill=c] (12.4279,6.53615) rectangle (12.4677,6.642);
\draw [color=c, fill=c] (12.4677,6.53615) rectangle (12.5075,6.642);
\draw [color=c, fill=c] (12.5075,6.53615) rectangle (12.5473,6.642);
\draw [color=c, fill=c] (12.5473,6.53615) rectangle (12.5871,6.642);
\draw [color=c, fill=c] (12.5871,6.53615) rectangle (12.6269,6.642);
\draw [color=c, fill=c] (12.6269,6.53615) rectangle (12.6667,6.642);
\draw [color=c, fill=c] (12.6667,6.53615) rectangle (12.7065,6.642);
\draw [color=c, fill=c] (12.7065,6.53615) rectangle (12.7463,6.642);
\draw [color=c, fill=c] (12.7463,6.53615) rectangle (12.7861,6.642);
\draw [color=c, fill=c] (12.7861,6.53615) rectangle (12.8259,6.642);
\draw [color=c, fill=c] (12.8259,6.53615) rectangle (12.8657,6.642);
\draw [color=c, fill=c] (12.8657,6.53615) rectangle (12.9055,6.642);
\draw [color=c, fill=c] (12.9055,6.53615) rectangle (12.9453,6.642);
\draw [color=c, fill=c] (12.9453,6.53615) rectangle (12.9851,6.642);
\draw [color=c, fill=c] (12.9851,6.53615) rectangle (13.0249,6.642);
\draw [color=c, fill=c] (13.0249,6.53615) rectangle (13.0647,6.642);
\draw [color=c, fill=c] (13.0647,6.53615) rectangle (13.1045,6.642);
\draw [color=c, fill=c] (13.1045,6.53615) rectangle (13.1443,6.642);
\draw [color=c, fill=c] (13.1443,6.53615) rectangle (13.1841,6.642);
\draw [color=c, fill=c] (13.1841,6.53615) rectangle (13.2239,6.642);
\draw [color=c, fill=c] (13.2239,6.53615) rectangle (13.2637,6.642);
\draw [color=c, fill=c] (13.2637,6.53615) rectangle (13.3035,6.642);
\draw [color=c, fill=c] (13.3035,6.53615) rectangle (13.3433,6.642);
\draw [color=c, fill=c] (13.3433,6.53615) rectangle (13.3831,6.642);
\draw [color=c, fill=c] (13.3831,6.53615) rectangle (13.4229,6.642);
\draw [color=c, fill=c] (13.4229,6.53615) rectangle (13.4627,6.642);
\draw [color=c, fill=c] (13.4627,6.53615) rectangle (13.5025,6.642);
\draw [color=c, fill=c] (13.5025,6.53615) rectangle (13.5423,6.642);
\draw [color=c, fill=c] (13.5423,6.53615) rectangle (13.5821,6.642);
\draw [color=c, fill=c] (13.5821,6.53615) rectangle (13.6219,6.642);
\draw [color=c, fill=c] (13.6219,6.53615) rectangle (13.6617,6.642);
\draw [color=c, fill=c] (13.6617,6.53615) rectangle (13.7015,6.642);
\draw [color=c, fill=c] (13.7015,6.53615) rectangle (13.7413,6.642);
\draw [color=c, fill=c] (13.7413,6.53615) rectangle (13.7811,6.642);
\draw [color=c, fill=c] (13.7811,6.53615) rectangle (13.8209,6.642);
\draw [color=c, fill=c] (13.8209,6.53615) rectangle (13.8607,6.642);
\draw [color=c, fill=c] (13.8607,6.53615) rectangle (13.9005,6.642);
\draw [color=c, fill=c] (13.9005,6.53615) rectangle (13.9403,6.642);
\draw [color=c, fill=c] (13.9403,6.53615) rectangle (13.9801,6.642);
\draw [color=c, fill=c] (13.9801,6.53615) rectangle (14.0199,6.642);
\draw [color=c, fill=c] (14.0199,6.53615) rectangle (14.0597,6.642);
\draw [color=c, fill=c] (14.0597,6.53615) rectangle (14.0995,6.642);
\draw [color=c, fill=c] (14.0995,6.53615) rectangle (14.1393,6.642);
\draw [color=c, fill=c] (14.1393,6.53615) rectangle (14.1791,6.642);
\draw [color=c, fill=c] (14.1791,6.53615) rectangle (14.2189,6.642);
\draw [color=c, fill=c] (14.2189,6.53615) rectangle (14.2587,6.642);
\draw [color=c, fill=c] (14.2587,6.53615) rectangle (14.2985,6.642);
\draw [color=c, fill=c] (14.2985,6.53615) rectangle (14.3383,6.642);
\draw [color=c, fill=c] (14.3383,6.53615) rectangle (14.3781,6.642);
\draw [color=c, fill=c] (14.3781,6.53615) rectangle (14.4179,6.642);
\draw [color=c, fill=c] (14.4179,6.53615) rectangle (14.4577,6.642);
\draw [color=c, fill=c] (14.4577,6.53615) rectangle (14.4975,6.642);
\draw [color=c, fill=c] (14.4975,6.53615) rectangle (14.5373,6.642);
\draw [color=c, fill=c] (14.5373,6.53615) rectangle (14.5771,6.642);
\draw [color=c, fill=c] (14.5771,6.53615) rectangle (14.6169,6.642);
\draw [color=c, fill=c] (14.6169,6.53615) rectangle (14.6567,6.642);
\draw [color=c, fill=c] (14.6567,6.53615) rectangle (14.6965,6.642);
\draw [color=c, fill=c] (14.6965,6.53615) rectangle (14.7363,6.642);
\draw [color=c, fill=c] (14.7363,6.53615) rectangle (14.7761,6.642);
\draw [color=c, fill=c] (14.7761,6.53615) rectangle (14.8159,6.642);
\draw [color=c, fill=c] (14.8159,6.53615) rectangle (14.8557,6.642);
\draw [color=c, fill=c] (14.8557,6.53615) rectangle (14.8955,6.642);
\draw [color=c, fill=c] (14.8955,6.53615) rectangle (14.9353,6.642);
\draw [color=c, fill=c] (14.9353,6.53615) rectangle (14.9751,6.642);
\draw [color=c, fill=c] (14.9751,6.53615) rectangle (15.0149,6.642);
\draw [color=c, fill=c] (15.0149,6.53615) rectangle (15.0547,6.642);
\draw [color=c, fill=c] (15.0547,6.53615) rectangle (15.0945,6.642);
\draw [color=c, fill=c] (15.0945,6.53615) rectangle (15.1343,6.642);
\draw [color=c, fill=c] (15.1343,6.53615) rectangle (15.1741,6.642);
\draw [color=c, fill=c] (15.1741,6.53615) rectangle (15.2139,6.642);
\draw [color=c, fill=c] (15.2139,6.53615) rectangle (15.2537,6.642);
\draw [color=c, fill=c] (15.2537,6.53615) rectangle (15.2935,6.642);
\draw [color=c, fill=c] (15.2935,6.53615) rectangle (15.3333,6.642);
\draw [color=c, fill=c] (15.3333,6.53615) rectangle (15.3731,6.642);
\draw [color=c, fill=c] (15.3731,6.53615) rectangle (15.4129,6.642);
\draw [color=c, fill=c] (15.4129,6.53615) rectangle (15.4527,6.642);
\draw [color=c, fill=c] (15.4527,6.53615) rectangle (15.4925,6.642);
\draw [color=c, fill=c] (15.4925,6.53615) rectangle (15.5323,6.642);
\draw [color=c, fill=c] (15.5323,6.53615) rectangle (15.5721,6.642);
\draw [color=c, fill=c] (15.5721,6.53615) rectangle (15.6119,6.642);
\draw [color=c, fill=c] (15.6119,6.53615) rectangle (15.6517,6.642);
\draw [color=c, fill=c] (15.6517,6.53615) rectangle (15.6915,6.642);
\draw [color=c, fill=c] (15.6915,6.53615) rectangle (15.7313,6.642);
\draw [color=c, fill=c] (15.7313,6.53615) rectangle (15.7711,6.642);
\draw [color=c, fill=c] (15.7711,6.53615) rectangle (15.8109,6.642);
\draw [color=c, fill=c] (15.8109,6.53615) rectangle (15.8507,6.642);
\draw [color=c, fill=c] (15.8507,6.53615) rectangle (15.8905,6.642);
\draw [color=c, fill=c] (15.8905,6.53615) rectangle (15.9303,6.642);
\draw [color=c, fill=c] (15.9303,6.53615) rectangle (15.9701,6.642);
\draw [color=c, fill=c] (15.9701,6.53615) rectangle (16.01,6.642);
\draw [color=c, fill=c] (16.01,6.53615) rectangle (16.0498,6.642);
\draw [color=c, fill=c] (16.0498,6.53615) rectangle (16.0896,6.642);
\draw [color=c, fill=c] (16.0896,6.53615) rectangle (16.1294,6.642);
\draw [color=c, fill=c] (16.1294,6.53615) rectangle (16.1692,6.642);
\draw [color=c, fill=c] (16.1692,6.53615) rectangle (16.209,6.642);
\draw [color=c, fill=c] (16.209,6.53615) rectangle (16.2488,6.642);
\draw [color=c, fill=c] (16.2488,6.53615) rectangle (16.2886,6.642);
\draw [color=c, fill=c] (16.2886,6.53615) rectangle (16.3284,6.642);
\draw [color=c, fill=c] (16.3284,6.53615) rectangle (16.3682,6.642);
\draw [color=c, fill=c] (16.3682,6.53615) rectangle (16.408,6.642);
\draw [color=c, fill=c] (16.408,6.53615) rectangle (16.4478,6.642);
\draw [color=c, fill=c] (16.4478,6.53615) rectangle (16.4876,6.642);
\draw [color=c, fill=c] (16.4876,6.53615) rectangle (16.5274,6.642);
\draw [color=c, fill=c] (16.5274,6.53615) rectangle (16.5672,6.642);
\draw [color=c, fill=c] (16.5672,6.53615) rectangle (16.607,6.642);
\draw [color=c, fill=c] (16.607,6.53615) rectangle (16.6468,6.642);
\draw [color=c, fill=c] (16.6468,6.53615) rectangle (16.6866,6.642);
\draw [color=c, fill=c] (16.6866,6.53615) rectangle (16.7264,6.642);
\draw [color=c, fill=c] (16.7264,6.53615) rectangle (16.7662,6.642);
\draw [color=c, fill=c] (16.7662,6.53615) rectangle (16.806,6.642);
\draw [color=c, fill=c] (16.806,6.53615) rectangle (16.8458,6.642);
\draw [color=c, fill=c] (16.8458,6.53615) rectangle (16.8856,6.642);
\draw [color=c, fill=c] (16.8856,6.53615) rectangle (16.9254,6.642);
\draw [color=c, fill=c] (16.9254,6.53615) rectangle (16.9652,6.642);
\draw [color=c, fill=c] (16.9652,6.53615) rectangle (17.005,6.642);
\draw [color=c, fill=c] (17.005,6.53615) rectangle (17.0448,6.642);
\draw [color=c, fill=c] (17.0448,6.53615) rectangle (17.0846,6.642);
\draw [color=c, fill=c] (17.0846,6.53615) rectangle (17.1244,6.642);
\draw [color=c, fill=c] (17.1244,6.53615) rectangle (17.1642,6.642);
\draw [color=c, fill=c] (17.1642,6.53615) rectangle (17.204,6.642);
\draw [color=c, fill=c] (17.204,6.53615) rectangle (17.2438,6.642);
\draw [color=c, fill=c] (17.2438,6.53615) rectangle (17.2836,6.642);
\draw [color=c, fill=c] (17.2836,6.53615) rectangle (17.3234,6.642);
\draw [color=c, fill=c] (17.3234,6.53615) rectangle (17.3632,6.642);
\draw [color=c, fill=c] (17.3632,6.53615) rectangle (17.403,6.642);
\draw [color=c, fill=c] (17.403,6.53615) rectangle (17.4428,6.642);
\draw [color=c, fill=c] (17.4428,6.53615) rectangle (17.4826,6.642);
\draw [color=c, fill=c] (17.4826,6.53615) rectangle (17.5224,6.642);
\draw [color=c, fill=c] (17.5224,6.53615) rectangle (17.5622,6.642);
\draw [color=c, fill=c] (17.5622,6.53615) rectangle (17.602,6.642);
\draw [color=c, fill=c] (17.602,6.53615) rectangle (17.6418,6.642);
\draw [color=c, fill=c] (17.6418,6.53615) rectangle (17.6816,6.642);
\draw [color=c, fill=c] (17.6816,6.53615) rectangle (17.7214,6.642);
\draw [color=c, fill=c] (17.7214,6.53615) rectangle (17.7612,6.642);
\draw [color=c, fill=c] (17.7612,6.53615) rectangle (17.801,6.642);
\draw [color=c, fill=c] (17.801,6.53615) rectangle (17.8408,6.642);
\draw [color=c, fill=c] (17.8408,6.53615) rectangle (17.8806,6.642);
\draw [color=c, fill=c] (17.8806,6.53615) rectangle (17.9204,6.642);
\draw [color=c, fill=c] (17.9204,6.53615) rectangle (17.9602,6.642);
\draw [color=c, fill=c] (17.9602,6.53615) rectangle (18,6.642);
\definecolor{c}{rgb}{0,0.0800001,1};
\draw [color=c, fill=c] (2,6.642) rectangle (2.0398,6.74785);
\draw [color=c, fill=c] (2.0398,6.642) rectangle (2.0796,6.74785);
\draw [color=c, fill=c] (2.0796,6.642) rectangle (2.1194,6.74785);
\draw [color=c, fill=c] (2.1194,6.642) rectangle (2.1592,6.74785);
\draw [color=c, fill=c] (2.1592,6.642) rectangle (2.19901,6.74785);
\draw [color=c, fill=c] (2.19901,6.642) rectangle (2.23881,6.74785);
\draw [color=c, fill=c] (2.23881,6.642) rectangle (2.27861,6.74785);
\draw [color=c, fill=c] (2.27861,6.642) rectangle (2.31841,6.74785);
\draw [color=c, fill=c] (2.31841,6.642) rectangle (2.35821,6.74785);
\draw [color=c, fill=c] (2.35821,6.642) rectangle (2.39801,6.74785);
\draw [color=c, fill=c] (2.39801,6.642) rectangle (2.43781,6.74785);
\draw [color=c, fill=c] (2.43781,6.642) rectangle (2.47761,6.74785);
\draw [color=c, fill=c] (2.47761,6.642) rectangle (2.51741,6.74785);
\draw [color=c, fill=c] (2.51741,6.642) rectangle (2.55721,6.74785);
\draw [color=c, fill=c] (2.55721,6.642) rectangle (2.59702,6.74785);
\draw [color=c, fill=c] (2.59702,6.642) rectangle (2.63682,6.74785);
\draw [color=c, fill=c] (2.63682,6.642) rectangle (2.67662,6.74785);
\draw [color=c, fill=c] (2.67662,6.642) rectangle (2.71642,6.74785);
\draw [color=c, fill=c] (2.71642,6.642) rectangle (2.75622,6.74785);
\draw [color=c, fill=c] (2.75622,6.642) rectangle (2.79602,6.74785);
\draw [color=c, fill=c] (2.79602,6.642) rectangle (2.83582,6.74785);
\draw [color=c, fill=c] (2.83582,6.642) rectangle (2.87562,6.74785);
\draw [color=c, fill=c] (2.87562,6.642) rectangle (2.91542,6.74785);
\draw [color=c, fill=c] (2.91542,6.642) rectangle (2.95522,6.74785);
\draw [color=c, fill=c] (2.95522,6.642) rectangle (2.99502,6.74785);
\draw [color=c, fill=c] (2.99502,6.642) rectangle (3.03483,6.74785);
\draw [color=c, fill=c] (3.03483,6.642) rectangle (3.07463,6.74785);
\draw [color=c, fill=c] (3.07463,6.642) rectangle (3.11443,6.74785);
\draw [color=c, fill=c] (3.11443,6.642) rectangle (3.15423,6.74785);
\draw [color=c, fill=c] (3.15423,6.642) rectangle (3.19403,6.74785);
\draw [color=c, fill=c] (3.19403,6.642) rectangle (3.23383,6.74785);
\draw [color=c, fill=c] (3.23383,6.642) rectangle (3.27363,6.74785);
\draw [color=c, fill=c] (3.27363,6.642) rectangle (3.31343,6.74785);
\draw [color=c, fill=c] (3.31343,6.642) rectangle (3.35323,6.74785);
\draw [color=c, fill=c] (3.35323,6.642) rectangle (3.39303,6.74785);
\draw [color=c, fill=c] (3.39303,6.642) rectangle (3.43284,6.74785);
\draw [color=c, fill=c] (3.43284,6.642) rectangle (3.47264,6.74785);
\draw [color=c, fill=c] (3.47264,6.642) rectangle (3.51244,6.74785);
\draw [color=c, fill=c] (3.51244,6.642) rectangle (3.55224,6.74785);
\draw [color=c, fill=c] (3.55224,6.642) rectangle (3.59204,6.74785);
\draw [color=c, fill=c] (3.59204,6.642) rectangle (3.63184,6.74785);
\draw [color=c, fill=c] (3.63184,6.642) rectangle (3.67164,6.74785);
\draw [color=c, fill=c] (3.67164,6.642) rectangle (3.71144,6.74785);
\draw [color=c, fill=c] (3.71144,6.642) rectangle (3.75124,6.74785);
\draw [color=c, fill=c] (3.75124,6.642) rectangle (3.79104,6.74785);
\draw [color=c, fill=c] (3.79104,6.642) rectangle (3.83085,6.74785);
\draw [color=c, fill=c] (3.83085,6.642) rectangle (3.87065,6.74785);
\draw [color=c, fill=c] (3.87065,6.642) rectangle (3.91045,6.74785);
\draw [color=c, fill=c] (3.91045,6.642) rectangle (3.95025,6.74785);
\draw [color=c, fill=c] (3.95025,6.642) rectangle (3.99005,6.74785);
\draw [color=c, fill=c] (3.99005,6.642) rectangle (4.02985,6.74785);
\draw [color=c, fill=c] (4.02985,6.642) rectangle (4.06965,6.74785);
\draw [color=c, fill=c] (4.06965,6.642) rectangle (4.10945,6.74785);
\draw [color=c, fill=c] (4.10945,6.642) rectangle (4.14925,6.74785);
\draw [color=c, fill=c] (4.14925,6.642) rectangle (4.18905,6.74785);
\draw [color=c, fill=c] (4.18905,6.642) rectangle (4.22886,6.74785);
\draw [color=c, fill=c] (4.22886,6.642) rectangle (4.26866,6.74785);
\definecolor{c}{rgb}{0.2,0,1};
\draw [color=c, fill=c] (4.26866,6.642) rectangle (4.30846,6.74785);
\draw [color=c, fill=c] (4.30846,6.642) rectangle (4.34826,6.74785);
\draw [color=c, fill=c] (4.34826,6.642) rectangle (4.38806,6.74785);
\draw [color=c, fill=c] (4.38806,6.642) rectangle (4.42786,6.74785);
\draw [color=c, fill=c] (4.42786,6.642) rectangle (4.46766,6.74785);
\draw [color=c, fill=c] (4.46766,6.642) rectangle (4.50746,6.74785);
\draw [color=c, fill=c] (4.50746,6.642) rectangle (4.54726,6.74785);
\draw [color=c, fill=c] (4.54726,6.642) rectangle (4.58706,6.74785);
\draw [color=c, fill=c] (4.58706,6.642) rectangle (4.62687,6.74785);
\draw [color=c, fill=c] (4.62687,6.642) rectangle (4.66667,6.74785);
\draw [color=c, fill=c] (4.66667,6.642) rectangle (4.70647,6.74785);
\draw [color=c, fill=c] (4.70647,6.642) rectangle (4.74627,6.74785);
\draw [color=c, fill=c] (4.74627,6.642) rectangle (4.78607,6.74785);
\draw [color=c, fill=c] (4.78607,6.642) rectangle (4.82587,6.74785);
\draw [color=c, fill=c] (4.82587,6.642) rectangle (4.86567,6.74785);
\draw [color=c, fill=c] (4.86567,6.642) rectangle (4.90547,6.74785);
\draw [color=c, fill=c] (4.90547,6.642) rectangle (4.94527,6.74785);
\draw [color=c, fill=c] (4.94527,6.642) rectangle (4.98507,6.74785);
\draw [color=c, fill=c] (4.98507,6.642) rectangle (5.02488,6.74785);
\draw [color=c, fill=c] (5.02488,6.642) rectangle (5.06468,6.74785);
\draw [color=c, fill=c] (5.06468,6.642) rectangle (5.10448,6.74785);
\draw [color=c, fill=c] (5.10448,6.642) rectangle (5.14428,6.74785);
\draw [color=c, fill=c] (5.14428,6.642) rectangle (5.18408,6.74785);
\draw [color=c, fill=c] (5.18408,6.642) rectangle (5.22388,6.74785);
\draw [color=c, fill=c] (5.22388,6.642) rectangle (5.26368,6.74785);
\draw [color=c, fill=c] (5.26368,6.642) rectangle (5.30348,6.74785);
\draw [color=c, fill=c] (5.30348,6.642) rectangle (5.34328,6.74785);
\draw [color=c, fill=c] (5.34328,6.642) rectangle (5.38308,6.74785);
\draw [color=c, fill=c] (5.38308,6.642) rectangle (5.42289,6.74785);
\draw [color=c, fill=c] (5.42289,6.642) rectangle (5.46269,6.74785);
\draw [color=c, fill=c] (5.46269,6.642) rectangle (5.50249,6.74785);
\draw [color=c, fill=c] (5.50249,6.642) rectangle (5.54229,6.74785);
\draw [color=c, fill=c] (5.54229,6.642) rectangle (5.58209,6.74785);
\draw [color=c, fill=c] (5.58209,6.642) rectangle (5.62189,6.74785);
\draw [color=c, fill=c] (5.62189,6.642) rectangle (5.66169,6.74785);
\draw [color=c, fill=c] (5.66169,6.642) rectangle (5.70149,6.74785);
\draw [color=c, fill=c] (5.70149,6.642) rectangle (5.74129,6.74785);
\draw [color=c, fill=c] (5.74129,6.642) rectangle (5.78109,6.74785);
\draw [color=c, fill=c] (5.78109,6.642) rectangle (5.8209,6.74785);
\draw [color=c, fill=c] (5.8209,6.642) rectangle (5.8607,6.74785);
\draw [color=c, fill=c] (5.8607,6.642) rectangle (5.9005,6.74785);
\draw [color=c, fill=c] (5.9005,6.642) rectangle (5.9403,6.74785);
\draw [color=c, fill=c] (5.9403,6.642) rectangle (5.9801,6.74785);
\draw [color=c, fill=c] (5.9801,6.642) rectangle (6.0199,6.74785);
\draw [color=c, fill=c] (6.0199,6.642) rectangle (6.0597,6.74785);
\draw [color=c, fill=c] (6.0597,6.642) rectangle (6.0995,6.74785);
\draw [color=c, fill=c] (6.0995,6.642) rectangle (6.1393,6.74785);
\draw [color=c, fill=c] (6.1393,6.642) rectangle (6.1791,6.74785);
\draw [color=c, fill=c] (6.1791,6.642) rectangle (6.21891,6.74785);
\draw [color=c, fill=c] (6.21891,6.642) rectangle (6.25871,6.74785);
\draw [color=c, fill=c] (6.25871,6.642) rectangle (6.29851,6.74785);
\draw [color=c, fill=c] (6.29851,6.642) rectangle (6.33831,6.74785);
\draw [color=c, fill=c] (6.33831,6.642) rectangle (6.37811,6.74785);
\draw [color=c, fill=c] (6.37811,6.642) rectangle (6.41791,6.74785);
\draw [color=c, fill=c] (6.41791,6.642) rectangle (6.45771,6.74785);
\draw [color=c, fill=c] (6.45771,6.642) rectangle (6.49751,6.74785);
\draw [color=c, fill=c] (6.49751,6.642) rectangle (6.53731,6.74785);
\draw [color=c, fill=c] (6.53731,6.642) rectangle (6.57711,6.74785);
\draw [color=c, fill=c] (6.57711,6.642) rectangle (6.61692,6.74785);
\draw [color=c, fill=c] (6.61692,6.642) rectangle (6.65672,6.74785);
\draw [color=c, fill=c] (6.65672,6.642) rectangle (6.69652,6.74785);
\draw [color=c, fill=c] (6.69652,6.642) rectangle (6.73632,6.74785);
\draw [color=c, fill=c] (6.73632,6.642) rectangle (6.77612,6.74785);
\draw [color=c, fill=c] (6.77612,6.642) rectangle (6.81592,6.74785);
\draw [color=c, fill=c] (6.81592,6.642) rectangle (6.85572,6.74785);
\draw [color=c, fill=c] (6.85572,6.642) rectangle (6.89552,6.74785);
\draw [color=c, fill=c] (6.89552,6.642) rectangle (6.93532,6.74785);
\draw [color=c, fill=c] (6.93532,6.642) rectangle (6.97512,6.74785);
\draw [color=c, fill=c] (6.97512,6.642) rectangle (7.01493,6.74785);
\draw [color=c, fill=c] (7.01493,6.642) rectangle (7.05473,6.74785);
\draw [color=c, fill=c] (7.05473,6.642) rectangle (7.09453,6.74785);
\draw [color=c, fill=c] (7.09453,6.642) rectangle (7.13433,6.74785);
\draw [color=c, fill=c] (7.13433,6.642) rectangle (7.17413,6.74785);
\draw [color=c, fill=c] (7.17413,6.642) rectangle (7.21393,6.74785);
\draw [color=c, fill=c] (7.21393,6.642) rectangle (7.25373,6.74785);
\draw [color=c, fill=c] (7.25373,6.642) rectangle (7.29353,6.74785);
\draw [color=c, fill=c] (7.29353,6.642) rectangle (7.33333,6.74785);
\draw [color=c, fill=c] (7.33333,6.642) rectangle (7.37313,6.74785);
\draw [color=c, fill=c] (7.37313,6.642) rectangle (7.41294,6.74785);
\draw [color=c, fill=c] (7.41294,6.642) rectangle (7.45274,6.74785);
\draw [color=c, fill=c] (7.45274,6.642) rectangle (7.49254,6.74785);
\draw [color=c, fill=c] (7.49254,6.642) rectangle (7.53234,6.74785);
\draw [color=c, fill=c] (7.53234,6.642) rectangle (7.57214,6.74785);
\draw [color=c, fill=c] (7.57214,6.642) rectangle (7.61194,6.74785);
\draw [color=c, fill=c] (7.61194,6.642) rectangle (7.65174,6.74785);
\draw [color=c, fill=c] (7.65174,6.642) rectangle (7.69154,6.74785);
\draw [color=c, fill=c] (7.69154,6.642) rectangle (7.73134,6.74785);
\draw [color=c, fill=c] (7.73134,6.642) rectangle (7.77114,6.74785);
\draw [color=c, fill=c] (7.77114,6.642) rectangle (7.81095,6.74785);
\draw [color=c, fill=c] (7.81095,6.642) rectangle (7.85075,6.74785);
\draw [color=c, fill=c] (7.85075,6.642) rectangle (7.89055,6.74785);
\draw [color=c, fill=c] (7.89055,6.642) rectangle (7.93035,6.74785);
\draw [color=c, fill=c] (7.93035,6.642) rectangle (7.97015,6.74785);
\draw [color=c, fill=c] (7.97015,6.642) rectangle (8.00995,6.74785);
\draw [color=c, fill=c] (8.00995,6.642) rectangle (8.04975,6.74785);
\draw [color=c, fill=c] (8.04975,6.642) rectangle (8.08955,6.74785);
\draw [color=c, fill=c] (8.08955,6.642) rectangle (8.12935,6.74785);
\draw [color=c, fill=c] (8.12935,6.642) rectangle (8.16915,6.74785);
\draw [color=c, fill=c] (8.16915,6.642) rectangle (8.20895,6.74785);
\draw [color=c, fill=c] (8.20895,6.642) rectangle (8.24876,6.74785);
\draw [color=c, fill=c] (8.24876,6.642) rectangle (8.28856,6.74785);
\draw [color=c, fill=c] (8.28856,6.642) rectangle (8.32836,6.74785);
\draw [color=c, fill=c] (8.32836,6.642) rectangle (8.36816,6.74785);
\draw [color=c, fill=c] (8.36816,6.642) rectangle (8.40796,6.74785);
\draw [color=c, fill=c] (8.40796,6.642) rectangle (8.44776,6.74785);
\draw [color=c, fill=c] (8.44776,6.642) rectangle (8.48756,6.74785);
\definecolor{c}{rgb}{0,0.0800001,1};
\draw [color=c, fill=c] (8.48756,6.642) rectangle (8.52736,6.74785);
\draw [color=c, fill=c] (8.52736,6.642) rectangle (8.56716,6.74785);
\draw [color=c, fill=c] (8.56716,6.642) rectangle (8.60697,6.74785);
\draw [color=c, fill=c] (8.60697,6.642) rectangle (8.64677,6.74785);
\draw [color=c, fill=c] (8.64677,6.642) rectangle (8.68657,6.74785);
\draw [color=c, fill=c] (8.68657,6.642) rectangle (8.72637,6.74785);
\draw [color=c, fill=c] (8.72637,6.642) rectangle (8.76617,6.74785);
\draw [color=c, fill=c] (8.76617,6.642) rectangle (8.80597,6.74785);
\draw [color=c, fill=c] (8.80597,6.642) rectangle (8.84577,6.74785);
\draw [color=c, fill=c] (8.84577,6.642) rectangle (8.88557,6.74785);
\draw [color=c, fill=c] (8.88557,6.642) rectangle (8.92537,6.74785);
\draw [color=c, fill=c] (8.92537,6.642) rectangle (8.96517,6.74785);
\draw [color=c, fill=c] (8.96517,6.642) rectangle (9.00498,6.74785);
\draw [color=c, fill=c] (9.00498,6.642) rectangle (9.04478,6.74785);
\draw [color=c, fill=c] (9.04478,6.642) rectangle (9.08458,6.74785);
\draw [color=c, fill=c] (9.08458,6.642) rectangle (9.12438,6.74785);
\draw [color=c, fill=c] (9.12438,6.642) rectangle (9.16418,6.74785);
\draw [color=c, fill=c] (9.16418,6.642) rectangle (9.20398,6.74785);
\draw [color=c, fill=c] (9.20398,6.642) rectangle (9.24378,6.74785);
\draw [color=c, fill=c] (9.24378,6.642) rectangle (9.28358,6.74785);
\draw [color=c, fill=c] (9.28358,6.642) rectangle (9.32338,6.74785);
\draw [color=c, fill=c] (9.32338,6.642) rectangle (9.36318,6.74785);
\draw [color=c, fill=c] (9.36318,6.642) rectangle (9.40298,6.74785);
\draw [color=c, fill=c] (9.40298,6.642) rectangle (9.44279,6.74785);
\draw [color=c, fill=c] (9.44279,6.642) rectangle (9.48259,6.74785);
\draw [color=c, fill=c] (9.48259,6.642) rectangle (9.52239,6.74785);
\draw [color=c, fill=c] (9.52239,6.642) rectangle (9.56219,6.74785);
\draw [color=c, fill=c] (9.56219,6.642) rectangle (9.60199,6.74785);
\draw [color=c, fill=c] (9.60199,6.642) rectangle (9.64179,6.74785);
\definecolor{c}{rgb}{0,0.266667,1};
\draw [color=c, fill=c] (9.64179,6.642) rectangle (9.68159,6.74785);
\draw [color=c, fill=c] (9.68159,6.642) rectangle (9.72139,6.74785);
\draw [color=c, fill=c] (9.72139,6.642) rectangle (9.76119,6.74785);
\draw [color=c, fill=c] (9.76119,6.642) rectangle (9.80099,6.74785);
\draw [color=c, fill=c] (9.80099,6.642) rectangle (9.8408,6.74785);
\draw [color=c, fill=c] (9.8408,6.642) rectangle (9.8806,6.74785);
\draw [color=c, fill=c] (9.8806,6.642) rectangle (9.9204,6.74785);
\draw [color=c, fill=c] (9.9204,6.642) rectangle (9.9602,6.74785);
\draw [color=c, fill=c] (9.9602,6.642) rectangle (10,6.74785);
\draw [color=c, fill=c] (10,6.642) rectangle (10.0398,6.74785);
\draw [color=c, fill=c] (10.0398,6.642) rectangle (10.0796,6.74785);
\draw [color=c, fill=c] (10.0796,6.642) rectangle (10.1194,6.74785);
\draw [color=c, fill=c] (10.1194,6.642) rectangle (10.1592,6.74785);
\draw [color=c, fill=c] (10.1592,6.642) rectangle (10.199,6.74785);
\definecolor{c}{rgb}{0,0.546666,1};
\draw [color=c, fill=c] (10.199,6.642) rectangle (10.2388,6.74785);
\draw [color=c, fill=c] (10.2388,6.642) rectangle (10.2786,6.74785);
\draw [color=c, fill=c] (10.2786,6.642) rectangle (10.3184,6.74785);
\draw [color=c, fill=c] (10.3184,6.642) rectangle (10.3582,6.74785);
\draw [color=c, fill=c] (10.3582,6.642) rectangle (10.398,6.74785);
\draw [color=c, fill=c] (10.398,6.642) rectangle (10.4378,6.74785);
\draw [color=c, fill=c] (10.4378,6.642) rectangle (10.4776,6.74785);
\draw [color=c, fill=c] (10.4776,6.642) rectangle (10.5174,6.74785);
\draw [color=c, fill=c] (10.5174,6.642) rectangle (10.5572,6.74785);
\draw [color=c, fill=c] (10.5572,6.642) rectangle (10.597,6.74785);
\draw [color=c, fill=c] (10.597,6.642) rectangle (10.6368,6.74785);
\draw [color=c, fill=c] (10.6368,6.642) rectangle (10.6766,6.74785);
\draw [color=c, fill=c] (10.6766,6.642) rectangle (10.7164,6.74785);
\draw [color=c, fill=c] (10.7164,6.642) rectangle (10.7562,6.74785);
\draw [color=c, fill=c] (10.7562,6.642) rectangle (10.796,6.74785);
\draw [color=c, fill=c] (10.796,6.642) rectangle (10.8358,6.74785);
\draw [color=c, fill=c] (10.8358,6.642) rectangle (10.8756,6.74785);
\draw [color=c, fill=c] (10.8756,6.642) rectangle (10.9154,6.74785);
\draw [color=c, fill=c] (10.9154,6.642) rectangle (10.9552,6.74785);
\draw [color=c, fill=c] (10.9552,6.642) rectangle (10.995,6.74785);
\draw [color=c, fill=c] (10.995,6.642) rectangle (11.0348,6.74785);
\draw [color=c, fill=c] (11.0348,6.642) rectangle (11.0746,6.74785);
\draw [color=c, fill=c] (11.0746,6.642) rectangle (11.1144,6.74785);
\definecolor{c}{rgb}{0,0.733333,1};
\draw [color=c, fill=c] (11.1144,6.642) rectangle (11.1542,6.74785);
\draw [color=c, fill=c] (11.1542,6.642) rectangle (11.194,6.74785);
\draw [color=c, fill=c] (11.194,6.642) rectangle (11.2338,6.74785);
\draw [color=c, fill=c] (11.2338,6.642) rectangle (11.2736,6.74785);
\draw [color=c, fill=c] (11.2736,6.642) rectangle (11.3134,6.74785);
\draw [color=c, fill=c] (11.3134,6.642) rectangle (11.3532,6.74785);
\draw [color=c, fill=c] (11.3532,6.642) rectangle (11.393,6.74785);
\draw [color=c, fill=c] (11.393,6.642) rectangle (11.4328,6.74785);
\draw [color=c, fill=c] (11.4328,6.642) rectangle (11.4726,6.74785);
\draw [color=c, fill=c] (11.4726,6.642) rectangle (11.5124,6.74785);
\draw [color=c, fill=c] (11.5124,6.642) rectangle (11.5522,6.74785);
\draw [color=c, fill=c] (11.5522,6.642) rectangle (11.592,6.74785);
\draw [color=c, fill=c] (11.592,6.642) rectangle (11.6318,6.74785);
\draw [color=c, fill=c] (11.6318,6.642) rectangle (11.6716,6.74785);
\draw [color=c, fill=c] (11.6716,6.642) rectangle (11.7114,6.74785);
\draw [color=c, fill=c] (11.7114,6.642) rectangle (11.7512,6.74785);
\draw [color=c, fill=c] (11.7512,6.642) rectangle (11.791,6.74785);
\draw [color=c, fill=c] (11.791,6.642) rectangle (11.8308,6.74785);
\draw [color=c, fill=c] (11.8308,6.642) rectangle (11.8706,6.74785);
\draw [color=c, fill=c] (11.8706,6.642) rectangle (11.9104,6.74785);
\draw [color=c, fill=c] (11.9104,6.642) rectangle (11.9502,6.74785);
\draw [color=c, fill=c] (11.9502,6.642) rectangle (11.99,6.74785);
\draw [color=c, fill=c] (11.99,6.642) rectangle (12.0299,6.74785);
\draw [color=c, fill=c] (12.0299,6.642) rectangle (12.0697,6.74785);
\draw [color=c, fill=c] (12.0697,6.642) rectangle (12.1095,6.74785);
\draw [color=c, fill=c] (12.1095,6.642) rectangle (12.1493,6.74785);
\draw [color=c, fill=c] (12.1493,6.642) rectangle (12.1891,6.74785);
\draw [color=c, fill=c] (12.1891,6.642) rectangle (12.2289,6.74785);
\draw [color=c, fill=c] (12.2289,6.642) rectangle (12.2687,6.74785);
\draw [color=c, fill=c] (12.2687,6.642) rectangle (12.3085,6.74785);
\draw [color=c, fill=c] (12.3085,6.642) rectangle (12.3483,6.74785);
\draw [color=c, fill=c] (12.3483,6.642) rectangle (12.3881,6.74785);
\draw [color=c, fill=c] (12.3881,6.642) rectangle (12.4279,6.74785);
\draw [color=c, fill=c] (12.4279,6.642) rectangle (12.4677,6.74785);
\draw [color=c, fill=c] (12.4677,6.642) rectangle (12.5075,6.74785);
\draw [color=c, fill=c] (12.5075,6.642) rectangle (12.5473,6.74785);
\draw [color=c, fill=c] (12.5473,6.642) rectangle (12.5871,6.74785);
\draw [color=c, fill=c] (12.5871,6.642) rectangle (12.6269,6.74785);
\draw [color=c, fill=c] (12.6269,6.642) rectangle (12.6667,6.74785);
\draw [color=c, fill=c] (12.6667,6.642) rectangle (12.7065,6.74785);
\draw [color=c, fill=c] (12.7065,6.642) rectangle (12.7463,6.74785);
\draw [color=c, fill=c] (12.7463,6.642) rectangle (12.7861,6.74785);
\draw [color=c, fill=c] (12.7861,6.642) rectangle (12.8259,6.74785);
\draw [color=c, fill=c] (12.8259,6.642) rectangle (12.8657,6.74785);
\draw [color=c, fill=c] (12.8657,6.642) rectangle (12.9055,6.74785);
\draw [color=c, fill=c] (12.9055,6.642) rectangle (12.9453,6.74785);
\draw [color=c, fill=c] (12.9453,6.642) rectangle (12.9851,6.74785);
\draw [color=c, fill=c] (12.9851,6.642) rectangle (13.0249,6.74785);
\draw [color=c, fill=c] (13.0249,6.642) rectangle (13.0647,6.74785);
\draw [color=c, fill=c] (13.0647,6.642) rectangle (13.1045,6.74785);
\draw [color=c, fill=c] (13.1045,6.642) rectangle (13.1443,6.74785);
\draw [color=c, fill=c] (13.1443,6.642) rectangle (13.1841,6.74785);
\draw [color=c, fill=c] (13.1841,6.642) rectangle (13.2239,6.74785);
\draw [color=c, fill=c] (13.2239,6.642) rectangle (13.2637,6.74785);
\draw [color=c, fill=c] (13.2637,6.642) rectangle (13.3035,6.74785);
\draw [color=c, fill=c] (13.3035,6.642) rectangle (13.3433,6.74785);
\draw [color=c, fill=c] (13.3433,6.642) rectangle (13.3831,6.74785);
\draw [color=c, fill=c] (13.3831,6.642) rectangle (13.4229,6.74785);
\draw [color=c, fill=c] (13.4229,6.642) rectangle (13.4627,6.74785);
\draw [color=c, fill=c] (13.4627,6.642) rectangle (13.5025,6.74785);
\draw [color=c, fill=c] (13.5025,6.642) rectangle (13.5423,6.74785);
\draw [color=c, fill=c] (13.5423,6.642) rectangle (13.5821,6.74785);
\draw [color=c, fill=c] (13.5821,6.642) rectangle (13.6219,6.74785);
\draw [color=c, fill=c] (13.6219,6.642) rectangle (13.6617,6.74785);
\draw [color=c, fill=c] (13.6617,6.642) rectangle (13.7015,6.74785);
\draw [color=c, fill=c] (13.7015,6.642) rectangle (13.7413,6.74785);
\draw [color=c, fill=c] (13.7413,6.642) rectangle (13.7811,6.74785);
\draw [color=c, fill=c] (13.7811,6.642) rectangle (13.8209,6.74785);
\draw [color=c, fill=c] (13.8209,6.642) rectangle (13.8607,6.74785);
\draw [color=c, fill=c] (13.8607,6.642) rectangle (13.9005,6.74785);
\draw [color=c, fill=c] (13.9005,6.642) rectangle (13.9403,6.74785);
\draw [color=c, fill=c] (13.9403,6.642) rectangle (13.9801,6.74785);
\draw [color=c, fill=c] (13.9801,6.642) rectangle (14.0199,6.74785);
\draw [color=c, fill=c] (14.0199,6.642) rectangle (14.0597,6.74785);
\draw [color=c, fill=c] (14.0597,6.642) rectangle (14.0995,6.74785);
\draw [color=c, fill=c] (14.0995,6.642) rectangle (14.1393,6.74785);
\draw [color=c, fill=c] (14.1393,6.642) rectangle (14.1791,6.74785);
\draw [color=c, fill=c] (14.1791,6.642) rectangle (14.2189,6.74785);
\draw [color=c, fill=c] (14.2189,6.642) rectangle (14.2587,6.74785);
\draw [color=c, fill=c] (14.2587,6.642) rectangle (14.2985,6.74785);
\draw [color=c, fill=c] (14.2985,6.642) rectangle (14.3383,6.74785);
\draw [color=c, fill=c] (14.3383,6.642) rectangle (14.3781,6.74785);
\draw [color=c, fill=c] (14.3781,6.642) rectangle (14.4179,6.74785);
\draw [color=c, fill=c] (14.4179,6.642) rectangle (14.4577,6.74785);
\draw [color=c, fill=c] (14.4577,6.642) rectangle (14.4975,6.74785);
\draw [color=c, fill=c] (14.4975,6.642) rectangle (14.5373,6.74785);
\draw [color=c, fill=c] (14.5373,6.642) rectangle (14.5771,6.74785);
\draw [color=c, fill=c] (14.5771,6.642) rectangle (14.6169,6.74785);
\draw [color=c, fill=c] (14.6169,6.642) rectangle (14.6567,6.74785);
\draw [color=c, fill=c] (14.6567,6.642) rectangle (14.6965,6.74785);
\draw [color=c, fill=c] (14.6965,6.642) rectangle (14.7363,6.74785);
\draw [color=c, fill=c] (14.7363,6.642) rectangle (14.7761,6.74785);
\draw [color=c, fill=c] (14.7761,6.642) rectangle (14.8159,6.74785);
\draw [color=c, fill=c] (14.8159,6.642) rectangle (14.8557,6.74785);
\draw [color=c, fill=c] (14.8557,6.642) rectangle (14.8955,6.74785);
\draw [color=c, fill=c] (14.8955,6.642) rectangle (14.9353,6.74785);
\draw [color=c, fill=c] (14.9353,6.642) rectangle (14.9751,6.74785);
\draw [color=c, fill=c] (14.9751,6.642) rectangle (15.0149,6.74785);
\draw [color=c, fill=c] (15.0149,6.642) rectangle (15.0547,6.74785);
\draw [color=c, fill=c] (15.0547,6.642) rectangle (15.0945,6.74785);
\draw [color=c, fill=c] (15.0945,6.642) rectangle (15.1343,6.74785);
\draw [color=c, fill=c] (15.1343,6.642) rectangle (15.1741,6.74785);
\draw [color=c, fill=c] (15.1741,6.642) rectangle (15.2139,6.74785);
\draw [color=c, fill=c] (15.2139,6.642) rectangle (15.2537,6.74785);
\draw [color=c, fill=c] (15.2537,6.642) rectangle (15.2935,6.74785);
\draw [color=c, fill=c] (15.2935,6.642) rectangle (15.3333,6.74785);
\draw [color=c, fill=c] (15.3333,6.642) rectangle (15.3731,6.74785);
\draw [color=c, fill=c] (15.3731,6.642) rectangle (15.4129,6.74785);
\draw [color=c, fill=c] (15.4129,6.642) rectangle (15.4527,6.74785);
\draw [color=c, fill=c] (15.4527,6.642) rectangle (15.4925,6.74785);
\draw [color=c, fill=c] (15.4925,6.642) rectangle (15.5323,6.74785);
\draw [color=c, fill=c] (15.5323,6.642) rectangle (15.5721,6.74785);
\draw [color=c, fill=c] (15.5721,6.642) rectangle (15.6119,6.74785);
\draw [color=c, fill=c] (15.6119,6.642) rectangle (15.6517,6.74785);
\draw [color=c, fill=c] (15.6517,6.642) rectangle (15.6915,6.74785);
\draw [color=c, fill=c] (15.6915,6.642) rectangle (15.7313,6.74785);
\draw [color=c, fill=c] (15.7313,6.642) rectangle (15.7711,6.74785);
\draw [color=c, fill=c] (15.7711,6.642) rectangle (15.8109,6.74785);
\draw [color=c, fill=c] (15.8109,6.642) rectangle (15.8507,6.74785);
\draw [color=c, fill=c] (15.8507,6.642) rectangle (15.8905,6.74785);
\draw [color=c, fill=c] (15.8905,6.642) rectangle (15.9303,6.74785);
\draw [color=c, fill=c] (15.9303,6.642) rectangle (15.9701,6.74785);
\draw [color=c, fill=c] (15.9701,6.642) rectangle (16.01,6.74785);
\draw [color=c, fill=c] (16.01,6.642) rectangle (16.0498,6.74785);
\draw [color=c, fill=c] (16.0498,6.642) rectangle (16.0896,6.74785);
\draw [color=c, fill=c] (16.0896,6.642) rectangle (16.1294,6.74785);
\draw [color=c, fill=c] (16.1294,6.642) rectangle (16.1692,6.74785);
\draw [color=c, fill=c] (16.1692,6.642) rectangle (16.209,6.74785);
\draw [color=c, fill=c] (16.209,6.642) rectangle (16.2488,6.74785);
\draw [color=c, fill=c] (16.2488,6.642) rectangle (16.2886,6.74785);
\draw [color=c, fill=c] (16.2886,6.642) rectangle (16.3284,6.74785);
\draw [color=c, fill=c] (16.3284,6.642) rectangle (16.3682,6.74785);
\draw [color=c, fill=c] (16.3682,6.642) rectangle (16.408,6.74785);
\draw [color=c, fill=c] (16.408,6.642) rectangle (16.4478,6.74785);
\draw [color=c, fill=c] (16.4478,6.642) rectangle (16.4876,6.74785);
\draw [color=c, fill=c] (16.4876,6.642) rectangle (16.5274,6.74785);
\draw [color=c, fill=c] (16.5274,6.642) rectangle (16.5672,6.74785);
\draw [color=c, fill=c] (16.5672,6.642) rectangle (16.607,6.74785);
\draw [color=c, fill=c] (16.607,6.642) rectangle (16.6468,6.74785);
\draw [color=c, fill=c] (16.6468,6.642) rectangle (16.6866,6.74785);
\draw [color=c, fill=c] (16.6866,6.642) rectangle (16.7264,6.74785);
\draw [color=c, fill=c] (16.7264,6.642) rectangle (16.7662,6.74785);
\draw [color=c, fill=c] (16.7662,6.642) rectangle (16.806,6.74785);
\draw [color=c, fill=c] (16.806,6.642) rectangle (16.8458,6.74785);
\draw [color=c, fill=c] (16.8458,6.642) rectangle (16.8856,6.74785);
\draw [color=c, fill=c] (16.8856,6.642) rectangle (16.9254,6.74785);
\draw [color=c, fill=c] (16.9254,6.642) rectangle (16.9652,6.74785);
\draw [color=c, fill=c] (16.9652,6.642) rectangle (17.005,6.74785);
\draw [color=c, fill=c] (17.005,6.642) rectangle (17.0448,6.74785);
\draw [color=c, fill=c] (17.0448,6.642) rectangle (17.0846,6.74785);
\draw [color=c, fill=c] (17.0846,6.642) rectangle (17.1244,6.74785);
\draw [color=c, fill=c] (17.1244,6.642) rectangle (17.1642,6.74785);
\draw [color=c, fill=c] (17.1642,6.642) rectangle (17.204,6.74785);
\draw [color=c, fill=c] (17.204,6.642) rectangle (17.2438,6.74785);
\draw [color=c, fill=c] (17.2438,6.642) rectangle (17.2836,6.74785);
\draw [color=c, fill=c] (17.2836,6.642) rectangle (17.3234,6.74785);
\draw [color=c, fill=c] (17.3234,6.642) rectangle (17.3632,6.74785);
\draw [color=c, fill=c] (17.3632,6.642) rectangle (17.403,6.74785);
\draw [color=c, fill=c] (17.403,6.642) rectangle (17.4428,6.74785);
\draw [color=c, fill=c] (17.4428,6.642) rectangle (17.4826,6.74785);
\draw [color=c, fill=c] (17.4826,6.642) rectangle (17.5224,6.74785);
\draw [color=c, fill=c] (17.5224,6.642) rectangle (17.5622,6.74785);
\draw [color=c, fill=c] (17.5622,6.642) rectangle (17.602,6.74785);
\draw [color=c, fill=c] (17.602,6.642) rectangle (17.6418,6.74785);
\draw [color=c, fill=c] (17.6418,6.642) rectangle (17.6816,6.74785);
\draw [color=c, fill=c] (17.6816,6.642) rectangle (17.7214,6.74785);
\draw [color=c, fill=c] (17.7214,6.642) rectangle (17.7612,6.74785);
\draw [color=c, fill=c] (17.7612,6.642) rectangle (17.801,6.74785);
\draw [color=c, fill=c] (17.801,6.642) rectangle (17.8408,6.74785);
\draw [color=c, fill=c] (17.8408,6.642) rectangle (17.8806,6.74785);
\draw [color=c, fill=c] (17.8806,6.642) rectangle (17.9204,6.74785);
\draw [color=c, fill=c] (17.9204,6.642) rectangle (17.9602,6.74785);
\draw [color=c, fill=c] (17.9602,6.642) rectangle (18,6.74785);
\definecolor{c}{rgb}{0,0.0800001,1};
\draw [color=c, fill=c] (2,6.74785) rectangle (2.0398,6.8537);
\draw [color=c, fill=c] (2.0398,6.74785) rectangle (2.0796,6.8537);
\draw [color=c, fill=c] (2.0796,6.74785) rectangle (2.1194,6.8537);
\draw [color=c, fill=c] (2.1194,6.74785) rectangle (2.1592,6.8537);
\draw [color=c, fill=c] (2.1592,6.74785) rectangle (2.19901,6.8537);
\draw [color=c, fill=c] (2.19901,6.74785) rectangle (2.23881,6.8537);
\draw [color=c, fill=c] (2.23881,6.74785) rectangle (2.27861,6.8537);
\draw [color=c, fill=c] (2.27861,6.74785) rectangle (2.31841,6.8537);
\draw [color=c, fill=c] (2.31841,6.74785) rectangle (2.35821,6.8537);
\draw [color=c, fill=c] (2.35821,6.74785) rectangle (2.39801,6.8537);
\draw [color=c, fill=c] (2.39801,6.74785) rectangle (2.43781,6.8537);
\draw [color=c, fill=c] (2.43781,6.74785) rectangle (2.47761,6.8537);
\draw [color=c, fill=c] (2.47761,6.74785) rectangle (2.51741,6.8537);
\draw [color=c, fill=c] (2.51741,6.74785) rectangle (2.55721,6.8537);
\draw [color=c, fill=c] (2.55721,6.74785) rectangle (2.59702,6.8537);
\draw [color=c, fill=c] (2.59702,6.74785) rectangle (2.63682,6.8537);
\draw [color=c, fill=c] (2.63682,6.74785) rectangle (2.67662,6.8537);
\draw [color=c, fill=c] (2.67662,6.74785) rectangle (2.71642,6.8537);
\draw [color=c, fill=c] (2.71642,6.74785) rectangle (2.75622,6.8537);
\draw [color=c, fill=c] (2.75622,6.74785) rectangle (2.79602,6.8537);
\draw [color=c, fill=c] (2.79602,6.74785) rectangle (2.83582,6.8537);
\draw [color=c, fill=c] (2.83582,6.74785) rectangle (2.87562,6.8537);
\draw [color=c, fill=c] (2.87562,6.74785) rectangle (2.91542,6.8537);
\draw [color=c, fill=c] (2.91542,6.74785) rectangle (2.95522,6.8537);
\draw [color=c, fill=c] (2.95522,6.74785) rectangle (2.99502,6.8537);
\draw [color=c, fill=c] (2.99502,6.74785) rectangle (3.03483,6.8537);
\draw [color=c, fill=c] (3.03483,6.74785) rectangle (3.07463,6.8537);
\draw [color=c, fill=c] (3.07463,6.74785) rectangle (3.11443,6.8537);
\draw [color=c, fill=c] (3.11443,6.74785) rectangle (3.15423,6.8537);
\draw [color=c, fill=c] (3.15423,6.74785) rectangle (3.19403,6.8537);
\draw [color=c, fill=c] (3.19403,6.74785) rectangle (3.23383,6.8537);
\draw [color=c, fill=c] (3.23383,6.74785) rectangle (3.27363,6.8537);
\draw [color=c, fill=c] (3.27363,6.74785) rectangle (3.31343,6.8537);
\draw [color=c, fill=c] (3.31343,6.74785) rectangle (3.35323,6.8537);
\draw [color=c, fill=c] (3.35323,6.74785) rectangle (3.39303,6.8537);
\draw [color=c, fill=c] (3.39303,6.74785) rectangle (3.43284,6.8537);
\draw [color=c, fill=c] (3.43284,6.74785) rectangle (3.47264,6.8537);
\draw [color=c, fill=c] (3.47264,6.74785) rectangle (3.51244,6.8537);
\draw [color=c, fill=c] (3.51244,6.74785) rectangle (3.55224,6.8537);
\draw [color=c, fill=c] (3.55224,6.74785) rectangle (3.59204,6.8537);
\draw [color=c, fill=c] (3.59204,6.74785) rectangle (3.63184,6.8537);
\draw [color=c, fill=c] (3.63184,6.74785) rectangle (3.67164,6.8537);
\draw [color=c, fill=c] (3.67164,6.74785) rectangle (3.71144,6.8537);
\draw [color=c, fill=c] (3.71144,6.74785) rectangle (3.75124,6.8537);
\definecolor{c}{rgb}{0.2,0,1};
\draw [color=c, fill=c] (3.75124,6.74785) rectangle (3.79104,6.8537);
\draw [color=c, fill=c] (3.79104,6.74785) rectangle (3.83085,6.8537);
\draw [color=c, fill=c] (3.83085,6.74785) rectangle (3.87065,6.8537);
\draw [color=c, fill=c] (3.87065,6.74785) rectangle (3.91045,6.8537);
\draw [color=c, fill=c] (3.91045,6.74785) rectangle (3.95025,6.8537);
\draw [color=c, fill=c] (3.95025,6.74785) rectangle (3.99005,6.8537);
\draw [color=c, fill=c] (3.99005,6.74785) rectangle (4.02985,6.8537);
\draw [color=c, fill=c] (4.02985,6.74785) rectangle (4.06965,6.8537);
\draw [color=c, fill=c] (4.06965,6.74785) rectangle (4.10945,6.8537);
\draw [color=c, fill=c] (4.10945,6.74785) rectangle (4.14925,6.8537);
\draw [color=c, fill=c] (4.14925,6.74785) rectangle (4.18905,6.8537);
\draw [color=c, fill=c] (4.18905,6.74785) rectangle (4.22886,6.8537);
\draw [color=c, fill=c] (4.22886,6.74785) rectangle (4.26866,6.8537);
\draw [color=c, fill=c] (4.26866,6.74785) rectangle (4.30846,6.8537);
\draw [color=c, fill=c] (4.30846,6.74785) rectangle (4.34826,6.8537);
\draw [color=c, fill=c] (4.34826,6.74785) rectangle (4.38806,6.8537);
\draw [color=c, fill=c] (4.38806,6.74785) rectangle (4.42786,6.8537);
\draw [color=c, fill=c] (4.42786,6.74785) rectangle (4.46766,6.8537);
\draw [color=c, fill=c] (4.46766,6.74785) rectangle (4.50746,6.8537);
\draw [color=c, fill=c] (4.50746,6.74785) rectangle (4.54726,6.8537);
\draw [color=c, fill=c] (4.54726,6.74785) rectangle (4.58706,6.8537);
\draw [color=c, fill=c] (4.58706,6.74785) rectangle (4.62687,6.8537);
\draw [color=c, fill=c] (4.62687,6.74785) rectangle (4.66667,6.8537);
\draw [color=c, fill=c] (4.66667,6.74785) rectangle (4.70647,6.8537);
\draw [color=c, fill=c] (4.70647,6.74785) rectangle (4.74627,6.8537);
\draw [color=c, fill=c] (4.74627,6.74785) rectangle (4.78607,6.8537);
\draw [color=c, fill=c] (4.78607,6.74785) rectangle (4.82587,6.8537);
\draw [color=c, fill=c] (4.82587,6.74785) rectangle (4.86567,6.8537);
\draw [color=c, fill=c] (4.86567,6.74785) rectangle (4.90547,6.8537);
\draw [color=c, fill=c] (4.90547,6.74785) rectangle (4.94527,6.8537);
\draw [color=c, fill=c] (4.94527,6.74785) rectangle (4.98507,6.8537);
\draw [color=c, fill=c] (4.98507,6.74785) rectangle (5.02488,6.8537);
\draw [color=c, fill=c] (5.02488,6.74785) rectangle (5.06468,6.8537);
\draw [color=c, fill=c] (5.06468,6.74785) rectangle (5.10448,6.8537);
\draw [color=c, fill=c] (5.10448,6.74785) rectangle (5.14428,6.8537);
\draw [color=c, fill=c] (5.14428,6.74785) rectangle (5.18408,6.8537);
\draw [color=c, fill=c] (5.18408,6.74785) rectangle (5.22388,6.8537);
\draw [color=c, fill=c] (5.22388,6.74785) rectangle (5.26368,6.8537);
\draw [color=c, fill=c] (5.26368,6.74785) rectangle (5.30348,6.8537);
\draw [color=c, fill=c] (5.30348,6.74785) rectangle (5.34328,6.8537);
\draw [color=c, fill=c] (5.34328,6.74785) rectangle (5.38308,6.8537);
\draw [color=c, fill=c] (5.38308,6.74785) rectangle (5.42289,6.8537);
\draw [color=c, fill=c] (5.42289,6.74785) rectangle (5.46269,6.8537);
\draw [color=c, fill=c] (5.46269,6.74785) rectangle (5.50249,6.8537);
\draw [color=c, fill=c] (5.50249,6.74785) rectangle (5.54229,6.8537);
\draw [color=c, fill=c] (5.54229,6.74785) rectangle (5.58209,6.8537);
\draw [color=c, fill=c] (5.58209,6.74785) rectangle (5.62189,6.8537);
\draw [color=c, fill=c] (5.62189,6.74785) rectangle (5.66169,6.8537);
\draw [color=c, fill=c] (5.66169,6.74785) rectangle (5.70149,6.8537);
\draw [color=c, fill=c] (5.70149,6.74785) rectangle (5.74129,6.8537);
\draw [color=c, fill=c] (5.74129,6.74785) rectangle (5.78109,6.8537);
\draw [color=c, fill=c] (5.78109,6.74785) rectangle (5.8209,6.8537);
\draw [color=c, fill=c] (5.8209,6.74785) rectangle (5.8607,6.8537);
\draw [color=c, fill=c] (5.8607,6.74785) rectangle (5.9005,6.8537);
\draw [color=c, fill=c] (5.9005,6.74785) rectangle (5.9403,6.8537);
\draw [color=c, fill=c] (5.9403,6.74785) rectangle (5.9801,6.8537);
\draw [color=c, fill=c] (5.9801,6.74785) rectangle (6.0199,6.8537);
\draw [color=c, fill=c] (6.0199,6.74785) rectangle (6.0597,6.8537);
\draw [color=c, fill=c] (6.0597,6.74785) rectangle (6.0995,6.8537);
\draw [color=c, fill=c] (6.0995,6.74785) rectangle (6.1393,6.8537);
\draw [color=c, fill=c] (6.1393,6.74785) rectangle (6.1791,6.8537);
\draw [color=c, fill=c] (6.1791,6.74785) rectangle (6.21891,6.8537);
\draw [color=c, fill=c] (6.21891,6.74785) rectangle (6.25871,6.8537);
\draw [color=c, fill=c] (6.25871,6.74785) rectangle (6.29851,6.8537);
\draw [color=c, fill=c] (6.29851,6.74785) rectangle (6.33831,6.8537);
\draw [color=c, fill=c] (6.33831,6.74785) rectangle (6.37811,6.8537);
\draw [color=c, fill=c] (6.37811,6.74785) rectangle (6.41791,6.8537);
\draw [color=c, fill=c] (6.41791,6.74785) rectangle (6.45771,6.8537);
\draw [color=c, fill=c] (6.45771,6.74785) rectangle (6.49751,6.8537);
\draw [color=c, fill=c] (6.49751,6.74785) rectangle (6.53731,6.8537);
\draw [color=c, fill=c] (6.53731,6.74785) rectangle (6.57711,6.8537);
\draw [color=c, fill=c] (6.57711,6.74785) rectangle (6.61692,6.8537);
\draw [color=c, fill=c] (6.61692,6.74785) rectangle (6.65672,6.8537);
\draw [color=c, fill=c] (6.65672,6.74785) rectangle (6.69652,6.8537);
\draw [color=c, fill=c] (6.69652,6.74785) rectangle (6.73632,6.8537);
\draw [color=c, fill=c] (6.73632,6.74785) rectangle (6.77612,6.8537);
\draw [color=c, fill=c] (6.77612,6.74785) rectangle (6.81592,6.8537);
\draw [color=c, fill=c] (6.81592,6.74785) rectangle (6.85572,6.8537);
\draw [color=c, fill=c] (6.85572,6.74785) rectangle (6.89552,6.8537);
\draw [color=c, fill=c] (6.89552,6.74785) rectangle (6.93532,6.8537);
\draw [color=c, fill=c] (6.93532,6.74785) rectangle (6.97512,6.8537);
\draw [color=c, fill=c] (6.97512,6.74785) rectangle (7.01493,6.8537);
\draw [color=c, fill=c] (7.01493,6.74785) rectangle (7.05473,6.8537);
\draw [color=c, fill=c] (7.05473,6.74785) rectangle (7.09453,6.8537);
\draw [color=c, fill=c] (7.09453,6.74785) rectangle (7.13433,6.8537);
\draw [color=c, fill=c] (7.13433,6.74785) rectangle (7.17413,6.8537);
\draw [color=c, fill=c] (7.17413,6.74785) rectangle (7.21393,6.8537);
\draw [color=c, fill=c] (7.21393,6.74785) rectangle (7.25373,6.8537);
\draw [color=c, fill=c] (7.25373,6.74785) rectangle (7.29353,6.8537);
\draw [color=c, fill=c] (7.29353,6.74785) rectangle (7.33333,6.8537);
\draw [color=c, fill=c] (7.33333,6.74785) rectangle (7.37313,6.8537);
\draw [color=c, fill=c] (7.37313,6.74785) rectangle (7.41294,6.8537);
\draw [color=c, fill=c] (7.41294,6.74785) rectangle (7.45274,6.8537);
\draw [color=c, fill=c] (7.45274,6.74785) rectangle (7.49254,6.8537);
\draw [color=c, fill=c] (7.49254,6.74785) rectangle (7.53234,6.8537);
\draw [color=c, fill=c] (7.53234,6.74785) rectangle (7.57214,6.8537);
\draw [color=c, fill=c] (7.57214,6.74785) rectangle (7.61194,6.8537);
\draw [color=c, fill=c] (7.61194,6.74785) rectangle (7.65174,6.8537);
\draw [color=c, fill=c] (7.65174,6.74785) rectangle (7.69154,6.8537);
\draw [color=c, fill=c] (7.69154,6.74785) rectangle (7.73134,6.8537);
\draw [color=c, fill=c] (7.73134,6.74785) rectangle (7.77114,6.8537);
\draw [color=c, fill=c] (7.77114,6.74785) rectangle (7.81095,6.8537);
\draw [color=c, fill=c] (7.81095,6.74785) rectangle (7.85075,6.8537);
\draw [color=c, fill=c] (7.85075,6.74785) rectangle (7.89055,6.8537);
\draw [color=c, fill=c] (7.89055,6.74785) rectangle (7.93035,6.8537);
\draw [color=c, fill=c] (7.93035,6.74785) rectangle (7.97015,6.8537);
\draw [color=c, fill=c] (7.97015,6.74785) rectangle (8.00995,6.8537);
\draw [color=c, fill=c] (8.00995,6.74785) rectangle (8.04975,6.8537);
\draw [color=c, fill=c] (8.04975,6.74785) rectangle (8.08955,6.8537);
\draw [color=c, fill=c] (8.08955,6.74785) rectangle (8.12935,6.8537);
\draw [color=c, fill=c] (8.12935,6.74785) rectangle (8.16915,6.8537);
\draw [color=c, fill=c] (8.16915,6.74785) rectangle (8.20895,6.8537);
\draw [color=c, fill=c] (8.20895,6.74785) rectangle (8.24876,6.8537);
\draw [color=c, fill=c] (8.24876,6.74785) rectangle (8.28856,6.8537);
\draw [color=c, fill=c] (8.28856,6.74785) rectangle (8.32836,6.8537);
\draw [color=c, fill=c] (8.32836,6.74785) rectangle (8.36816,6.8537);
\draw [color=c, fill=c] (8.36816,6.74785) rectangle (8.40796,6.8537);
\definecolor{c}{rgb}{0,0.0800001,1};
\draw [color=c, fill=c] (8.40796,6.74785) rectangle (8.44776,6.8537);
\draw [color=c, fill=c] (8.44776,6.74785) rectangle (8.48756,6.8537);
\draw [color=c, fill=c] (8.48756,6.74785) rectangle (8.52736,6.8537);
\draw [color=c, fill=c] (8.52736,6.74785) rectangle (8.56716,6.8537);
\draw [color=c, fill=c] (8.56716,6.74785) rectangle (8.60697,6.8537);
\draw [color=c, fill=c] (8.60697,6.74785) rectangle (8.64677,6.8537);
\draw [color=c, fill=c] (8.64677,6.74785) rectangle (8.68657,6.8537);
\draw [color=c, fill=c] (8.68657,6.74785) rectangle (8.72637,6.8537);
\draw [color=c, fill=c] (8.72637,6.74785) rectangle (8.76617,6.8537);
\draw [color=c, fill=c] (8.76617,6.74785) rectangle (8.80597,6.8537);
\draw [color=c, fill=c] (8.80597,6.74785) rectangle (8.84577,6.8537);
\draw [color=c, fill=c] (8.84577,6.74785) rectangle (8.88557,6.8537);
\draw [color=c, fill=c] (8.88557,6.74785) rectangle (8.92537,6.8537);
\draw [color=c, fill=c] (8.92537,6.74785) rectangle (8.96517,6.8537);
\draw [color=c, fill=c] (8.96517,6.74785) rectangle (9.00498,6.8537);
\draw [color=c, fill=c] (9.00498,6.74785) rectangle (9.04478,6.8537);
\draw [color=c, fill=c] (9.04478,6.74785) rectangle (9.08458,6.8537);
\draw [color=c, fill=c] (9.08458,6.74785) rectangle (9.12438,6.8537);
\draw [color=c, fill=c] (9.12438,6.74785) rectangle (9.16418,6.8537);
\draw [color=c, fill=c] (9.16418,6.74785) rectangle (9.20398,6.8537);
\draw [color=c, fill=c] (9.20398,6.74785) rectangle (9.24378,6.8537);
\draw [color=c, fill=c] (9.24378,6.74785) rectangle (9.28358,6.8537);
\draw [color=c, fill=c] (9.28358,6.74785) rectangle (9.32338,6.8537);
\draw [color=c, fill=c] (9.32338,6.74785) rectangle (9.36318,6.8537);
\draw [color=c, fill=c] (9.36318,6.74785) rectangle (9.40298,6.8537);
\draw [color=c, fill=c] (9.40298,6.74785) rectangle (9.44279,6.8537);
\draw [color=c, fill=c] (9.44279,6.74785) rectangle (9.48259,6.8537);
\draw [color=c, fill=c] (9.48259,6.74785) rectangle (9.52239,6.8537);
\draw [color=c, fill=c] (9.52239,6.74785) rectangle (9.56219,6.8537);
\draw [color=c, fill=c] (9.56219,6.74785) rectangle (9.60199,6.8537);
\draw [color=c, fill=c] (9.60199,6.74785) rectangle (9.64179,6.8537);
\definecolor{c}{rgb}{0,0.266667,1};
\draw [color=c, fill=c] (9.64179,6.74785) rectangle (9.68159,6.8537);
\draw [color=c, fill=c] (9.68159,6.74785) rectangle (9.72139,6.8537);
\draw [color=c, fill=c] (9.72139,6.74785) rectangle (9.76119,6.8537);
\draw [color=c, fill=c] (9.76119,6.74785) rectangle (9.80099,6.8537);
\draw [color=c, fill=c] (9.80099,6.74785) rectangle (9.8408,6.8537);
\draw [color=c, fill=c] (9.8408,6.74785) rectangle (9.8806,6.8537);
\draw [color=c, fill=c] (9.8806,6.74785) rectangle (9.9204,6.8537);
\draw [color=c, fill=c] (9.9204,6.74785) rectangle (9.9602,6.8537);
\draw [color=c, fill=c] (9.9602,6.74785) rectangle (10,6.8537);
\draw [color=c, fill=c] (10,6.74785) rectangle (10.0398,6.8537);
\draw [color=c, fill=c] (10.0398,6.74785) rectangle (10.0796,6.8537);
\draw [color=c, fill=c] (10.0796,6.74785) rectangle (10.1194,6.8537);
\draw [color=c, fill=c] (10.1194,6.74785) rectangle (10.1592,6.8537);
\draw [color=c, fill=c] (10.1592,6.74785) rectangle (10.199,6.8537);
\definecolor{c}{rgb}{0,0.546666,1};
\draw [color=c, fill=c] (10.199,6.74785) rectangle (10.2388,6.8537);
\draw [color=c, fill=c] (10.2388,6.74785) rectangle (10.2786,6.8537);
\draw [color=c, fill=c] (10.2786,6.74785) rectangle (10.3184,6.8537);
\draw [color=c, fill=c] (10.3184,6.74785) rectangle (10.3582,6.8537);
\draw [color=c, fill=c] (10.3582,6.74785) rectangle (10.398,6.8537);
\draw [color=c, fill=c] (10.398,6.74785) rectangle (10.4378,6.8537);
\draw [color=c, fill=c] (10.4378,6.74785) rectangle (10.4776,6.8537);
\draw [color=c, fill=c] (10.4776,6.74785) rectangle (10.5174,6.8537);
\draw [color=c, fill=c] (10.5174,6.74785) rectangle (10.5572,6.8537);
\draw [color=c, fill=c] (10.5572,6.74785) rectangle (10.597,6.8537);
\draw [color=c, fill=c] (10.597,6.74785) rectangle (10.6368,6.8537);
\draw [color=c, fill=c] (10.6368,6.74785) rectangle (10.6766,6.8537);
\draw [color=c, fill=c] (10.6766,6.74785) rectangle (10.7164,6.8537);
\draw [color=c, fill=c] (10.7164,6.74785) rectangle (10.7562,6.8537);
\draw [color=c, fill=c] (10.7562,6.74785) rectangle (10.796,6.8537);
\draw [color=c, fill=c] (10.796,6.74785) rectangle (10.8358,6.8537);
\draw [color=c, fill=c] (10.8358,6.74785) rectangle (10.8756,6.8537);
\draw [color=c, fill=c] (10.8756,6.74785) rectangle (10.9154,6.8537);
\draw [color=c, fill=c] (10.9154,6.74785) rectangle (10.9552,6.8537);
\draw [color=c, fill=c] (10.9552,6.74785) rectangle (10.995,6.8537);
\draw [color=c, fill=c] (10.995,6.74785) rectangle (11.0348,6.8537);
\draw [color=c, fill=c] (11.0348,6.74785) rectangle (11.0746,6.8537);
\draw [color=c, fill=c] (11.0746,6.74785) rectangle (11.1144,6.8537);
\draw [color=c, fill=c] (11.1144,6.74785) rectangle (11.1542,6.8537);
\draw [color=c, fill=c] (11.1542,6.74785) rectangle (11.194,6.8537);
\definecolor{c}{rgb}{0,0.733333,1};
\draw [color=c, fill=c] (11.194,6.74785) rectangle (11.2338,6.8537);
\draw [color=c, fill=c] (11.2338,6.74785) rectangle (11.2736,6.8537);
\draw [color=c, fill=c] (11.2736,6.74785) rectangle (11.3134,6.8537);
\draw [color=c, fill=c] (11.3134,6.74785) rectangle (11.3532,6.8537);
\draw [color=c, fill=c] (11.3532,6.74785) rectangle (11.393,6.8537);
\draw [color=c, fill=c] (11.393,6.74785) rectangle (11.4328,6.8537);
\draw [color=c, fill=c] (11.4328,6.74785) rectangle (11.4726,6.8537);
\draw [color=c, fill=c] (11.4726,6.74785) rectangle (11.5124,6.8537);
\draw [color=c, fill=c] (11.5124,6.74785) rectangle (11.5522,6.8537);
\draw [color=c, fill=c] (11.5522,6.74785) rectangle (11.592,6.8537);
\draw [color=c, fill=c] (11.592,6.74785) rectangle (11.6318,6.8537);
\draw [color=c, fill=c] (11.6318,6.74785) rectangle (11.6716,6.8537);
\draw [color=c, fill=c] (11.6716,6.74785) rectangle (11.7114,6.8537);
\draw [color=c, fill=c] (11.7114,6.74785) rectangle (11.7512,6.8537);
\draw [color=c, fill=c] (11.7512,6.74785) rectangle (11.791,6.8537);
\draw [color=c, fill=c] (11.791,6.74785) rectangle (11.8308,6.8537);
\draw [color=c, fill=c] (11.8308,6.74785) rectangle (11.8706,6.8537);
\draw [color=c, fill=c] (11.8706,6.74785) rectangle (11.9104,6.8537);
\draw [color=c, fill=c] (11.9104,6.74785) rectangle (11.9502,6.8537);
\draw [color=c, fill=c] (11.9502,6.74785) rectangle (11.99,6.8537);
\draw [color=c, fill=c] (11.99,6.74785) rectangle (12.0299,6.8537);
\draw [color=c, fill=c] (12.0299,6.74785) rectangle (12.0697,6.8537);
\draw [color=c, fill=c] (12.0697,6.74785) rectangle (12.1095,6.8537);
\draw [color=c, fill=c] (12.1095,6.74785) rectangle (12.1493,6.8537);
\draw [color=c, fill=c] (12.1493,6.74785) rectangle (12.1891,6.8537);
\draw [color=c, fill=c] (12.1891,6.74785) rectangle (12.2289,6.8537);
\draw [color=c, fill=c] (12.2289,6.74785) rectangle (12.2687,6.8537);
\draw [color=c, fill=c] (12.2687,6.74785) rectangle (12.3085,6.8537);
\draw [color=c, fill=c] (12.3085,6.74785) rectangle (12.3483,6.8537);
\draw [color=c, fill=c] (12.3483,6.74785) rectangle (12.3881,6.8537);
\draw [color=c, fill=c] (12.3881,6.74785) rectangle (12.4279,6.8537);
\draw [color=c, fill=c] (12.4279,6.74785) rectangle (12.4677,6.8537);
\draw [color=c, fill=c] (12.4677,6.74785) rectangle (12.5075,6.8537);
\draw [color=c, fill=c] (12.5075,6.74785) rectangle (12.5473,6.8537);
\draw [color=c, fill=c] (12.5473,6.74785) rectangle (12.5871,6.8537);
\draw [color=c, fill=c] (12.5871,6.74785) rectangle (12.6269,6.8537);
\draw [color=c, fill=c] (12.6269,6.74785) rectangle (12.6667,6.8537);
\draw [color=c, fill=c] (12.6667,6.74785) rectangle (12.7065,6.8537);
\draw [color=c, fill=c] (12.7065,6.74785) rectangle (12.7463,6.8537);
\draw [color=c, fill=c] (12.7463,6.74785) rectangle (12.7861,6.8537);
\draw [color=c, fill=c] (12.7861,6.74785) rectangle (12.8259,6.8537);
\draw [color=c, fill=c] (12.8259,6.74785) rectangle (12.8657,6.8537);
\draw [color=c, fill=c] (12.8657,6.74785) rectangle (12.9055,6.8537);
\draw [color=c, fill=c] (12.9055,6.74785) rectangle (12.9453,6.8537);
\draw [color=c, fill=c] (12.9453,6.74785) rectangle (12.9851,6.8537);
\draw [color=c, fill=c] (12.9851,6.74785) rectangle (13.0249,6.8537);
\draw [color=c, fill=c] (13.0249,6.74785) rectangle (13.0647,6.8537);
\draw [color=c, fill=c] (13.0647,6.74785) rectangle (13.1045,6.8537);
\draw [color=c, fill=c] (13.1045,6.74785) rectangle (13.1443,6.8537);
\draw [color=c, fill=c] (13.1443,6.74785) rectangle (13.1841,6.8537);
\draw [color=c, fill=c] (13.1841,6.74785) rectangle (13.2239,6.8537);
\draw [color=c, fill=c] (13.2239,6.74785) rectangle (13.2637,6.8537);
\draw [color=c, fill=c] (13.2637,6.74785) rectangle (13.3035,6.8537);
\draw [color=c, fill=c] (13.3035,6.74785) rectangle (13.3433,6.8537);
\draw [color=c, fill=c] (13.3433,6.74785) rectangle (13.3831,6.8537);
\draw [color=c, fill=c] (13.3831,6.74785) rectangle (13.4229,6.8537);
\draw [color=c, fill=c] (13.4229,6.74785) rectangle (13.4627,6.8537);
\draw [color=c, fill=c] (13.4627,6.74785) rectangle (13.5025,6.8537);
\draw [color=c, fill=c] (13.5025,6.74785) rectangle (13.5423,6.8537);
\draw [color=c, fill=c] (13.5423,6.74785) rectangle (13.5821,6.8537);
\draw [color=c, fill=c] (13.5821,6.74785) rectangle (13.6219,6.8537);
\draw [color=c, fill=c] (13.6219,6.74785) rectangle (13.6617,6.8537);
\draw [color=c, fill=c] (13.6617,6.74785) rectangle (13.7015,6.8537);
\draw [color=c, fill=c] (13.7015,6.74785) rectangle (13.7413,6.8537);
\draw [color=c, fill=c] (13.7413,6.74785) rectangle (13.7811,6.8537);
\draw [color=c, fill=c] (13.7811,6.74785) rectangle (13.8209,6.8537);
\draw [color=c, fill=c] (13.8209,6.74785) rectangle (13.8607,6.8537);
\draw [color=c, fill=c] (13.8607,6.74785) rectangle (13.9005,6.8537);
\draw [color=c, fill=c] (13.9005,6.74785) rectangle (13.9403,6.8537);
\draw [color=c, fill=c] (13.9403,6.74785) rectangle (13.9801,6.8537);
\draw [color=c, fill=c] (13.9801,6.74785) rectangle (14.0199,6.8537);
\draw [color=c, fill=c] (14.0199,6.74785) rectangle (14.0597,6.8537);
\draw [color=c, fill=c] (14.0597,6.74785) rectangle (14.0995,6.8537);
\draw [color=c, fill=c] (14.0995,6.74785) rectangle (14.1393,6.8537);
\draw [color=c, fill=c] (14.1393,6.74785) rectangle (14.1791,6.8537);
\draw [color=c, fill=c] (14.1791,6.74785) rectangle (14.2189,6.8537);
\draw [color=c, fill=c] (14.2189,6.74785) rectangle (14.2587,6.8537);
\draw [color=c, fill=c] (14.2587,6.74785) rectangle (14.2985,6.8537);
\draw [color=c, fill=c] (14.2985,6.74785) rectangle (14.3383,6.8537);
\draw [color=c, fill=c] (14.3383,6.74785) rectangle (14.3781,6.8537);
\draw [color=c, fill=c] (14.3781,6.74785) rectangle (14.4179,6.8537);
\draw [color=c, fill=c] (14.4179,6.74785) rectangle (14.4577,6.8537);
\draw [color=c, fill=c] (14.4577,6.74785) rectangle (14.4975,6.8537);
\draw [color=c, fill=c] (14.4975,6.74785) rectangle (14.5373,6.8537);
\draw [color=c, fill=c] (14.5373,6.74785) rectangle (14.5771,6.8537);
\draw [color=c, fill=c] (14.5771,6.74785) rectangle (14.6169,6.8537);
\draw [color=c, fill=c] (14.6169,6.74785) rectangle (14.6567,6.8537);
\draw [color=c, fill=c] (14.6567,6.74785) rectangle (14.6965,6.8537);
\draw [color=c, fill=c] (14.6965,6.74785) rectangle (14.7363,6.8537);
\draw [color=c, fill=c] (14.7363,6.74785) rectangle (14.7761,6.8537);
\draw [color=c, fill=c] (14.7761,6.74785) rectangle (14.8159,6.8537);
\draw [color=c, fill=c] (14.8159,6.74785) rectangle (14.8557,6.8537);
\draw [color=c, fill=c] (14.8557,6.74785) rectangle (14.8955,6.8537);
\draw [color=c, fill=c] (14.8955,6.74785) rectangle (14.9353,6.8537);
\draw [color=c, fill=c] (14.9353,6.74785) rectangle (14.9751,6.8537);
\draw [color=c, fill=c] (14.9751,6.74785) rectangle (15.0149,6.8537);
\draw [color=c, fill=c] (15.0149,6.74785) rectangle (15.0547,6.8537);
\draw [color=c, fill=c] (15.0547,6.74785) rectangle (15.0945,6.8537);
\draw [color=c, fill=c] (15.0945,6.74785) rectangle (15.1343,6.8537);
\draw [color=c, fill=c] (15.1343,6.74785) rectangle (15.1741,6.8537);
\draw [color=c, fill=c] (15.1741,6.74785) rectangle (15.2139,6.8537);
\draw [color=c, fill=c] (15.2139,6.74785) rectangle (15.2537,6.8537);
\draw [color=c, fill=c] (15.2537,6.74785) rectangle (15.2935,6.8537);
\draw [color=c, fill=c] (15.2935,6.74785) rectangle (15.3333,6.8537);
\draw [color=c, fill=c] (15.3333,6.74785) rectangle (15.3731,6.8537);
\draw [color=c, fill=c] (15.3731,6.74785) rectangle (15.4129,6.8537);
\draw [color=c, fill=c] (15.4129,6.74785) rectangle (15.4527,6.8537);
\draw [color=c, fill=c] (15.4527,6.74785) rectangle (15.4925,6.8537);
\draw [color=c, fill=c] (15.4925,6.74785) rectangle (15.5323,6.8537);
\draw [color=c, fill=c] (15.5323,6.74785) rectangle (15.5721,6.8537);
\draw [color=c, fill=c] (15.5721,6.74785) rectangle (15.6119,6.8537);
\draw [color=c, fill=c] (15.6119,6.74785) rectangle (15.6517,6.8537);
\draw [color=c, fill=c] (15.6517,6.74785) rectangle (15.6915,6.8537);
\draw [color=c, fill=c] (15.6915,6.74785) rectangle (15.7313,6.8537);
\draw [color=c, fill=c] (15.7313,6.74785) rectangle (15.7711,6.8537);
\draw [color=c, fill=c] (15.7711,6.74785) rectangle (15.8109,6.8537);
\draw [color=c, fill=c] (15.8109,6.74785) rectangle (15.8507,6.8537);
\draw [color=c, fill=c] (15.8507,6.74785) rectangle (15.8905,6.8537);
\draw [color=c, fill=c] (15.8905,6.74785) rectangle (15.9303,6.8537);
\draw [color=c, fill=c] (15.9303,6.74785) rectangle (15.9701,6.8537);
\draw [color=c, fill=c] (15.9701,6.74785) rectangle (16.01,6.8537);
\draw [color=c, fill=c] (16.01,6.74785) rectangle (16.0498,6.8537);
\draw [color=c, fill=c] (16.0498,6.74785) rectangle (16.0896,6.8537);
\draw [color=c, fill=c] (16.0896,6.74785) rectangle (16.1294,6.8537);
\draw [color=c, fill=c] (16.1294,6.74785) rectangle (16.1692,6.8537);
\draw [color=c, fill=c] (16.1692,6.74785) rectangle (16.209,6.8537);
\draw [color=c, fill=c] (16.209,6.74785) rectangle (16.2488,6.8537);
\draw [color=c, fill=c] (16.2488,6.74785) rectangle (16.2886,6.8537);
\draw [color=c, fill=c] (16.2886,6.74785) rectangle (16.3284,6.8537);
\draw [color=c, fill=c] (16.3284,6.74785) rectangle (16.3682,6.8537);
\draw [color=c, fill=c] (16.3682,6.74785) rectangle (16.408,6.8537);
\draw [color=c, fill=c] (16.408,6.74785) rectangle (16.4478,6.8537);
\draw [color=c, fill=c] (16.4478,6.74785) rectangle (16.4876,6.8537);
\draw [color=c, fill=c] (16.4876,6.74785) rectangle (16.5274,6.8537);
\draw [color=c, fill=c] (16.5274,6.74785) rectangle (16.5672,6.8537);
\draw [color=c, fill=c] (16.5672,6.74785) rectangle (16.607,6.8537);
\draw [color=c, fill=c] (16.607,6.74785) rectangle (16.6468,6.8537);
\draw [color=c, fill=c] (16.6468,6.74785) rectangle (16.6866,6.8537);
\draw [color=c, fill=c] (16.6866,6.74785) rectangle (16.7264,6.8537);
\draw [color=c, fill=c] (16.7264,6.74785) rectangle (16.7662,6.8537);
\draw [color=c, fill=c] (16.7662,6.74785) rectangle (16.806,6.8537);
\draw [color=c, fill=c] (16.806,6.74785) rectangle (16.8458,6.8537);
\draw [color=c, fill=c] (16.8458,6.74785) rectangle (16.8856,6.8537);
\draw [color=c, fill=c] (16.8856,6.74785) rectangle (16.9254,6.8537);
\draw [color=c, fill=c] (16.9254,6.74785) rectangle (16.9652,6.8537);
\draw [color=c, fill=c] (16.9652,6.74785) rectangle (17.005,6.8537);
\draw [color=c, fill=c] (17.005,6.74785) rectangle (17.0448,6.8537);
\draw [color=c, fill=c] (17.0448,6.74785) rectangle (17.0846,6.8537);
\draw [color=c, fill=c] (17.0846,6.74785) rectangle (17.1244,6.8537);
\draw [color=c, fill=c] (17.1244,6.74785) rectangle (17.1642,6.8537);
\draw [color=c, fill=c] (17.1642,6.74785) rectangle (17.204,6.8537);
\draw [color=c, fill=c] (17.204,6.74785) rectangle (17.2438,6.8537);
\draw [color=c, fill=c] (17.2438,6.74785) rectangle (17.2836,6.8537);
\draw [color=c, fill=c] (17.2836,6.74785) rectangle (17.3234,6.8537);
\draw [color=c, fill=c] (17.3234,6.74785) rectangle (17.3632,6.8537);
\draw [color=c, fill=c] (17.3632,6.74785) rectangle (17.403,6.8537);
\draw [color=c, fill=c] (17.403,6.74785) rectangle (17.4428,6.8537);
\draw [color=c, fill=c] (17.4428,6.74785) rectangle (17.4826,6.8537);
\draw [color=c, fill=c] (17.4826,6.74785) rectangle (17.5224,6.8537);
\draw [color=c, fill=c] (17.5224,6.74785) rectangle (17.5622,6.8537);
\draw [color=c, fill=c] (17.5622,6.74785) rectangle (17.602,6.8537);
\draw [color=c, fill=c] (17.602,6.74785) rectangle (17.6418,6.8537);
\draw [color=c, fill=c] (17.6418,6.74785) rectangle (17.6816,6.8537);
\draw [color=c, fill=c] (17.6816,6.74785) rectangle (17.7214,6.8537);
\draw [color=c, fill=c] (17.7214,6.74785) rectangle (17.7612,6.8537);
\draw [color=c, fill=c] (17.7612,6.74785) rectangle (17.801,6.8537);
\draw [color=c, fill=c] (17.801,6.74785) rectangle (17.8408,6.8537);
\draw [color=c, fill=c] (17.8408,6.74785) rectangle (17.8806,6.8537);
\draw [color=c, fill=c] (17.8806,6.74785) rectangle (17.9204,6.8537);
\draw [color=c, fill=c] (17.9204,6.74785) rectangle (17.9602,6.8537);
\draw [color=c, fill=c] (17.9602,6.74785) rectangle (18,6.8537);
\definecolor{c}{rgb}{0,0.0800001,1};
\draw [color=c, fill=c] (2,6.8537) rectangle (2.0398,6.95955);
\draw [color=c, fill=c] (2.0398,6.8537) rectangle (2.0796,6.95955);
\draw [color=c, fill=c] (2.0796,6.8537) rectangle (2.1194,6.95955);
\draw [color=c, fill=c] (2.1194,6.8537) rectangle (2.1592,6.95955);
\draw [color=c, fill=c] (2.1592,6.8537) rectangle (2.19901,6.95955);
\draw [color=c, fill=c] (2.19901,6.8537) rectangle (2.23881,6.95955);
\draw [color=c, fill=c] (2.23881,6.8537) rectangle (2.27861,6.95955);
\draw [color=c, fill=c] (2.27861,6.8537) rectangle (2.31841,6.95955);
\draw [color=c, fill=c] (2.31841,6.8537) rectangle (2.35821,6.95955);
\draw [color=c, fill=c] (2.35821,6.8537) rectangle (2.39801,6.95955);
\draw [color=c, fill=c] (2.39801,6.8537) rectangle (2.43781,6.95955);
\draw [color=c, fill=c] (2.43781,6.8537) rectangle (2.47761,6.95955);
\draw [color=c, fill=c] (2.47761,6.8537) rectangle (2.51741,6.95955);
\draw [color=c, fill=c] (2.51741,6.8537) rectangle (2.55721,6.95955);
\draw [color=c, fill=c] (2.55721,6.8537) rectangle (2.59702,6.95955);
\draw [color=c, fill=c] (2.59702,6.8537) rectangle (2.63682,6.95955);
\draw [color=c, fill=c] (2.63682,6.8537) rectangle (2.67662,6.95955);
\draw [color=c, fill=c] (2.67662,6.8537) rectangle (2.71642,6.95955);
\draw [color=c, fill=c] (2.71642,6.8537) rectangle (2.75622,6.95955);
\draw [color=c, fill=c] (2.75622,6.8537) rectangle (2.79602,6.95955);
\definecolor{c}{rgb}{0.2,0,1};
\draw [color=c, fill=c] (2.79602,6.8537) rectangle (2.83582,6.95955);
\draw [color=c, fill=c] (2.83582,6.8537) rectangle (2.87562,6.95955);
\draw [color=c, fill=c] (2.87562,6.8537) rectangle (2.91542,6.95955);
\draw [color=c, fill=c] (2.91542,6.8537) rectangle (2.95522,6.95955);
\draw [color=c, fill=c] (2.95522,6.8537) rectangle (2.99502,6.95955);
\draw [color=c, fill=c] (2.99502,6.8537) rectangle (3.03483,6.95955);
\draw [color=c, fill=c] (3.03483,6.8537) rectangle (3.07463,6.95955);
\draw [color=c, fill=c] (3.07463,6.8537) rectangle (3.11443,6.95955);
\draw [color=c, fill=c] (3.11443,6.8537) rectangle (3.15423,6.95955);
\draw [color=c, fill=c] (3.15423,6.8537) rectangle (3.19403,6.95955);
\draw [color=c, fill=c] (3.19403,6.8537) rectangle (3.23383,6.95955);
\draw [color=c, fill=c] (3.23383,6.8537) rectangle (3.27363,6.95955);
\draw [color=c, fill=c] (3.27363,6.8537) rectangle (3.31343,6.95955);
\draw [color=c, fill=c] (3.31343,6.8537) rectangle (3.35323,6.95955);
\draw [color=c, fill=c] (3.35323,6.8537) rectangle (3.39303,6.95955);
\draw [color=c, fill=c] (3.39303,6.8537) rectangle (3.43284,6.95955);
\draw [color=c, fill=c] (3.43284,6.8537) rectangle (3.47264,6.95955);
\draw [color=c, fill=c] (3.47264,6.8537) rectangle (3.51244,6.95955);
\draw [color=c, fill=c] (3.51244,6.8537) rectangle (3.55224,6.95955);
\draw [color=c, fill=c] (3.55224,6.8537) rectangle (3.59204,6.95955);
\draw [color=c, fill=c] (3.59204,6.8537) rectangle (3.63184,6.95955);
\draw [color=c, fill=c] (3.63184,6.8537) rectangle (3.67164,6.95955);
\draw [color=c, fill=c] (3.67164,6.8537) rectangle (3.71144,6.95955);
\draw [color=c, fill=c] (3.71144,6.8537) rectangle (3.75124,6.95955);
\draw [color=c, fill=c] (3.75124,6.8537) rectangle (3.79104,6.95955);
\draw [color=c, fill=c] (3.79104,6.8537) rectangle (3.83085,6.95955);
\draw [color=c, fill=c] (3.83085,6.8537) rectangle (3.87065,6.95955);
\draw [color=c, fill=c] (3.87065,6.8537) rectangle (3.91045,6.95955);
\draw [color=c, fill=c] (3.91045,6.8537) rectangle (3.95025,6.95955);
\draw [color=c, fill=c] (3.95025,6.8537) rectangle (3.99005,6.95955);
\draw [color=c, fill=c] (3.99005,6.8537) rectangle (4.02985,6.95955);
\draw [color=c, fill=c] (4.02985,6.8537) rectangle (4.06965,6.95955);
\draw [color=c, fill=c] (4.06965,6.8537) rectangle (4.10945,6.95955);
\draw [color=c, fill=c] (4.10945,6.8537) rectangle (4.14925,6.95955);
\draw [color=c, fill=c] (4.14925,6.8537) rectangle (4.18905,6.95955);
\draw [color=c, fill=c] (4.18905,6.8537) rectangle (4.22886,6.95955);
\draw [color=c, fill=c] (4.22886,6.8537) rectangle (4.26866,6.95955);
\draw [color=c, fill=c] (4.26866,6.8537) rectangle (4.30846,6.95955);
\draw [color=c, fill=c] (4.30846,6.8537) rectangle (4.34826,6.95955);
\draw [color=c, fill=c] (4.34826,6.8537) rectangle (4.38806,6.95955);
\draw [color=c, fill=c] (4.38806,6.8537) rectangle (4.42786,6.95955);
\draw [color=c, fill=c] (4.42786,6.8537) rectangle (4.46766,6.95955);
\draw [color=c, fill=c] (4.46766,6.8537) rectangle (4.50746,6.95955);
\draw [color=c, fill=c] (4.50746,6.8537) rectangle (4.54726,6.95955);
\draw [color=c, fill=c] (4.54726,6.8537) rectangle (4.58706,6.95955);
\draw [color=c, fill=c] (4.58706,6.8537) rectangle (4.62687,6.95955);
\draw [color=c, fill=c] (4.62687,6.8537) rectangle (4.66667,6.95955);
\draw [color=c, fill=c] (4.66667,6.8537) rectangle (4.70647,6.95955);
\draw [color=c, fill=c] (4.70647,6.8537) rectangle (4.74627,6.95955);
\draw [color=c, fill=c] (4.74627,6.8537) rectangle (4.78607,6.95955);
\draw [color=c, fill=c] (4.78607,6.8537) rectangle (4.82587,6.95955);
\draw [color=c, fill=c] (4.82587,6.8537) rectangle (4.86567,6.95955);
\draw [color=c, fill=c] (4.86567,6.8537) rectangle (4.90547,6.95955);
\draw [color=c, fill=c] (4.90547,6.8537) rectangle (4.94527,6.95955);
\draw [color=c, fill=c] (4.94527,6.8537) rectangle (4.98507,6.95955);
\draw [color=c, fill=c] (4.98507,6.8537) rectangle (5.02488,6.95955);
\draw [color=c, fill=c] (5.02488,6.8537) rectangle (5.06468,6.95955);
\draw [color=c, fill=c] (5.06468,6.8537) rectangle (5.10448,6.95955);
\draw [color=c, fill=c] (5.10448,6.8537) rectangle (5.14428,6.95955);
\draw [color=c, fill=c] (5.14428,6.8537) rectangle (5.18408,6.95955);
\draw [color=c, fill=c] (5.18408,6.8537) rectangle (5.22388,6.95955);
\draw [color=c, fill=c] (5.22388,6.8537) rectangle (5.26368,6.95955);
\draw [color=c, fill=c] (5.26368,6.8537) rectangle (5.30348,6.95955);
\draw [color=c, fill=c] (5.30348,6.8537) rectangle (5.34328,6.95955);
\draw [color=c, fill=c] (5.34328,6.8537) rectangle (5.38308,6.95955);
\draw [color=c, fill=c] (5.38308,6.8537) rectangle (5.42289,6.95955);
\draw [color=c, fill=c] (5.42289,6.8537) rectangle (5.46269,6.95955);
\draw [color=c, fill=c] (5.46269,6.8537) rectangle (5.50249,6.95955);
\draw [color=c, fill=c] (5.50249,6.8537) rectangle (5.54229,6.95955);
\draw [color=c, fill=c] (5.54229,6.8537) rectangle (5.58209,6.95955);
\draw [color=c, fill=c] (5.58209,6.8537) rectangle (5.62189,6.95955);
\draw [color=c, fill=c] (5.62189,6.8537) rectangle (5.66169,6.95955);
\draw [color=c, fill=c] (5.66169,6.8537) rectangle (5.70149,6.95955);
\draw [color=c, fill=c] (5.70149,6.8537) rectangle (5.74129,6.95955);
\draw [color=c, fill=c] (5.74129,6.8537) rectangle (5.78109,6.95955);
\draw [color=c, fill=c] (5.78109,6.8537) rectangle (5.8209,6.95955);
\draw [color=c, fill=c] (5.8209,6.8537) rectangle (5.8607,6.95955);
\draw [color=c, fill=c] (5.8607,6.8537) rectangle (5.9005,6.95955);
\draw [color=c, fill=c] (5.9005,6.8537) rectangle (5.9403,6.95955);
\draw [color=c, fill=c] (5.9403,6.8537) rectangle (5.9801,6.95955);
\draw [color=c, fill=c] (5.9801,6.8537) rectangle (6.0199,6.95955);
\draw [color=c, fill=c] (6.0199,6.8537) rectangle (6.0597,6.95955);
\draw [color=c, fill=c] (6.0597,6.8537) rectangle (6.0995,6.95955);
\draw [color=c, fill=c] (6.0995,6.8537) rectangle (6.1393,6.95955);
\draw [color=c, fill=c] (6.1393,6.8537) rectangle (6.1791,6.95955);
\draw [color=c, fill=c] (6.1791,6.8537) rectangle (6.21891,6.95955);
\draw [color=c, fill=c] (6.21891,6.8537) rectangle (6.25871,6.95955);
\draw [color=c, fill=c] (6.25871,6.8537) rectangle (6.29851,6.95955);
\draw [color=c, fill=c] (6.29851,6.8537) rectangle (6.33831,6.95955);
\draw [color=c, fill=c] (6.33831,6.8537) rectangle (6.37811,6.95955);
\draw [color=c, fill=c] (6.37811,6.8537) rectangle (6.41791,6.95955);
\draw [color=c, fill=c] (6.41791,6.8537) rectangle (6.45771,6.95955);
\draw [color=c, fill=c] (6.45771,6.8537) rectangle (6.49751,6.95955);
\draw [color=c, fill=c] (6.49751,6.8537) rectangle (6.53731,6.95955);
\draw [color=c, fill=c] (6.53731,6.8537) rectangle (6.57711,6.95955);
\draw [color=c, fill=c] (6.57711,6.8537) rectangle (6.61692,6.95955);
\draw [color=c, fill=c] (6.61692,6.8537) rectangle (6.65672,6.95955);
\draw [color=c, fill=c] (6.65672,6.8537) rectangle (6.69652,6.95955);
\draw [color=c, fill=c] (6.69652,6.8537) rectangle (6.73632,6.95955);
\draw [color=c, fill=c] (6.73632,6.8537) rectangle (6.77612,6.95955);
\draw [color=c, fill=c] (6.77612,6.8537) rectangle (6.81592,6.95955);
\draw [color=c, fill=c] (6.81592,6.8537) rectangle (6.85572,6.95955);
\draw [color=c, fill=c] (6.85572,6.8537) rectangle (6.89552,6.95955);
\draw [color=c, fill=c] (6.89552,6.8537) rectangle (6.93532,6.95955);
\draw [color=c, fill=c] (6.93532,6.8537) rectangle (6.97512,6.95955);
\draw [color=c, fill=c] (6.97512,6.8537) rectangle (7.01493,6.95955);
\draw [color=c, fill=c] (7.01493,6.8537) rectangle (7.05473,6.95955);
\draw [color=c, fill=c] (7.05473,6.8537) rectangle (7.09453,6.95955);
\draw [color=c, fill=c] (7.09453,6.8537) rectangle (7.13433,6.95955);
\draw [color=c, fill=c] (7.13433,6.8537) rectangle (7.17413,6.95955);
\draw [color=c, fill=c] (7.17413,6.8537) rectangle (7.21393,6.95955);
\draw [color=c, fill=c] (7.21393,6.8537) rectangle (7.25373,6.95955);
\draw [color=c, fill=c] (7.25373,6.8537) rectangle (7.29353,6.95955);
\draw [color=c, fill=c] (7.29353,6.8537) rectangle (7.33333,6.95955);
\draw [color=c, fill=c] (7.33333,6.8537) rectangle (7.37313,6.95955);
\draw [color=c, fill=c] (7.37313,6.8537) rectangle (7.41294,6.95955);
\draw [color=c, fill=c] (7.41294,6.8537) rectangle (7.45274,6.95955);
\draw [color=c, fill=c] (7.45274,6.8537) rectangle (7.49254,6.95955);
\draw [color=c, fill=c] (7.49254,6.8537) rectangle (7.53234,6.95955);
\draw [color=c, fill=c] (7.53234,6.8537) rectangle (7.57214,6.95955);
\draw [color=c, fill=c] (7.57214,6.8537) rectangle (7.61194,6.95955);
\draw [color=c, fill=c] (7.61194,6.8537) rectangle (7.65174,6.95955);
\draw [color=c, fill=c] (7.65174,6.8537) rectangle (7.69154,6.95955);
\draw [color=c, fill=c] (7.69154,6.8537) rectangle (7.73134,6.95955);
\draw [color=c, fill=c] (7.73134,6.8537) rectangle (7.77114,6.95955);
\draw [color=c, fill=c] (7.77114,6.8537) rectangle (7.81095,6.95955);
\draw [color=c, fill=c] (7.81095,6.8537) rectangle (7.85075,6.95955);
\draw [color=c, fill=c] (7.85075,6.8537) rectangle (7.89055,6.95955);
\draw [color=c, fill=c] (7.89055,6.8537) rectangle (7.93035,6.95955);
\draw [color=c, fill=c] (7.93035,6.8537) rectangle (7.97015,6.95955);
\draw [color=c, fill=c] (7.97015,6.8537) rectangle (8.00995,6.95955);
\draw [color=c, fill=c] (8.00995,6.8537) rectangle (8.04975,6.95955);
\draw [color=c, fill=c] (8.04975,6.8537) rectangle (8.08955,6.95955);
\draw [color=c, fill=c] (8.08955,6.8537) rectangle (8.12935,6.95955);
\draw [color=c, fill=c] (8.12935,6.8537) rectangle (8.16915,6.95955);
\draw [color=c, fill=c] (8.16915,6.8537) rectangle (8.20895,6.95955);
\draw [color=c, fill=c] (8.20895,6.8537) rectangle (8.24876,6.95955);
\draw [color=c, fill=c] (8.24876,6.8537) rectangle (8.28856,6.95955);
\draw [color=c, fill=c] (8.28856,6.8537) rectangle (8.32836,6.95955);
\definecolor{c}{rgb}{0,0.0800001,1};
\draw [color=c, fill=c] (8.32836,6.8537) rectangle (8.36816,6.95955);
\draw [color=c, fill=c] (8.36816,6.8537) rectangle (8.40796,6.95955);
\draw [color=c, fill=c] (8.40796,6.8537) rectangle (8.44776,6.95955);
\draw [color=c, fill=c] (8.44776,6.8537) rectangle (8.48756,6.95955);
\draw [color=c, fill=c] (8.48756,6.8537) rectangle (8.52736,6.95955);
\draw [color=c, fill=c] (8.52736,6.8537) rectangle (8.56716,6.95955);
\draw [color=c, fill=c] (8.56716,6.8537) rectangle (8.60697,6.95955);
\draw [color=c, fill=c] (8.60697,6.8537) rectangle (8.64677,6.95955);
\draw [color=c, fill=c] (8.64677,6.8537) rectangle (8.68657,6.95955);
\draw [color=c, fill=c] (8.68657,6.8537) rectangle (8.72637,6.95955);
\draw [color=c, fill=c] (8.72637,6.8537) rectangle (8.76617,6.95955);
\draw [color=c, fill=c] (8.76617,6.8537) rectangle (8.80597,6.95955);
\draw [color=c, fill=c] (8.80597,6.8537) rectangle (8.84577,6.95955);
\draw [color=c, fill=c] (8.84577,6.8537) rectangle (8.88557,6.95955);
\draw [color=c, fill=c] (8.88557,6.8537) rectangle (8.92537,6.95955);
\draw [color=c, fill=c] (8.92537,6.8537) rectangle (8.96517,6.95955);
\draw [color=c, fill=c] (8.96517,6.8537) rectangle (9.00498,6.95955);
\draw [color=c, fill=c] (9.00498,6.8537) rectangle (9.04478,6.95955);
\draw [color=c, fill=c] (9.04478,6.8537) rectangle (9.08458,6.95955);
\draw [color=c, fill=c] (9.08458,6.8537) rectangle (9.12438,6.95955);
\draw [color=c, fill=c] (9.12438,6.8537) rectangle (9.16418,6.95955);
\draw [color=c, fill=c] (9.16418,6.8537) rectangle (9.20398,6.95955);
\draw [color=c, fill=c] (9.20398,6.8537) rectangle (9.24378,6.95955);
\draw [color=c, fill=c] (9.24378,6.8537) rectangle (9.28358,6.95955);
\draw [color=c, fill=c] (9.28358,6.8537) rectangle (9.32338,6.95955);
\draw [color=c, fill=c] (9.32338,6.8537) rectangle (9.36318,6.95955);
\draw [color=c, fill=c] (9.36318,6.8537) rectangle (9.40298,6.95955);
\draw [color=c, fill=c] (9.40298,6.8537) rectangle (9.44279,6.95955);
\draw [color=c, fill=c] (9.44279,6.8537) rectangle (9.48259,6.95955);
\draw [color=c, fill=c] (9.48259,6.8537) rectangle (9.52239,6.95955);
\draw [color=c, fill=c] (9.52239,6.8537) rectangle (9.56219,6.95955);
\draw [color=c, fill=c] (9.56219,6.8537) rectangle (9.60199,6.95955);
\definecolor{c}{rgb}{0,0.266667,1};
\draw [color=c, fill=c] (9.60199,6.8537) rectangle (9.64179,6.95955);
\draw [color=c, fill=c] (9.64179,6.8537) rectangle (9.68159,6.95955);
\draw [color=c, fill=c] (9.68159,6.8537) rectangle (9.72139,6.95955);
\draw [color=c, fill=c] (9.72139,6.8537) rectangle (9.76119,6.95955);
\draw [color=c, fill=c] (9.76119,6.8537) rectangle (9.80099,6.95955);
\draw [color=c, fill=c] (9.80099,6.8537) rectangle (9.8408,6.95955);
\draw [color=c, fill=c] (9.8408,6.8537) rectangle (9.8806,6.95955);
\draw [color=c, fill=c] (9.8806,6.8537) rectangle (9.9204,6.95955);
\draw [color=c, fill=c] (9.9204,6.8537) rectangle (9.9602,6.95955);
\draw [color=c, fill=c] (9.9602,6.8537) rectangle (10,6.95955);
\draw [color=c, fill=c] (10,6.8537) rectangle (10.0398,6.95955);
\draw [color=c, fill=c] (10.0398,6.8537) rectangle (10.0796,6.95955);
\draw [color=c, fill=c] (10.0796,6.8537) rectangle (10.1194,6.95955);
\draw [color=c, fill=c] (10.1194,6.8537) rectangle (10.1592,6.95955);
\draw [color=c, fill=c] (10.1592,6.8537) rectangle (10.199,6.95955);
\draw [color=c, fill=c] (10.199,6.8537) rectangle (10.2388,6.95955);
\definecolor{c}{rgb}{0,0.546666,1};
\draw [color=c, fill=c] (10.2388,6.8537) rectangle (10.2786,6.95955);
\draw [color=c, fill=c] (10.2786,6.8537) rectangle (10.3184,6.95955);
\draw [color=c, fill=c] (10.3184,6.8537) rectangle (10.3582,6.95955);
\draw [color=c, fill=c] (10.3582,6.8537) rectangle (10.398,6.95955);
\draw [color=c, fill=c] (10.398,6.8537) rectangle (10.4378,6.95955);
\draw [color=c, fill=c] (10.4378,6.8537) rectangle (10.4776,6.95955);
\draw [color=c, fill=c] (10.4776,6.8537) rectangle (10.5174,6.95955);
\draw [color=c, fill=c] (10.5174,6.8537) rectangle (10.5572,6.95955);
\draw [color=c, fill=c] (10.5572,6.8537) rectangle (10.597,6.95955);
\draw [color=c, fill=c] (10.597,6.8537) rectangle (10.6368,6.95955);
\draw [color=c, fill=c] (10.6368,6.8537) rectangle (10.6766,6.95955);
\draw [color=c, fill=c] (10.6766,6.8537) rectangle (10.7164,6.95955);
\draw [color=c, fill=c] (10.7164,6.8537) rectangle (10.7562,6.95955);
\draw [color=c, fill=c] (10.7562,6.8537) rectangle (10.796,6.95955);
\draw [color=c, fill=c] (10.796,6.8537) rectangle (10.8358,6.95955);
\draw [color=c, fill=c] (10.8358,6.8537) rectangle (10.8756,6.95955);
\draw [color=c, fill=c] (10.8756,6.8537) rectangle (10.9154,6.95955);
\draw [color=c, fill=c] (10.9154,6.8537) rectangle (10.9552,6.95955);
\draw [color=c, fill=c] (10.9552,6.8537) rectangle (10.995,6.95955);
\draw [color=c, fill=c] (10.995,6.8537) rectangle (11.0348,6.95955);
\draw [color=c, fill=c] (11.0348,6.8537) rectangle (11.0746,6.95955);
\draw [color=c, fill=c] (11.0746,6.8537) rectangle (11.1144,6.95955);
\draw [color=c, fill=c] (11.1144,6.8537) rectangle (11.1542,6.95955);
\draw [color=c, fill=c] (11.1542,6.8537) rectangle (11.194,6.95955);
\draw [color=c, fill=c] (11.194,6.8537) rectangle (11.2338,6.95955);
\draw [color=c, fill=c] (11.2338,6.8537) rectangle (11.2736,6.95955);
\definecolor{c}{rgb}{0,0.733333,1};
\draw [color=c, fill=c] (11.2736,6.8537) rectangle (11.3134,6.95955);
\draw [color=c, fill=c] (11.3134,6.8537) rectangle (11.3532,6.95955);
\draw [color=c, fill=c] (11.3532,6.8537) rectangle (11.393,6.95955);
\draw [color=c, fill=c] (11.393,6.8537) rectangle (11.4328,6.95955);
\draw [color=c, fill=c] (11.4328,6.8537) rectangle (11.4726,6.95955);
\draw [color=c, fill=c] (11.4726,6.8537) rectangle (11.5124,6.95955);
\draw [color=c, fill=c] (11.5124,6.8537) rectangle (11.5522,6.95955);
\draw [color=c, fill=c] (11.5522,6.8537) rectangle (11.592,6.95955);
\draw [color=c, fill=c] (11.592,6.8537) rectangle (11.6318,6.95955);
\draw [color=c, fill=c] (11.6318,6.8537) rectangle (11.6716,6.95955);
\draw [color=c, fill=c] (11.6716,6.8537) rectangle (11.7114,6.95955);
\draw [color=c, fill=c] (11.7114,6.8537) rectangle (11.7512,6.95955);
\draw [color=c, fill=c] (11.7512,6.8537) rectangle (11.791,6.95955);
\draw [color=c, fill=c] (11.791,6.8537) rectangle (11.8308,6.95955);
\draw [color=c, fill=c] (11.8308,6.8537) rectangle (11.8706,6.95955);
\draw [color=c, fill=c] (11.8706,6.8537) rectangle (11.9104,6.95955);
\draw [color=c, fill=c] (11.9104,6.8537) rectangle (11.9502,6.95955);
\draw [color=c, fill=c] (11.9502,6.8537) rectangle (11.99,6.95955);
\draw [color=c, fill=c] (11.99,6.8537) rectangle (12.0299,6.95955);
\draw [color=c, fill=c] (12.0299,6.8537) rectangle (12.0697,6.95955);
\draw [color=c, fill=c] (12.0697,6.8537) rectangle (12.1095,6.95955);
\draw [color=c, fill=c] (12.1095,6.8537) rectangle (12.1493,6.95955);
\draw [color=c, fill=c] (12.1493,6.8537) rectangle (12.1891,6.95955);
\draw [color=c, fill=c] (12.1891,6.8537) rectangle (12.2289,6.95955);
\draw [color=c, fill=c] (12.2289,6.8537) rectangle (12.2687,6.95955);
\draw [color=c, fill=c] (12.2687,6.8537) rectangle (12.3085,6.95955);
\draw [color=c, fill=c] (12.3085,6.8537) rectangle (12.3483,6.95955);
\draw [color=c, fill=c] (12.3483,6.8537) rectangle (12.3881,6.95955);
\draw [color=c, fill=c] (12.3881,6.8537) rectangle (12.4279,6.95955);
\draw [color=c, fill=c] (12.4279,6.8537) rectangle (12.4677,6.95955);
\draw [color=c, fill=c] (12.4677,6.8537) rectangle (12.5075,6.95955);
\draw [color=c, fill=c] (12.5075,6.8537) rectangle (12.5473,6.95955);
\draw [color=c, fill=c] (12.5473,6.8537) rectangle (12.5871,6.95955);
\draw [color=c, fill=c] (12.5871,6.8537) rectangle (12.6269,6.95955);
\draw [color=c, fill=c] (12.6269,6.8537) rectangle (12.6667,6.95955);
\draw [color=c, fill=c] (12.6667,6.8537) rectangle (12.7065,6.95955);
\draw [color=c, fill=c] (12.7065,6.8537) rectangle (12.7463,6.95955);
\draw [color=c, fill=c] (12.7463,6.8537) rectangle (12.7861,6.95955);
\draw [color=c, fill=c] (12.7861,6.8537) rectangle (12.8259,6.95955);
\draw [color=c, fill=c] (12.8259,6.8537) rectangle (12.8657,6.95955);
\draw [color=c, fill=c] (12.8657,6.8537) rectangle (12.9055,6.95955);
\draw [color=c, fill=c] (12.9055,6.8537) rectangle (12.9453,6.95955);
\draw [color=c, fill=c] (12.9453,6.8537) rectangle (12.9851,6.95955);
\draw [color=c, fill=c] (12.9851,6.8537) rectangle (13.0249,6.95955);
\draw [color=c, fill=c] (13.0249,6.8537) rectangle (13.0647,6.95955);
\draw [color=c, fill=c] (13.0647,6.8537) rectangle (13.1045,6.95955);
\draw [color=c, fill=c] (13.1045,6.8537) rectangle (13.1443,6.95955);
\draw [color=c, fill=c] (13.1443,6.8537) rectangle (13.1841,6.95955);
\draw [color=c, fill=c] (13.1841,6.8537) rectangle (13.2239,6.95955);
\draw [color=c, fill=c] (13.2239,6.8537) rectangle (13.2637,6.95955);
\draw [color=c, fill=c] (13.2637,6.8537) rectangle (13.3035,6.95955);
\draw [color=c, fill=c] (13.3035,6.8537) rectangle (13.3433,6.95955);
\draw [color=c, fill=c] (13.3433,6.8537) rectangle (13.3831,6.95955);
\draw [color=c, fill=c] (13.3831,6.8537) rectangle (13.4229,6.95955);
\draw [color=c, fill=c] (13.4229,6.8537) rectangle (13.4627,6.95955);
\draw [color=c, fill=c] (13.4627,6.8537) rectangle (13.5025,6.95955);
\draw [color=c, fill=c] (13.5025,6.8537) rectangle (13.5423,6.95955);
\draw [color=c, fill=c] (13.5423,6.8537) rectangle (13.5821,6.95955);
\draw [color=c, fill=c] (13.5821,6.8537) rectangle (13.6219,6.95955);
\draw [color=c, fill=c] (13.6219,6.8537) rectangle (13.6617,6.95955);
\draw [color=c, fill=c] (13.6617,6.8537) rectangle (13.7015,6.95955);
\draw [color=c, fill=c] (13.7015,6.8537) rectangle (13.7413,6.95955);
\draw [color=c, fill=c] (13.7413,6.8537) rectangle (13.7811,6.95955);
\draw [color=c, fill=c] (13.7811,6.8537) rectangle (13.8209,6.95955);
\draw [color=c, fill=c] (13.8209,6.8537) rectangle (13.8607,6.95955);
\draw [color=c, fill=c] (13.8607,6.8537) rectangle (13.9005,6.95955);
\draw [color=c, fill=c] (13.9005,6.8537) rectangle (13.9403,6.95955);
\draw [color=c, fill=c] (13.9403,6.8537) rectangle (13.9801,6.95955);
\draw [color=c, fill=c] (13.9801,6.8537) rectangle (14.0199,6.95955);
\draw [color=c, fill=c] (14.0199,6.8537) rectangle (14.0597,6.95955);
\draw [color=c, fill=c] (14.0597,6.8537) rectangle (14.0995,6.95955);
\draw [color=c, fill=c] (14.0995,6.8537) rectangle (14.1393,6.95955);
\draw [color=c, fill=c] (14.1393,6.8537) rectangle (14.1791,6.95955);
\draw [color=c, fill=c] (14.1791,6.8537) rectangle (14.2189,6.95955);
\draw [color=c, fill=c] (14.2189,6.8537) rectangle (14.2587,6.95955);
\draw [color=c, fill=c] (14.2587,6.8537) rectangle (14.2985,6.95955);
\draw [color=c, fill=c] (14.2985,6.8537) rectangle (14.3383,6.95955);
\draw [color=c, fill=c] (14.3383,6.8537) rectangle (14.3781,6.95955);
\draw [color=c, fill=c] (14.3781,6.8537) rectangle (14.4179,6.95955);
\draw [color=c, fill=c] (14.4179,6.8537) rectangle (14.4577,6.95955);
\draw [color=c, fill=c] (14.4577,6.8537) rectangle (14.4975,6.95955);
\draw [color=c, fill=c] (14.4975,6.8537) rectangle (14.5373,6.95955);
\draw [color=c, fill=c] (14.5373,6.8537) rectangle (14.5771,6.95955);
\draw [color=c, fill=c] (14.5771,6.8537) rectangle (14.6169,6.95955);
\draw [color=c, fill=c] (14.6169,6.8537) rectangle (14.6567,6.95955);
\draw [color=c, fill=c] (14.6567,6.8537) rectangle (14.6965,6.95955);
\draw [color=c, fill=c] (14.6965,6.8537) rectangle (14.7363,6.95955);
\draw [color=c, fill=c] (14.7363,6.8537) rectangle (14.7761,6.95955);
\draw [color=c, fill=c] (14.7761,6.8537) rectangle (14.8159,6.95955);
\draw [color=c, fill=c] (14.8159,6.8537) rectangle (14.8557,6.95955);
\draw [color=c, fill=c] (14.8557,6.8537) rectangle (14.8955,6.95955);
\draw [color=c, fill=c] (14.8955,6.8537) rectangle (14.9353,6.95955);
\draw [color=c, fill=c] (14.9353,6.8537) rectangle (14.9751,6.95955);
\draw [color=c, fill=c] (14.9751,6.8537) rectangle (15.0149,6.95955);
\draw [color=c, fill=c] (15.0149,6.8537) rectangle (15.0547,6.95955);
\draw [color=c, fill=c] (15.0547,6.8537) rectangle (15.0945,6.95955);
\draw [color=c, fill=c] (15.0945,6.8537) rectangle (15.1343,6.95955);
\draw [color=c, fill=c] (15.1343,6.8537) rectangle (15.1741,6.95955);
\draw [color=c, fill=c] (15.1741,6.8537) rectangle (15.2139,6.95955);
\draw [color=c, fill=c] (15.2139,6.8537) rectangle (15.2537,6.95955);
\draw [color=c, fill=c] (15.2537,6.8537) rectangle (15.2935,6.95955);
\draw [color=c, fill=c] (15.2935,6.8537) rectangle (15.3333,6.95955);
\draw [color=c, fill=c] (15.3333,6.8537) rectangle (15.3731,6.95955);
\draw [color=c, fill=c] (15.3731,6.8537) rectangle (15.4129,6.95955);
\draw [color=c, fill=c] (15.4129,6.8537) rectangle (15.4527,6.95955);
\draw [color=c, fill=c] (15.4527,6.8537) rectangle (15.4925,6.95955);
\draw [color=c, fill=c] (15.4925,6.8537) rectangle (15.5323,6.95955);
\draw [color=c, fill=c] (15.5323,6.8537) rectangle (15.5721,6.95955);
\draw [color=c, fill=c] (15.5721,6.8537) rectangle (15.6119,6.95955);
\draw [color=c, fill=c] (15.6119,6.8537) rectangle (15.6517,6.95955);
\draw [color=c, fill=c] (15.6517,6.8537) rectangle (15.6915,6.95955);
\draw [color=c, fill=c] (15.6915,6.8537) rectangle (15.7313,6.95955);
\draw [color=c, fill=c] (15.7313,6.8537) rectangle (15.7711,6.95955);
\draw [color=c, fill=c] (15.7711,6.8537) rectangle (15.8109,6.95955);
\draw [color=c, fill=c] (15.8109,6.8537) rectangle (15.8507,6.95955);
\draw [color=c, fill=c] (15.8507,6.8537) rectangle (15.8905,6.95955);
\draw [color=c, fill=c] (15.8905,6.8537) rectangle (15.9303,6.95955);
\draw [color=c, fill=c] (15.9303,6.8537) rectangle (15.9701,6.95955);
\draw [color=c, fill=c] (15.9701,6.8537) rectangle (16.01,6.95955);
\draw [color=c, fill=c] (16.01,6.8537) rectangle (16.0498,6.95955);
\draw [color=c, fill=c] (16.0498,6.8537) rectangle (16.0896,6.95955);
\draw [color=c, fill=c] (16.0896,6.8537) rectangle (16.1294,6.95955);
\draw [color=c, fill=c] (16.1294,6.8537) rectangle (16.1692,6.95955);
\draw [color=c, fill=c] (16.1692,6.8537) rectangle (16.209,6.95955);
\draw [color=c, fill=c] (16.209,6.8537) rectangle (16.2488,6.95955);
\draw [color=c, fill=c] (16.2488,6.8537) rectangle (16.2886,6.95955);
\draw [color=c, fill=c] (16.2886,6.8537) rectangle (16.3284,6.95955);
\draw [color=c, fill=c] (16.3284,6.8537) rectangle (16.3682,6.95955);
\draw [color=c, fill=c] (16.3682,6.8537) rectangle (16.408,6.95955);
\draw [color=c, fill=c] (16.408,6.8537) rectangle (16.4478,6.95955);
\draw [color=c, fill=c] (16.4478,6.8537) rectangle (16.4876,6.95955);
\draw [color=c, fill=c] (16.4876,6.8537) rectangle (16.5274,6.95955);
\draw [color=c, fill=c] (16.5274,6.8537) rectangle (16.5672,6.95955);
\draw [color=c, fill=c] (16.5672,6.8537) rectangle (16.607,6.95955);
\draw [color=c, fill=c] (16.607,6.8537) rectangle (16.6468,6.95955);
\draw [color=c, fill=c] (16.6468,6.8537) rectangle (16.6866,6.95955);
\draw [color=c, fill=c] (16.6866,6.8537) rectangle (16.7264,6.95955);
\draw [color=c, fill=c] (16.7264,6.8537) rectangle (16.7662,6.95955);
\draw [color=c, fill=c] (16.7662,6.8537) rectangle (16.806,6.95955);
\draw [color=c, fill=c] (16.806,6.8537) rectangle (16.8458,6.95955);
\draw [color=c, fill=c] (16.8458,6.8537) rectangle (16.8856,6.95955);
\draw [color=c, fill=c] (16.8856,6.8537) rectangle (16.9254,6.95955);
\draw [color=c, fill=c] (16.9254,6.8537) rectangle (16.9652,6.95955);
\draw [color=c, fill=c] (16.9652,6.8537) rectangle (17.005,6.95955);
\draw [color=c, fill=c] (17.005,6.8537) rectangle (17.0448,6.95955);
\draw [color=c, fill=c] (17.0448,6.8537) rectangle (17.0846,6.95955);
\draw [color=c, fill=c] (17.0846,6.8537) rectangle (17.1244,6.95955);
\draw [color=c, fill=c] (17.1244,6.8537) rectangle (17.1642,6.95955);
\draw [color=c, fill=c] (17.1642,6.8537) rectangle (17.204,6.95955);
\draw [color=c, fill=c] (17.204,6.8537) rectangle (17.2438,6.95955);
\draw [color=c, fill=c] (17.2438,6.8537) rectangle (17.2836,6.95955);
\draw [color=c, fill=c] (17.2836,6.8537) rectangle (17.3234,6.95955);
\draw [color=c, fill=c] (17.3234,6.8537) rectangle (17.3632,6.95955);
\draw [color=c, fill=c] (17.3632,6.8537) rectangle (17.403,6.95955);
\draw [color=c, fill=c] (17.403,6.8537) rectangle (17.4428,6.95955);
\draw [color=c, fill=c] (17.4428,6.8537) rectangle (17.4826,6.95955);
\draw [color=c, fill=c] (17.4826,6.8537) rectangle (17.5224,6.95955);
\draw [color=c, fill=c] (17.5224,6.8537) rectangle (17.5622,6.95955);
\draw [color=c, fill=c] (17.5622,6.8537) rectangle (17.602,6.95955);
\draw [color=c, fill=c] (17.602,6.8537) rectangle (17.6418,6.95955);
\draw [color=c, fill=c] (17.6418,6.8537) rectangle (17.6816,6.95955);
\draw [color=c, fill=c] (17.6816,6.8537) rectangle (17.7214,6.95955);
\draw [color=c, fill=c] (17.7214,6.8537) rectangle (17.7612,6.95955);
\draw [color=c, fill=c] (17.7612,6.8537) rectangle (17.801,6.95955);
\draw [color=c, fill=c] (17.801,6.8537) rectangle (17.8408,6.95955);
\draw [color=c, fill=c] (17.8408,6.8537) rectangle (17.8806,6.95955);
\draw [color=c, fill=c] (17.8806,6.8537) rectangle (17.9204,6.95955);
\draw [color=c, fill=c] (17.9204,6.8537) rectangle (17.9602,6.95955);
\draw [color=c, fill=c] (17.9602,6.8537) rectangle (18,6.95955);
\definecolor{c}{rgb}{0.2,0,1};
\draw [color=c, fill=c] (2,6.95955) rectangle (2.0398,7.0654);
\draw [color=c, fill=c] (2.0398,6.95955) rectangle (2.0796,7.0654);
\draw [color=c, fill=c] (2.0796,6.95955) rectangle (2.1194,7.0654);
\draw [color=c, fill=c] (2.1194,6.95955) rectangle (2.1592,7.0654);
\draw [color=c, fill=c] (2.1592,6.95955) rectangle (2.19901,7.0654);
\draw [color=c, fill=c] (2.19901,6.95955) rectangle (2.23881,7.0654);
\draw [color=c, fill=c] (2.23881,6.95955) rectangle (2.27861,7.0654);
\draw [color=c, fill=c] (2.27861,6.95955) rectangle (2.31841,7.0654);
\draw [color=c, fill=c] (2.31841,6.95955) rectangle (2.35821,7.0654);
\draw [color=c, fill=c] (2.35821,6.95955) rectangle (2.39801,7.0654);
\draw [color=c, fill=c] (2.39801,6.95955) rectangle (2.43781,7.0654);
\draw [color=c, fill=c] (2.43781,6.95955) rectangle (2.47761,7.0654);
\draw [color=c, fill=c] (2.47761,6.95955) rectangle (2.51741,7.0654);
\draw [color=c, fill=c] (2.51741,6.95955) rectangle (2.55721,7.0654);
\draw [color=c, fill=c] (2.55721,6.95955) rectangle (2.59702,7.0654);
\draw [color=c, fill=c] (2.59702,6.95955) rectangle (2.63682,7.0654);
\draw [color=c, fill=c] (2.63682,6.95955) rectangle (2.67662,7.0654);
\draw [color=c, fill=c] (2.67662,6.95955) rectangle (2.71642,7.0654);
\draw [color=c, fill=c] (2.71642,6.95955) rectangle (2.75622,7.0654);
\draw [color=c, fill=c] (2.75622,6.95955) rectangle (2.79602,7.0654);
\draw [color=c, fill=c] (2.79602,6.95955) rectangle (2.83582,7.0654);
\draw [color=c, fill=c] (2.83582,6.95955) rectangle (2.87562,7.0654);
\draw [color=c, fill=c] (2.87562,6.95955) rectangle (2.91542,7.0654);
\draw [color=c, fill=c] (2.91542,6.95955) rectangle (2.95522,7.0654);
\draw [color=c, fill=c] (2.95522,6.95955) rectangle (2.99502,7.0654);
\draw [color=c, fill=c] (2.99502,6.95955) rectangle (3.03483,7.0654);
\draw [color=c, fill=c] (3.03483,6.95955) rectangle (3.07463,7.0654);
\draw [color=c, fill=c] (3.07463,6.95955) rectangle (3.11443,7.0654);
\draw [color=c, fill=c] (3.11443,6.95955) rectangle (3.15423,7.0654);
\draw [color=c, fill=c] (3.15423,6.95955) rectangle (3.19403,7.0654);
\draw [color=c, fill=c] (3.19403,6.95955) rectangle (3.23383,7.0654);
\draw [color=c, fill=c] (3.23383,6.95955) rectangle (3.27363,7.0654);
\draw [color=c, fill=c] (3.27363,6.95955) rectangle (3.31343,7.0654);
\draw [color=c, fill=c] (3.31343,6.95955) rectangle (3.35323,7.0654);
\draw [color=c, fill=c] (3.35323,6.95955) rectangle (3.39303,7.0654);
\draw [color=c, fill=c] (3.39303,6.95955) rectangle (3.43284,7.0654);
\draw [color=c, fill=c] (3.43284,6.95955) rectangle (3.47264,7.0654);
\draw [color=c, fill=c] (3.47264,6.95955) rectangle (3.51244,7.0654);
\draw [color=c, fill=c] (3.51244,6.95955) rectangle (3.55224,7.0654);
\draw [color=c, fill=c] (3.55224,6.95955) rectangle (3.59204,7.0654);
\draw [color=c, fill=c] (3.59204,6.95955) rectangle (3.63184,7.0654);
\draw [color=c, fill=c] (3.63184,6.95955) rectangle (3.67164,7.0654);
\draw [color=c, fill=c] (3.67164,6.95955) rectangle (3.71144,7.0654);
\draw [color=c, fill=c] (3.71144,6.95955) rectangle (3.75124,7.0654);
\draw [color=c, fill=c] (3.75124,6.95955) rectangle (3.79104,7.0654);
\draw [color=c, fill=c] (3.79104,6.95955) rectangle (3.83085,7.0654);
\draw [color=c, fill=c] (3.83085,6.95955) rectangle (3.87065,7.0654);
\draw [color=c, fill=c] (3.87065,6.95955) rectangle (3.91045,7.0654);
\draw [color=c, fill=c] (3.91045,6.95955) rectangle (3.95025,7.0654);
\draw [color=c, fill=c] (3.95025,6.95955) rectangle (3.99005,7.0654);
\draw [color=c, fill=c] (3.99005,6.95955) rectangle (4.02985,7.0654);
\draw [color=c, fill=c] (4.02985,6.95955) rectangle (4.06965,7.0654);
\draw [color=c, fill=c] (4.06965,6.95955) rectangle (4.10945,7.0654);
\draw [color=c, fill=c] (4.10945,6.95955) rectangle (4.14925,7.0654);
\draw [color=c, fill=c] (4.14925,6.95955) rectangle (4.18905,7.0654);
\draw [color=c, fill=c] (4.18905,6.95955) rectangle (4.22886,7.0654);
\draw [color=c, fill=c] (4.22886,6.95955) rectangle (4.26866,7.0654);
\draw [color=c, fill=c] (4.26866,6.95955) rectangle (4.30846,7.0654);
\draw [color=c, fill=c] (4.30846,6.95955) rectangle (4.34826,7.0654);
\draw [color=c, fill=c] (4.34826,6.95955) rectangle (4.38806,7.0654);
\draw [color=c, fill=c] (4.38806,6.95955) rectangle (4.42786,7.0654);
\draw [color=c, fill=c] (4.42786,6.95955) rectangle (4.46766,7.0654);
\draw [color=c, fill=c] (4.46766,6.95955) rectangle (4.50746,7.0654);
\draw [color=c, fill=c] (4.50746,6.95955) rectangle (4.54726,7.0654);
\draw [color=c, fill=c] (4.54726,6.95955) rectangle (4.58706,7.0654);
\draw [color=c, fill=c] (4.58706,6.95955) rectangle (4.62687,7.0654);
\draw [color=c, fill=c] (4.62687,6.95955) rectangle (4.66667,7.0654);
\draw [color=c, fill=c] (4.66667,6.95955) rectangle (4.70647,7.0654);
\draw [color=c, fill=c] (4.70647,6.95955) rectangle (4.74627,7.0654);
\draw [color=c, fill=c] (4.74627,6.95955) rectangle (4.78607,7.0654);
\draw [color=c, fill=c] (4.78607,6.95955) rectangle (4.82587,7.0654);
\draw [color=c, fill=c] (4.82587,6.95955) rectangle (4.86567,7.0654);
\draw [color=c, fill=c] (4.86567,6.95955) rectangle (4.90547,7.0654);
\draw [color=c, fill=c] (4.90547,6.95955) rectangle (4.94527,7.0654);
\draw [color=c, fill=c] (4.94527,6.95955) rectangle (4.98507,7.0654);
\draw [color=c, fill=c] (4.98507,6.95955) rectangle (5.02488,7.0654);
\draw [color=c, fill=c] (5.02488,6.95955) rectangle (5.06468,7.0654);
\draw [color=c, fill=c] (5.06468,6.95955) rectangle (5.10448,7.0654);
\draw [color=c, fill=c] (5.10448,6.95955) rectangle (5.14428,7.0654);
\draw [color=c, fill=c] (5.14428,6.95955) rectangle (5.18408,7.0654);
\draw [color=c, fill=c] (5.18408,6.95955) rectangle (5.22388,7.0654);
\draw [color=c, fill=c] (5.22388,6.95955) rectangle (5.26368,7.0654);
\draw [color=c, fill=c] (5.26368,6.95955) rectangle (5.30348,7.0654);
\draw [color=c, fill=c] (5.30348,6.95955) rectangle (5.34328,7.0654);
\draw [color=c, fill=c] (5.34328,6.95955) rectangle (5.38308,7.0654);
\draw [color=c, fill=c] (5.38308,6.95955) rectangle (5.42289,7.0654);
\draw [color=c, fill=c] (5.42289,6.95955) rectangle (5.46269,7.0654);
\draw [color=c, fill=c] (5.46269,6.95955) rectangle (5.50249,7.0654);
\draw [color=c, fill=c] (5.50249,6.95955) rectangle (5.54229,7.0654);
\draw [color=c, fill=c] (5.54229,6.95955) rectangle (5.58209,7.0654);
\draw [color=c, fill=c] (5.58209,6.95955) rectangle (5.62189,7.0654);
\draw [color=c, fill=c] (5.62189,6.95955) rectangle (5.66169,7.0654);
\draw [color=c, fill=c] (5.66169,6.95955) rectangle (5.70149,7.0654);
\draw [color=c, fill=c] (5.70149,6.95955) rectangle (5.74129,7.0654);
\draw [color=c, fill=c] (5.74129,6.95955) rectangle (5.78109,7.0654);
\draw [color=c, fill=c] (5.78109,6.95955) rectangle (5.8209,7.0654);
\draw [color=c, fill=c] (5.8209,6.95955) rectangle (5.8607,7.0654);
\draw [color=c, fill=c] (5.8607,6.95955) rectangle (5.9005,7.0654);
\draw [color=c, fill=c] (5.9005,6.95955) rectangle (5.9403,7.0654);
\draw [color=c, fill=c] (5.9403,6.95955) rectangle (5.9801,7.0654);
\draw [color=c, fill=c] (5.9801,6.95955) rectangle (6.0199,7.0654);
\draw [color=c, fill=c] (6.0199,6.95955) rectangle (6.0597,7.0654);
\draw [color=c, fill=c] (6.0597,6.95955) rectangle (6.0995,7.0654);
\draw [color=c, fill=c] (6.0995,6.95955) rectangle (6.1393,7.0654);
\draw [color=c, fill=c] (6.1393,6.95955) rectangle (6.1791,7.0654);
\draw [color=c, fill=c] (6.1791,6.95955) rectangle (6.21891,7.0654);
\draw [color=c, fill=c] (6.21891,6.95955) rectangle (6.25871,7.0654);
\draw [color=c, fill=c] (6.25871,6.95955) rectangle (6.29851,7.0654);
\draw [color=c, fill=c] (6.29851,6.95955) rectangle (6.33831,7.0654);
\draw [color=c, fill=c] (6.33831,6.95955) rectangle (6.37811,7.0654);
\draw [color=c, fill=c] (6.37811,6.95955) rectangle (6.41791,7.0654);
\draw [color=c, fill=c] (6.41791,6.95955) rectangle (6.45771,7.0654);
\draw [color=c, fill=c] (6.45771,6.95955) rectangle (6.49751,7.0654);
\draw [color=c, fill=c] (6.49751,6.95955) rectangle (6.53731,7.0654);
\draw [color=c, fill=c] (6.53731,6.95955) rectangle (6.57711,7.0654);
\draw [color=c, fill=c] (6.57711,6.95955) rectangle (6.61692,7.0654);
\draw [color=c, fill=c] (6.61692,6.95955) rectangle (6.65672,7.0654);
\draw [color=c, fill=c] (6.65672,6.95955) rectangle (6.69652,7.0654);
\draw [color=c, fill=c] (6.69652,6.95955) rectangle (6.73632,7.0654);
\draw [color=c, fill=c] (6.73632,6.95955) rectangle (6.77612,7.0654);
\draw [color=c, fill=c] (6.77612,6.95955) rectangle (6.81592,7.0654);
\draw [color=c, fill=c] (6.81592,6.95955) rectangle (6.85572,7.0654);
\draw [color=c, fill=c] (6.85572,6.95955) rectangle (6.89552,7.0654);
\draw [color=c, fill=c] (6.89552,6.95955) rectangle (6.93532,7.0654);
\draw [color=c, fill=c] (6.93532,6.95955) rectangle (6.97512,7.0654);
\draw [color=c, fill=c] (6.97512,6.95955) rectangle (7.01493,7.0654);
\draw [color=c, fill=c] (7.01493,6.95955) rectangle (7.05473,7.0654);
\draw [color=c, fill=c] (7.05473,6.95955) rectangle (7.09453,7.0654);
\draw [color=c, fill=c] (7.09453,6.95955) rectangle (7.13433,7.0654);
\draw [color=c, fill=c] (7.13433,6.95955) rectangle (7.17413,7.0654);
\draw [color=c, fill=c] (7.17413,6.95955) rectangle (7.21393,7.0654);
\draw [color=c, fill=c] (7.21393,6.95955) rectangle (7.25373,7.0654);
\draw [color=c, fill=c] (7.25373,6.95955) rectangle (7.29353,7.0654);
\draw [color=c, fill=c] (7.29353,6.95955) rectangle (7.33333,7.0654);
\draw [color=c, fill=c] (7.33333,6.95955) rectangle (7.37313,7.0654);
\draw [color=c, fill=c] (7.37313,6.95955) rectangle (7.41294,7.0654);
\draw [color=c, fill=c] (7.41294,6.95955) rectangle (7.45274,7.0654);
\draw [color=c, fill=c] (7.45274,6.95955) rectangle (7.49254,7.0654);
\draw [color=c, fill=c] (7.49254,6.95955) rectangle (7.53234,7.0654);
\draw [color=c, fill=c] (7.53234,6.95955) rectangle (7.57214,7.0654);
\draw [color=c, fill=c] (7.57214,6.95955) rectangle (7.61194,7.0654);
\draw [color=c, fill=c] (7.61194,6.95955) rectangle (7.65174,7.0654);
\draw [color=c, fill=c] (7.65174,6.95955) rectangle (7.69154,7.0654);
\draw [color=c, fill=c] (7.69154,6.95955) rectangle (7.73134,7.0654);
\draw [color=c, fill=c] (7.73134,6.95955) rectangle (7.77114,7.0654);
\draw [color=c, fill=c] (7.77114,6.95955) rectangle (7.81095,7.0654);
\draw [color=c, fill=c] (7.81095,6.95955) rectangle (7.85075,7.0654);
\draw [color=c, fill=c] (7.85075,6.95955) rectangle (7.89055,7.0654);
\draw [color=c, fill=c] (7.89055,6.95955) rectangle (7.93035,7.0654);
\draw [color=c, fill=c] (7.93035,6.95955) rectangle (7.97015,7.0654);
\draw [color=c, fill=c] (7.97015,6.95955) rectangle (8.00995,7.0654);
\draw [color=c, fill=c] (8.00995,6.95955) rectangle (8.04975,7.0654);
\draw [color=c, fill=c] (8.04975,6.95955) rectangle (8.08955,7.0654);
\draw [color=c, fill=c] (8.08955,6.95955) rectangle (8.12935,7.0654);
\draw [color=c, fill=c] (8.12935,6.95955) rectangle (8.16915,7.0654);
\draw [color=c, fill=c] (8.16915,6.95955) rectangle (8.20895,7.0654);
\draw [color=c, fill=c] (8.20895,6.95955) rectangle (8.24876,7.0654);
\definecolor{c}{rgb}{0,0.0800001,1};
\draw [color=c, fill=c] (8.24876,6.95955) rectangle (8.28856,7.0654);
\draw [color=c, fill=c] (8.28856,6.95955) rectangle (8.32836,7.0654);
\draw [color=c, fill=c] (8.32836,6.95955) rectangle (8.36816,7.0654);
\draw [color=c, fill=c] (8.36816,6.95955) rectangle (8.40796,7.0654);
\draw [color=c, fill=c] (8.40796,6.95955) rectangle (8.44776,7.0654);
\draw [color=c, fill=c] (8.44776,6.95955) rectangle (8.48756,7.0654);
\draw [color=c, fill=c] (8.48756,6.95955) rectangle (8.52736,7.0654);
\draw [color=c, fill=c] (8.52736,6.95955) rectangle (8.56716,7.0654);
\draw [color=c, fill=c] (8.56716,6.95955) rectangle (8.60697,7.0654);
\draw [color=c, fill=c] (8.60697,6.95955) rectangle (8.64677,7.0654);
\draw [color=c, fill=c] (8.64677,6.95955) rectangle (8.68657,7.0654);
\draw [color=c, fill=c] (8.68657,6.95955) rectangle (8.72637,7.0654);
\draw [color=c, fill=c] (8.72637,6.95955) rectangle (8.76617,7.0654);
\draw [color=c, fill=c] (8.76617,6.95955) rectangle (8.80597,7.0654);
\draw [color=c, fill=c] (8.80597,6.95955) rectangle (8.84577,7.0654);
\draw [color=c, fill=c] (8.84577,6.95955) rectangle (8.88557,7.0654);
\draw [color=c, fill=c] (8.88557,6.95955) rectangle (8.92537,7.0654);
\draw [color=c, fill=c] (8.92537,6.95955) rectangle (8.96517,7.0654);
\draw [color=c, fill=c] (8.96517,6.95955) rectangle (9.00498,7.0654);
\draw [color=c, fill=c] (9.00498,6.95955) rectangle (9.04478,7.0654);
\draw [color=c, fill=c] (9.04478,6.95955) rectangle (9.08458,7.0654);
\draw [color=c, fill=c] (9.08458,6.95955) rectangle (9.12438,7.0654);
\draw [color=c, fill=c] (9.12438,6.95955) rectangle (9.16418,7.0654);
\draw [color=c, fill=c] (9.16418,6.95955) rectangle (9.20398,7.0654);
\draw [color=c, fill=c] (9.20398,6.95955) rectangle (9.24378,7.0654);
\draw [color=c, fill=c] (9.24378,6.95955) rectangle (9.28358,7.0654);
\draw [color=c, fill=c] (9.28358,6.95955) rectangle (9.32338,7.0654);
\draw [color=c, fill=c] (9.32338,6.95955) rectangle (9.36318,7.0654);
\draw [color=c, fill=c] (9.36318,6.95955) rectangle (9.40298,7.0654);
\draw [color=c, fill=c] (9.40298,6.95955) rectangle (9.44279,7.0654);
\draw [color=c, fill=c] (9.44279,6.95955) rectangle (9.48259,7.0654);
\draw [color=c, fill=c] (9.48259,6.95955) rectangle (9.52239,7.0654);
\draw [color=c, fill=c] (9.52239,6.95955) rectangle (9.56219,7.0654);
\draw [color=c, fill=c] (9.56219,6.95955) rectangle (9.60199,7.0654);
\definecolor{c}{rgb}{0,0.266667,1};
\draw [color=c, fill=c] (9.60199,6.95955) rectangle (9.64179,7.0654);
\draw [color=c, fill=c] (9.64179,6.95955) rectangle (9.68159,7.0654);
\draw [color=c, fill=c] (9.68159,6.95955) rectangle (9.72139,7.0654);
\draw [color=c, fill=c] (9.72139,6.95955) rectangle (9.76119,7.0654);
\draw [color=c, fill=c] (9.76119,6.95955) rectangle (9.80099,7.0654);
\draw [color=c, fill=c] (9.80099,6.95955) rectangle (9.8408,7.0654);
\draw [color=c, fill=c] (9.8408,6.95955) rectangle (9.8806,7.0654);
\draw [color=c, fill=c] (9.8806,6.95955) rectangle (9.9204,7.0654);
\draw [color=c, fill=c] (9.9204,6.95955) rectangle (9.9602,7.0654);
\draw [color=c, fill=c] (9.9602,6.95955) rectangle (10,7.0654);
\draw [color=c, fill=c] (10,6.95955) rectangle (10.0398,7.0654);
\draw [color=c, fill=c] (10.0398,6.95955) rectangle (10.0796,7.0654);
\draw [color=c, fill=c] (10.0796,6.95955) rectangle (10.1194,7.0654);
\draw [color=c, fill=c] (10.1194,6.95955) rectangle (10.1592,7.0654);
\draw [color=c, fill=c] (10.1592,6.95955) rectangle (10.199,7.0654);
\draw [color=c, fill=c] (10.199,6.95955) rectangle (10.2388,7.0654);
\definecolor{c}{rgb}{0,0.546666,1};
\draw [color=c, fill=c] (10.2388,6.95955) rectangle (10.2786,7.0654);
\draw [color=c, fill=c] (10.2786,6.95955) rectangle (10.3184,7.0654);
\draw [color=c, fill=c] (10.3184,6.95955) rectangle (10.3582,7.0654);
\draw [color=c, fill=c] (10.3582,6.95955) rectangle (10.398,7.0654);
\draw [color=c, fill=c] (10.398,6.95955) rectangle (10.4378,7.0654);
\draw [color=c, fill=c] (10.4378,6.95955) rectangle (10.4776,7.0654);
\draw [color=c, fill=c] (10.4776,6.95955) rectangle (10.5174,7.0654);
\draw [color=c, fill=c] (10.5174,6.95955) rectangle (10.5572,7.0654);
\draw [color=c, fill=c] (10.5572,6.95955) rectangle (10.597,7.0654);
\draw [color=c, fill=c] (10.597,6.95955) rectangle (10.6368,7.0654);
\draw [color=c, fill=c] (10.6368,6.95955) rectangle (10.6766,7.0654);
\draw [color=c, fill=c] (10.6766,6.95955) rectangle (10.7164,7.0654);
\draw [color=c, fill=c] (10.7164,6.95955) rectangle (10.7562,7.0654);
\draw [color=c, fill=c] (10.7562,6.95955) rectangle (10.796,7.0654);
\draw [color=c, fill=c] (10.796,6.95955) rectangle (10.8358,7.0654);
\draw [color=c, fill=c] (10.8358,6.95955) rectangle (10.8756,7.0654);
\draw [color=c, fill=c] (10.8756,6.95955) rectangle (10.9154,7.0654);
\draw [color=c, fill=c] (10.9154,6.95955) rectangle (10.9552,7.0654);
\draw [color=c, fill=c] (10.9552,6.95955) rectangle (10.995,7.0654);
\draw [color=c, fill=c] (10.995,6.95955) rectangle (11.0348,7.0654);
\draw [color=c, fill=c] (11.0348,6.95955) rectangle (11.0746,7.0654);
\draw [color=c, fill=c] (11.0746,6.95955) rectangle (11.1144,7.0654);
\draw [color=c, fill=c] (11.1144,6.95955) rectangle (11.1542,7.0654);
\draw [color=c, fill=c] (11.1542,6.95955) rectangle (11.194,7.0654);
\draw [color=c, fill=c] (11.194,6.95955) rectangle (11.2338,7.0654);
\draw [color=c, fill=c] (11.2338,6.95955) rectangle (11.2736,7.0654);
\draw [color=c, fill=c] (11.2736,6.95955) rectangle (11.3134,7.0654);
\draw [color=c, fill=c] (11.3134,6.95955) rectangle (11.3532,7.0654);
\definecolor{c}{rgb}{0,0.733333,1};
\draw [color=c, fill=c] (11.3532,6.95955) rectangle (11.393,7.0654);
\draw [color=c, fill=c] (11.393,6.95955) rectangle (11.4328,7.0654);
\draw [color=c, fill=c] (11.4328,6.95955) rectangle (11.4726,7.0654);
\draw [color=c, fill=c] (11.4726,6.95955) rectangle (11.5124,7.0654);
\draw [color=c, fill=c] (11.5124,6.95955) rectangle (11.5522,7.0654);
\draw [color=c, fill=c] (11.5522,6.95955) rectangle (11.592,7.0654);
\draw [color=c, fill=c] (11.592,6.95955) rectangle (11.6318,7.0654);
\draw [color=c, fill=c] (11.6318,6.95955) rectangle (11.6716,7.0654);
\draw [color=c, fill=c] (11.6716,6.95955) rectangle (11.7114,7.0654);
\draw [color=c, fill=c] (11.7114,6.95955) rectangle (11.7512,7.0654);
\draw [color=c, fill=c] (11.7512,6.95955) rectangle (11.791,7.0654);
\draw [color=c, fill=c] (11.791,6.95955) rectangle (11.8308,7.0654);
\draw [color=c, fill=c] (11.8308,6.95955) rectangle (11.8706,7.0654);
\draw [color=c, fill=c] (11.8706,6.95955) rectangle (11.9104,7.0654);
\draw [color=c, fill=c] (11.9104,6.95955) rectangle (11.9502,7.0654);
\draw [color=c, fill=c] (11.9502,6.95955) rectangle (11.99,7.0654);
\draw [color=c, fill=c] (11.99,6.95955) rectangle (12.0299,7.0654);
\draw [color=c, fill=c] (12.0299,6.95955) rectangle (12.0697,7.0654);
\draw [color=c, fill=c] (12.0697,6.95955) rectangle (12.1095,7.0654);
\draw [color=c, fill=c] (12.1095,6.95955) rectangle (12.1493,7.0654);
\draw [color=c, fill=c] (12.1493,6.95955) rectangle (12.1891,7.0654);
\draw [color=c, fill=c] (12.1891,6.95955) rectangle (12.2289,7.0654);
\draw [color=c, fill=c] (12.2289,6.95955) rectangle (12.2687,7.0654);
\draw [color=c, fill=c] (12.2687,6.95955) rectangle (12.3085,7.0654);
\draw [color=c, fill=c] (12.3085,6.95955) rectangle (12.3483,7.0654);
\draw [color=c, fill=c] (12.3483,6.95955) rectangle (12.3881,7.0654);
\draw [color=c, fill=c] (12.3881,6.95955) rectangle (12.4279,7.0654);
\draw [color=c, fill=c] (12.4279,6.95955) rectangle (12.4677,7.0654);
\draw [color=c, fill=c] (12.4677,6.95955) rectangle (12.5075,7.0654);
\draw [color=c, fill=c] (12.5075,6.95955) rectangle (12.5473,7.0654);
\draw [color=c, fill=c] (12.5473,6.95955) rectangle (12.5871,7.0654);
\draw [color=c, fill=c] (12.5871,6.95955) rectangle (12.6269,7.0654);
\draw [color=c, fill=c] (12.6269,6.95955) rectangle (12.6667,7.0654);
\draw [color=c, fill=c] (12.6667,6.95955) rectangle (12.7065,7.0654);
\draw [color=c, fill=c] (12.7065,6.95955) rectangle (12.7463,7.0654);
\draw [color=c, fill=c] (12.7463,6.95955) rectangle (12.7861,7.0654);
\draw [color=c, fill=c] (12.7861,6.95955) rectangle (12.8259,7.0654);
\draw [color=c, fill=c] (12.8259,6.95955) rectangle (12.8657,7.0654);
\draw [color=c, fill=c] (12.8657,6.95955) rectangle (12.9055,7.0654);
\draw [color=c, fill=c] (12.9055,6.95955) rectangle (12.9453,7.0654);
\draw [color=c, fill=c] (12.9453,6.95955) rectangle (12.9851,7.0654);
\draw [color=c, fill=c] (12.9851,6.95955) rectangle (13.0249,7.0654);
\draw [color=c, fill=c] (13.0249,6.95955) rectangle (13.0647,7.0654);
\draw [color=c, fill=c] (13.0647,6.95955) rectangle (13.1045,7.0654);
\draw [color=c, fill=c] (13.1045,6.95955) rectangle (13.1443,7.0654);
\draw [color=c, fill=c] (13.1443,6.95955) rectangle (13.1841,7.0654);
\draw [color=c, fill=c] (13.1841,6.95955) rectangle (13.2239,7.0654);
\draw [color=c, fill=c] (13.2239,6.95955) rectangle (13.2637,7.0654);
\draw [color=c, fill=c] (13.2637,6.95955) rectangle (13.3035,7.0654);
\draw [color=c, fill=c] (13.3035,6.95955) rectangle (13.3433,7.0654);
\draw [color=c, fill=c] (13.3433,6.95955) rectangle (13.3831,7.0654);
\draw [color=c, fill=c] (13.3831,6.95955) rectangle (13.4229,7.0654);
\draw [color=c, fill=c] (13.4229,6.95955) rectangle (13.4627,7.0654);
\draw [color=c, fill=c] (13.4627,6.95955) rectangle (13.5025,7.0654);
\draw [color=c, fill=c] (13.5025,6.95955) rectangle (13.5423,7.0654);
\draw [color=c, fill=c] (13.5423,6.95955) rectangle (13.5821,7.0654);
\draw [color=c, fill=c] (13.5821,6.95955) rectangle (13.6219,7.0654);
\draw [color=c, fill=c] (13.6219,6.95955) rectangle (13.6617,7.0654);
\draw [color=c, fill=c] (13.6617,6.95955) rectangle (13.7015,7.0654);
\draw [color=c, fill=c] (13.7015,6.95955) rectangle (13.7413,7.0654);
\draw [color=c, fill=c] (13.7413,6.95955) rectangle (13.7811,7.0654);
\draw [color=c, fill=c] (13.7811,6.95955) rectangle (13.8209,7.0654);
\draw [color=c, fill=c] (13.8209,6.95955) rectangle (13.8607,7.0654);
\draw [color=c, fill=c] (13.8607,6.95955) rectangle (13.9005,7.0654);
\draw [color=c, fill=c] (13.9005,6.95955) rectangle (13.9403,7.0654);
\draw [color=c, fill=c] (13.9403,6.95955) rectangle (13.9801,7.0654);
\draw [color=c, fill=c] (13.9801,6.95955) rectangle (14.0199,7.0654);
\draw [color=c, fill=c] (14.0199,6.95955) rectangle (14.0597,7.0654);
\draw [color=c, fill=c] (14.0597,6.95955) rectangle (14.0995,7.0654);
\draw [color=c, fill=c] (14.0995,6.95955) rectangle (14.1393,7.0654);
\draw [color=c, fill=c] (14.1393,6.95955) rectangle (14.1791,7.0654);
\draw [color=c, fill=c] (14.1791,6.95955) rectangle (14.2189,7.0654);
\draw [color=c, fill=c] (14.2189,6.95955) rectangle (14.2587,7.0654);
\draw [color=c, fill=c] (14.2587,6.95955) rectangle (14.2985,7.0654);
\draw [color=c, fill=c] (14.2985,6.95955) rectangle (14.3383,7.0654);
\draw [color=c, fill=c] (14.3383,6.95955) rectangle (14.3781,7.0654);
\draw [color=c, fill=c] (14.3781,6.95955) rectangle (14.4179,7.0654);
\draw [color=c, fill=c] (14.4179,6.95955) rectangle (14.4577,7.0654);
\draw [color=c, fill=c] (14.4577,6.95955) rectangle (14.4975,7.0654);
\draw [color=c, fill=c] (14.4975,6.95955) rectangle (14.5373,7.0654);
\draw [color=c, fill=c] (14.5373,6.95955) rectangle (14.5771,7.0654);
\draw [color=c, fill=c] (14.5771,6.95955) rectangle (14.6169,7.0654);
\draw [color=c, fill=c] (14.6169,6.95955) rectangle (14.6567,7.0654);
\draw [color=c, fill=c] (14.6567,6.95955) rectangle (14.6965,7.0654);
\draw [color=c, fill=c] (14.6965,6.95955) rectangle (14.7363,7.0654);
\draw [color=c, fill=c] (14.7363,6.95955) rectangle (14.7761,7.0654);
\draw [color=c, fill=c] (14.7761,6.95955) rectangle (14.8159,7.0654);
\draw [color=c, fill=c] (14.8159,6.95955) rectangle (14.8557,7.0654);
\draw [color=c, fill=c] (14.8557,6.95955) rectangle (14.8955,7.0654);
\draw [color=c, fill=c] (14.8955,6.95955) rectangle (14.9353,7.0654);
\draw [color=c, fill=c] (14.9353,6.95955) rectangle (14.9751,7.0654);
\draw [color=c, fill=c] (14.9751,6.95955) rectangle (15.0149,7.0654);
\draw [color=c, fill=c] (15.0149,6.95955) rectangle (15.0547,7.0654);
\draw [color=c, fill=c] (15.0547,6.95955) rectangle (15.0945,7.0654);
\draw [color=c, fill=c] (15.0945,6.95955) rectangle (15.1343,7.0654);
\draw [color=c, fill=c] (15.1343,6.95955) rectangle (15.1741,7.0654);
\draw [color=c, fill=c] (15.1741,6.95955) rectangle (15.2139,7.0654);
\draw [color=c, fill=c] (15.2139,6.95955) rectangle (15.2537,7.0654);
\draw [color=c, fill=c] (15.2537,6.95955) rectangle (15.2935,7.0654);
\draw [color=c, fill=c] (15.2935,6.95955) rectangle (15.3333,7.0654);
\draw [color=c, fill=c] (15.3333,6.95955) rectangle (15.3731,7.0654);
\draw [color=c, fill=c] (15.3731,6.95955) rectangle (15.4129,7.0654);
\draw [color=c, fill=c] (15.4129,6.95955) rectangle (15.4527,7.0654);
\draw [color=c, fill=c] (15.4527,6.95955) rectangle (15.4925,7.0654);
\draw [color=c, fill=c] (15.4925,6.95955) rectangle (15.5323,7.0654);
\draw [color=c, fill=c] (15.5323,6.95955) rectangle (15.5721,7.0654);
\draw [color=c, fill=c] (15.5721,6.95955) rectangle (15.6119,7.0654);
\draw [color=c, fill=c] (15.6119,6.95955) rectangle (15.6517,7.0654);
\draw [color=c, fill=c] (15.6517,6.95955) rectangle (15.6915,7.0654);
\draw [color=c, fill=c] (15.6915,6.95955) rectangle (15.7313,7.0654);
\draw [color=c, fill=c] (15.7313,6.95955) rectangle (15.7711,7.0654);
\draw [color=c, fill=c] (15.7711,6.95955) rectangle (15.8109,7.0654);
\draw [color=c, fill=c] (15.8109,6.95955) rectangle (15.8507,7.0654);
\draw [color=c, fill=c] (15.8507,6.95955) rectangle (15.8905,7.0654);
\draw [color=c, fill=c] (15.8905,6.95955) rectangle (15.9303,7.0654);
\draw [color=c, fill=c] (15.9303,6.95955) rectangle (15.9701,7.0654);
\draw [color=c, fill=c] (15.9701,6.95955) rectangle (16.01,7.0654);
\draw [color=c, fill=c] (16.01,6.95955) rectangle (16.0498,7.0654);
\draw [color=c, fill=c] (16.0498,6.95955) rectangle (16.0896,7.0654);
\draw [color=c, fill=c] (16.0896,6.95955) rectangle (16.1294,7.0654);
\draw [color=c, fill=c] (16.1294,6.95955) rectangle (16.1692,7.0654);
\draw [color=c, fill=c] (16.1692,6.95955) rectangle (16.209,7.0654);
\draw [color=c, fill=c] (16.209,6.95955) rectangle (16.2488,7.0654);
\draw [color=c, fill=c] (16.2488,6.95955) rectangle (16.2886,7.0654);
\draw [color=c, fill=c] (16.2886,6.95955) rectangle (16.3284,7.0654);
\draw [color=c, fill=c] (16.3284,6.95955) rectangle (16.3682,7.0654);
\draw [color=c, fill=c] (16.3682,6.95955) rectangle (16.408,7.0654);
\draw [color=c, fill=c] (16.408,6.95955) rectangle (16.4478,7.0654);
\draw [color=c, fill=c] (16.4478,6.95955) rectangle (16.4876,7.0654);
\draw [color=c, fill=c] (16.4876,6.95955) rectangle (16.5274,7.0654);
\draw [color=c, fill=c] (16.5274,6.95955) rectangle (16.5672,7.0654);
\draw [color=c, fill=c] (16.5672,6.95955) rectangle (16.607,7.0654);
\draw [color=c, fill=c] (16.607,6.95955) rectangle (16.6468,7.0654);
\draw [color=c, fill=c] (16.6468,6.95955) rectangle (16.6866,7.0654);
\draw [color=c, fill=c] (16.6866,6.95955) rectangle (16.7264,7.0654);
\draw [color=c, fill=c] (16.7264,6.95955) rectangle (16.7662,7.0654);
\draw [color=c, fill=c] (16.7662,6.95955) rectangle (16.806,7.0654);
\draw [color=c, fill=c] (16.806,6.95955) rectangle (16.8458,7.0654);
\draw [color=c, fill=c] (16.8458,6.95955) rectangle (16.8856,7.0654);
\draw [color=c, fill=c] (16.8856,6.95955) rectangle (16.9254,7.0654);
\draw [color=c, fill=c] (16.9254,6.95955) rectangle (16.9652,7.0654);
\draw [color=c, fill=c] (16.9652,6.95955) rectangle (17.005,7.0654);
\draw [color=c, fill=c] (17.005,6.95955) rectangle (17.0448,7.0654);
\draw [color=c, fill=c] (17.0448,6.95955) rectangle (17.0846,7.0654);
\draw [color=c, fill=c] (17.0846,6.95955) rectangle (17.1244,7.0654);
\draw [color=c, fill=c] (17.1244,6.95955) rectangle (17.1642,7.0654);
\draw [color=c, fill=c] (17.1642,6.95955) rectangle (17.204,7.0654);
\draw [color=c, fill=c] (17.204,6.95955) rectangle (17.2438,7.0654);
\draw [color=c, fill=c] (17.2438,6.95955) rectangle (17.2836,7.0654);
\draw [color=c, fill=c] (17.2836,6.95955) rectangle (17.3234,7.0654);
\draw [color=c, fill=c] (17.3234,6.95955) rectangle (17.3632,7.0654);
\draw [color=c, fill=c] (17.3632,6.95955) rectangle (17.403,7.0654);
\draw [color=c, fill=c] (17.403,6.95955) rectangle (17.4428,7.0654);
\draw [color=c, fill=c] (17.4428,6.95955) rectangle (17.4826,7.0654);
\draw [color=c, fill=c] (17.4826,6.95955) rectangle (17.5224,7.0654);
\draw [color=c, fill=c] (17.5224,6.95955) rectangle (17.5622,7.0654);
\draw [color=c, fill=c] (17.5622,6.95955) rectangle (17.602,7.0654);
\draw [color=c, fill=c] (17.602,6.95955) rectangle (17.6418,7.0654);
\draw [color=c, fill=c] (17.6418,6.95955) rectangle (17.6816,7.0654);
\draw [color=c, fill=c] (17.6816,6.95955) rectangle (17.7214,7.0654);
\draw [color=c, fill=c] (17.7214,6.95955) rectangle (17.7612,7.0654);
\draw [color=c, fill=c] (17.7612,6.95955) rectangle (17.801,7.0654);
\draw [color=c, fill=c] (17.801,6.95955) rectangle (17.8408,7.0654);
\draw [color=c, fill=c] (17.8408,6.95955) rectangle (17.8806,7.0654);
\draw [color=c, fill=c] (17.8806,6.95955) rectangle (17.9204,7.0654);
\draw [color=c, fill=c] (17.9204,6.95955) rectangle (17.9602,7.0654);
\draw [color=c, fill=c] (17.9602,6.95955) rectangle (18,7.0654);
\definecolor{c}{rgb}{0.2,0,1};
\draw [color=c, fill=c] (2,7.0654) rectangle (2.0398,7.17125);
\draw [color=c, fill=c] (2.0398,7.0654) rectangle (2.0796,7.17125);
\draw [color=c, fill=c] (2.0796,7.0654) rectangle (2.1194,7.17125);
\draw [color=c, fill=c] (2.1194,7.0654) rectangle (2.1592,7.17125);
\draw [color=c, fill=c] (2.1592,7.0654) rectangle (2.19901,7.17125);
\draw [color=c, fill=c] (2.19901,7.0654) rectangle (2.23881,7.17125);
\draw [color=c, fill=c] (2.23881,7.0654) rectangle (2.27861,7.17125);
\draw [color=c, fill=c] (2.27861,7.0654) rectangle (2.31841,7.17125);
\draw [color=c, fill=c] (2.31841,7.0654) rectangle (2.35821,7.17125);
\draw [color=c, fill=c] (2.35821,7.0654) rectangle (2.39801,7.17125);
\draw [color=c, fill=c] (2.39801,7.0654) rectangle (2.43781,7.17125);
\draw [color=c, fill=c] (2.43781,7.0654) rectangle (2.47761,7.17125);
\draw [color=c, fill=c] (2.47761,7.0654) rectangle (2.51741,7.17125);
\draw [color=c, fill=c] (2.51741,7.0654) rectangle (2.55721,7.17125);
\draw [color=c, fill=c] (2.55721,7.0654) rectangle (2.59702,7.17125);
\draw [color=c, fill=c] (2.59702,7.0654) rectangle (2.63682,7.17125);
\draw [color=c, fill=c] (2.63682,7.0654) rectangle (2.67662,7.17125);
\draw [color=c, fill=c] (2.67662,7.0654) rectangle (2.71642,7.17125);
\draw [color=c, fill=c] (2.71642,7.0654) rectangle (2.75622,7.17125);
\draw [color=c, fill=c] (2.75622,7.0654) rectangle (2.79602,7.17125);
\draw [color=c, fill=c] (2.79602,7.0654) rectangle (2.83582,7.17125);
\draw [color=c, fill=c] (2.83582,7.0654) rectangle (2.87562,7.17125);
\draw [color=c, fill=c] (2.87562,7.0654) rectangle (2.91542,7.17125);
\draw [color=c, fill=c] (2.91542,7.0654) rectangle (2.95522,7.17125);
\draw [color=c, fill=c] (2.95522,7.0654) rectangle (2.99502,7.17125);
\draw [color=c, fill=c] (2.99502,7.0654) rectangle (3.03483,7.17125);
\draw [color=c, fill=c] (3.03483,7.0654) rectangle (3.07463,7.17125);
\draw [color=c, fill=c] (3.07463,7.0654) rectangle (3.11443,7.17125);
\draw [color=c, fill=c] (3.11443,7.0654) rectangle (3.15423,7.17125);
\draw [color=c, fill=c] (3.15423,7.0654) rectangle (3.19403,7.17125);
\draw [color=c, fill=c] (3.19403,7.0654) rectangle (3.23383,7.17125);
\draw [color=c, fill=c] (3.23383,7.0654) rectangle (3.27363,7.17125);
\draw [color=c, fill=c] (3.27363,7.0654) rectangle (3.31343,7.17125);
\draw [color=c, fill=c] (3.31343,7.0654) rectangle (3.35323,7.17125);
\draw [color=c, fill=c] (3.35323,7.0654) rectangle (3.39303,7.17125);
\draw [color=c, fill=c] (3.39303,7.0654) rectangle (3.43284,7.17125);
\draw [color=c, fill=c] (3.43284,7.0654) rectangle (3.47264,7.17125);
\draw [color=c, fill=c] (3.47264,7.0654) rectangle (3.51244,7.17125);
\draw [color=c, fill=c] (3.51244,7.0654) rectangle (3.55224,7.17125);
\draw [color=c, fill=c] (3.55224,7.0654) rectangle (3.59204,7.17125);
\draw [color=c, fill=c] (3.59204,7.0654) rectangle (3.63184,7.17125);
\draw [color=c, fill=c] (3.63184,7.0654) rectangle (3.67164,7.17125);
\draw [color=c, fill=c] (3.67164,7.0654) rectangle (3.71144,7.17125);
\draw [color=c, fill=c] (3.71144,7.0654) rectangle (3.75124,7.17125);
\draw [color=c, fill=c] (3.75124,7.0654) rectangle (3.79104,7.17125);
\draw [color=c, fill=c] (3.79104,7.0654) rectangle (3.83085,7.17125);
\draw [color=c, fill=c] (3.83085,7.0654) rectangle (3.87065,7.17125);
\draw [color=c, fill=c] (3.87065,7.0654) rectangle (3.91045,7.17125);
\draw [color=c, fill=c] (3.91045,7.0654) rectangle (3.95025,7.17125);
\draw [color=c, fill=c] (3.95025,7.0654) rectangle (3.99005,7.17125);
\draw [color=c, fill=c] (3.99005,7.0654) rectangle (4.02985,7.17125);
\draw [color=c, fill=c] (4.02985,7.0654) rectangle (4.06965,7.17125);
\draw [color=c, fill=c] (4.06965,7.0654) rectangle (4.10945,7.17125);
\draw [color=c, fill=c] (4.10945,7.0654) rectangle (4.14925,7.17125);
\draw [color=c, fill=c] (4.14925,7.0654) rectangle (4.18905,7.17125);
\draw [color=c, fill=c] (4.18905,7.0654) rectangle (4.22886,7.17125);
\draw [color=c, fill=c] (4.22886,7.0654) rectangle (4.26866,7.17125);
\draw [color=c, fill=c] (4.26866,7.0654) rectangle (4.30846,7.17125);
\draw [color=c, fill=c] (4.30846,7.0654) rectangle (4.34826,7.17125);
\draw [color=c, fill=c] (4.34826,7.0654) rectangle (4.38806,7.17125);
\draw [color=c, fill=c] (4.38806,7.0654) rectangle (4.42786,7.17125);
\draw [color=c, fill=c] (4.42786,7.0654) rectangle (4.46766,7.17125);
\draw [color=c, fill=c] (4.46766,7.0654) rectangle (4.50746,7.17125);
\draw [color=c, fill=c] (4.50746,7.0654) rectangle (4.54726,7.17125);
\draw [color=c, fill=c] (4.54726,7.0654) rectangle (4.58706,7.17125);
\draw [color=c, fill=c] (4.58706,7.0654) rectangle (4.62687,7.17125);
\draw [color=c, fill=c] (4.62687,7.0654) rectangle (4.66667,7.17125);
\draw [color=c, fill=c] (4.66667,7.0654) rectangle (4.70647,7.17125);
\draw [color=c, fill=c] (4.70647,7.0654) rectangle (4.74627,7.17125);
\draw [color=c, fill=c] (4.74627,7.0654) rectangle (4.78607,7.17125);
\draw [color=c, fill=c] (4.78607,7.0654) rectangle (4.82587,7.17125);
\draw [color=c, fill=c] (4.82587,7.0654) rectangle (4.86567,7.17125);
\draw [color=c, fill=c] (4.86567,7.0654) rectangle (4.90547,7.17125);
\draw [color=c, fill=c] (4.90547,7.0654) rectangle (4.94527,7.17125);
\draw [color=c, fill=c] (4.94527,7.0654) rectangle (4.98507,7.17125);
\draw [color=c, fill=c] (4.98507,7.0654) rectangle (5.02488,7.17125);
\draw [color=c, fill=c] (5.02488,7.0654) rectangle (5.06468,7.17125);
\draw [color=c, fill=c] (5.06468,7.0654) rectangle (5.10448,7.17125);
\draw [color=c, fill=c] (5.10448,7.0654) rectangle (5.14428,7.17125);
\draw [color=c, fill=c] (5.14428,7.0654) rectangle (5.18408,7.17125);
\draw [color=c, fill=c] (5.18408,7.0654) rectangle (5.22388,7.17125);
\draw [color=c, fill=c] (5.22388,7.0654) rectangle (5.26368,7.17125);
\draw [color=c, fill=c] (5.26368,7.0654) rectangle (5.30348,7.17125);
\draw [color=c, fill=c] (5.30348,7.0654) rectangle (5.34328,7.17125);
\draw [color=c, fill=c] (5.34328,7.0654) rectangle (5.38308,7.17125);
\draw [color=c, fill=c] (5.38308,7.0654) rectangle (5.42289,7.17125);
\draw [color=c, fill=c] (5.42289,7.0654) rectangle (5.46269,7.17125);
\draw [color=c, fill=c] (5.46269,7.0654) rectangle (5.50249,7.17125);
\draw [color=c, fill=c] (5.50249,7.0654) rectangle (5.54229,7.17125);
\draw [color=c, fill=c] (5.54229,7.0654) rectangle (5.58209,7.17125);
\draw [color=c, fill=c] (5.58209,7.0654) rectangle (5.62189,7.17125);
\draw [color=c, fill=c] (5.62189,7.0654) rectangle (5.66169,7.17125);
\draw [color=c, fill=c] (5.66169,7.0654) rectangle (5.70149,7.17125);
\draw [color=c, fill=c] (5.70149,7.0654) rectangle (5.74129,7.17125);
\draw [color=c, fill=c] (5.74129,7.0654) rectangle (5.78109,7.17125);
\draw [color=c, fill=c] (5.78109,7.0654) rectangle (5.8209,7.17125);
\draw [color=c, fill=c] (5.8209,7.0654) rectangle (5.8607,7.17125);
\draw [color=c, fill=c] (5.8607,7.0654) rectangle (5.9005,7.17125);
\draw [color=c, fill=c] (5.9005,7.0654) rectangle (5.9403,7.17125);
\draw [color=c, fill=c] (5.9403,7.0654) rectangle (5.9801,7.17125);
\draw [color=c, fill=c] (5.9801,7.0654) rectangle (6.0199,7.17125);
\draw [color=c, fill=c] (6.0199,7.0654) rectangle (6.0597,7.17125);
\draw [color=c, fill=c] (6.0597,7.0654) rectangle (6.0995,7.17125);
\draw [color=c, fill=c] (6.0995,7.0654) rectangle (6.1393,7.17125);
\draw [color=c, fill=c] (6.1393,7.0654) rectangle (6.1791,7.17125);
\draw [color=c, fill=c] (6.1791,7.0654) rectangle (6.21891,7.17125);
\draw [color=c, fill=c] (6.21891,7.0654) rectangle (6.25871,7.17125);
\draw [color=c, fill=c] (6.25871,7.0654) rectangle (6.29851,7.17125);
\draw [color=c, fill=c] (6.29851,7.0654) rectangle (6.33831,7.17125);
\draw [color=c, fill=c] (6.33831,7.0654) rectangle (6.37811,7.17125);
\draw [color=c, fill=c] (6.37811,7.0654) rectangle (6.41791,7.17125);
\draw [color=c, fill=c] (6.41791,7.0654) rectangle (6.45771,7.17125);
\draw [color=c, fill=c] (6.45771,7.0654) rectangle (6.49751,7.17125);
\draw [color=c, fill=c] (6.49751,7.0654) rectangle (6.53731,7.17125);
\draw [color=c, fill=c] (6.53731,7.0654) rectangle (6.57711,7.17125);
\draw [color=c, fill=c] (6.57711,7.0654) rectangle (6.61692,7.17125);
\draw [color=c, fill=c] (6.61692,7.0654) rectangle (6.65672,7.17125);
\draw [color=c, fill=c] (6.65672,7.0654) rectangle (6.69652,7.17125);
\draw [color=c, fill=c] (6.69652,7.0654) rectangle (6.73632,7.17125);
\draw [color=c, fill=c] (6.73632,7.0654) rectangle (6.77612,7.17125);
\draw [color=c, fill=c] (6.77612,7.0654) rectangle (6.81592,7.17125);
\draw [color=c, fill=c] (6.81592,7.0654) rectangle (6.85572,7.17125);
\draw [color=c, fill=c] (6.85572,7.0654) rectangle (6.89552,7.17125);
\draw [color=c, fill=c] (6.89552,7.0654) rectangle (6.93532,7.17125);
\draw [color=c, fill=c] (6.93532,7.0654) rectangle (6.97512,7.17125);
\draw [color=c, fill=c] (6.97512,7.0654) rectangle (7.01493,7.17125);
\draw [color=c, fill=c] (7.01493,7.0654) rectangle (7.05473,7.17125);
\draw [color=c, fill=c] (7.05473,7.0654) rectangle (7.09453,7.17125);
\draw [color=c, fill=c] (7.09453,7.0654) rectangle (7.13433,7.17125);
\draw [color=c, fill=c] (7.13433,7.0654) rectangle (7.17413,7.17125);
\draw [color=c, fill=c] (7.17413,7.0654) rectangle (7.21393,7.17125);
\draw [color=c, fill=c] (7.21393,7.0654) rectangle (7.25373,7.17125);
\draw [color=c, fill=c] (7.25373,7.0654) rectangle (7.29353,7.17125);
\draw [color=c, fill=c] (7.29353,7.0654) rectangle (7.33333,7.17125);
\draw [color=c, fill=c] (7.33333,7.0654) rectangle (7.37313,7.17125);
\draw [color=c, fill=c] (7.37313,7.0654) rectangle (7.41294,7.17125);
\draw [color=c, fill=c] (7.41294,7.0654) rectangle (7.45274,7.17125);
\draw [color=c, fill=c] (7.45274,7.0654) rectangle (7.49254,7.17125);
\draw [color=c, fill=c] (7.49254,7.0654) rectangle (7.53234,7.17125);
\draw [color=c, fill=c] (7.53234,7.0654) rectangle (7.57214,7.17125);
\draw [color=c, fill=c] (7.57214,7.0654) rectangle (7.61194,7.17125);
\draw [color=c, fill=c] (7.61194,7.0654) rectangle (7.65174,7.17125);
\draw [color=c, fill=c] (7.65174,7.0654) rectangle (7.69154,7.17125);
\draw [color=c, fill=c] (7.69154,7.0654) rectangle (7.73134,7.17125);
\draw [color=c, fill=c] (7.73134,7.0654) rectangle (7.77114,7.17125);
\draw [color=c, fill=c] (7.77114,7.0654) rectangle (7.81095,7.17125);
\draw [color=c, fill=c] (7.81095,7.0654) rectangle (7.85075,7.17125);
\draw [color=c, fill=c] (7.85075,7.0654) rectangle (7.89055,7.17125);
\draw [color=c, fill=c] (7.89055,7.0654) rectangle (7.93035,7.17125);
\draw [color=c, fill=c] (7.93035,7.0654) rectangle (7.97015,7.17125);
\draw [color=c, fill=c] (7.97015,7.0654) rectangle (8.00995,7.17125);
\draw [color=c, fill=c] (8.00995,7.0654) rectangle (8.04975,7.17125);
\draw [color=c, fill=c] (8.04975,7.0654) rectangle (8.08955,7.17125);
\draw [color=c, fill=c] (8.08955,7.0654) rectangle (8.12935,7.17125);
\draw [color=c, fill=c] (8.12935,7.0654) rectangle (8.16915,7.17125);
\definecolor{c}{rgb}{0,0.0800001,1};
\draw [color=c, fill=c] (8.16915,7.0654) rectangle (8.20895,7.17125);
\draw [color=c, fill=c] (8.20895,7.0654) rectangle (8.24876,7.17125);
\draw [color=c, fill=c] (8.24876,7.0654) rectangle (8.28856,7.17125);
\draw [color=c, fill=c] (8.28856,7.0654) rectangle (8.32836,7.17125);
\draw [color=c, fill=c] (8.32836,7.0654) rectangle (8.36816,7.17125);
\draw [color=c, fill=c] (8.36816,7.0654) rectangle (8.40796,7.17125);
\draw [color=c, fill=c] (8.40796,7.0654) rectangle (8.44776,7.17125);
\draw [color=c, fill=c] (8.44776,7.0654) rectangle (8.48756,7.17125);
\draw [color=c, fill=c] (8.48756,7.0654) rectangle (8.52736,7.17125);
\draw [color=c, fill=c] (8.52736,7.0654) rectangle (8.56716,7.17125);
\draw [color=c, fill=c] (8.56716,7.0654) rectangle (8.60697,7.17125);
\draw [color=c, fill=c] (8.60697,7.0654) rectangle (8.64677,7.17125);
\draw [color=c, fill=c] (8.64677,7.0654) rectangle (8.68657,7.17125);
\draw [color=c, fill=c] (8.68657,7.0654) rectangle (8.72637,7.17125);
\draw [color=c, fill=c] (8.72637,7.0654) rectangle (8.76617,7.17125);
\draw [color=c, fill=c] (8.76617,7.0654) rectangle (8.80597,7.17125);
\draw [color=c, fill=c] (8.80597,7.0654) rectangle (8.84577,7.17125);
\draw [color=c, fill=c] (8.84577,7.0654) rectangle (8.88557,7.17125);
\draw [color=c, fill=c] (8.88557,7.0654) rectangle (8.92537,7.17125);
\draw [color=c, fill=c] (8.92537,7.0654) rectangle (8.96517,7.17125);
\draw [color=c, fill=c] (8.96517,7.0654) rectangle (9.00498,7.17125);
\draw [color=c, fill=c] (9.00498,7.0654) rectangle (9.04478,7.17125);
\draw [color=c, fill=c] (9.04478,7.0654) rectangle (9.08458,7.17125);
\draw [color=c, fill=c] (9.08458,7.0654) rectangle (9.12438,7.17125);
\draw [color=c, fill=c] (9.12438,7.0654) rectangle (9.16418,7.17125);
\draw [color=c, fill=c] (9.16418,7.0654) rectangle (9.20398,7.17125);
\draw [color=c, fill=c] (9.20398,7.0654) rectangle (9.24378,7.17125);
\draw [color=c, fill=c] (9.24378,7.0654) rectangle (9.28358,7.17125);
\draw [color=c, fill=c] (9.28358,7.0654) rectangle (9.32338,7.17125);
\draw [color=c, fill=c] (9.32338,7.0654) rectangle (9.36318,7.17125);
\draw [color=c, fill=c] (9.36318,7.0654) rectangle (9.40298,7.17125);
\draw [color=c, fill=c] (9.40298,7.0654) rectangle (9.44279,7.17125);
\draw [color=c, fill=c] (9.44279,7.0654) rectangle (9.48259,7.17125);
\draw [color=c, fill=c] (9.48259,7.0654) rectangle (9.52239,7.17125);
\draw [color=c, fill=c] (9.52239,7.0654) rectangle (9.56219,7.17125);
\draw [color=c, fill=c] (9.56219,7.0654) rectangle (9.60199,7.17125);
\definecolor{c}{rgb}{0,0.266667,1};
\draw [color=c, fill=c] (9.60199,7.0654) rectangle (9.64179,7.17125);
\draw [color=c, fill=c] (9.64179,7.0654) rectangle (9.68159,7.17125);
\draw [color=c, fill=c] (9.68159,7.0654) rectangle (9.72139,7.17125);
\draw [color=c, fill=c] (9.72139,7.0654) rectangle (9.76119,7.17125);
\draw [color=c, fill=c] (9.76119,7.0654) rectangle (9.80099,7.17125);
\draw [color=c, fill=c] (9.80099,7.0654) rectangle (9.8408,7.17125);
\draw [color=c, fill=c] (9.8408,7.0654) rectangle (9.8806,7.17125);
\draw [color=c, fill=c] (9.8806,7.0654) rectangle (9.9204,7.17125);
\draw [color=c, fill=c] (9.9204,7.0654) rectangle (9.9602,7.17125);
\draw [color=c, fill=c] (9.9602,7.0654) rectangle (10,7.17125);
\draw [color=c, fill=c] (10,7.0654) rectangle (10.0398,7.17125);
\draw [color=c, fill=c] (10.0398,7.0654) rectangle (10.0796,7.17125);
\draw [color=c, fill=c] (10.0796,7.0654) rectangle (10.1194,7.17125);
\draw [color=c, fill=c] (10.1194,7.0654) rectangle (10.1592,7.17125);
\draw [color=c, fill=c] (10.1592,7.0654) rectangle (10.199,7.17125);
\draw [color=c, fill=c] (10.199,7.0654) rectangle (10.2388,7.17125);
\draw [color=c, fill=c] (10.2388,7.0654) rectangle (10.2786,7.17125);
\definecolor{c}{rgb}{0,0.546666,1};
\draw [color=c, fill=c] (10.2786,7.0654) rectangle (10.3184,7.17125);
\draw [color=c, fill=c] (10.3184,7.0654) rectangle (10.3582,7.17125);
\draw [color=c, fill=c] (10.3582,7.0654) rectangle (10.398,7.17125);
\draw [color=c, fill=c] (10.398,7.0654) rectangle (10.4378,7.17125);
\draw [color=c, fill=c] (10.4378,7.0654) rectangle (10.4776,7.17125);
\draw [color=c, fill=c] (10.4776,7.0654) rectangle (10.5174,7.17125);
\draw [color=c, fill=c] (10.5174,7.0654) rectangle (10.5572,7.17125);
\draw [color=c, fill=c] (10.5572,7.0654) rectangle (10.597,7.17125);
\draw [color=c, fill=c] (10.597,7.0654) rectangle (10.6368,7.17125);
\draw [color=c, fill=c] (10.6368,7.0654) rectangle (10.6766,7.17125);
\draw [color=c, fill=c] (10.6766,7.0654) rectangle (10.7164,7.17125);
\draw [color=c, fill=c] (10.7164,7.0654) rectangle (10.7562,7.17125);
\draw [color=c, fill=c] (10.7562,7.0654) rectangle (10.796,7.17125);
\draw [color=c, fill=c] (10.796,7.0654) rectangle (10.8358,7.17125);
\draw [color=c, fill=c] (10.8358,7.0654) rectangle (10.8756,7.17125);
\draw [color=c, fill=c] (10.8756,7.0654) rectangle (10.9154,7.17125);
\draw [color=c, fill=c] (10.9154,7.0654) rectangle (10.9552,7.17125);
\draw [color=c, fill=c] (10.9552,7.0654) rectangle (10.995,7.17125);
\draw [color=c, fill=c] (10.995,7.0654) rectangle (11.0348,7.17125);
\draw [color=c, fill=c] (11.0348,7.0654) rectangle (11.0746,7.17125);
\draw [color=c, fill=c] (11.0746,7.0654) rectangle (11.1144,7.17125);
\draw [color=c, fill=c] (11.1144,7.0654) rectangle (11.1542,7.17125);
\draw [color=c, fill=c] (11.1542,7.0654) rectangle (11.194,7.17125);
\draw [color=c, fill=c] (11.194,7.0654) rectangle (11.2338,7.17125);
\draw [color=c, fill=c] (11.2338,7.0654) rectangle (11.2736,7.17125);
\draw [color=c, fill=c] (11.2736,7.0654) rectangle (11.3134,7.17125);
\draw [color=c, fill=c] (11.3134,7.0654) rectangle (11.3532,7.17125);
\draw [color=c, fill=c] (11.3532,7.0654) rectangle (11.393,7.17125);
\draw [color=c, fill=c] (11.393,7.0654) rectangle (11.4328,7.17125);
\definecolor{c}{rgb}{0,0.733333,1};
\draw [color=c, fill=c] (11.4328,7.0654) rectangle (11.4726,7.17125);
\draw [color=c, fill=c] (11.4726,7.0654) rectangle (11.5124,7.17125);
\draw [color=c, fill=c] (11.5124,7.0654) rectangle (11.5522,7.17125);
\draw [color=c, fill=c] (11.5522,7.0654) rectangle (11.592,7.17125);
\draw [color=c, fill=c] (11.592,7.0654) rectangle (11.6318,7.17125);
\draw [color=c, fill=c] (11.6318,7.0654) rectangle (11.6716,7.17125);
\draw [color=c, fill=c] (11.6716,7.0654) rectangle (11.7114,7.17125);
\draw [color=c, fill=c] (11.7114,7.0654) rectangle (11.7512,7.17125);
\draw [color=c, fill=c] (11.7512,7.0654) rectangle (11.791,7.17125);
\draw [color=c, fill=c] (11.791,7.0654) rectangle (11.8308,7.17125);
\draw [color=c, fill=c] (11.8308,7.0654) rectangle (11.8706,7.17125);
\draw [color=c, fill=c] (11.8706,7.0654) rectangle (11.9104,7.17125);
\draw [color=c, fill=c] (11.9104,7.0654) rectangle (11.9502,7.17125);
\draw [color=c, fill=c] (11.9502,7.0654) rectangle (11.99,7.17125);
\draw [color=c, fill=c] (11.99,7.0654) rectangle (12.0299,7.17125);
\draw [color=c, fill=c] (12.0299,7.0654) rectangle (12.0697,7.17125);
\draw [color=c, fill=c] (12.0697,7.0654) rectangle (12.1095,7.17125);
\draw [color=c, fill=c] (12.1095,7.0654) rectangle (12.1493,7.17125);
\draw [color=c, fill=c] (12.1493,7.0654) rectangle (12.1891,7.17125);
\draw [color=c, fill=c] (12.1891,7.0654) rectangle (12.2289,7.17125);
\draw [color=c, fill=c] (12.2289,7.0654) rectangle (12.2687,7.17125);
\draw [color=c, fill=c] (12.2687,7.0654) rectangle (12.3085,7.17125);
\draw [color=c, fill=c] (12.3085,7.0654) rectangle (12.3483,7.17125);
\draw [color=c, fill=c] (12.3483,7.0654) rectangle (12.3881,7.17125);
\draw [color=c, fill=c] (12.3881,7.0654) rectangle (12.4279,7.17125);
\draw [color=c, fill=c] (12.4279,7.0654) rectangle (12.4677,7.17125);
\draw [color=c, fill=c] (12.4677,7.0654) rectangle (12.5075,7.17125);
\draw [color=c, fill=c] (12.5075,7.0654) rectangle (12.5473,7.17125);
\draw [color=c, fill=c] (12.5473,7.0654) rectangle (12.5871,7.17125);
\draw [color=c, fill=c] (12.5871,7.0654) rectangle (12.6269,7.17125);
\draw [color=c, fill=c] (12.6269,7.0654) rectangle (12.6667,7.17125);
\draw [color=c, fill=c] (12.6667,7.0654) rectangle (12.7065,7.17125);
\draw [color=c, fill=c] (12.7065,7.0654) rectangle (12.7463,7.17125);
\draw [color=c, fill=c] (12.7463,7.0654) rectangle (12.7861,7.17125);
\draw [color=c, fill=c] (12.7861,7.0654) rectangle (12.8259,7.17125);
\draw [color=c, fill=c] (12.8259,7.0654) rectangle (12.8657,7.17125);
\draw [color=c, fill=c] (12.8657,7.0654) rectangle (12.9055,7.17125);
\draw [color=c, fill=c] (12.9055,7.0654) rectangle (12.9453,7.17125);
\draw [color=c, fill=c] (12.9453,7.0654) rectangle (12.9851,7.17125);
\draw [color=c, fill=c] (12.9851,7.0654) rectangle (13.0249,7.17125);
\draw [color=c, fill=c] (13.0249,7.0654) rectangle (13.0647,7.17125);
\draw [color=c, fill=c] (13.0647,7.0654) rectangle (13.1045,7.17125);
\draw [color=c, fill=c] (13.1045,7.0654) rectangle (13.1443,7.17125);
\draw [color=c, fill=c] (13.1443,7.0654) rectangle (13.1841,7.17125);
\draw [color=c, fill=c] (13.1841,7.0654) rectangle (13.2239,7.17125);
\draw [color=c, fill=c] (13.2239,7.0654) rectangle (13.2637,7.17125);
\draw [color=c, fill=c] (13.2637,7.0654) rectangle (13.3035,7.17125);
\draw [color=c, fill=c] (13.3035,7.0654) rectangle (13.3433,7.17125);
\draw [color=c, fill=c] (13.3433,7.0654) rectangle (13.3831,7.17125);
\draw [color=c, fill=c] (13.3831,7.0654) rectangle (13.4229,7.17125);
\draw [color=c, fill=c] (13.4229,7.0654) rectangle (13.4627,7.17125);
\draw [color=c, fill=c] (13.4627,7.0654) rectangle (13.5025,7.17125);
\draw [color=c, fill=c] (13.5025,7.0654) rectangle (13.5423,7.17125);
\draw [color=c, fill=c] (13.5423,7.0654) rectangle (13.5821,7.17125);
\draw [color=c, fill=c] (13.5821,7.0654) rectangle (13.6219,7.17125);
\draw [color=c, fill=c] (13.6219,7.0654) rectangle (13.6617,7.17125);
\draw [color=c, fill=c] (13.6617,7.0654) rectangle (13.7015,7.17125);
\draw [color=c, fill=c] (13.7015,7.0654) rectangle (13.7413,7.17125);
\draw [color=c, fill=c] (13.7413,7.0654) rectangle (13.7811,7.17125);
\draw [color=c, fill=c] (13.7811,7.0654) rectangle (13.8209,7.17125);
\draw [color=c, fill=c] (13.8209,7.0654) rectangle (13.8607,7.17125);
\draw [color=c, fill=c] (13.8607,7.0654) rectangle (13.9005,7.17125);
\draw [color=c, fill=c] (13.9005,7.0654) rectangle (13.9403,7.17125);
\draw [color=c, fill=c] (13.9403,7.0654) rectangle (13.9801,7.17125);
\draw [color=c, fill=c] (13.9801,7.0654) rectangle (14.0199,7.17125);
\draw [color=c, fill=c] (14.0199,7.0654) rectangle (14.0597,7.17125);
\draw [color=c, fill=c] (14.0597,7.0654) rectangle (14.0995,7.17125);
\draw [color=c, fill=c] (14.0995,7.0654) rectangle (14.1393,7.17125);
\draw [color=c, fill=c] (14.1393,7.0654) rectangle (14.1791,7.17125);
\draw [color=c, fill=c] (14.1791,7.0654) rectangle (14.2189,7.17125);
\draw [color=c, fill=c] (14.2189,7.0654) rectangle (14.2587,7.17125);
\draw [color=c, fill=c] (14.2587,7.0654) rectangle (14.2985,7.17125);
\draw [color=c, fill=c] (14.2985,7.0654) rectangle (14.3383,7.17125);
\draw [color=c, fill=c] (14.3383,7.0654) rectangle (14.3781,7.17125);
\draw [color=c, fill=c] (14.3781,7.0654) rectangle (14.4179,7.17125);
\draw [color=c, fill=c] (14.4179,7.0654) rectangle (14.4577,7.17125);
\draw [color=c, fill=c] (14.4577,7.0654) rectangle (14.4975,7.17125);
\draw [color=c, fill=c] (14.4975,7.0654) rectangle (14.5373,7.17125);
\draw [color=c, fill=c] (14.5373,7.0654) rectangle (14.5771,7.17125);
\draw [color=c, fill=c] (14.5771,7.0654) rectangle (14.6169,7.17125);
\draw [color=c, fill=c] (14.6169,7.0654) rectangle (14.6567,7.17125);
\draw [color=c, fill=c] (14.6567,7.0654) rectangle (14.6965,7.17125);
\draw [color=c, fill=c] (14.6965,7.0654) rectangle (14.7363,7.17125);
\draw [color=c, fill=c] (14.7363,7.0654) rectangle (14.7761,7.17125);
\draw [color=c, fill=c] (14.7761,7.0654) rectangle (14.8159,7.17125);
\draw [color=c, fill=c] (14.8159,7.0654) rectangle (14.8557,7.17125);
\draw [color=c, fill=c] (14.8557,7.0654) rectangle (14.8955,7.17125);
\draw [color=c, fill=c] (14.8955,7.0654) rectangle (14.9353,7.17125);
\draw [color=c, fill=c] (14.9353,7.0654) rectangle (14.9751,7.17125);
\draw [color=c, fill=c] (14.9751,7.0654) rectangle (15.0149,7.17125);
\draw [color=c, fill=c] (15.0149,7.0654) rectangle (15.0547,7.17125);
\draw [color=c, fill=c] (15.0547,7.0654) rectangle (15.0945,7.17125);
\draw [color=c, fill=c] (15.0945,7.0654) rectangle (15.1343,7.17125);
\draw [color=c, fill=c] (15.1343,7.0654) rectangle (15.1741,7.17125);
\draw [color=c, fill=c] (15.1741,7.0654) rectangle (15.2139,7.17125);
\draw [color=c, fill=c] (15.2139,7.0654) rectangle (15.2537,7.17125);
\draw [color=c, fill=c] (15.2537,7.0654) rectangle (15.2935,7.17125);
\draw [color=c, fill=c] (15.2935,7.0654) rectangle (15.3333,7.17125);
\draw [color=c, fill=c] (15.3333,7.0654) rectangle (15.3731,7.17125);
\draw [color=c, fill=c] (15.3731,7.0654) rectangle (15.4129,7.17125);
\draw [color=c, fill=c] (15.4129,7.0654) rectangle (15.4527,7.17125);
\draw [color=c, fill=c] (15.4527,7.0654) rectangle (15.4925,7.17125);
\draw [color=c, fill=c] (15.4925,7.0654) rectangle (15.5323,7.17125);
\draw [color=c, fill=c] (15.5323,7.0654) rectangle (15.5721,7.17125);
\draw [color=c, fill=c] (15.5721,7.0654) rectangle (15.6119,7.17125);
\draw [color=c, fill=c] (15.6119,7.0654) rectangle (15.6517,7.17125);
\draw [color=c, fill=c] (15.6517,7.0654) rectangle (15.6915,7.17125);
\draw [color=c, fill=c] (15.6915,7.0654) rectangle (15.7313,7.17125);
\draw [color=c, fill=c] (15.7313,7.0654) rectangle (15.7711,7.17125);
\draw [color=c, fill=c] (15.7711,7.0654) rectangle (15.8109,7.17125);
\draw [color=c, fill=c] (15.8109,7.0654) rectangle (15.8507,7.17125);
\draw [color=c, fill=c] (15.8507,7.0654) rectangle (15.8905,7.17125);
\draw [color=c, fill=c] (15.8905,7.0654) rectangle (15.9303,7.17125);
\draw [color=c, fill=c] (15.9303,7.0654) rectangle (15.9701,7.17125);
\draw [color=c, fill=c] (15.9701,7.0654) rectangle (16.01,7.17125);
\draw [color=c, fill=c] (16.01,7.0654) rectangle (16.0498,7.17125);
\draw [color=c, fill=c] (16.0498,7.0654) rectangle (16.0896,7.17125);
\draw [color=c, fill=c] (16.0896,7.0654) rectangle (16.1294,7.17125);
\draw [color=c, fill=c] (16.1294,7.0654) rectangle (16.1692,7.17125);
\draw [color=c, fill=c] (16.1692,7.0654) rectangle (16.209,7.17125);
\draw [color=c, fill=c] (16.209,7.0654) rectangle (16.2488,7.17125);
\draw [color=c, fill=c] (16.2488,7.0654) rectangle (16.2886,7.17125);
\draw [color=c, fill=c] (16.2886,7.0654) rectangle (16.3284,7.17125);
\draw [color=c, fill=c] (16.3284,7.0654) rectangle (16.3682,7.17125);
\draw [color=c, fill=c] (16.3682,7.0654) rectangle (16.408,7.17125);
\draw [color=c, fill=c] (16.408,7.0654) rectangle (16.4478,7.17125);
\draw [color=c, fill=c] (16.4478,7.0654) rectangle (16.4876,7.17125);
\draw [color=c, fill=c] (16.4876,7.0654) rectangle (16.5274,7.17125);
\draw [color=c, fill=c] (16.5274,7.0654) rectangle (16.5672,7.17125);
\draw [color=c, fill=c] (16.5672,7.0654) rectangle (16.607,7.17125);
\draw [color=c, fill=c] (16.607,7.0654) rectangle (16.6468,7.17125);
\draw [color=c, fill=c] (16.6468,7.0654) rectangle (16.6866,7.17125);
\draw [color=c, fill=c] (16.6866,7.0654) rectangle (16.7264,7.17125);
\draw [color=c, fill=c] (16.7264,7.0654) rectangle (16.7662,7.17125);
\draw [color=c, fill=c] (16.7662,7.0654) rectangle (16.806,7.17125);
\draw [color=c, fill=c] (16.806,7.0654) rectangle (16.8458,7.17125);
\draw [color=c, fill=c] (16.8458,7.0654) rectangle (16.8856,7.17125);
\draw [color=c, fill=c] (16.8856,7.0654) rectangle (16.9254,7.17125);
\draw [color=c, fill=c] (16.9254,7.0654) rectangle (16.9652,7.17125);
\draw [color=c, fill=c] (16.9652,7.0654) rectangle (17.005,7.17125);
\draw [color=c, fill=c] (17.005,7.0654) rectangle (17.0448,7.17125);
\draw [color=c, fill=c] (17.0448,7.0654) rectangle (17.0846,7.17125);
\draw [color=c, fill=c] (17.0846,7.0654) rectangle (17.1244,7.17125);
\draw [color=c, fill=c] (17.1244,7.0654) rectangle (17.1642,7.17125);
\draw [color=c, fill=c] (17.1642,7.0654) rectangle (17.204,7.17125);
\draw [color=c, fill=c] (17.204,7.0654) rectangle (17.2438,7.17125);
\draw [color=c, fill=c] (17.2438,7.0654) rectangle (17.2836,7.17125);
\draw [color=c, fill=c] (17.2836,7.0654) rectangle (17.3234,7.17125);
\draw [color=c, fill=c] (17.3234,7.0654) rectangle (17.3632,7.17125);
\draw [color=c, fill=c] (17.3632,7.0654) rectangle (17.403,7.17125);
\draw [color=c, fill=c] (17.403,7.0654) rectangle (17.4428,7.17125);
\draw [color=c, fill=c] (17.4428,7.0654) rectangle (17.4826,7.17125);
\draw [color=c, fill=c] (17.4826,7.0654) rectangle (17.5224,7.17125);
\draw [color=c, fill=c] (17.5224,7.0654) rectangle (17.5622,7.17125);
\draw [color=c, fill=c] (17.5622,7.0654) rectangle (17.602,7.17125);
\draw [color=c, fill=c] (17.602,7.0654) rectangle (17.6418,7.17125);
\draw [color=c, fill=c] (17.6418,7.0654) rectangle (17.6816,7.17125);
\draw [color=c, fill=c] (17.6816,7.0654) rectangle (17.7214,7.17125);
\draw [color=c, fill=c] (17.7214,7.0654) rectangle (17.7612,7.17125);
\draw [color=c, fill=c] (17.7612,7.0654) rectangle (17.801,7.17125);
\draw [color=c, fill=c] (17.801,7.0654) rectangle (17.8408,7.17125);
\draw [color=c, fill=c] (17.8408,7.0654) rectangle (17.8806,7.17125);
\draw [color=c, fill=c] (17.8806,7.0654) rectangle (17.9204,7.17125);
\draw [color=c, fill=c] (17.9204,7.0654) rectangle (17.9602,7.17125);
\draw [color=c, fill=c] (17.9602,7.0654) rectangle (18,7.17125);
\definecolor{c}{rgb}{0.2,0,1};
\draw [color=c, fill=c] (2,7.17125) rectangle (2.0398,7.27709);
\draw [color=c, fill=c] (2.0398,7.17125) rectangle (2.0796,7.27709);
\draw [color=c, fill=c] (2.0796,7.17125) rectangle (2.1194,7.27709);
\draw [color=c, fill=c] (2.1194,7.17125) rectangle (2.1592,7.27709);
\draw [color=c, fill=c] (2.1592,7.17125) rectangle (2.19901,7.27709);
\draw [color=c, fill=c] (2.19901,7.17125) rectangle (2.23881,7.27709);
\draw [color=c, fill=c] (2.23881,7.17125) rectangle (2.27861,7.27709);
\draw [color=c, fill=c] (2.27861,7.17125) rectangle (2.31841,7.27709);
\draw [color=c, fill=c] (2.31841,7.17125) rectangle (2.35821,7.27709);
\draw [color=c, fill=c] (2.35821,7.17125) rectangle (2.39801,7.27709);
\draw [color=c, fill=c] (2.39801,7.17125) rectangle (2.43781,7.27709);
\draw [color=c, fill=c] (2.43781,7.17125) rectangle (2.47761,7.27709);
\draw [color=c, fill=c] (2.47761,7.17125) rectangle (2.51741,7.27709);
\draw [color=c, fill=c] (2.51741,7.17125) rectangle (2.55721,7.27709);
\draw [color=c, fill=c] (2.55721,7.17125) rectangle (2.59702,7.27709);
\draw [color=c, fill=c] (2.59702,7.17125) rectangle (2.63682,7.27709);
\draw [color=c, fill=c] (2.63682,7.17125) rectangle (2.67662,7.27709);
\draw [color=c, fill=c] (2.67662,7.17125) rectangle (2.71642,7.27709);
\draw [color=c, fill=c] (2.71642,7.17125) rectangle (2.75622,7.27709);
\draw [color=c, fill=c] (2.75622,7.17125) rectangle (2.79602,7.27709);
\draw [color=c, fill=c] (2.79602,7.17125) rectangle (2.83582,7.27709);
\draw [color=c, fill=c] (2.83582,7.17125) rectangle (2.87562,7.27709);
\draw [color=c, fill=c] (2.87562,7.17125) rectangle (2.91542,7.27709);
\draw [color=c, fill=c] (2.91542,7.17125) rectangle (2.95522,7.27709);
\draw [color=c, fill=c] (2.95522,7.17125) rectangle (2.99502,7.27709);
\draw [color=c, fill=c] (2.99502,7.17125) rectangle (3.03483,7.27709);
\draw [color=c, fill=c] (3.03483,7.17125) rectangle (3.07463,7.27709);
\draw [color=c, fill=c] (3.07463,7.17125) rectangle (3.11443,7.27709);
\draw [color=c, fill=c] (3.11443,7.17125) rectangle (3.15423,7.27709);
\draw [color=c, fill=c] (3.15423,7.17125) rectangle (3.19403,7.27709);
\draw [color=c, fill=c] (3.19403,7.17125) rectangle (3.23383,7.27709);
\draw [color=c, fill=c] (3.23383,7.17125) rectangle (3.27363,7.27709);
\draw [color=c, fill=c] (3.27363,7.17125) rectangle (3.31343,7.27709);
\draw [color=c, fill=c] (3.31343,7.17125) rectangle (3.35323,7.27709);
\draw [color=c, fill=c] (3.35323,7.17125) rectangle (3.39303,7.27709);
\draw [color=c, fill=c] (3.39303,7.17125) rectangle (3.43284,7.27709);
\draw [color=c, fill=c] (3.43284,7.17125) rectangle (3.47264,7.27709);
\draw [color=c, fill=c] (3.47264,7.17125) rectangle (3.51244,7.27709);
\draw [color=c, fill=c] (3.51244,7.17125) rectangle (3.55224,7.27709);
\draw [color=c, fill=c] (3.55224,7.17125) rectangle (3.59204,7.27709);
\draw [color=c, fill=c] (3.59204,7.17125) rectangle (3.63184,7.27709);
\draw [color=c, fill=c] (3.63184,7.17125) rectangle (3.67164,7.27709);
\draw [color=c, fill=c] (3.67164,7.17125) rectangle (3.71144,7.27709);
\draw [color=c, fill=c] (3.71144,7.17125) rectangle (3.75124,7.27709);
\draw [color=c, fill=c] (3.75124,7.17125) rectangle (3.79104,7.27709);
\draw [color=c, fill=c] (3.79104,7.17125) rectangle (3.83085,7.27709);
\draw [color=c, fill=c] (3.83085,7.17125) rectangle (3.87065,7.27709);
\draw [color=c, fill=c] (3.87065,7.17125) rectangle (3.91045,7.27709);
\draw [color=c, fill=c] (3.91045,7.17125) rectangle (3.95025,7.27709);
\draw [color=c, fill=c] (3.95025,7.17125) rectangle (3.99005,7.27709);
\draw [color=c, fill=c] (3.99005,7.17125) rectangle (4.02985,7.27709);
\draw [color=c, fill=c] (4.02985,7.17125) rectangle (4.06965,7.27709);
\draw [color=c, fill=c] (4.06965,7.17125) rectangle (4.10945,7.27709);
\draw [color=c, fill=c] (4.10945,7.17125) rectangle (4.14925,7.27709);
\draw [color=c, fill=c] (4.14925,7.17125) rectangle (4.18905,7.27709);
\draw [color=c, fill=c] (4.18905,7.17125) rectangle (4.22886,7.27709);
\draw [color=c, fill=c] (4.22886,7.17125) rectangle (4.26866,7.27709);
\draw [color=c, fill=c] (4.26866,7.17125) rectangle (4.30846,7.27709);
\draw [color=c, fill=c] (4.30846,7.17125) rectangle (4.34826,7.27709);
\draw [color=c, fill=c] (4.34826,7.17125) rectangle (4.38806,7.27709);
\draw [color=c, fill=c] (4.38806,7.17125) rectangle (4.42786,7.27709);
\draw [color=c, fill=c] (4.42786,7.17125) rectangle (4.46766,7.27709);
\draw [color=c, fill=c] (4.46766,7.17125) rectangle (4.50746,7.27709);
\draw [color=c, fill=c] (4.50746,7.17125) rectangle (4.54726,7.27709);
\draw [color=c, fill=c] (4.54726,7.17125) rectangle (4.58706,7.27709);
\draw [color=c, fill=c] (4.58706,7.17125) rectangle (4.62687,7.27709);
\draw [color=c, fill=c] (4.62687,7.17125) rectangle (4.66667,7.27709);
\draw [color=c, fill=c] (4.66667,7.17125) rectangle (4.70647,7.27709);
\draw [color=c, fill=c] (4.70647,7.17125) rectangle (4.74627,7.27709);
\draw [color=c, fill=c] (4.74627,7.17125) rectangle (4.78607,7.27709);
\draw [color=c, fill=c] (4.78607,7.17125) rectangle (4.82587,7.27709);
\draw [color=c, fill=c] (4.82587,7.17125) rectangle (4.86567,7.27709);
\draw [color=c, fill=c] (4.86567,7.17125) rectangle (4.90547,7.27709);
\draw [color=c, fill=c] (4.90547,7.17125) rectangle (4.94527,7.27709);
\draw [color=c, fill=c] (4.94527,7.17125) rectangle (4.98507,7.27709);
\draw [color=c, fill=c] (4.98507,7.17125) rectangle (5.02488,7.27709);
\draw [color=c, fill=c] (5.02488,7.17125) rectangle (5.06468,7.27709);
\draw [color=c, fill=c] (5.06468,7.17125) rectangle (5.10448,7.27709);
\draw [color=c, fill=c] (5.10448,7.17125) rectangle (5.14428,7.27709);
\draw [color=c, fill=c] (5.14428,7.17125) rectangle (5.18408,7.27709);
\draw [color=c, fill=c] (5.18408,7.17125) rectangle (5.22388,7.27709);
\draw [color=c, fill=c] (5.22388,7.17125) rectangle (5.26368,7.27709);
\draw [color=c, fill=c] (5.26368,7.17125) rectangle (5.30348,7.27709);
\draw [color=c, fill=c] (5.30348,7.17125) rectangle (5.34328,7.27709);
\draw [color=c, fill=c] (5.34328,7.17125) rectangle (5.38308,7.27709);
\draw [color=c, fill=c] (5.38308,7.17125) rectangle (5.42289,7.27709);
\draw [color=c, fill=c] (5.42289,7.17125) rectangle (5.46269,7.27709);
\draw [color=c, fill=c] (5.46269,7.17125) rectangle (5.50249,7.27709);
\draw [color=c, fill=c] (5.50249,7.17125) rectangle (5.54229,7.27709);
\draw [color=c, fill=c] (5.54229,7.17125) rectangle (5.58209,7.27709);
\draw [color=c, fill=c] (5.58209,7.17125) rectangle (5.62189,7.27709);
\draw [color=c, fill=c] (5.62189,7.17125) rectangle (5.66169,7.27709);
\draw [color=c, fill=c] (5.66169,7.17125) rectangle (5.70149,7.27709);
\draw [color=c, fill=c] (5.70149,7.17125) rectangle (5.74129,7.27709);
\draw [color=c, fill=c] (5.74129,7.17125) rectangle (5.78109,7.27709);
\draw [color=c, fill=c] (5.78109,7.17125) rectangle (5.8209,7.27709);
\draw [color=c, fill=c] (5.8209,7.17125) rectangle (5.8607,7.27709);
\draw [color=c, fill=c] (5.8607,7.17125) rectangle (5.9005,7.27709);
\draw [color=c, fill=c] (5.9005,7.17125) rectangle (5.9403,7.27709);
\draw [color=c, fill=c] (5.9403,7.17125) rectangle (5.9801,7.27709);
\draw [color=c, fill=c] (5.9801,7.17125) rectangle (6.0199,7.27709);
\draw [color=c, fill=c] (6.0199,7.17125) rectangle (6.0597,7.27709);
\draw [color=c, fill=c] (6.0597,7.17125) rectangle (6.0995,7.27709);
\draw [color=c, fill=c] (6.0995,7.17125) rectangle (6.1393,7.27709);
\draw [color=c, fill=c] (6.1393,7.17125) rectangle (6.1791,7.27709);
\draw [color=c, fill=c] (6.1791,7.17125) rectangle (6.21891,7.27709);
\draw [color=c, fill=c] (6.21891,7.17125) rectangle (6.25871,7.27709);
\draw [color=c, fill=c] (6.25871,7.17125) rectangle (6.29851,7.27709);
\draw [color=c, fill=c] (6.29851,7.17125) rectangle (6.33831,7.27709);
\draw [color=c, fill=c] (6.33831,7.17125) rectangle (6.37811,7.27709);
\draw [color=c, fill=c] (6.37811,7.17125) rectangle (6.41791,7.27709);
\draw [color=c, fill=c] (6.41791,7.17125) rectangle (6.45771,7.27709);
\draw [color=c, fill=c] (6.45771,7.17125) rectangle (6.49751,7.27709);
\draw [color=c, fill=c] (6.49751,7.17125) rectangle (6.53731,7.27709);
\draw [color=c, fill=c] (6.53731,7.17125) rectangle (6.57711,7.27709);
\draw [color=c, fill=c] (6.57711,7.17125) rectangle (6.61692,7.27709);
\draw [color=c, fill=c] (6.61692,7.17125) rectangle (6.65672,7.27709);
\draw [color=c, fill=c] (6.65672,7.17125) rectangle (6.69652,7.27709);
\draw [color=c, fill=c] (6.69652,7.17125) rectangle (6.73632,7.27709);
\draw [color=c, fill=c] (6.73632,7.17125) rectangle (6.77612,7.27709);
\draw [color=c, fill=c] (6.77612,7.17125) rectangle (6.81592,7.27709);
\draw [color=c, fill=c] (6.81592,7.17125) rectangle (6.85572,7.27709);
\draw [color=c, fill=c] (6.85572,7.17125) rectangle (6.89552,7.27709);
\draw [color=c, fill=c] (6.89552,7.17125) rectangle (6.93532,7.27709);
\draw [color=c, fill=c] (6.93532,7.17125) rectangle (6.97512,7.27709);
\draw [color=c, fill=c] (6.97512,7.17125) rectangle (7.01493,7.27709);
\draw [color=c, fill=c] (7.01493,7.17125) rectangle (7.05473,7.27709);
\draw [color=c, fill=c] (7.05473,7.17125) rectangle (7.09453,7.27709);
\draw [color=c, fill=c] (7.09453,7.17125) rectangle (7.13433,7.27709);
\draw [color=c, fill=c] (7.13433,7.17125) rectangle (7.17413,7.27709);
\draw [color=c, fill=c] (7.17413,7.17125) rectangle (7.21393,7.27709);
\draw [color=c, fill=c] (7.21393,7.17125) rectangle (7.25373,7.27709);
\draw [color=c, fill=c] (7.25373,7.17125) rectangle (7.29353,7.27709);
\draw [color=c, fill=c] (7.29353,7.17125) rectangle (7.33333,7.27709);
\draw [color=c, fill=c] (7.33333,7.17125) rectangle (7.37313,7.27709);
\draw [color=c, fill=c] (7.37313,7.17125) rectangle (7.41294,7.27709);
\draw [color=c, fill=c] (7.41294,7.17125) rectangle (7.45274,7.27709);
\draw [color=c, fill=c] (7.45274,7.17125) rectangle (7.49254,7.27709);
\draw [color=c, fill=c] (7.49254,7.17125) rectangle (7.53234,7.27709);
\draw [color=c, fill=c] (7.53234,7.17125) rectangle (7.57214,7.27709);
\draw [color=c, fill=c] (7.57214,7.17125) rectangle (7.61194,7.27709);
\draw [color=c, fill=c] (7.61194,7.17125) rectangle (7.65174,7.27709);
\draw [color=c, fill=c] (7.65174,7.17125) rectangle (7.69154,7.27709);
\draw [color=c, fill=c] (7.69154,7.17125) rectangle (7.73134,7.27709);
\draw [color=c, fill=c] (7.73134,7.17125) rectangle (7.77114,7.27709);
\draw [color=c, fill=c] (7.77114,7.17125) rectangle (7.81095,7.27709);
\draw [color=c, fill=c] (7.81095,7.17125) rectangle (7.85075,7.27709);
\draw [color=c, fill=c] (7.85075,7.17125) rectangle (7.89055,7.27709);
\draw [color=c, fill=c] (7.89055,7.17125) rectangle (7.93035,7.27709);
\draw [color=c, fill=c] (7.93035,7.17125) rectangle (7.97015,7.27709);
\draw [color=c, fill=c] (7.97015,7.17125) rectangle (8.00995,7.27709);
\draw [color=c, fill=c] (8.00995,7.17125) rectangle (8.04975,7.27709);
\draw [color=c, fill=c] (8.04975,7.17125) rectangle (8.08955,7.27709);
\definecolor{c}{rgb}{0,0.0800001,1};
\draw [color=c, fill=c] (8.08955,7.17125) rectangle (8.12935,7.27709);
\draw [color=c, fill=c] (8.12935,7.17125) rectangle (8.16915,7.27709);
\draw [color=c, fill=c] (8.16915,7.17125) rectangle (8.20895,7.27709);
\draw [color=c, fill=c] (8.20895,7.17125) rectangle (8.24876,7.27709);
\draw [color=c, fill=c] (8.24876,7.17125) rectangle (8.28856,7.27709);
\draw [color=c, fill=c] (8.28856,7.17125) rectangle (8.32836,7.27709);
\draw [color=c, fill=c] (8.32836,7.17125) rectangle (8.36816,7.27709);
\draw [color=c, fill=c] (8.36816,7.17125) rectangle (8.40796,7.27709);
\draw [color=c, fill=c] (8.40796,7.17125) rectangle (8.44776,7.27709);
\draw [color=c, fill=c] (8.44776,7.17125) rectangle (8.48756,7.27709);
\draw [color=c, fill=c] (8.48756,7.17125) rectangle (8.52736,7.27709);
\draw [color=c, fill=c] (8.52736,7.17125) rectangle (8.56716,7.27709);
\draw [color=c, fill=c] (8.56716,7.17125) rectangle (8.60697,7.27709);
\draw [color=c, fill=c] (8.60697,7.17125) rectangle (8.64677,7.27709);
\draw [color=c, fill=c] (8.64677,7.17125) rectangle (8.68657,7.27709);
\draw [color=c, fill=c] (8.68657,7.17125) rectangle (8.72637,7.27709);
\draw [color=c, fill=c] (8.72637,7.17125) rectangle (8.76617,7.27709);
\draw [color=c, fill=c] (8.76617,7.17125) rectangle (8.80597,7.27709);
\draw [color=c, fill=c] (8.80597,7.17125) rectangle (8.84577,7.27709);
\draw [color=c, fill=c] (8.84577,7.17125) rectangle (8.88557,7.27709);
\draw [color=c, fill=c] (8.88557,7.17125) rectangle (8.92537,7.27709);
\draw [color=c, fill=c] (8.92537,7.17125) rectangle (8.96517,7.27709);
\draw [color=c, fill=c] (8.96517,7.17125) rectangle (9.00498,7.27709);
\draw [color=c, fill=c] (9.00498,7.17125) rectangle (9.04478,7.27709);
\draw [color=c, fill=c] (9.04478,7.17125) rectangle (9.08458,7.27709);
\draw [color=c, fill=c] (9.08458,7.17125) rectangle (9.12438,7.27709);
\draw [color=c, fill=c] (9.12438,7.17125) rectangle (9.16418,7.27709);
\draw [color=c, fill=c] (9.16418,7.17125) rectangle (9.20398,7.27709);
\draw [color=c, fill=c] (9.20398,7.17125) rectangle (9.24378,7.27709);
\draw [color=c, fill=c] (9.24378,7.17125) rectangle (9.28358,7.27709);
\draw [color=c, fill=c] (9.28358,7.17125) rectangle (9.32338,7.27709);
\draw [color=c, fill=c] (9.32338,7.17125) rectangle (9.36318,7.27709);
\draw [color=c, fill=c] (9.36318,7.17125) rectangle (9.40298,7.27709);
\draw [color=c, fill=c] (9.40298,7.17125) rectangle (9.44279,7.27709);
\draw [color=c, fill=c] (9.44279,7.17125) rectangle (9.48259,7.27709);
\draw [color=c, fill=c] (9.48259,7.17125) rectangle (9.52239,7.27709);
\draw [color=c, fill=c] (9.52239,7.17125) rectangle (9.56219,7.27709);
\definecolor{c}{rgb}{0,0.266667,1};
\draw [color=c, fill=c] (9.56219,7.17125) rectangle (9.60199,7.27709);
\draw [color=c, fill=c] (9.60199,7.17125) rectangle (9.64179,7.27709);
\draw [color=c, fill=c] (9.64179,7.17125) rectangle (9.68159,7.27709);
\draw [color=c, fill=c] (9.68159,7.17125) rectangle (9.72139,7.27709);
\draw [color=c, fill=c] (9.72139,7.17125) rectangle (9.76119,7.27709);
\draw [color=c, fill=c] (9.76119,7.17125) rectangle (9.80099,7.27709);
\draw [color=c, fill=c] (9.80099,7.17125) rectangle (9.8408,7.27709);
\draw [color=c, fill=c] (9.8408,7.17125) rectangle (9.8806,7.27709);
\draw [color=c, fill=c] (9.8806,7.17125) rectangle (9.9204,7.27709);
\draw [color=c, fill=c] (9.9204,7.17125) rectangle (9.9602,7.27709);
\draw [color=c, fill=c] (9.9602,7.17125) rectangle (10,7.27709);
\draw [color=c, fill=c] (10,7.17125) rectangle (10.0398,7.27709);
\draw [color=c, fill=c] (10.0398,7.17125) rectangle (10.0796,7.27709);
\draw [color=c, fill=c] (10.0796,7.17125) rectangle (10.1194,7.27709);
\draw [color=c, fill=c] (10.1194,7.17125) rectangle (10.1592,7.27709);
\draw [color=c, fill=c] (10.1592,7.17125) rectangle (10.199,7.27709);
\draw [color=c, fill=c] (10.199,7.17125) rectangle (10.2388,7.27709);
\draw [color=c, fill=c] (10.2388,7.17125) rectangle (10.2786,7.27709);
\definecolor{c}{rgb}{0,0.546666,1};
\draw [color=c, fill=c] (10.2786,7.17125) rectangle (10.3184,7.27709);
\draw [color=c, fill=c] (10.3184,7.17125) rectangle (10.3582,7.27709);
\draw [color=c, fill=c] (10.3582,7.17125) rectangle (10.398,7.27709);
\draw [color=c, fill=c] (10.398,7.17125) rectangle (10.4378,7.27709);
\draw [color=c, fill=c] (10.4378,7.17125) rectangle (10.4776,7.27709);
\draw [color=c, fill=c] (10.4776,7.17125) rectangle (10.5174,7.27709);
\draw [color=c, fill=c] (10.5174,7.17125) rectangle (10.5572,7.27709);
\draw [color=c, fill=c] (10.5572,7.17125) rectangle (10.597,7.27709);
\draw [color=c, fill=c] (10.597,7.17125) rectangle (10.6368,7.27709);
\draw [color=c, fill=c] (10.6368,7.17125) rectangle (10.6766,7.27709);
\draw [color=c, fill=c] (10.6766,7.17125) rectangle (10.7164,7.27709);
\draw [color=c, fill=c] (10.7164,7.17125) rectangle (10.7562,7.27709);
\draw [color=c, fill=c] (10.7562,7.17125) rectangle (10.796,7.27709);
\draw [color=c, fill=c] (10.796,7.17125) rectangle (10.8358,7.27709);
\draw [color=c, fill=c] (10.8358,7.17125) rectangle (10.8756,7.27709);
\draw [color=c, fill=c] (10.8756,7.17125) rectangle (10.9154,7.27709);
\draw [color=c, fill=c] (10.9154,7.17125) rectangle (10.9552,7.27709);
\draw [color=c, fill=c] (10.9552,7.17125) rectangle (10.995,7.27709);
\draw [color=c, fill=c] (10.995,7.17125) rectangle (11.0348,7.27709);
\draw [color=c, fill=c] (11.0348,7.17125) rectangle (11.0746,7.27709);
\draw [color=c, fill=c] (11.0746,7.17125) rectangle (11.1144,7.27709);
\draw [color=c, fill=c] (11.1144,7.17125) rectangle (11.1542,7.27709);
\draw [color=c, fill=c] (11.1542,7.17125) rectangle (11.194,7.27709);
\draw [color=c, fill=c] (11.194,7.17125) rectangle (11.2338,7.27709);
\draw [color=c, fill=c] (11.2338,7.17125) rectangle (11.2736,7.27709);
\draw [color=c, fill=c] (11.2736,7.17125) rectangle (11.3134,7.27709);
\draw [color=c, fill=c] (11.3134,7.17125) rectangle (11.3532,7.27709);
\draw [color=c, fill=c] (11.3532,7.17125) rectangle (11.393,7.27709);
\draw [color=c, fill=c] (11.393,7.17125) rectangle (11.4328,7.27709);
\draw [color=c, fill=c] (11.4328,7.17125) rectangle (11.4726,7.27709);
\draw [color=c, fill=c] (11.4726,7.17125) rectangle (11.5124,7.27709);
\definecolor{c}{rgb}{0,0.733333,1};
\draw [color=c, fill=c] (11.5124,7.17125) rectangle (11.5522,7.27709);
\draw [color=c, fill=c] (11.5522,7.17125) rectangle (11.592,7.27709);
\draw [color=c, fill=c] (11.592,7.17125) rectangle (11.6318,7.27709);
\draw [color=c, fill=c] (11.6318,7.17125) rectangle (11.6716,7.27709);
\draw [color=c, fill=c] (11.6716,7.17125) rectangle (11.7114,7.27709);
\draw [color=c, fill=c] (11.7114,7.17125) rectangle (11.7512,7.27709);
\draw [color=c, fill=c] (11.7512,7.17125) rectangle (11.791,7.27709);
\draw [color=c, fill=c] (11.791,7.17125) rectangle (11.8308,7.27709);
\draw [color=c, fill=c] (11.8308,7.17125) rectangle (11.8706,7.27709);
\draw [color=c, fill=c] (11.8706,7.17125) rectangle (11.9104,7.27709);
\draw [color=c, fill=c] (11.9104,7.17125) rectangle (11.9502,7.27709);
\draw [color=c, fill=c] (11.9502,7.17125) rectangle (11.99,7.27709);
\draw [color=c, fill=c] (11.99,7.17125) rectangle (12.0299,7.27709);
\draw [color=c, fill=c] (12.0299,7.17125) rectangle (12.0697,7.27709);
\draw [color=c, fill=c] (12.0697,7.17125) rectangle (12.1095,7.27709);
\draw [color=c, fill=c] (12.1095,7.17125) rectangle (12.1493,7.27709);
\draw [color=c, fill=c] (12.1493,7.17125) rectangle (12.1891,7.27709);
\draw [color=c, fill=c] (12.1891,7.17125) rectangle (12.2289,7.27709);
\draw [color=c, fill=c] (12.2289,7.17125) rectangle (12.2687,7.27709);
\draw [color=c, fill=c] (12.2687,7.17125) rectangle (12.3085,7.27709);
\draw [color=c, fill=c] (12.3085,7.17125) rectangle (12.3483,7.27709);
\draw [color=c, fill=c] (12.3483,7.17125) rectangle (12.3881,7.27709);
\draw [color=c, fill=c] (12.3881,7.17125) rectangle (12.4279,7.27709);
\draw [color=c, fill=c] (12.4279,7.17125) rectangle (12.4677,7.27709);
\draw [color=c, fill=c] (12.4677,7.17125) rectangle (12.5075,7.27709);
\draw [color=c, fill=c] (12.5075,7.17125) rectangle (12.5473,7.27709);
\draw [color=c, fill=c] (12.5473,7.17125) rectangle (12.5871,7.27709);
\draw [color=c, fill=c] (12.5871,7.17125) rectangle (12.6269,7.27709);
\draw [color=c, fill=c] (12.6269,7.17125) rectangle (12.6667,7.27709);
\draw [color=c, fill=c] (12.6667,7.17125) rectangle (12.7065,7.27709);
\draw [color=c, fill=c] (12.7065,7.17125) rectangle (12.7463,7.27709);
\draw [color=c, fill=c] (12.7463,7.17125) rectangle (12.7861,7.27709);
\draw [color=c, fill=c] (12.7861,7.17125) rectangle (12.8259,7.27709);
\draw [color=c, fill=c] (12.8259,7.17125) rectangle (12.8657,7.27709);
\draw [color=c, fill=c] (12.8657,7.17125) rectangle (12.9055,7.27709);
\draw [color=c, fill=c] (12.9055,7.17125) rectangle (12.9453,7.27709);
\draw [color=c, fill=c] (12.9453,7.17125) rectangle (12.9851,7.27709);
\draw [color=c, fill=c] (12.9851,7.17125) rectangle (13.0249,7.27709);
\draw [color=c, fill=c] (13.0249,7.17125) rectangle (13.0647,7.27709);
\draw [color=c, fill=c] (13.0647,7.17125) rectangle (13.1045,7.27709);
\draw [color=c, fill=c] (13.1045,7.17125) rectangle (13.1443,7.27709);
\draw [color=c, fill=c] (13.1443,7.17125) rectangle (13.1841,7.27709);
\draw [color=c, fill=c] (13.1841,7.17125) rectangle (13.2239,7.27709);
\draw [color=c, fill=c] (13.2239,7.17125) rectangle (13.2637,7.27709);
\draw [color=c, fill=c] (13.2637,7.17125) rectangle (13.3035,7.27709);
\draw [color=c, fill=c] (13.3035,7.17125) rectangle (13.3433,7.27709);
\draw [color=c, fill=c] (13.3433,7.17125) rectangle (13.3831,7.27709);
\draw [color=c, fill=c] (13.3831,7.17125) rectangle (13.4229,7.27709);
\draw [color=c, fill=c] (13.4229,7.17125) rectangle (13.4627,7.27709);
\draw [color=c, fill=c] (13.4627,7.17125) rectangle (13.5025,7.27709);
\draw [color=c, fill=c] (13.5025,7.17125) rectangle (13.5423,7.27709);
\draw [color=c, fill=c] (13.5423,7.17125) rectangle (13.5821,7.27709);
\draw [color=c, fill=c] (13.5821,7.17125) rectangle (13.6219,7.27709);
\draw [color=c, fill=c] (13.6219,7.17125) rectangle (13.6617,7.27709);
\draw [color=c, fill=c] (13.6617,7.17125) rectangle (13.7015,7.27709);
\draw [color=c, fill=c] (13.7015,7.17125) rectangle (13.7413,7.27709);
\draw [color=c, fill=c] (13.7413,7.17125) rectangle (13.7811,7.27709);
\draw [color=c, fill=c] (13.7811,7.17125) rectangle (13.8209,7.27709);
\draw [color=c, fill=c] (13.8209,7.17125) rectangle (13.8607,7.27709);
\draw [color=c, fill=c] (13.8607,7.17125) rectangle (13.9005,7.27709);
\draw [color=c, fill=c] (13.9005,7.17125) rectangle (13.9403,7.27709);
\draw [color=c, fill=c] (13.9403,7.17125) rectangle (13.9801,7.27709);
\draw [color=c, fill=c] (13.9801,7.17125) rectangle (14.0199,7.27709);
\draw [color=c, fill=c] (14.0199,7.17125) rectangle (14.0597,7.27709);
\draw [color=c, fill=c] (14.0597,7.17125) rectangle (14.0995,7.27709);
\draw [color=c, fill=c] (14.0995,7.17125) rectangle (14.1393,7.27709);
\draw [color=c, fill=c] (14.1393,7.17125) rectangle (14.1791,7.27709);
\draw [color=c, fill=c] (14.1791,7.17125) rectangle (14.2189,7.27709);
\draw [color=c, fill=c] (14.2189,7.17125) rectangle (14.2587,7.27709);
\draw [color=c, fill=c] (14.2587,7.17125) rectangle (14.2985,7.27709);
\draw [color=c, fill=c] (14.2985,7.17125) rectangle (14.3383,7.27709);
\draw [color=c, fill=c] (14.3383,7.17125) rectangle (14.3781,7.27709);
\draw [color=c, fill=c] (14.3781,7.17125) rectangle (14.4179,7.27709);
\draw [color=c, fill=c] (14.4179,7.17125) rectangle (14.4577,7.27709);
\draw [color=c, fill=c] (14.4577,7.17125) rectangle (14.4975,7.27709);
\draw [color=c, fill=c] (14.4975,7.17125) rectangle (14.5373,7.27709);
\draw [color=c, fill=c] (14.5373,7.17125) rectangle (14.5771,7.27709);
\draw [color=c, fill=c] (14.5771,7.17125) rectangle (14.6169,7.27709);
\draw [color=c, fill=c] (14.6169,7.17125) rectangle (14.6567,7.27709);
\draw [color=c, fill=c] (14.6567,7.17125) rectangle (14.6965,7.27709);
\draw [color=c, fill=c] (14.6965,7.17125) rectangle (14.7363,7.27709);
\draw [color=c, fill=c] (14.7363,7.17125) rectangle (14.7761,7.27709);
\draw [color=c, fill=c] (14.7761,7.17125) rectangle (14.8159,7.27709);
\draw [color=c, fill=c] (14.8159,7.17125) rectangle (14.8557,7.27709);
\draw [color=c, fill=c] (14.8557,7.17125) rectangle (14.8955,7.27709);
\draw [color=c, fill=c] (14.8955,7.17125) rectangle (14.9353,7.27709);
\draw [color=c, fill=c] (14.9353,7.17125) rectangle (14.9751,7.27709);
\draw [color=c, fill=c] (14.9751,7.17125) rectangle (15.0149,7.27709);
\draw [color=c, fill=c] (15.0149,7.17125) rectangle (15.0547,7.27709);
\draw [color=c, fill=c] (15.0547,7.17125) rectangle (15.0945,7.27709);
\draw [color=c, fill=c] (15.0945,7.17125) rectangle (15.1343,7.27709);
\draw [color=c, fill=c] (15.1343,7.17125) rectangle (15.1741,7.27709);
\draw [color=c, fill=c] (15.1741,7.17125) rectangle (15.2139,7.27709);
\draw [color=c, fill=c] (15.2139,7.17125) rectangle (15.2537,7.27709);
\draw [color=c, fill=c] (15.2537,7.17125) rectangle (15.2935,7.27709);
\draw [color=c, fill=c] (15.2935,7.17125) rectangle (15.3333,7.27709);
\draw [color=c, fill=c] (15.3333,7.17125) rectangle (15.3731,7.27709);
\draw [color=c, fill=c] (15.3731,7.17125) rectangle (15.4129,7.27709);
\draw [color=c, fill=c] (15.4129,7.17125) rectangle (15.4527,7.27709);
\draw [color=c, fill=c] (15.4527,7.17125) rectangle (15.4925,7.27709);
\draw [color=c, fill=c] (15.4925,7.17125) rectangle (15.5323,7.27709);
\draw [color=c, fill=c] (15.5323,7.17125) rectangle (15.5721,7.27709);
\draw [color=c, fill=c] (15.5721,7.17125) rectangle (15.6119,7.27709);
\draw [color=c, fill=c] (15.6119,7.17125) rectangle (15.6517,7.27709);
\draw [color=c, fill=c] (15.6517,7.17125) rectangle (15.6915,7.27709);
\draw [color=c, fill=c] (15.6915,7.17125) rectangle (15.7313,7.27709);
\draw [color=c, fill=c] (15.7313,7.17125) rectangle (15.7711,7.27709);
\draw [color=c, fill=c] (15.7711,7.17125) rectangle (15.8109,7.27709);
\draw [color=c, fill=c] (15.8109,7.17125) rectangle (15.8507,7.27709);
\draw [color=c, fill=c] (15.8507,7.17125) rectangle (15.8905,7.27709);
\draw [color=c, fill=c] (15.8905,7.17125) rectangle (15.9303,7.27709);
\draw [color=c, fill=c] (15.9303,7.17125) rectangle (15.9701,7.27709);
\draw [color=c, fill=c] (15.9701,7.17125) rectangle (16.01,7.27709);
\draw [color=c, fill=c] (16.01,7.17125) rectangle (16.0498,7.27709);
\draw [color=c, fill=c] (16.0498,7.17125) rectangle (16.0896,7.27709);
\draw [color=c, fill=c] (16.0896,7.17125) rectangle (16.1294,7.27709);
\draw [color=c, fill=c] (16.1294,7.17125) rectangle (16.1692,7.27709);
\draw [color=c, fill=c] (16.1692,7.17125) rectangle (16.209,7.27709);
\draw [color=c, fill=c] (16.209,7.17125) rectangle (16.2488,7.27709);
\draw [color=c, fill=c] (16.2488,7.17125) rectangle (16.2886,7.27709);
\draw [color=c, fill=c] (16.2886,7.17125) rectangle (16.3284,7.27709);
\draw [color=c, fill=c] (16.3284,7.17125) rectangle (16.3682,7.27709);
\draw [color=c, fill=c] (16.3682,7.17125) rectangle (16.408,7.27709);
\draw [color=c, fill=c] (16.408,7.17125) rectangle (16.4478,7.27709);
\draw [color=c, fill=c] (16.4478,7.17125) rectangle (16.4876,7.27709);
\draw [color=c, fill=c] (16.4876,7.17125) rectangle (16.5274,7.27709);
\draw [color=c, fill=c] (16.5274,7.17125) rectangle (16.5672,7.27709);
\draw [color=c, fill=c] (16.5672,7.17125) rectangle (16.607,7.27709);
\draw [color=c, fill=c] (16.607,7.17125) rectangle (16.6468,7.27709);
\draw [color=c, fill=c] (16.6468,7.17125) rectangle (16.6866,7.27709);
\draw [color=c, fill=c] (16.6866,7.17125) rectangle (16.7264,7.27709);
\draw [color=c, fill=c] (16.7264,7.17125) rectangle (16.7662,7.27709);
\draw [color=c, fill=c] (16.7662,7.17125) rectangle (16.806,7.27709);
\draw [color=c, fill=c] (16.806,7.17125) rectangle (16.8458,7.27709);
\draw [color=c, fill=c] (16.8458,7.17125) rectangle (16.8856,7.27709);
\draw [color=c, fill=c] (16.8856,7.17125) rectangle (16.9254,7.27709);
\draw [color=c, fill=c] (16.9254,7.17125) rectangle (16.9652,7.27709);
\draw [color=c, fill=c] (16.9652,7.17125) rectangle (17.005,7.27709);
\draw [color=c, fill=c] (17.005,7.17125) rectangle (17.0448,7.27709);
\draw [color=c, fill=c] (17.0448,7.17125) rectangle (17.0846,7.27709);
\draw [color=c, fill=c] (17.0846,7.17125) rectangle (17.1244,7.27709);
\draw [color=c, fill=c] (17.1244,7.17125) rectangle (17.1642,7.27709);
\draw [color=c, fill=c] (17.1642,7.17125) rectangle (17.204,7.27709);
\draw [color=c, fill=c] (17.204,7.17125) rectangle (17.2438,7.27709);
\draw [color=c, fill=c] (17.2438,7.17125) rectangle (17.2836,7.27709);
\draw [color=c, fill=c] (17.2836,7.17125) rectangle (17.3234,7.27709);
\draw [color=c, fill=c] (17.3234,7.17125) rectangle (17.3632,7.27709);
\draw [color=c, fill=c] (17.3632,7.17125) rectangle (17.403,7.27709);
\draw [color=c, fill=c] (17.403,7.17125) rectangle (17.4428,7.27709);
\draw [color=c, fill=c] (17.4428,7.17125) rectangle (17.4826,7.27709);
\draw [color=c, fill=c] (17.4826,7.17125) rectangle (17.5224,7.27709);
\draw [color=c, fill=c] (17.5224,7.17125) rectangle (17.5622,7.27709);
\draw [color=c, fill=c] (17.5622,7.17125) rectangle (17.602,7.27709);
\draw [color=c, fill=c] (17.602,7.17125) rectangle (17.6418,7.27709);
\draw [color=c, fill=c] (17.6418,7.17125) rectangle (17.6816,7.27709);
\draw [color=c, fill=c] (17.6816,7.17125) rectangle (17.7214,7.27709);
\draw [color=c, fill=c] (17.7214,7.17125) rectangle (17.7612,7.27709);
\draw [color=c, fill=c] (17.7612,7.17125) rectangle (17.801,7.27709);
\draw [color=c, fill=c] (17.801,7.17125) rectangle (17.8408,7.27709);
\draw [color=c, fill=c] (17.8408,7.17125) rectangle (17.8806,7.27709);
\draw [color=c, fill=c] (17.8806,7.17125) rectangle (17.9204,7.27709);
\draw [color=c, fill=c] (17.9204,7.17125) rectangle (17.9602,7.27709);
\draw [color=c, fill=c] (17.9602,7.17125) rectangle (18,7.27709);
\definecolor{c}{rgb}{0.2,0,1};
\draw [color=c, fill=c] (2,7.27709) rectangle (2.0398,7.38294);
\draw [color=c, fill=c] (2.0398,7.27709) rectangle (2.0796,7.38294);
\draw [color=c, fill=c] (2.0796,7.27709) rectangle (2.1194,7.38294);
\draw [color=c, fill=c] (2.1194,7.27709) rectangle (2.1592,7.38294);
\draw [color=c, fill=c] (2.1592,7.27709) rectangle (2.19901,7.38294);
\draw [color=c, fill=c] (2.19901,7.27709) rectangle (2.23881,7.38294);
\draw [color=c, fill=c] (2.23881,7.27709) rectangle (2.27861,7.38294);
\draw [color=c, fill=c] (2.27861,7.27709) rectangle (2.31841,7.38294);
\draw [color=c, fill=c] (2.31841,7.27709) rectangle (2.35821,7.38294);
\draw [color=c, fill=c] (2.35821,7.27709) rectangle (2.39801,7.38294);
\draw [color=c, fill=c] (2.39801,7.27709) rectangle (2.43781,7.38294);
\draw [color=c, fill=c] (2.43781,7.27709) rectangle (2.47761,7.38294);
\draw [color=c, fill=c] (2.47761,7.27709) rectangle (2.51741,7.38294);
\draw [color=c, fill=c] (2.51741,7.27709) rectangle (2.55721,7.38294);
\draw [color=c, fill=c] (2.55721,7.27709) rectangle (2.59702,7.38294);
\draw [color=c, fill=c] (2.59702,7.27709) rectangle (2.63682,7.38294);
\draw [color=c, fill=c] (2.63682,7.27709) rectangle (2.67662,7.38294);
\draw [color=c, fill=c] (2.67662,7.27709) rectangle (2.71642,7.38294);
\draw [color=c, fill=c] (2.71642,7.27709) rectangle (2.75622,7.38294);
\draw [color=c, fill=c] (2.75622,7.27709) rectangle (2.79602,7.38294);
\draw [color=c, fill=c] (2.79602,7.27709) rectangle (2.83582,7.38294);
\draw [color=c, fill=c] (2.83582,7.27709) rectangle (2.87562,7.38294);
\draw [color=c, fill=c] (2.87562,7.27709) rectangle (2.91542,7.38294);
\draw [color=c, fill=c] (2.91542,7.27709) rectangle (2.95522,7.38294);
\draw [color=c, fill=c] (2.95522,7.27709) rectangle (2.99502,7.38294);
\draw [color=c, fill=c] (2.99502,7.27709) rectangle (3.03483,7.38294);
\draw [color=c, fill=c] (3.03483,7.27709) rectangle (3.07463,7.38294);
\draw [color=c, fill=c] (3.07463,7.27709) rectangle (3.11443,7.38294);
\draw [color=c, fill=c] (3.11443,7.27709) rectangle (3.15423,7.38294);
\draw [color=c, fill=c] (3.15423,7.27709) rectangle (3.19403,7.38294);
\draw [color=c, fill=c] (3.19403,7.27709) rectangle (3.23383,7.38294);
\draw [color=c, fill=c] (3.23383,7.27709) rectangle (3.27363,7.38294);
\draw [color=c, fill=c] (3.27363,7.27709) rectangle (3.31343,7.38294);
\draw [color=c, fill=c] (3.31343,7.27709) rectangle (3.35323,7.38294);
\draw [color=c, fill=c] (3.35323,7.27709) rectangle (3.39303,7.38294);
\draw [color=c, fill=c] (3.39303,7.27709) rectangle (3.43284,7.38294);
\draw [color=c, fill=c] (3.43284,7.27709) rectangle (3.47264,7.38294);
\draw [color=c, fill=c] (3.47264,7.27709) rectangle (3.51244,7.38294);
\draw [color=c, fill=c] (3.51244,7.27709) rectangle (3.55224,7.38294);
\draw [color=c, fill=c] (3.55224,7.27709) rectangle (3.59204,7.38294);
\draw [color=c, fill=c] (3.59204,7.27709) rectangle (3.63184,7.38294);
\draw [color=c, fill=c] (3.63184,7.27709) rectangle (3.67164,7.38294);
\draw [color=c, fill=c] (3.67164,7.27709) rectangle (3.71144,7.38294);
\draw [color=c, fill=c] (3.71144,7.27709) rectangle (3.75124,7.38294);
\draw [color=c, fill=c] (3.75124,7.27709) rectangle (3.79104,7.38294);
\draw [color=c, fill=c] (3.79104,7.27709) rectangle (3.83085,7.38294);
\draw [color=c, fill=c] (3.83085,7.27709) rectangle (3.87065,7.38294);
\draw [color=c, fill=c] (3.87065,7.27709) rectangle (3.91045,7.38294);
\draw [color=c, fill=c] (3.91045,7.27709) rectangle (3.95025,7.38294);
\draw [color=c, fill=c] (3.95025,7.27709) rectangle (3.99005,7.38294);
\draw [color=c, fill=c] (3.99005,7.27709) rectangle (4.02985,7.38294);
\draw [color=c, fill=c] (4.02985,7.27709) rectangle (4.06965,7.38294);
\draw [color=c, fill=c] (4.06965,7.27709) rectangle (4.10945,7.38294);
\draw [color=c, fill=c] (4.10945,7.27709) rectangle (4.14925,7.38294);
\draw [color=c, fill=c] (4.14925,7.27709) rectangle (4.18905,7.38294);
\draw [color=c, fill=c] (4.18905,7.27709) rectangle (4.22886,7.38294);
\draw [color=c, fill=c] (4.22886,7.27709) rectangle (4.26866,7.38294);
\draw [color=c, fill=c] (4.26866,7.27709) rectangle (4.30846,7.38294);
\draw [color=c, fill=c] (4.30846,7.27709) rectangle (4.34826,7.38294);
\draw [color=c, fill=c] (4.34826,7.27709) rectangle (4.38806,7.38294);
\draw [color=c, fill=c] (4.38806,7.27709) rectangle (4.42786,7.38294);
\draw [color=c, fill=c] (4.42786,7.27709) rectangle (4.46766,7.38294);
\draw [color=c, fill=c] (4.46766,7.27709) rectangle (4.50746,7.38294);
\draw [color=c, fill=c] (4.50746,7.27709) rectangle (4.54726,7.38294);
\draw [color=c, fill=c] (4.54726,7.27709) rectangle (4.58706,7.38294);
\draw [color=c, fill=c] (4.58706,7.27709) rectangle (4.62687,7.38294);
\draw [color=c, fill=c] (4.62687,7.27709) rectangle (4.66667,7.38294);
\draw [color=c, fill=c] (4.66667,7.27709) rectangle (4.70647,7.38294);
\draw [color=c, fill=c] (4.70647,7.27709) rectangle (4.74627,7.38294);
\draw [color=c, fill=c] (4.74627,7.27709) rectangle (4.78607,7.38294);
\draw [color=c, fill=c] (4.78607,7.27709) rectangle (4.82587,7.38294);
\draw [color=c, fill=c] (4.82587,7.27709) rectangle (4.86567,7.38294);
\draw [color=c, fill=c] (4.86567,7.27709) rectangle (4.90547,7.38294);
\draw [color=c, fill=c] (4.90547,7.27709) rectangle (4.94527,7.38294);
\draw [color=c, fill=c] (4.94527,7.27709) rectangle (4.98507,7.38294);
\draw [color=c, fill=c] (4.98507,7.27709) rectangle (5.02488,7.38294);
\draw [color=c, fill=c] (5.02488,7.27709) rectangle (5.06468,7.38294);
\draw [color=c, fill=c] (5.06468,7.27709) rectangle (5.10448,7.38294);
\draw [color=c, fill=c] (5.10448,7.27709) rectangle (5.14428,7.38294);
\draw [color=c, fill=c] (5.14428,7.27709) rectangle (5.18408,7.38294);
\draw [color=c, fill=c] (5.18408,7.27709) rectangle (5.22388,7.38294);
\draw [color=c, fill=c] (5.22388,7.27709) rectangle (5.26368,7.38294);
\draw [color=c, fill=c] (5.26368,7.27709) rectangle (5.30348,7.38294);
\draw [color=c, fill=c] (5.30348,7.27709) rectangle (5.34328,7.38294);
\draw [color=c, fill=c] (5.34328,7.27709) rectangle (5.38308,7.38294);
\draw [color=c, fill=c] (5.38308,7.27709) rectangle (5.42289,7.38294);
\draw [color=c, fill=c] (5.42289,7.27709) rectangle (5.46269,7.38294);
\draw [color=c, fill=c] (5.46269,7.27709) rectangle (5.50249,7.38294);
\draw [color=c, fill=c] (5.50249,7.27709) rectangle (5.54229,7.38294);
\draw [color=c, fill=c] (5.54229,7.27709) rectangle (5.58209,7.38294);
\draw [color=c, fill=c] (5.58209,7.27709) rectangle (5.62189,7.38294);
\draw [color=c, fill=c] (5.62189,7.27709) rectangle (5.66169,7.38294);
\draw [color=c, fill=c] (5.66169,7.27709) rectangle (5.70149,7.38294);
\draw [color=c, fill=c] (5.70149,7.27709) rectangle (5.74129,7.38294);
\draw [color=c, fill=c] (5.74129,7.27709) rectangle (5.78109,7.38294);
\draw [color=c, fill=c] (5.78109,7.27709) rectangle (5.8209,7.38294);
\draw [color=c, fill=c] (5.8209,7.27709) rectangle (5.8607,7.38294);
\draw [color=c, fill=c] (5.8607,7.27709) rectangle (5.9005,7.38294);
\draw [color=c, fill=c] (5.9005,7.27709) rectangle (5.9403,7.38294);
\draw [color=c, fill=c] (5.9403,7.27709) rectangle (5.9801,7.38294);
\draw [color=c, fill=c] (5.9801,7.27709) rectangle (6.0199,7.38294);
\draw [color=c, fill=c] (6.0199,7.27709) rectangle (6.0597,7.38294);
\draw [color=c, fill=c] (6.0597,7.27709) rectangle (6.0995,7.38294);
\draw [color=c, fill=c] (6.0995,7.27709) rectangle (6.1393,7.38294);
\draw [color=c, fill=c] (6.1393,7.27709) rectangle (6.1791,7.38294);
\draw [color=c, fill=c] (6.1791,7.27709) rectangle (6.21891,7.38294);
\draw [color=c, fill=c] (6.21891,7.27709) rectangle (6.25871,7.38294);
\draw [color=c, fill=c] (6.25871,7.27709) rectangle (6.29851,7.38294);
\draw [color=c, fill=c] (6.29851,7.27709) rectangle (6.33831,7.38294);
\draw [color=c, fill=c] (6.33831,7.27709) rectangle (6.37811,7.38294);
\draw [color=c, fill=c] (6.37811,7.27709) rectangle (6.41791,7.38294);
\draw [color=c, fill=c] (6.41791,7.27709) rectangle (6.45771,7.38294);
\draw [color=c, fill=c] (6.45771,7.27709) rectangle (6.49751,7.38294);
\draw [color=c, fill=c] (6.49751,7.27709) rectangle (6.53731,7.38294);
\draw [color=c, fill=c] (6.53731,7.27709) rectangle (6.57711,7.38294);
\draw [color=c, fill=c] (6.57711,7.27709) rectangle (6.61692,7.38294);
\draw [color=c, fill=c] (6.61692,7.27709) rectangle (6.65672,7.38294);
\draw [color=c, fill=c] (6.65672,7.27709) rectangle (6.69652,7.38294);
\draw [color=c, fill=c] (6.69652,7.27709) rectangle (6.73632,7.38294);
\draw [color=c, fill=c] (6.73632,7.27709) rectangle (6.77612,7.38294);
\draw [color=c, fill=c] (6.77612,7.27709) rectangle (6.81592,7.38294);
\draw [color=c, fill=c] (6.81592,7.27709) rectangle (6.85572,7.38294);
\draw [color=c, fill=c] (6.85572,7.27709) rectangle (6.89552,7.38294);
\draw [color=c, fill=c] (6.89552,7.27709) rectangle (6.93532,7.38294);
\draw [color=c, fill=c] (6.93532,7.27709) rectangle (6.97512,7.38294);
\draw [color=c, fill=c] (6.97512,7.27709) rectangle (7.01493,7.38294);
\draw [color=c, fill=c] (7.01493,7.27709) rectangle (7.05473,7.38294);
\draw [color=c, fill=c] (7.05473,7.27709) rectangle (7.09453,7.38294);
\draw [color=c, fill=c] (7.09453,7.27709) rectangle (7.13433,7.38294);
\draw [color=c, fill=c] (7.13433,7.27709) rectangle (7.17413,7.38294);
\draw [color=c, fill=c] (7.17413,7.27709) rectangle (7.21393,7.38294);
\draw [color=c, fill=c] (7.21393,7.27709) rectangle (7.25373,7.38294);
\draw [color=c, fill=c] (7.25373,7.27709) rectangle (7.29353,7.38294);
\draw [color=c, fill=c] (7.29353,7.27709) rectangle (7.33333,7.38294);
\draw [color=c, fill=c] (7.33333,7.27709) rectangle (7.37313,7.38294);
\draw [color=c, fill=c] (7.37313,7.27709) rectangle (7.41294,7.38294);
\draw [color=c, fill=c] (7.41294,7.27709) rectangle (7.45274,7.38294);
\draw [color=c, fill=c] (7.45274,7.27709) rectangle (7.49254,7.38294);
\draw [color=c, fill=c] (7.49254,7.27709) rectangle (7.53234,7.38294);
\draw [color=c, fill=c] (7.53234,7.27709) rectangle (7.57214,7.38294);
\draw [color=c, fill=c] (7.57214,7.27709) rectangle (7.61194,7.38294);
\draw [color=c, fill=c] (7.61194,7.27709) rectangle (7.65174,7.38294);
\draw [color=c, fill=c] (7.65174,7.27709) rectangle (7.69154,7.38294);
\draw [color=c, fill=c] (7.69154,7.27709) rectangle (7.73134,7.38294);
\draw [color=c, fill=c] (7.73134,7.27709) rectangle (7.77114,7.38294);
\draw [color=c, fill=c] (7.77114,7.27709) rectangle (7.81095,7.38294);
\draw [color=c, fill=c] (7.81095,7.27709) rectangle (7.85075,7.38294);
\draw [color=c, fill=c] (7.85075,7.27709) rectangle (7.89055,7.38294);
\draw [color=c, fill=c] (7.89055,7.27709) rectangle (7.93035,7.38294);
\draw [color=c, fill=c] (7.93035,7.27709) rectangle (7.97015,7.38294);
\draw [color=c, fill=c] (7.97015,7.27709) rectangle (8.00995,7.38294);
\definecolor{c}{rgb}{0,0.0800001,1};
\draw [color=c, fill=c] (8.00995,7.27709) rectangle (8.04975,7.38294);
\draw [color=c, fill=c] (8.04975,7.27709) rectangle (8.08955,7.38294);
\draw [color=c, fill=c] (8.08955,7.27709) rectangle (8.12935,7.38294);
\draw [color=c, fill=c] (8.12935,7.27709) rectangle (8.16915,7.38294);
\draw [color=c, fill=c] (8.16915,7.27709) rectangle (8.20895,7.38294);
\draw [color=c, fill=c] (8.20895,7.27709) rectangle (8.24876,7.38294);
\draw [color=c, fill=c] (8.24876,7.27709) rectangle (8.28856,7.38294);
\draw [color=c, fill=c] (8.28856,7.27709) rectangle (8.32836,7.38294);
\draw [color=c, fill=c] (8.32836,7.27709) rectangle (8.36816,7.38294);
\draw [color=c, fill=c] (8.36816,7.27709) rectangle (8.40796,7.38294);
\draw [color=c, fill=c] (8.40796,7.27709) rectangle (8.44776,7.38294);
\draw [color=c, fill=c] (8.44776,7.27709) rectangle (8.48756,7.38294);
\draw [color=c, fill=c] (8.48756,7.27709) rectangle (8.52736,7.38294);
\draw [color=c, fill=c] (8.52736,7.27709) rectangle (8.56716,7.38294);
\draw [color=c, fill=c] (8.56716,7.27709) rectangle (8.60697,7.38294);
\draw [color=c, fill=c] (8.60697,7.27709) rectangle (8.64677,7.38294);
\draw [color=c, fill=c] (8.64677,7.27709) rectangle (8.68657,7.38294);
\draw [color=c, fill=c] (8.68657,7.27709) rectangle (8.72637,7.38294);
\draw [color=c, fill=c] (8.72637,7.27709) rectangle (8.76617,7.38294);
\draw [color=c, fill=c] (8.76617,7.27709) rectangle (8.80597,7.38294);
\draw [color=c, fill=c] (8.80597,7.27709) rectangle (8.84577,7.38294);
\draw [color=c, fill=c] (8.84577,7.27709) rectangle (8.88557,7.38294);
\draw [color=c, fill=c] (8.88557,7.27709) rectangle (8.92537,7.38294);
\draw [color=c, fill=c] (8.92537,7.27709) rectangle (8.96517,7.38294);
\draw [color=c, fill=c] (8.96517,7.27709) rectangle (9.00498,7.38294);
\draw [color=c, fill=c] (9.00498,7.27709) rectangle (9.04478,7.38294);
\draw [color=c, fill=c] (9.04478,7.27709) rectangle (9.08458,7.38294);
\draw [color=c, fill=c] (9.08458,7.27709) rectangle (9.12438,7.38294);
\draw [color=c, fill=c] (9.12438,7.27709) rectangle (9.16418,7.38294);
\draw [color=c, fill=c] (9.16418,7.27709) rectangle (9.20398,7.38294);
\draw [color=c, fill=c] (9.20398,7.27709) rectangle (9.24378,7.38294);
\draw [color=c, fill=c] (9.24378,7.27709) rectangle (9.28358,7.38294);
\draw [color=c, fill=c] (9.28358,7.27709) rectangle (9.32338,7.38294);
\draw [color=c, fill=c] (9.32338,7.27709) rectangle (9.36318,7.38294);
\draw [color=c, fill=c] (9.36318,7.27709) rectangle (9.40298,7.38294);
\draw [color=c, fill=c] (9.40298,7.27709) rectangle (9.44279,7.38294);
\draw [color=c, fill=c] (9.44279,7.27709) rectangle (9.48259,7.38294);
\draw [color=c, fill=c] (9.48259,7.27709) rectangle (9.52239,7.38294);
\draw [color=c, fill=c] (9.52239,7.27709) rectangle (9.56219,7.38294);
\definecolor{c}{rgb}{0,0.266667,1};
\draw [color=c, fill=c] (9.56219,7.27709) rectangle (9.60199,7.38294);
\draw [color=c, fill=c] (9.60199,7.27709) rectangle (9.64179,7.38294);
\draw [color=c, fill=c] (9.64179,7.27709) rectangle (9.68159,7.38294);
\draw [color=c, fill=c] (9.68159,7.27709) rectangle (9.72139,7.38294);
\draw [color=c, fill=c] (9.72139,7.27709) rectangle (9.76119,7.38294);
\draw [color=c, fill=c] (9.76119,7.27709) rectangle (9.80099,7.38294);
\draw [color=c, fill=c] (9.80099,7.27709) rectangle (9.8408,7.38294);
\draw [color=c, fill=c] (9.8408,7.27709) rectangle (9.8806,7.38294);
\draw [color=c, fill=c] (9.8806,7.27709) rectangle (9.9204,7.38294);
\draw [color=c, fill=c] (9.9204,7.27709) rectangle (9.9602,7.38294);
\draw [color=c, fill=c] (9.9602,7.27709) rectangle (10,7.38294);
\draw [color=c, fill=c] (10,7.27709) rectangle (10.0398,7.38294);
\draw [color=c, fill=c] (10.0398,7.27709) rectangle (10.0796,7.38294);
\draw [color=c, fill=c] (10.0796,7.27709) rectangle (10.1194,7.38294);
\draw [color=c, fill=c] (10.1194,7.27709) rectangle (10.1592,7.38294);
\draw [color=c, fill=c] (10.1592,7.27709) rectangle (10.199,7.38294);
\draw [color=c, fill=c] (10.199,7.27709) rectangle (10.2388,7.38294);
\draw [color=c, fill=c] (10.2388,7.27709) rectangle (10.2786,7.38294);
\definecolor{c}{rgb}{0,0.546666,1};
\draw [color=c, fill=c] (10.2786,7.27709) rectangle (10.3184,7.38294);
\draw [color=c, fill=c] (10.3184,7.27709) rectangle (10.3582,7.38294);
\draw [color=c, fill=c] (10.3582,7.27709) rectangle (10.398,7.38294);
\draw [color=c, fill=c] (10.398,7.27709) rectangle (10.4378,7.38294);
\draw [color=c, fill=c] (10.4378,7.27709) rectangle (10.4776,7.38294);
\draw [color=c, fill=c] (10.4776,7.27709) rectangle (10.5174,7.38294);
\draw [color=c, fill=c] (10.5174,7.27709) rectangle (10.5572,7.38294);
\draw [color=c, fill=c] (10.5572,7.27709) rectangle (10.597,7.38294);
\draw [color=c, fill=c] (10.597,7.27709) rectangle (10.6368,7.38294);
\draw [color=c, fill=c] (10.6368,7.27709) rectangle (10.6766,7.38294);
\draw [color=c, fill=c] (10.6766,7.27709) rectangle (10.7164,7.38294);
\draw [color=c, fill=c] (10.7164,7.27709) rectangle (10.7562,7.38294);
\draw [color=c, fill=c] (10.7562,7.27709) rectangle (10.796,7.38294);
\draw [color=c, fill=c] (10.796,7.27709) rectangle (10.8358,7.38294);
\draw [color=c, fill=c] (10.8358,7.27709) rectangle (10.8756,7.38294);
\draw [color=c, fill=c] (10.8756,7.27709) rectangle (10.9154,7.38294);
\draw [color=c, fill=c] (10.9154,7.27709) rectangle (10.9552,7.38294);
\draw [color=c, fill=c] (10.9552,7.27709) rectangle (10.995,7.38294);
\draw [color=c, fill=c] (10.995,7.27709) rectangle (11.0348,7.38294);
\draw [color=c, fill=c] (11.0348,7.27709) rectangle (11.0746,7.38294);
\draw [color=c, fill=c] (11.0746,7.27709) rectangle (11.1144,7.38294);
\draw [color=c, fill=c] (11.1144,7.27709) rectangle (11.1542,7.38294);
\draw [color=c, fill=c] (11.1542,7.27709) rectangle (11.194,7.38294);
\draw [color=c, fill=c] (11.194,7.27709) rectangle (11.2338,7.38294);
\draw [color=c, fill=c] (11.2338,7.27709) rectangle (11.2736,7.38294);
\draw [color=c, fill=c] (11.2736,7.27709) rectangle (11.3134,7.38294);
\draw [color=c, fill=c] (11.3134,7.27709) rectangle (11.3532,7.38294);
\draw [color=c, fill=c] (11.3532,7.27709) rectangle (11.393,7.38294);
\draw [color=c, fill=c] (11.393,7.27709) rectangle (11.4328,7.38294);
\draw [color=c, fill=c] (11.4328,7.27709) rectangle (11.4726,7.38294);
\draw [color=c, fill=c] (11.4726,7.27709) rectangle (11.5124,7.38294);
\draw [color=c, fill=c] (11.5124,7.27709) rectangle (11.5522,7.38294);
\draw [color=c, fill=c] (11.5522,7.27709) rectangle (11.592,7.38294);
\definecolor{c}{rgb}{0,0.733333,1};
\draw [color=c, fill=c] (11.592,7.27709) rectangle (11.6318,7.38294);
\draw [color=c, fill=c] (11.6318,7.27709) rectangle (11.6716,7.38294);
\draw [color=c, fill=c] (11.6716,7.27709) rectangle (11.7114,7.38294);
\draw [color=c, fill=c] (11.7114,7.27709) rectangle (11.7512,7.38294);
\draw [color=c, fill=c] (11.7512,7.27709) rectangle (11.791,7.38294);
\draw [color=c, fill=c] (11.791,7.27709) rectangle (11.8308,7.38294);
\draw [color=c, fill=c] (11.8308,7.27709) rectangle (11.8706,7.38294);
\draw [color=c, fill=c] (11.8706,7.27709) rectangle (11.9104,7.38294);
\draw [color=c, fill=c] (11.9104,7.27709) rectangle (11.9502,7.38294);
\draw [color=c, fill=c] (11.9502,7.27709) rectangle (11.99,7.38294);
\draw [color=c, fill=c] (11.99,7.27709) rectangle (12.0299,7.38294);
\draw [color=c, fill=c] (12.0299,7.27709) rectangle (12.0697,7.38294);
\draw [color=c, fill=c] (12.0697,7.27709) rectangle (12.1095,7.38294);
\draw [color=c, fill=c] (12.1095,7.27709) rectangle (12.1493,7.38294);
\draw [color=c, fill=c] (12.1493,7.27709) rectangle (12.1891,7.38294);
\draw [color=c, fill=c] (12.1891,7.27709) rectangle (12.2289,7.38294);
\draw [color=c, fill=c] (12.2289,7.27709) rectangle (12.2687,7.38294);
\draw [color=c, fill=c] (12.2687,7.27709) rectangle (12.3085,7.38294);
\draw [color=c, fill=c] (12.3085,7.27709) rectangle (12.3483,7.38294);
\draw [color=c, fill=c] (12.3483,7.27709) rectangle (12.3881,7.38294);
\draw [color=c, fill=c] (12.3881,7.27709) rectangle (12.4279,7.38294);
\draw [color=c, fill=c] (12.4279,7.27709) rectangle (12.4677,7.38294);
\draw [color=c, fill=c] (12.4677,7.27709) rectangle (12.5075,7.38294);
\draw [color=c, fill=c] (12.5075,7.27709) rectangle (12.5473,7.38294);
\draw [color=c, fill=c] (12.5473,7.27709) rectangle (12.5871,7.38294);
\draw [color=c, fill=c] (12.5871,7.27709) rectangle (12.6269,7.38294);
\draw [color=c, fill=c] (12.6269,7.27709) rectangle (12.6667,7.38294);
\draw [color=c, fill=c] (12.6667,7.27709) rectangle (12.7065,7.38294);
\draw [color=c, fill=c] (12.7065,7.27709) rectangle (12.7463,7.38294);
\draw [color=c, fill=c] (12.7463,7.27709) rectangle (12.7861,7.38294);
\draw [color=c, fill=c] (12.7861,7.27709) rectangle (12.8259,7.38294);
\draw [color=c, fill=c] (12.8259,7.27709) rectangle (12.8657,7.38294);
\draw [color=c, fill=c] (12.8657,7.27709) rectangle (12.9055,7.38294);
\draw [color=c, fill=c] (12.9055,7.27709) rectangle (12.9453,7.38294);
\draw [color=c, fill=c] (12.9453,7.27709) rectangle (12.9851,7.38294);
\draw [color=c, fill=c] (12.9851,7.27709) rectangle (13.0249,7.38294);
\draw [color=c, fill=c] (13.0249,7.27709) rectangle (13.0647,7.38294);
\draw [color=c, fill=c] (13.0647,7.27709) rectangle (13.1045,7.38294);
\draw [color=c, fill=c] (13.1045,7.27709) rectangle (13.1443,7.38294);
\draw [color=c, fill=c] (13.1443,7.27709) rectangle (13.1841,7.38294);
\draw [color=c, fill=c] (13.1841,7.27709) rectangle (13.2239,7.38294);
\draw [color=c, fill=c] (13.2239,7.27709) rectangle (13.2637,7.38294);
\draw [color=c, fill=c] (13.2637,7.27709) rectangle (13.3035,7.38294);
\draw [color=c, fill=c] (13.3035,7.27709) rectangle (13.3433,7.38294);
\draw [color=c, fill=c] (13.3433,7.27709) rectangle (13.3831,7.38294);
\draw [color=c, fill=c] (13.3831,7.27709) rectangle (13.4229,7.38294);
\draw [color=c, fill=c] (13.4229,7.27709) rectangle (13.4627,7.38294);
\draw [color=c, fill=c] (13.4627,7.27709) rectangle (13.5025,7.38294);
\draw [color=c, fill=c] (13.5025,7.27709) rectangle (13.5423,7.38294);
\draw [color=c, fill=c] (13.5423,7.27709) rectangle (13.5821,7.38294);
\draw [color=c, fill=c] (13.5821,7.27709) rectangle (13.6219,7.38294);
\draw [color=c, fill=c] (13.6219,7.27709) rectangle (13.6617,7.38294);
\draw [color=c, fill=c] (13.6617,7.27709) rectangle (13.7015,7.38294);
\draw [color=c, fill=c] (13.7015,7.27709) rectangle (13.7413,7.38294);
\draw [color=c, fill=c] (13.7413,7.27709) rectangle (13.7811,7.38294);
\draw [color=c, fill=c] (13.7811,7.27709) rectangle (13.8209,7.38294);
\draw [color=c, fill=c] (13.8209,7.27709) rectangle (13.8607,7.38294);
\draw [color=c, fill=c] (13.8607,7.27709) rectangle (13.9005,7.38294);
\draw [color=c, fill=c] (13.9005,7.27709) rectangle (13.9403,7.38294);
\draw [color=c, fill=c] (13.9403,7.27709) rectangle (13.9801,7.38294);
\draw [color=c, fill=c] (13.9801,7.27709) rectangle (14.0199,7.38294);
\draw [color=c, fill=c] (14.0199,7.27709) rectangle (14.0597,7.38294);
\draw [color=c, fill=c] (14.0597,7.27709) rectangle (14.0995,7.38294);
\draw [color=c, fill=c] (14.0995,7.27709) rectangle (14.1393,7.38294);
\draw [color=c, fill=c] (14.1393,7.27709) rectangle (14.1791,7.38294);
\draw [color=c, fill=c] (14.1791,7.27709) rectangle (14.2189,7.38294);
\draw [color=c, fill=c] (14.2189,7.27709) rectangle (14.2587,7.38294);
\draw [color=c, fill=c] (14.2587,7.27709) rectangle (14.2985,7.38294);
\draw [color=c, fill=c] (14.2985,7.27709) rectangle (14.3383,7.38294);
\draw [color=c, fill=c] (14.3383,7.27709) rectangle (14.3781,7.38294);
\draw [color=c, fill=c] (14.3781,7.27709) rectangle (14.4179,7.38294);
\draw [color=c, fill=c] (14.4179,7.27709) rectangle (14.4577,7.38294);
\draw [color=c, fill=c] (14.4577,7.27709) rectangle (14.4975,7.38294);
\draw [color=c, fill=c] (14.4975,7.27709) rectangle (14.5373,7.38294);
\draw [color=c, fill=c] (14.5373,7.27709) rectangle (14.5771,7.38294);
\draw [color=c, fill=c] (14.5771,7.27709) rectangle (14.6169,7.38294);
\draw [color=c, fill=c] (14.6169,7.27709) rectangle (14.6567,7.38294);
\draw [color=c, fill=c] (14.6567,7.27709) rectangle (14.6965,7.38294);
\draw [color=c, fill=c] (14.6965,7.27709) rectangle (14.7363,7.38294);
\draw [color=c, fill=c] (14.7363,7.27709) rectangle (14.7761,7.38294);
\draw [color=c, fill=c] (14.7761,7.27709) rectangle (14.8159,7.38294);
\draw [color=c, fill=c] (14.8159,7.27709) rectangle (14.8557,7.38294);
\draw [color=c, fill=c] (14.8557,7.27709) rectangle (14.8955,7.38294);
\draw [color=c, fill=c] (14.8955,7.27709) rectangle (14.9353,7.38294);
\draw [color=c, fill=c] (14.9353,7.27709) rectangle (14.9751,7.38294);
\draw [color=c, fill=c] (14.9751,7.27709) rectangle (15.0149,7.38294);
\draw [color=c, fill=c] (15.0149,7.27709) rectangle (15.0547,7.38294);
\draw [color=c, fill=c] (15.0547,7.27709) rectangle (15.0945,7.38294);
\draw [color=c, fill=c] (15.0945,7.27709) rectangle (15.1343,7.38294);
\draw [color=c, fill=c] (15.1343,7.27709) rectangle (15.1741,7.38294);
\draw [color=c, fill=c] (15.1741,7.27709) rectangle (15.2139,7.38294);
\draw [color=c, fill=c] (15.2139,7.27709) rectangle (15.2537,7.38294);
\draw [color=c, fill=c] (15.2537,7.27709) rectangle (15.2935,7.38294);
\draw [color=c, fill=c] (15.2935,7.27709) rectangle (15.3333,7.38294);
\draw [color=c, fill=c] (15.3333,7.27709) rectangle (15.3731,7.38294);
\draw [color=c, fill=c] (15.3731,7.27709) rectangle (15.4129,7.38294);
\draw [color=c, fill=c] (15.4129,7.27709) rectangle (15.4527,7.38294);
\draw [color=c, fill=c] (15.4527,7.27709) rectangle (15.4925,7.38294);
\draw [color=c, fill=c] (15.4925,7.27709) rectangle (15.5323,7.38294);
\draw [color=c, fill=c] (15.5323,7.27709) rectangle (15.5721,7.38294);
\draw [color=c, fill=c] (15.5721,7.27709) rectangle (15.6119,7.38294);
\draw [color=c, fill=c] (15.6119,7.27709) rectangle (15.6517,7.38294);
\draw [color=c, fill=c] (15.6517,7.27709) rectangle (15.6915,7.38294);
\draw [color=c, fill=c] (15.6915,7.27709) rectangle (15.7313,7.38294);
\draw [color=c, fill=c] (15.7313,7.27709) rectangle (15.7711,7.38294);
\draw [color=c, fill=c] (15.7711,7.27709) rectangle (15.8109,7.38294);
\draw [color=c, fill=c] (15.8109,7.27709) rectangle (15.8507,7.38294);
\draw [color=c, fill=c] (15.8507,7.27709) rectangle (15.8905,7.38294);
\draw [color=c, fill=c] (15.8905,7.27709) rectangle (15.9303,7.38294);
\draw [color=c, fill=c] (15.9303,7.27709) rectangle (15.9701,7.38294);
\draw [color=c, fill=c] (15.9701,7.27709) rectangle (16.01,7.38294);
\draw [color=c, fill=c] (16.01,7.27709) rectangle (16.0498,7.38294);
\draw [color=c, fill=c] (16.0498,7.27709) rectangle (16.0896,7.38294);
\draw [color=c, fill=c] (16.0896,7.27709) rectangle (16.1294,7.38294);
\draw [color=c, fill=c] (16.1294,7.27709) rectangle (16.1692,7.38294);
\draw [color=c, fill=c] (16.1692,7.27709) rectangle (16.209,7.38294);
\draw [color=c, fill=c] (16.209,7.27709) rectangle (16.2488,7.38294);
\draw [color=c, fill=c] (16.2488,7.27709) rectangle (16.2886,7.38294);
\draw [color=c, fill=c] (16.2886,7.27709) rectangle (16.3284,7.38294);
\draw [color=c, fill=c] (16.3284,7.27709) rectangle (16.3682,7.38294);
\draw [color=c, fill=c] (16.3682,7.27709) rectangle (16.408,7.38294);
\draw [color=c, fill=c] (16.408,7.27709) rectangle (16.4478,7.38294);
\draw [color=c, fill=c] (16.4478,7.27709) rectangle (16.4876,7.38294);
\draw [color=c, fill=c] (16.4876,7.27709) rectangle (16.5274,7.38294);
\draw [color=c, fill=c] (16.5274,7.27709) rectangle (16.5672,7.38294);
\draw [color=c, fill=c] (16.5672,7.27709) rectangle (16.607,7.38294);
\draw [color=c, fill=c] (16.607,7.27709) rectangle (16.6468,7.38294);
\draw [color=c, fill=c] (16.6468,7.27709) rectangle (16.6866,7.38294);
\draw [color=c, fill=c] (16.6866,7.27709) rectangle (16.7264,7.38294);
\draw [color=c, fill=c] (16.7264,7.27709) rectangle (16.7662,7.38294);
\draw [color=c, fill=c] (16.7662,7.27709) rectangle (16.806,7.38294);
\draw [color=c, fill=c] (16.806,7.27709) rectangle (16.8458,7.38294);
\draw [color=c, fill=c] (16.8458,7.27709) rectangle (16.8856,7.38294);
\draw [color=c, fill=c] (16.8856,7.27709) rectangle (16.9254,7.38294);
\draw [color=c, fill=c] (16.9254,7.27709) rectangle (16.9652,7.38294);
\draw [color=c, fill=c] (16.9652,7.27709) rectangle (17.005,7.38294);
\draw [color=c, fill=c] (17.005,7.27709) rectangle (17.0448,7.38294);
\draw [color=c, fill=c] (17.0448,7.27709) rectangle (17.0846,7.38294);
\draw [color=c, fill=c] (17.0846,7.27709) rectangle (17.1244,7.38294);
\draw [color=c, fill=c] (17.1244,7.27709) rectangle (17.1642,7.38294);
\draw [color=c, fill=c] (17.1642,7.27709) rectangle (17.204,7.38294);
\draw [color=c, fill=c] (17.204,7.27709) rectangle (17.2438,7.38294);
\draw [color=c, fill=c] (17.2438,7.27709) rectangle (17.2836,7.38294);
\draw [color=c, fill=c] (17.2836,7.27709) rectangle (17.3234,7.38294);
\draw [color=c, fill=c] (17.3234,7.27709) rectangle (17.3632,7.38294);
\draw [color=c, fill=c] (17.3632,7.27709) rectangle (17.403,7.38294);
\draw [color=c, fill=c] (17.403,7.27709) rectangle (17.4428,7.38294);
\draw [color=c, fill=c] (17.4428,7.27709) rectangle (17.4826,7.38294);
\draw [color=c, fill=c] (17.4826,7.27709) rectangle (17.5224,7.38294);
\draw [color=c, fill=c] (17.5224,7.27709) rectangle (17.5622,7.38294);
\draw [color=c, fill=c] (17.5622,7.27709) rectangle (17.602,7.38294);
\draw [color=c, fill=c] (17.602,7.27709) rectangle (17.6418,7.38294);
\draw [color=c, fill=c] (17.6418,7.27709) rectangle (17.6816,7.38294);
\draw [color=c, fill=c] (17.6816,7.27709) rectangle (17.7214,7.38294);
\draw [color=c, fill=c] (17.7214,7.27709) rectangle (17.7612,7.38294);
\draw [color=c, fill=c] (17.7612,7.27709) rectangle (17.801,7.38294);
\draw [color=c, fill=c] (17.801,7.27709) rectangle (17.8408,7.38294);
\draw [color=c, fill=c] (17.8408,7.27709) rectangle (17.8806,7.38294);
\draw [color=c, fill=c] (17.8806,7.27709) rectangle (17.9204,7.38294);
\draw [color=c, fill=c] (17.9204,7.27709) rectangle (17.9602,7.38294);
\draw [color=c, fill=c] (17.9602,7.27709) rectangle (18,7.38294);
\definecolor{c}{rgb}{0.2,0,1};
\draw [color=c, fill=c] (2,7.38294) rectangle (2.0398,7.48879);
\draw [color=c, fill=c] (2.0398,7.38294) rectangle (2.0796,7.48879);
\draw [color=c, fill=c] (2.0796,7.38294) rectangle (2.1194,7.48879);
\draw [color=c, fill=c] (2.1194,7.38294) rectangle (2.1592,7.48879);
\draw [color=c, fill=c] (2.1592,7.38294) rectangle (2.19901,7.48879);
\draw [color=c, fill=c] (2.19901,7.38294) rectangle (2.23881,7.48879);
\draw [color=c, fill=c] (2.23881,7.38294) rectangle (2.27861,7.48879);
\draw [color=c, fill=c] (2.27861,7.38294) rectangle (2.31841,7.48879);
\draw [color=c, fill=c] (2.31841,7.38294) rectangle (2.35821,7.48879);
\draw [color=c, fill=c] (2.35821,7.38294) rectangle (2.39801,7.48879);
\draw [color=c, fill=c] (2.39801,7.38294) rectangle (2.43781,7.48879);
\draw [color=c, fill=c] (2.43781,7.38294) rectangle (2.47761,7.48879);
\draw [color=c, fill=c] (2.47761,7.38294) rectangle (2.51741,7.48879);
\draw [color=c, fill=c] (2.51741,7.38294) rectangle (2.55721,7.48879);
\draw [color=c, fill=c] (2.55721,7.38294) rectangle (2.59702,7.48879);
\draw [color=c, fill=c] (2.59702,7.38294) rectangle (2.63682,7.48879);
\draw [color=c, fill=c] (2.63682,7.38294) rectangle (2.67662,7.48879);
\draw [color=c, fill=c] (2.67662,7.38294) rectangle (2.71642,7.48879);
\draw [color=c, fill=c] (2.71642,7.38294) rectangle (2.75622,7.48879);
\draw [color=c, fill=c] (2.75622,7.38294) rectangle (2.79602,7.48879);
\draw [color=c, fill=c] (2.79602,7.38294) rectangle (2.83582,7.48879);
\draw [color=c, fill=c] (2.83582,7.38294) rectangle (2.87562,7.48879);
\draw [color=c, fill=c] (2.87562,7.38294) rectangle (2.91542,7.48879);
\draw [color=c, fill=c] (2.91542,7.38294) rectangle (2.95522,7.48879);
\draw [color=c, fill=c] (2.95522,7.38294) rectangle (2.99502,7.48879);
\draw [color=c, fill=c] (2.99502,7.38294) rectangle (3.03483,7.48879);
\draw [color=c, fill=c] (3.03483,7.38294) rectangle (3.07463,7.48879);
\draw [color=c, fill=c] (3.07463,7.38294) rectangle (3.11443,7.48879);
\draw [color=c, fill=c] (3.11443,7.38294) rectangle (3.15423,7.48879);
\draw [color=c, fill=c] (3.15423,7.38294) rectangle (3.19403,7.48879);
\draw [color=c, fill=c] (3.19403,7.38294) rectangle (3.23383,7.48879);
\draw [color=c, fill=c] (3.23383,7.38294) rectangle (3.27363,7.48879);
\draw [color=c, fill=c] (3.27363,7.38294) rectangle (3.31343,7.48879);
\draw [color=c, fill=c] (3.31343,7.38294) rectangle (3.35323,7.48879);
\draw [color=c, fill=c] (3.35323,7.38294) rectangle (3.39303,7.48879);
\draw [color=c, fill=c] (3.39303,7.38294) rectangle (3.43284,7.48879);
\draw [color=c, fill=c] (3.43284,7.38294) rectangle (3.47264,7.48879);
\draw [color=c, fill=c] (3.47264,7.38294) rectangle (3.51244,7.48879);
\draw [color=c, fill=c] (3.51244,7.38294) rectangle (3.55224,7.48879);
\draw [color=c, fill=c] (3.55224,7.38294) rectangle (3.59204,7.48879);
\draw [color=c, fill=c] (3.59204,7.38294) rectangle (3.63184,7.48879);
\draw [color=c, fill=c] (3.63184,7.38294) rectangle (3.67164,7.48879);
\draw [color=c, fill=c] (3.67164,7.38294) rectangle (3.71144,7.48879);
\draw [color=c, fill=c] (3.71144,7.38294) rectangle (3.75124,7.48879);
\draw [color=c, fill=c] (3.75124,7.38294) rectangle (3.79104,7.48879);
\draw [color=c, fill=c] (3.79104,7.38294) rectangle (3.83085,7.48879);
\draw [color=c, fill=c] (3.83085,7.38294) rectangle (3.87065,7.48879);
\draw [color=c, fill=c] (3.87065,7.38294) rectangle (3.91045,7.48879);
\draw [color=c, fill=c] (3.91045,7.38294) rectangle (3.95025,7.48879);
\draw [color=c, fill=c] (3.95025,7.38294) rectangle (3.99005,7.48879);
\draw [color=c, fill=c] (3.99005,7.38294) rectangle (4.02985,7.48879);
\draw [color=c, fill=c] (4.02985,7.38294) rectangle (4.06965,7.48879);
\draw [color=c, fill=c] (4.06965,7.38294) rectangle (4.10945,7.48879);
\draw [color=c, fill=c] (4.10945,7.38294) rectangle (4.14925,7.48879);
\draw [color=c, fill=c] (4.14925,7.38294) rectangle (4.18905,7.48879);
\draw [color=c, fill=c] (4.18905,7.38294) rectangle (4.22886,7.48879);
\draw [color=c, fill=c] (4.22886,7.38294) rectangle (4.26866,7.48879);
\draw [color=c, fill=c] (4.26866,7.38294) rectangle (4.30846,7.48879);
\draw [color=c, fill=c] (4.30846,7.38294) rectangle (4.34826,7.48879);
\draw [color=c, fill=c] (4.34826,7.38294) rectangle (4.38806,7.48879);
\draw [color=c, fill=c] (4.38806,7.38294) rectangle (4.42786,7.48879);
\draw [color=c, fill=c] (4.42786,7.38294) rectangle (4.46766,7.48879);
\draw [color=c, fill=c] (4.46766,7.38294) rectangle (4.50746,7.48879);
\draw [color=c, fill=c] (4.50746,7.38294) rectangle (4.54726,7.48879);
\draw [color=c, fill=c] (4.54726,7.38294) rectangle (4.58706,7.48879);
\draw [color=c, fill=c] (4.58706,7.38294) rectangle (4.62687,7.48879);
\draw [color=c, fill=c] (4.62687,7.38294) rectangle (4.66667,7.48879);
\draw [color=c, fill=c] (4.66667,7.38294) rectangle (4.70647,7.48879);
\draw [color=c, fill=c] (4.70647,7.38294) rectangle (4.74627,7.48879);
\draw [color=c, fill=c] (4.74627,7.38294) rectangle (4.78607,7.48879);
\draw [color=c, fill=c] (4.78607,7.38294) rectangle (4.82587,7.48879);
\draw [color=c, fill=c] (4.82587,7.38294) rectangle (4.86567,7.48879);
\draw [color=c, fill=c] (4.86567,7.38294) rectangle (4.90547,7.48879);
\draw [color=c, fill=c] (4.90547,7.38294) rectangle (4.94527,7.48879);
\draw [color=c, fill=c] (4.94527,7.38294) rectangle (4.98507,7.48879);
\draw [color=c, fill=c] (4.98507,7.38294) rectangle (5.02488,7.48879);
\draw [color=c, fill=c] (5.02488,7.38294) rectangle (5.06468,7.48879);
\draw [color=c, fill=c] (5.06468,7.38294) rectangle (5.10448,7.48879);
\draw [color=c, fill=c] (5.10448,7.38294) rectangle (5.14428,7.48879);
\draw [color=c, fill=c] (5.14428,7.38294) rectangle (5.18408,7.48879);
\draw [color=c, fill=c] (5.18408,7.38294) rectangle (5.22388,7.48879);
\draw [color=c, fill=c] (5.22388,7.38294) rectangle (5.26368,7.48879);
\draw [color=c, fill=c] (5.26368,7.38294) rectangle (5.30348,7.48879);
\draw [color=c, fill=c] (5.30348,7.38294) rectangle (5.34328,7.48879);
\draw [color=c, fill=c] (5.34328,7.38294) rectangle (5.38308,7.48879);
\draw [color=c, fill=c] (5.38308,7.38294) rectangle (5.42289,7.48879);
\draw [color=c, fill=c] (5.42289,7.38294) rectangle (5.46269,7.48879);
\draw [color=c, fill=c] (5.46269,7.38294) rectangle (5.50249,7.48879);
\draw [color=c, fill=c] (5.50249,7.38294) rectangle (5.54229,7.48879);
\draw [color=c, fill=c] (5.54229,7.38294) rectangle (5.58209,7.48879);
\draw [color=c, fill=c] (5.58209,7.38294) rectangle (5.62189,7.48879);
\draw [color=c, fill=c] (5.62189,7.38294) rectangle (5.66169,7.48879);
\draw [color=c, fill=c] (5.66169,7.38294) rectangle (5.70149,7.48879);
\draw [color=c, fill=c] (5.70149,7.38294) rectangle (5.74129,7.48879);
\draw [color=c, fill=c] (5.74129,7.38294) rectangle (5.78109,7.48879);
\draw [color=c, fill=c] (5.78109,7.38294) rectangle (5.8209,7.48879);
\draw [color=c, fill=c] (5.8209,7.38294) rectangle (5.8607,7.48879);
\draw [color=c, fill=c] (5.8607,7.38294) rectangle (5.9005,7.48879);
\draw [color=c, fill=c] (5.9005,7.38294) rectangle (5.9403,7.48879);
\draw [color=c, fill=c] (5.9403,7.38294) rectangle (5.9801,7.48879);
\draw [color=c, fill=c] (5.9801,7.38294) rectangle (6.0199,7.48879);
\draw [color=c, fill=c] (6.0199,7.38294) rectangle (6.0597,7.48879);
\draw [color=c, fill=c] (6.0597,7.38294) rectangle (6.0995,7.48879);
\draw [color=c, fill=c] (6.0995,7.38294) rectangle (6.1393,7.48879);
\draw [color=c, fill=c] (6.1393,7.38294) rectangle (6.1791,7.48879);
\draw [color=c, fill=c] (6.1791,7.38294) rectangle (6.21891,7.48879);
\draw [color=c, fill=c] (6.21891,7.38294) rectangle (6.25871,7.48879);
\draw [color=c, fill=c] (6.25871,7.38294) rectangle (6.29851,7.48879);
\draw [color=c, fill=c] (6.29851,7.38294) rectangle (6.33831,7.48879);
\draw [color=c, fill=c] (6.33831,7.38294) rectangle (6.37811,7.48879);
\draw [color=c, fill=c] (6.37811,7.38294) rectangle (6.41791,7.48879);
\draw [color=c, fill=c] (6.41791,7.38294) rectangle (6.45771,7.48879);
\draw [color=c, fill=c] (6.45771,7.38294) rectangle (6.49751,7.48879);
\draw [color=c, fill=c] (6.49751,7.38294) rectangle (6.53731,7.48879);
\draw [color=c, fill=c] (6.53731,7.38294) rectangle (6.57711,7.48879);
\draw [color=c, fill=c] (6.57711,7.38294) rectangle (6.61692,7.48879);
\draw [color=c, fill=c] (6.61692,7.38294) rectangle (6.65672,7.48879);
\draw [color=c, fill=c] (6.65672,7.38294) rectangle (6.69652,7.48879);
\draw [color=c, fill=c] (6.69652,7.38294) rectangle (6.73632,7.48879);
\draw [color=c, fill=c] (6.73632,7.38294) rectangle (6.77612,7.48879);
\draw [color=c, fill=c] (6.77612,7.38294) rectangle (6.81592,7.48879);
\draw [color=c, fill=c] (6.81592,7.38294) rectangle (6.85572,7.48879);
\draw [color=c, fill=c] (6.85572,7.38294) rectangle (6.89552,7.48879);
\draw [color=c, fill=c] (6.89552,7.38294) rectangle (6.93532,7.48879);
\draw [color=c, fill=c] (6.93532,7.38294) rectangle (6.97512,7.48879);
\draw [color=c, fill=c] (6.97512,7.38294) rectangle (7.01493,7.48879);
\draw [color=c, fill=c] (7.01493,7.38294) rectangle (7.05473,7.48879);
\draw [color=c, fill=c] (7.05473,7.38294) rectangle (7.09453,7.48879);
\draw [color=c, fill=c] (7.09453,7.38294) rectangle (7.13433,7.48879);
\draw [color=c, fill=c] (7.13433,7.38294) rectangle (7.17413,7.48879);
\draw [color=c, fill=c] (7.17413,7.38294) rectangle (7.21393,7.48879);
\draw [color=c, fill=c] (7.21393,7.38294) rectangle (7.25373,7.48879);
\draw [color=c, fill=c] (7.25373,7.38294) rectangle (7.29353,7.48879);
\draw [color=c, fill=c] (7.29353,7.38294) rectangle (7.33333,7.48879);
\draw [color=c, fill=c] (7.33333,7.38294) rectangle (7.37313,7.48879);
\draw [color=c, fill=c] (7.37313,7.38294) rectangle (7.41294,7.48879);
\draw [color=c, fill=c] (7.41294,7.38294) rectangle (7.45274,7.48879);
\draw [color=c, fill=c] (7.45274,7.38294) rectangle (7.49254,7.48879);
\draw [color=c, fill=c] (7.49254,7.38294) rectangle (7.53234,7.48879);
\draw [color=c, fill=c] (7.53234,7.38294) rectangle (7.57214,7.48879);
\draw [color=c, fill=c] (7.57214,7.38294) rectangle (7.61194,7.48879);
\draw [color=c, fill=c] (7.61194,7.38294) rectangle (7.65174,7.48879);
\draw [color=c, fill=c] (7.65174,7.38294) rectangle (7.69154,7.48879);
\draw [color=c, fill=c] (7.69154,7.38294) rectangle (7.73134,7.48879);
\draw [color=c, fill=c] (7.73134,7.38294) rectangle (7.77114,7.48879);
\draw [color=c, fill=c] (7.77114,7.38294) rectangle (7.81095,7.48879);
\draw [color=c, fill=c] (7.81095,7.38294) rectangle (7.85075,7.48879);
\draw [color=c, fill=c] (7.85075,7.38294) rectangle (7.89055,7.48879);
\draw [color=c, fill=c] (7.89055,7.38294) rectangle (7.93035,7.48879);
\definecolor{c}{rgb}{0,0.0800001,1};
\draw [color=c, fill=c] (7.93035,7.38294) rectangle (7.97015,7.48879);
\draw [color=c, fill=c] (7.97015,7.38294) rectangle (8.00995,7.48879);
\draw [color=c, fill=c] (8.00995,7.38294) rectangle (8.04975,7.48879);
\draw [color=c, fill=c] (8.04975,7.38294) rectangle (8.08955,7.48879);
\draw [color=c, fill=c] (8.08955,7.38294) rectangle (8.12935,7.48879);
\draw [color=c, fill=c] (8.12935,7.38294) rectangle (8.16915,7.48879);
\draw [color=c, fill=c] (8.16915,7.38294) rectangle (8.20895,7.48879);
\draw [color=c, fill=c] (8.20895,7.38294) rectangle (8.24876,7.48879);
\draw [color=c, fill=c] (8.24876,7.38294) rectangle (8.28856,7.48879);
\draw [color=c, fill=c] (8.28856,7.38294) rectangle (8.32836,7.48879);
\draw [color=c, fill=c] (8.32836,7.38294) rectangle (8.36816,7.48879);
\draw [color=c, fill=c] (8.36816,7.38294) rectangle (8.40796,7.48879);
\draw [color=c, fill=c] (8.40796,7.38294) rectangle (8.44776,7.48879);
\draw [color=c, fill=c] (8.44776,7.38294) rectangle (8.48756,7.48879);
\draw [color=c, fill=c] (8.48756,7.38294) rectangle (8.52736,7.48879);
\draw [color=c, fill=c] (8.52736,7.38294) rectangle (8.56716,7.48879);
\draw [color=c, fill=c] (8.56716,7.38294) rectangle (8.60697,7.48879);
\draw [color=c, fill=c] (8.60697,7.38294) rectangle (8.64677,7.48879);
\draw [color=c, fill=c] (8.64677,7.38294) rectangle (8.68657,7.48879);
\draw [color=c, fill=c] (8.68657,7.38294) rectangle (8.72637,7.48879);
\draw [color=c, fill=c] (8.72637,7.38294) rectangle (8.76617,7.48879);
\draw [color=c, fill=c] (8.76617,7.38294) rectangle (8.80597,7.48879);
\draw [color=c, fill=c] (8.80597,7.38294) rectangle (8.84577,7.48879);
\draw [color=c, fill=c] (8.84577,7.38294) rectangle (8.88557,7.48879);
\draw [color=c, fill=c] (8.88557,7.38294) rectangle (8.92537,7.48879);
\draw [color=c, fill=c] (8.92537,7.38294) rectangle (8.96517,7.48879);
\draw [color=c, fill=c] (8.96517,7.38294) rectangle (9.00498,7.48879);
\draw [color=c, fill=c] (9.00498,7.38294) rectangle (9.04478,7.48879);
\draw [color=c, fill=c] (9.04478,7.38294) rectangle (9.08458,7.48879);
\draw [color=c, fill=c] (9.08458,7.38294) rectangle (9.12438,7.48879);
\draw [color=c, fill=c] (9.12438,7.38294) rectangle (9.16418,7.48879);
\draw [color=c, fill=c] (9.16418,7.38294) rectangle (9.20398,7.48879);
\draw [color=c, fill=c] (9.20398,7.38294) rectangle (9.24378,7.48879);
\draw [color=c, fill=c] (9.24378,7.38294) rectangle (9.28358,7.48879);
\draw [color=c, fill=c] (9.28358,7.38294) rectangle (9.32338,7.48879);
\draw [color=c, fill=c] (9.32338,7.38294) rectangle (9.36318,7.48879);
\draw [color=c, fill=c] (9.36318,7.38294) rectangle (9.40298,7.48879);
\draw [color=c, fill=c] (9.40298,7.38294) rectangle (9.44279,7.48879);
\draw [color=c, fill=c] (9.44279,7.38294) rectangle (9.48259,7.48879);
\draw [color=c, fill=c] (9.48259,7.38294) rectangle (9.52239,7.48879);
\draw [color=c, fill=c] (9.52239,7.38294) rectangle (9.56219,7.48879);
\definecolor{c}{rgb}{0,0.266667,1};
\draw [color=c, fill=c] (9.56219,7.38294) rectangle (9.60199,7.48879);
\draw [color=c, fill=c] (9.60199,7.38294) rectangle (9.64179,7.48879);
\draw [color=c, fill=c] (9.64179,7.38294) rectangle (9.68159,7.48879);
\draw [color=c, fill=c] (9.68159,7.38294) rectangle (9.72139,7.48879);
\draw [color=c, fill=c] (9.72139,7.38294) rectangle (9.76119,7.48879);
\draw [color=c, fill=c] (9.76119,7.38294) rectangle (9.80099,7.48879);
\draw [color=c, fill=c] (9.80099,7.38294) rectangle (9.8408,7.48879);
\draw [color=c, fill=c] (9.8408,7.38294) rectangle (9.8806,7.48879);
\draw [color=c, fill=c] (9.8806,7.38294) rectangle (9.9204,7.48879);
\draw [color=c, fill=c] (9.9204,7.38294) rectangle (9.9602,7.48879);
\draw [color=c, fill=c] (9.9602,7.38294) rectangle (10,7.48879);
\draw [color=c, fill=c] (10,7.38294) rectangle (10.0398,7.48879);
\draw [color=c, fill=c] (10.0398,7.38294) rectangle (10.0796,7.48879);
\draw [color=c, fill=c] (10.0796,7.38294) rectangle (10.1194,7.48879);
\draw [color=c, fill=c] (10.1194,7.38294) rectangle (10.1592,7.48879);
\draw [color=c, fill=c] (10.1592,7.38294) rectangle (10.199,7.48879);
\draw [color=c, fill=c] (10.199,7.38294) rectangle (10.2388,7.48879);
\draw [color=c, fill=c] (10.2388,7.38294) rectangle (10.2786,7.48879);
\draw [color=c, fill=c] (10.2786,7.38294) rectangle (10.3184,7.48879);
\definecolor{c}{rgb}{0,0.546666,1};
\draw [color=c, fill=c] (10.3184,7.38294) rectangle (10.3582,7.48879);
\draw [color=c, fill=c] (10.3582,7.38294) rectangle (10.398,7.48879);
\draw [color=c, fill=c] (10.398,7.38294) rectangle (10.4378,7.48879);
\draw [color=c, fill=c] (10.4378,7.38294) rectangle (10.4776,7.48879);
\draw [color=c, fill=c] (10.4776,7.38294) rectangle (10.5174,7.48879);
\draw [color=c, fill=c] (10.5174,7.38294) rectangle (10.5572,7.48879);
\draw [color=c, fill=c] (10.5572,7.38294) rectangle (10.597,7.48879);
\draw [color=c, fill=c] (10.597,7.38294) rectangle (10.6368,7.48879);
\draw [color=c, fill=c] (10.6368,7.38294) rectangle (10.6766,7.48879);
\draw [color=c, fill=c] (10.6766,7.38294) rectangle (10.7164,7.48879);
\draw [color=c, fill=c] (10.7164,7.38294) rectangle (10.7562,7.48879);
\draw [color=c, fill=c] (10.7562,7.38294) rectangle (10.796,7.48879);
\draw [color=c, fill=c] (10.796,7.38294) rectangle (10.8358,7.48879);
\draw [color=c, fill=c] (10.8358,7.38294) rectangle (10.8756,7.48879);
\draw [color=c, fill=c] (10.8756,7.38294) rectangle (10.9154,7.48879);
\draw [color=c, fill=c] (10.9154,7.38294) rectangle (10.9552,7.48879);
\draw [color=c, fill=c] (10.9552,7.38294) rectangle (10.995,7.48879);
\draw [color=c, fill=c] (10.995,7.38294) rectangle (11.0348,7.48879);
\draw [color=c, fill=c] (11.0348,7.38294) rectangle (11.0746,7.48879);
\draw [color=c, fill=c] (11.0746,7.38294) rectangle (11.1144,7.48879);
\draw [color=c, fill=c] (11.1144,7.38294) rectangle (11.1542,7.48879);
\draw [color=c, fill=c] (11.1542,7.38294) rectangle (11.194,7.48879);
\draw [color=c, fill=c] (11.194,7.38294) rectangle (11.2338,7.48879);
\draw [color=c, fill=c] (11.2338,7.38294) rectangle (11.2736,7.48879);
\draw [color=c, fill=c] (11.2736,7.38294) rectangle (11.3134,7.48879);
\draw [color=c, fill=c] (11.3134,7.38294) rectangle (11.3532,7.48879);
\draw [color=c, fill=c] (11.3532,7.38294) rectangle (11.393,7.48879);
\draw [color=c, fill=c] (11.393,7.38294) rectangle (11.4328,7.48879);
\draw [color=c, fill=c] (11.4328,7.38294) rectangle (11.4726,7.48879);
\draw [color=c, fill=c] (11.4726,7.38294) rectangle (11.5124,7.48879);
\draw [color=c, fill=c] (11.5124,7.38294) rectangle (11.5522,7.48879);
\draw [color=c, fill=c] (11.5522,7.38294) rectangle (11.592,7.48879);
\draw [color=c, fill=c] (11.592,7.38294) rectangle (11.6318,7.48879);
\draw [color=c, fill=c] (11.6318,7.38294) rectangle (11.6716,7.48879);
\draw [color=c, fill=c] (11.6716,7.38294) rectangle (11.7114,7.48879);
\definecolor{c}{rgb}{0,0.733333,1};
\draw [color=c, fill=c] (11.7114,7.38294) rectangle (11.7512,7.48879);
\draw [color=c, fill=c] (11.7512,7.38294) rectangle (11.791,7.48879);
\draw [color=c, fill=c] (11.791,7.38294) rectangle (11.8308,7.48879);
\draw [color=c, fill=c] (11.8308,7.38294) rectangle (11.8706,7.48879);
\draw [color=c, fill=c] (11.8706,7.38294) rectangle (11.9104,7.48879);
\draw [color=c, fill=c] (11.9104,7.38294) rectangle (11.9502,7.48879);
\draw [color=c, fill=c] (11.9502,7.38294) rectangle (11.99,7.48879);
\draw [color=c, fill=c] (11.99,7.38294) rectangle (12.0299,7.48879);
\draw [color=c, fill=c] (12.0299,7.38294) rectangle (12.0697,7.48879);
\draw [color=c, fill=c] (12.0697,7.38294) rectangle (12.1095,7.48879);
\draw [color=c, fill=c] (12.1095,7.38294) rectangle (12.1493,7.48879);
\draw [color=c, fill=c] (12.1493,7.38294) rectangle (12.1891,7.48879);
\draw [color=c, fill=c] (12.1891,7.38294) rectangle (12.2289,7.48879);
\draw [color=c, fill=c] (12.2289,7.38294) rectangle (12.2687,7.48879);
\draw [color=c, fill=c] (12.2687,7.38294) rectangle (12.3085,7.48879);
\draw [color=c, fill=c] (12.3085,7.38294) rectangle (12.3483,7.48879);
\draw [color=c, fill=c] (12.3483,7.38294) rectangle (12.3881,7.48879);
\draw [color=c, fill=c] (12.3881,7.38294) rectangle (12.4279,7.48879);
\draw [color=c, fill=c] (12.4279,7.38294) rectangle (12.4677,7.48879);
\draw [color=c, fill=c] (12.4677,7.38294) rectangle (12.5075,7.48879);
\draw [color=c, fill=c] (12.5075,7.38294) rectangle (12.5473,7.48879);
\draw [color=c, fill=c] (12.5473,7.38294) rectangle (12.5871,7.48879);
\draw [color=c, fill=c] (12.5871,7.38294) rectangle (12.6269,7.48879);
\draw [color=c, fill=c] (12.6269,7.38294) rectangle (12.6667,7.48879);
\draw [color=c, fill=c] (12.6667,7.38294) rectangle (12.7065,7.48879);
\draw [color=c, fill=c] (12.7065,7.38294) rectangle (12.7463,7.48879);
\draw [color=c, fill=c] (12.7463,7.38294) rectangle (12.7861,7.48879);
\draw [color=c, fill=c] (12.7861,7.38294) rectangle (12.8259,7.48879);
\draw [color=c, fill=c] (12.8259,7.38294) rectangle (12.8657,7.48879);
\draw [color=c, fill=c] (12.8657,7.38294) rectangle (12.9055,7.48879);
\draw [color=c, fill=c] (12.9055,7.38294) rectangle (12.9453,7.48879);
\draw [color=c, fill=c] (12.9453,7.38294) rectangle (12.9851,7.48879);
\draw [color=c, fill=c] (12.9851,7.38294) rectangle (13.0249,7.48879);
\draw [color=c, fill=c] (13.0249,7.38294) rectangle (13.0647,7.48879);
\draw [color=c, fill=c] (13.0647,7.38294) rectangle (13.1045,7.48879);
\draw [color=c, fill=c] (13.1045,7.38294) rectangle (13.1443,7.48879);
\draw [color=c, fill=c] (13.1443,7.38294) rectangle (13.1841,7.48879);
\draw [color=c, fill=c] (13.1841,7.38294) rectangle (13.2239,7.48879);
\draw [color=c, fill=c] (13.2239,7.38294) rectangle (13.2637,7.48879);
\draw [color=c, fill=c] (13.2637,7.38294) rectangle (13.3035,7.48879);
\draw [color=c, fill=c] (13.3035,7.38294) rectangle (13.3433,7.48879);
\draw [color=c, fill=c] (13.3433,7.38294) rectangle (13.3831,7.48879);
\draw [color=c, fill=c] (13.3831,7.38294) rectangle (13.4229,7.48879);
\draw [color=c, fill=c] (13.4229,7.38294) rectangle (13.4627,7.48879);
\draw [color=c, fill=c] (13.4627,7.38294) rectangle (13.5025,7.48879);
\draw [color=c, fill=c] (13.5025,7.38294) rectangle (13.5423,7.48879);
\draw [color=c, fill=c] (13.5423,7.38294) rectangle (13.5821,7.48879);
\draw [color=c, fill=c] (13.5821,7.38294) rectangle (13.6219,7.48879);
\draw [color=c, fill=c] (13.6219,7.38294) rectangle (13.6617,7.48879);
\draw [color=c, fill=c] (13.6617,7.38294) rectangle (13.7015,7.48879);
\draw [color=c, fill=c] (13.7015,7.38294) rectangle (13.7413,7.48879);
\draw [color=c, fill=c] (13.7413,7.38294) rectangle (13.7811,7.48879);
\draw [color=c, fill=c] (13.7811,7.38294) rectangle (13.8209,7.48879);
\draw [color=c, fill=c] (13.8209,7.38294) rectangle (13.8607,7.48879);
\draw [color=c, fill=c] (13.8607,7.38294) rectangle (13.9005,7.48879);
\draw [color=c, fill=c] (13.9005,7.38294) rectangle (13.9403,7.48879);
\draw [color=c, fill=c] (13.9403,7.38294) rectangle (13.9801,7.48879);
\draw [color=c, fill=c] (13.9801,7.38294) rectangle (14.0199,7.48879);
\draw [color=c, fill=c] (14.0199,7.38294) rectangle (14.0597,7.48879);
\draw [color=c, fill=c] (14.0597,7.38294) rectangle (14.0995,7.48879);
\draw [color=c, fill=c] (14.0995,7.38294) rectangle (14.1393,7.48879);
\draw [color=c, fill=c] (14.1393,7.38294) rectangle (14.1791,7.48879);
\draw [color=c, fill=c] (14.1791,7.38294) rectangle (14.2189,7.48879);
\draw [color=c, fill=c] (14.2189,7.38294) rectangle (14.2587,7.48879);
\draw [color=c, fill=c] (14.2587,7.38294) rectangle (14.2985,7.48879);
\draw [color=c, fill=c] (14.2985,7.38294) rectangle (14.3383,7.48879);
\draw [color=c, fill=c] (14.3383,7.38294) rectangle (14.3781,7.48879);
\draw [color=c, fill=c] (14.3781,7.38294) rectangle (14.4179,7.48879);
\draw [color=c, fill=c] (14.4179,7.38294) rectangle (14.4577,7.48879);
\draw [color=c, fill=c] (14.4577,7.38294) rectangle (14.4975,7.48879);
\draw [color=c, fill=c] (14.4975,7.38294) rectangle (14.5373,7.48879);
\draw [color=c, fill=c] (14.5373,7.38294) rectangle (14.5771,7.48879);
\draw [color=c, fill=c] (14.5771,7.38294) rectangle (14.6169,7.48879);
\draw [color=c, fill=c] (14.6169,7.38294) rectangle (14.6567,7.48879);
\draw [color=c, fill=c] (14.6567,7.38294) rectangle (14.6965,7.48879);
\draw [color=c, fill=c] (14.6965,7.38294) rectangle (14.7363,7.48879);
\draw [color=c, fill=c] (14.7363,7.38294) rectangle (14.7761,7.48879);
\draw [color=c, fill=c] (14.7761,7.38294) rectangle (14.8159,7.48879);
\draw [color=c, fill=c] (14.8159,7.38294) rectangle (14.8557,7.48879);
\draw [color=c, fill=c] (14.8557,7.38294) rectangle (14.8955,7.48879);
\draw [color=c, fill=c] (14.8955,7.38294) rectangle (14.9353,7.48879);
\draw [color=c, fill=c] (14.9353,7.38294) rectangle (14.9751,7.48879);
\draw [color=c, fill=c] (14.9751,7.38294) rectangle (15.0149,7.48879);
\draw [color=c, fill=c] (15.0149,7.38294) rectangle (15.0547,7.48879);
\draw [color=c, fill=c] (15.0547,7.38294) rectangle (15.0945,7.48879);
\draw [color=c, fill=c] (15.0945,7.38294) rectangle (15.1343,7.48879);
\draw [color=c, fill=c] (15.1343,7.38294) rectangle (15.1741,7.48879);
\draw [color=c, fill=c] (15.1741,7.38294) rectangle (15.2139,7.48879);
\draw [color=c, fill=c] (15.2139,7.38294) rectangle (15.2537,7.48879);
\draw [color=c, fill=c] (15.2537,7.38294) rectangle (15.2935,7.48879);
\draw [color=c, fill=c] (15.2935,7.38294) rectangle (15.3333,7.48879);
\draw [color=c, fill=c] (15.3333,7.38294) rectangle (15.3731,7.48879);
\draw [color=c, fill=c] (15.3731,7.38294) rectangle (15.4129,7.48879);
\draw [color=c, fill=c] (15.4129,7.38294) rectangle (15.4527,7.48879);
\draw [color=c, fill=c] (15.4527,7.38294) rectangle (15.4925,7.48879);
\draw [color=c, fill=c] (15.4925,7.38294) rectangle (15.5323,7.48879);
\draw [color=c, fill=c] (15.5323,7.38294) rectangle (15.5721,7.48879);
\draw [color=c, fill=c] (15.5721,7.38294) rectangle (15.6119,7.48879);
\draw [color=c, fill=c] (15.6119,7.38294) rectangle (15.6517,7.48879);
\draw [color=c, fill=c] (15.6517,7.38294) rectangle (15.6915,7.48879);
\draw [color=c, fill=c] (15.6915,7.38294) rectangle (15.7313,7.48879);
\draw [color=c, fill=c] (15.7313,7.38294) rectangle (15.7711,7.48879);
\draw [color=c, fill=c] (15.7711,7.38294) rectangle (15.8109,7.48879);
\draw [color=c, fill=c] (15.8109,7.38294) rectangle (15.8507,7.48879);
\draw [color=c, fill=c] (15.8507,7.38294) rectangle (15.8905,7.48879);
\draw [color=c, fill=c] (15.8905,7.38294) rectangle (15.9303,7.48879);
\draw [color=c, fill=c] (15.9303,7.38294) rectangle (15.9701,7.48879);
\draw [color=c, fill=c] (15.9701,7.38294) rectangle (16.01,7.48879);
\draw [color=c, fill=c] (16.01,7.38294) rectangle (16.0498,7.48879);
\draw [color=c, fill=c] (16.0498,7.38294) rectangle (16.0896,7.48879);
\draw [color=c, fill=c] (16.0896,7.38294) rectangle (16.1294,7.48879);
\draw [color=c, fill=c] (16.1294,7.38294) rectangle (16.1692,7.48879);
\draw [color=c, fill=c] (16.1692,7.38294) rectangle (16.209,7.48879);
\draw [color=c, fill=c] (16.209,7.38294) rectangle (16.2488,7.48879);
\draw [color=c, fill=c] (16.2488,7.38294) rectangle (16.2886,7.48879);
\draw [color=c, fill=c] (16.2886,7.38294) rectangle (16.3284,7.48879);
\draw [color=c, fill=c] (16.3284,7.38294) rectangle (16.3682,7.48879);
\draw [color=c, fill=c] (16.3682,7.38294) rectangle (16.408,7.48879);
\draw [color=c, fill=c] (16.408,7.38294) rectangle (16.4478,7.48879);
\draw [color=c, fill=c] (16.4478,7.38294) rectangle (16.4876,7.48879);
\draw [color=c, fill=c] (16.4876,7.38294) rectangle (16.5274,7.48879);
\draw [color=c, fill=c] (16.5274,7.38294) rectangle (16.5672,7.48879);
\draw [color=c, fill=c] (16.5672,7.38294) rectangle (16.607,7.48879);
\draw [color=c, fill=c] (16.607,7.38294) rectangle (16.6468,7.48879);
\draw [color=c, fill=c] (16.6468,7.38294) rectangle (16.6866,7.48879);
\draw [color=c, fill=c] (16.6866,7.38294) rectangle (16.7264,7.48879);
\draw [color=c, fill=c] (16.7264,7.38294) rectangle (16.7662,7.48879);
\draw [color=c, fill=c] (16.7662,7.38294) rectangle (16.806,7.48879);
\draw [color=c, fill=c] (16.806,7.38294) rectangle (16.8458,7.48879);
\draw [color=c, fill=c] (16.8458,7.38294) rectangle (16.8856,7.48879);
\draw [color=c, fill=c] (16.8856,7.38294) rectangle (16.9254,7.48879);
\draw [color=c, fill=c] (16.9254,7.38294) rectangle (16.9652,7.48879);
\draw [color=c, fill=c] (16.9652,7.38294) rectangle (17.005,7.48879);
\draw [color=c, fill=c] (17.005,7.38294) rectangle (17.0448,7.48879);
\draw [color=c, fill=c] (17.0448,7.38294) rectangle (17.0846,7.48879);
\draw [color=c, fill=c] (17.0846,7.38294) rectangle (17.1244,7.48879);
\draw [color=c, fill=c] (17.1244,7.38294) rectangle (17.1642,7.48879);
\draw [color=c, fill=c] (17.1642,7.38294) rectangle (17.204,7.48879);
\draw [color=c, fill=c] (17.204,7.38294) rectangle (17.2438,7.48879);
\draw [color=c, fill=c] (17.2438,7.38294) rectangle (17.2836,7.48879);
\draw [color=c, fill=c] (17.2836,7.38294) rectangle (17.3234,7.48879);
\draw [color=c, fill=c] (17.3234,7.38294) rectangle (17.3632,7.48879);
\draw [color=c, fill=c] (17.3632,7.38294) rectangle (17.403,7.48879);
\draw [color=c, fill=c] (17.403,7.38294) rectangle (17.4428,7.48879);
\draw [color=c, fill=c] (17.4428,7.38294) rectangle (17.4826,7.48879);
\draw [color=c, fill=c] (17.4826,7.38294) rectangle (17.5224,7.48879);
\draw [color=c, fill=c] (17.5224,7.38294) rectangle (17.5622,7.48879);
\draw [color=c, fill=c] (17.5622,7.38294) rectangle (17.602,7.48879);
\draw [color=c, fill=c] (17.602,7.38294) rectangle (17.6418,7.48879);
\draw [color=c, fill=c] (17.6418,7.38294) rectangle (17.6816,7.48879);
\draw [color=c, fill=c] (17.6816,7.38294) rectangle (17.7214,7.48879);
\draw [color=c, fill=c] (17.7214,7.38294) rectangle (17.7612,7.48879);
\draw [color=c, fill=c] (17.7612,7.38294) rectangle (17.801,7.48879);
\draw [color=c, fill=c] (17.801,7.38294) rectangle (17.8408,7.48879);
\draw [color=c, fill=c] (17.8408,7.38294) rectangle (17.8806,7.48879);
\draw [color=c, fill=c] (17.8806,7.38294) rectangle (17.9204,7.48879);
\draw [color=c, fill=c] (17.9204,7.38294) rectangle (17.9602,7.48879);
\draw [color=c, fill=c] (17.9602,7.38294) rectangle (18,7.48879);
\definecolor{c}{rgb}{0.2,0,1};
\draw [color=c, fill=c] (2,7.48879) rectangle (2.0398,7.59464);
\draw [color=c, fill=c] (2.0398,7.48879) rectangle (2.0796,7.59464);
\draw [color=c, fill=c] (2.0796,7.48879) rectangle (2.1194,7.59464);
\draw [color=c, fill=c] (2.1194,7.48879) rectangle (2.1592,7.59464);
\draw [color=c, fill=c] (2.1592,7.48879) rectangle (2.19901,7.59464);
\draw [color=c, fill=c] (2.19901,7.48879) rectangle (2.23881,7.59464);
\draw [color=c, fill=c] (2.23881,7.48879) rectangle (2.27861,7.59464);
\draw [color=c, fill=c] (2.27861,7.48879) rectangle (2.31841,7.59464);
\draw [color=c, fill=c] (2.31841,7.48879) rectangle (2.35821,7.59464);
\draw [color=c, fill=c] (2.35821,7.48879) rectangle (2.39801,7.59464);
\draw [color=c, fill=c] (2.39801,7.48879) rectangle (2.43781,7.59464);
\draw [color=c, fill=c] (2.43781,7.48879) rectangle (2.47761,7.59464);
\draw [color=c, fill=c] (2.47761,7.48879) rectangle (2.51741,7.59464);
\draw [color=c, fill=c] (2.51741,7.48879) rectangle (2.55721,7.59464);
\draw [color=c, fill=c] (2.55721,7.48879) rectangle (2.59702,7.59464);
\draw [color=c, fill=c] (2.59702,7.48879) rectangle (2.63682,7.59464);
\draw [color=c, fill=c] (2.63682,7.48879) rectangle (2.67662,7.59464);
\draw [color=c, fill=c] (2.67662,7.48879) rectangle (2.71642,7.59464);
\draw [color=c, fill=c] (2.71642,7.48879) rectangle (2.75622,7.59464);
\draw [color=c, fill=c] (2.75622,7.48879) rectangle (2.79602,7.59464);
\draw [color=c, fill=c] (2.79602,7.48879) rectangle (2.83582,7.59464);
\draw [color=c, fill=c] (2.83582,7.48879) rectangle (2.87562,7.59464);
\draw [color=c, fill=c] (2.87562,7.48879) rectangle (2.91542,7.59464);
\draw [color=c, fill=c] (2.91542,7.48879) rectangle (2.95522,7.59464);
\draw [color=c, fill=c] (2.95522,7.48879) rectangle (2.99502,7.59464);
\draw [color=c, fill=c] (2.99502,7.48879) rectangle (3.03483,7.59464);
\draw [color=c, fill=c] (3.03483,7.48879) rectangle (3.07463,7.59464);
\draw [color=c, fill=c] (3.07463,7.48879) rectangle (3.11443,7.59464);
\draw [color=c, fill=c] (3.11443,7.48879) rectangle (3.15423,7.59464);
\draw [color=c, fill=c] (3.15423,7.48879) rectangle (3.19403,7.59464);
\draw [color=c, fill=c] (3.19403,7.48879) rectangle (3.23383,7.59464);
\draw [color=c, fill=c] (3.23383,7.48879) rectangle (3.27363,7.59464);
\draw [color=c, fill=c] (3.27363,7.48879) rectangle (3.31343,7.59464);
\draw [color=c, fill=c] (3.31343,7.48879) rectangle (3.35323,7.59464);
\draw [color=c, fill=c] (3.35323,7.48879) rectangle (3.39303,7.59464);
\draw [color=c, fill=c] (3.39303,7.48879) rectangle (3.43284,7.59464);
\draw [color=c, fill=c] (3.43284,7.48879) rectangle (3.47264,7.59464);
\draw [color=c, fill=c] (3.47264,7.48879) rectangle (3.51244,7.59464);
\draw [color=c, fill=c] (3.51244,7.48879) rectangle (3.55224,7.59464);
\draw [color=c, fill=c] (3.55224,7.48879) rectangle (3.59204,7.59464);
\draw [color=c, fill=c] (3.59204,7.48879) rectangle (3.63184,7.59464);
\draw [color=c, fill=c] (3.63184,7.48879) rectangle (3.67164,7.59464);
\draw [color=c, fill=c] (3.67164,7.48879) rectangle (3.71144,7.59464);
\draw [color=c, fill=c] (3.71144,7.48879) rectangle (3.75124,7.59464);
\draw [color=c, fill=c] (3.75124,7.48879) rectangle (3.79104,7.59464);
\draw [color=c, fill=c] (3.79104,7.48879) rectangle (3.83085,7.59464);
\draw [color=c, fill=c] (3.83085,7.48879) rectangle (3.87065,7.59464);
\draw [color=c, fill=c] (3.87065,7.48879) rectangle (3.91045,7.59464);
\draw [color=c, fill=c] (3.91045,7.48879) rectangle (3.95025,7.59464);
\draw [color=c, fill=c] (3.95025,7.48879) rectangle (3.99005,7.59464);
\draw [color=c, fill=c] (3.99005,7.48879) rectangle (4.02985,7.59464);
\draw [color=c, fill=c] (4.02985,7.48879) rectangle (4.06965,7.59464);
\draw [color=c, fill=c] (4.06965,7.48879) rectangle (4.10945,7.59464);
\draw [color=c, fill=c] (4.10945,7.48879) rectangle (4.14925,7.59464);
\draw [color=c, fill=c] (4.14925,7.48879) rectangle (4.18905,7.59464);
\draw [color=c, fill=c] (4.18905,7.48879) rectangle (4.22886,7.59464);
\draw [color=c, fill=c] (4.22886,7.48879) rectangle (4.26866,7.59464);
\draw [color=c, fill=c] (4.26866,7.48879) rectangle (4.30846,7.59464);
\draw [color=c, fill=c] (4.30846,7.48879) rectangle (4.34826,7.59464);
\draw [color=c, fill=c] (4.34826,7.48879) rectangle (4.38806,7.59464);
\draw [color=c, fill=c] (4.38806,7.48879) rectangle (4.42786,7.59464);
\draw [color=c, fill=c] (4.42786,7.48879) rectangle (4.46766,7.59464);
\draw [color=c, fill=c] (4.46766,7.48879) rectangle (4.50746,7.59464);
\draw [color=c, fill=c] (4.50746,7.48879) rectangle (4.54726,7.59464);
\draw [color=c, fill=c] (4.54726,7.48879) rectangle (4.58706,7.59464);
\draw [color=c, fill=c] (4.58706,7.48879) rectangle (4.62687,7.59464);
\draw [color=c, fill=c] (4.62687,7.48879) rectangle (4.66667,7.59464);
\draw [color=c, fill=c] (4.66667,7.48879) rectangle (4.70647,7.59464);
\draw [color=c, fill=c] (4.70647,7.48879) rectangle (4.74627,7.59464);
\draw [color=c, fill=c] (4.74627,7.48879) rectangle (4.78607,7.59464);
\draw [color=c, fill=c] (4.78607,7.48879) rectangle (4.82587,7.59464);
\draw [color=c, fill=c] (4.82587,7.48879) rectangle (4.86567,7.59464);
\draw [color=c, fill=c] (4.86567,7.48879) rectangle (4.90547,7.59464);
\draw [color=c, fill=c] (4.90547,7.48879) rectangle (4.94527,7.59464);
\draw [color=c, fill=c] (4.94527,7.48879) rectangle (4.98507,7.59464);
\draw [color=c, fill=c] (4.98507,7.48879) rectangle (5.02488,7.59464);
\draw [color=c, fill=c] (5.02488,7.48879) rectangle (5.06468,7.59464);
\draw [color=c, fill=c] (5.06468,7.48879) rectangle (5.10448,7.59464);
\draw [color=c, fill=c] (5.10448,7.48879) rectangle (5.14428,7.59464);
\draw [color=c, fill=c] (5.14428,7.48879) rectangle (5.18408,7.59464);
\draw [color=c, fill=c] (5.18408,7.48879) rectangle (5.22388,7.59464);
\draw [color=c, fill=c] (5.22388,7.48879) rectangle (5.26368,7.59464);
\draw [color=c, fill=c] (5.26368,7.48879) rectangle (5.30348,7.59464);
\draw [color=c, fill=c] (5.30348,7.48879) rectangle (5.34328,7.59464);
\draw [color=c, fill=c] (5.34328,7.48879) rectangle (5.38308,7.59464);
\draw [color=c, fill=c] (5.38308,7.48879) rectangle (5.42289,7.59464);
\draw [color=c, fill=c] (5.42289,7.48879) rectangle (5.46269,7.59464);
\draw [color=c, fill=c] (5.46269,7.48879) rectangle (5.50249,7.59464);
\draw [color=c, fill=c] (5.50249,7.48879) rectangle (5.54229,7.59464);
\draw [color=c, fill=c] (5.54229,7.48879) rectangle (5.58209,7.59464);
\draw [color=c, fill=c] (5.58209,7.48879) rectangle (5.62189,7.59464);
\draw [color=c, fill=c] (5.62189,7.48879) rectangle (5.66169,7.59464);
\draw [color=c, fill=c] (5.66169,7.48879) rectangle (5.70149,7.59464);
\draw [color=c, fill=c] (5.70149,7.48879) rectangle (5.74129,7.59464);
\draw [color=c, fill=c] (5.74129,7.48879) rectangle (5.78109,7.59464);
\draw [color=c, fill=c] (5.78109,7.48879) rectangle (5.8209,7.59464);
\draw [color=c, fill=c] (5.8209,7.48879) rectangle (5.8607,7.59464);
\draw [color=c, fill=c] (5.8607,7.48879) rectangle (5.9005,7.59464);
\draw [color=c, fill=c] (5.9005,7.48879) rectangle (5.9403,7.59464);
\draw [color=c, fill=c] (5.9403,7.48879) rectangle (5.9801,7.59464);
\draw [color=c, fill=c] (5.9801,7.48879) rectangle (6.0199,7.59464);
\draw [color=c, fill=c] (6.0199,7.48879) rectangle (6.0597,7.59464);
\draw [color=c, fill=c] (6.0597,7.48879) rectangle (6.0995,7.59464);
\draw [color=c, fill=c] (6.0995,7.48879) rectangle (6.1393,7.59464);
\draw [color=c, fill=c] (6.1393,7.48879) rectangle (6.1791,7.59464);
\draw [color=c, fill=c] (6.1791,7.48879) rectangle (6.21891,7.59464);
\draw [color=c, fill=c] (6.21891,7.48879) rectangle (6.25871,7.59464);
\draw [color=c, fill=c] (6.25871,7.48879) rectangle (6.29851,7.59464);
\draw [color=c, fill=c] (6.29851,7.48879) rectangle (6.33831,7.59464);
\draw [color=c, fill=c] (6.33831,7.48879) rectangle (6.37811,7.59464);
\draw [color=c, fill=c] (6.37811,7.48879) rectangle (6.41791,7.59464);
\draw [color=c, fill=c] (6.41791,7.48879) rectangle (6.45771,7.59464);
\draw [color=c, fill=c] (6.45771,7.48879) rectangle (6.49751,7.59464);
\draw [color=c, fill=c] (6.49751,7.48879) rectangle (6.53731,7.59464);
\draw [color=c, fill=c] (6.53731,7.48879) rectangle (6.57711,7.59464);
\draw [color=c, fill=c] (6.57711,7.48879) rectangle (6.61692,7.59464);
\draw [color=c, fill=c] (6.61692,7.48879) rectangle (6.65672,7.59464);
\draw [color=c, fill=c] (6.65672,7.48879) rectangle (6.69652,7.59464);
\draw [color=c, fill=c] (6.69652,7.48879) rectangle (6.73632,7.59464);
\draw [color=c, fill=c] (6.73632,7.48879) rectangle (6.77612,7.59464);
\draw [color=c, fill=c] (6.77612,7.48879) rectangle (6.81592,7.59464);
\draw [color=c, fill=c] (6.81592,7.48879) rectangle (6.85572,7.59464);
\draw [color=c, fill=c] (6.85572,7.48879) rectangle (6.89552,7.59464);
\draw [color=c, fill=c] (6.89552,7.48879) rectangle (6.93532,7.59464);
\draw [color=c, fill=c] (6.93532,7.48879) rectangle (6.97512,7.59464);
\draw [color=c, fill=c] (6.97512,7.48879) rectangle (7.01493,7.59464);
\draw [color=c, fill=c] (7.01493,7.48879) rectangle (7.05473,7.59464);
\draw [color=c, fill=c] (7.05473,7.48879) rectangle (7.09453,7.59464);
\draw [color=c, fill=c] (7.09453,7.48879) rectangle (7.13433,7.59464);
\draw [color=c, fill=c] (7.13433,7.48879) rectangle (7.17413,7.59464);
\draw [color=c, fill=c] (7.17413,7.48879) rectangle (7.21393,7.59464);
\draw [color=c, fill=c] (7.21393,7.48879) rectangle (7.25373,7.59464);
\draw [color=c, fill=c] (7.25373,7.48879) rectangle (7.29353,7.59464);
\draw [color=c, fill=c] (7.29353,7.48879) rectangle (7.33333,7.59464);
\draw [color=c, fill=c] (7.33333,7.48879) rectangle (7.37313,7.59464);
\draw [color=c, fill=c] (7.37313,7.48879) rectangle (7.41294,7.59464);
\draw [color=c, fill=c] (7.41294,7.48879) rectangle (7.45274,7.59464);
\draw [color=c, fill=c] (7.45274,7.48879) rectangle (7.49254,7.59464);
\draw [color=c, fill=c] (7.49254,7.48879) rectangle (7.53234,7.59464);
\draw [color=c, fill=c] (7.53234,7.48879) rectangle (7.57214,7.59464);
\draw [color=c, fill=c] (7.57214,7.48879) rectangle (7.61194,7.59464);
\draw [color=c, fill=c] (7.61194,7.48879) rectangle (7.65174,7.59464);
\draw [color=c, fill=c] (7.65174,7.48879) rectangle (7.69154,7.59464);
\draw [color=c, fill=c] (7.69154,7.48879) rectangle (7.73134,7.59464);
\draw [color=c, fill=c] (7.73134,7.48879) rectangle (7.77114,7.59464);
\draw [color=c, fill=c] (7.77114,7.48879) rectangle (7.81095,7.59464);
\draw [color=c, fill=c] (7.81095,7.48879) rectangle (7.85075,7.59464);
\draw [color=c, fill=c] (7.85075,7.48879) rectangle (7.89055,7.59464);
\definecolor{c}{rgb}{0,0.0800001,1};
\draw [color=c, fill=c] (7.89055,7.48879) rectangle (7.93035,7.59464);
\draw [color=c, fill=c] (7.93035,7.48879) rectangle (7.97015,7.59464);
\draw [color=c, fill=c] (7.97015,7.48879) rectangle (8.00995,7.59464);
\draw [color=c, fill=c] (8.00995,7.48879) rectangle (8.04975,7.59464);
\draw [color=c, fill=c] (8.04975,7.48879) rectangle (8.08955,7.59464);
\draw [color=c, fill=c] (8.08955,7.48879) rectangle (8.12935,7.59464);
\draw [color=c, fill=c] (8.12935,7.48879) rectangle (8.16915,7.59464);
\draw [color=c, fill=c] (8.16915,7.48879) rectangle (8.20895,7.59464);
\draw [color=c, fill=c] (8.20895,7.48879) rectangle (8.24876,7.59464);
\draw [color=c, fill=c] (8.24876,7.48879) rectangle (8.28856,7.59464);
\draw [color=c, fill=c] (8.28856,7.48879) rectangle (8.32836,7.59464);
\draw [color=c, fill=c] (8.32836,7.48879) rectangle (8.36816,7.59464);
\draw [color=c, fill=c] (8.36816,7.48879) rectangle (8.40796,7.59464);
\draw [color=c, fill=c] (8.40796,7.48879) rectangle (8.44776,7.59464);
\draw [color=c, fill=c] (8.44776,7.48879) rectangle (8.48756,7.59464);
\draw [color=c, fill=c] (8.48756,7.48879) rectangle (8.52736,7.59464);
\draw [color=c, fill=c] (8.52736,7.48879) rectangle (8.56716,7.59464);
\draw [color=c, fill=c] (8.56716,7.48879) rectangle (8.60697,7.59464);
\draw [color=c, fill=c] (8.60697,7.48879) rectangle (8.64677,7.59464);
\draw [color=c, fill=c] (8.64677,7.48879) rectangle (8.68657,7.59464);
\draw [color=c, fill=c] (8.68657,7.48879) rectangle (8.72637,7.59464);
\draw [color=c, fill=c] (8.72637,7.48879) rectangle (8.76617,7.59464);
\draw [color=c, fill=c] (8.76617,7.48879) rectangle (8.80597,7.59464);
\draw [color=c, fill=c] (8.80597,7.48879) rectangle (8.84577,7.59464);
\draw [color=c, fill=c] (8.84577,7.48879) rectangle (8.88557,7.59464);
\draw [color=c, fill=c] (8.88557,7.48879) rectangle (8.92537,7.59464);
\draw [color=c, fill=c] (8.92537,7.48879) rectangle (8.96517,7.59464);
\draw [color=c, fill=c] (8.96517,7.48879) rectangle (9.00498,7.59464);
\draw [color=c, fill=c] (9.00498,7.48879) rectangle (9.04478,7.59464);
\draw [color=c, fill=c] (9.04478,7.48879) rectangle (9.08458,7.59464);
\draw [color=c, fill=c] (9.08458,7.48879) rectangle (9.12438,7.59464);
\draw [color=c, fill=c] (9.12438,7.48879) rectangle (9.16418,7.59464);
\draw [color=c, fill=c] (9.16418,7.48879) rectangle (9.20398,7.59464);
\draw [color=c, fill=c] (9.20398,7.48879) rectangle (9.24378,7.59464);
\draw [color=c, fill=c] (9.24378,7.48879) rectangle (9.28358,7.59464);
\draw [color=c, fill=c] (9.28358,7.48879) rectangle (9.32338,7.59464);
\draw [color=c, fill=c] (9.32338,7.48879) rectangle (9.36318,7.59464);
\draw [color=c, fill=c] (9.36318,7.48879) rectangle (9.40298,7.59464);
\draw [color=c, fill=c] (9.40298,7.48879) rectangle (9.44279,7.59464);
\draw [color=c, fill=c] (9.44279,7.48879) rectangle (9.48259,7.59464);
\draw [color=c, fill=c] (9.48259,7.48879) rectangle (9.52239,7.59464);
\draw [color=c, fill=c] (9.52239,7.48879) rectangle (9.56219,7.59464);
\definecolor{c}{rgb}{0,0.266667,1};
\draw [color=c, fill=c] (9.56219,7.48879) rectangle (9.60199,7.59464);
\draw [color=c, fill=c] (9.60199,7.48879) rectangle (9.64179,7.59464);
\draw [color=c, fill=c] (9.64179,7.48879) rectangle (9.68159,7.59464);
\draw [color=c, fill=c] (9.68159,7.48879) rectangle (9.72139,7.59464);
\draw [color=c, fill=c] (9.72139,7.48879) rectangle (9.76119,7.59464);
\draw [color=c, fill=c] (9.76119,7.48879) rectangle (9.80099,7.59464);
\draw [color=c, fill=c] (9.80099,7.48879) rectangle (9.8408,7.59464);
\draw [color=c, fill=c] (9.8408,7.48879) rectangle (9.8806,7.59464);
\draw [color=c, fill=c] (9.8806,7.48879) rectangle (9.9204,7.59464);
\draw [color=c, fill=c] (9.9204,7.48879) rectangle (9.9602,7.59464);
\draw [color=c, fill=c] (9.9602,7.48879) rectangle (10,7.59464);
\draw [color=c, fill=c] (10,7.48879) rectangle (10.0398,7.59464);
\draw [color=c, fill=c] (10.0398,7.48879) rectangle (10.0796,7.59464);
\draw [color=c, fill=c] (10.0796,7.48879) rectangle (10.1194,7.59464);
\draw [color=c, fill=c] (10.1194,7.48879) rectangle (10.1592,7.59464);
\draw [color=c, fill=c] (10.1592,7.48879) rectangle (10.199,7.59464);
\draw [color=c, fill=c] (10.199,7.48879) rectangle (10.2388,7.59464);
\draw [color=c, fill=c] (10.2388,7.48879) rectangle (10.2786,7.59464);
\draw [color=c, fill=c] (10.2786,7.48879) rectangle (10.3184,7.59464);
\definecolor{c}{rgb}{0,0.546666,1};
\draw [color=c, fill=c] (10.3184,7.48879) rectangle (10.3582,7.59464);
\draw [color=c, fill=c] (10.3582,7.48879) rectangle (10.398,7.59464);
\draw [color=c, fill=c] (10.398,7.48879) rectangle (10.4378,7.59464);
\draw [color=c, fill=c] (10.4378,7.48879) rectangle (10.4776,7.59464);
\draw [color=c, fill=c] (10.4776,7.48879) rectangle (10.5174,7.59464);
\draw [color=c, fill=c] (10.5174,7.48879) rectangle (10.5572,7.59464);
\draw [color=c, fill=c] (10.5572,7.48879) rectangle (10.597,7.59464);
\draw [color=c, fill=c] (10.597,7.48879) rectangle (10.6368,7.59464);
\draw [color=c, fill=c] (10.6368,7.48879) rectangle (10.6766,7.59464);
\draw [color=c, fill=c] (10.6766,7.48879) rectangle (10.7164,7.59464);
\draw [color=c, fill=c] (10.7164,7.48879) rectangle (10.7562,7.59464);
\draw [color=c, fill=c] (10.7562,7.48879) rectangle (10.796,7.59464);
\draw [color=c, fill=c] (10.796,7.48879) rectangle (10.8358,7.59464);
\draw [color=c, fill=c] (10.8358,7.48879) rectangle (10.8756,7.59464);
\draw [color=c, fill=c] (10.8756,7.48879) rectangle (10.9154,7.59464);
\draw [color=c, fill=c] (10.9154,7.48879) rectangle (10.9552,7.59464);
\draw [color=c, fill=c] (10.9552,7.48879) rectangle (10.995,7.59464);
\draw [color=c, fill=c] (10.995,7.48879) rectangle (11.0348,7.59464);
\draw [color=c, fill=c] (11.0348,7.48879) rectangle (11.0746,7.59464);
\draw [color=c, fill=c] (11.0746,7.48879) rectangle (11.1144,7.59464);
\draw [color=c, fill=c] (11.1144,7.48879) rectangle (11.1542,7.59464);
\draw [color=c, fill=c] (11.1542,7.48879) rectangle (11.194,7.59464);
\draw [color=c, fill=c] (11.194,7.48879) rectangle (11.2338,7.59464);
\draw [color=c, fill=c] (11.2338,7.48879) rectangle (11.2736,7.59464);
\draw [color=c, fill=c] (11.2736,7.48879) rectangle (11.3134,7.59464);
\draw [color=c, fill=c] (11.3134,7.48879) rectangle (11.3532,7.59464);
\draw [color=c, fill=c] (11.3532,7.48879) rectangle (11.393,7.59464);
\draw [color=c, fill=c] (11.393,7.48879) rectangle (11.4328,7.59464);
\draw [color=c, fill=c] (11.4328,7.48879) rectangle (11.4726,7.59464);
\draw [color=c, fill=c] (11.4726,7.48879) rectangle (11.5124,7.59464);
\draw [color=c, fill=c] (11.5124,7.48879) rectangle (11.5522,7.59464);
\draw [color=c, fill=c] (11.5522,7.48879) rectangle (11.592,7.59464);
\draw [color=c, fill=c] (11.592,7.48879) rectangle (11.6318,7.59464);
\draw [color=c, fill=c] (11.6318,7.48879) rectangle (11.6716,7.59464);
\draw [color=c, fill=c] (11.6716,7.48879) rectangle (11.7114,7.59464);
\draw [color=c, fill=c] (11.7114,7.48879) rectangle (11.7512,7.59464);
\draw [color=c, fill=c] (11.7512,7.48879) rectangle (11.791,7.59464);
\definecolor{c}{rgb}{0,0.733333,1};
\draw [color=c, fill=c] (11.791,7.48879) rectangle (11.8308,7.59464);
\draw [color=c, fill=c] (11.8308,7.48879) rectangle (11.8706,7.59464);
\draw [color=c, fill=c] (11.8706,7.48879) rectangle (11.9104,7.59464);
\draw [color=c, fill=c] (11.9104,7.48879) rectangle (11.9502,7.59464);
\draw [color=c, fill=c] (11.9502,7.48879) rectangle (11.99,7.59464);
\draw [color=c, fill=c] (11.99,7.48879) rectangle (12.0299,7.59464);
\draw [color=c, fill=c] (12.0299,7.48879) rectangle (12.0697,7.59464);
\draw [color=c, fill=c] (12.0697,7.48879) rectangle (12.1095,7.59464);
\draw [color=c, fill=c] (12.1095,7.48879) rectangle (12.1493,7.59464);
\draw [color=c, fill=c] (12.1493,7.48879) rectangle (12.1891,7.59464);
\draw [color=c, fill=c] (12.1891,7.48879) rectangle (12.2289,7.59464);
\draw [color=c, fill=c] (12.2289,7.48879) rectangle (12.2687,7.59464);
\draw [color=c, fill=c] (12.2687,7.48879) rectangle (12.3085,7.59464);
\draw [color=c, fill=c] (12.3085,7.48879) rectangle (12.3483,7.59464);
\draw [color=c, fill=c] (12.3483,7.48879) rectangle (12.3881,7.59464);
\draw [color=c, fill=c] (12.3881,7.48879) rectangle (12.4279,7.59464);
\draw [color=c, fill=c] (12.4279,7.48879) rectangle (12.4677,7.59464);
\draw [color=c, fill=c] (12.4677,7.48879) rectangle (12.5075,7.59464);
\draw [color=c, fill=c] (12.5075,7.48879) rectangle (12.5473,7.59464);
\draw [color=c, fill=c] (12.5473,7.48879) rectangle (12.5871,7.59464);
\draw [color=c, fill=c] (12.5871,7.48879) rectangle (12.6269,7.59464);
\draw [color=c, fill=c] (12.6269,7.48879) rectangle (12.6667,7.59464);
\draw [color=c, fill=c] (12.6667,7.48879) rectangle (12.7065,7.59464);
\draw [color=c, fill=c] (12.7065,7.48879) rectangle (12.7463,7.59464);
\draw [color=c, fill=c] (12.7463,7.48879) rectangle (12.7861,7.59464);
\draw [color=c, fill=c] (12.7861,7.48879) rectangle (12.8259,7.59464);
\draw [color=c, fill=c] (12.8259,7.48879) rectangle (12.8657,7.59464);
\draw [color=c, fill=c] (12.8657,7.48879) rectangle (12.9055,7.59464);
\draw [color=c, fill=c] (12.9055,7.48879) rectangle (12.9453,7.59464);
\draw [color=c, fill=c] (12.9453,7.48879) rectangle (12.9851,7.59464);
\draw [color=c, fill=c] (12.9851,7.48879) rectangle (13.0249,7.59464);
\draw [color=c, fill=c] (13.0249,7.48879) rectangle (13.0647,7.59464);
\draw [color=c, fill=c] (13.0647,7.48879) rectangle (13.1045,7.59464);
\draw [color=c, fill=c] (13.1045,7.48879) rectangle (13.1443,7.59464);
\draw [color=c, fill=c] (13.1443,7.48879) rectangle (13.1841,7.59464);
\draw [color=c, fill=c] (13.1841,7.48879) rectangle (13.2239,7.59464);
\draw [color=c, fill=c] (13.2239,7.48879) rectangle (13.2637,7.59464);
\draw [color=c, fill=c] (13.2637,7.48879) rectangle (13.3035,7.59464);
\draw [color=c, fill=c] (13.3035,7.48879) rectangle (13.3433,7.59464);
\draw [color=c, fill=c] (13.3433,7.48879) rectangle (13.3831,7.59464);
\draw [color=c, fill=c] (13.3831,7.48879) rectangle (13.4229,7.59464);
\draw [color=c, fill=c] (13.4229,7.48879) rectangle (13.4627,7.59464);
\draw [color=c, fill=c] (13.4627,7.48879) rectangle (13.5025,7.59464);
\draw [color=c, fill=c] (13.5025,7.48879) rectangle (13.5423,7.59464);
\draw [color=c, fill=c] (13.5423,7.48879) rectangle (13.5821,7.59464);
\draw [color=c, fill=c] (13.5821,7.48879) rectangle (13.6219,7.59464);
\draw [color=c, fill=c] (13.6219,7.48879) rectangle (13.6617,7.59464);
\draw [color=c, fill=c] (13.6617,7.48879) rectangle (13.7015,7.59464);
\draw [color=c, fill=c] (13.7015,7.48879) rectangle (13.7413,7.59464);
\draw [color=c, fill=c] (13.7413,7.48879) rectangle (13.7811,7.59464);
\draw [color=c, fill=c] (13.7811,7.48879) rectangle (13.8209,7.59464);
\draw [color=c, fill=c] (13.8209,7.48879) rectangle (13.8607,7.59464);
\draw [color=c, fill=c] (13.8607,7.48879) rectangle (13.9005,7.59464);
\draw [color=c, fill=c] (13.9005,7.48879) rectangle (13.9403,7.59464);
\draw [color=c, fill=c] (13.9403,7.48879) rectangle (13.9801,7.59464);
\draw [color=c, fill=c] (13.9801,7.48879) rectangle (14.0199,7.59464);
\draw [color=c, fill=c] (14.0199,7.48879) rectangle (14.0597,7.59464);
\draw [color=c, fill=c] (14.0597,7.48879) rectangle (14.0995,7.59464);
\draw [color=c, fill=c] (14.0995,7.48879) rectangle (14.1393,7.59464);
\draw [color=c, fill=c] (14.1393,7.48879) rectangle (14.1791,7.59464);
\draw [color=c, fill=c] (14.1791,7.48879) rectangle (14.2189,7.59464);
\draw [color=c, fill=c] (14.2189,7.48879) rectangle (14.2587,7.59464);
\draw [color=c, fill=c] (14.2587,7.48879) rectangle (14.2985,7.59464);
\draw [color=c, fill=c] (14.2985,7.48879) rectangle (14.3383,7.59464);
\draw [color=c, fill=c] (14.3383,7.48879) rectangle (14.3781,7.59464);
\draw [color=c, fill=c] (14.3781,7.48879) rectangle (14.4179,7.59464);
\draw [color=c, fill=c] (14.4179,7.48879) rectangle (14.4577,7.59464);
\draw [color=c, fill=c] (14.4577,7.48879) rectangle (14.4975,7.59464);
\draw [color=c, fill=c] (14.4975,7.48879) rectangle (14.5373,7.59464);
\draw [color=c, fill=c] (14.5373,7.48879) rectangle (14.5771,7.59464);
\draw [color=c, fill=c] (14.5771,7.48879) rectangle (14.6169,7.59464);
\draw [color=c, fill=c] (14.6169,7.48879) rectangle (14.6567,7.59464);
\draw [color=c, fill=c] (14.6567,7.48879) rectangle (14.6965,7.59464);
\draw [color=c, fill=c] (14.6965,7.48879) rectangle (14.7363,7.59464);
\draw [color=c, fill=c] (14.7363,7.48879) rectangle (14.7761,7.59464);
\draw [color=c, fill=c] (14.7761,7.48879) rectangle (14.8159,7.59464);
\draw [color=c, fill=c] (14.8159,7.48879) rectangle (14.8557,7.59464);
\draw [color=c, fill=c] (14.8557,7.48879) rectangle (14.8955,7.59464);
\draw [color=c, fill=c] (14.8955,7.48879) rectangle (14.9353,7.59464);
\draw [color=c, fill=c] (14.9353,7.48879) rectangle (14.9751,7.59464);
\draw [color=c, fill=c] (14.9751,7.48879) rectangle (15.0149,7.59464);
\draw [color=c, fill=c] (15.0149,7.48879) rectangle (15.0547,7.59464);
\draw [color=c, fill=c] (15.0547,7.48879) rectangle (15.0945,7.59464);
\draw [color=c, fill=c] (15.0945,7.48879) rectangle (15.1343,7.59464);
\draw [color=c, fill=c] (15.1343,7.48879) rectangle (15.1741,7.59464);
\draw [color=c, fill=c] (15.1741,7.48879) rectangle (15.2139,7.59464);
\draw [color=c, fill=c] (15.2139,7.48879) rectangle (15.2537,7.59464);
\draw [color=c, fill=c] (15.2537,7.48879) rectangle (15.2935,7.59464);
\draw [color=c, fill=c] (15.2935,7.48879) rectangle (15.3333,7.59464);
\draw [color=c, fill=c] (15.3333,7.48879) rectangle (15.3731,7.59464);
\draw [color=c, fill=c] (15.3731,7.48879) rectangle (15.4129,7.59464);
\draw [color=c, fill=c] (15.4129,7.48879) rectangle (15.4527,7.59464);
\draw [color=c, fill=c] (15.4527,7.48879) rectangle (15.4925,7.59464);
\draw [color=c, fill=c] (15.4925,7.48879) rectangle (15.5323,7.59464);
\draw [color=c, fill=c] (15.5323,7.48879) rectangle (15.5721,7.59464);
\draw [color=c, fill=c] (15.5721,7.48879) rectangle (15.6119,7.59464);
\draw [color=c, fill=c] (15.6119,7.48879) rectangle (15.6517,7.59464);
\draw [color=c, fill=c] (15.6517,7.48879) rectangle (15.6915,7.59464);
\draw [color=c, fill=c] (15.6915,7.48879) rectangle (15.7313,7.59464);
\draw [color=c, fill=c] (15.7313,7.48879) rectangle (15.7711,7.59464);
\draw [color=c, fill=c] (15.7711,7.48879) rectangle (15.8109,7.59464);
\draw [color=c, fill=c] (15.8109,7.48879) rectangle (15.8507,7.59464);
\draw [color=c, fill=c] (15.8507,7.48879) rectangle (15.8905,7.59464);
\draw [color=c, fill=c] (15.8905,7.48879) rectangle (15.9303,7.59464);
\draw [color=c, fill=c] (15.9303,7.48879) rectangle (15.9701,7.59464);
\draw [color=c, fill=c] (15.9701,7.48879) rectangle (16.01,7.59464);
\draw [color=c, fill=c] (16.01,7.48879) rectangle (16.0498,7.59464);
\draw [color=c, fill=c] (16.0498,7.48879) rectangle (16.0896,7.59464);
\draw [color=c, fill=c] (16.0896,7.48879) rectangle (16.1294,7.59464);
\draw [color=c, fill=c] (16.1294,7.48879) rectangle (16.1692,7.59464);
\draw [color=c, fill=c] (16.1692,7.48879) rectangle (16.209,7.59464);
\draw [color=c, fill=c] (16.209,7.48879) rectangle (16.2488,7.59464);
\draw [color=c, fill=c] (16.2488,7.48879) rectangle (16.2886,7.59464);
\draw [color=c, fill=c] (16.2886,7.48879) rectangle (16.3284,7.59464);
\draw [color=c, fill=c] (16.3284,7.48879) rectangle (16.3682,7.59464);
\draw [color=c, fill=c] (16.3682,7.48879) rectangle (16.408,7.59464);
\draw [color=c, fill=c] (16.408,7.48879) rectangle (16.4478,7.59464);
\draw [color=c, fill=c] (16.4478,7.48879) rectangle (16.4876,7.59464);
\draw [color=c, fill=c] (16.4876,7.48879) rectangle (16.5274,7.59464);
\draw [color=c, fill=c] (16.5274,7.48879) rectangle (16.5672,7.59464);
\draw [color=c, fill=c] (16.5672,7.48879) rectangle (16.607,7.59464);
\draw [color=c, fill=c] (16.607,7.48879) rectangle (16.6468,7.59464);
\draw [color=c, fill=c] (16.6468,7.48879) rectangle (16.6866,7.59464);
\draw [color=c, fill=c] (16.6866,7.48879) rectangle (16.7264,7.59464);
\draw [color=c, fill=c] (16.7264,7.48879) rectangle (16.7662,7.59464);
\draw [color=c, fill=c] (16.7662,7.48879) rectangle (16.806,7.59464);
\draw [color=c, fill=c] (16.806,7.48879) rectangle (16.8458,7.59464);
\draw [color=c, fill=c] (16.8458,7.48879) rectangle (16.8856,7.59464);
\draw [color=c, fill=c] (16.8856,7.48879) rectangle (16.9254,7.59464);
\draw [color=c, fill=c] (16.9254,7.48879) rectangle (16.9652,7.59464);
\draw [color=c, fill=c] (16.9652,7.48879) rectangle (17.005,7.59464);
\draw [color=c, fill=c] (17.005,7.48879) rectangle (17.0448,7.59464);
\draw [color=c, fill=c] (17.0448,7.48879) rectangle (17.0846,7.59464);
\draw [color=c, fill=c] (17.0846,7.48879) rectangle (17.1244,7.59464);
\draw [color=c, fill=c] (17.1244,7.48879) rectangle (17.1642,7.59464);
\draw [color=c, fill=c] (17.1642,7.48879) rectangle (17.204,7.59464);
\draw [color=c, fill=c] (17.204,7.48879) rectangle (17.2438,7.59464);
\draw [color=c, fill=c] (17.2438,7.48879) rectangle (17.2836,7.59464);
\draw [color=c, fill=c] (17.2836,7.48879) rectangle (17.3234,7.59464);
\draw [color=c, fill=c] (17.3234,7.48879) rectangle (17.3632,7.59464);
\draw [color=c, fill=c] (17.3632,7.48879) rectangle (17.403,7.59464);
\draw [color=c, fill=c] (17.403,7.48879) rectangle (17.4428,7.59464);
\draw [color=c, fill=c] (17.4428,7.48879) rectangle (17.4826,7.59464);
\draw [color=c, fill=c] (17.4826,7.48879) rectangle (17.5224,7.59464);
\draw [color=c, fill=c] (17.5224,7.48879) rectangle (17.5622,7.59464);
\draw [color=c, fill=c] (17.5622,7.48879) rectangle (17.602,7.59464);
\draw [color=c, fill=c] (17.602,7.48879) rectangle (17.6418,7.59464);
\draw [color=c, fill=c] (17.6418,7.48879) rectangle (17.6816,7.59464);
\draw [color=c, fill=c] (17.6816,7.48879) rectangle (17.7214,7.59464);
\draw [color=c, fill=c] (17.7214,7.48879) rectangle (17.7612,7.59464);
\draw [color=c, fill=c] (17.7612,7.48879) rectangle (17.801,7.59464);
\draw [color=c, fill=c] (17.801,7.48879) rectangle (17.8408,7.59464);
\draw [color=c, fill=c] (17.8408,7.48879) rectangle (17.8806,7.59464);
\draw [color=c, fill=c] (17.8806,7.48879) rectangle (17.9204,7.59464);
\draw [color=c, fill=c] (17.9204,7.48879) rectangle (17.9602,7.59464);
\draw [color=c, fill=c] (17.9602,7.48879) rectangle (18,7.59464);
\definecolor{c}{rgb}{0.2,0,1};
\draw [color=c, fill=c] (2,7.59464) rectangle (2.0398,7.70049);
\draw [color=c, fill=c] (2.0398,7.59464) rectangle (2.0796,7.70049);
\draw [color=c, fill=c] (2.0796,7.59464) rectangle (2.1194,7.70049);
\draw [color=c, fill=c] (2.1194,7.59464) rectangle (2.1592,7.70049);
\draw [color=c, fill=c] (2.1592,7.59464) rectangle (2.19901,7.70049);
\draw [color=c, fill=c] (2.19901,7.59464) rectangle (2.23881,7.70049);
\draw [color=c, fill=c] (2.23881,7.59464) rectangle (2.27861,7.70049);
\draw [color=c, fill=c] (2.27861,7.59464) rectangle (2.31841,7.70049);
\draw [color=c, fill=c] (2.31841,7.59464) rectangle (2.35821,7.70049);
\draw [color=c, fill=c] (2.35821,7.59464) rectangle (2.39801,7.70049);
\draw [color=c, fill=c] (2.39801,7.59464) rectangle (2.43781,7.70049);
\draw [color=c, fill=c] (2.43781,7.59464) rectangle (2.47761,7.70049);
\draw [color=c, fill=c] (2.47761,7.59464) rectangle (2.51741,7.70049);
\draw [color=c, fill=c] (2.51741,7.59464) rectangle (2.55721,7.70049);
\draw [color=c, fill=c] (2.55721,7.59464) rectangle (2.59702,7.70049);
\draw [color=c, fill=c] (2.59702,7.59464) rectangle (2.63682,7.70049);
\draw [color=c, fill=c] (2.63682,7.59464) rectangle (2.67662,7.70049);
\draw [color=c, fill=c] (2.67662,7.59464) rectangle (2.71642,7.70049);
\draw [color=c, fill=c] (2.71642,7.59464) rectangle (2.75622,7.70049);
\draw [color=c, fill=c] (2.75622,7.59464) rectangle (2.79602,7.70049);
\draw [color=c, fill=c] (2.79602,7.59464) rectangle (2.83582,7.70049);
\draw [color=c, fill=c] (2.83582,7.59464) rectangle (2.87562,7.70049);
\draw [color=c, fill=c] (2.87562,7.59464) rectangle (2.91542,7.70049);
\draw [color=c, fill=c] (2.91542,7.59464) rectangle (2.95522,7.70049);
\draw [color=c, fill=c] (2.95522,7.59464) rectangle (2.99502,7.70049);
\draw [color=c, fill=c] (2.99502,7.59464) rectangle (3.03483,7.70049);
\draw [color=c, fill=c] (3.03483,7.59464) rectangle (3.07463,7.70049);
\draw [color=c, fill=c] (3.07463,7.59464) rectangle (3.11443,7.70049);
\draw [color=c, fill=c] (3.11443,7.59464) rectangle (3.15423,7.70049);
\draw [color=c, fill=c] (3.15423,7.59464) rectangle (3.19403,7.70049);
\draw [color=c, fill=c] (3.19403,7.59464) rectangle (3.23383,7.70049);
\draw [color=c, fill=c] (3.23383,7.59464) rectangle (3.27363,7.70049);
\draw [color=c, fill=c] (3.27363,7.59464) rectangle (3.31343,7.70049);
\draw [color=c, fill=c] (3.31343,7.59464) rectangle (3.35323,7.70049);
\draw [color=c, fill=c] (3.35323,7.59464) rectangle (3.39303,7.70049);
\draw [color=c, fill=c] (3.39303,7.59464) rectangle (3.43284,7.70049);
\draw [color=c, fill=c] (3.43284,7.59464) rectangle (3.47264,7.70049);
\draw [color=c, fill=c] (3.47264,7.59464) rectangle (3.51244,7.70049);
\draw [color=c, fill=c] (3.51244,7.59464) rectangle (3.55224,7.70049);
\draw [color=c, fill=c] (3.55224,7.59464) rectangle (3.59204,7.70049);
\draw [color=c, fill=c] (3.59204,7.59464) rectangle (3.63184,7.70049);
\draw [color=c, fill=c] (3.63184,7.59464) rectangle (3.67164,7.70049);
\draw [color=c, fill=c] (3.67164,7.59464) rectangle (3.71144,7.70049);
\draw [color=c, fill=c] (3.71144,7.59464) rectangle (3.75124,7.70049);
\draw [color=c, fill=c] (3.75124,7.59464) rectangle (3.79104,7.70049);
\draw [color=c, fill=c] (3.79104,7.59464) rectangle (3.83085,7.70049);
\draw [color=c, fill=c] (3.83085,7.59464) rectangle (3.87065,7.70049);
\draw [color=c, fill=c] (3.87065,7.59464) rectangle (3.91045,7.70049);
\draw [color=c, fill=c] (3.91045,7.59464) rectangle (3.95025,7.70049);
\draw [color=c, fill=c] (3.95025,7.59464) rectangle (3.99005,7.70049);
\draw [color=c, fill=c] (3.99005,7.59464) rectangle (4.02985,7.70049);
\draw [color=c, fill=c] (4.02985,7.59464) rectangle (4.06965,7.70049);
\draw [color=c, fill=c] (4.06965,7.59464) rectangle (4.10945,7.70049);
\draw [color=c, fill=c] (4.10945,7.59464) rectangle (4.14925,7.70049);
\draw [color=c, fill=c] (4.14925,7.59464) rectangle (4.18905,7.70049);
\draw [color=c, fill=c] (4.18905,7.59464) rectangle (4.22886,7.70049);
\draw [color=c, fill=c] (4.22886,7.59464) rectangle (4.26866,7.70049);
\draw [color=c, fill=c] (4.26866,7.59464) rectangle (4.30846,7.70049);
\draw [color=c, fill=c] (4.30846,7.59464) rectangle (4.34826,7.70049);
\draw [color=c, fill=c] (4.34826,7.59464) rectangle (4.38806,7.70049);
\draw [color=c, fill=c] (4.38806,7.59464) rectangle (4.42786,7.70049);
\draw [color=c, fill=c] (4.42786,7.59464) rectangle (4.46766,7.70049);
\draw [color=c, fill=c] (4.46766,7.59464) rectangle (4.50746,7.70049);
\draw [color=c, fill=c] (4.50746,7.59464) rectangle (4.54726,7.70049);
\draw [color=c, fill=c] (4.54726,7.59464) rectangle (4.58706,7.70049);
\draw [color=c, fill=c] (4.58706,7.59464) rectangle (4.62687,7.70049);
\draw [color=c, fill=c] (4.62687,7.59464) rectangle (4.66667,7.70049);
\draw [color=c, fill=c] (4.66667,7.59464) rectangle (4.70647,7.70049);
\draw [color=c, fill=c] (4.70647,7.59464) rectangle (4.74627,7.70049);
\draw [color=c, fill=c] (4.74627,7.59464) rectangle (4.78607,7.70049);
\draw [color=c, fill=c] (4.78607,7.59464) rectangle (4.82587,7.70049);
\draw [color=c, fill=c] (4.82587,7.59464) rectangle (4.86567,7.70049);
\draw [color=c, fill=c] (4.86567,7.59464) rectangle (4.90547,7.70049);
\draw [color=c, fill=c] (4.90547,7.59464) rectangle (4.94527,7.70049);
\draw [color=c, fill=c] (4.94527,7.59464) rectangle (4.98507,7.70049);
\draw [color=c, fill=c] (4.98507,7.59464) rectangle (5.02488,7.70049);
\draw [color=c, fill=c] (5.02488,7.59464) rectangle (5.06468,7.70049);
\draw [color=c, fill=c] (5.06468,7.59464) rectangle (5.10448,7.70049);
\draw [color=c, fill=c] (5.10448,7.59464) rectangle (5.14428,7.70049);
\draw [color=c, fill=c] (5.14428,7.59464) rectangle (5.18408,7.70049);
\draw [color=c, fill=c] (5.18408,7.59464) rectangle (5.22388,7.70049);
\draw [color=c, fill=c] (5.22388,7.59464) rectangle (5.26368,7.70049);
\draw [color=c, fill=c] (5.26368,7.59464) rectangle (5.30348,7.70049);
\draw [color=c, fill=c] (5.30348,7.59464) rectangle (5.34328,7.70049);
\draw [color=c, fill=c] (5.34328,7.59464) rectangle (5.38308,7.70049);
\draw [color=c, fill=c] (5.38308,7.59464) rectangle (5.42289,7.70049);
\draw [color=c, fill=c] (5.42289,7.59464) rectangle (5.46269,7.70049);
\draw [color=c, fill=c] (5.46269,7.59464) rectangle (5.50249,7.70049);
\draw [color=c, fill=c] (5.50249,7.59464) rectangle (5.54229,7.70049);
\draw [color=c, fill=c] (5.54229,7.59464) rectangle (5.58209,7.70049);
\draw [color=c, fill=c] (5.58209,7.59464) rectangle (5.62189,7.70049);
\draw [color=c, fill=c] (5.62189,7.59464) rectangle (5.66169,7.70049);
\draw [color=c, fill=c] (5.66169,7.59464) rectangle (5.70149,7.70049);
\draw [color=c, fill=c] (5.70149,7.59464) rectangle (5.74129,7.70049);
\draw [color=c, fill=c] (5.74129,7.59464) rectangle (5.78109,7.70049);
\draw [color=c, fill=c] (5.78109,7.59464) rectangle (5.8209,7.70049);
\draw [color=c, fill=c] (5.8209,7.59464) rectangle (5.8607,7.70049);
\draw [color=c, fill=c] (5.8607,7.59464) rectangle (5.9005,7.70049);
\draw [color=c, fill=c] (5.9005,7.59464) rectangle (5.9403,7.70049);
\draw [color=c, fill=c] (5.9403,7.59464) rectangle (5.9801,7.70049);
\draw [color=c, fill=c] (5.9801,7.59464) rectangle (6.0199,7.70049);
\draw [color=c, fill=c] (6.0199,7.59464) rectangle (6.0597,7.70049);
\draw [color=c, fill=c] (6.0597,7.59464) rectangle (6.0995,7.70049);
\draw [color=c, fill=c] (6.0995,7.59464) rectangle (6.1393,7.70049);
\draw [color=c, fill=c] (6.1393,7.59464) rectangle (6.1791,7.70049);
\draw [color=c, fill=c] (6.1791,7.59464) rectangle (6.21891,7.70049);
\draw [color=c, fill=c] (6.21891,7.59464) rectangle (6.25871,7.70049);
\draw [color=c, fill=c] (6.25871,7.59464) rectangle (6.29851,7.70049);
\draw [color=c, fill=c] (6.29851,7.59464) rectangle (6.33831,7.70049);
\draw [color=c, fill=c] (6.33831,7.59464) rectangle (6.37811,7.70049);
\draw [color=c, fill=c] (6.37811,7.59464) rectangle (6.41791,7.70049);
\draw [color=c, fill=c] (6.41791,7.59464) rectangle (6.45771,7.70049);
\draw [color=c, fill=c] (6.45771,7.59464) rectangle (6.49751,7.70049);
\draw [color=c, fill=c] (6.49751,7.59464) rectangle (6.53731,7.70049);
\draw [color=c, fill=c] (6.53731,7.59464) rectangle (6.57711,7.70049);
\draw [color=c, fill=c] (6.57711,7.59464) rectangle (6.61692,7.70049);
\draw [color=c, fill=c] (6.61692,7.59464) rectangle (6.65672,7.70049);
\draw [color=c, fill=c] (6.65672,7.59464) rectangle (6.69652,7.70049);
\draw [color=c, fill=c] (6.69652,7.59464) rectangle (6.73632,7.70049);
\draw [color=c, fill=c] (6.73632,7.59464) rectangle (6.77612,7.70049);
\draw [color=c, fill=c] (6.77612,7.59464) rectangle (6.81592,7.70049);
\draw [color=c, fill=c] (6.81592,7.59464) rectangle (6.85572,7.70049);
\draw [color=c, fill=c] (6.85572,7.59464) rectangle (6.89552,7.70049);
\draw [color=c, fill=c] (6.89552,7.59464) rectangle (6.93532,7.70049);
\draw [color=c, fill=c] (6.93532,7.59464) rectangle (6.97512,7.70049);
\draw [color=c, fill=c] (6.97512,7.59464) rectangle (7.01493,7.70049);
\draw [color=c, fill=c] (7.01493,7.59464) rectangle (7.05473,7.70049);
\draw [color=c, fill=c] (7.05473,7.59464) rectangle (7.09453,7.70049);
\draw [color=c, fill=c] (7.09453,7.59464) rectangle (7.13433,7.70049);
\draw [color=c, fill=c] (7.13433,7.59464) rectangle (7.17413,7.70049);
\draw [color=c, fill=c] (7.17413,7.59464) rectangle (7.21393,7.70049);
\draw [color=c, fill=c] (7.21393,7.59464) rectangle (7.25373,7.70049);
\draw [color=c, fill=c] (7.25373,7.59464) rectangle (7.29353,7.70049);
\draw [color=c, fill=c] (7.29353,7.59464) rectangle (7.33333,7.70049);
\draw [color=c, fill=c] (7.33333,7.59464) rectangle (7.37313,7.70049);
\draw [color=c, fill=c] (7.37313,7.59464) rectangle (7.41294,7.70049);
\draw [color=c, fill=c] (7.41294,7.59464) rectangle (7.45274,7.70049);
\draw [color=c, fill=c] (7.45274,7.59464) rectangle (7.49254,7.70049);
\draw [color=c, fill=c] (7.49254,7.59464) rectangle (7.53234,7.70049);
\draw [color=c, fill=c] (7.53234,7.59464) rectangle (7.57214,7.70049);
\draw [color=c, fill=c] (7.57214,7.59464) rectangle (7.61194,7.70049);
\draw [color=c, fill=c] (7.61194,7.59464) rectangle (7.65174,7.70049);
\draw [color=c, fill=c] (7.65174,7.59464) rectangle (7.69154,7.70049);
\draw [color=c, fill=c] (7.69154,7.59464) rectangle (7.73134,7.70049);
\draw [color=c, fill=c] (7.73134,7.59464) rectangle (7.77114,7.70049);
\draw [color=c, fill=c] (7.77114,7.59464) rectangle (7.81095,7.70049);
\definecolor{c}{rgb}{0,0.0800001,1};
\draw [color=c, fill=c] (7.81095,7.59464) rectangle (7.85075,7.70049);
\draw [color=c, fill=c] (7.85075,7.59464) rectangle (7.89055,7.70049);
\draw [color=c, fill=c] (7.89055,7.59464) rectangle (7.93035,7.70049);
\draw [color=c, fill=c] (7.93035,7.59464) rectangle (7.97015,7.70049);
\draw [color=c, fill=c] (7.97015,7.59464) rectangle (8.00995,7.70049);
\draw [color=c, fill=c] (8.00995,7.59464) rectangle (8.04975,7.70049);
\draw [color=c, fill=c] (8.04975,7.59464) rectangle (8.08955,7.70049);
\draw [color=c, fill=c] (8.08955,7.59464) rectangle (8.12935,7.70049);
\draw [color=c, fill=c] (8.12935,7.59464) rectangle (8.16915,7.70049);
\draw [color=c, fill=c] (8.16915,7.59464) rectangle (8.20895,7.70049);
\draw [color=c, fill=c] (8.20895,7.59464) rectangle (8.24876,7.70049);
\draw [color=c, fill=c] (8.24876,7.59464) rectangle (8.28856,7.70049);
\draw [color=c, fill=c] (8.28856,7.59464) rectangle (8.32836,7.70049);
\draw [color=c, fill=c] (8.32836,7.59464) rectangle (8.36816,7.70049);
\draw [color=c, fill=c] (8.36816,7.59464) rectangle (8.40796,7.70049);
\draw [color=c, fill=c] (8.40796,7.59464) rectangle (8.44776,7.70049);
\draw [color=c, fill=c] (8.44776,7.59464) rectangle (8.48756,7.70049);
\draw [color=c, fill=c] (8.48756,7.59464) rectangle (8.52736,7.70049);
\draw [color=c, fill=c] (8.52736,7.59464) rectangle (8.56716,7.70049);
\draw [color=c, fill=c] (8.56716,7.59464) rectangle (8.60697,7.70049);
\draw [color=c, fill=c] (8.60697,7.59464) rectangle (8.64677,7.70049);
\draw [color=c, fill=c] (8.64677,7.59464) rectangle (8.68657,7.70049);
\draw [color=c, fill=c] (8.68657,7.59464) rectangle (8.72637,7.70049);
\draw [color=c, fill=c] (8.72637,7.59464) rectangle (8.76617,7.70049);
\draw [color=c, fill=c] (8.76617,7.59464) rectangle (8.80597,7.70049);
\draw [color=c, fill=c] (8.80597,7.59464) rectangle (8.84577,7.70049);
\draw [color=c, fill=c] (8.84577,7.59464) rectangle (8.88557,7.70049);
\draw [color=c, fill=c] (8.88557,7.59464) rectangle (8.92537,7.70049);
\draw [color=c, fill=c] (8.92537,7.59464) rectangle (8.96517,7.70049);
\draw [color=c, fill=c] (8.96517,7.59464) rectangle (9.00498,7.70049);
\draw [color=c, fill=c] (9.00498,7.59464) rectangle (9.04478,7.70049);
\draw [color=c, fill=c] (9.04478,7.59464) rectangle (9.08458,7.70049);
\draw [color=c, fill=c] (9.08458,7.59464) rectangle (9.12438,7.70049);
\draw [color=c, fill=c] (9.12438,7.59464) rectangle (9.16418,7.70049);
\draw [color=c, fill=c] (9.16418,7.59464) rectangle (9.20398,7.70049);
\draw [color=c, fill=c] (9.20398,7.59464) rectangle (9.24378,7.70049);
\draw [color=c, fill=c] (9.24378,7.59464) rectangle (9.28358,7.70049);
\draw [color=c, fill=c] (9.28358,7.59464) rectangle (9.32338,7.70049);
\draw [color=c, fill=c] (9.32338,7.59464) rectangle (9.36318,7.70049);
\draw [color=c, fill=c] (9.36318,7.59464) rectangle (9.40298,7.70049);
\draw [color=c, fill=c] (9.40298,7.59464) rectangle (9.44279,7.70049);
\draw [color=c, fill=c] (9.44279,7.59464) rectangle (9.48259,7.70049);
\draw [color=c, fill=c] (9.48259,7.59464) rectangle (9.52239,7.70049);
\draw [color=c, fill=c] (9.52239,7.59464) rectangle (9.56219,7.70049);
\definecolor{c}{rgb}{0,0.266667,1};
\draw [color=c, fill=c] (9.56219,7.59464) rectangle (9.60199,7.70049);
\draw [color=c, fill=c] (9.60199,7.59464) rectangle (9.64179,7.70049);
\draw [color=c, fill=c] (9.64179,7.59464) rectangle (9.68159,7.70049);
\draw [color=c, fill=c] (9.68159,7.59464) rectangle (9.72139,7.70049);
\draw [color=c, fill=c] (9.72139,7.59464) rectangle (9.76119,7.70049);
\draw [color=c, fill=c] (9.76119,7.59464) rectangle (9.80099,7.70049);
\draw [color=c, fill=c] (9.80099,7.59464) rectangle (9.8408,7.70049);
\draw [color=c, fill=c] (9.8408,7.59464) rectangle (9.8806,7.70049);
\draw [color=c, fill=c] (9.8806,7.59464) rectangle (9.9204,7.70049);
\draw [color=c, fill=c] (9.9204,7.59464) rectangle (9.9602,7.70049);
\draw [color=c, fill=c] (9.9602,7.59464) rectangle (10,7.70049);
\draw [color=c, fill=c] (10,7.59464) rectangle (10.0398,7.70049);
\draw [color=c, fill=c] (10.0398,7.59464) rectangle (10.0796,7.70049);
\draw [color=c, fill=c] (10.0796,7.59464) rectangle (10.1194,7.70049);
\draw [color=c, fill=c] (10.1194,7.59464) rectangle (10.1592,7.70049);
\draw [color=c, fill=c] (10.1592,7.59464) rectangle (10.199,7.70049);
\draw [color=c, fill=c] (10.199,7.59464) rectangle (10.2388,7.70049);
\draw [color=c, fill=c] (10.2388,7.59464) rectangle (10.2786,7.70049);
\draw [color=c, fill=c] (10.2786,7.59464) rectangle (10.3184,7.70049);
\draw [color=c, fill=c] (10.3184,7.59464) rectangle (10.3582,7.70049);
\definecolor{c}{rgb}{0,0.546666,1};
\draw [color=c, fill=c] (10.3582,7.59464) rectangle (10.398,7.70049);
\draw [color=c, fill=c] (10.398,7.59464) rectangle (10.4378,7.70049);
\draw [color=c, fill=c] (10.4378,7.59464) rectangle (10.4776,7.70049);
\draw [color=c, fill=c] (10.4776,7.59464) rectangle (10.5174,7.70049);
\draw [color=c, fill=c] (10.5174,7.59464) rectangle (10.5572,7.70049);
\draw [color=c, fill=c] (10.5572,7.59464) rectangle (10.597,7.70049);
\draw [color=c, fill=c] (10.597,7.59464) rectangle (10.6368,7.70049);
\draw [color=c, fill=c] (10.6368,7.59464) rectangle (10.6766,7.70049);
\draw [color=c, fill=c] (10.6766,7.59464) rectangle (10.7164,7.70049);
\draw [color=c, fill=c] (10.7164,7.59464) rectangle (10.7562,7.70049);
\draw [color=c, fill=c] (10.7562,7.59464) rectangle (10.796,7.70049);
\draw [color=c, fill=c] (10.796,7.59464) rectangle (10.8358,7.70049);
\draw [color=c, fill=c] (10.8358,7.59464) rectangle (10.8756,7.70049);
\draw [color=c, fill=c] (10.8756,7.59464) rectangle (10.9154,7.70049);
\draw [color=c, fill=c] (10.9154,7.59464) rectangle (10.9552,7.70049);
\draw [color=c, fill=c] (10.9552,7.59464) rectangle (10.995,7.70049);
\draw [color=c, fill=c] (10.995,7.59464) rectangle (11.0348,7.70049);
\draw [color=c, fill=c] (11.0348,7.59464) rectangle (11.0746,7.70049);
\draw [color=c, fill=c] (11.0746,7.59464) rectangle (11.1144,7.70049);
\draw [color=c, fill=c] (11.1144,7.59464) rectangle (11.1542,7.70049);
\draw [color=c, fill=c] (11.1542,7.59464) rectangle (11.194,7.70049);
\draw [color=c, fill=c] (11.194,7.59464) rectangle (11.2338,7.70049);
\draw [color=c, fill=c] (11.2338,7.59464) rectangle (11.2736,7.70049);
\draw [color=c, fill=c] (11.2736,7.59464) rectangle (11.3134,7.70049);
\draw [color=c, fill=c] (11.3134,7.59464) rectangle (11.3532,7.70049);
\draw [color=c, fill=c] (11.3532,7.59464) rectangle (11.393,7.70049);
\draw [color=c, fill=c] (11.393,7.59464) rectangle (11.4328,7.70049);
\draw [color=c, fill=c] (11.4328,7.59464) rectangle (11.4726,7.70049);
\draw [color=c, fill=c] (11.4726,7.59464) rectangle (11.5124,7.70049);
\draw [color=c, fill=c] (11.5124,7.59464) rectangle (11.5522,7.70049);
\draw [color=c, fill=c] (11.5522,7.59464) rectangle (11.592,7.70049);
\draw [color=c, fill=c] (11.592,7.59464) rectangle (11.6318,7.70049);
\draw [color=c, fill=c] (11.6318,7.59464) rectangle (11.6716,7.70049);
\draw [color=c, fill=c] (11.6716,7.59464) rectangle (11.7114,7.70049);
\draw [color=c, fill=c] (11.7114,7.59464) rectangle (11.7512,7.70049);
\draw [color=c, fill=c] (11.7512,7.59464) rectangle (11.791,7.70049);
\draw [color=c, fill=c] (11.791,7.59464) rectangle (11.8308,7.70049);
\draw [color=c, fill=c] (11.8308,7.59464) rectangle (11.8706,7.70049);
\definecolor{c}{rgb}{0,0.733333,1};
\draw [color=c, fill=c] (11.8706,7.59464) rectangle (11.9104,7.70049);
\draw [color=c, fill=c] (11.9104,7.59464) rectangle (11.9502,7.70049);
\draw [color=c, fill=c] (11.9502,7.59464) rectangle (11.99,7.70049);
\draw [color=c, fill=c] (11.99,7.59464) rectangle (12.0299,7.70049);
\draw [color=c, fill=c] (12.0299,7.59464) rectangle (12.0697,7.70049);
\draw [color=c, fill=c] (12.0697,7.59464) rectangle (12.1095,7.70049);
\draw [color=c, fill=c] (12.1095,7.59464) rectangle (12.1493,7.70049);
\draw [color=c, fill=c] (12.1493,7.59464) rectangle (12.1891,7.70049);
\draw [color=c, fill=c] (12.1891,7.59464) rectangle (12.2289,7.70049);
\draw [color=c, fill=c] (12.2289,7.59464) rectangle (12.2687,7.70049);
\draw [color=c, fill=c] (12.2687,7.59464) rectangle (12.3085,7.70049);
\draw [color=c, fill=c] (12.3085,7.59464) rectangle (12.3483,7.70049);
\draw [color=c, fill=c] (12.3483,7.59464) rectangle (12.3881,7.70049);
\draw [color=c, fill=c] (12.3881,7.59464) rectangle (12.4279,7.70049);
\draw [color=c, fill=c] (12.4279,7.59464) rectangle (12.4677,7.70049);
\draw [color=c, fill=c] (12.4677,7.59464) rectangle (12.5075,7.70049);
\draw [color=c, fill=c] (12.5075,7.59464) rectangle (12.5473,7.70049);
\draw [color=c, fill=c] (12.5473,7.59464) rectangle (12.5871,7.70049);
\draw [color=c, fill=c] (12.5871,7.59464) rectangle (12.6269,7.70049);
\draw [color=c, fill=c] (12.6269,7.59464) rectangle (12.6667,7.70049);
\draw [color=c, fill=c] (12.6667,7.59464) rectangle (12.7065,7.70049);
\draw [color=c, fill=c] (12.7065,7.59464) rectangle (12.7463,7.70049);
\draw [color=c, fill=c] (12.7463,7.59464) rectangle (12.7861,7.70049);
\draw [color=c, fill=c] (12.7861,7.59464) rectangle (12.8259,7.70049);
\draw [color=c, fill=c] (12.8259,7.59464) rectangle (12.8657,7.70049);
\draw [color=c, fill=c] (12.8657,7.59464) rectangle (12.9055,7.70049);
\draw [color=c, fill=c] (12.9055,7.59464) rectangle (12.9453,7.70049);
\draw [color=c, fill=c] (12.9453,7.59464) rectangle (12.9851,7.70049);
\draw [color=c, fill=c] (12.9851,7.59464) rectangle (13.0249,7.70049);
\draw [color=c, fill=c] (13.0249,7.59464) rectangle (13.0647,7.70049);
\draw [color=c, fill=c] (13.0647,7.59464) rectangle (13.1045,7.70049);
\draw [color=c, fill=c] (13.1045,7.59464) rectangle (13.1443,7.70049);
\draw [color=c, fill=c] (13.1443,7.59464) rectangle (13.1841,7.70049);
\draw [color=c, fill=c] (13.1841,7.59464) rectangle (13.2239,7.70049);
\draw [color=c, fill=c] (13.2239,7.59464) rectangle (13.2637,7.70049);
\draw [color=c, fill=c] (13.2637,7.59464) rectangle (13.3035,7.70049);
\draw [color=c, fill=c] (13.3035,7.59464) rectangle (13.3433,7.70049);
\draw [color=c, fill=c] (13.3433,7.59464) rectangle (13.3831,7.70049);
\draw [color=c, fill=c] (13.3831,7.59464) rectangle (13.4229,7.70049);
\draw [color=c, fill=c] (13.4229,7.59464) rectangle (13.4627,7.70049);
\draw [color=c, fill=c] (13.4627,7.59464) rectangle (13.5025,7.70049);
\draw [color=c, fill=c] (13.5025,7.59464) rectangle (13.5423,7.70049);
\draw [color=c, fill=c] (13.5423,7.59464) rectangle (13.5821,7.70049);
\draw [color=c, fill=c] (13.5821,7.59464) rectangle (13.6219,7.70049);
\draw [color=c, fill=c] (13.6219,7.59464) rectangle (13.6617,7.70049);
\draw [color=c, fill=c] (13.6617,7.59464) rectangle (13.7015,7.70049);
\draw [color=c, fill=c] (13.7015,7.59464) rectangle (13.7413,7.70049);
\draw [color=c, fill=c] (13.7413,7.59464) rectangle (13.7811,7.70049);
\draw [color=c, fill=c] (13.7811,7.59464) rectangle (13.8209,7.70049);
\draw [color=c, fill=c] (13.8209,7.59464) rectangle (13.8607,7.70049);
\draw [color=c, fill=c] (13.8607,7.59464) rectangle (13.9005,7.70049);
\draw [color=c, fill=c] (13.9005,7.59464) rectangle (13.9403,7.70049);
\draw [color=c, fill=c] (13.9403,7.59464) rectangle (13.9801,7.70049);
\draw [color=c, fill=c] (13.9801,7.59464) rectangle (14.0199,7.70049);
\draw [color=c, fill=c] (14.0199,7.59464) rectangle (14.0597,7.70049);
\draw [color=c, fill=c] (14.0597,7.59464) rectangle (14.0995,7.70049);
\draw [color=c, fill=c] (14.0995,7.59464) rectangle (14.1393,7.70049);
\draw [color=c, fill=c] (14.1393,7.59464) rectangle (14.1791,7.70049);
\draw [color=c, fill=c] (14.1791,7.59464) rectangle (14.2189,7.70049);
\draw [color=c, fill=c] (14.2189,7.59464) rectangle (14.2587,7.70049);
\draw [color=c, fill=c] (14.2587,7.59464) rectangle (14.2985,7.70049);
\draw [color=c, fill=c] (14.2985,7.59464) rectangle (14.3383,7.70049);
\draw [color=c, fill=c] (14.3383,7.59464) rectangle (14.3781,7.70049);
\draw [color=c, fill=c] (14.3781,7.59464) rectangle (14.4179,7.70049);
\draw [color=c, fill=c] (14.4179,7.59464) rectangle (14.4577,7.70049);
\draw [color=c, fill=c] (14.4577,7.59464) rectangle (14.4975,7.70049);
\draw [color=c, fill=c] (14.4975,7.59464) rectangle (14.5373,7.70049);
\draw [color=c, fill=c] (14.5373,7.59464) rectangle (14.5771,7.70049);
\draw [color=c, fill=c] (14.5771,7.59464) rectangle (14.6169,7.70049);
\draw [color=c, fill=c] (14.6169,7.59464) rectangle (14.6567,7.70049);
\draw [color=c, fill=c] (14.6567,7.59464) rectangle (14.6965,7.70049);
\draw [color=c, fill=c] (14.6965,7.59464) rectangle (14.7363,7.70049);
\draw [color=c, fill=c] (14.7363,7.59464) rectangle (14.7761,7.70049);
\draw [color=c, fill=c] (14.7761,7.59464) rectangle (14.8159,7.70049);
\draw [color=c, fill=c] (14.8159,7.59464) rectangle (14.8557,7.70049);
\draw [color=c, fill=c] (14.8557,7.59464) rectangle (14.8955,7.70049);
\draw [color=c, fill=c] (14.8955,7.59464) rectangle (14.9353,7.70049);
\draw [color=c, fill=c] (14.9353,7.59464) rectangle (14.9751,7.70049);
\draw [color=c, fill=c] (14.9751,7.59464) rectangle (15.0149,7.70049);
\draw [color=c, fill=c] (15.0149,7.59464) rectangle (15.0547,7.70049);
\draw [color=c, fill=c] (15.0547,7.59464) rectangle (15.0945,7.70049);
\draw [color=c, fill=c] (15.0945,7.59464) rectangle (15.1343,7.70049);
\draw [color=c, fill=c] (15.1343,7.59464) rectangle (15.1741,7.70049);
\draw [color=c, fill=c] (15.1741,7.59464) rectangle (15.2139,7.70049);
\draw [color=c, fill=c] (15.2139,7.59464) rectangle (15.2537,7.70049);
\draw [color=c, fill=c] (15.2537,7.59464) rectangle (15.2935,7.70049);
\draw [color=c, fill=c] (15.2935,7.59464) rectangle (15.3333,7.70049);
\draw [color=c, fill=c] (15.3333,7.59464) rectangle (15.3731,7.70049);
\draw [color=c, fill=c] (15.3731,7.59464) rectangle (15.4129,7.70049);
\draw [color=c, fill=c] (15.4129,7.59464) rectangle (15.4527,7.70049);
\draw [color=c, fill=c] (15.4527,7.59464) rectangle (15.4925,7.70049);
\draw [color=c, fill=c] (15.4925,7.59464) rectangle (15.5323,7.70049);
\draw [color=c, fill=c] (15.5323,7.59464) rectangle (15.5721,7.70049);
\draw [color=c, fill=c] (15.5721,7.59464) rectangle (15.6119,7.70049);
\draw [color=c, fill=c] (15.6119,7.59464) rectangle (15.6517,7.70049);
\draw [color=c, fill=c] (15.6517,7.59464) rectangle (15.6915,7.70049);
\draw [color=c, fill=c] (15.6915,7.59464) rectangle (15.7313,7.70049);
\draw [color=c, fill=c] (15.7313,7.59464) rectangle (15.7711,7.70049);
\draw [color=c, fill=c] (15.7711,7.59464) rectangle (15.8109,7.70049);
\draw [color=c, fill=c] (15.8109,7.59464) rectangle (15.8507,7.70049);
\draw [color=c, fill=c] (15.8507,7.59464) rectangle (15.8905,7.70049);
\draw [color=c, fill=c] (15.8905,7.59464) rectangle (15.9303,7.70049);
\draw [color=c, fill=c] (15.9303,7.59464) rectangle (15.9701,7.70049);
\draw [color=c, fill=c] (15.9701,7.59464) rectangle (16.01,7.70049);
\draw [color=c, fill=c] (16.01,7.59464) rectangle (16.0498,7.70049);
\draw [color=c, fill=c] (16.0498,7.59464) rectangle (16.0896,7.70049);
\draw [color=c, fill=c] (16.0896,7.59464) rectangle (16.1294,7.70049);
\draw [color=c, fill=c] (16.1294,7.59464) rectangle (16.1692,7.70049);
\draw [color=c, fill=c] (16.1692,7.59464) rectangle (16.209,7.70049);
\draw [color=c, fill=c] (16.209,7.59464) rectangle (16.2488,7.70049);
\draw [color=c, fill=c] (16.2488,7.59464) rectangle (16.2886,7.70049);
\draw [color=c, fill=c] (16.2886,7.59464) rectangle (16.3284,7.70049);
\draw [color=c, fill=c] (16.3284,7.59464) rectangle (16.3682,7.70049);
\draw [color=c, fill=c] (16.3682,7.59464) rectangle (16.408,7.70049);
\draw [color=c, fill=c] (16.408,7.59464) rectangle (16.4478,7.70049);
\draw [color=c, fill=c] (16.4478,7.59464) rectangle (16.4876,7.70049);
\draw [color=c, fill=c] (16.4876,7.59464) rectangle (16.5274,7.70049);
\draw [color=c, fill=c] (16.5274,7.59464) rectangle (16.5672,7.70049);
\draw [color=c, fill=c] (16.5672,7.59464) rectangle (16.607,7.70049);
\draw [color=c, fill=c] (16.607,7.59464) rectangle (16.6468,7.70049);
\draw [color=c, fill=c] (16.6468,7.59464) rectangle (16.6866,7.70049);
\draw [color=c, fill=c] (16.6866,7.59464) rectangle (16.7264,7.70049);
\draw [color=c, fill=c] (16.7264,7.59464) rectangle (16.7662,7.70049);
\draw [color=c, fill=c] (16.7662,7.59464) rectangle (16.806,7.70049);
\draw [color=c, fill=c] (16.806,7.59464) rectangle (16.8458,7.70049);
\draw [color=c, fill=c] (16.8458,7.59464) rectangle (16.8856,7.70049);
\draw [color=c, fill=c] (16.8856,7.59464) rectangle (16.9254,7.70049);
\draw [color=c, fill=c] (16.9254,7.59464) rectangle (16.9652,7.70049);
\draw [color=c, fill=c] (16.9652,7.59464) rectangle (17.005,7.70049);
\draw [color=c, fill=c] (17.005,7.59464) rectangle (17.0448,7.70049);
\draw [color=c, fill=c] (17.0448,7.59464) rectangle (17.0846,7.70049);
\draw [color=c, fill=c] (17.0846,7.59464) rectangle (17.1244,7.70049);
\draw [color=c, fill=c] (17.1244,7.59464) rectangle (17.1642,7.70049);
\draw [color=c, fill=c] (17.1642,7.59464) rectangle (17.204,7.70049);
\draw [color=c, fill=c] (17.204,7.59464) rectangle (17.2438,7.70049);
\draw [color=c, fill=c] (17.2438,7.59464) rectangle (17.2836,7.70049);
\draw [color=c, fill=c] (17.2836,7.59464) rectangle (17.3234,7.70049);
\draw [color=c, fill=c] (17.3234,7.59464) rectangle (17.3632,7.70049);
\draw [color=c, fill=c] (17.3632,7.59464) rectangle (17.403,7.70049);
\draw [color=c, fill=c] (17.403,7.59464) rectangle (17.4428,7.70049);
\draw [color=c, fill=c] (17.4428,7.59464) rectangle (17.4826,7.70049);
\draw [color=c, fill=c] (17.4826,7.59464) rectangle (17.5224,7.70049);
\draw [color=c, fill=c] (17.5224,7.59464) rectangle (17.5622,7.70049);
\draw [color=c, fill=c] (17.5622,7.59464) rectangle (17.602,7.70049);
\draw [color=c, fill=c] (17.602,7.59464) rectangle (17.6418,7.70049);
\draw [color=c, fill=c] (17.6418,7.59464) rectangle (17.6816,7.70049);
\draw [color=c, fill=c] (17.6816,7.59464) rectangle (17.7214,7.70049);
\draw [color=c, fill=c] (17.7214,7.59464) rectangle (17.7612,7.70049);
\draw [color=c, fill=c] (17.7612,7.59464) rectangle (17.801,7.70049);
\draw [color=c, fill=c] (17.801,7.59464) rectangle (17.8408,7.70049);
\draw [color=c, fill=c] (17.8408,7.59464) rectangle (17.8806,7.70049);
\draw [color=c, fill=c] (17.8806,7.59464) rectangle (17.9204,7.70049);
\draw [color=c, fill=c] (17.9204,7.59464) rectangle (17.9602,7.70049);
\draw [color=c, fill=c] (17.9602,7.59464) rectangle (18,7.70049);
\definecolor{c}{rgb}{0.2,0,1};
\draw [color=c, fill=c] (2,7.70049) rectangle (2.0398,7.80634);
\draw [color=c, fill=c] (2.0398,7.70049) rectangle (2.0796,7.80634);
\draw [color=c, fill=c] (2.0796,7.70049) rectangle (2.1194,7.80634);
\draw [color=c, fill=c] (2.1194,7.70049) rectangle (2.1592,7.80634);
\draw [color=c, fill=c] (2.1592,7.70049) rectangle (2.19901,7.80634);
\draw [color=c, fill=c] (2.19901,7.70049) rectangle (2.23881,7.80634);
\draw [color=c, fill=c] (2.23881,7.70049) rectangle (2.27861,7.80634);
\draw [color=c, fill=c] (2.27861,7.70049) rectangle (2.31841,7.80634);
\draw [color=c, fill=c] (2.31841,7.70049) rectangle (2.35821,7.80634);
\draw [color=c, fill=c] (2.35821,7.70049) rectangle (2.39801,7.80634);
\draw [color=c, fill=c] (2.39801,7.70049) rectangle (2.43781,7.80634);
\draw [color=c, fill=c] (2.43781,7.70049) rectangle (2.47761,7.80634);
\draw [color=c, fill=c] (2.47761,7.70049) rectangle (2.51741,7.80634);
\draw [color=c, fill=c] (2.51741,7.70049) rectangle (2.55721,7.80634);
\draw [color=c, fill=c] (2.55721,7.70049) rectangle (2.59702,7.80634);
\draw [color=c, fill=c] (2.59702,7.70049) rectangle (2.63682,7.80634);
\draw [color=c, fill=c] (2.63682,7.70049) rectangle (2.67662,7.80634);
\draw [color=c, fill=c] (2.67662,7.70049) rectangle (2.71642,7.80634);
\draw [color=c, fill=c] (2.71642,7.70049) rectangle (2.75622,7.80634);
\draw [color=c, fill=c] (2.75622,7.70049) rectangle (2.79602,7.80634);
\draw [color=c, fill=c] (2.79602,7.70049) rectangle (2.83582,7.80634);
\draw [color=c, fill=c] (2.83582,7.70049) rectangle (2.87562,7.80634);
\draw [color=c, fill=c] (2.87562,7.70049) rectangle (2.91542,7.80634);
\draw [color=c, fill=c] (2.91542,7.70049) rectangle (2.95522,7.80634);
\draw [color=c, fill=c] (2.95522,7.70049) rectangle (2.99502,7.80634);
\draw [color=c, fill=c] (2.99502,7.70049) rectangle (3.03483,7.80634);
\draw [color=c, fill=c] (3.03483,7.70049) rectangle (3.07463,7.80634);
\draw [color=c, fill=c] (3.07463,7.70049) rectangle (3.11443,7.80634);
\draw [color=c, fill=c] (3.11443,7.70049) rectangle (3.15423,7.80634);
\draw [color=c, fill=c] (3.15423,7.70049) rectangle (3.19403,7.80634);
\draw [color=c, fill=c] (3.19403,7.70049) rectangle (3.23383,7.80634);
\draw [color=c, fill=c] (3.23383,7.70049) rectangle (3.27363,7.80634);
\draw [color=c, fill=c] (3.27363,7.70049) rectangle (3.31343,7.80634);
\draw [color=c, fill=c] (3.31343,7.70049) rectangle (3.35323,7.80634);
\draw [color=c, fill=c] (3.35323,7.70049) rectangle (3.39303,7.80634);
\draw [color=c, fill=c] (3.39303,7.70049) rectangle (3.43284,7.80634);
\draw [color=c, fill=c] (3.43284,7.70049) rectangle (3.47264,7.80634);
\draw [color=c, fill=c] (3.47264,7.70049) rectangle (3.51244,7.80634);
\draw [color=c, fill=c] (3.51244,7.70049) rectangle (3.55224,7.80634);
\draw [color=c, fill=c] (3.55224,7.70049) rectangle (3.59204,7.80634);
\draw [color=c, fill=c] (3.59204,7.70049) rectangle (3.63184,7.80634);
\draw [color=c, fill=c] (3.63184,7.70049) rectangle (3.67164,7.80634);
\draw [color=c, fill=c] (3.67164,7.70049) rectangle (3.71144,7.80634);
\draw [color=c, fill=c] (3.71144,7.70049) rectangle (3.75124,7.80634);
\draw [color=c, fill=c] (3.75124,7.70049) rectangle (3.79104,7.80634);
\draw [color=c, fill=c] (3.79104,7.70049) rectangle (3.83085,7.80634);
\draw [color=c, fill=c] (3.83085,7.70049) rectangle (3.87065,7.80634);
\draw [color=c, fill=c] (3.87065,7.70049) rectangle (3.91045,7.80634);
\draw [color=c, fill=c] (3.91045,7.70049) rectangle (3.95025,7.80634);
\draw [color=c, fill=c] (3.95025,7.70049) rectangle (3.99005,7.80634);
\draw [color=c, fill=c] (3.99005,7.70049) rectangle (4.02985,7.80634);
\draw [color=c, fill=c] (4.02985,7.70049) rectangle (4.06965,7.80634);
\draw [color=c, fill=c] (4.06965,7.70049) rectangle (4.10945,7.80634);
\draw [color=c, fill=c] (4.10945,7.70049) rectangle (4.14925,7.80634);
\draw [color=c, fill=c] (4.14925,7.70049) rectangle (4.18905,7.80634);
\draw [color=c, fill=c] (4.18905,7.70049) rectangle (4.22886,7.80634);
\draw [color=c, fill=c] (4.22886,7.70049) rectangle (4.26866,7.80634);
\draw [color=c, fill=c] (4.26866,7.70049) rectangle (4.30846,7.80634);
\draw [color=c, fill=c] (4.30846,7.70049) rectangle (4.34826,7.80634);
\draw [color=c, fill=c] (4.34826,7.70049) rectangle (4.38806,7.80634);
\draw [color=c, fill=c] (4.38806,7.70049) rectangle (4.42786,7.80634);
\draw [color=c, fill=c] (4.42786,7.70049) rectangle (4.46766,7.80634);
\draw [color=c, fill=c] (4.46766,7.70049) rectangle (4.50746,7.80634);
\draw [color=c, fill=c] (4.50746,7.70049) rectangle (4.54726,7.80634);
\draw [color=c, fill=c] (4.54726,7.70049) rectangle (4.58706,7.80634);
\draw [color=c, fill=c] (4.58706,7.70049) rectangle (4.62687,7.80634);
\draw [color=c, fill=c] (4.62687,7.70049) rectangle (4.66667,7.80634);
\draw [color=c, fill=c] (4.66667,7.70049) rectangle (4.70647,7.80634);
\draw [color=c, fill=c] (4.70647,7.70049) rectangle (4.74627,7.80634);
\draw [color=c, fill=c] (4.74627,7.70049) rectangle (4.78607,7.80634);
\draw [color=c, fill=c] (4.78607,7.70049) rectangle (4.82587,7.80634);
\draw [color=c, fill=c] (4.82587,7.70049) rectangle (4.86567,7.80634);
\draw [color=c, fill=c] (4.86567,7.70049) rectangle (4.90547,7.80634);
\draw [color=c, fill=c] (4.90547,7.70049) rectangle (4.94527,7.80634);
\draw [color=c, fill=c] (4.94527,7.70049) rectangle (4.98507,7.80634);
\draw [color=c, fill=c] (4.98507,7.70049) rectangle (5.02488,7.80634);
\draw [color=c, fill=c] (5.02488,7.70049) rectangle (5.06468,7.80634);
\draw [color=c, fill=c] (5.06468,7.70049) rectangle (5.10448,7.80634);
\draw [color=c, fill=c] (5.10448,7.70049) rectangle (5.14428,7.80634);
\draw [color=c, fill=c] (5.14428,7.70049) rectangle (5.18408,7.80634);
\draw [color=c, fill=c] (5.18408,7.70049) rectangle (5.22388,7.80634);
\draw [color=c, fill=c] (5.22388,7.70049) rectangle (5.26368,7.80634);
\draw [color=c, fill=c] (5.26368,7.70049) rectangle (5.30348,7.80634);
\draw [color=c, fill=c] (5.30348,7.70049) rectangle (5.34328,7.80634);
\draw [color=c, fill=c] (5.34328,7.70049) rectangle (5.38308,7.80634);
\draw [color=c, fill=c] (5.38308,7.70049) rectangle (5.42289,7.80634);
\draw [color=c, fill=c] (5.42289,7.70049) rectangle (5.46269,7.80634);
\draw [color=c, fill=c] (5.46269,7.70049) rectangle (5.50249,7.80634);
\draw [color=c, fill=c] (5.50249,7.70049) rectangle (5.54229,7.80634);
\draw [color=c, fill=c] (5.54229,7.70049) rectangle (5.58209,7.80634);
\draw [color=c, fill=c] (5.58209,7.70049) rectangle (5.62189,7.80634);
\draw [color=c, fill=c] (5.62189,7.70049) rectangle (5.66169,7.80634);
\draw [color=c, fill=c] (5.66169,7.70049) rectangle (5.70149,7.80634);
\draw [color=c, fill=c] (5.70149,7.70049) rectangle (5.74129,7.80634);
\draw [color=c, fill=c] (5.74129,7.70049) rectangle (5.78109,7.80634);
\draw [color=c, fill=c] (5.78109,7.70049) rectangle (5.8209,7.80634);
\draw [color=c, fill=c] (5.8209,7.70049) rectangle (5.8607,7.80634);
\draw [color=c, fill=c] (5.8607,7.70049) rectangle (5.9005,7.80634);
\draw [color=c, fill=c] (5.9005,7.70049) rectangle (5.9403,7.80634);
\draw [color=c, fill=c] (5.9403,7.70049) rectangle (5.9801,7.80634);
\draw [color=c, fill=c] (5.9801,7.70049) rectangle (6.0199,7.80634);
\draw [color=c, fill=c] (6.0199,7.70049) rectangle (6.0597,7.80634);
\draw [color=c, fill=c] (6.0597,7.70049) rectangle (6.0995,7.80634);
\draw [color=c, fill=c] (6.0995,7.70049) rectangle (6.1393,7.80634);
\draw [color=c, fill=c] (6.1393,7.70049) rectangle (6.1791,7.80634);
\draw [color=c, fill=c] (6.1791,7.70049) rectangle (6.21891,7.80634);
\draw [color=c, fill=c] (6.21891,7.70049) rectangle (6.25871,7.80634);
\draw [color=c, fill=c] (6.25871,7.70049) rectangle (6.29851,7.80634);
\draw [color=c, fill=c] (6.29851,7.70049) rectangle (6.33831,7.80634);
\draw [color=c, fill=c] (6.33831,7.70049) rectangle (6.37811,7.80634);
\draw [color=c, fill=c] (6.37811,7.70049) rectangle (6.41791,7.80634);
\draw [color=c, fill=c] (6.41791,7.70049) rectangle (6.45771,7.80634);
\draw [color=c, fill=c] (6.45771,7.70049) rectangle (6.49751,7.80634);
\draw [color=c, fill=c] (6.49751,7.70049) rectangle (6.53731,7.80634);
\draw [color=c, fill=c] (6.53731,7.70049) rectangle (6.57711,7.80634);
\draw [color=c, fill=c] (6.57711,7.70049) rectangle (6.61692,7.80634);
\draw [color=c, fill=c] (6.61692,7.70049) rectangle (6.65672,7.80634);
\draw [color=c, fill=c] (6.65672,7.70049) rectangle (6.69652,7.80634);
\draw [color=c, fill=c] (6.69652,7.70049) rectangle (6.73632,7.80634);
\draw [color=c, fill=c] (6.73632,7.70049) rectangle (6.77612,7.80634);
\draw [color=c, fill=c] (6.77612,7.70049) rectangle (6.81592,7.80634);
\draw [color=c, fill=c] (6.81592,7.70049) rectangle (6.85572,7.80634);
\draw [color=c, fill=c] (6.85572,7.70049) rectangle (6.89552,7.80634);
\draw [color=c, fill=c] (6.89552,7.70049) rectangle (6.93532,7.80634);
\draw [color=c, fill=c] (6.93532,7.70049) rectangle (6.97512,7.80634);
\draw [color=c, fill=c] (6.97512,7.70049) rectangle (7.01493,7.80634);
\draw [color=c, fill=c] (7.01493,7.70049) rectangle (7.05473,7.80634);
\draw [color=c, fill=c] (7.05473,7.70049) rectangle (7.09453,7.80634);
\draw [color=c, fill=c] (7.09453,7.70049) rectangle (7.13433,7.80634);
\draw [color=c, fill=c] (7.13433,7.70049) rectangle (7.17413,7.80634);
\draw [color=c, fill=c] (7.17413,7.70049) rectangle (7.21393,7.80634);
\draw [color=c, fill=c] (7.21393,7.70049) rectangle (7.25373,7.80634);
\draw [color=c, fill=c] (7.25373,7.70049) rectangle (7.29353,7.80634);
\draw [color=c, fill=c] (7.29353,7.70049) rectangle (7.33333,7.80634);
\draw [color=c, fill=c] (7.33333,7.70049) rectangle (7.37313,7.80634);
\draw [color=c, fill=c] (7.37313,7.70049) rectangle (7.41294,7.80634);
\draw [color=c, fill=c] (7.41294,7.70049) rectangle (7.45274,7.80634);
\draw [color=c, fill=c] (7.45274,7.70049) rectangle (7.49254,7.80634);
\draw [color=c, fill=c] (7.49254,7.70049) rectangle (7.53234,7.80634);
\draw [color=c, fill=c] (7.53234,7.70049) rectangle (7.57214,7.80634);
\draw [color=c, fill=c] (7.57214,7.70049) rectangle (7.61194,7.80634);
\draw [color=c, fill=c] (7.61194,7.70049) rectangle (7.65174,7.80634);
\draw [color=c, fill=c] (7.65174,7.70049) rectangle (7.69154,7.80634);
\draw [color=c, fill=c] (7.69154,7.70049) rectangle (7.73134,7.80634);
\draw [color=c, fill=c] (7.73134,7.70049) rectangle (7.77114,7.80634);
\definecolor{c}{rgb}{0,0.0800001,1};
\draw [color=c, fill=c] (7.77114,7.70049) rectangle (7.81095,7.80634);
\draw [color=c, fill=c] (7.81095,7.70049) rectangle (7.85075,7.80634);
\draw [color=c, fill=c] (7.85075,7.70049) rectangle (7.89055,7.80634);
\draw [color=c, fill=c] (7.89055,7.70049) rectangle (7.93035,7.80634);
\draw [color=c, fill=c] (7.93035,7.70049) rectangle (7.97015,7.80634);
\draw [color=c, fill=c] (7.97015,7.70049) rectangle (8.00995,7.80634);
\draw [color=c, fill=c] (8.00995,7.70049) rectangle (8.04975,7.80634);
\draw [color=c, fill=c] (8.04975,7.70049) rectangle (8.08955,7.80634);
\draw [color=c, fill=c] (8.08955,7.70049) rectangle (8.12935,7.80634);
\draw [color=c, fill=c] (8.12935,7.70049) rectangle (8.16915,7.80634);
\draw [color=c, fill=c] (8.16915,7.70049) rectangle (8.20895,7.80634);
\draw [color=c, fill=c] (8.20895,7.70049) rectangle (8.24876,7.80634);
\draw [color=c, fill=c] (8.24876,7.70049) rectangle (8.28856,7.80634);
\draw [color=c, fill=c] (8.28856,7.70049) rectangle (8.32836,7.80634);
\draw [color=c, fill=c] (8.32836,7.70049) rectangle (8.36816,7.80634);
\draw [color=c, fill=c] (8.36816,7.70049) rectangle (8.40796,7.80634);
\draw [color=c, fill=c] (8.40796,7.70049) rectangle (8.44776,7.80634);
\draw [color=c, fill=c] (8.44776,7.70049) rectangle (8.48756,7.80634);
\draw [color=c, fill=c] (8.48756,7.70049) rectangle (8.52736,7.80634);
\draw [color=c, fill=c] (8.52736,7.70049) rectangle (8.56716,7.80634);
\draw [color=c, fill=c] (8.56716,7.70049) rectangle (8.60697,7.80634);
\draw [color=c, fill=c] (8.60697,7.70049) rectangle (8.64677,7.80634);
\draw [color=c, fill=c] (8.64677,7.70049) rectangle (8.68657,7.80634);
\draw [color=c, fill=c] (8.68657,7.70049) rectangle (8.72637,7.80634);
\draw [color=c, fill=c] (8.72637,7.70049) rectangle (8.76617,7.80634);
\draw [color=c, fill=c] (8.76617,7.70049) rectangle (8.80597,7.80634);
\draw [color=c, fill=c] (8.80597,7.70049) rectangle (8.84577,7.80634);
\draw [color=c, fill=c] (8.84577,7.70049) rectangle (8.88557,7.80634);
\draw [color=c, fill=c] (8.88557,7.70049) rectangle (8.92537,7.80634);
\draw [color=c, fill=c] (8.92537,7.70049) rectangle (8.96517,7.80634);
\draw [color=c, fill=c] (8.96517,7.70049) rectangle (9.00498,7.80634);
\draw [color=c, fill=c] (9.00498,7.70049) rectangle (9.04478,7.80634);
\draw [color=c, fill=c] (9.04478,7.70049) rectangle (9.08458,7.80634);
\draw [color=c, fill=c] (9.08458,7.70049) rectangle (9.12438,7.80634);
\draw [color=c, fill=c] (9.12438,7.70049) rectangle (9.16418,7.80634);
\draw [color=c, fill=c] (9.16418,7.70049) rectangle (9.20398,7.80634);
\draw [color=c, fill=c] (9.20398,7.70049) rectangle (9.24378,7.80634);
\draw [color=c, fill=c] (9.24378,7.70049) rectangle (9.28358,7.80634);
\draw [color=c, fill=c] (9.28358,7.70049) rectangle (9.32338,7.80634);
\draw [color=c, fill=c] (9.32338,7.70049) rectangle (9.36318,7.80634);
\draw [color=c, fill=c] (9.36318,7.70049) rectangle (9.40298,7.80634);
\draw [color=c, fill=c] (9.40298,7.70049) rectangle (9.44279,7.80634);
\draw [color=c, fill=c] (9.44279,7.70049) rectangle (9.48259,7.80634);
\draw [color=c, fill=c] (9.48259,7.70049) rectangle (9.52239,7.80634);
\definecolor{c}{rgb}{0,0.266667,1};
\draw [color=c, fill=c] (9.52239,7.70049) rectangle (9.56219,7.80634);
\draw [color=c, fill=c] (9.56219,7.70049) rectangle (9.60199,7.80634);
\draw [color=c, fill=c] (9.60199,7.70049) rectangle (9.64179,7.80634);
\draw [color=c, fill=c] (9.64179,7.70049) rectangle (9.68159,7.80634);
\draw [color=c, fill=c] (9.68159,7.70049) rectangle (9.72139,7.80634);
\draw [color=c, fill=c] (9.72139,7.70049) rectangle (9.76119,7.80634);
\draw [color=c, fill=c] (9.76119,7.70049) rectangle (9.80099,7.80634);
\draw [color=c, fill=c] (9.80099,7.70049) rectangle (9.8408,7.80634);
\draw [color=c, fill=c] (9.8408,7.70049) rectangle (9.8806,7.80634);
\draw [color=c, fill=c] (9.8806,7.70049) rectangle (9.9204,7.80634);
\draw [color=c, fill=c] (9.9204,7.70049) rectangle (9.9602,7.80634);
\draw [color=c, fill=c] (9.9602,7.70049) rectangle (10,7.80634);
\draw [color=c, fill=c] (10,7.70049) rectangle (10.0398,7.80634);
\draw [color=c, fill=c] (10.0398,7.70049) rectangle (10.0796,7.80634);
\draw [color=c, fill=c] (10.0796,7.70049) rectangle (10.1194,7.80634);
\draw [color=c, fill=c] (10.1194,7.70049) rectangle (10.1592,7.80634);
\draw [color=c, fill=c] (10.1592,7.70049) rectangle (10.199,7.80634);
\draw [color=c, fill=c] (10.199,7.70049) rectangle (10.2388,7.80634);
\draw [color=c, fill=c] (10.2388,7.70049) rectangle (10.2786,7.80634);
\draw [color=c, fill=c] (10.2786,7.70049) rectangle (10.3184,7.80634);
\draw [color=c, fill=c] (10.3184,7.70049) rectangle (10.3582,7.80634);
\definecolor{c}{rgb}{0,0.546666,1};
\draw [color=c, fill=c] (10.3582,7.70049) rectangle (10.398,7.80634);
\draw [color=c, fill=c] (10.398,7.70049) rectangle (10.4378,7.80634);
\draw [color=c, fill=c] (10.4378,7.70049) rectangle (10.4776,7.80634);
\draw [color=c, fill=c] (10.4776,7.70049) rectangle (10.5174,7.80634);
\draw [color=c, fill=c] (10.5174,7.70049) rectangle (10.5572,7.80634);
\draw [color=c, fill=c] (10.5572,7.70049) rectangle (10.597,7.80634);
\draw [color=c, fill=c] (10.597,7.70049) rectangle (10.6368,7.80634);
\draw [color=c, fill=c] (10.6368,7.70049) rectangle (10.6766,7.80634);
\draw [color=c, fill=c] (10.6766,7.70049) rectangle (10.7164,7.80634);
\draw [color=c, fill=c] (10.7164,7.70049) rectangle (10.7562,7.80634);
\draw [color=c, fill=c] (10.7562,7.70049) rectangle (10.796,7.80634);
\draw [color=c, fill=c] (10.796,7.70049) rectangle (10.8358,7.80634);
\draw [color=c, fill=c] (10.8358,7.70049) rectangle (10.8756,7.80634);
\draw [color=c, fill=c] (10.8756,7.70049) rectangle (10.9154,7.80634);
\draw [color=c, fill=c] (10.9154,7.70049) rectangle (10.9552,7.80634);
\draw [color=c, fill=c] (10.9552,7.70049) rectangle (10.995,7.80634);
\draw [color=c, fill=c] (10.995,7.70049) rectangle (11.0348,7.80634);
\draw [color=c, fill=c] (11.0348,7.70049) rectangle (11.0746,7.80634);
\draw [color=c, fill=c] (11.0746,7.70049) rectangle (11.1144,7.80634);
\draw [color=c, fill=c] (11.1144,7.70049) rectangle (11.1542,7.80634);
\draw [color=c, fill=c] (11.1542,7.70049) rectangle (11.194,7.80634);
\draw [color=c, fill=c] (11.194,7.70049) rectangle (11.2338,7.80634);
\draw [color=c, fill=c] (11.2338,7.70049) rectangle (11.2736,7.80634);
\draw [color=c, fill=c] (11.2736,7.70049) rectangle (11.3134,7.80634);
\draw [color=c, fill=c] (11.3134,7.70049) rectangle (11.3532,7.80634);
\draw [color=c, fill=c] (11.3532,7.70049) rectangle (11.393,7.80634);
\draw [color=c, fill=c] (11.393,7.70049) rectangle (11.4328,7.80634);
\draw [color=c, fill=c] (11.4328,7.70049) rectangle (11.4726,7.80634);
\draw [color=c, fill=c] (11.4726,7.70049) rectangle (11.5124,7.80634);
\draw [color=c, fill=c] (11.5124,7.70049) rectangle (11.5522,7.80634);
\draw [color=c, fill=c] (11.5522,7.70049) rectangle (11.592,7.80634);
\draw [color=c, fill=c] (11.592,7.70049) rectangle (11.6318,7.80634);
\draw [color=c, fill=c] (11.6318,7.70049) rectangle (11.6716,7.80634);
\draw [color=c, fill=c] (11.6716,7.70049) rectangle (11.7114,7.80634);
\draw [color=c, fill=c] (11.7114,7.70049) rectangle (11.7512,7.80634);
\draw [color=c, fill=c] (11.7512,7.70049) rectangle (11.791,7.80634);
\draw [color=c, fill=c] (11.791,7.70049) rectangle (11.8308,7.80634);
\draw [color=c, fill=c] (11.8308,7.70049) rectangle (11.8706,7.80634);
\draw [color=c, fill=c] (11.8706,7.70049) rectangle (11.9104,7.80634);
\draw [color=c, fill=c] (11.9104,7.70049) rectangle (11.9502,7.80634);
\draw [color=c, fill=c] (11.9502,7.70049) rectangle (11.99,7.80634);
\definecolor{c}{rgb}{0,0.733333,1};
\draw [color=c, fill=c] (11.99,7.70049) rectangle (12.0299,7.80634);
\draw [color=c, fill=c] (12.0299,7.70049) rectangle (12.0697,7.80634);
\draw [color=c, fill=c] (12.0697,7.70049) rectangle (12.1095,7.80634);
\draw [color=c, fill=c] (12.1095,7.70049) rectangle (12.1493,7.80634);
\draw [color=c, fill=c] (12.1493,7.70049) rectangle (12.1891,7.80634);
\draw [color=c, fill=c] (12.1891,7.70049) rectangle (12.2289,7.80634);
\draw [color=c, fill=c] (12.2289,7.70049) rectangle (12.2687,7.80634);
\draw [color=c, fill=c] (12.2687,7.70049) rectangle (12.3085,7.80634);
\draw [color=c, fill=c] (12.3085,7.70049) rectangle (12.3483,7.80634);
\draw [color=c, fill=c] (12.3483,7.70049) rectangle (12.3881,7.80634);
\draw [color=c, fill=c] (12.3881,7.70049) rectangle (12.4279,7.80634);
\draw [color=c, fill=c] (12.4279,7.70049) rectangle (12.4677,7.80634);
\draw [color=c, fill=c] (12.4677,7.70049) rectangle (12.5075,7.80634);
\draw [color=c, fill=c] (12.5075,7.70049) rectangle (12.5473,7.80634);
\draw [color=c, fill=c] (12.5473,7.70049) rectangle (12.5871,7.80634);
\draw [color=c, fill=c] (12.5871,7.70049) rectangle (12.6269,7.80634);
\draw [color=c, fill=c] (12.6269,7.70049) rectangle (12.6667,7.80634);
\draw [color=c, fill=c] (12.6667,7.70049) rectangle (12.7065,7.80634);
\draw [color=c, fill=c] (12.7065,7.70049) rectangle (12.7463,7.80634);
\draw [color=c, fill=c] (12.7463,7.70049) rectangle (12.7861,7.80634);
\draw [color=c, fill=c] (12.7861,7.70049) rectangle (12.8259,7.80634);
\draw [color=c, fill=c] (12.8259,7.70049) rectangle (12.8657,7.80634);
\draw [color=c, fill=c] (12.8657,7.70049) rectangle (12.9055,7.80634);
\draw [color=c, fill=c] (12.9055,7.70049) rectangle (12.9453,7.80634);
\draw [color=c, fill=c] (12.9453,7.70049) rectangle (12.9851,7.80634);
\draw [color=c, fill=c] (12.9851,7.70049) rectangle (13.0249,7.80634);
\draw [color=c, fill=c] (13.0249,7.70049) rectangle (13.0647,7.80634);
\draw [color=c, fill=c] (13.0647,7.70049) rectangle (13.1045,7.80634);
\draw [color=c, fill=c] (13.1045,7.70049) rectangle (13.1443,7.80634);
\draw [color=c, fill=c] (13.1443,7.70049) rectangle (13.1841,7.80634);
\draw [color=c, fill=c] (13.1841,7.70049) rectangle (13.2239,7.80634);
\draw [color=c, fill=c] (13.2239,7.70049) rectangle (13.2637,7.80634);
\draw [color=c, fill=c] (13.2637,7.70049) rectangle (13.3035,7.80634);
\draw [color=c, fill=c] (13.3035,7.70049) rectangle (13.3433,7.80634);
\draw [color=c, fill=c] (13.3433,7.70049) rectangle (13.3831,7.80634);
\draw [color=c, fill=c] (13.3831,7.70049) rectangle (13.4229,7.80634);
\draw [color=c, fill=c] (13.4229,7.70049) rectangle (13.4627,7.80634);
\draw [color=c, fill=c] (13.4627,7.70049) rectangle (13.5025,7.80634);
\draw [color=c, fill=c] (13.5025,7.70049) rectangle (13.5423,7.80634);
\draw [color=c, fill=c] (13.5423,7.70049) rectangle (13.5821,7.80634);
\draw [color=c, fill=c] (13.5821,7.70049) rectangle (13.6219,7.80634);
\draw [color=c, fill=c] (13.6219,7.70049) rectangle (13.6617,7.80634);
\draw [color=c, fill=c] (13.6617,7.70049) rectangle (13.7015,7.80634);
\draw [color=c, fill=c] (13.7015,7.70049) rectangle (13.7413,7.80634);
\draw [color=c, fill=c] (13.7413,7.70049) rectangle (13.7811,7.80634);
\draw [color=c, fill=c] (13.7811,7.70049) rectangle (13.8209,7.80634);
\draw [color=c, fill=c] (13.8209,7.70049) rectangle (13.8607,7.80634);
\draw [color=c, fill=c] (13.8607,7.70049) rectangle (13.9005,7.80634);
\draw [color=c, fill=c] (13.9005,7.70049) rectangle (13.9403,7.80634);
\draw [color=c, fill=c] (13.9403,7.70049) rectangle (13.9801,7.80634);
\draw [color=c, fill=c] (13.9801,7.70049) rectangle (14.0199,7.80634);
\draw [color=c, fill=c] (14.0199,7.70049) rectangle (14.0597,7.80634);
\draw [color=c, fill=c] (14.0597,7.70049) rectangle (14.0995,7.80634);
\draw [color=c, fill=c] (14.0995,7.70049) rectangle (14.1393,7.80634);
\draw [color=c, fill=c] (14.1393,7.70049) rectangle (14.1791,7.80634);
\draw [color=c, fill=c] (14.1791,7.70049) rectangle (14.2189,7.80634);
\draw [color=c, fill=c] (14.2189,7.70049) rectangle (14.2587,7.80634);
\draw [color=c, fill=c] (14.2587,7.70049) rectangle (14.2985,7.80634);
\draw [color=c, fill=c] (14.2985,7.70049) rectangle (14.3383,7.80634);
\draw [color=c, fill=c] (14.3383,7.70049) rectangle (14.3781,7.80634);
\draw [color=c, fill=c] (14.3781,7.70049) rectangle (14.4179,7.80634);
\draw [color=c, fill=c] (14.4179,7.70049) rectangle (14.4577,7.80634);
\draw [color=c, fill=c] (14.4577,7.70049) rectangle (14.4975,7.80634);
\draw [color=c, fill=c] (14.4975,7.70049) rectangle (14.5373,7.80634);
\draw [color=c, fill=c] (14.5373,7.70049) rectangle (14.5771,7.80634);
\draw [color=c, fill=c] (14.5771,7.70049) rectangle (14.6169,7.80634);
\draw [color=c, fill=c] (14.6169,7.70049) rectangle (14.6567,7.80634);
\draw [color=c, fill=c] (14.6567,7.70049) rectangle (14.6965,7.80634);
\draw [color=c, fill=c] (14.6965,7.70049) rectangle (14.7363,7.80634);
\draw [color=c, fill=c] (14.7363,7.70049) rectangle (14.7761,7.80634);
\draw [color=c, fill=c] (14.7761,7.70049) rectangle (14.8159,7.80634);
\draw [color=c, fill=c] (14.8159,7.70049) rectangle (14.8557,7.80634);
\draw [color=c, fill=c] (14.8557,7.70049) rectangle (14.8955,7.80634);
\draw [color=c, fill=c] (14.8955,7.70049) rectangle (14.9353,7.80634);
\draw [color=c, fill=c] (14.9353,7.70049) rectangle (14.9751,7.80634);
\draw [color=c, fill=c] (14.9751,7.70049) rectangle (15.0149,7.80634);
\draw [color=c, fill=c] (15.0149,7.70049) rectangle (15.0547,7.80634);
\draw [color=c, fill=c] (15.0547,7.70049) rectangle (15.0945,7.80634);
\draw [color=c, fill=c] (15.0945,7.70049) rectangle (15.1343,7.80634);
\draw [color=c, fill=c] (15.1343,7.70049) rectangle (15.1741,7.80634);
\draw [color=c, fill=c] (15.1741,7.70049) rectangle (15.2139,7.80634);
\draw [color=c, fill=c] (15.2139,7.70049) rectangle (15.2537,7.80634);
\draw [color=c, fill=c] (15.2537,7.70049) rectangle (15.2935,7.80634);
\draw [color=c, fill=c] (15.2935,7.70049) rectangle (15.3333,7.80634);
\draw [color=c, fill=c] (15.3333,7.70049) rectangle (15.3731,7.80634);
\draw [color=c, fill=c] (15.3731,7.70049) rectangle (15.4129,7.80634);
\draw [color=c, fill=c] (15.4129,7.70049) rectangle (15.4527,7.80634);
\draw [color=c, fill=c] (15.4527,7.70049) rectangle (15.4925,7.80634);
\draw [color=c, fill=c] (15.4925,7.70049) rectangle (15.5323,7.80634);
\draw [color=c, fill=c] (15.5323,7.70049) rectangle (15.5721,7.80634);
\draw [color=c, fill=c] (15.5721,7.70049) rectangle (15.6119,7.80634);
\draw [color=c, fill=c] (15.6119,7.70049) rectangle (15.6517,7.80634);
\draw [color=c, fill=c] (15.6517,7.70049) rectangle (15.6915,7.80634);
\draw [color=c, fill=c] (15.6915,7.70049) rectangle (15.7313,7.80634);
\draw [color=c, fill=c] (15.7313,7.70049) rectangle (15.7711,7.80634);
\draw [color=c, fill=c] (15.7711,7.70049) rectangle (15.8109,7.80634);
\draw [color=c, fill=c] (15.8109,7.70049) rectangle (15.8507,7.80634);
\draw [color=c, fill=c] (15.8507,7.70049) rectangle (15.8905,7.80634);
\draw [color=c, fill=c] (15.8905,7.70049) rectangle (15.9303,7.80634);
\draw [color=c, fill=c] (15.9303,7.70049) rectangle (15.9701,7.80634);
\draw [color=c, fill=c] (15.9701,7.70049) rectangle (16.01,7.80634);
\draw [color=c, fill=c] (16.01,7.70049) rectangle (16.0498,7.80634);
\draw [color=c, fill=c] (16.0498,7.70049) rectangle (16.0896,7.80634);
\draw [color=c, fill=c] (16.0896,7.70049) rectangle (16.1294,7.80634);
\draw [color=c, fill=c] (16.1294,7.70049) rectangle (16.1692,7.80634);
\draw [color=c, fill=c] (16.1692,7.70049) rectangle (16.209,7.80634);
\draw [color=c, fill=c] (16.209,7.70049) rectangle (16.2488,7.80634);
\draw [color=c, fill=c] (16.2488,7.70049) rectangle (16.2886,7.80634);
\draw [color=c, fill=c] (16.2886,7.70049) rectangle (16.3284,7.80634);
\draw [color=c, fill=c] (16.3284,7.70049) rectangle (16.3682,7.80634);
\draw [color=c, fill=c] (16.3682,7.70049) rectangle (16.408,7.80634);
\draw [color=c, fill=c] (16.408,7.70049) rectangle (16.4478,7.80634);
\draw [color=c, fill=c] (16.4478,7.70049) rectangle (16.4876,7.80634);
\draw [color=c, fill=c] (16.4876,7.70049) rectangle (16.5274,7.80634);
\draw [color=c, fill=c] (16.5274,7.70049) rectangle (16.5672,7.80634);
\draw [color=c, fill=c] (16.5672,7.70049) rectangle (16.607,7.80634);
\draw [color=c, fill=c] (16.607,7.70049) rectangle (16.6468,7.80634);
\draw [color=c, fill=c] (16.6468,7.70049) rectangle (16.6866,7.80634);
\draw [color=c, fill=c] (16.6866,7.70049) rectangle (16.7264,7.80634);
\draw [color=c, fill=c] (16.7264,7.70049) rectangle (16.7662,7.80634);
\draw [color=c, fill=c] (16.7662,7.70049) rectangle (16.806,7.80634);
\draw [color=c, fill=c] (16.806,7.70049) rectangle (16.8458,7.80634);
\draw [color=c, fill=c] (16.8458,7.70049) rectangle (16.8856,7.80634);
\draw [color=c, fill=c] (16.8856,7.70049) rectangle (16.9254,7.80634);
\draw [color=c, fill=c] (16.9254,7.70049) rectangle (16.9652,7.80634);
\draw [color=c, fill=c] (16.9652,7.70049) rectangle (17.005,7.80634);
\draw [color=c, fill=c] (17.005,7.70049) rectangle (17.0448,7.80634);
\draw [color=c, fill=c] (17.0448,7.70049) rectangle (17.0846,7.80634);
\draw [color=c, fill=c] (17.0846,7.70049) rectangle (17.1244,7.80634);
\draw [color=c, fill=c] (17.1244,7.70049) rectangle (17.1642,7.80634);
\draw [color=c, fill=c] (17.1642,7.70049) rectangle (17.204,7.80634);
\draw [color=c, fill=c] (17.204,7.70049) rectangle (17.2438,7.80634);
\draw [color=c, fill=c] (17.2438,7.70049) rectangle (17.2836,7.80634);
\draw [color=c, fill=c] (17.2836,7.70049) rectangle (17.3234,7.80634);
\draw [color=c, fill=c] (17.3234,7.70049) rectangle (17.3632,7.80634);
\draw [color=c, fill=c] (17.3632,7.70049) rectangle (17.403,7.80634);
\draw [color=c, fill=c] (17.403,7.70049) rectangle (17.4428,7.80634);
\draw [color=c, fill=c] (17.4428,7.70049) rectangle (17.4826,7.80634);
\draw [color=c, fill=c] (17.4826,7.70049) rectangle (17.5224,7.80634);
\draw [color=c, fill=c] (17.5224,7.70049) rectangle (17.5622,7.80634);
\draw [color=c, fill=c] (17.5622,7.70049) rectangle (17.602,7.80634);
\draw [color=c, fill=c] (17.602,7.70049) rectangle (17.6418,7.80634);
\draw [color=c, fill=c] (17.6418,7.70049) rectangle (17.6816,7.80634);
\draw [color=c, fill=c] (17.6816,7.70049) rectangle (17.7214,7.80634);
\draw [color=c, fill=c] (17.7214,7.70049) rectangle (17.7612,7.80634);
\draw [color=c, fill=c] (17.7612,7.70049) rectangle (17.801,7.80634);
\draw [color=c, fill=c] (17.801,7.70049) rectangle (17.8408,7.80634);
\draw [color=c, fill=c] (17.8408,7.70049) rectangle (17.8806,7.80634);
\draw [color=c, fill=c] (17.8806,7.70049) rectangle (17.9204,7.80634);
\draw [color=c, fill=c] (17.9204,7.70049) rectangle (17.9602,7.80634);
\draw [color=c, fill=c] (17.9602,7.70049) rectangle (18,7.80634);
\definecolor{c}{rgb}{0.2,0,1};
\draw [color=c, fill=c] (2,7.80634) rectangle (2.0398,7.91219);
\draw [color=c, fill=c] (2.0398,7.80634) rectangle (2.0796,7.91219);
\draw [color=c, fill=c] (2.0796,7.80634) rectangle (2.1194,7.91219);
\draw [color=c, fill=c] (2.1194,7.80634) rectangle (2.1592,7.91219);
\draw [color=c, fill=c] (2.1592,7.80634) rectangle (2.19901,7.91219);
\draw [color=c, fill=c] (2.19901,7.80634) rectangle (2.23881,7.91219);
\draw [color=c, fill=c] (2.23881,7.80634) rectangle (2.27861,7.91219);
\draw [color=c, fill=c] (2.27861,7.80634) rectangle (2.31841,7.91219);
\draw [color=c, fill=c] (2.31841,7.80634) rectangle (2.35821,7.91219);
\draw [color=c, fill=c] (2.35821,7.80634) rectangle (2.39801,7.91219);
\draw [color=c, fill=c] (2.39801,7.80634) rectangle (2.43781,7.91219);
\draw [color=c, fill=c] (2.43781,7.80634) rectangle (2.47761,7.91219);
\draw [color=c, fill=c] (2.47761,7.80634) rectangle (2.51741,7.91219);
\draw [color=c, fill=c] (2.51741,7.80634) rectangle (2.55721,7.91219);
\draw [color=c, fill=c] (2.55721,7.80634) rectangle (2.59702,7.91219);
\draw [color=c, fill=c] (2.59702,7.80634) rectangle (2.63682,7.91219);
\draw [color=c, fill=c] (2.63682,7.80634) rectangle (2.67662,7.91219);
\draw [color=c, fill=c] (2.67662,7.80634) rectangle (2.71642,7.91219);
\draw [color=c, fill=c] (2.71642,7.80634) rectangle (2.75622,7.91219);
\draw [color=c, fill=c] (2.75622,7.80634) rectangle (2.79602,7.91219);
\draw [color=c, fill=c] (2.79602,7.80634) rectangle (2.83582,7.91219);
\draw [color=c, fill=c] (2.83582,7.80634) rectangle (2.87562,7.91219);
\draw [color=c, fill=c] (2.87562,7.80634) rectangle (2.91542,7.91219);
\draw [color=c, fill=c] (2.91542,7.80634) rectangle (2.95522,7.91219);
\draw [color=c, fill=c] (2.95522,7.80634) rectangle (2.99502,7.91219);
\draw [color=c, fill=c] (2.99502,7.80634) rectangle (3.03483,7.91219);
\draw [color=c, fill=c] (3.03483,7.80634) rectangle (3.07463,7.91219);
\draw [color=c, fill=c] (3.07463,7.80634) rectangle (3.11443,7.91219);
\draw [color=c, fill=c] (3.11443,7.80634) rectangle (3.15423,7.91219);
\draw [color=c, fill=c] (3.15423,7.80634) rectangle (3.19403,7.91219);
\draw [color=c, fill=c] (3.19403,7.80634) rectangle (3.23383,7.91219);
\draw [color=c, fill=c] (3.23383,7.80634) rectangle (3.27363,7.91219);
\draw [color=c, fill=c] (3.27363,7.80634) rectangle (3.31343,7.91219);
\draw [color=c, fill=c] (3.31343,7.80634) rectangle (3.35323,7.91219);
\draw [color=c, fill=c] (3.35323,7.80634) rectangle (3.39303,7.91219);
\draw [color=c, fill=c] (3.39303,7.80634) rectangle (3.43284,7.91219);
\draw [color=c, fill=c] (3.43284,7.80634) rectangle (3.47264,7.91219);
\draw [color=c, fill=c] (3.47264,7.80634) rectangle (3.51244,7.91219);
\draw [color=c, fill=c] (3.51244,7.80634) rectangle (3.55224,7.91219);
\draw [color=c, fill=c] (3.55224,7.80634) rectangle (3.59204,7.91219);
\draw [color=c, fill=c] (3.59204,7.80634) rectangle (3.63184,7.91219);
\draw [color=c, fill=c] (3.63184,7.80634) rectangle (3.67164,7.91219);
\draw [color=c, fill=c] (3.67164,7.80634) rectangle (3.71144,7.91219);
\draw [color=c, fill=c] (3.71144,7.80634) rectangle (3.75124,7.91219);
\draw [color=c, fill=c] (3.75124,7.80634) rectangle (3.79104,7.91219);
\draw [color=c, fill=c] (3.79104,7.80634) rectangle (3.83085,7.91219);
\draw [color=c, fill=c] (3.83085,7.80634) rectangle (3.87065,7.91219);
\draw [color=c, fill=c] (3.87065,7.80634) rectangle (3.91045,7.91219);
\draw [color=c, fill=c] (3.91045,7.80634) rectangle (3.95025,7.91219);
\draw [color=c, fill=c] (3.95025,7.80634) rectangle (3.99005,7.91219);
\draw [color=c, fill=c] (3.99005,7.80634) rectangle (4.02985,7.91219);
\draw [color=c, fill=c] (4.02985,7.80634) rectangle (4.06965,7.91219);
\draw [color=c, fill=c] (4.06965,7.80634) rectangle (4.10945,7.91219);
\draw [color=c, fill=c] (4.10945,7.80634) rectangle (4.14925,7.91219);
\draw [color=c, fill=c] (4.14925,7.80634) rectangle (4.18905,7.91219);
\draw [color=c, fill=c] (4.18905,7.80634) rectangle (4.22886,7.91219);
\draw [color=c, fill=c] (4.22886,7.80634) rectangle (4.26866,7.91219);
\draw [color=c, fill=c] (4.26866,7.80634) rectangle (4.30846,7.91219);
\draw [color=c, fill=c] (4.30846,7.80634) rectangle (4.34826,7.91219);
\draw [color=c, fill=c] (4.34826,7.80634) rectangle (4.38806,7.91219);
\draw [color=c, fill=c] (4.38806,7.80634) rectangle (4.42786,7.91219);
\draw [color=c, fill=c] (4.42786,7.80634) rectangle (4.46766,7.91219);
\draw [color=c, fill=c] (4.46766,7.80634) rectangle (4.50746,7.91219);
\draw [color=c, fill=c] (4.50746,7.80634) rectangle (4.54726,7.91219);
\draw [color=c, fill=c] (4.54726,7.80634) rectangle (4.58706,7.91219);
\draw [color=c, fill=c] (4.58706,7.80634) rectangle (4.62687,7.91219);
\draw [color=c, fill=c] (4.62687,7.80634) rectangle (4.66667,7.91219);
\draw [color=c, fill=c] (4.66667,7.80634) rectangle (4.70647,7.91219);
\draw [color=c, fill=c] (4.70647,7.80634) rectangle (4.74627,7.91219);
\draw [color=c, fill=c] (4.74627,7.80634) rectangle (4.78607,7.91219);
\draw [color=c, fill=c] (4.78607,7.80634) rectangle (4.82587,7.91219);
\draw [color=c, fill=c] (4.82587,7.80634) rectangle (4.86567,7.91219);
\draw [color=c, fill=c] (4.86567,7.80634) rectangle (4.90547,7.91219);
\draw [color=c, fill=c] (4.90547,7.80634) rectangle (4.94527,7.91219);
\draw [color=c, fill=c] (4.94527,7.80634) rectangle (4.98507,7.91219);
\draw [color=c, fill=c] (4.98507,7.80634) rectangle (5.02488,7.91219);
\draw [color=c, fill=c] (5.02488,7.80634) rectangle (5.06468,7.91219);
\draw [color=c, fill=c] (5.06468,7.80634) rectangle (5.10448,7.91219);
\draw [color=c, fill=c] (5.10448,7.80634) rectangle (5.14428,7.91219);
\draw [color=c, fill=c] (5.14428,7.80634) rectangle (5.18408,7.91219);
\draw [color=c, fill=c] (5.18408,7.80634) rectangle (5.22388,7.91219);
\draw [color=c, fill=c] (5.22388,7.80634) rectangle (5.26368,7.91219);
\draw [color=c, fill=c] (5.26368,7.80634) rectangle (5.30348,7.91219);
\draw [color=c, fill=c] (5.30348,7.80634) rectangle (5.34328,7.91219);
\draw [color=c, fill=c] (5.34328,7.80634) rectangle (5.38308,7.91219);
\draw [color=c, fill=c] (5.38308,7.80634) rectangle (5.42289,7.91219);
\draw [color=c, fill=c] (5.42289,7.80634) rectangle (5.46269,7.91219);
\draw [color=c, fill=c] (5.46269,7.80634) rectangle (5.50249,7.91219);
\draw [color=c, fill=c] (5.50249,7.80634) rectangle (5.54229,7.91219);
\draw [color=c, fill=c] (5.54229,7.80634) rectangle (5.58209,7.91219);
\draw [color=c, fill=c] (5.58209,7.80634) rectangle (5.62189,7.91219);
\draw [color=c, fill=c] (5.62189,7.80634) rectangle (5.66169,7.91219);
\draw [color=c, fill=c] (5.66169,7.80634) rectangle (5.70149,7.91219);
\draw [color=c, fill=c] (5.70149,7.80634) rectangle (5.74129,7.91219);
\draw [color=c, fill=c] (5.74129,7.80634) rectangle (5.78109,7.91219);
\draw [color=c, fill=c] (5.78109,7.80634) rectangle (5.8209,7.91219);
\draw [color=c, fill=c] (5.8209,7.80634) rectangle (5.8607,7.91219);
\draw [color=c, fill=c] (5.8607,7.80634) rectangle (5.9005,7.91219);
\draw [color=c, fill=c] (5.9005,7.80634) rectangle (5.9403,7.91219);
\draw [color=c, fill=c] (5.9403,7.80634) rectangle (5.9801,7.91219);
\draw [color=c, fill=c] (5.9801,7.80634) rectangle (6.0199,7.91219);
\draw [color=c, fill=c] (6.0199,7.80634) rectangle (6.0597,7.91219);
\draw [color=c, fill=c] (6.0597,7.80634) rectangle (6.0995,7.91219);
\draw [color=c, fill=c] (6.0995,7.80634) rectangle (6.1393,7.91219);
\draw [color=c, fill=c] (6.1393,7.80634) rectangle (6.1791,7.91219);
\draw [color=c, fill=c] (6.1791,7.80634) rectangle (6.21891,7.91219);
\draw [color=c, fill=c] (6.21891,7.80634) rectangle (6.25871,7.91219);
\draw [color=c, fill=c] (6.25871,7.80634) rectangle (6.29851,7.91219);
\draw [color=c, fill=c] (6.29851,7.80634) rectangle (6.33831,7.91219);
\draw [color=c, fill=c] (6.33831,7.80634) rectangle (6.37811,7.91219);
\draw [color=c, fill=c] (6.37811,7.80634) rectangle (6.41791,7.91219);
\draw [color=c, fill=c] (6.41791,7.80634) rectangle (6.45771,7.91219);
\draw [color=c, fill=c] (6.45771,7.80634) rectangle (6.49751,7.91219);
\draw [color=c, fill=c] (6.49751,7.80634) rectangle (6.53731,7.91219);
\draw [color=c, fill=c] (6.53731,7.80634) rectangle (6.57711,7.91219);
\draw [color=c, fill=c] (6.57711,7.80634) rectangle (6.61692,7.91219);
\draw [color=c, fill=c] (6.61692,7.80634) rectangle (6.65672,7.91219);
\draw [color=c, fill=c] (6.65672,7.80634) rectangle (6.69652,7.91219);
\draw [color=c, fill=c] (6.69652,7.80634) rectangle (6.73632,7.91219);
\draw [color=c, fill=c] (6.73632,7.80634) rectangle (6.77612,7.91219);
\draw [color=c, fill=c] (6.77612,7.80634) rectangle (6.81592,7.91219);
\draw [color=c, fill=c] (6.81592,7.80634) rectangle (6.85572,7.91219);
\draw [color=c, fill=c] (6.85572,7.80634) rectangle (6.89552,7.91219);
\draw [color=c, fill=c] (6.89552,7.80634) rectangle (6.93532,7.91219);
\draw [color=c, fill=c] (6.93532,7.80634) rectangle (6.97512,7.91219);
\draw [color=c, fill=c] (6.97512,7.80634) rectangle (7.01493,7.91219);
\draw [color=c, fill=c] (7.01493,7.80634) rectangle (7.05473,7.91219);
\draw [color=c, fill=c] (7.05473,7.80634) rectangle (7.09453,7.91219);
\draw [color=c, fill=c] (7.09453,7.80634) rectangle (7.13433,7.91219);
\draw [color=c, fill=c] (7.13433,7.80634) rectangle (7.17413,7.91219);
\draw [color=c, fill=c] (7.17413,7.80634) rectangle (7.21393,7.91219);
\draw [color=c, fill=c] (7.21393,7.80634) rectangle (7.25373,7.91219);
\draw [color=c, fill=c] (7.25373,7.80634) rectangle (7.29353,7.91219);
\draw [color=c, fill=c] (7.29353,7.80634) rectangle (7.33333,7.91219);
\draw [color=c, fill=c] (7.33333,7.80634) rectangle (7.37313,7.91219);
\draw [color=c, fill=c] (7.37313,7.80634) rectangle (7.41294,7.91219);
\draw [color=c, fill=c] (7.41294,7.80634) rectangle (7.45274,7.91219);
\draw [color=c, fill=c] (7.45274,7.80634) rectangle (7.49254,7.91219);
\draw [color=c, fill=c] (7.49254,7.80634) rectangle (7.53234,7.91219);
\draw [color=c, fill=c] (7.53234,7.80634) rectangle (7.57214,7.91219);
\draw [color=c, fill=c] (7.57214,7.80634) rectangle (7.61194,7.91219);
\draw [color=c, fill=c] (7.61194,7.80634) rectangle (7.65174,7.91219);
\draw [color=c, fill=c] (7.65174,7.80634) rectangle (7.69154,7.91219);
\draw [color=c, fill=c] (7.69154,7.80634) rectangle (7.73134,7.91219);
\definecolor{c}{rgb}{0,0.0800001,1};
\draw [color=c, fill=c] (7.73134,7.80634) rectangle (7.77114,7.91219);
\draw [color=c, fill=c] (7.77114,7.80634) rectangle (7.81095,7.91219);
\draw [color=c, fill=c] (7.81095,7.80634) rectangle (7.85075,7.91219);
\draw [color=c, fill=c] (7.85075,7.80634) rectangle (7.89055,7.91219);
\draw [color=c, fill=c] (7.89055,7.80634) rectangle (7.93035,7.91219);
\draw [color=c, fill=c] (7.93035,7.80634) rectangle (7.97015,7.91219);
\draw [color=c, fill=c] (7.97015,7.80634) rectangle (8.00995,7.91219);
\draw [color=c, fill=c] (8.00995,7.80634) rectangle (8.04975,7.91219);
\draw [color=c, fill=c] (8.04975,7.80634) rectangle (8.08955,7.91219);
\draw [color=c, fill=c] (8.08955,7.80634) rectangle (8.12935,7.91219);
\draw [color=c, fill=c] (8.12935,7.80634) rectangle (8.16915,7.91219);
\draw [color=c, fill=c] (8.16915,7.80634) rectangle (8.20895,7.91219);
\draw [color=c, fill=c] (8.20895,7.80634) rectangle (8.24876,7.91219);
\draw [color=c, fill=c] (8.24876,7.80634) rectangle (8.28856,7.91219);
\draw [color=c, fill=c] (8.28856,7.80634) rectangle (8.32836,7.91219);
\draw [color=c, fill=c] (8.32836,7.80634) rectangle (8.36816,7.91219);
\draw [color=c, fill=c] (8.36816,7.80634) rectangle (8.40796,7.91219);
\draw [color=c, fill=c] (8.40796,7.80634) rectangle (8.44776,7.91219);
\draw [color=c, fill=c] (8.44776,7.80634) rectangle (8.48756,7.91219);
\draw [color=c, fill=c] (8.48756,7.80634) rectangle (8.52736,7.91219);
\draw [color=c, fill=c] (8.52736,7.80634) rectangle (8.56716,7.91219);
\draw [color=c, fill=c] (8.56716,7.80634) rectangle (8.60697,7.91219);
\draw [color=c, fill=c] (8.60697,7.80634) rectangle (8.64677,7.91219);
\draw [color=c, fill=c] (8.64677,7.80634) rectangle (8.68657,7.91219);
\draw [color=c, fill=c] (8.68657,7.80634) rectangle (8.72637,7.91219);
\draw [color=c, fill=c] (8.72637,7.80634) rectangle (8.76617,7.91219);
\draw [color=c, fill=c] (8.76617,7.80634) rectangle (8.80597,7.91219);
\draw [color=c, fill=c] (8.80597,7.80634) rectangle (8.84577,7.91219);
\draw [color=c, fill=c] (8.84577,7.80634) rectangle (8.88557,7.91219);
\draw [color=c, fill=c] (8.88557,7.80634) rectangle (8.92537,7.91219);
\draw [color=c, fill=c] (8.92537,7.80634) rectangle (8.96517,7.91219);
\draw [color=c, fill=c] (8.96517,7.80634) rectangle (9.00498,7.91219);
\draw [color=c, fill=c] (9.00498,7.80634) rectangle (9.04478,7.91219);
\draw [color=c, fill=c] (9.04478,7.80634) rectangle (9.08458,7.91219);
\draw [color=c, fill=c] (9.08458,7.80634) rectangle (9.12438,7.91219);
\draw [color=c, fill=c] (9.12438,7.80634) rectangle (9.16418,7.91219);
\draw [color=c, fill=c] (9.16418,7.80634) rectangle (9.20398,7.91219);
\draw [color=c, fill=c] (9.20398,7.80634) rectangle (9.24378,7.91219);
\draw [color=c, fill=c] (9.24378,7.80634) rectangle (9.28358,7.91219);
\draw [color=c, fill=c] (9.28358,7.80634) rectangle (9.32338,7.91219);
\draw [color=c, fill=c] (9.32338,7.80634) rectangle (9.36318,7.91219);
\draw [color=c, fill=c] (9.36318,7.80634) rectangle (9.40298,7.91219);
\draw [color=c, fill=c] (9.40298,7.80634) rectangle (9.44279,7.91219);
\draw [color=c, fill=c] (9.44279,7.80634) rectangle (9.48259,7.91219);
\draw [color=c, fill=c] (9.48259,7.80634) rectangle (9.52239,7.91219);
\definecolor{c}{rgb}{0,0.266667,1};
\draw [color=c, fill=c] (9.52239,7.80634) rectangle (9.56219,7.91219);
\draw [color=c, fill=c] (9.56219,7.80634) rectangle (9.60199,7.91219);
\draw [color=c, fill=c] (9.60199,7.80634) rectangle (9.64179,7.91219);
\draw [color=c, fill=c] (9.64179,7.80634) rectangle (9.68159,7.91219);
\draw [color=c, fill=c] (9.68159,7.80634) rectangle (9.72139,7.91219);
\draw [color=c, fill=c] (9.72139,7.80634) rectangle (9.76119,7.91219);
\draw [color=c, fill=c] (9.76119,7.80634) rectangle (9.80099,7.91219);
\draw [color=c, fill=c] (9.80099,7.80634) rectangle (9.8408,7.91219);
\draw [color=c, fill=c] (9.8408,7.80634) rectangle (9.8806,7.91219);
\draw [color=c, fill=c] (9.8806,7.80634) rectangle (9.9204,7.91219);
\draw [color=c, fill=c] (9.9204,7.80634) rectangle (9.9602,7.91219);
\draw [color=c, fill=c] (9.9602,7.80634) rectangle (10,7.91219);
\draw [color=c, fill=c] (10,7.80634) rectangle (10.0398,7.91219);
\draw [color=c, fill=c] (10.0398,7.80634) rectangle (10.0796,7.91219);
\draw [color=c, fill=c] (10.0796,7.80634) rectangle (10.1194,7.91219);
\draw [color=c, fill=c] (10.1194,7.80634) rectangle (10.1592,7.91219);
\draw [color=c, fill=c] (10.1592,7.80634) rectangle (10.199,7.91219);
\draw [color=c, fill=c] (10.199,7.80634) rectangle (10.2388,7.91219);
\draw [color=c, fill=c] (10.2388,7.80634) rectangle (10.2786,7.91219);
\draw [color=c, fill=c] (10.2786,7.80634) rectangle (10.3184,7.91219);
\draw [color=c, fill=c] (10.3184,7.80634) rectangle (10.3582,7.91219);
\definecolor{c}{rgb}{0,0.546666,1};
\draw [color=c, fill=c] (10.3582,7.80634) rectangle (10.398,7.91219);
\draw [color=c, fill=c] (10.398,7.80634) rectangle (10.4378,7.91219);
\draw [color=c, fill=c] (10.4378,7.80634) rectangle (10.4776,7.91219);
\draw [color=c, fill=c] (10.4776,7.80634) rectangle (10.5174,7.91219);
\draw [color=c, fill=c] (10.5174,7.80634) rectangle (10.5572,7.91219);
\draw [color=c, fill=c] (10.5572,7.80634) rectangle (10.597,7.91219);
\draw [color=c, fill=c] (10.597,7.80634) rectangle (10.6368,7.91219);
\draw [color=c, fill=c] (10.6368,7.80634) rectangle (10.6766,7.91219);
\draw [color=c, fill=c] (10.6766,7.80634) rectangle (10.7164,7.91219);
\draw [color=c, fill=c] (10.7164,7.80634) rectangle (10.7562,7.91219);
\draw [color=c, fill=c] (10.7562,7.80634) rectangle (10.796,7.91219);
\draw [color=c, fill=c] (10.796,7.80634) rectangle (10.8358,7.91219);
\draw [color=c, fill=c] (10.8358,7.80634) rectangle (10.8756,7.91219);
\draw [color=c, fill=c] (10.8756,7.80634) rectangle (10.9154,7.91219);
\draw [color=c, fill=c] (10.9154,7.80634) rectangle (10.9552,7.91219);
\draw [color=c, fill=c] (10.9552,7.80634) rectangle (10.995,7.91219);
\draw [color=c, fill=c] (10.995,7.80634) rectangle (11.0348,7.91219);
\draw [color=c, fill=c] (11.0348,7.80634) rectangle (11.0746,7.91219);
\draw [color=c, fill=c] (11.0746,7.80634) rectangle (11.1144,7.91219);
\draw [color=c, fill=c] (11.1144,7.80634) rectangle (11.1542,7.91219);
\draw [color=c, fill=c] (11.1542,7.80634) rectangle (11.194,7.91219);
\draw [color=c, fill=c] (11.194,7.80634) rectangle (11.2338,7.91219);
\draw [color=c, fill=c] (11.2338,7.80634) rectangle (11.2736,7.91219);
\draw [color=c, fill=c] (11.2736,7.80634) rectangle (11.3134,7.91219);
\draw [color=c, fill=c] (11.3134,7.80634) rectangle (11.3532,7.91219);
\draw [color=c, fill=c] (11.3532,7.80634) rectangle (11.393,7.91219);
\draw [color=c, fill=c] (11.393,7.80634) rectangle (11.4328,7.91219);
\draw [color=c, fill=c] (11.4328,7.80634) rectangle (11.4726,7.91219);
\draw [color=c, fill=c] (11.4726,7.80634) rectangle (11.5124,7.91219);
\draw [color=c, fill=c] (11.5124,7.80634) rectangle (11.5522,7.91219);
\draw [color=c, fill=c] (11.5522,7.80634) rectangle (11.592,7.91219);
\draw [color=c, fill=c] (11.592,7.80634) rectangle (11.6318,7.91219);
\draw [color=c, fill=c] (11.6318,7.80634) rectangle (11.6716,7.91219);
\draw [color=c, fill=c] (11.6716,7.80634) rectangle (11.7114,7.91219);
\draw [color=c, fill=c] (11.7114,7.80634) rectangle (11.7512,7.91219);
\draw [color=c, fill=c] (11.7512,7.80634) rectangle (11.791,7.91219);
\draw [color=c, fill=c] (11.791,7.80634) rectangle (11.8308,7.91219);
\draw [color=c, fill=c] (11.8308,7.80634) rectangle (11.8706,7.91219);
\draw [color=c, fill=c] (11.8706,7.80634) rectangle (11.9104,7.91219);
\draw [color=c, fill=c] (11.9104,7.80634) rectangle (11.9502,7.91219);
\draw [color=c, fill=c] (11.9502,7.80634) rectangle (11.99,7.91219);
\draw [color=c, fill=c] (11.99,7.80634) rectangle (12.0299,7.91219);
\draw [color=c, fill=c] (12.0299,7.80634) rectangle (12.0697,7.91219);
\draw [color=c, fill=c] (12.0697,7.80634) rectangle (12.1095,7.91219);
\definecolor{c}{rgb}{0,0.733333,1};
\draw [color=c, fill=c] (12.1095,7.80634) rectangle (12.1493,7.91219);
\draw [color=c, fill=c] (12.1493,7.80634) rectangle (12.1891,7.91219);
\draw [color=c, fill=c] (12.1891,7.80634) rectangle (12.2289,7.91219);
\draw [color=c, fill=c] (12.2289,7.80634) rectangle (12.2687,7.91219);
\draw [color=c, fill=c] (12.2687,7.80634) rectangle (12.3085,7.91219);
\draw [color=c, fill=c] (12.3085,7.80634) rectangle (12.3483,7.91219);
\draw [color=c, fill=c] (12.3483,7.80634) rectangle (12.3881,7.91219);
\draw [color=c, fill=c] (12.3881,7.80634) rectangle (12.4279,7.91219);
\draw [color=c, fill=c] (12.4279,7.80634) rectangle (12.4677,7.91219);
\draw [color=c, fill=c] (12.4677,7.80634) rectangle (12.5075,7.91219);
\draw [color=c, fill=c] (12.5075,7.80634) rectangle (12.5473,7.91219);
\draw [color=c, fill=c] (12.5473,7.80634) rectangle (12.5871,7.91219);
\draw [color=c, fill=c] (12.5871,7.80634) rectangle (12.6269,7.91219);
\draw [color=c, fill=c] (12.6269,7.80634) rectangle (12.6667,7.91219);
\draw [color=c, fill=c] (12.6667,7.80634) rectangle (12.7065,7.91219);
\draw [color=c, fill=c] (12.7065,7.80634) rectangle (12.7463,7.91219);
\draw [color=c, fill=c] (12.7463,7.80634) rectangle (12.7861,7.91219);
\draw [color=c, fill=c] (12.7861,7.80634) rectangle (12.8259,7.91219);
\draw [color=c, fill=c] (12.8259,7.80634) rectangle (12.8657,7.91219);
\draw [color=c, fill=c] (12.8657,7.80634) rectangle (12.9055,7.91219);
\draw [color=c, fill=c] (12.9055,7.80634) rectangle (12.9453,7.91219);
\draw [color=c, fill=c] (12.9453,7.80634) rectangle (12.9851,7.91219);
\draw [color=c, fill=c] (12.9851,7.80634) rectangle (13.0249,7.91219);
\draw [color=c, fill=c] (13.0249,7.80634) rectangle (13.0647,7.91219);
\draw [color=c, fill=c] (13.0647,7.80634) rectangle (13.1045,7.91219);
\draw [color=c, fill=c] (13.1045,7.80634) rectangle (13.1443,7.91219);
\draw [color=c, fill=c] (13.1443,7.80634) rectangle (13.1841,7.91219);
\draw [color=c, fill=c] (13.1841,7.80634) rectangle (13.2239,7.91219);
\draw [color=c, fill=c] (13.2239,7.80634) rectangle (13.2637,7.91219);
\draw [color=c, fill=c] (13.2637,7.80634) rectangle (13.3035,7.91219);
\draw [color=c, fill=c] (13.3035,7.80634) rectangle (13.3433,7.91219);
\draw [color=c, fill=c] (13.3433,7.80634) rectangle (13.3831,7.91219);
\draw [color=c, fill=c] (13.3831,7.80634) rectangle (13.4229,7.91219);
\draw [color=c, fill=c] (13.4229,7.80634) rectangle (13.4627,7.91219);
\draw [color=c, fill=c] (13.4627,7.80634) rectangle (13.5025,7.91219);
\draw [color=c, fill=c] (13.5025,7.80634) rectangle (13.5423,7.91219);
\draw [color=c, fill=c] (13.5423,7.80634) rectangle (13.5821,7.91219);
\draw [color=c, fill=c] (13.5821,7.80634) rectangle (13.6219,7.91219);
\draw [color=c, fill=c] (13.6219,7.80634) rectangle (13.6617,7.91219);
\draw [color=c, fill=c] (13.6617,7.80634) rectangle (13.7015,7.91219);
\draw [color=c, fill=c] (13.7015,7.80634) rectangle (13.7413,7.91219);
\draw [color=c, fill=c] (13.7413,7.80634) rectangle (13.7811,7.91219);
\draw [color=c, fill=c] (13.7811,7.80634) rectangle (13.8209,7.91219);
\draw [color=c, fill=c] (13.8209,7.80634) rectangle (13.8607,7.91219);
\draw [color=c, fill=c] (13.8607,7.80634) rectangle (13.9005,7.91219);
\draw [color=c, fill=c] (13.9005,7.80634) rectangle (13.9403,7.91219);
\draw [color=c, fill=c] (13.9403,7.80634) rectangle (13.9801,7.91219);
\draw [color=c, fill=c] (13.9801,7.80634) rectangle (14.0199,7.91219);
\draw [color=c, fill=c] (14.0199,7.80634) rectangle (14.0597,7.91219);
\draw [color=c, fill=c] (14.0597,7.80634) rectangle (14.0995,7.91219);
\draw [color=c, fill=c] (14.0995,7.80634) rectangle (14.1393,7.91219);
\draw [color=c, fill=c] (14.1393,7.80634) rectangle (14.1791,7.91219);
\draw [color=c, fill=c] (14.1791,7.80634) rectangle (14.2189,7.91219);
\draw [color=c, fill=c] (14.2189,7.80634) rectangle (14.2587,7.91219);
\draw [color=c, fill=c] (14.2587,7.80634) rectangle (14.2985,7.91219);
\draw [color=c, fill=c] (14.2985,7.80634) rectangle (14.3383,7.91219);
\draw [color=c, fill=c] (14.3383,7.80634) rectangle (14.3781,7.91219);
\draw [color=c, fill=c] (14.3781,7.80634) rectangle (14.4179,7.91219);
\draw [color=c, fill=c] (14.4179,7.80634) rectangle (14.4577,7.91219);
\draw [color=c, fill=c] (14.4577,7.80634) rectangle (14.4975,7.91219);
\draw [color=c, fill=c] (14.4975,7.80634) rectangle (14.5373,7.91219);
\draw [color=c, fill=c] (14.5373,7.80634) rectangle (14.5771,7.91219);
\draw [color=c, fill=c] (14.5771,7.80634) rectangle (14.6169,7.91219);
\draw [color=c, fill=c] (14.6169,7.80634) rectangle (14.6567,7.91219);
\draw [color=c, fill=c] (14.6567,7.80634) rectangle (14.6965,7.91219);
\draw [color=c, fill=c] (14.6965,7.80634) rectangle (14.7363,7.91219);
\draw [color=c, fill=c] (14.7363,7.80634) rectangle (14.7761,7.91219);
\draw [color=c, fill=c] (14.7761,7.80634) rectangle (14.8159,7.91219);
\draw [color=c, fill=c] (14.8159,7.80634) rectangle (14.8557,7.91219);
\draw [color=c, fill=c] (14.8557,7.80634) rectangle (14.8955,7.91219);
\draw [color=c, fill=c] (14.8955,7.80634) rectangle (14.9353,7.91219);
\draw [color=c, fill=c] (14.9353,7.80634) rectangle (14.9751,7.91219);
\draw [color=c, fill=c] (14.9751,7.80634) rectangle (15.0149,7.91219);
\draw [color=c, fill=c] (15.0149,7.80634) rectangle (15.0547,7.91219);
\draw [color=c, fill=c] (15.0547,7.80634) rectangle (15.0945,7.91219);
\draw [color=c, fill=c] (15.0945,7.80634) rectangle (15.1343,7.91219);
\draw [color=c, fill=c] (15.1343,7.80634) rectangle (15.1741,7.91219);
\draw [color=c, fill=c] (15.1741,7.80634) rectangle (15.2139,7.91219);
\draw [color=c, fill=c] (15.2139,7.80634) rectangle (15.2537,7.91219);
\draw [color=c, fill=c] (15.2537,7.80634) rectangle (15.2935,7.91219);
\draw [color=c, fill=c] (15.2935,7.80634) rectangle (15.3333,7.91219);
\draw [color=c, fill=c] (15.3333,7.80634) rectangle (15.3731,7.91219);
\draw [color=c, fill=c] (15.3731,7.80634) rectangle (15.4129,7.91219);
\draw [color=c, fill=c] (15.4129,7.80634) rectangle (15.4527,7.91219);
\draw [color=c, fill=c] (15.4527,7.80634) rectangle (15.4925,7.91219);
\draw [color=c, fill=c] (15.4925,7.80634) rectangle (15.5323,7.91219);
\draw [color=c, fill=c] (15.5323,7.80634) rectangle (15.5721,7.91219);
\draw [color=c, fill=c] (15.5721,7.80634) rectangle (15.6119,7.91219);
\draw [color=c, fill=c] (15.6119,7.80634) rectangle (15.6517,7.91219);
\draw [color=c, fill=c] (15.6517,7.80634) rectangle (15.6915,7.91219);
\draw [color=c, fill=c] (15.6915,7.80634) rectangle (15.7313,7.91219);
\draw [color=c, fill=c] (15.7313,7.80634) rectangle (15.7711,7.91219);
\draw [color=c, fill=c] (15.7711,7.80634) rectangle (15.8109,7.91219);
\draw [color=c, fill=c] (15.8109,7.80634) rectangle (15.8507,7.91219);
\draw [color=c, fill=c] (15.8507,7.80634) rectangle (15.8905,7.91219);
\draw [color=c, fill=c] (15.8905,7.80634) rectangle (15.9303,7.91219);
\draw [color=c, fill=c] (15.9303,7.80634) rectangle (15.9701,7.91219);
\draw [color=c, fill=c] (15.9701,7.80634) rectangle (16.01,7.91219);
\draw [color=c, fill=c] (16.01,7.80634) rectangle (16.0498,7.91219);
\draw [color=c, fill=c] (16.0498,7.80634) rectangle (16.0896,7.91219);
\draw [color=c, fill=c] (16.0896,7.80634) rectangle (16.1294,7.91219);
\draw [color=c, fill=c] (16.1294,7.80634) rectangle (16.1692,7.91219);
\draw [color=c, fill=c] (16.1692,7.80634) rectangle (16.209,7.91219);
\draw [color=c, fill=c] (16.209,7.80634) rectangle (16.2488,7.91219);
\draw [color=c, fill=c] (16.2488,7.80634) rectangle (16.2886,7.91219);
\draw [color=c, fill=c] (16.2886,7.80634) rectangle (16.3284,7.91219);
\draw [color=c, fill=c] (16.3284,7.80634) rectangle (16.3682,7.91219);
\draw [color=c, fill=c] (16.3682,7.80634) rectangle (16.408,7.91219);
\draw [color=c, fill=c] (16.408,7.80634) rectangle (16.4478,7.91219);
\draw [color=c, fill=c] (16.4478,7.80634) rectangle (16.4876,7.91219);
\draw [color=c, fill=c] (16.4876,7.80634) rectangle (16.5274,7.91219);
\draw [color=c, fill=c] (16.5274,7.80634) rectangle (16.5672,7.91219);
\draw [color=c, fill=c] (16.5672,7.80634) rectangle (16.607,7.91219);
\draw [color=c, fill=c] (16.607,7.80634) rectangle (16.6468,7.91219);
\draw [color=c, fill=c] (16.6468,7.80634) rectangle (16.6866,7.91219);
\draw [color=c, fill=c] (16.6866,7.80634) rectangle (16.7264,7.91219);
\draw [color=c, fill=c] (16.7264,7.80634) rectangle (16.7662,7.91219);
\draw [color=c, fill=c] (16.7662,7.80634) rectangle (16.806,7.91219);
\draw [color=c, fill=c] (16.806,7.80634) rectangle (16.8458,7.91219);
\draw [color=c, fill=c] (16.8458,7.80634) rectangle (16.8856,7.91219);
\draw [color=c, fill=c] (16.8856,7.80634) rectangle (16.9254,7.91219);
\draw [color=c, fill=c] (16.9254,7.80634) rectangle (16.9652,7.91219);
\draw [color=c, fill=c] (16.9652,7.80634) rectangle (17.005,7.91219);
\draw [color=c, fill=c] (17.005,7.80634) rectangle (17.0448,7.91219);
\draw [color=c, fill=c] (17.0448,7.80634) rectangle (17.0846,7.91219);
\draw [color=c, fill=c] (17.0846,7.80634) rectangle (17.1244,7.91219);
\draw [color=c, fill=c] (17.1244,7.80634) rectangle (17.1642,7.91219);
\draw [color=c, fill=c] (17.1642,7.80634) rectangle (17.204,7.91219);
\draw [color=c, fill=c] (17.204,7.80634) rectangle (17.2438,7.91219);
\draw [color=c, fill=c] (17.2438,7.80634) rectangle (17.2836,7.91219);
\draw [color=c, fill=c] (17.2836,7.80634) rectangle (17.3234,7.91219);
\draw [color=c, fill=c] (17.3234,7.80634) rectangle (17.3632,7.91219);
\draw [color=c, fill=c] (17.3632,7.80634) rectangle (17.403,7.91219);
\draw [color=c, fill=c] (17.403,7.80634) rectangle (17.4428,7.91219);
\draw [color=c, fill=c] (17.4428,7.80634) rectangle (17.4826,7.91219);
\draw [color=c, fill=c] (17.4826,7.80634) rectangle (17.5224,7.91219);
\draw [color=c, fill=c] (17.5224,7.80634) rectangle (17.5622,7.91219);
\draw [color=c, fill=c] (17.5622,7.80634) rectangle (17.602,7.91219);
\draw [color=c, fill=c] (17.602,7.80634) rectangle (17.6418,7.91219);
\draw [color=c, fill=c] (17.6418,7.80634) rectangle (17.6816,7.91219);
\draw [color=c, fill=c] (17.6816,7.80634) rectangle (17.7214,7.91219);
\draw [color=c, fill=c] (17.7214,7.80634) rectangle (17.7612,7.91219);
\draw [color=c, fill=c] (17.7612,7.80634) rectangle (17.801,7.91219);
\draw [color=c, fill=c] (17.801,7.80634) rectangle (17.8408,7.91219);
\draw [color=c, fill=c] (17.8408,7.80634) rectangle (17.8806,7.91219);
\draw [color=c, fill=c] (17.8806,7.80634) rectangle (17.9204,7.91219);
\draw [color=c, fill=c] (17.9204,7.80634) rectangle (17.9602,7.91219);
\draw [color=c, fill=c] (17.9602,7.80634) rectangle (18,7.91219);
\definecolor{c}{rgb}{0.2,0,1};
\draw [color=c, fill=c] (2,7.91219) rectangle (2.0398,8.01803);
\draw [color=c, fill=c] (2.0398,7.91219) rectangle (2.0796,8.01803);
\draw [color=c, fill=c] (2.0796,7.91219) rectangle (2.1194,8.01803);
\draw [color=c, fill=c] (2.1194,7.91219) rectangle (2.1592,8.01803);
\draw [color=c, fill=c] (2.1592,7.91219) rectangle (2.19901,8.01803);
\draw [color=c, fill=c] (2.19901,7.91219) rectangle (2.23881,8.01803);
\draw [color=c, fill=c] (2.23881,7.91219) rectangle (2.27861,8.01803);
\draw [color=c, fill=c] (2.27861,7.91219) rectangle (2.31841,8.01803);
\draw [color=c, fill=c] (2.31841,7.91219) rectangle (2.35821,8.01803);
\draw [color=c, fill=c] (2.35821,7.91219) rectangle (2.39801,8.01803);
\draw [color=c, fill=c] (2.39801,7.91219) rectangle (2.43781,8.01803);
\draw [color=c, fill=c] (2.43781,7.91219) rectangle (2.47761,8.01803);
\draw [color=c, fill=c] (2.47761,7.91219) rectangle (2.51741,8.01803);
\draw [color=c, fill=c] (2.51741,7.91219) rectangle (2.55721,8.01803);
\draw [color=c, fill=c] (2.55721,7.91219) rectangle (2.59702,8.01803);
\draw [color=c, fill=c] (2.59702,7.91219) rectangle (2.63682,8.01803);
\draw [color=c, fill=c] (2.63682,7.91219) rectangle (2.67662,8.01803);
\draw [color=c, fill=c] (2.67662,7.91219) rectangle (2.71642,8.01803);
\draw [color=c, fill=c] (2.71642,7.91219) rectangle (2.75622,8.01803);
\draw [color=c, fill=c] (2.75622,7.91219) rectangle (2.79602,8.01803);
\draw [color=c, fill=c] (2.79602,7.91219) rectangle (2.83582,8.01803);
\draw [color=c, fill=c] (2.83582,7.91219) rectangle (2.87562,8.01803);
\draw [color=c, fill=c] (2.87562,7.91219) rectangle (2.91542,8.01803);
\draw [color=c, fill=c] (2.91542,7.91219) rectangle (2.95522,8.01803);
\draw [color=c, fill=c] (2.95522,7.91219) rectangle (2.99502,8.01803);
\draw [color=c, fill=c] (2.99502,7.91219) rectangle (3.03483,8.01803);
\draw [color=c, fill=c] (3.03483,7.91219) rectangle (3.07463,8.01803);
\draw [color=c, fill=c] (3.07463,7.91219) rectangle (3.11443,8.01803);
\draw [color=c, fill=c] (3.11443,7.91219) rectangle (3.15423,8.01803);
\draw [color=c, fill=c] (3.15423,7.91219) rectangle (3.19403,8.01803);
\draw [color=c, fill=c] (3.19403,7.91219) rectangle (3.23383,8.01803);
\draw [color=c, fill=c] (3.23383,7.91219) rectangle (3.27363,8.01803);
\draw [color=c, fill=c] (3.27363,7.91219) rectangle (3.31343,8.01803);
\draw [color=c, fill=c] (3.31343,7.91219) rectangle (3.35323,8.01803);
\draw [color=c, fill=c] (3.35323,7.91219) rectangle (3.39303,8.01803);
\draw [color=c, fill=c] (3.39303,7.91219) rectangle (3.43284,8.01803);
\draw [color=c, fill=c] (3.43284,7.91219) rectangle (3.47264,8.01803);
\draw [color=c, fill=c] (3.47264,7.91219) rectangle (3.51244,8.01803);
\draw [color=c, fill=c] (3.51244,7.91219) rectangle (3.55224,8.01803);
\draw [color=c, fill=c] (3.55224,7.91219) rectangle (3.59204,8.01803);
\draw [color=c, fill=c] (3.59204,7.91219) rectangle (3.63184,8.01803);
\draw [color=c, fill=c] (3.63184,7.91219) rectangle (3.67164,8.01803);
\draw [color=c, fill=c] (3.67164,7.91219) rectangle (3.71144,8.01803);
\draw [color=c, fill=c] (3.71144,7.91219) rectangle (3.75124,8.01803);
\draw [color=c, fill=c] (3.75124,7.91219) rectangle (3.79104,8.01803);
\draw [color=c, fill=c] (3.79104,7.91219) rectangle (3.83085,8.01803);
\draw [color=c, fill=c] (3.83085,7.91219) rectangle (3.87065,8.01803);
\draw [color=c, fill=c] (3.87065,7.91219) rectangle (3.91045,8.01803);
\draw [color=c, fill=c] (3.91045,7.91219) rectangle (3.95025,8.01803);
\draw [color=c, fill=c] (3.95025,7.91219) rectangle (3.99005,8.01803);
\draw [color=c, fill=c] (3.99005,7.91219) rectangle (4.02985,8.01803);
\draw [color=c, fill=c] (4.02985,7.91219) rectangle (4.06965,8.01803);
\draw [color=c, fill=c] (4.06965,7.91219) rectangle (4.10945,8.01803);
\draw [color=c, fill=c] (4.10945,7.91219) rectangle (4.14925,8.01803);
\draw [color=c, fill=c] (4.14925,7.91219) rectangle (4.18905,8.01803);
\draw [color=c, fill=c] (4.18905,7.91219) rectangle (4.22886,8.01803);
\draw [color=c, fill=c] (4.22886,7.91219) rectangle (4.26866,8.01803);
\draw [color=c, fill=c] (4.26866,7.91219) rectangle (4.30846,8.01803);
\draw [color=c, fill=c] (4.30846,7.91219) rectangle (4.34826,8.01803);
\draw [color=c, fill=c] (4.34826,7.91219) rectangle (4.38806,8.01803);
\draw [color=c, fill=c] (4.38806,7.91219) rectangle (4.42786,8.01803);
\draw [color=c, fill=c] (4.42786,7.91219) rectangle (4.46766,8.01803);
\draw [color=c, fill=c] (4.46766,7.91219) rectangle (4.50746,8.01803);
\draw [color=c, fill=c] (4.50746,7.91219) rectangle (4.54726,8.01803);
\draw [color=c, fill=c] (4.54726,7.91219) rectangle (4.58706,8.01803);
\draw [color=c, fill=c] (4.58706,7.91219) rectangle (4.62687,8.01803);
\draw [color=c, fill=c] (4.62687,7.91219) rectangle (4.66667,8.01803);
\draw [color=c, fill=c] (4.66667,7.91219) rectangle (4.70647,8.01803);
\draw [color=c, fill=c] (4.70647,7.91219) rectangle (4.74627,8.01803);
\draw [color=c, fill=c] (4.74627,7.91219) rectangle (4.78607,8.01803);
\draw [color=c, fill=c] (4.78607,7.91219) rectangle (4.82587,8.01803);
\draw [color=c, fill=c] (4.82587,7.91219) rectangle (4.86567,8.01803);
\draw [color=c, fill=c] (4.86567,7.91219) rectangle (4.90547,8.01803);
\draw [color=c, fill=c] (4.90547,7.91219) rectangle (4.94527,8.01803);
\draw [color=c, fill=c] (4.94527,7.91219) rectangle (4.98507,8.01803);
\draw [color=c, fill=c] (4.98507,7.91219) rectangle (5.02488,8.01803);
\draw [color=c, fill=c] (5.02488,7.91219) rectangle (5.06468,8.01803);
\draw [color=c, fill=c] (5.06468,7.91219) rectangle (5.10448,8.01803);
\draw [color=c, fill=c] (5.10448,7.91219) rectangle (5.14428,8.01803);
\draw [color=c, fill=c] (5.14428,7.91219) rectangle (5.18408,8.01803);
\draw [color=c, fill=c] (5.18408,7.91219) rectangle (5.22388,8.01803);
\draw [color=c, fill=c] (5.22388,7.91219) rectangle (5.26368,8.01803);
\draw [color=c, fill=c] (5.26368,7.91219) rectangle (5.30348,8.01803);
\draw [color=c, fill=c] (5.30348,7.91219) rectangle (5.34328,8.01803);
\draw [color=c, fill=c] (5.34328,7.91219) rectangle (5.38308,8.01803);
\draw [color=c, fill=c] (5.38308,7.91219) rectangle (5.42289,8.01803);
\draw [color=c, fill=c] (5.42289,7.91219) rectangle (5.46269,8.01803);
\draw [color=c, fill=c] (5.46269,7.91219) rectangle (5.50249,8.01803);
\draw [color=c, fill=c] (5.50249,7.91219) rectangle (5.54229,8.01803);
\draw [color=c, fill=c] (5.54229,7.91219) rectangle (5.58209,8.01803);
\draw [color=c, fill=c] (5.58209,7.91219) rectangle (5.62189,8.01803);
\draw [color=c, fill=c] (5.62189,7.91219) rectangle (5.66169,8.01803);
\draw [color=c, fill=c] (5.66169,7.91219) rectangle (5.70149,8.01803);
\draw [color=c, fill=c] (5.70149,7.91219) rectangle (5.74129,8.01803);
\draw [color=c, fill=c] (5.74129,7.91219) rectangle (5.78109,8.01803);
\draw [color=c, fill=c] (5.78109,7.91219) rectangle (5.8209,8.01803);
\draw [color=c, fill=c] (5.8209,7.91219) rectangle (5.8607,8.01803);
\draw [color=c, fill=c] (5.8607,7.91219) rectangle (5.9005,8.01803);
\draw [color=c, fill=c] (5.9005,7.91219) rectangle (5.9403,8.01803);
\draw [color=c, fill=c] (5.9403,7.91219) rectangle (5.9801,8.01803);
\draw [color=c, fill=c] (5.9801,7.91219) rectangle (6.0199,8.01803);
\draw [color=c, fill=c] (6.0199,7.91219) rectangle (6.0597,8.01803);
\draw [color=c, fill=c] (6.0597,7.91219) rectangle (6.0995,8.01803);
\draw [color=c, fill=c] (6.0995,7.91219) rectangle (6.1393,8.01803);
\draw [color=c, fill=c] (6.1393,7.91219) rectangle (6.1791,8.01803);
\draw [color=c, fill=c] (6.1791,7.91219) rectangle (6.21891,8.01803);
\draw [color=c, fill=c] (6.21891,7.91219) rectangle (6.25871,8.01803);
\draw [color=c, fill=c] (6.25871,7.91219) rectangle (6.29851,8.01803);
\draw [color=c, fill=c] (6.29851,7.91219) rectangle (6.33831,8.01803);
\draw [color=c, fill=c] (6.33831,7.91219) rectangle (6.37811,8.01803);
\draw [color=c, fill=c] (6.37811,7.91219) rectangle (6.41791,8.01803);
\draw [color=c, fill=c] (6.41791,7.91219) rectangle (6.45771,8.01803);
\draw [color=c, fill=c] (6.45771,7.91219) rectangle (6.49751,8.01803);
\draw [color=c, fill=c] (6.49751,7.91219) rectangle (6.53731,8.01803);
\draw [color=c, fill=c] (6.53731,7.91219) rectangle (6.57711,8.01803);
\draw [color=c, fill=c] (6.57711,7.91219) rectangle (6.61692,8.01803);
\draw [color=c, fill=c] (6.61692,7.91219) rectangle (6.65672,8.01803);
\draw [color=c, fill=c] (6.65672,7.91219) rectangle (6.69652,8.01803);
\draw [color=c, fill=c] (6.69652,7.91219) rectangle (6.73632,8.01803);
\draw [color=c, fill=c] (6.73632,7.91219) rectangle (6.77612,8.01803);
\draw [color=c, fill=c] (6.77612,7.91219) rectangle (6.81592,8.01803);
\draw [color=c, fill=c] (6.81592,7.91219) rectangle (6.85572,8.01803);
\draw [color=c, fill=c] (6.85572,7.91219) rectangle (6.89552,8.01803);
\draw [color=c, fill=c] (6.89552,7.91219) rectangle (6.93532,8.01803);
\draw [color=c, fill=c] (6.93532,7.91219) rectangle (6.97512,8.01803);
\draw [color=c, fill=c] (6.97512,7.91219) rectangle (7.01493,8.01803);
\draw [color=c, fill=c] (7.01493,7.91219) rectangle (7.05473,8.01803);
\draw [color=c, fill=c] (7.05473,7.91219) rectangle (7.09453,8.01803);
\draw [color=c, fill=c] (7.09453,7.91219) rectangle (7.13433,8.01803);
\draw [color=c, fill=c] (7.13433,7.91219) rectangle (7.17413,8.01803);
\draw [color=c, fill=c] (7.17413,7.91219) rectangle (7.21393,8.01803);
\draw [color=c, fill=c] (7.21393,7.91219) rectangle (7.25373,8.01803);
\draw [color=c, fill=c] (7.25373,7.91219) rectangle (7.29353,8.01803);
\draw [color=c, fill=c] (7.29353,7.91219) rectangle (7.33333,8.01803);
\draw [color=c, fill=c] (7.33333,7.91219) rectangle (7.37313,8.01803);
\draw [color=c, fill=c] (7.37313,7.91219) rectangle (7.41294,8.01803);
\draw [color=c, fill=c] (7.41294,7.91219) rectangle (7.45274,8.01803);
\draw [color=c, fill=c] (7.45274,7.91219) rectangle (7.49254,8.01803);
\draw [color=c, fill=c] (7.49254,7.91219) rectangle (7.53234,8.01803);
\draw [color=c, fill=c] (7.53234,7.91219) rectangle (7.57214,8.01803);
\draw [color=c, fill=c] (7.57214,7.91219) rectangle (7.61194,8.01803);
\draw [color=c, fill=c] (7.61194,7.91219) rectangle (7.65174,8.01803);
\draw [color=c, fill=c] (7.65174,7.91219) rectangle (7.69154,8.01803);
\definecolor{c}{rgb}{0,0.0800001,1};
\draw [color=c, fill=c] (7.69154,7.91219) rectangle (7.73134,8.01803);
\draw [color=c, fill=c] (7.73134,7.91219) rectangle (7.77114,8.01803);
\draw [color=c, fill=c] (7.77114,7.91219) rectangle (7.81095,8.01803);
\draw [color=c, fill=c] (7.81095,7.91219) rectangle (7.85075,8.01803);
\draw [color=c, fill=c] (7.85075,7.91219) rectangle (7.89055,8.01803);
\draw [color=c, fill=c] (7.89055,7.91219) rectangle (7.93035,8.01803);
\draw [color=c, fill=c] (7.93035,7.91219) rectangle (7.97015,8.01803);
\draw [color=c, fill=c] (7.97015,7.91219) rectangle (8.00995,8.01803);
\draw [color=c, fill=c] (8.00995,7.91219) rectangle (8.04975,8.01803);
\draw [color=c, fill=c] (8.04975,7.91219) rectangle (8.08955,8.01803);
\draw [color=c, fill=c] (8.08955,7.91219) rectangle (8.12935,8.01803);
\draw [color=c, fill=c] (8.12935,7.91219) rectangle (8.16915,8.01803);
\draw [color=c, fill=c] (8.16915,7.91219) rectangle (8.20895,8.01803);
\draw [color=c, fill=c] (8.20895,7.91219) rectangle (8.24876,8.01803);
\draw [color=c, fill=c] (8.24876,7.91219) rectangle (8.28856,8.01803);
\draw [color=c, fill=c] (8.28856,7.91219) rectangle (8.32836,8.01803);
\draw [color=c, fill=c] (8.32836,7.91219) rectangle (8.36816,8.01803);
\draw [color=c, fill=c] (8.36816,7.91219) rectangle (8.40796,8.01803);
\draw [color=c, fill=c] (8.40796,7.91219) rectangle (8.44776,8.01803);
\draw [color=c, fill=c] (8.44776,7.91219) rectangle (8.48756,8.01803);
\draw [color=c, fill=c] (8.48756,7.91219) rectangle (8.52736,8.01803);
\draw [color=c, fill=c] (8.52736,7.91219) rectangle (8.56716,8.01803);
\draw [color=c, fill=c] (8.56716,7.91219) rectangle (8.60697,8.01803);
\draw [color=c, fill=c] (8.60697,7.91219) rectangle (8.64677,8.01803);
\draw [color=c, fill=c] (8.64677,7.91219) rectangle (8.68657,8.01803);
\draw [color=c, fill=c] (8.68657,7.91219) rectangle (8.72637,8.01803);
\draw [color=c, fill=c] (8.72637,7.91219) rectangle (8.76617,8.01803);
\draw [color=c, fill=c] (8.76617,7.91219) rectangle (8.80597,8.01803);
\draw [color=c, fill=c] (8.80597,7.91219) rectangle (8.84577,8.01803);
\draw [color=c, fill=c] (8.84577,7.91219) rectangle (8.88557,8.01803);
\draw [color=c, fill=c] (8.88557,7.91219) rectangle (8.92537,8.01803);
\draw [color=c, fill=c] (8.92537,7.91219) rectangle (8.96517,8.01803);
\draw [color=c, fill=c] (8.96517,7.91219) rectangle (9.00498,8.01803);
\draw [color=c, fill=c] (9.00498,7.91219) rectangle (9.04478,8.01803);
\draw [color=c, fill=c] (9.04478,7.91219) rectangle (9.08458,8.01803);
\draw [color=c, fill=c] (9.08458,7.91219) rectangle (9.12438,8.01803);
\draw [color=c, fill=c] (9.12438,7.91219) rectangle (9.16418,8.01803);
\draw [color=c, fill=c] (9.16418,7.91219) rectangle (9.20398,8.01803);
\draw [color=c, fill=c] (9.20398,7.91219) rectangle (9.24378,8.01803);
\draw [color=c, fill=c] (9.24378,7.91219) rectangle (9.28358,8.01803);
\draw [color=c, fill=c] (9.28358,7.91219) rectangle (9.32338,8.01803);
\draw [color=c, fill=c] (9.32338,7.91219) rectangle (9.36318,8.01803);
\draw [color=c, fill=c] (9.36318,7.91219) rectangle (9.40298,8.01803);
\draw [color=c, fill=c] (9.40298,7.91219) rectangle (9.44279,8.01803);
\draw [color=c, fill=c] (9.44279,7.91219) rectangle (9.48259,8.01803);
\draw [color=c, fill=c] (9.48259,7.91219) rectangle (9.52239,8.01803);
\definecolor{c}{rgb}{0,0.266667,1};
\draw [color=c, fill=c] (9.52239,7.91219) rectangle (9.56219,8.01803);
\draw [color=c, fill=c] (9.56219,7.91219) rectangle (9.60199,8.01803);
\draw [color=c, fill=c] (9.60199,7.91219) rectangle (9.64179,8.01803);
\draw [color=c, fill=c] (9.64179,7.91219) rectangle (9.68159,8.01803);
\draw [color=c, fill=c] (9.68159,7.91219) rectangle (9.72139,8.01803);
\draw [color=c, fill=c] (9.72139,7.91219) rectangle (9.76119,8.01803);
\draw [color=c, fill=c] (9.76119,7.91219) rectangle (9.80099,8.01803);
\draw [color=c, fill=c] (9.80099,7.91219) rectangle (9.8408,8.01803);
\draw [color=c, fill=c] (9.8408,7.91219) rectangle (9.8806,8.01803);
\draw [color=c, fill=c] (9.8806,7.91219) rectangle (9.9204,8.01803);
\draw [color=c, fill=c] (9.9204,7.91219) rectangle (9.9602,8.01803);
\draw [color=c, fill=c] (9.9602,7.91219) rectangle (10,8.01803);
\draw [color=c, fill=c] (10,7.91219) rectangle (10.0398,8.01803);
\draw [color=c, fill=c] (10.0398,7.91219) rectangle (10.0796,8.01803);
\draw [color=c, fill=c] (10.0796,7.91219) rectangle (10.1194,8.01803);
\draw [color=c, fill=c] (10.1194,7.91219) rectangle (10.1592,8.01803);
\draw [color=c, fill=c] (10.1592,7.91219) rectangle (10.199,8.01803);
\draw [color=c, fill=c] (10.199,7.91219) rectangle (10.2388,8.01803);
\draw [color=c, fill=c] (10.2388,7.91219) rectangle (10.2786,8.01803);
\draw [color=c, fill=c] (10.2786,7.91219) rectangle (10.3184,8.01803);
\draw [color=c, fill=c] (10.3184,7.91219) rectangle (10.3582,8.01803);
\draw [color=c, fill=c] (10.3582,7.91219) rectangle (10.398,8.01803);
\definecolor{c}{rgb}{0,0.546666,1};
\draw [color=c, fill=c] (10.398,7.91219) rectangle (10.4378,8.01803);
\draw [color=c, fill=c] (10.4378,7.91219) rectangle (10.4776,8.01803);
\draw [color=c, fill=c] (10.4776,7.91219) rectangle (10.5174,8.01803);
\draw [color=c, fill=c] (10.5174,7.91219) rectangle (10.5572,8.01803);
\draw [color=c, fill=c] (10.5572,7.91219) rectangle (10.597,8.01803);
\draw [color=c, fill=c] (10.597,7.91219) rectangle (10.6368,8.01803);
\draw [color=c, fill=c] (10.6368,7.91219) rectangle (10.6766,8.01803);
\draw [color=c, fill=c] (10.6766,7.91219) rectangle (10.7164,8.01803);
\draw [color=c, fill=c] (10.7164,7.91219) rectangle (10.7562,8.01803);
\draw [color=c, fill=c] (10.7562,7.91219) rectangle (10.796,8.01803);
\draw [color=c, fill=c] (10.796,7.91219) rectangle (10.8358,8.01803);
\draw [color=c, fill=c] (10.8358,7.91219) rectangle (10.8756,8.01803);
\draw [color=c, fill=c] (10.8756,7.91219) rectangle (10.9154,8.01803);
\draw [color=c, fill=c] (10.9154,7.91219) rectangle (10.9552,8.01803);
\draw [color=c, fill=c] (10.9552,7.91219) rectangle (10.995,8.01803);
\draw [color=c, fill=c] (10.995,7.91219) rectangle (11.0348,8.01803);
\draw [color=c, fill=c] (11.0348,7.91219) rectangle (11.0746,8.01803);
\draw [color=c, fill=c] (11.0746,7.91219) rectangle (11.1144,8.01803);
\draw [color=c, fill=c] (11.1144,7.91219) rectangle (11.1542,8.01803);
\draw [color=c, fill=c] (11.1542,7.91219) rectangle (11.194,8.01803);
\draw [color=c, fill=c] (11.194,7.91219) rectangle (11.2338,8.01803);
\draw [color=c, fill=c] (11.2338,7.91219) rectangle (11.2736,8.01803);
\draw [color=c, fill=c] (11.2736,7.91219) rectangle (11.3134,8.01803);
\draw [color=c, fill=c] (11.3134,7.91219) rectangle (11.3532,8.01803);
\draw [color=c, fill=c] (11.3532,7.91219) rectangle (11.393,8.01803);
\draw [color=c, fill=c] (11.393,7.91219) rectangle (11.4328,8.01803);
\draw [color=c, fill=c] (11.4328,7.91219) rectangle (11.4726,8.01803);
\draw [color=c, fill=c] (11.4726,7.91219) rectangle (11.5124,8.01803);
\draw [color=c, fill=c] (11.5124,7.91219) rectangle (11.5522,8.01803);
\draw [color=c, fill=c] (11.5522,7.91219) rectangle (11.592,8.01803);
\draw [color=c, fill=c] (11.592,7.91219) rectangle (11.6318,8.01803);
\draw [color=c, fill=c] (11.6318,7.91219) rectangle (11.6716,8.01803);
\draw [color=c, fill=c] (11.6716,7.91219) rectangle (11.7114,8.01803);
\draw [color=c, fill=c] (11.7114,7.91219) rectangle (11.7512,8.01803);
\draw [color=c, fill=c] (11.7512,7.91219) rectangle (11.791,8.01803);
\draw [color=c, fill=c] (11.791,7.91219) rectangle (11.8308,8.01803);
\draw [color=c, fill=c] (11.8308,7.91219) rectangle (11.8706,8.01803);
\draw [color=c, fill=c] (11.8706,7.91219) rectangle (11.9104,8.01803);
\draw [color=c, fill=c] (11.9104,7.91219) rectangle (11.9502,8.01803);
\draw [color=c, fill=c] (11.9502,7.91219) rectangle (11.99,8.01803);
\draw [color=c, fill=c] (11.99,7.91219) rectangle (12.0299,8.01803);
\draw [color=c, fill=c] (12.0299,7.91219) rectangle (12.0697,8.01803);
\draw [color=c, fill=c] (12.0697,7.91219) rectangle (12.1095,8.01803);
\draw [color=c, fill=c] (12.1095,7.91219) rectangle (12.1493,8.01803);
\draw [color=c, fill=c] (12.1493,7.91219) rectangle (12.1891,8.01803);
\definecolor{c}{rgb}{0,0.733333,1};
\draw [color=c, fill=c] (12.1891,7.91219) rectangle (12.2289,8.01803);
\draw [color=c, fill=c] (12.2289,7.91219) rectangle (12.2687,8.01803);
\draw [color=c, fill=c] (12.2687,7.91219) rectangle (12.3085,8.01803);
\draw [color=c, fill=c] (12.3085,7.91219) rectangle (12.3483,8.01803);
\draw [color=c, fill=c] (12.3483,7.91219) rectangle (12.3881,8.01803);
\draw [color=c, fill=c] (12.3881,7.91219) rectangle (12.4279,8.01803);
\draw [color=c, fill=c] (12.4279,7.91219) rectangle (12.4677,8.01803);
\draw [color=c, fill=c] (12.4677,7.91219) rectangle (12.5075,8.01803);
\draw [color=c, fill=c] (12.5075,7.91219) rectangle (12.5473,8.01803);
\draw [color=c, fill=c] (12.5473,7.91219) rectangle (12.5871,8.01803);
\draw [color=c, fill=c] (12.5871,7.91219) rectangle (12.6269,8.01803);
\draw [color=c, fill=c] (12.6269,7.91219) rectangle (12.6667,8.01803);
\draw [color=c, fill=c] (12.6667,7.91219) rectangle (12.7065,8.01803);
\draw [color=c, fill=c] (12.7065,7.91219) rectangle (12.7463,8.01803);
\draw [color=c, fill=c] (12.7463,7.91219) rectangle (12.7861,8.01803);
\draw [color=c, fill=c] (12.7861,7.91219) rectangle (12.8259,8.01803);
\draw [color=c, fill=c] (12.8259,7.91219) rectangle (12.8657,8.01803);
\draw [color=c, fill=c] (12.8657,7.91219) rectangle (12.9055,8.01803);
\draw [color=c, fill=c] (12.9055,7.91219) rectangle (12.9453,8.01803);
\draw [color=c, fill=c] (12.9453,7.91219) rectangle (12.9851,8.01803);
\draw [color=c, fill=c] (12.9851,7.91219) rectangle (13.0249,8.01803);
\draw [color=c, fill=c] (13.0249,7.91219) rectangle (13.0647,8.01803);
\draw [color=c, fill=c] (13.0647,7.91219) rectangle (13.1045,8.01803);
\draw [color=c, fill=c] (13.1045,7.91219) rectangle (13.1443,8.01803);
\draw [color=c, fill=c] (13.1443,7.91219) rectangle (13.1841,8.01803);
\draw [color=c, fill=c] (13.1841,7.91219) rectangle (13.2239,8.01803);
\draw [color=c, fill=c] (13.2239,7.91219) rectangle (13.2637,8.01803);
\draw [color=c, fill=c] (13.2637,7.91219) rectangle (13.3035,8.01803);
\draw [color=c, fill=c] (13.3035,7.91219) rectangle (13.3433,8.01803);
\draw [color=c, fill=c] (13.3433,7.91219) rectangle (13.3831,8.01803);
\draw [color=c, fill=c] (13.3831,7.91219) rectangle (13.4229,8.01803);
\draw [color=c, fill=c] (13.4229,7.91219) rectangle (13.4627,8.01803);
\draw [color=c, fill=c] (13.4627,7.91219) rectangle (13.5025,8.01803);
\draw [color=c, fill=c] (13.5025,7.91219) rectangle (13.5423,8.01803);
\draw [color=c, fill=c] (13.5423,7.91219) rectangle (13.5821,8.01803);
\draw [color=c, fill=c] (13.5821,7.91219) rectangle (13.6219,8.01803);
\draw [color=c, fill=c] (13.6219,7.91219) rectangle (13.6617,8.01803);
\draw [color=c, fill=c] (13.6617,7.91219) rectangle (13.7015,8.01803);
\draw [color=c, fill=c] (13.7015,7.91219) rectangle (13.7413,8.01803);
\draw [color=c, fill=c] (13.7413,7.91219) rectangle (13.7811,8.01803);
\draw [color=c, fill=c] (13.7811,7.91219) rectangle (13.8209,8.01803);
\draw [color=c, fill=c] (13.8209,7.91219) rectangle (13.8607,8.01803);
\draw [color=c, fill=c] (13.8607,7.91219) rectangle (13.9005,8.01803);
\draw [color=c, fill=c] (13.9005,7.91219) rectangle (13.9403,8.01803);
\draw [color=c, fill=c] (13.9403,7.91219) rectangle (13.9801,8.01803);
\draw [color=c, fill=c] (13.9801,7.91219) rectangle (14.0199,8.01803);
\draw [color=c, fill=c] (14.0199,7.91219) rectangle (14.0597,8.01803);
\draw [color=c, fill=c] (14.0597,7.91219) rectangle (14.0995,8.01803);
\draw [color=c, fill=c] (14.0995,7.91219) rectangle (14.1393,8.01803);
\draw [color=c, fill=c] (14.1393,7.91219) rectangle (14.1791,8.01803);
\draw [color=c, fill=c] (14.1791,7.91219) rectangle (14.2189,8.01803);
\draw [color=c, fill=c] (14.2189,7.91219) rectangle (14.2587,8.01803);
\draw [color=c, fill=c] (14.2587,7.91219) rectangle (14.2985,8.01803);
\draw [color=c, fill=c] (14.2985,7.91219) rectangle (14.3383,8.01803);
\draw [color=c, fill=c] (14.3383,7.91219) rectangle (14.3781,8.01803);
\draw [color=c, fill=c] (14.3781,7.91219) rectangle (14.4179,8.01803);
\draw [color=c, fill=c] (14.4179,7.91219) rectangle (14.4577,8.01803);
\draw [color=c, fill=c] (14.4577,7.91219) rectangle (14.4975,8.01803);
\draw [color=c, fill=c] (14.4975,7.91219) rectangle (14.5373,8.01803);
\draw [color=c, fill=c] (14.5373,7.91219) rectangle (14.5771,8.01803);
\draw [color=c, fill=c] (14.5771,7.91219) rectangle (14.6169,8.01803);
\draw [color=c, fill=c] (14.6169,7.91219) rectangle (14.6567,8.01803);
\draw [color=c, fill=c] (14.6567,7.91219) rectangle (14.6965,8.01803);
\draw [color=c, fill=c] (14.6965,7.91219) rectangle (14.7363,8.01803);
\draw [color=c, fill=c] (14.7363,7.91219) rectangle (14.7761,8.01803);
\draw [color=c, fill=c] (14.7761,7.91219) rectangle (14.8159,8.01803);
\draw [color=c, fill=c] (14.8159,7.91219) rectangle (14.8557,8.01803);
\draw [color=c, fill=c] (14.8557,7.91219) rectangle (14.8955,8.01803);
\draw [color=c, fill=c] (14.8955,7.91219) rectangle (14.9353,8.01803);
\draw [color=c, fill=c] (14.9353,7.91219) rectangle (14.9751,8.01803);
\draw [color=c, fill=c] (14.9751,7.91219) rectangle (15.0149,8.01803);
\draw [color=c, fill=c] (15.0149,7.91219) rectangle (15.0547,8.01803);
\draw [color=c, fill=c] (15.0547,7.91219) rectangle (15.0945,8.01803);
\draw [color=c, fill=c] (15.0945,7.91219) rectangle (15.1343,8.01803);
\draw [color=c, fill=c] (15.1343,7.91219) rectangle (15.1741,8.01803);
\draw [color=c, fill=c] (15.1741,7.91219) rectangle (15.2139,8.01803);
\draw [color=c, fill=c] (15.2139,7.91219) rectangle (15.2537,8.01803);
\draw [color=c, fill=c] (15.2537,7.91219) rectangle (15.2935,8.01803);
\draw [color=c, fill=c] (15.2935,7.91219) rectangle (15.3333,8.01803);
\draw [color=c, fill=c] (15.3333,7.91219) rectangle (15.3731,8.01803);
\draw [color=c, fill=c] (15.3731,7.91219) rectangle (15.4129,8.01803);
\draw [color=c, fill=c] (15.4129,7.91219) rectangle (15.4527,8.01803);
\draw [color=c, fill=c] (15.4527,7.91219) rectangle (15.4925,8.01803);
\draw [color=c, fill=c] (15.4925,7.91219) rectangle (15.5323,8.01803);
\draw [color=c, fill=c] (15.5323,7.91219) rectangle (15.5721,8.01803);
\draw [color=c, fill=c] (15.5721,7.91219) rectangle (15.6119,8.01803);
\draw [color=c, fill=c] (15.6119,7.91219) rectangle (15.6517,8.01803);
\draw [color=c, fill=c] (15.6517,7.91219) rectangle (15.6915,8.01803);
\draw [color=c, fill=c] (15.6915,7.91219) rectangle (15.7313,8.01803);
\draw [color=c, fill=c] (15.7313,7.91219) rectangle (15.7711,8.01803);
\draw [color=c, fill=c] (15.7711,7.91219) rectangle (15.8109,8.01803);
\draw [color=c, fill=c] (15.8109,7.91219) rectangle (15.8507,8.01803);
\draw [color=c, fill=c] (15.8507,7.91219) rectangle (15.8905,8.01803);
\draw [color=c, fill=c] (15.8905,7.91219) rectangle (15.9303,8.01803);
\draw [color=c, fill=c] (15.9303,7.91219) rectangle (15.9701,8.01803);
\draw [color=c, fill=c] (15.9701,7.91219) rectangle (16.01,8.01803);
\draw [color=c, fill=c] (16.01,7.91219) rectangle (16.0498,8.01803);
\draw [color=c, fill=c] (16.0498,7.91219) rectangle (16.0896,8.01803);
\draw [color=c, fill=c] (16.0896,7.91219) rectangle (16.1294,8.01803);
\draw [color=c, fill=c] (16.1294,7.91219) rectangle (16.1692,8.01803);
\draw [color=c, fill=c] (16.1692,7.91219) rectangle (16.209,8.01803);
\draw [color=c, fill=c] (16.209,7.91219) rectangle (16.2488,8.01803);
\draw [color=c, fill=c] (16.2488,7.91219) rectangle (16.2886,8.01803);
\draw [color=c, fill=c] (16.2886,7.91219) rectangle (16.3284,8.01803);
\draw [color=c, fill=c] (16.3284,7.91219) rectangle (16.3682,8.01803);
\draw [color=c, fill=c] (16.3682,7.91219) rectangle (16.408,8.01803);
\draw [color=c, fill=c] (16.408,7.91219) rectangle (16.4478,8.01803);
\draw [color=c, fill=c] (16.4478,7.91219) rectangle (16.4876,8.01803);
\draw [color=c, fill=c] (16.4876,7.91219) rectangle (16.5274,8.01803);
\draw [color=c, fill=c] (16.5274,7.91219) rectangle (16.5672,8.01803);
\draw [color=c, fill=c] (16.5672,7.91219) rectangle (16.607,8.01803);
\draw [color=c, fill=c] (16.607,7.91219) rectangle (16.6468,8.01803);
\draw [color=c, fill=c] (16.6468,7.91219) rectangle (16.6866,8.01803);
\draw [color=c, fill=c] (16.6866,7.91219) rectangle (16.7264,8.01803);
\draw [color=c, fill=c] (16.7264,7.91219) rectangle (16.7662,8.01803);
\draw [color=c, fill=c] (16.7662,7.91219) rectangle (16.806,8.01803);
\draw [color=c, fill=c] (16.806,7.91219) rectangle (16.8458,8.01803);
\draw [color=c, fill=c] (16.8458,7.91219) rectangle (16.8856,8.01803);
\draw [color=c, fill=c] (16.8856,7.91219) rectangle (16.9254,8.01803);
\draw [color=c, fill=c] (16.9254,7.91219) rectangle (16.9652,8.01803);
\draw [color=c, fill=c] (16.9652,7.91219) rectangle (17.005,8.01803);
\draw [color=c, fill=c] (17.005,7.91219) rectangle (17.0448,8.01803);
\draw [color=c, fill=c] (17.0448,7.91219) rectangle (17.0846,8.01803);
\draw [color=c, fill=c] (17.0846,7.91219) rectangle (17.1244,8.01803);
\draw [color=c, fill=c] (17.1244,7.91219) rectangle (17.1642,8.01803);
\draw [color=c, fill=c] (17.1642,7.91219) rectangle (17.204,8.01803);
\draw [color=c, fill=c] (17.204,7.91219) rectangle (17.2438,8.01803);
\draw [color=c, fill=c] (17.2438,7.91219) rectangle (17.2836,8.01803);
\draw [color=c, fill=c] (17.2836,7.91219) rectangle (17.3234,8.01803);
\draw [color=c, fill=c] (17.3234,7.91219) rectangle (17.3632,8.01803);
\draw [color=c, fill=c] (17.3632,7.91219) rectangle (17.403,8.01803);
\draw [color=c, fill=c] (17.403,7.91219) rectangle (17.4428,8.01803);
\draw [color=c, fill=c] (17.4428,7.91219) rectangle (17.4826,8.01803);
\draw [color=c, fill=c] (17.4826,7.91219) rectangle (17.5224,8.01803);
\draw [color=c, fill=c] (17.5224,7.91219) rectangle (17.5622,8.01803);
\draw [color=c, fill=c] (17.5622,7.91219) rectangle (17.602,8.01803);
\draw [color=c, fill=c] (17.602,7.91219) rectangle (17.6418,8.01803);
\draw [color=c, fill=c] (17.6418,7.91219) rectangle (17.6816,8.01803);
\draw [color=c, fill=c] (17.6816,7.91219) rectangle (17.7214,8.01803);
\draw [color=c, fill=c] (17.7214,7.91219) rectangle (17.7612,8.01803);
\draw [color=c, fill=c] (17.7612,7.91219) rectangle (17.801,8.01803);
\draw [color=c, fill=c] (17.801,7.91219) rectangle (17.8408,8.01803);
\draw [color=c, fill=c] (17.8408,7.91219) rectangle (17.8806,8.01803);
\draw [color=c, fill=c] (17.8806,7.91219) rectangle (17.9204,8.01803);
\draw [color=c, fill=c] (17.9204,7.91219) rectangle (17.9602,8.01803);
\draw [color=c, fill=c] (17.9602,7.91219) rectangle (18,8.01803);
\definecolor{c}{rgb}{0.2,0,1};
\draw [color=c, fill=c] (2,8.01803) rectangle (2.0398,8.12388);
\draw [color=c, fill=c] (2.0398,8.01803) rectangle (2.0796,8.12388);
\draw [color=c, fill=c] (2.0796,8.01803) rectangle (2.1194,8.12388);
\draw [color=c, fill=c] (2.1194,8.01803) rectangle (2.1592,8.12388);
\draw [color=c, fill=c] (2.1592,8.01803) rectangle (2.19901,8.12388);
\draw [color=c, fill=c] (2.19901,8.01803) rectangle (2.23881,8.12388);
\draw [color=c, fill=c] (2.23881,8.01803) rectangle (2.27861,8.12388);
\draw [color=c, fill=c] (2.27861,8.01803) rectangle (2.31841,8.12388);
\draw [color=c, fill=c] (2.31841,8.01803) rectangle (2.35821,8.12388);
\draw [color=c, fill=c] (2.35821,8.01803) rectangle (2.39801,8.12388);
\draw [color=c, fill=c] (2.39801,8.01803) rectangle (2.43781,8.12388);
\draw [color=c, fill=c] (2.43781,8.01803) rectangle (2.47761,8.12388);
\draw [color=c, fill=c] (2.47761,8.01803) rectangle (2.51741,8.12388);
\draw [color=c, fill=c] (2.51741,8.01803) rectangle (2.55721,8.12388);
\draw [color=c, fill=c] (2.55721,8.01803) rectangle (2.59702,8.12388);
\draw [color=c, fill=c] (2.59702,8.01803) rectangle (2.63682,8.12388);
\draw [color=c, fill=c] (2.63682,8.01803) rectangle (2.67662,8.12388);
\draw [color=c, fill=c] (2.67662,8.01803) rectangle (2.71642,8.12388);
\draw [color=c, fill=c] (2.71642,8.01803) rectangle (2.75622,8.12388);
\draw [color=c, fill=c] (2.75622,8.01803) rectangle (2.79602,8.12388);
\draw [color=c, fill=c] (2.79602,8.01803) rectangle (2.83582,8.12388);
\draw [color=c, fill=c] (2.83582,8.01803) rectangle (2.87562,8.12388);
\draw [color=c, fill=c] (2.87562,8.01803) rectangle (2.91542,8.12388);
\draw [color=c, fill=c] (2.91542,8.01803) rectangle (2.95522,8.12388);
\draw [color=c, fill=c] (2.95522,8.01803) rectangle (2.99502,8.12388);
\draw [color=c, fill=c] (2.99502,8.01803) rectangle (3.03483,8.12388);
\draw [color=c, fill=c] (3.03483,8.01803) rectangle (3.07463,8.12388);
\draw [color=c, fill=c] (3.07463,8.01803) rectangle (3.11443,8.12388);
\draw [color=c, fill=c] (3.11443,8.01803) rectangle (3.15423,8.12388);
\draw [color=c, fill=c] (3.15423,8.01803) rectangle (3.19403,8.12388);
\draw [color=c, fill=c] (3.19403,8.01803) rectangle (3.23383,8.12388);
\draw [color=c, fill=c] (3.23383,8.01803) rectangle (3.27363,8.12388);
\draw [color=c, fill=c] (3.27363,8.01803) rectangle (3.31343,8.12388);
\draw [color=c, fill=c] (3.31343,8.01803) rectangle (3.35323,8.12388);
\draw [color=c, fill=c] (3.35323,8.01803) rectangle (3.39303,8.12388);
\draw [color=c, fill=c] (3.39303,8.01803) rectangle (3.43284,8.12388);
\draw [color=c, fill=c] (3.43284,8.01803) rectangle (3.47264,8.12388);
\draw [color=c, fill=c] (3.47264,8.01803) rectangle (3.51244,8.12388);
\draw [color=c, fill=c] (3.51244,8.01803) rectangle (3.55224,8.12388);
\draw [color=c, fill=c] (3.55224,8.01803) rectangle (3.59204,8.12388);
\draw [color=c, fill=c] (3.59204,8.01803) rectangle (3.63184,8.12388);
\draw [color=c, fill=c] (3.63184,8.01803) rectangle (3.67164,8.12388);
\draw [color=c, fill=c] (3.67164,8.01803) rectangle (3.71144,8.12388);
\draw [color=c, fill=c] (3.71144,8.01803) rectangle (3.75124,8.12388);
\draw [color=c, fill=c] (3.75124,8.01803) rectangle (3.79104,8.12388);
\draw [color=c, fill=c] (3.79104,8.01803) rectangle (3.83085,8.12388);
\draw [color=c, fill=c] (3.83085,8.01803) rectangle (3.87065,8.12388);
\draw [color=c, fill=c] (3.87065,8.01803) rectangle (3.91045,8.12388);
\draw [color=c, fill=c] (3.91045,8.01803) rectangle (3.95025,8.12388);
\draw [color=c, fill=c] (3.95025,8.01803) rectangle (3.99005,8.12388);
\draw [color=c, fill=c] (3.99005,8.01803) rectangle (4.02985,8.12388);
\draw [color=c, fill=c] (4.02985,8.01803) rectangle (4.06965,8.12388);
\draw [color=c, fill=c] (4.06965,8.01803) rectangle (4.10945,8.12388);
\draw [color=c, fill=c] (4.10945,8.01803) rectangle (4.14925,8.12388);
\draw [color=c, fill=c] (4.14925,8.01803) rectangle (4.18905,8.12388);
\draw [color=c, fill=c] (4.18905,8.01803) rectangle (4.22886,8.12388);
\draw [color=c, fill=c] (4.22886,8.01803) rectangle (4.26866,8.12388);
\draw [color=c, fill=c] (4.26866,8.01803) rectangle (4.30846,8.12388);
\draw [color=c, fill=c] (4.30846,8.01803) rectangle (4.34826,8.12388);
\draw [color=c, fill=c] (4.34826,8.01803) rectangle (4.38806,8.12388);
\draw [color=c, fill=c] (4.38806,8.01803) rectangle (4.42786,8.12388);
\draw [color=c, fill=c] (4.42786,8.01803) rectangle (4.46766,8.12388);
\draw [color=c, fill=c] (4.46766,8.01803) rectangle (4.50746,8.12388);
\draw [color=c, fill=c] (4.50746,8.01803) rectangle (4.54726,8.12388);
\draw [color=c, fill=c] (4.54726,8.01803) rectangle (4.58706,8.12388);
\draw [color=c, fill=c] (4.58706,8.01803) rectangle (4.62687,8.12388);
\draw [color=c, fill=c] (4.62687,8.01803) rectangle (4.66667,8.12388);
\draw [color=c, fill=c] (4.66667,8.01803) rectangle (4.70647,8.12388);
\draw [color=c, fill=c] (4.70647,8.01803) rectangle (4.74627,8.12388);
\draw [color=c, fill=c] (4.74627,8.01803) rectangle (4.78607,8.12388);
\draw [color=c, fill=c] (4.78607,8.01803) rectangle (4.82587,8.12388);
\draw [color=c, fill=c] (4.82587,8.01803) rectangle (4.86567,8.12388);
\draw [color=c, fill=c] (4.86567,8.01803) rectangle (4.90547,8.12388);
\draw [color=c, fill=c] (4.90547,8.01803) rectangle (4.94527,8.12388);
\draw [color=c, fill=c] (4.94527,8.01803) rectangle (4.98507,8.12388);
\draw [color=c, fill=c] (4.98507,8.01803) rectangle (5.02488,8.12388);
\draw [color=c, fill=c] (5.02488,8.01803) rectangle (5.06468,8.12388);
\draw [color=c, fill=c] (5.06468,8.01803) rectangle (5.10448,8.12388);
\draw [color=c, fill=c] (5.10448,8.01803) rectangle (5.14428,8.12388);
\draw [color=c, fill=c] (5.14428,8.01803) rectangle (5.18408,8.12388);
\draw [color=c, fill=c] (5.18408,8.01803) rectangle (5.22388,8.12388);
\draw [color=c, fill=c] (5.22388,8.01803) rectangle (5.26368,8.12388);
\draw [color=c, fill=c] (5.26368,8.01803) rectangle (5.30348,8.12388);
\draw [color=c, fill=c] (5.30348,8.01803) rectangle (5.34328,8.12388);
\draw [color=c, fill=c] (5.34328,8.01803) rectangle (5.38308,8.12388);
\draw [color=c, fill=c] (5.38308,8.01803) rectangle (5.42289,8.12388);
\draw [color=c, fill=c] (5.42289,8.01803) rectangle (5.46269,8.12388);
\draw [color=c, fill=c] (5.46269,8.01803) rectangle (5.50249,8.12388);
\draw [color=c, fill=c] (5.50249,8.01803) rectangle (5.54229,8.12388);
\draw [color=c, fill=c] (5.54229,8.01803) rectangle (5.58209,8.12388);
\draw [color=c, fill=c] (5.58209,8.01803) rectangle (5.62189,8.12388);
\draw [color=c, fill=c] (5.62189,8.01803) rectangle (5.66169,8.12388);
\draw [color=c, fill=c] (5.66169,8.01803) rectangle (5.70149,8.12388);
\draw [color=c, fill=c] (5.70149,8.01803) rectangle (5.74129,8.12388);
\draw [color=c, fill=c] (5.74129,8.01803) rectangle (5.78109,8.12388);
\draw [color=c, fill=c] (5.78109,8.01803) rectangle (5.8209,8.12388);
\draw [color=c, fill=c] (5.8209,8.01803) rectangle (5.8607,8.12388);
\draw [color=c, fill=c] (5.8607,8.01803) rectangle (5.9005,8.12388);
\draw [color=c, fill=c] (5.9005,8.01803) rectangle (5.9403,8.12388);
\draw [color=c, fill=c] (5.9403,8.01803) rectangle (5.9801,8.12388);
\draw [color=c, fill=c] (5.9801,8.01803) rectangle (6.0199,8.12388);
\draw [color=c, fill=c] (6.0199,8.01803) rectangle (6.0597,8.12388);
\draw [color=c, fill=c] (6.0597,8.01803) rectangle (6.0995,8.12388);
\draw [color=c, fill=c] (6.0995,8.01803) rectangle (6.1393,8.12388);
\draw [color=c, fill=c] (6.1393,8.01803) rectangle (6.1791,8.12388);
\draw [color=c, fill=c] (6.1791,8.01803) rectangle (6.21891,8.12388);
\draw [color=c, fill=c] (6.21891,8.01803) rectangle (6.25871,8.12388);
\draw [color=c, fill=c] (6.25871,8.01803) rectangle (6.29851,8.12388);
\draw [color=c, fill=c] (6.29851,8.01803) rectangle (6.33831,8.12388);
\draw [color=c, fill=c] (6.33831,8.01803) rectangle (6.37811,8.12388);
\draw [color=c, fill=c] (6.37811,8.01803) rectangle (6.41791,8.12388);
\draw [color=c, fill=c] (6.41791,8.01803) rectangle (6.45771,8.12388);
\draw [color=c, fill=c] (6.45771,8.01803) rectangle (6.49751,8.12388);
\draw [color=c, fill=c] (6.49751,8.01803) rectangle (6.53731,8.12388);
\draw [color=c, fill=c] (6.53731,8.01803) rectangle (6.57711,8.12388);
\draw [color=c, fill=c] (6.57711,8.01803) rectangle (6.61692,8.12388);
\draw [color=c, fill=c] (6.61692,8.01803) rectangle (6.65672,8.12388);
\draw [color=c, fill=c] (6.65672,8.01803) rectangle (6.69652,8.12388);
\draw [color=c, fill=c] (6.69652,8.01803) rectangle (6.73632,8.12388);
\draw [color=c, fill=c] (6.73632,8.01803) rectangle (6.77612,8.12388);
\draw [color=c, fill=c] (6.77612,8.01803) rectangle (6.81592,8.12388);
\draw [color=c, fill=c] (6.81592,8.01803) rectangle (6.85572,8.12388);
\draw [color=c, fill=c] (6.85572,8.01803) rectangle (6.89552,8.12388);
\draw [color=c, fill=c] (6.89552,8.01803) rectangle (6.93532,8.12388);
\draw [color=c, fill=c] (6.93532,8.01803) rectangle (6.97512,8.12388);
\draw [color=c, fill=c] (6.97512,8.01803) rectangle (7.01493,8.12388);
\draw [color=c, fill=c] (7.01493,8.01803) rectangle (7.05473,8.12388);
\draw [color=c, fill=c] (7.05473,8.01803) rectangle (7.09453,8.12388);
\draw [color=c, fill=c] (7.09453,8.01803) rectangle (7.13433,8.12388);
\draw [color=c, fill=c] (7.13433,8.01803) rectangle (7.17413,8.12388);
\draw [color=c, fill=c] (7.17413,8.01803) rectangle (7.21393,8.12388);
\draw [color=c, fill=c] (7.21393,8.01803) rectangle (7.25373,8.12388);
\draw [color=c, fill=c] (7.25373,8.01803) rectangle (7.29353,8.12388);
\draw [color=c, fill=c] (7.29353,8.01803) rectangle (7.33333,8.12388);
\draw [color=c, fill=c] (7.33333,8.01803) rectangle (7.37313,8.12388);
\draw [color=c, fill=c] (7.37313,8.01803) rectangle (7.41294,8.12388);
\draw [color=c, fill=c] (7.41294,8.01803) rectangle (7.45274,8.12388);
\draw [color=c, fill=c] (7.45274,8.01803) rectangle (7.49254,8.12388);
\draw [color=c, fill=c] (7.49254,8.01803) rectangle (7.53234,8.12388);
\draw [color=c, fill=c] (7.53234,8.01803) rectangle (7.57214,8.12388);
\draw [color=c, fill=c] (7.57214,8.01803) rectangle (7.61194,8.12388);
\draw [color=c, fill=c] (7.61194,8.01803) rectangle (7.65174,8.12388);
\draw [color=c, fill=c] (7.65174,8.01803) rectangle (7.69154,8.12388);
\definecolor{c}{rgb}{0,0.0800001,1};
\draw [color=c, fill=c] (7.69154,8.01803) rectangle (7.73134,8.12388);
\draw [color=c, fill=c] (7.73134,8.01803) rectangle (7.77114,8.12388);
\draw [color=c, fill=c] (7.77114,8.01803) rectangle (7.81095,8.12388);
\draw [color=c, fill=c] (7.81095,8.01803) rectangle (7.85075,8.12388);
\draw [color=c, fill=c] (7.85075,8.01803) rectangle (7.89055,8.12388);
\draw [color=c, fill=c] (7.89055,8.01803) rectangle (7.93035,8.12388);
\draw [color=c, fill=c] (7.93035,8.01803) rectangle (7.97015,8.12388);
\draw [color=c, fill=c] (7.97015,8.01803) rectangle (8.00995,8.12388);
\draw [color=c, fill=c] (8.00995,8.01803) rectangle (8.04975,8.12388);
\draw [color=c, fill=c] (8.04975,8.01803) rectangle (8.08955,8.12388);
\draw [color=c, fill=c] (8.08955,8.01803) rectangle (8.12935,8.12388);
\draw [color=c, fill=c] (8.12935,8.01803) rectangle (8.16915,8.12388);
\draw [color=c, fill=c] (8.16915,8.01803) rectangle (8.20895,8.12388);
\draw [color=c, fill=c] (8.20895,8.01803) rectangle (8.24876,8.12388);
\draw [color=c, fill=c] (8.24876,8.01803) rectangle (8.28856,8.12388);
\draw [color=c, fill=c] (8.28856,8.01803) rectangle (8.32836,8.12388);
\draw [color=c, fill=c] (8.32836,8.01803) rectangle (8.36816,8.12388);
\draw [color=c, fill=c] (8.36816,8.01803) rectangle (8.40796,8.12388);
\draw [color=c, fill=c] (8.40796,8.01803) rectangle (8.44776,8.12388);
\draw [color=c, fill=c] (8.44776,8.01803) rectangle (8.48756,8.12388);
\draw [color=c, fill=c] (8.48756,8.01803) rectangle (8.52736,8.12388);
\draw [color=c, fill=c] (8.52736,8.01803) rectangle (8.56716,8.12388);
\draw [color=c, fill=c] (8.56716,8.01803) rectangle (8.60697,8.12388);
\draw [color=c, fill=c] (8.60697,8.01803) rectangle (8.64677,8.12388);
\draw [color=c, fill=c] (8.64677,8.01803) rectangle (8.68657,8.12388);
\draw [color=c, fill=c] (8.68657,8.01803) rectangle (8.72637,8.12388);
\draw [color=c, fill=c] (8.72637,8.01803) rectangle (8.76617,8.12388);
\draw [color=c, fill=c] (8.76617,8.01803) rectangle (8.80597,8.12388);
\draw [color=c, fill=c] (8.80597,8.01803) rectangle (8.84577,8.12388);
\draw [color=c, fill=c] (8.84577,8.01803) rectangle (8.88557,8.12388);
\draw [color=c, fill=c] (8.88557,8.01803) rectangle (8.92537,8.12388);
\draw [color=c, fill=c] (8.92537,8.01803) rectangle (8.96517,8.12388);
\draw [color=c, fill=c] (8.96517,8.01803) rectangle (9.00498,8.12388);
\draw [color=c, fill=c] (9.00498,8.01803) rectangle (9.04478,8.12388);
\draw [color=c, fill=c] (9.04478,8.01803) rectangle (9.08458,8.12388);
\draw [color=c, fill=c] (9.08458,8.01803) rectangle (9.12438,8.12388);
\draw [color=c, fill=c] (9.12438,8.01803) rectangle (9.16418,8.12388);
\draw [color=c, fill=c] (9.16418,8.01803) rectangle (9.20398,8.12388);
\draw [color=c, fill=c] (9.20398,8.01803) rectangle (9.24378,8.12388);
\draw [color=c, fill=c] (9.24378,8.01803) rectangle (9.28358,8.12388);
\draw [color=c, fill=c] (9.28358,8.01803) rectangle (9.32338,8.12388);
\draw [color=c, fill=c] (9.32338,8.01803) rectangle (9.36318,8.12388);
\draw [color=c, fill=c] (9.36318,8.01803) rectangle (9.40298,8.12388);
\draw [color=c, fill=c] (9.40298,8.01803) rectangle (9.44279,8.12388);
\draw [color=c, fill=c] (9.44279,8.01803) rectangle (9.48259,8.12388);
\draw [color=c, fill=c] (9.48259,8.01803) rectangle (9.52239,8.12388);
\definecolor{c}{rgb}{0,0.266667,1};
\draw [color=c, fill=c] (9.52239,8.01803) rectangle (9.56219,8.12388);
\draw [color=c, fill=c] (9.56219,8.01803) rectangle (9.60199,8.12388);
\draw [color=c, fill=c] (9.60199,8.01803) rectangle (9.64179,8.12388);
\draw [color=c, fill=c] (9.64179,8.01803) rectangle (9.68159,8.12388);
\draw [color=c, fill=c] (9.68159,8.01803) rectangle (9.72139,8.12388);
\draw [color=c, fill=c] (9.72139,8.01803) rectangle (9.76119,8.12388);
\draw [color=c, fill=c] (9.76119,8.01803) rectangle (9.80099,8.12388);
\draw [color=c, fill=c] (9.80099,8.01803) rectangle (9.8408,8.12388);
\draw [color=c, fill=c] (9.8408,8.01803) rectangle (9.8806,8.12388);
\draw [color=c, fill=c] (9.8806,8.01803) rectangle (9.9204,8.12388);
\draw [color=c, fill=c] (9.9204,8.01803) rectangle (9.9602,8.12388);
\draw [color=c, fill=c] (9.9602,8.01803) rectangle (10,8.12388);
\draw [color=c, fill=c] (10,8.01803) rectangle (10.0398,8.12388);
\draw [color=c, fill=c] (10.0398,8.01803) rectangle (10.0796,8.12388);
\draw [color=c, fill=c] (10.0796,8.01803) rectangle (10.1194,8.12388);
\draw [color=c, fill=c] (10.1194,8.01803) rectangle (10.1592,8.12388);
\draw [color=c, fill=c] (10.1592,8.01803) rectangle (10.199,8.12388);
\draw [color=c, fill=c] (10.199,8.01803) rectangle (10.2388,8.12388);
\draw [color=c, fill=c] (10.2388,8.01803) rectangle (10.2786,8.12388);
\draw [color=c, fill=c] (10.2786,8.01803) rectangle (10.3184,8.12388);
\draw [color=c, fill=c] (10.3184,8.01803) rectangle (10.3582,8.12388);
\draw [color=c, fill=c] (10.3582,8.01803) rectangle (10.398,8.12388);
\definecolor{c}{rgb}{0,0.546666,1};
\draw [color=c, fill=c] (10.398,8.01803) rectangle (10.4378,8.12388);
\draw [color=c, fill=c] (10.4378,8.01803) rectangle (10.4776,8.12388);
\draw [color=c, fill=c] (10.4776,8.01803) rectangle (10.5174,8.12388);
\draw [color=c, fill=c] (10.5174,8.01803) rectangle (10.5572,8.12388);
\draw [color=c, fill=c] (10.5572,8.01803) rectangle (10.597,8.12388);
\draw [color=c, fill=c] (10.597,8.01803) rectangle (10.6368,8.12388);
\draw [color=c, fill=c] (10.6368,8.01803) rectangle (10.6766,8.12388);
\draw [color=c, fill=c] (10.6766,8.01803) rectangle (10.7164,8.12388);
\draw [color=c, fill=c] (10.7164,8.01803) rectangle (10.7562,8.12388);
\draw [color=c, fill=c] (10.7562,8.01803) rectangle (10.796,8.12388);
\draw [color=c, fill=c] (10.796,8.01803) rectangle (10.8358,8.12388);
\draw [color=c, fill=c] (10.8358,8.01803) rectangle (10.8756,8.12388);
\draw [color=c, fill=c] (10.8756,8.01803) rectangle (10.9154,8.12388);
\draw [color=c, fill=c] (10.9154,8.01803) rectangle (10.9552,8.12388);
\draw [color=c, fill=c] (10.9552,8.01803) rectangle (10.995,8.12388);
\draw [color=c, fill=c] (10.995,8.01803) rectangle (11.0348,8.12388);
\draw [color=c, fill=c] (11.0348,8.01803) rectangle (11.0746,8.12388);
\draw [color=c, fill=c] (11.0746,8.01803) rectangle (11.1144,8.12388);
\draw [color=c, fill=c] (11.1144,8.01803) rectangle (11.1542,8.12388);
\draw [color=c, fill=c] (11.1542,8.01803) rectangle (11.194,8.12388);
\draw [color=c, fill=c] (11.194,8.01803) rectangle (11.2338,8.12388);
\draw [color=c, fill=c] (11.2338,8.01803) rectangle (11.2736,8.12388);
\draw [color=c, fill=c] (11.2736,8.01803) rectangle (11.3134,8.12388);
\draw [color=c, fill=c] (11.3134,8.01803) rectangle (11.3532,8.12388);
\draw [color=c, fill=c] (11.3532,8.01803) rectangle (11.393,8.12388);
\draw [color=c, fill=c] (11.393,8.01803) rectangle (11.4328,8.12388);
\draw [color=c, fill=c] (11.4328,8.01803) rectangle (11.4726,8.12388);
\draw [color=c, fill=c] (11.4726,8.01803) rectangle (11.5124,8.12388);
\draw [color=c, fill=c] (11.5124,8.01803) rectangle (11.5522,8.12388);
\draw [color=c, fill=c] (11.5522,8.01803) rectangle (11.592,8.12388);
\draw [color=c, fill=c] (11.592,8.01803) rectangle (11.6318,8.12388);
\draw [color=c, fill=c] (11.6318,8.01803) rectangle (11.6716,8.12388);
\draw [color=c, fill=c] (11.6716,8.01803) rectangle (11.7114,8.12388);
\draw [color=c, fill=c] (11.7114,8.01803) rectangle (11.7512,8.12388);
\draw [color=c, fill=c] (11.7512,8.01803) rectangle (11.791,8.12388);
\draw [color=c, fill=c] (11.791,8.01803) rectangle (11.8308,8.12388);
\draw [color=c, fill=c] (11.8308,8.01803) rectangle (11.8706,8.12388);
\draw [color=c, fill=c] (11.8706,8.01803) rectangle (11.9104,8.12388);
\draw [color=c, fill=c] (11.9104,8.01803) rectangle (11.9502,8.12388);
\draw [color=c, fill=c] (11.9502,8.01803) rectangle (11.99,8.12388);
\draw [color=c, fill=c] (11.99,8.01803) rectangle (12.0299,8.12388);
\draw [color=c, fill=c] (12.0299,8.01803) rectangle (12.0697,8.12388);
\draw [color=c, fill=c] (12.0697,8.01803) rectangle (12.1095,8.12388);
\draw [color=c, fill=c] (12.1095,8.01803) rectangle (12.1493,8.12388);
\draw [color=c, fill=c] (12.1493,8.01803) rectangle (12.1891,8.12388);
\draw [color=c, fill=c] (12.1891,8.01803) rectangle (12.2289,8.12388);
\draw [color=c, fill=c] (12.2289,8.01803) rectangle (12.2687,8.12388);
\draw [color=c, fill=c] (12.2687,8.01803) rectangle (12.3085,8.12388);
\definecolor{c}{rgb}{0,0.733333,1};
\draw [color=c, fill=c] (12.3085,8.01803) rectangle (12.3483,8.12388);
\draw [color=c, fill=c] (12.3483,8.01803) rectangle (12.3881,8.12388);
\draw [color=c, fill=c] (12.3881,8.01803) rectangle (12.4279,8.12388);
\draw [color=c, fill=c] (12.4279,8.01803) rectangle (12.4677,8.12388);
\draw [color=c, fill=c] (12.4677,8.01803) rectangle (12.5075,8.12388);
\draw [color=c, fill=c] (12.5075,8.01803) rectangle (12.5473,8.12388);
\draw [color=c, fill=c] (12.5473,8.01803) rectangle (12.5871,8.12388);
\draw [color=c, fill=c] (12.5871,8.01803) rectangle (12.6269,8.12388);
\draw [color=c, fill=c] (12.6269,8.01803) rectangle (12.6667,8.12388);
\draw [color=c, fill=c] (12.6667,8.01803) rectangle (12.7065,8.12388);
\draw [color=c, fill=c] (12.7065,8.01803) rectangle (12.7463,8.12388);
\draw [color=c, fill=c] (12.7463,8.01803) rectangle (12.7861,8.12388);
\draw [color=c, fill=c] (12.7861,8.01803) rectangle (12.8259,8.12388);
\draw [color=c, fill=c] (12.8259,8.01803) rectangle (12.8657,8.12388);
\draw [color=c, fill=c] (12.8657,8.01803) rectangle (12.9055,8.12388);
\draw [color=c, fill=c] (12.9055,8.01803) rectangle (12.9453,8.12388);
\draw [color=c, fill=c] (12.9453,8.01803) rectangle (12.9851,8.12388);
\draw [color=c, fill=c] (12.9851,8.01803) rectangle (13.0249,8.12388);
\draw [color=c, fill=c] (13.0249,8.01803) rectangle (13.0647,8.12388);
\draw [color=c, fill=c] (13.0647,8.01803) rectangle (13.1045,8.12388);
\draw [color=c, fill=c] (13.1045,8.01803) rectangle (13.1443,8.12388);
\draw [color=c, fill=c] (13.1443,8.01803) rectangle (13.1841,8.12388);
\draw [color=c, fill=c] (13.1841,8.01803) rectangle (13.2239,8.12388);
\draw [color=c, fill=c] (13.2239,8.01803) rectangle (13.2637,8.12388);
\draw [color=c, fill=c] (13.2637,8.01803) rectangle (13.3035,8.12388);
\draw [color=c, fill=c] (13.3035,8.01803) rectangle (13.3433,8.12388);
\draw [color=c, fill=c] (13.3433,8.01803) rectangle (13.3831,8.12388);
\draw [color=c, fill=c] (13.3831,8.01803) rectangle (13.4229,8.12388);
\draw [color=c, fill=c] (13.4229,8.01803) rectangle (13.4627,8.12388);
\draw [color=c, fill=c] (13.4627,8.01803) rectangle (13.5025,8.12388);
\draw [color=c, fill=c] (13.5025,8.01803) rectangle (13.5423,8.12388);
\draw [color=c, fill=c] (13.5423,8.01803) rectangle (13.5821,8.12388);
\draw [color=c, fill=c] (13.5821,8.01803) rectangle (13.6219,8.12388);
\draw [color=c, fill=c] (13.6219,8.01803) rectangle (13.6617,8.12388);
\draw [color=c, fill=c] (13.6617,8.01803) rectangle (13.7015,8.12388);
\draw [color=c, fill=c] (13.7015,8.01803) rectangle (13.7413,8.12388);
\draw [color=c, fill=c] (13.7413,8.01803) rectangle (13.7811,8.12388);
\draw [color=c, fill=c] (13.7811,8.01803) rectangle (13.8209,8.12388);
\draw [color=c, fill=c] (13.8209,8.01803) rectangle (13.8607,8.12388);
\draw [color=c, fill=c] (13.8607,8.01803) rectangle (13.9005,8.12388);
\draw [color=c, fill=c] (13.9005,8.01803) rectangle (13.9403,8.12388);
\draw [color=c, fill=c] (13.9403,8.01803) rectangle (13.9801,8.12388);
\draw [color=c, fill=c] (13.9801,8.01803) rectangle (14.0199,8.12388);
\draw [color=c, fill=c] (14.0199,8.01803) rectangle (14.0597,8.12388);
\draw [color=c, fill=c] (14.0597,8.01803) rectangle (14.0995,8.12388);
\draw [color=c, fill=c] (14.0995,8.01803) rectangle (14.1393,8.12388);
\draw [color=c, fill=c] (14.1393,8.01803) rectangle (14.1791,8.12388);
\draw [color=c, fill=c] (14.1791,8.01803) rectangle (14.2189,8.12388);
\draw [color=c, fill=c] (14.2189,8.01803) rectangle (14.2587,8.12388);
\draw [color=c, fill=c] (14.2587,8.01803) rectangle (14.2985,8.12388);
\draw [color=c, fill=c] (14.2985,8.01803) rectangle (14.3383,8.12388);
\draw [color=c, fill=c] (14.3383,8.01803) rectangle (14.3781,8.12388);
\draw [color=c, fill=c] (14.3781,8.01803) rectangle (14.4179,8.12388);
\draw [color=c, fill=c] (14.4179,8.01803) rectangle (14.4577,8.12388);
\draw [color=c, fill=c] (14.4577,8.01803) rectangle (14.4975,8.12388);
\draw [color=c, fill=c] (14.4975,8.01803) rectangle (14.5373,8.12388);
\draw [color=c, fill=c] (14.5373,8.01803) rectangle (14.5771,8.12388);
\draw [color=c, fill=c] (14.5771,8.01803) rectangle (14.6169,8.12388);
\draw [color=c, fill=c] (14.6169,8.01803) rectangle (14.6567,8.12388);
\draw [color=c, fill=c] (14.6567,8.01803) rectangle (14.6965,8.12388);
\draw [color=c, fill=c] (14.6965,8.01803) rectangle (14.7363,8.12388);
\draw [color=c, fill=c] (14.7363,8.01803) rectangle (14.7761,8.12388);
\draw [color=c, fill=c] (14.7761,8.01803) rectangle (14.8159,8.12388);
\draw [color=c, fill=c] (14.8159,8.01803) rectangle (14.8557,8.12388);
\draw [color=c, fill=c] (14.8557,8.01803) rectangle (14.8955,8.12388);
\draw [color=c, fill=c] (14.8955,8.01803) rectangle (14.9353,8.12388);
\draw [color=c, fill=c] (14.9353,8.01803) rectangle (14.9751,8.12388);
\draw [color=c, fill=c] (14.9751,8.01803) rectangle (15.0149,8.12388);
\draw [color=c, fill=c] (15.0149,8.01803) rectangle (15.0547,8.12388);
\draw [color=c, fill=c] (15.0547,8.01803) rectangle (15.0945,8.12388);
\draw [color=c, fill=c] (15.0945,8.01803) rectangle (15.1343,8.12388);
\draw [color=c, fill=c] (15.1343,8.01803) rectangle (15.1741,8.12388);
\draw [color=c, fill=c] (15.1741,8.01803) rectangle (15.2139,8.12388);
\draw [color=c, fill=c] (15.2139,8.01803) rectangle (15.2537,8.12388);
\draw [color=c, fill=c] (15.2537,8.01803) rectangle (15.2935,8.12388);
\draw [color=c, fill=c] (15.2935,8.01803) rectangle (15.3333,8.12388);
\draw [color=c, fill=c] (15.3333,8.01803) rectangle (15.3731,8.12388);
\draw [color=c, fill=c] (15.3731,8.01803) rectangle (15.4129,8.12388);
\draw [color=c, fill=c] (15.4129,8.01803) rectangle (15.4527,8.12388);
\draw [color=c, fill=c] (15.4527,8.01803) rectangle (15.4925,8.12388);
\draw [color=c, fill=c] (15.4925,8.01803) rectangle (15.5323,8.12388);
\draw [color=c, fill=c] (15.5323,8.01803) rectangle (15.5721,8.12388);
\draw [color=c, fill=c] (15.5721,8.01803) rectangle (15.6119,8.12388);
\draw [color=c, fill=c] (15.6119,8.01803) rectangle (15.6517,8.12388);
\draw [color=c, fill=c] (15.6517,8.01803) rectangle (15.6915,8.12388);
\draw [color=c, fill=c] (15.6915,8.01803) rectangle (15.7313,8.12388);
\draw [color=c, fill=c] (15.7313,8.01803) rectangle (15.7711,8.12388);
\draw [color=c, fill=c] (15.7711,8.01803) rectangle (15.8109,8.12388);
\draw [color=c, fill=c] (15.8109,8.01803) rectangle (15.8507,8.12388);
\draw [color=c, fill=c] (15.8507,8.01803) rectangle (15.8905,8.12388);
\draw [color=c, fill=c] (15.8905,8.01803) rectangle (15.9303,8.12388);
\draw [color=c, fill=c] (15.9303,8.01803) rectangle (15.9701,8.12388);
\draw [color=c, fill=c] (15.9701,8.01803) rectangle (16.01,8.12388);
\draw [color=c, fill=c] (16.01,8.01803) rectangle (16.0498,8.12388);
\draw [color=c, fill=c] (16.0498,8.01803) rectangle (16.0896,8.12388);
\draw [color=c, fill=c] (16.0896,8.01803) rectangle (16.1294,8.12388);
\draw [color=c, fill=c] (16.1294,8.01803) rectangle (16.1692,8.12388);
\draw [color=c, fill=c] (16.1692,8.01803) rectangle (16.209,8.12388);
\draw [color=c, fill=c] (16.209,8.01803) rectangle (16.2488,8.12388);
\draw [color=c, fill=c] (16.2488,8.01803) rectangle (16.2886,8.12388);
\draw [color=c, fill=c] (16.2886,8.01803) rectangle (16.3284,8.12388);
\draw [color=c, fill=c] (16.3284,8.01803) rectangle (16.3682,8.12388);
\draw [color=c, fill=c] (16.3682,8.01803) rectangle (16.408,8.12388);
\draw [color=c, fill=c] (16.408,8.01803) rectangle (16.4478,8.12388);
\draw [color=c, fill=c] (16.4478,8.01803) rectangle (16.4876,8.12388);
\draw [color=c, fill=c] (16.4876,8.01803) rectangle (16.5274,8.12388);
\draw [color=c, fill=c] (16.5274,8.01803) rectangle (16.5672,8.12388);
\draw [color=c, fill=c] (16.5672,8.01803) rectangle (16.607,8.12388);
\draw [color=c, fill=c] (16.607,8.01803) rectangle (16.6468,8.12388);
\draw [color=c, fill=c] (16.6468,8.01803) rectangle (16.6866,8.12388);
\draw [color=c, fill=c] (16.6866,8.01803) rectangle (16.7264,8.12388);
\draw [color=c, fill=c] (16.7264,8.01803) rectangle (16.7662,8.12388);
\draw [color=c, fill=c] (16.7662,8.01803) rectangle (16.806,8.12388);
\draw [color=c, fill=c] (16.806,8.01803) rectangle (16.8458,8.12388);
\draw [color=c, fill=c] (16.8458,8.01803) rectangle (16.8856,8.12388);
\draw [color=c, fill=c] (16.8856,8.01803) rectangle (16.9254,8.12388);
\draw [color=c, fill=c] (16.9254,8.01803) rectangle (16.9652,8.12388);
\draw [color=c, fill=c] (16.9652,8.01803) rectangle (17.005,8.12388);
\draw [color=c, fill=c] (17.005,8.01803) rectangle (17.0448,8.12388);
\draw [color=c, fill=c] (17.0448,8.01803) rectangle (17.0846,8.12388);
\draw [color=c, fill=c] (17.0846,8.01803) rectangle (17.1244,8.12388);
\draw [color=c, fill=c] (17.1244,8.01803) rectangle (17.1642,8.12388);
\draw [color=c, fill=c] (17.1642,8.01803) rectangle (17.204,8.12388);
\draw [color=c, fill=c] (17.204,8.01803) rectangle (17.2438,8.12388);
\draw [color=c, fill=c] (17.2438,8.01803) rectangle (17.2836,8.12388);
\draw [color=c, fill=c] (17.2836,8.01803) rectangle (17.3234,8.12388);
\draw [color=c, fill=c] (17.3234,8.01803) rectangle (17.3632,8.12388);
\draw [color=c, fill=c] (17.3632,8.01803) rectangle (17.403,8.12388);
\draw [color=c, fill=c] (17.403,8.01803) rectangle (17.4428,8.12388);
\draw [color=c, fill=c] (17.4428,8.01803) rectangle (17.4826,8.12388);
\draw [color=c, fill=c] (17.4826,8.01803) rectangle (17.5224,8.12388);
\draw [color=c, fill=c] (17.5224,8.01803) rectangle (17.5622,8.12388);
\draw [color=c, fill=c] (17.5622,8.01803) rectangle (17.602,8.12388);
\draw [color=c, fill=c] (17.602,8.01803) rectangle (17.6418,8.12388);
\draw [color=c, fill=c] (17.6418,8.01803) rectangle (17.6816,8.12388);
\draw [color=c, fill=c] (17.6816,8.01803) rectangle (17.7214,8.12388);
\draw [color=c, fill=c] (17.7214,8.01803) rectangle (17.7612,8.12388);
\draw [color=c, fill=c] (17.7612,8.01803) rectangle (17.801,8.12388);
\draw [color=c, fill=c] (17.801,8.01803) rectangle (17.8408,8.12388);
\draw [color=c, fill=c] (17.8408,8.01803) rectangle (17.8806,8.12388);
\draw [color=c, fill=c] (17.8806,8.01803) rectangle (17.9204,8.12388);
\draw [color=c, fill=c] (17.9204,8.01803) rectangle (17.9602,8.12388);
\draw [color=c, fill=c] (17.9602,8.01803) rectangle (18,8.12388);
\definecolor{c}{rgb}{0.2,0,1};
\draw [color=c, fill=c] (2,8.12388) rectangle (2.0398,8.22973);
\draw [color=c, fill=c] (2.0398,8.12388) rectangle (2.0796,8.22973);
\draw [color=c, fill=c] (2.0796,8.12388) rectangle (2.1194,8.22973);
\draw [color=c, fill=c] (2.1194,8.12388) rectangle (2.1592,8.22973);
\draw [color=c, fill=c] (2.1592,8.12388) rectangle (2.19901,8.22973);
\draw [color=c, fill=c] (2.19901,8.12388) rectangle (2.23881,8.22973);
\draw [color=c, fill=c] (2.23881,8.12388) rectangle (2.27861,8.22973);
\draw [color=c, fill=c] (2.27861,8.12388) rectangle (2.31841,8.22973);
\draw [color=c, fill=c] (2.31841,8.12388) rectangle (2.35821,8.22973);
\draw [color=c, fill=c] (2.35821,8.12388) rectangle (2.39801,8.22973);
\draw [color=c, fill=c] (2.39801,8.12388) rectangle (2.43781,8.22973);
\draw [color=c, fill=c] (2.43781,8.12388) rectangle (2.47761,8.22973);
\draw [color=c, fill=c] (2.47761,8.12388) rectangle (2.51741,8.22973);
\draw [color=c, fill=c] (2.51741,8.12388) rectangle (2.55721,8.22973);
\draw [color=c, fill=c] (2.55721,8.12388) rectangle (2.59702,8.22973);
\draw [color=c, fill=c] (2.59702,8.12388) rectangle (2.63682,8.22973);
\draw [color=c, fill=c] (2.63682,8.12388) rectangle (2.67662,8.22973);
\draw [color=c, fill=c] (2.67662,8.12388) rectangle (2.71642,8.22973);
\draw [color=c, fill=c] (2.71642,8.12388) rectangle (2.75622,8.22973);
\draw [color=c, fill=c] (2.75622,8.12388) rectangle (2.79602,8.22973);
\draw [color=c, fill=c] (2.79602,8.12388) rectangle (2.83582,8.22973);
\draw [color=c, fill=c] (2.83582,8.12388) rectangle (2.87562,8.22973);
\draw [color=c, fill=c] (2.87562,8.12388) rectangle (2.91542,8.22973);
\draw [color=c, fill=c] (2.91542,8.12388) rectangle (2.95522,8.22973);
\draw [color=c, fill=c] (2.95522,8.12388) rectangle (2.99502,8.22973);
\draw [color=c, fill=c] (2.99502,8.12388) rectangle (3.03483,8.22973);
\draw [color=c, fill=c] (3.03483,8.12388) rectangle (3.07463,8.22973);
\draw [color=c, fill=c] (3.07463,8.12388) rectangle (3.11443,8.22973);
\draw [color=c, fill=c] (3.11443,8.12388) rectangle (3.15423,8.22973);
\draw [color=c, fill=c] (3.15423,8.12388) rectangle (3.19403,8.22973);
\draw [color=c, fill=c] (3.19403,8.12388) rectangle (3.23383,8.22973);
\draw [color=c, fill=c] (3.23383,8.12388) rectangle (3.27363,8.22973);
\draw [color=c, fill=c] (3.27363,8.12388) rectangle (3.31343,8.22973);
\draw [color=c, fill=c] (3.31343,8.12388) rectangle (3.35323,8.22973);
\draw [color=c, fill=c] (3.35323,8.12388) rectangle (3.39303,8.22973);
\draw [color=c, fill=c] (3.39303,8.12388) rectangle (3.43284,8.22973);
\draw [color=c, fill=c] (3.43284,8.12388) rectangle (3.47264,8.22973);
\draw [color=c, fill=c] (3.47264,8.12388) rectangle (3.51244,8.22973);
\draw [color=c, fill=c] (3.51244,8.12388) rectangle (3.55224,8.22973);
\draw [color=c, fill=c] (3.55224,8.12388) rectangle (3.59204,8.22973);
\draw [color=c, fill=c] (3.59204,8.12388) rectangle (3.63184,8.22973);
\draw [color=c, fill=c] (3.63184,8.12388) rectangle (3.67164,8.22973);
\draw [color=c, fill=c] (3.67164,8.12388) rectangle (3.71144,8.22973);
\draw [color=c, fill=c] (3.71144,8.12388) rectangle (3.75124,8.22973);
\draw [color=c, fill=c] (3.75124,8.12388) rectangle (3.79104,8.22973);
\draw [color=c, fill=c] (3.79104,8.12388) rectangle (3.83085,8.22973);
\draw [color=c, fill=c] (3.83085,8.12388) rectangle (3.87065,8.22973);
\draw [color=c, fill=c] (3.87065,8.12388) rectangle (3.91045,8.22973);
\draw [color=c, fill=c] (3.91045,8.12388) rectangle (3.95025,8.22973);
\draw [color=c, fill=c] (3.95025,8.12388) rectangle (3.99005,8.22973);
\draw [color=c, fill=c] (3.99005,8.12388) rectangle (4.02985,8.22973);
\draw [color=c, fill=c] (4.02985,8.12388) rectangle (4.06965,8.22973);
\draw [color=c, fill=c] (4.06965,8.12388) rectangle (4.10945,8.22973);
\draw [color=c, fill=c] (4.10945,8.12388) rectangle (4.14925,8.22973);
\draw [color=c, fill=c] (4.14925,8.12388) rectangle (4.18905,8.22973);
\draw [color=c, fill=c] (4.18905,8.12388) rectangle (4.22886,8.22973);
\draw [color=c, fill=c] (4.22886,8.12388) rectangle (4.26866,8.22973);
\draw [color=c, fill=c] (4.26866,8.12388) rectangle (4.30846,8.22973);
\draw [color=c, fill=c] (4.30846,8.12388) rectangle (4.34826,8.22973);
\draw [color=c, fill=c] (4.34826,8.12388) rectangle (4.38806,8.22973);
\draw [color=c, fill=c] (4.38806,8.12388) rectangle (4.42786,8.22973);
\draw [color=c, fill=c] (4.42786,8.12388) rectangle (4.46766,8.22973);
\draw [color=c, fill=c] (4.46766,8.12388) rectangle (4.50746,8.22973);
\draw [color=c, fill=c] (4.50746,8.12388) rectangle (4.54726,8.22973);
\draw [color=c, fill=c] (4.54726,8.12388) rectangle (4.58706,8.22973);
\draw [color=c, fill=c] (4.58706,8.12388) rectangle (4.62687,8.22973);
\draw [color=c, fill=c] (4.62687,8.12388) rectangle (4.66667,8.22973);
\draw [color=c, fill=c] (4.66667,8.12388) rectangle (4.70647,8.22973);
\draw [color=c, fill=c] (4.70647,8.12388) rectangle (4.74627,8.22973);
\draw [color=c, fill=c] (4.74627,8.12388) rectangle (4.78607,8.22973);
\draw [color=c, fill=c] (4.78607,8.12388) rectangle (4.82587,8.22973);
\draw [color=c, fill=c] (4.82587,8.12388) rectangle (4.86567,8.22973);
\draw [color=c, fill=c] (4.86567,8.12388) rectangle (4.90547,8.22973);
\draw [color=c, fill=c] (4.90547,8.12388) rectangle (4.94527,8.22973);
\draw [color=c, fill=c] (4.94527,8.12388) rectangle (4.98507,8.22973);
\draw [color=c, fill=c] (4.98507,8.12388) rectangle (5.02488,8.22973);
\draw [color=c, fill=c] (5.02488,8.12388) rectangle (5.06468,8.22973);
\draw [color=c, fill=c] (5.06468,8.12388) rectangle (5.10448,8.22973);
\draw [color=c, fill=c] (5.10448,8.12388) rectangle (5.14428,8.22973);
\draw [color=c, fill=c] (5.14428,8.12388) rectangle (5.18408,8.22973);
\draw [color=c, fill=c] (5.18408,8.12388) rectangle (5.22388,8.22973);
\draw [color=c, fill=c] (5.22388,8.12388) rectangle (5.26368,8.22973);
\draw [color=c, fill=c] (5.26368,8.12388) rectangle (5.30348,8.22973);
\draw [color=c, fill=c] (5.30348,8.12388) rectangle (5.34328,8.22973);
\draw [color=c, fill=c] (5.34328,8.12388) rectangle (5.38308,8.22973);
\draw [color=c, fill=c] (5.38308,8.12388) rectangle (5.42289,8.22973);
\draw [color=c, fill=c] (5.42289,8.12388) rectangle (5.46269,8.22973);
\draw [color=c, fill=c] (5.46269,8.12388) rectangle (5.50249,8.22973);
\draw [color=c, fill=c] (5.50249,8.12388) rectangle (5.54229,8.22973);
\draw [color=c, fill=c] (5.54229,8.12388) rectangle (5.58209,8.22973);
\draw [color=c, fill=c] (5.58209,8.12388) rectangle (5.62189,8.22973);
\draw [color=c, fill=c] (5.62189,8.12388) rectangle (5.66169,8.22973);
\draw [color=c, fill=c] (5.66169,8.12388) rectangle (5.70149,8.22973);
\draw [color=c, fill=c] (5.70149,8.12388) rectangle (5.74129,8.22973);
\draw [color=c, fill=c] (5.74129,8.12388) rectangle (5.78109,8.22973);
\draw [color=c, fill=c] (5.78109,8.12388) rectangle (5.8209,8.22973);
\draw [color=c, fill=c] (5.8209,8.12388) rectangle (5.8607,8.22973);
\draw [color=c, fill=c] (5.8607,8.12388) rectangle (5.9005,8.22973);
\draw [color=c, fill=c] (5.9005,8.12388) rectangle (5.9403,8.22973);
\draw [color=c, fill=c] (5.9403,8.12388) rectangle (5.9801,8.22973);
\draw [color=c, fill=c] (5.9801,8.12388) rectangle (6.0199,8.22973);
\draw [color=c, fill=c] (6.0199,8.12388) rectangle (6.0597,8.22973);
\draw [color=c, fill=c] (6.0597,8.12388) rectangle (6.0995,8.22973);
\draw [color=c, fill=c] (6.0995,8.12388) rectangle (6.1393,8.22973);
\draw [color=c, fill=c] (6.1393,8.12388) rectangle (6.1791,8.22973);
\draw [color=c, fill=c] (6.1791,8.12388) rectangle (6.21891,8.22973);
\draw [color=c, fill=c] (6.21891,8.12388) rectangle (6.25871,8.22973);
\draw [color=c, fill=c] (6.25871,8.12388) rectangle (6.29851,8.22973);
\draw [color=c, fill=c] (6.29851,8.12388) rectangle (6.33831,8.22973);
\draw [color=c, fill=c] (6.33831,8.12388) rectangle (6.37811,8.22973);
\draw [color=c, fill=c] (6.37811,8.12388) rectangle (6.41791,8.22973);
\draw [color=c, fill=c] (6.41791,8.12388) rectangle (6.45771,8.22973);
\draw [color=c, fill=c] (6.45771,8.12388) rectangle (6.49751,8.22973);
\draw [color=c, fill=c] (6.49751,8.12388) rectangle (6.53731,8.22973);
\draw [color=c, fill=c] (6.53731,8.12388) rectangle (6.57711,8.22973);
\draw [color=c, fill=c] (6.57711,8.12388) rectangle (6.61692,8.22973);
\draw [color=c, fill=c] (6.61692,8.12388) rectangle (6.65672,8.22973);
\draw [color=c, fill=c] (6.65672,8.12388) rectangle (6.69652,8.22973);
\draw [color=c, fill=c] (6.69652,8.12388) rectangle (6.73632,8.22973);
\draw [color=c, fill=c] (6.73632,8.12388) rectangle (6.77612,8.22973);
\draw [color=c, fill=c] (6.77612,8.12388) rectangle (6.81592,8.22973);
\draw [color=c, fill=c] (6.81592,8.12388) rectangle (6.85572,8.22973);
\draw [color=c, fill=c] (6.85572,8.12388) rectangle (6.89552,8.22973);
\draw [color=c, fill=c] (6.89552,8.12388) rectangle (6.93532,8.22973);
\draw [color=c, fill=c] (6.93532,8.12388) rectangle (6.97512,8.22973);
\draw [color=c, fill=c] (6.97512,8.12388) rectangle (7.01493,8.22973);
\draw [color=c, fill=c] (7.01493,8.12388) rectangle (7.05473,8.22973);
\draw [color=c, fill=c] (7.05473,8.12388) rectangle (7.09453,8.22973);
\draw [color=c, fill=c] (7.09453,8.12388) rectangle (7.13433,8.22973);
\draw [color=c, fill=c] (7.13433,8.12388) rectangle (7.17413,8.22973);
\draw [color=c, fill=c] (7.17413,8.12388) rectangle (7.21393,8.22973);
\draw [color=c, fill=c] (7.21393,8.12388) rectangle (7.25373,8.22973);
\draw [color=c, fill=c] (7.25373,8.12388) rectangle (7.29353,8.22973);
\draw [color=c, fill=c] (7.29353,8.12388) rectangle (7.33333,8.22973);
\draw [color=c, fill=c] (7.33333,8.12388) rectangle (7.37313,8.22973);
\draw [color=c, fill=c] (7.37313,8.12388) rectangle (7.41294,8.22973);
\draw [color=c, fill=c] (7.41294,8.12388) rectangle (7.45274,8.22973);
\draw [color=c, fill=c] (7.45274,8.12388) rectangle (7.49254,8.22973);
\draw [color=c, fill=c] (7.49254,8.12388) rectangle (7.53234,8.22973);
\draw [color=c, fill=c] (7.53234,8.12388) rectangle (7.57214,8.22973);
\draw [color=c, fill=c] (7.57214,8.12388) rectangle (7.61194,8.22973);
\draw [color=c, fill=c] (7.61194,8.12388) rectangle (7.65174,8.22973);
\definecolor{c}{rgb}{0,0.0800001,1};
\draw [color=c, fill=c] (7.65174,8.12388) rectangle (7.69154,8.22973);
\draw [color=c, fill=c] (7.69154,8.12388) rectangle (7.73134,8.22973);
\draw [color=c, fill=c] (7.73134,8.12388) rectangle (7.77114,8.22973);
\draw [color=c, fill=c] (7.77114,8.12388) rectangle (7.81095,8.22973);
\draw [color=c, fill=c] (7.81095,8.12388) rectangle (7.85075,8.22973);
\draw [color=c, fill=c] (7.85075,8.12388) rectangle (7.89055,8.22973);
\draw [color=c, fill=c] (7.89055,8.12388) rectangle (7.93035,8.22973);
\draw [color=c, fill=c] (7.93035,8.12388) rectangle (7.97015,8.22973);
\draw [color=c, fill=c] (7.97015,8.12388) rectangle (8.00995,8.22973);
\draw [color=c, fill=c] (8.00995,8.12388) rectangle (8.04975,8.22973);
\draw [color=c, fill=c] (8.04975,8.12388) rectangle (8.08955,8.22973);
\draw [color=c, fill=c] (8.08955,8.12388) rectangle (8.12935,8.22973);
\draw [color=c, fill=c] (8.12935,8.12388) rectangle (8.16915,8.22973);
\draw [color=c, fill=c] (8.16915,8.12388) rectangle (8.20895,8.22973);
\draw [color=c, fill=c] (8.20895,8.12388) rectangle (8.24876,8.22973);
\draw [color=c, fill=c] (8.24876,8.12388) rectangle (8.28856,8.22973);
\draw [color=c, fill=c] (8.28856,8.12388) rectangle (8.32836,8.22973);
\draw [color=c, fill=c] (8.32836,8.12388) rectangle (8.36816,8.22973);
\draw [color=c, fill=c] (8.36816,8.12388) rectangle (8.40796,8.22973);
\draw [color=c, fill=c] (8.40796,8.12388) rectangle (8.44776,8.22973);
\draw [color=c, fill=c] (8.44776,8.12388) rectangle (8.48756,8.22973);
\draw [color=c, fill=c] (8.48756,8.12388) rectangle (8.52736,8.22973);
\draw [color=c, fill=c] (8.52736,8.12388) rectangle (8.56716,8.22973);
\draw [color=c, fill=c] (8.56716,8.12388) rectangle (8.60697,8.22973);
\draw [color=c, fill=c] (8.60697,8.12388) rectangle (8.64677,8.22973);
\draw [color=c, fill=c] (8.64677,8.12388) rectangle (8.68657,8.22973);
\draw [color=c, fill=c] (8.68657,8.12388) rectangle (8.72637,8.22973);
\draw [color=c, fill=c] (8.72637,8.12388) rectangle (8.76617,8.22973);
\draw [color=c, fill=c] (8.76617,8.12388) rectangle (8.80597,8.22973);
\draw [color=c, fill=c] (8.80597,8.12388) rectangle (8.84577,8.22973);
\draw [color=c, fill=c] (8.84577,8.12388) rectangle (8.88557,8.22973);
\draw [color=c, fill=c] (8.88557,8.12388) rectangle (8.92537,8.22973);
\draw [color=c, fill=c] (8.92537,8.12388) rectangle (8.96517,8.22973);
\draw [color=c, fill=c] (8.96517,8.12388) rectangle (9.00498,8.22973);
\draw [color=c, fill=c] (9.00498,8.12388) rectangle (9.04478,8.22973);
\draw [color=c, fill=c] (9.04478,8.12388) rectangle (9.08458,8.22973);
\draw [color=c, fill=c] (9.08458,8.12388) rectangle (9.12438,8.22973);
\draw [color=c, fill=c] (9.12438,8.12388) rectangle (9.16418,8.22973);
\draw [color=c, fill=c] (9.16418,8.12388) rectangle (9.20398,8.22973);
\draw [color=c, fill=c] (9.20398,8.12388) rectangle (9.24378,8.22973);
\draw [color=c, fill=c] (9.24378,8.12388) rectangle (9.28358,8.22973);
\draw [color=c, fill=c] (9.28358,8.12388) rectangle (9.32338,8.22973);
\draw [color=c, fill=c] (9.32338,8.12388) rectangle (9.36318,8.22973);
\draw [color=c, fill=c] (9.36318,8.12388) rectangle (9.40298,8.22973);
\draw [color=c, fill=c] (9.40298,8.12388) rectangle (9.44279,8.22973);
\draw [color=c, fill=c] (9.44279,8.12388) rectangle (9.48259,8.22973);
\draw [color=c, fill=c] (9.48259,8.12388) rectangle (9.52239,8.22973);
\definecolor{c}{rgb}{0,0.266667,1};
\draw [color=c, fill=c] (9.52239,8.12388) rectangle (9.56219,8.22973);
\draw [color=c, fill=c] (9.56219,8.12388) rectangle (9.60199,8.22973);
\draw [color=c, fill=c] (9.60199,8.12388) rectangle (9.64179,8.22973);
\draw [color=c, fill=c] (9.64179,8.12388) rectangle (9.68159,8.22973);
\draw [color=c, fill=c] (9.68159,8.12388) rectangle (9.72139,8.22973);
\draw [color=c, fill=c] (9.72139,8.12388) rectangle (9.76119,8.22973);
\draw [color=c, fill=c] (9.76119,8.12388) rectangle (9.80099,8.22973);
\draw [color=c, fill=c] (9.80099,8.12388) rectangle (9.8408,8.22973);
\draw [color=c, fill=c] (9.8408,8.12388) rectangle (9.8806,8.22973);
\draw [color=c, fill=c] (9.8806,8.12388) rectangle (9.9204,8.22973);
\draw [color=c, fill=c] (9.9204,8.12388) rectangle (9.9602,8.22973);
\draw [color=c, fill=c] (9.9602,8.12388) rectangle (10,8.22973);
\draw [color=c, fill=c] (10,8.12388) rectangle (10.0398,8.22973);
\draw [color=c, fill=c] (10.0398,8.12388) rectangle (10.0796,8.22973);
\draw [color=c, fill=c] (10.0796,8.12388) rectangle (10.1194,8.22973);
\draw [color=c, fill=c] (10.1194,8.12388) rectangle (10.1592,8.22973);
\draw [color=c, fill=c] (10.1592,8.12388) rectangle (10.199,8.22973);
\draw [color=c, fill=c] (10.199,8.12388) rectangle (10.2388,8.22973);
\draw [color=c, fill=c] (10.2388,8.12388) rectangle (10.2786,8.22973);
\draw [color=c, fill=c] (10.2786,8.12388) rectangle (10.3184,8.22973);
\draw [color=c, fill=c] (10.3184,8.12388) rectangle (10.3582,8.22973);
\draw [color=c, fill=c] (10.3582,8.12388) rectangle (10.398,8.22973);
\draw [color=c, fill=c] (10.398,8.12388) rectangle (10.4378,8.22973);
\definecolor{c}{rgb}{0,0.546666,1};
\draw [color=c, fill=c] (10.4378,8.12388) rectangle (10.4776,8.22973);
\draw [color=c, fill=c] (10.4776,8.12388) rectangle (10.5174,8.22973);
\draw [color=c, fill=c] (10.5174,8.12388) rectangle (10.5572,8.22973);
\draw [color=c, fill=c] (10.5572,8.12388) rectangle (10.597,8.22973);
\draw [color=c, fill=c] (10.597,8.12388) rectangle (10.6368,8.22973);
\draw [color=c, fill=c] (10.6368,8.12388) rectangle (10.6766,8.22973);
\draw [color=c, fill=c] (10.6766,8.12388) rectangle (10.7164,8.22973);
\draw [color=c, fill=c] (10.7164,8.12388) rectangle (10.7562,8.22973);
\draw [color=c, fill=c] (10.7562,8.12388) rectangle (10.796,8.22973);
\draw [color=c, fill=c] (10.796,8.12388) rectangle (10.8358,8.22973);
\draw [color=c, fill=c] (10.8358,8.12388) rectangle (10.8756,8.22973);
\draw [color=c, fill=c] (10.8756,8.12388) rectangle (10.9154,8.22973);
\draw [color=c, fill=c] (10.9154,8.12388) rectangle (10.9552,8.22973);
\draw [color=c, fill=c] (10.9552,8.12388) rectangle (10.995,8.22973);
\draw [color=c, fill=c] (10.995,8.12388) rectangle (11.0348,8.22973);
\draw [color=c, fill=c] (11.0348,8.12388) rectangle (11.0746,8.22973);
\draw [color=c, fill=c] (11.0746,8.12388) rectangle (11.1144,8.22973);
\draw [color=c, fill=c] (11.1144,8.12388) rectangle (11.1542,8.22973);
\draw [color=c, fill=c] (11.1542,8.12388) rectangle (11.194,8.22973);
\draw [color=c, fill=c] (11.194,8.12388) rectangle (11.2338,8.22973);
\draw [color=c, fill=c] (11.2338,8.12388) rectangle (11.2736,8.22973);
\draw [color=c, fill=c] (11.2736,8.12388) rectangle (11.3134,8.22973);
\draw [color=c, fill=c] (11.3134,8.12388) rectangle (11.3532,8.22973);
\draw [color=c, fill=c] (11.3532,8.12388) rectangle (11.393,8.22973);
\draw [color=c, fill=c] (11.393,8.12388) rectangle (11.4328,8.22973);
\draw [color=c, fill=c] (11.4328,8.12388) rectangle (11.4726,8.22973);
\draw [color=c, fill=c] (11.4726,8.12388) rectangle (11.5124,8.22973);
\draw [color=c, fill=c] (11.5124,8.12388) rectangle (11.5522,8.22973);
\draw [color=c, fill=c] (11.5522,8.12388) rectangle (11.592,8.22973);
\draw [color=c, fill=c] (11.592,8.12388) rectangle (11.6318,8.22973);
\draw [color=c, fill=c] (11.6318,8.12388) rectangle (11.6716,8.22973);
\draw [color=c, fill=c] (11.6716,8.12388) rectangle (11.7114,8.22973);
\draw [color=c, fill=c] (11.7114,8.12388) rectangle (11.7512,8.22973);
\draw [color=c, fill=c] (11.7512,8.12388) rectangle (11.791,8.22973);
\draw [color=c, fill=c] (11.791,8.12388) rectangle (11.8308,8.22973);
\draw [color=c, fill=c] (11.8308,8.12388) rectangle (11.8706,8.22973);
\draw [color=c, fill=c] (11.8706,8.12388) rectangle (11.9104,8.22973);
\draw [color=c, fill=c] (11.9104,8.12388) rectangle (11.9502,8.22973);
\draw [color=c, fill=c] (11.9502,8.12388) rectangle (11.99,8.22973);
\draw [color=c, fill=c] (11.99,8.12388) rectangle (12.0299,8.22973);
\draw [color=c, fill=c] (12.0299,8.12388) rectangle (12.0697,8.22973);
\draw [color=c, fill=c] (12.0697,8.12388) rectangle (12.1095,8.22973);
\draw [color=c, fill=c] (12.1095,8.12388) rectangle (12.1493,8.22973);
\draw [color=c, fill=c] (12.1493,8.12388) rectangle (12.1891,8.22973);
\draw [color=c, fill=c] (12.1891,8.12388) rectangle (12.2289,8.22973);
\draw [color=c, fill=c] (12.2289,8.12388) rectangle (12.2687,8.22973);
\draw [color=c, fill=c] (12.2687,8.12388) rectangle (12.3085,8.22973);
\draw [color=c, fill=c] (12.3085,8.12388) rectangle (12.3483,8.22973);
\draw [color=c, fill=c] (12.3483,8.12388) rectangle (12.3881,8.22973);
\definecolor{c}{rgb}{0,0.733333,1};
\draw [color=c, fill=c] (12.3881,8.12388) rectangle (12.4279,8.22973);
\draw [color=c, fill=c] (12.4279,8.12388) rectangle (12.4677,8.22973);
\draw [color=c, fill=c] (12.4677,8.12388) rectangle (12.5075,8.22973);
\draw [color=c, fill=c] (12.5075,8.12388) rectangle (12.5473,8.22973);
\draw [color=c, fill=c] (12.5473,8.12388) rectangle (12.5871,8.22973);
\draw [color=c, fill=c] (12.5871,8.12388) rectangle (12.6269,8.22973);
\draw [color=c, fill=c] (12.6269,8.12388) rectangle (12.6667,8.22973);
\draw [color=c, fill=c] (12.6667,8.12388) rectangle (12.7065,8.22973);
\draw [color=c, fill=c] (12.7065,8.12388) rectangle (12.7463,8.22973);
\draw [color=c, fill=c] (12.7463,8.12388) rectangle (12.7861,8.22973);
\draw [color=c, fill=c] (12.7861,8.12388) rectangle (12.8259,8.22973);
\draw [color=c, fill=c] (12.8259,8.12388) rectangle (12.8657,8.22973);
\draw [color=c, fill=c] (12.8657,8.12388) rectangle (12.9055,8.22973);
\draw [color=c, fill=c] (12.9055,8.12388) rectangle (12.9453,8.22973);
\draw [color=c, fill=c] (12.9453,8.12388) rectangle (12.9851,8.22973);
\draw [color=c, fill=c] (12.9851,8.12388) rectangle (13.0249,8.22973);
\draw [color=c, fill=c] (13.0249,8.12388) rectangle (13.0647,8.22973);
\draw [color=c, fill=c] (13.0647,8.12388) rectangle (13.1045,8.22973);
\draw [color=c, fill=c] (13.1045,8.12388) rectangle (13.1443,8.22973);
\draw [color=c, fill=c] (13.1443,8.12388) rectangle (13.1841,8.22973);
\draw [color=c, fill=c] (13.1841,8.12388) rectangle (13.2239,8.22973);
\draw [color=c, fill=c] (13.2239,8.12388) rectangle (13.2637,8.22973);
\draw [color=c, fill=c] (13.2637,8.12388) rectangle (13.3035,8.22973);
\draw [color=c, fill=c] (13.3035,8.12388) rectangle (13.3433,8.22973);
\draw [color=c, fill=c] (13.3433,8.12388) rectangle (13.3831,8.22973);
\draw [color=c, fill=c] (13.3831,8.12388) rectangle (13.4229,8.22973);
\draw [color=c, fill=c] (13.4229,8.12388) rectangle (13.4627,8.22973);
\draw [color=c, fill=c] (13.4627,8.12388) rectangle (13.5025,8.22973);
\draw [color=c, fill=c] (13.5025,8.12388) rectangle (13.5423,8.22973);
\draw [color=c, fill=c] (13.5423,8.12388) rectangle (13.5821,8.22973);
\draw [color=c, fill=c] (13.5821,8.12388) rectangle (13.6219,8.22973);
\draw [color=c, fill=c] (13.6219,8.12388) rectangle (13.6617,8.22973);
\draw [color=c, fill=c] (13.6617,8.12388) rectangle (13.7015,8.22973);
\draw [color=c, fill=c] (13.7015,8.12388) rectangle (13.7413,8.22973);
\draw [color=c, fill=c] (13.7413,8.12388) rectangle (13.7811,8.22973);
\draw [color=c, fill=c] (13.7811,8.12388) rectangle (13.8209,8.22973);
\draw [color=c, fill=c] (13.8209,8.12388) rectangle (13.8607,8.22973);
\draw [color=c, fill=c] (13.8607,8.12388) rectangle (13.9005,8.22973);
\draw [color=c, fill=c] (13.9005,8.12388) rectangle (13.9403,8.22973);
\draw [color=c, fill=c] (13.9403,8.12388) rectangle (13.9801,8.22973);
\draw [color=c, fill=c] (13.9801,8.12388) rectangle (14.0199,8.22973);
\draw [color=c, fill=c] (14.0199,8.12388) rectangle (14.0597,8.22973);
\draw [color=c, fill=c] (14.0597,8.12388) rectangle (14.0995,8.22973);
\draw [color=c, fill=c] (14.0995,8.12388) rectangle (14.1393,8.22973);
\draw [color=c, fill=c] (14.1393,8.12388) rectangle (14.1791,8.22973);
\draw [color=c, fill=c] (14.1791,8.12388) rectangle (14.2189,8.22973);
\draw [color=c, fill=c] (14.2189,8.12388) rectangle (14.2587,8.22973);
\draw [color=c, fill=c] (14.2587,8.12388) rectangle (14.2985,8.22973);
\draw [color=c, fill=c] (14.2985,8.12388) rectangle (14.3383,8.22973);
\draw [color=c, fill=c] (14.3383,8.12388) rectangle (14.3781,8.22973);
\draw [color=c, fill=c] (14.3781,8.12388) rectangle (14.4179,8.22973);
\draw [color=c, fill=c] (14.4179,8.12388) rectangle (14.4577,8.22973);
\draw [color=c, fill=c] (14.4577,8.12388) rectangle (14.4975,8.22973);
\draw [color=c, fill=c] (14.4975,8.12388) rectangle (14.5373,8.22973);
\draw [color=c, fill=c] (14.5373,8.12388) rectangle (14.5771,8.22973);
\draw [color=c, fill=c] (14.5771,8.12388) rectangle (14.6169,8.22973);
\draw [color=c, fill=c] (14.6169,8.12388) rectangle (14.6567,8.22973);
\draw [color=c, fill=c] (14.6567,8.12388) rectangle (14.6965,8.22973);
\draw [color=c, fill=c] (14.6965,8.12388) rectangle (14.7363,8.22973);
\draw [color=c, fill=c] (14.7363,8.12388) rectangle (14.7761,8.22973);
\draw [color=c, fill=c] (14.7761,8.12388) rectangle (14.8159,8.22973);
\draw [color=c, fill=c] (14.8159,8.12388) rectangle (14.8557,8.22973);
\draw [color=c, fill=c] (14.8557,8.12388) rectangle (14.8955,8.22973);
\draw [color=c, fill=c] (14.8955,8.12388) rectangle (14.9353,8.22973);
\draw [color=c, fill=c] (14.9353,8.12388) rectangle (14.9751,8.22973);
\draw [color=c, fill=c] (14.9751,8.12388) rectangle (15.0149,8.22973);
\draw [color=c, fill=c] (15.0149,8.12388) rectangle (15.0547,8.22973);
\draw [color=c, fill=c] (15.0547,8.12388) rectangle (15.0945,8.22973);
\draw [color=c, fill=c] (15.0945,8.12388) rectangle (15.1343,8.22973);
\draw [color=c, fill=c] (15.1343,8.12388) rectangle (15.1741,8.22973);
\draw [color=c, fill=c] (15.1741,8.12388) rectangle (15.2139,8.22973);
\draw [color=c, fill=c] (15.2139,8.12388) rectangle (15.2537,8.22973);
\draw [color=c, fill=c] (15.2537,8.12388) rectangle (15.2935,8.22973);
\draw [color=c, fill=c] (15.2935,8.12388) rectangle (15.3333,8.22973);
\draw [color=c, fill=c] (15.3333,8.12388) rectangle (15.3731,8.22973);
\draw [color=c, fill=c] (15.3731,8.12388) rectangle (15.4129,8.22973);
\draw [color=c, fill=c] (15.4129,8.12388) rectangle (15.4527,8.22973);
\draw [color=c, fill=c] (15.4527,8.12388) rectangle (15.4925,8.22973);
\draw [color=c, fill=c] (15.4925,8.12388) rectangle (15.5323,8.22973);
\draw [color=c, fill=c] (15.5323,8.12388) rectangle (15.5721,8.22973);
\draw [color=c, fill=c] (15.5721,8.12388) rectangle (15.6119,8.22973);
\draw [color=c, fill=c] (15.6119,8.12388) rectangle (15.6517,8.22973);
\draw [color=c, fill=c] (15.6517,8.12388) rectangle (15.6915,8.22973);
\draw [color=c, fill=c] (15.6915,8.12388) rectangle (15.7313,8.22973);
\draw [color=c, fill=c] (15.7313,8.12388) rectangle (15.7711,8.22973);
\draw [color=c, fill=c] (15.7711,8.12388) rectangle (15.8109,8.22973);
\draw [color=c, fill=c] (15.8109,8.12388) rectangle (15.8507,8.22973);
\draw [color=c, fill=c] (15.8507,8.12388) rectangle (15.8905,8.22973);
\draw [color=c, fill=c] (15.8905,8.12388) rectangle (15.9303,8.22973);
\draw [color=c, fill=c] (15.9303,8.12388) rectangle (15.9701,8.22973);
\draw [color=c, fill=c] (15.9701,8.12388) rectangle (16.01,8.22973);
\draw [color=c, fill=c] (16.01,8.12388) rectangle (16.0498,8.22973);
\draw [color=c, fill=c] (16.0498,8.12388) rectangle (16.0896,8.22973);
\draw [color=c, fill=c] (16.0896,8.12388) rectangle (16.1294,8.22973);
\draw [color=c, fill=c] (16.1294,8.12388) rectangle (16.1692,8.22973);
\draw [color=c, fill=c] (16.1692,8.12388) rectangle (16.209,8.22973);
\draw [color=c, fill=c] (16.209,8.12388) rectangle (16.2488,8.22973);
\draw [color=c, fill=c] (16.2488,8.12388) rectangle (16.2886,8.22973);
\draw [color=c, fill=c] (16.2886,8.12388) rectangle (16.3284,8.22973);
\draw [color=c, fill=c] (16.3284,8.12388) rectangle (16.3682,8.22973);
\draw [color=c, fill=c] (16.3682,8.12388) rectangle (16.408,8.22973);
\draw [color=c, fill=c] (16.408,8.12388) rectangle (16.4478,8.22973);
\draw [color=c, fill=c] (16.4478,8.12388) rectangle (16.4876,8.22973);
\draw [color=c, fill=c] (16.4876,8.12388) rectangle (16.5274,8.22973);
\draw [color=c, fill=c] (16.5274,8.12388) rectangle (16.5672,8.22973);
\draw [color=c, fill=c] (16.5672,8.12388) rectangle (16.607,8.22973);
\draw [color=c, fill=c] (16.607,8.12388) rectangle (16.6468,8.22973);
\draw [color=c, fill=c] (16.6468,8.12388) rectangle (16.6866,8.22973);
\draw [color=c, fill=c] (16.6866,8.12388) rectangle (16.7264,8.22973);
\draw [color=c, fill=c] (16.7264,8.12388) rectangle (16.7662,8.22973);
\draw [color=c, fill=c] (16.7662,8.12388) rectangle (16.806,8.22973);
\draw [color=c, fill=c] (16.806,8.12388) rectangle (16.8458,8.22973);
\draw [color=c, fill=c] (16.8458,8.12388) rectangle (16.8856,8.22973);
\draw [color=c, fill=c] (16.8856,8.12388) rectangle (16.9254,8.22973);
\draw [color=c, fill=c] (16.9254,8.12388) rectangle (16.9652,8.22973);
\draw [color=c, fill=c] (16.9652,8.12388) rectangle (17.005,8.22973);
\draw [color=c, fill=c] (17.005,8.12388) rectangle (17.0448,8.22973);
\draw [color=c, fill=c] (17.0448,8.12388) rectangle (17.0846,8.22973);
\draw [color=c, fill=c] (17.0846,8.12388) rectangle (17.1244,8.22973);
\draw [color=c, fill=c] (17.1244,8.12388) rectangle (17.1642,8.22973);
\draw [color=c, fill=c] (17.1642,8.12388) rectangle (17.204,8.22973);
\draw [color=c, fill=c] (17.204,8.12388) rectangle (17.2438,8.22973);
\draw [color=c, fill=c] (17.2438,8.12388) rectangle (17.2836,8.22973);
\draw [color=c, fill=c] (17.2836,8.12388) rectangle (17.3234,8.22973);
\draw [color=c, fill=c] (17.3234,8.12388) rectangle (17.3632,8.22973);
\draw [color=c, fill=c] (17.3632,8.12388) rectangle (17.403,8.22973);
\draw [color=c, fill=c] (17.403,8.12388) rectangle (17.4428,8.22973);
\draw [color=c, fill=c] (17.4428,8.12388) rectangle (17.4826,8.22973);
\draw [color=c, fill=c] (17.4826,8.12388) rectangle (17.5224,8.22973);
\draw [color=c, fill=c] (17.5224,8.12388) rectangle (17.5622,8.22973);
\draw [color=c, fill=c] (17.5622,8.12388) rectangle (17.602,8.22973);
\draw [color=c, fill=c] (17.602,8.12388) rectangle (17.6418,8.22973);
\draw [color=c, fill=c] (17.6418,8.12388) rectangle (17.6816,8.22973);
\draw [color=c, fill=c] (17.6816,8.12388) rectangle (17.7214,8.22973);
\draw [color=c, fill=c] (17.7214,8.12388) rectangle (17.7612,8.22973);
\draw [color=c, fill=c] (17.7612,8.12388) rectangle (17.801,8.22973);
\draw [color=c, fill=c] (17.801,8.12388) rectangle (17.8408,8.22973);
\draw [color=c, fill=c] (17.8408,8.12388) rectangle (17.8806,8.22973);
\draw [color=c, fill=c] (17.8806,8.12388) rectangle (17.9204,8.22973);
\draw [color=c, fill=c] (17.9204,8.12388) rectangle (17.9602,8.22973);
\draw [color=c, fill=c] (17.9602,8.12388) rectangle (18,8.22973);
\definecolor{c}{rgb}{0.2,0,1};
\draw [color=c, fill=c] (2,8.22973) rectangle (2.0398,8.33558);
\draw [color=c, fill=c] (2.0398,8.22973) rectangle (2.0796,8.33558);
\draw [color=c, fill=c] (2.0796,8.22973) rectangle (2.1194,8.33558);
\draw [color=c, fill=c] (2.1194,8.22973) rectangle (2.1592,8.33558);
\draw [color=c, fill=c] (2.1592,8.22973) rectangle (2.19901,8.33558);
\draw [color=c, fill=c] (2.19901,8.22973) rectangle (2.23881,8.33558);
\draw [color=c, fill=c] (2.23881,8.22973) rectangle (2.27861,8.33558);
\draw [color=c, fill=c] (2.27861,8.22973) rectangle (2.31841,8.33558);
\draw [color=c, fill=c] (2.31841,8.22973) rectangle (2.35821,8.33558);
\draw [color=c, fill=c] (2.35821,8.22973) rectangle (2.39801,8.33558);
\draw [color=c, fill=c] (2.39801,8.22973) rectangle (2.43781,8.33558);
\draw [color=c, fill=c] (2.43781,8.22973) rectangle (2.47761,8.33558);
\draw [color=c, fill=c] (2.47761,8.22973) rectangle (2.51741,8.33558);
\draw [color=c, fill=c] (2.51741,8.22973) rectangle (2.55721,8.33558);
\draw [color=c, fill=c] (2.55721,8.22973) rectangle (2.59702,8.33558);
\draw [color=c, fill=c] (2.59702,8.22973) rectangle (2.63682,8.33558);
\draw [color=c, fill=c] (2.63682,8.22973) rectangle (2.67662,8.33558);
\draw [color=c, fill=c] (2.67662,8.22973) rectangle (2.71642,8.33558);
\draw [color=c, fill=c] (2.71642,8.22973) rectangle (2.75622,8.33558);
\draw [color=c, fill=c] (2.75622,8.22973) rectangle (2.79602,8.33558);
\draw [color=c, fill=c] (2.79602,8.22973) rectangle (2.83582,8.33558);
\draw [color=c, fill=c] (2.83582,8.22973) rectangle (2.87562,8.33558);
\draw [color=c, fill=c] (2.87562,8.22973) rectangle (2.91542,8.33558);
\draw [color=c, fill=c] (2.91542,8.22973) rectangle (2.95522,8.33558);
\draw [color=c, fill=c] (2.95522,8.22973) rectangle (2.99502,8.33558);
\draw [color=c, fill=c] (2.99502,8.22973) rectangle (3.03483,8.33558);
\draw [color=c, fill=c] (3.03483,8.22973) rectangle (3.07463,8.33558);
\draw [color=c, fill=c] (3.07463,8.22973) rectangle (3.11443,8.33558);
\draw [color=c, fill=c] (3.11443,8.22973) rectangle (3.15423,8.33558);
\draw [color=c, fill=c] (3.15423,8.22973) rectangle (3.19403,8.33558);
\draw [color=c, fill=c] (3.19403,8.22973) rectangle (3.23383,8.33558);
\draw [color=c, fill=c] (3.23383,8.22973) rectangle (3.27363,8.33558);
\draw [color=c, fill=c] (3.27363,8.22973) rectangle (3.31343,8.33558);
\draw [color=c, fill=c] (3.31343,8.22973) rectangle (3.35323,8.33558);
\draw [color=c, fill=c] (3.35323,8.22973) rectangle (3.39303,8.33558);
\draw [color=c, fill=c] (3.39303,8.22973) rectangle (3.43284,8.33558);
\draw [color=c, fill=c] (3.43284,8.22973) rectangle (3.47264,8.33558);
\draw [color=c, fill=c] (3.47264,8.22973) rectangle (3.51244,8.33558);
\draw [color=c, fill=c] (3.51244,8.22973) rectangle (3.55224,8.33558);
\draw [color=c, fill=c] (3.55224,8.22973) rectangle (3.59204,8.33558);
\draw [color=c, fill=c] (3.59204,8.22973) rectangle (3.63184,8.33558);
\draw [color=c, fill=c] (3.63184,8.22973) rectangle (3.67164,8.33558);
\draw [color=c, fill=c] (3.67164,8.22973) rectangle (3.71144,8.33558);
\draw [color=c, fill=c] (3.71144,8.22973) rectangle (3.75124,8.33558);
\draw [color=c, fill=c] (3.75124,8.22973) rectangle (3.79104,8.33558);
\draw [color=c, fill=c] (3.79104,8.22973) rectangle (3.83085,8.33558);
\draw [color=c, fill=c] (3.83085,8.22973) rectangle (3.87065,8.33558);
\draw [color=c, fill=c] (3.87065,8.22973) rectangle (3.91045,8.33558);
\draw [color=c, fill=c] (3.91045,8.22973) rectangle (3.95025,8.33558);
\draw [color=c, fill=c] (3.95025,8.22973) rectangle (3.99005,8.33558);
\draw [color=c, fill=c] (3.99005,8.22973) rectangle (4.02985,8.33558);
\draw [color=c, fill=c] (4.02985,8.22973) rectangle (4.06965,8.33558);
\draw [color=c, fill=c] (4.06965,8.22973) rectangle (4.10945,8.33558);
\draw [color=c, fill=c] (4.10945,8.22973) rectangle (4.14925,8.33558);
\draw [color=c, fill=c] (4.14925,8.22973) rectangle (4.18905,8.33558);
\draw [color=c, fill=c] (4.18905,8.22973) rectangle (4.22886,8.33558);
\draw [color=c, fill=c] (4.22886,8.22973) rectangle (4.26866,8.33558);
\draw [color=c, fill=c] (4.26866,8.22973) rectangle (4.30846,8.33558);
\draw [color=c, fill=c] (4.30846,8.22973) rectangle (4.34826,8.33558);
\draw [color=c, fill=c] (4.34826,8.22973) rectangle (4.38806,8.33558);
\draw [color=c, fill=c] (4.38806,8.22973) rectangle (4.42786,8.33558);
\draw [color=c, fill=c] (4.42786,8.22973) rectangle (4.46766,8.33558);
\draw [color=c, fill=c] (4.46766,8.22973) rectangle (4.50746,8.33558);
\draw [color=c, fill=c] (4.50746,8.22973) rectangle (4.54726,8.33558);
\draw [color=c, fill=c] (4.54726,8.22973) rectangle (4.58706,8.33558);
\draw [color=c, fill=c] (4.58706,8.22973) rectangle (4.62687,8.33558);
\draw [color=c, fill=c] (4.62687,8.22973) rectangle (4.66667,8.33558);
\draw [color=c, fill=c] (4.66667,8.22973) rectangle (4.70647,8.33558);
\draw [color=c, fill=c] (4.70647,8.22973) rectangle (4.74627,8.33558);
\draw [color=c, fill=c] (4.74627,8.22973) rectangle (4.78607,8.33558);
\draw [color=c, fill=c] (4.78607,8.22973) rectangle (4.82587,8.33558);
\draw [color=c, fill=c] (4.82587,8.22973) rectangle (4.86567,8.33558);
\draw [color=c, fill=c] (4.86567,8.22973) rectangle (4.90547,8.33558);
\draw [color=c, fill=c] (4.90547,8.22973) rectangle (4.94527,8.33558);
\draw [color=c, fill=c] (4.94527,8.22973) rectangle (4.98507,8.33558);
\draw [color=c, fill=c] (4.98507,8.22973) rectangle (5.02488,8.33558);
\draw [color=c, fill=c] (5.02488,8.22973) rectangle (5.06468,8.33558);
\draw [color=c, fill=c] (5.06468,8.22973) rectangle (5.10448,8.33558);
\draw [color=c, fill=c] (5.10448,8.22973) rectangle (5.14428,8.33558);
\draw [color=c, fill=c] (5.14428,8.22973) rectangle (5.18408,8.33558);
\draw [color=c, fill=c] (5.18408,8.22973) rectangle (5.22388,8.33558);
\draw [color=c, fill=c] (5.22388,8.22973) rectangle (5.26368,8.33558);
\draw [color=c, fill=c] (5.26368,8.22973) rectangle (5.30348,8.33558);
\draw [color=c, fill=c] (5.30348,8.22973) rectangle (5.34328,8.33558);
\draw [color=c, fill=c] (5.34328,8.22973) rectangle (5.38308,8.33558);
\draw [color=c, fill=c] (5.38308,8.22973) rectangle (5.42289,8.33558);
\draw [color=c, fill=c] (5.42289,8.22973) rectangle (5.46269,8.33558);
\draw [color=c, fill=c] (5.46269,8.22973) rectangle (5.50249,8.33558);
\draw [color=c, fill=c] (5.50249,8.22973) rectangle (5.54229,8.33558);
\draw [color=c, fill=c] (5.54229,8.22973) rectangle (5.58209,8.33558);
\draw [color=c, fill=c] (5.58209,8.22973) rectangle (5.62189,8.33558);
\draw [color=c, fill=c] (5.62189,8.22973) rectangle (5.66169,8.33558);
\draw [color=c, fill=c] (5.66169,8.22973) rectangle (5.70149,8.33558);
\draw [color=c, fill=c] (5.70149,8.22973) rectangle (5.74129,8.33558);
\draw [color=c, fill=c] (5.74129,8.22973) rectangle (5.78109,8.33558);
\draw [color=c, fill=c] (5.78109,8.22973) rectangle (5.8209,8.33558);
\draw [color=c, fill=c] (5.8209,8.22973) rectangle (5.8607,8.33558);
\draw [color=c, fill=c] (5.8607,8.22973) rectangle (5.9005,8.33558);
\draw [color=c, fill=c] (5.9005,8.22973) rectangle (5.9403,8.33558);
\draw [color=c, fill=c] (5.9403,8.22973) rectangle (5.9801,8.33558);
\draw [color=c, fill=c] (5.9801,8.22973) rectangle (6.0199,8.33558);
\draw [color=c, fill=c] (6.0199,8.22973) rectangle (6.0597,8.33558);
\draw [color=c, fill=c] (6.0597,8.22973) rectangle (6.0995,8.33558);
\draw [color=c, fill=c] (6.0995,8.22973) rectangle (6.1393,8.33558);
\draw [color=c, fill=c] (6.1393,8.22973) rectangle (6.1791,8.33558);
\draw [color=c, fill=c] (6.1791,8.22973) rectangle (6.21891,8.33558);
\draw [color=c, fill=c] (6.21891,8.22973) rectangle (6.25871,8.33558);
\draw [color=c, fill=c] (6.25871,8.22973) rectangle (6.29851,8.33558);
\draw [color=c, fill=c] (6.29851,8.22973) rectangle (6.33831,8.33558);
\draw [color=c, fill=c] (6.33831,8.22973) rectangle (6.37811,8.33558);
\draw [color=c, fill=c] (6.37811,8.22973) rectangle (6.41791,8.33558);
\draw [color=c, fill=c] (6.41791,8.22973) rectangle (6.45771,8.33558);
\draw [color=c, fill=c] (6.45771,8.22973) rectangle (6.49751,8.33558);
\draw [color=c, fill=c] (6.49751,8.22973) rectangle (6.53731,8.33558);
\draw [color=c, fill=c] (6.53731,8.22973) rectangle (6.57711,8.33558);
\draw [color=c, fill=c] (6.57711,8.22973) rectangle (6.61692,8.33558);
\draw [color=c, fill=c] (6.61692,8.22973) rectangle (6.65672,8.33558);
\draw [color=c, fill=c] (6.65672,8.22973) rectangle (6.69652,8.33558);
\draw [color=c, fill=c] (6.69652,8.22973) rectangle (6.73632,8.33558);
\draw [color=c, fill=c] (6.73632,8.22973) rectangle (6.77612,8.33558);
\draw [color=c, fill=c] (6.77612,8.22973) rectangle (6.81592,8.33558);
\draw [color=c, fill=c] (6.81592,8.22973) rectangle (6.85572,8.33558);
\draw [color=c, fill=c] (6.85572,8.22973) rectangle (6.89552,8.33558);
\draw [color=c, fill=c] (6.89552,8.22973) rectangle (6.93532,8.33558);
\draw [color=c, fill=c] (6.93532,8.22973) rectangle (6.97512,8.33558);
\draw [color=c, fill=c] (6.97512,8.22973) rectangle (7.01493,8.33558);
\draw [color=c, fill=c] (7.01493,8.22973) rectangle (7.05473,8.33558);
\draw [color=c, fill=c] (7.05473,8.22973) rectangle (7.09453,8.33558);
\draw [color=c, fill=c] (7.09453,8.22973) rectangle (7.13433,8.33558);
\draw [color=c, fill=c] (7.13433,8.22973) rectangle (7.17413,8.33558);
\draw [color=c, fill=c] (7.17413,8.22973) rectangle (7.21393,8.33558);
\draw [color=c, fill=c] (7.21393,8.22973) rectangle (7.25373,8.33558);
\draw [color=c, fill=c] (7.25373,8.22973) rectangle (7.29353,8.33558);
\draw [color=c, fill=c] (7.29353,8.22973) rectangle (7.33333,8.33558);
\draw [color=c, fill=c] (7.33333,8.22973) rectangle (7.37313,8.33558);
\draw [color=c, fill=c] (7.37313,8.22973) rectangle (7.41294,8.33558);
\draw [color=c, fill=c] (7.41294,8.22973) rectangle (7.45274,8.33558);
\draw [color=c, fill=c] (7.45274,8.22973) rectangle (7.49254,8.33558);
\draw [color=c, fill=c] (7.49254,8.22973) rectangle (7.53234,8.33558);
\draw [color=c, fill=c] (7.53234,8.22973) rectangle (7.57214,8.33558);
\draw [color=c, fill=c] (7.57214,8.22973) rectangle (7.61194,8.33558);
\draw [color=c, fill=c] (7.61194,8.22973) rectangle (7.65174,8.33558);
\definecolor{c}{rgb}{0,0.0800001,1};
\draw [color=c, fill=c] (7.65174,8.22973) rectangle (7.69154,8.33558);
\draw [color=c, fill=c] (7.69154,8.22973) rectangle (7.73134,8.33558);
\draw [color=c, fill=c] (7.73134,8.22973) rectangle (7.77114,8.33558);
\draw [color=c, fill=c] (7.77114,8.22973) rectangle (7.81095,8.33558);
\draw [color=c, fill=c] (7.81095,8.22973) rectangle (7.85075,8.33558);
\draw [color=c, fill=c] (7.85075,8.22973) rectangle (7.89055,8.33558);
\draw [color=c, fill=c] (7.89055,8.22973) rectangle (7.93035,8.33558);
\draw [color=c, fill=c] (7.93035,8.22973) rectangle (7.97015,8.33558);
\draw [color=c, fill=c] (7.97015,8.22973) rectangle (8.00995,8.33558);
\draw [color=c, fill=c] (8.00995,8.22973) rectangle (8.04975,8.33558);
\draw [color=c, fill=c] (8.04975,8.22973) rectangle (8.08955,8.33558);
\draw [color=c, fill=c] (8.08955,8.22973) rectangle (8.12935,8.33558);
\draw [color=c, fill=c] (8.12935,8.22973) rectangle (8.16915,8.33558);
\draw [color=c, fill=c] (8.16915,8.22973) rectangle (8.20895,8.33558);
\draw [color=c, fill=c] (8.20895,8.22973) rectangle (8.24876,8.33558);
\draw [color=c, fill=c] (8.24876,8.22973) rectangle (8.28856,8.33558);
\draw [color=c, fill=c] (8.28856,8.22973) rectangle (8.32836,8.33558);
\draw [color=c, fill=c] (8.32836,8.22973) rectangle (8.36816,8.33558);
\draw [color=c, fill=c] (8.36816,8.22973) rectangle (8.40796,8.33558);
\draw [color=c, fill=c] (8.40796,8.22973) rectangle (8.44776,8.33558);
\draw [color=c, fill=c] (8.44776,8.22973) rectangle (8.48756,8.33558);
\draw [color=c, fill=c] (8.48756,8.22973) rectangle (8.52736,8.33558);
\draw [color=c, fill=c] (8.52736,8.22973) rectangle (8.56716,8.33558);
\draw [color=c, fill=c] (8.56716,8.22973) rectangle (8.60697,8.33558);
\draw [color=c, fill=c] (8.60697,8.22973) rectangle (8.64677,8.33558);
\draw [color=c, fill=c] (8.64677,8.22973) rectangle (8.68657,8.33558);
\draw [color=c, fill=c] (8.68657,8.22973) rectangle (8.72637,8.33558);
\draw [color=c, fill=c] (8.72637,8.22973) rectangle (8.76617,8.33558);
\draw [color=c, fill=c] (8.76617,8.22973) rectangle (8.80597,8.33558);
\draw [color=c, fill=c] (8.80597,8.22973) rectangle (8.84577,8.33558);
\draw [color=c, fill=c] (8.84577,8.22973) rectangle (8.88557,8.33558);
\draw [color=c, fill=c] (8.88557,8.22973) rectangle (8.92537,8.33558);
\draw [color=c, fill=c] (8.92537,8.22973) rectangle (8.96517,8.33558);
\draw [color=c, fill=c] (8.96517,8.22973) rectangle (9.00498,8.33558);
\draw [color=c, fill=c] (9.00498,8.22973) rectangle (9.04478,8.33558);
\draw [color=c, fill=c] (9.04478,8.22973) rectangle (9.08458,8.33558);
\draw [color=c, fill=c] (9.08458,8.22973) rectangle (9.12438,8.33558);
\draw [color=c, fill=c] (9.12438,8.22973) rectangle (9.16418,8.33558);
\draw [color=c, fill=c] (9.16418,8.22973) rectangle (9.20398,8.33558);
\draw [color=c, fill=c] (9.20398,8.22973) rectangle (9.24378,8.33558);
\draw [color=c, fill=c] (9.24378,8.22973) rectangle (9.28358,8.33558);
\draw [color=c, fill=c] (9.28358,8.22973) rectangle (9.32338,8.33558);
\draw [color=c, fill=c] (9.32338,8.22973) rectangle (9.36318,8.33558);
\draw [color=c, fill=c] (9.36318,8.22973) rectangle (9.40298,8.33558);
\draw [color=c, fill=c] (9.40298,8.22973) rectangle (9.44279,8.33558);
\draw [color=c, fill=c] (9.44279,8.22973) rectangle (9.48259,8.33558);
\draw [color=c, fill=c] (9.48259,8.22973) rectangle (9.52239,8.33558);
\definecolor{c}{rgb}{0,0.266667,1};
\draw [color=c, fill=c] (9.52239,8.22973) rectangle (9.56219,8.33558);
\draw [color=c, fill=c] (9.56219,8.22973) rectangle (9.60199,8.33558);
\draw [color=c, fill=c] (9.60199,8.22973) rectangle (9.64179,8.33558);
\draw [color=c, fill=c] (9.64179,8.22973) rectangle (9.68159,8.33558);
\draw [color=c, fill=c] (9.68159,8.22973) rectangle (9.72139,8.33558);
\draw [color=c, fill=c] (9.72139,8.22973) rectangle (9.76119,8.33558);
\draw [color=c, fill=c] (9.76119,8.22973) rectangle (9.80099,8.33558);
\draw [color=c, fill=c] (9.80099,8.22973) rectangle (9.8408,8.33558);
\draw [color=c, fill=c] (9.8408,8.22973) rectangle (9.8806,8.33558);
\draw [color=c, fill=c] (9.8806,8.22973) rectangle (9.9204,8.33558);
\draw [color=c, fill=c] (9.9204,8.22973) rectangle (9.9602,8.33558);
\draw [color=c, fill=c] (9.9602,8.22973) rectangle (10,8.33558);
\draw [color=c, fill=c] (10,8.22973) rectangle (10.0398,8.33558);
\draw [color=c, fill=c] (10.0398,8.22973) rectangle (10.0796,8.33558);
\draw [color=c, fill=c] (10.0796,8.22973) rectangle (10.1194,8.33558);
\draw [color=c, fill=c] (10.1194,8.22973) rectangle (10.1592,8.33558);
\draw [color=c, fill=c] (10.1592,8.22973) rectangle (10.199,8.33558);
\draw [color=c, fill=c] (10.199,8.22973) rectangle (10.2388,8.33558);
\draw [color=c, fill=c] (10.2388,8.22973) rectangle (10.2786,8.33558);
\draw [color=c, fill=c] (10.2786,8.22973) rectangle (10.3184,8.33558);
\draw [color=c, fill=c] (10.3184,8.22973) rectangle (10.3582,8.33558);
\draw [color=c, fill=c] (10.3582,8.22973) rectangle (10.398,8.33558);
\draw [color=c, fill=c] (10.398,8.22973) rectangle (10.4378,8.33558);
\definecolor{c}{rgb}{0,0.546666,1};
\draw [color=c, fill=c] (10.4378,8.22973) rectangle (10.4776,8.33558);
\draw [color=c, fill=c] (10.4776,8.22973) rectangle (10.5174,8.33558);
\draw [color=c, fill=c] (10.5174,8.22973) rectangle (10.5572,8.33558);
\draw [color=c, fill=c] (10.5572,8.22973) rectangle (10.597,8.33558);
\draw [color=c, fill=c] (10.597,8.22973) rectangle (10.6368,8.33558);
\draw [color=c, fill=c] (10.6368,8.22973) rectangle (10.6766,8.33558);
\draw [color=c, fill=c] (10.6766,8.22973) rectangle (10.7164,8.33558);
\draw [color=c, fill=c] (10.7164,8.22973) rectangle (10.7562,8.33558);
\draw [color=c, fill=c] (10.7562,8.22973) rectangle (10.796,8.33558);
\draw [color=c, fill=c] (10.796,8.22973) rectangle (10.8358,8.33558);
\draw [color=c, fill=c] (10.8358,8.22973) rectangle (10.8756,8.33558);
\draw [color=c, fill=c] (10.8756,8.22973) rectangle (10.9154,8.33558);
\draw [color=c, fill=c] (10.9154,8.22973) rectangle (10.9552,8.33558);
\draw [color=c, fill=c] (10.9552,8.22973) rectangle (10.995,8.33558);
\draw [color=c, fill=c] (10.995,8.22973) rectangle (11.0348,8.33558);
\draw [color=c, fill=c] (11.0348,8.22973) rectangle (11.0746,8.33558);
\draw [color=c, fill=c] (11.0746,8.22973) rectangle (11.1144,8.33558);
\draw [color=c, fill=c] (11.1144,8.22973) rectangle (11.1542,8.33558);
\draw [color=c, fill=c] (11.1542,8.22973) rectangle (11.194,8.33558);
\draw [color=c, fill=c] (11.194,8.22973) rectangle (11.2338,8.33558);
\draw [color=c, fill=c] (11.2338,8.22973) rectangle (11.2736,8.33558);
\draw [color=c, fill=c] (11.2736,8.22973) rectangle (11.3134,8.33558);
\draw [color=c, fill=c] (11.3134,8.22973) rectangle (11.3532,8.33558);
\draw [color=c, fill=c] (11.3532,8.22973) rectangle (11.393,8.33558);
\draw [color=c, fill=c] (11.393,8.22973) rectangle (11.4328,8.33558);
\draw [color=c, fill=c] (11.4328,8.22973) rectangle (11.4726,8.33558);
\draw [color=c, fill=c] (11.4726,8.22973) rectangle (11.5124,8.33558);
\draw [color=c, fill=c] (11.5124,8.22973) rectangle (11.5522,8.33558);
\draw [color=c, fill=c] (11.5522,8.22973) rectangle (11.592,8.33558);
\draw [color=c, fill=c] (11.592,8.22973) rectangle (11.6318,8.33558);
\draw [color=c, fill=c] (11.6318,8.22973) rectangle (11.6716,8.33558);
\draw [color=c, fill=c] (11.6716,8.22973) rectangle (11.7114,8.33558);
\draw [color=c, fill=c] (11.7114,8.22973) rectangle (11.7512,8.33558);
\draw [color=c, fill=c] (11.7512,8.22973) rectangle (11.791,8.33558);
\draw [color=c, fill=c] (11.791,8.22973) rectangle (11.8308,8.33558);
\draw [color=c, fill=c] (11.8308,8.22973) rectangle (11.8706,8.33558);
\draw [color=c, fill=c] (11.8706,8.22973) rectangle (11.9104,8.33558);
\draw [color=c, fill=c] (11.9104,8.22973) rectangle (11.9502,8.33558);
\draw [color=c, fill=c] (11.9502,8.22973) rectangle (11.99,8.33558);
\draw [color=c, fill=c] (11.99,8.22973) rectangle (12.0299,8.33558);
\draw [color=c, fill=c] (12.0299,8.22973) rectangle (12.0697,8.33558);
\draw [color=c, fill=c] (12.0697,8.22973) rectangle (12.1095,8.33558);
\draw [color=c, fill=c] (12.1095,8.22973) rectangle (12.1493,8.33558);
\draw [color=c, fill=c] (12.1493,8.22973) rectangle (12.1891,8.33558);
\draw [color=c, fill=c] (12.1891,8.22973) rectangle (12.2289,8.33558);
\draw [color=c, fill=c] (12.2289,8.22973) rectangle (12.2687,8.33558);
\draw [color=c, fill=c] (12.2687,8.22973) rectangle (12.3085,8.33558);
\draw [color=c, fill=c] (12.3085,8.22973) rectangle (12.3483,8.33558);
\draw [color=c, fill=c] (12.3483,8.22973) rectangle (12.3881,8.33558);
\draw [color=c, fill=c] (12.3881,8.22973) rectangle (12.4279,8.33558);
\draw [color=c, fill=c] (12.4279,8.22973) rectangle (12.4677,8.33558);
\draw [color=c, fill=c] (12.4677,8.22973) rectangle (12.5075,8.33558);
\definecolor{c}{rgb}{0,0.733333,1};
\draw [color=c, fill=c] (12.5075,8.22973) rectangle (12.5473,8.33558);
\draw [color=c, fill=c] (12.5473,8.22973) rectangle (12.5871,8.33558);
\draw [color=c, fill=c] (12.5871,8.22973) rectangle (12.6269,8.33558);
\draw [color=c, fill=c] (12.6269,8.22973) rectangle (12.6667,8.33558);
\draw [color=c, fill=c] (12.6667,8.22973) rectangle (12.7065,8.33558);
\draw [color=c, fill=c] (12.7065,8.22973) rectangle (12.7463,8.33558);
\draw [color=c, fill=c] (12.7463,8.22973) rectangle (12.7861,8.33558);
\draw [color=c, fill=c] (12.7861,8.22973) rectangle (12.8259,8.33558);
\draw [color=c, fill=c] (12.8259,8.22973) rectangle (12.8657,8.33558);
\draw [color=c, fill=c] (12.8657,8.22973) rectangle (12.9055,8.33558);
\draw [color=c, fill=c] (12.9055,8.22973) rectangle (12.9453,8.33558);
\draw [color=c, fill=c] (12.9453,8.22973) rectangle (12.9851,8.33558);
\draw [color=c, fill=c] (12.9851,8.22973) rectangle (13.0249,8.33558);
\draw [color=c, fill=c] (13.0249,8.22973) rectangle (13.0647,8.33558);
\draw [color=c, fill=c] (13.0647,8.22973) rectangle (13.1045,8.33558);
\draw [color=c, fill=c] (13.1045,8.22973) rectangle (13.1443,8.33558);
\draw [color=c, fill=c] (13.1443,8.22973) rectangle (13.1841,8.33558);
\draw [color=c, fill=c] (13.1841,8.22973) rectangle (13.2239,8.33558);
\draw [color=c, fill=c] (13.2239,8.22973) rectangle (13.2637,8.33558);
\draw [color=c, fill=c] (13.2637,8.22973) rectangle (13.3035,8.33558);
\draw [color=c, fill=c] (13.3035,8.22973) rectangle (13.3433,8.33558);
\draw [color=c, fill=c] (13.3433,8.22973) rectangle (13.3831,8.33558);
\draw [color=c, fill=c] (13.3831,8.22973) rectangle (13.4229,8.33558);
\draw [color=c, fill=c] (13.4229,8.22973) rectangle (13.4627,8.33558);
\draw [color=c, fill=c] (13.4627,8.22973) rectangle (13.5025,8.33558);
\draw [color=c, fill=c] (13.5025,8.22973) rectangle (13.5423,8.33558);
\draw [color=c, fill=c] (13.5423,8.22973) rectangle (13.5821,8.33558);
\draw [color=c, fill=c] (13.5821,8.22973) rectangle (13.6219,8.33558);
\draw [color=c, fill=c] (13.6219,8.22973) rectangle (13.6617,8.33558);
\draw [color=c, fill=c] (13.6617,8.22973) rectangle (13.7015,8.33558);
\draw [color=c, fill=c] (13.7015,8.22973) rectangle (13.7413,8.33558);
\draw [color=c, fill=c] (13.7413,8.22973) rectangle (13.7811,8.33558);
\draw [color=c, fill=c] (13.7811,8.22973) rectangle (13.8209,8.33558);
\draw [color=c, fill=c] (13.8209,8.22973) rectangle (13.8607,8.33558);
\draw [color=c, fill=c] (13.8607,8.22973) rectangle (13.9005,8.33558);
\draw [color=c, fill=c] (13.9005,8.22973) rectangle (13.9403,8.33558);
\draw [color=c, fill=c] (13.9403,8.22973) rectangle (13.9801,8.33558);
\draw [color=c, fill=c] (13.9801,8.22973) rectangle (14.0199,8.33558);
\draw [color=c, fill=c] (14.0199,8.22973) rectangle (14.0597,8.33558);
\draw [color=c, fill=c] (14.0597,8.22973) rectangle (14.0995,8.33558);
\draw [color=c, fill=c] (14.0995,8.22973) rectangle (14.1393,8.33558);
\draw [color=c, fill=c] (14.1393,8.22973) rectangle (14.1791,8.33558);
\draw [color=c, fill=c] (14.1791,8.22973) rectangle (14.2189,8.33558);
\draw [color=c, fill=c] (14.2189,8.22973) rectangle (14.2587,8.33558);
\draw [color=c, fill=c] (14.2587,8.22973) rectangle (14.2985,8.33558);
\draw [color=c, fill=c] (14.2985,8.22973) rectangle (14.3383,8.33558);
\draw [color=c, fill=c] (14.3383,8.22973) rectangle (14.3781,8.33558);
\draw [color=c, fill=c] (14.3781,8.22973) rectangle (14.4179,8.33558);
\draw [color=c, fill=c] (14.4179,8.22973) rectangle (14.4577,8.33558);
\draw [color=c, fill=c] (14.4577,8.22973) rectangle (14.4975,8.33558);
\draw [color=c, fill=c] (14.4975,8.22973) rectangle (14.5373,8.33558);
\draw [color=c, fill=c] (14.5373,8.22973) rectangle (14.5771,8.33558);
\draw [color=c, fill=c] (14.5771,8.22973) rectangle (14.6169,8.33558);
\draw [color=c, fill=c] (14.6169,8.22973) rectangle (14.6567,8.33558);
\draw [color=c, fill=c] (14.6567,8.22973) rectangle (14.6965,8.33558);
\draw [color=c, fill=c] (14.6965,8.22973) rectangle (14.7363,8.33558);
\draw [color=c, fill=c] (14.7363,8.22973) rectangle (14.7761,8.33558);
\draw [color=c, fill=c] (14.7761,8.22973) rectangle (14.8159,8.33558);
\draw [color=c, fill=c] (14.8159,8.22973) rectangle (14.8557,8.33558);
\draw [color=c, fill=c] (14.8557,8.22973) rectangle (14.8955,8.33558);
\draw [color=c, fill=c] (14.8955,8.22973) rectangle (14.9353,8.33558);
\draw [color=c, fill=c] (14.9353,8.22973) rectangle (14.9751,8.33558);
\draw [color=c, fill=c] (14.9751,8.22973) rectangle (15.0149,8.33558);
\draw [color=c, fill=c] (15.0149,8.22973) rectangle (15.0547,8.33558);
\draw [color=c, fill=c] (15.0547,8.22973) rectangle (15.0945,8.33558);
\draw [color=c, fill=c] (15.0945,8.22973) rectangle (15.1343,8.33558);
\draw [color=c, fill=c] (15.1343,8.22973) rectangle (15.1741,8.33558);
\draw [color=c, fill=c] (15.1741,8.22973) rectangle (15.2139,8.33558);
\draw [color=c, fill=c] (15.2139,8.22973) rectangle (15.2537,8.33558);
\draw [color=c, fill=c] (15.2537,8.22973) rectangle (15.2935,8.33558);
\draw [color=c, fill=c] (15.2935,8.22973) rectangle (15.3333,8.33558);
\draw [color=c, fill=c] (15.3333,8.22973) rectangle (15.3731,8.33558);
\draw [color=c, fill=c] (15.3731,8.22973) rectangle (15.4129,8.33558);
\draw [color=c, fill=c] (15.4129,8.22973) rectangle (15.4527,8.33558);
\draw [color=c, fill=c] (15.4527,8.22973) rectangle (15.4925,8.33558);
\draw [color=c, fill=c] (15.4925,8.22973) rectangle (15.5323,8.33558);
\draw [color=c, fill=c] (15.5323,8.22973) rectangle (15.5721,8.33558);
\draw [color=c, fill=c] (15.5721,8.22973) rectangle (15.6119,8.33558);
\draw [color=c, fill=c] (15.6119,8.22973) rectangle (15.6517,8.33558);
\draw [color=c, fill=c] (15.6517,8.22973) rectangle (15.6915,8.33558);
\draw [color=c, fill=c] (15.6915,8.22973) rectangle (15.7313,8.33558);
\draw [color=c, fill=c] (15.7313,8.22973) rectangle (15.7711,8.33558);
\draw [color=c, fill=c] (15.7711,8.22973) rectangle (15.8109,8.33558);
\draw [color=c, fill=c] (15.8109,8.22973) rectangle (15.8507,8.33558);
\draw [color=c, fill=c] (15.8507,8.22973) rectangle (15.8905,8.33558);
\draw [color=c, fill=c] (15.8905,8.22973) rectangle (15.9303,8.33558);
\draw [color=c, fill=c] (15.9303,8.22973) rectangle (15.9701,8.33558);
\draw [color=c, fill=c] (15.9701,8.22973) rectangle (16.01,8.33558);
\draw [color=c, fill=c] (16.01,8.22973) rectangle (16.0498,8.33558);
\draw [color=c, fill=c] (16.0498,8.22973) rectangle (16.0896,8.33558);
\draw [color=c, fill=c] (16.0896,8.22973) rectangle (16.1294,8.33558);
\draw [color=c, fill=c] (16.1294,8.22973) rectangle (16.1692,8.33558);
\draw [color=c, fill=c] (16.1692,8.22973) rectangle (16.209,8.33558);
\draw [color=c, fill=c] (16.209,8.22973) rectangle (16.2488,8.33558);
\draw [color=c, fill=c] (16.2488,8.22973) rectangle (16.2886,8.33558);
\draw [color=c, fill=c] (16.2886,8.22973) rectangle (16.3284,8.33558);
\draw [color=c, fill=c] (16.3284,8.22973) rectangle (16.3682,8.33558);
\draw [color=c, fill=c] (16.3682,8.22973) rectangle (16.408,8.33558);
\draw [color=c, fill=c] (16.408,8.22973) rectangle (16.4478,8.33558);
\draw [color=c, fill=c] (16.4478,8.22973) rectangle (16.4876,8.33558);
\draw [color=c, fill=c] (16.4876,8.22973) rectangle (16.5274,8.33558);
\draw [color=c, fill=c] (16.5274,8.22973) rectangle (16.5672,8.33558);
\draw [color=c, fill=c] (16.5672,8.22973) rectangle (16.607,8.33558);
\draw [color=c, fill=c] (16.607,8.22973) rectangle (16.6468,8.33558);
\draw [color=c, fill=c] (16.6468,8.22973) rectangle (16.6866,8.33558);
\draw [color=c, fill=c] (16.6866,8.22973) rectangle (16.7264,8.33558);
\draw [color=c, fill=c] (16.7264,8.22973) rectangle (16.7662,8.33558);
\draw [color=c, fill=c] (16.7662,8.22973) rectangle (16.806,8.33558);
\draw [color=c, fill=c] (16.806,8.22973) rectangle (16.8458,8.33558);
\draw [color=c, fill=c] (16.8458,8.22973) rectangle (16.8856,8.33558);
\draw [color=c, fill=c] (16.8856,8.22973) rectangle (16.9254,8.33558);
\draw [color=c, fill=c] (16.9254,8.22973) rectangle (16.9652,8.33558);
\draw [color=c, fill=c] (16.9652,8.22973) rectangle (17.005,8.33558);
\draw [color=c, fill=c] (17.005,8.22973) rectangle (17.0448,8.33558);
\draw [color=c, fill=c] (17.0448,8.22973) rectangle (17.0846,8.33558);
\draw [color=c, fill=c] (17.0846,8.22973) rectangle (17.1244,8.33558);
\draw [color=c, fill=c] (17.1244,8.22973) rectangle (17.1642,8.33558);
\draw [color=c, fill=c] (17.1642,8.22973) rectangle (17.204,8.33558);
\draw [color=c, fill=c] (17.204,8.22973) rectangle (17.2438,8.33558);
\draw [color=c, fill=c] (17.2438,8.22973) rectangle (17.2836,8.33558);
\draw [color=c, fill=c] (17.2836,8.22973) rectangle (17.3234,8.33558);
\draw [color=c, fill=c] (17.3234,8.22973) rectangle (17.3632,8.33558);
\draw [color=c, fill=c] (17.3632,8.22973) rectangle (17.403,8.33558);
\draw [color=c, fill=c] (17.403,8.22973) rectangle (17.4428,8.33558);
\draw [color=c, fill=c] (17.4428,8.22973) rectangle (17.4826,8.33558);
\draw [color=c, fill=c] (17.4826,8.22973) rectangle (17.5224,8.33558);
\draw [color=c, fill=c] (17.5224,8.22973) rectangle (17.5622,8.33558);
\draw [color=c, fill=c] (17.5622,8.22973) rectangle (17.602,8.33558);
\draw [color=c, fill=c] (17.602,8.22973) rectangle (17.6418,8.33558);
\draw [color=c, fill=c] (17.6418,8.22973) rectangle (17.6816,8.33558);
\draw [color=c, fill=c] (17.6816,8.22973) rectangle (17.7214,8.33558);
\draw [color=c, fill=c] (17.7214,8.22973) rectangle (17.7612,8.33558);
\draw [color=c, fill=c] (17.7612,8.22973) rectangle (17.801,8.33558);
\draw [color=c, fill=c] (17.801,8.22973) rectangle (17.8408,8.33558);
\draw [color=c, fill=c] (17.8408,8.22973) rectangle (17.8806,8.33558);
\draw [color=c, fill=c] (17.8806,8.22973) rectangle (17.9204,8.33558);
\draw [color=c, fill=c] (17.9204,8.22973) rectangle (17.9602,8.33558);
\draw [color=c, fill=c] (17.9602,8.22973) rectangle (18,8.33558);
\definecolor{c}{rgb}{0.2,0,1};
\draw [color=c, fill=c] (2,8.33558) rectangle (2.0398,8.44143);
\draw [color=c, fill=c] (2.0398,8.33558) rectangle (2.0796,8.44143);
\draw [color=c, fill=c] (2.0796,8.33558) rectangle (2.1194,8.44143);
\draw [color=c, fill=c] (2.1194,8.33558) rectangle (2.1592,8.44143);
\draw [color=c, fill=c] (2.1592,8.33558) rectangle (2.19901,8.44143);
\draw [color=c, fill=c] (2.19901,8.33558) rectangle (2.23881,8.44143);
\draw [color=c, fill=c] (2.23881,8.33558) rectangle (2.27861,8.44143);
\draw [color=c, fill=c] (2.27861,8.33558) rectangle (2.31841,8.44143);
\draw [color=c, fill=c] (2.31841,8.33558) rectangle (2.35821,8.44143);
\draw [color=c, fill=c] (2.35821,8.33558) rectangle (2.39801,8.44143);
\draw [color=c, fill=c] (2.39801,8.33558) rectangle (2.43781,8.44143);
\draw [color=c, fill=c] (2.43781,8.33558) rectangle (2.47761,8.44143);
\draw [color=c, fill=c] (2.47761,8.33558) rectangle (2.51741,8.44143);
\draw [color=c, fill=c] (2.51741,8.33558) rectangle (2.55721,8.44143);
\draw [color=c, fill=c] (2.55721,8.33558) rectangle (2.59702,8.44143);
\draw [color=c, fill=c] (2.59702,8.33558) rectangle (2.63682,8.44143);
\draw [color=c, fill=c] (2.63682,8.33558) rectangle (2.67662,8.44143);
\draw [color=c, fill=c] (2.67662,8.33558) rectangle (2.71642,8.44143);
\draw [color=c, fill=c] (2.71642,8.33558) rectangle (2.75622,8.44143);
\draw [color=c, fill=c] (2.75622,8.33558) rectangle (2.79602,8.44143);
\draw [color=c, fill=c] (2.79602,8.33558) rectangle (2.83582,8.44143);
\draw [color=c, fill=c] (2.83582,8.33558) rectangle (2.87562,8.44143);
\draw [color=c, fill=c] (2.87562,8.33558) rectangle (2.91542,8.44143);
\draw [color=c, fill=c] (2.91542,8.33558) rectangle (2.95522,8.44143);
\draw [color=c, fill=c] (2.95522,8.33558) rectangle (2.99502,8.44143);
\draw [color=c, fill=c] (2.99502,8.33558) rectangle (3.03483,8.44143);
\draw [color=c, fill=c] (3.03483,8.33558) rectangle (3.07463,8.44143);
\draw [color=c, fill=c] (3.07463,8.33558) rectangle (3.11443,8.44143);
\draw [color=c, fill=c] (3.11443,8.33558) rectangle (3.15423,8.44143);
\draw [color=c, fill=c] (3.15423,8.33558) rectangle (3.19403,8.44143);
\draw [color=c, fill=c] (3.19403,8.33558) rectangle (3.23383,8.44143);
\draw [color=c, fill=c] (3.23383,8.33558) rectangle (3.27363,8.44143);
\draw [color=c, fill=c] (3.27363,8.33558) rectangle (3.31343,8.44143);
\draw [color=c, fill=c] (3.31343,8.33558) rectangle (3.35323,8.44143);
\draw [color=c, fill=c] (3.35323,8.33558) rectangle (3.39303,8.44143);
\draw [color=c, fill=c] (3.39303,8.33558) rectangle (3.43284,8.44143);
\draw [color=c, fill=c] (3.43284,8.33558) rectangle (3.47264,8.44143);
\draw [color=c, fill=c] (3.47264,8.33558) rectangle (3.51244,8.44143);
\draw [color=c, fill=c] (3.51244,8.33558) rectangle (3.55224,8.44143);
\draw [color=c, fill=c] (3.55224,8.33558) rectangle (3.59204,8.44143);
\draw [color=c, fill=c] (3.59204,8.33558) rectangle (3.63184,8.44143);
\draw [color=c, fill=c] (3.63184,8.33558) rectangle (3.67164,8.44143);
\draw [color=c, fill=c] (3.67164,8.33558) rectangle (3.71144,8.44143);
\draw [color=c, fill=c] (3.71144,8.33558) rectangle (3.75124,8.44143);
\draw [color=c, fill=c] (3.75124,8.33558) rectangle (3.79104,8.44143);
\draw [color=c, fill=c] (3.79104,8.33558) rectangle (3.83085,8.44143);
\draw [color=c, fill=c] (3.83085,8.33558) rectangle (3.87065,8.44143);
\draw [color=c, fill=c] (3.87065,8.33558) rectangle (3.91045,8.44143);
\draw [color=c, fill=c] (3.91045,8.33558) rectangle (3.95025,8.44143);
\draw [color=c, fill=c] (3.95025,8.33558) rectangle (3.99005,8.44143);
\draw [color=c, fill=c] (3.99005,8.33558) rectangle (4.02985,8.44143);
\draw [color=c, fill=c] (4.02985,8.33558) rectangle (4.06965,8.44143);
\draw [color=c, fill=c] (4.06965,8.33558) rectangle (4.10945,8.44143);
\draw [color=c, fill=c] (4.10945,8.33558) rectangle (4.14925,8.44143);
\draw [color=c, fill=c] (4.14925,8.33558) rectangle (4.18905,8.44143);
\draw [color=c, fill=c] (4.18905,8.33558) rectangle (4.22886,8.44143);
\draw [color=c, fill=c] (4.22886,8.33558) rectangle (4.26866,8.44143);
\draw [color=c, fill=c] (4.26866,8.33558) rectangle (4.30846,8.44143);
\draw [color=c, fill=c] (4.30846,8.33558) rectangle (4.34826,8.44143);
\draw [color=c, fill=c] (4.34826,8.33558) rectangle (4.38806,8.44143);
\draw [color=c, fill=c] (4.38806,8.33558) rectangle (4.42786,8.44143);
\draw [color=c, fill=c] (4.42786,8.33558) rectangle (4.46766,8.44143);
\draw [color=c, fill=c] (4.46766,8.33558) rectangle (4.50746,8.44143);
\draw [color=c, fill=c] (4.50746,8.33558) rectangle (4.54726,8.44143);
\draw [color=c, fill=c] (4.54726,8.33558) rectangle (4.58706,8.44143);
\draw [color=c, fill=c] (4.58706,8.33558) rectangle (4.62687,8.44143);
\draw [color=c, fill=c] (4.62687,8.33558) rectangle (4.66667,8.44143);
\draw [color=c, fill=c] (4.66667,8.33558) rectangle (4.70647,8.44143);
\draw [color=c, fill=c] (4.70647,8.33558) rectangle (4.74627,8.44143);
\draw [color=c, fill=c] (4.74627,8.33558) rectangle (4.78607,8.44143);
\draw [color=c, fill=c] (4.78607,8.33558) rectangle (4.82587,8.44143);
\draw [color=c, fill=c] (4.82587,8.33558) rectangle (4.86567,8.44143);
\draw [color=c, fill=c] (4.86567,8.33558) rectangle (4.90547,8.44143);
\draw [color=c, fill=c] (4.90547,8.33558) rectangle (4.94527,8.44143);
\draw [color=c, fill=c] (4.94527,8.33558) rectangle (4.98507,8.44143);
\draw [color=c, fill=c] (4.98507,8.33558) rectangle (5.02488,8.44143);
\draw [color=c, fill=c] (5.02488,8.33558) rectangle (5.06468,8.44143);
\draw [color=c, fill=c] (5.06468,8.33558) rectangle (5.10448,8.44143);
\draw [color=c, fill=c] (5.10448,8.33558) rectangle (5.14428,8.44143);
\draw [color=c, fill=c] (5.14428,8.33558) rectangle (5.18408,8.44143);
\draw [color=c, fill=c] (5.18408,8.33558) rectangle (5.22388,8.44143);
\draw [color=c, fill=c] (5.22388,8.33558) rectangle (5.26368,8.44143);
\draw [color=c, fill=c] (5.26368,8.33558) rectangle (5.30348,8.44143);
\draw [color=c, fill=c] (5.30348,8.33558) rectangle (5.34328,8.44143);
\draw [color=c, fill=c] (5.34328,8.33558) rectangle (5.38308,8.44143);
\draw [color=c, fill=c] (5.38308,8.33558) rectangle (5.42289,8.44143);
\draw [color=c, fill=c] (5.42289,8.33558) rectangle (5.46269,8.44143);
\draw [color=c, fill=c] (5.46269,8.33558) rectangle (5.50249,8.44143);
\draw [color=c, fill=c] (5.50249,8.33558) rectangle (5.54229,8.44143);
\draw [color=c, fill=c] (5.54229,8.33558) rectangle (5.58209,8.44143);
\draw [color=c, fill=c] (5.58209,8.33558) rectangle (5.62189,8.44143);
\draw [color=c, fill=c] (5.62189,8.33558) rectangle (5.66169,8.44143);
\draw [color=c, fill=c] (5.66169,8.33558) rectangle (5.70149,8.44143);
\draw [color=c, fill=c] (5.70149,8.33558) rectangle (5.74129,8.44143);
\draw [color=c, fill=c] (5.74129,8.33558) rectangle (5.78109,8.44143);
\draw [color=c, fill=c] (5.78109,8.33558) rectangle (5.8209,8.44143);
\draw [color=c, fill=c] (5.8209,8.33558) rectangle (5.8607,8.44143);
\draw [color=c, fill=c] (5.8607,8.33558) rectangle (5.9005,8.44143);
\draw [color=c, fill=c] (5.9005,8.33558) rectangle (5.9403,8.44143);
\draw [color=c, fill=c] (5.9403,8.33558) rectangle (5.9801,8.44143);
\draw [color=c, fill=c] (5.9801,8.33558) rectangle (6.0199,8.44143);
\draw [color=c, fill=c] (6.0199,8.33558) rectangle (6.0597,8.44143);
\draw [color=c, fill=c] (6.0597,8.33558) rectangle (6.0995,8.44143);
\draw [color=c, fill=c] (6.0995,8.33558) rectangle (6.1393,8.44143);
\draw [color=c, fill=c] (6.1393,8.33558) rectangle (6.1791,8.44143);
\draw [color=c, fill=c] (6.1791,8.33558) rectangle (6.21891,8.44143);
\draw [color=c, fill=c] (6.21891,8.33558) rectangle (6.25871,8.44143);
\draw [color=c, fill=c] (6.25871,8.33558) rectangle (6.29851,8.44143);
\draw [color=c, fill=c] (6.29851,8.33558) rectangle (6.33831,8.44143);
\draw [color=c, fill=c] (6.33831,8.33558) rectangle (6.37811,8.44143);
\draw [color=c, fill=c] (6.37811,8.33558) rectangle (6.41791,8.44143);
\draw [color=c, fill=c] (6.41791,8.33558) rectangle (6.45771,8.44143);
\draw [color=c, fill=c] (6.45771,8.33558) rectangle (6.49751,8.44143);
\draw [color=c, fill=c] (6.49751,8.33558) rectangle (6.53731,8.44143);
\draw [color=c, fill=c] (6.53731,8.33558) rectangle (6.57711,8.44143);
\draw [color=c, fill=c] (6.57711,8.33558) rectangle (6.61692,8.44143);
\draw [color=c, fill=c] (6.61692,8.33558) rectangle (6.65672,8.44143);
\draw [color=c, fill=c] (6.65672,8.33558) rectangle (6.69652,8.44143);
\draw [color=c, fill=c] (6.69652,8.33558) rectangle (6.73632,8.44143);
\draw [color=c, fill=c] (6.73632,8.33558) rectangle (6.77612,8.44143);
\draw [color=c, fill=c] (6.77612,8.33558) rectangle (6.81592,8.44143);
\draw [color=c, fill=c] (6.81592,8.33558) rectangle (6.85572,8.44143);
\draw [color=c, fill=c] (6.85572,8.33558) rectangle (6.89552,8.44143);
\draw [color=c, fill=c] (6.89552,8.33558) rectangle (6.93532,8.44143);
\draw [color=c, fill=c] (6.93532,8.33558) rectangle (6.97512,8.44143);
\draw [color=c, fill=c] (6.97512,8.33558) rectangle (7.01493,8.44143);
\draw [color=c, fill=c] (7.01493,8.33558) rectangle (7.05473,8.44143);
\draw [color=c, fill=c] (7.05473,8.33558) rectangle (7.09453,8.44143);
\draw [color=c, fill=c] (7.09453,8.33558) rectangle (7.13433,8.44143);
\draw [color=c, fill=c] (7.13433,8.33558) rectangle (7.17413,8.44143);
\draw [color=c, fill=c] (7.17413,8.33558) rectangle (7.21393,8.44143);
\draw [color=c, fill=c] (7.21393,8.33558) rectangle (7.25373,8.44143);
\draw [color=c, fill=c] (7.25373,8.33558) rectangle (7.29353,8.44143);
\draw [color=c, fill=c] (7.29353,8.33558) rectangle (7.33333,8.44143);
\draw [color=c, fill=c] (7.33333,8.33558) rectangle (7.37313,8.44143);
\draw [color=c, fill=c] (7.37313,8.33558) rectangle (7.41294,8.44143);
\draw [color=c, fill=c] (7.41294,8.33558) rectangle (7.45274,8.44143);
\draw [color=c, fill=c] (7.45274,8.33558) rectangle (7.49254,8.44143);
\draw [color=c, fill=c] (7.49254,8.33558) rectangle (7.53234,8.44143);
\draw [color=c, fill=c] (7.53234,8.33558) rectangle (7.57214,8.44143);
\draw [color=c, fill=c] (7.57214,8.33558) rectangle (7.61194,8.44143);
\definecolor{c}{rgb}{0,0.0800001,1};
\draw [color=c, fill=c] (7.61194,8.33558) rectangle (7.65174,8.44143);
\draw [color=c, fill=c] (7.65174,8.33558) rectangle (7.69154,8.44143);
\draw [color=c, fill=c] (7.69154,8.33558) rectangle (7.73134,8.44143);
\draw [color=c, fill=c] (7.73134,8.33558) rectangle (7.77114,8.44143);
\draw [color=c, fill=c] (7.77114,8.33558) rectangle (7.81095,8.44143);
\draw [color=c, fill=c] (7.81095,8.33558) rectangle (7.85075,8.44143);
\draw [color=c, fill=c] (7.85075,8.33558) rectangle (7.89055,8.44143);
\draw [color=c, fill=c] (7.89055,8.33558) rectangle (7.93035,8.44143);
\draw [color=c, fill=c] (7.93035,8.33558) rectangle (7.97015,8.44143);
\draw [color=c, fill=c] (7.97015,8.33558) rectangle (8.00995,8.44143);
\draw [color=c, fill=c] (8.00995,8.33558) rectangle (8.04975,8.44143);
\draw [color=c, fill=c] (8.04975,8.33558) rectangle (8.08955,8.44143);
\draw [color=c, fill=c] (8.08955,8.33558) rectangle (8.12935,8.44143);
\draw [color=c, fill=c] (8.12935,8.33558) rectangle (8.16915,8.44143);
\draw [color=c, fill=c] (8.16915,8.33558) rectangle (8.20895,8.44143);
\draw [color=c, fill=c] (8.20895,8.33558) rectangle (8.24876,8.44143);
\draw [color=c, fill=c] (8.24876,8.33558) rectangle (8.28856,8.44143);
\draw [color=c, fill=c] (8.28856,8.33558) rectangle (8.32836,8.44143);
\draw [color=c, fill=c] (8.32836,8.33558) rectangle (8.36816,8.44143);
\draw [color=c, fill=c] (8.36816,8.33558) rectangle (8.40796,8.44143);
\draw [color=c, fill=c] (8.40796,8.33558) rectangle (8.44776,8.44143);
\draw [color=c, fill=c] (8.44776,8.33558) rectangle (8.48756,8.44143);
\draw [color=c, fill=c] (8.48756,8.33558) rectangle (8.52736,8.44143);
\draw [color=c, fill=c] (8.52736,8.33558) rectangle (8.56716,8.44143);
\draw [color=c, fill=c] (8.56716,8.33558) rectangle (8.60697,8.44143);
\draw [color=c, fill=c] (8.60697,8.33558) rectangle (8.64677,8.44143);
\draw [color=c, fill=c] (8.64677,8.33558) rectangle (8.68657,8.44143);
\draw [color=c, fill=c] (8.68657,8.33558) rectangle (8.72637,8.44143);
\draw [color=c, fill=c] (8.72637,8.33558) rectangle (8.76617,8.44143);
\draw [color=c, fill=c] (8.76617,8.33558) rectangle (8.80597,8.44143);
\draw [color=c, fill=c] (8.80597,8.33558) rectangle (8.84577,8.44143);
\draw [color=c, fill=c] (8.84577,8.33558) rectangle (8.88557,8.44143);
\draw [color=c, fill=c] (8.88557,8.33558) rectangle (8.92537,8.44143);
\draw [color=c, fill=c] (8.92537,8.33558) rectangle (8.96517,8.44143);
\draw [color=c, fill=c] (8.96517,8.33558) rectangle (9.00498,8.44143);
\draw [color=c, fill=c] (9.00498,8.33558) rectangle (9.04478,8.44143);
\draw [color=c, fill=c] (9.04478,8.33558) rectangle (9.08458,8.44143);
\draw [color=c, fill=c] (9.08458,8.33558) rectangle (9.12438,8.44143);
\draw [color=c, fill=c] (9.12438,8.33558) rectangle (9.16418,8.44143);
\draw [color=c, fill=c] (9.16418,8.33558) rectangle (9.20398,8.44143);
\draw [color=c, fill=c] (9.20398,8.33558) rectangle (9.24378,8.44143);
\draw [color=c, fill=c] (9.24378,8.33558) rectangle (9.28358,8.44143);
\draw [color=c, fill=c] (9.28358,8.33558) rectangle (9.32338,8.44143);
\draw [color=c, fill=c] (9.32338,8.33558) rectangle (9.36318,8.44143);
\draw [color=c, fill=c] (9.36318,8.33558) rectangle (9.40298,8.44143);
\draw [color=c, fill=c] (9.40298,8.33558) rectangle (9.44279,8.44143);
\draw [color=c, fill=c] (9.44279,8.33558) rectangle (9.48259,8.44143);
\draw [color=c, fill=c] (9.48259,8.33558) rectangle (9.52239,8.44143);
\definecolor{c}{rgb}{0,0.266667,1};
\draw [color=c, fill=c] (9.52239,8.33558) rectangle (9.56219,8.44143);
\draw [color=c, fill=c] (9.56219,8.33558) rectangle (9.60199,8.44143);
\draw [color=c, fill=c] (9.60199,8.33558) rectangle (9.64179,8.44143);
\draw [color=c, fill=c] (9.64179,8.33558) rectangle (9.68159,8.44143);
\draw [color=c, fill=c] (9.68159,8.33558) rectangle (9.72139,8.44143);
\draw [color=c, fill=c] (9.72139,8.33558) rectangle (9.76119,8.44143);
\draw [color=c, fill=c] (9.76119,8.33558) rectangle (9.80099,8.44143);
\draw [color=c, fill=c] (9.80099,8.33558) rectangle (9.8408,8.44143);
\draw [color=c, fill=c] (9.8408,8.33558) rectangle (9.8806,8.44143);
\draw [color=c, fill=c] (9.8806,8.33558) rectangle (9.9204,8.44143);
\draw [color=c, fill=c] (9.9204,8.33558) rectangle (9.9602,8.44143);
\draw [color=c, fill=c] (9.9602,8.33558) rectangle (10,8.44143);
\draw [color=c, fill=c] (10,8.33558) rectangle (10.0398,8.44143);
\draw [color=c, fill=c] (10.0398,8.33558) rectangle (10.0796,8.44143);
\draw [color=c, fill=c] (10.0796,8.33558) rectangle (10.1194,8.44143);
\draw [color=c, fill=c] (10.1194,8.33558) rectangle (10.1592,8.44143);
\draw [color=c, fill=c] (10.1592,8.33558) rectangle (10.199,8.44143);
\draw [color=c, fill=c] (10.199,8.33558) rectangle (10.2388,8.44143);
\draw [color=c, fill=c] (10.2388,8.33558) rectangle (10.2786,8.44143);
\draw [color=c, fill=c] (10.2786,8.33558) rectangle (10.3184,8.44143);
\draw [color=c, fill=c] (10.3184,8.33558) rectangle (10.3582,8.44143);
\draw [color=c, fill=c] (10.3582,8.33558) rectangle (10.398,8.44143);
\draw [color=c, fill=c] (10.398,8.33558) rectangle (10.4378,8.44143);
\draw [color=c, fill=c] (10.4378,8.33558) rectangle (10.4776,8.44143);
\definecolor{c}{rgb}{0,0.546666,1};
\draw [color=c, fill=c] (10.4776,8.33558) rectangle (10.5174,8.44143);
\draw [color=c, fill=c] (10.5174,8.33558) rectangle (10.5572,8.44143);
\draw [color=c, fill=c] (10.5572,8.33558) rectangle (10.597,8.44143);
\draw [color=c, fill=c] (10.597,8.33558) rectangle (10.6368,8.44143);
\draw [color=c, fill=c] (10.6368,8.33558) rectangle (10.6766,8.44143);
\draw [color=c, fill=c] (10.6766,8.33558) rectangle (10.7164,8.44143);
\draw [color=c, fill=c] (10.7164,8.33558) rectangle (10.7562,8.44143);
\draw [color=c, fill=c] (10.7562,8.33558) rectangle (10.796,8.44143);
\draw [color=c, fill=c] (10.796,8.33558) rectangle (10.8358,8.44143);
\draw [color=c, fill=c] (10.8358,8.33558) rectangle (10.8756,8.44143);
\draw [color=c, fill=c] (10.8756,8.33558) rectangle (10.9154,8.44143);
\draw [color=c, fill=c] (10.9154,8.33558) rectangle (10.9552,8.44143);
\draw [color=c, fill=c] (10.9552,8.33558) rectangle (10.995,8.44143);
\draw [color=c, fill=c] (10.995,8.33558) rectangle (11.0348,8.44143);
\draw [color=c, fill=c] (11.0348,8.33558) rectangle (11.0746,8.44143);
\draw [color=c, fill=c] (11.0746,8.33558) rectangle (11.1144,8.44143);
\draw [color=c, fill=c] (11.1144,8.33558) rectangle (11.1542,8.44143);
\draw [color=c, fill=c] (11.1542,8.33558) rectangle (11.194,8.44143);
\draw [color=c, fill=c] (11.194,8.33558) rectangle (11.2338,8.44143);
\draw [color=c, fill=c] (11.2338,8.33558) rectangle (11.2736,8.44143);
\draw [color=c, fill=c] (11.2736,8.33558) rectangle (11.3134,8.44143);
\draw [color=c, fill=c] (11.3134,8.33558) rectangle (11.3532,8.44143);
\draw [color=c, fill=c] (11.3532,8.33558) rectangle (11.393,8.44143);
\draw [color=c, fill=c] (11.393,8.33558) rectangle (11.4328,8.44143);
\draw [color=c, fill=c] (11.4328,8.33558) rectangle (11.4726,8.44143);
\draw [color=c, fill=c] (11.4726,8.33558) rectangle (11.5124,8.44143);
\draw [color=c, fill=c] (11.5124,8.33558) rectangle (11.5522,8.44143);
\draw [color=c, fill=c] (11.5522,8.33558) rectangle (11.592,8.44143);
\draw [color=c, fill=c] (11.592,8.33558) rectangle (11.6318,8.44143);
\draw [color=c, fill=c] (11.6318,8.33558) rectangle (11.6716,8.44143);
\draw [color=c, fill=c] (11.6716,8.33558) rectangle (11.7114,8.44143);
\draw [color=c, fill=c] (11.7114,8.33558) rectangle (11.7512,8.44143);
\draw [color=c, fill=c] (11.7512,8.33558) rectangle (11.791,8.44143);
\draw [color=c, fill=c] (11.791,8.33558) rectangle (11.8308,8.44143);
\draw [color=c, fill=c] (11.8308,8.33558) rectangle (11.8706,8.44143);
\draw [color=c, fill=c] (11.8706,8.33558) rectangle (11.9104,8.44143);
\draw [color=c, fill=c] (11.9104,8.33558) rectangle (11.9502,8.44143);
\draw [color=c, fill=c] (11.9502,8.33558) rectangle (11.99,8.44143);
\draw [color=c, fill=c] (11.99,8.33558) rectangle (12.0299,8.44143);
\draw [color=c, fill=c] (12.0299,8.33558) rectangle (12.0697,8.44143);
\draw [color=c, fill=c] (12.0697,8.33558) rectangle (12.1095,8.44143);
\draw [color=c, fill=c] (12.1095,8.33558) rectangle (12.1493,8.44143);
\draw [color=c, fill=c] (12.1493,8.33558) rectangle (12.1891,8.44143);
\draw [color=c, fill=c] (12.1891,8.33558) rectangle (12.2289,8.44143);
\draw [color=c, fill=c] (12.2289,8.33558) rectangle (12.2687,8.44143);
\draw [color=c, fill=c] (12.2687,8.33558) rectangle (12.3085,8.44143);
\draw [color=c, fill=c] (12.3085,8.33558) rectangle (12.3483,8.44143);
\draw [color=c, fill=c] (12.3483,8.33558) rectangle (12.3881,8.44143);
\draw [color=c, fill=c] (12.3881,8.33558) rectangle (12.4279,8.44143);
\draw [color=c, fill=c] (12.4279,8.33558) rectangle (12.4677,8.44143);
\draw [color=c, fill=c] (12.4677,8.33558) rectangle (12.5075,8.44143);
\draw [color=c, fill=c] (12.5075,8.33558) rectangle (12.5473,8.44143);
\draw [color=c, fill=c] (12.5473,8.33558) rectangle (12.5871,8.44143);
\definecolor{c}{rgb}{0,0.733333,1};
\draw [color=c, fill=c] (12.5871,8.33558) rectangle (12.6269,8.44143);
\draw [color=c, fill=c] (12.6269,8.33558) rectangle (12.6667,8.44143);
\draw [color=c, fill=c] (12.6667,8.33558) rectangle (12.7065,8.44143);
\draw [color=c, fill=c] (12.7065,8.33558) rectangle (12.7463,8.44143);
\draw [color=c, fill=c] (12.7463,8.33558) rectangle (12.7861,8.44143);
\draw [color=c, fill=c] (12.7861,8.33558) rectangle (12.8259,8.44143);
\draw [color=c, fill=c] (12.8259,8.33558) rectangle (12.8657,8.44143);
\draw [color=c, fill=c] (12.8657,8.33558) rectangle (12.9055,8.44143);
\draw [color=c, fill=c] (12.9055,8.33558) rectangle (12.9453,8.44143);
\draw [color=c, fill=c] (12.9453,8.33558) rectangle (12.9851,8.44143);
\draw [color=c, fill=c] (12.9851,8.33558) rectangle (13.0249,8.44143);
\draw [color=c, fill=c] (13.0249,8.33558) rectangle (13.0647,8.44143);
\draw [color=c, fill=c] (13.0647,8.33558) rectangle (13.1045,8.44143);
\draw [color=c, fill=c] (13.1045,8.33558) rectangle (13.1443,8.44143);
\draw [color=c, fill=c] (13.1443,8.33558) rectangle (13.1841,8.44143);
\draw [color=c, fill=c] (13.1841,8.33558) rectangle (13.2239,8.44143);
\draw [color=c, fill=c] (13.2239,8.33558) rectangle (13.2637,8.44143);
\draw [color=c, fill=c] (13.2637,8.33558) rectangle (13.3035,8.44143);
\draw [color=c, fill=c] (13.3035,8.33558) rectangle (13.3433,8.44143);
\draw [color=c, fill=c] (13.3433,8.33558) rectangle (13.3831,8.44143);
\draw [color=c, fill=c] (13.3831,8.33558) rectangle (13.4229,8.44143);
\draw [color=c, fill=c] (13.4229,8.33558) rectangle (13.4627,8.44143);
\draw [color=c, fill=c] (13.4627,8.33558) rectangle (13.5025,8.44143);
\draw [color=c, fill=c] (13.5025,8.33558) rectangle (13.5423,8.44143);
\draw [color=c, fill=c] (13.5423,8.33558) rectangle (13.5821,8.44143);
\draw [color=c, fill=c] (13.5821,8.33558) rectangle (13.6219,8.44143);
\draw [color=c, fill=c] (13.6219,8.33558) rectangle (13.6617,8.44143);
\draw [color=c, fill=c] (13.6617,8.33558) rectangle (13.7015,8.44143);
\draw [color=c, fill=c] (13.7015,8.33558) rectangle (13.7413,8.44143);
\draw [color=c, fill=c] (13.7413,8.33558) rectangle (13.7811,8.44143);
\draw [color=c, fill=c] (13.7811,8.33558) rectangle (13.8209,8.44143);
\draw [color=c, fill=c] (13.8209,8.33558) rectangle (13.8607,8.44143);
\draw [color=c, fill=c] (13.8607,8.33558) rectangle (13.9005,8.44143);
\draw [color=c, fill=c] (13.9005,8.33558) rectangle (13.9403,8.44143);
\draw [color=c, fill=c] (13.9403,8.33558) rectangle (13.9801,8.44143);
\draw [color=c, fill=c] (13.9801,8.33558) rectangle (14.0199,8.44143);
\draw [color=c, fill=c] (14.0199,8.33558) rectangle (14.0597,8.44143);
\draw [color=c, fill=c] (14.0597,8.33558) rectangle (14.0995,8.44143);
\draw [color=c, fill=c] (14.0995,8.33558) rectangle (14.1393,8.44143);
\draw [color=c, fill=c] (14.1393,8.33558) rectangle (14.1791,8.44143);
\draw [color=c, fill=c] (14.1791,8.33558) rectangle (14.2189,8.44143);
\draw [color=c, fill=c] (14.2189,8.33558) rectangle (14.2587,8.44143);
\draw [color=c, fill=c] (14.2587,8.33558) rectangle (14.2985,8.44143);
\draw [color=c, fill=c] (14.2985,8.33558) rectangle (14.3383,8.44143);
\draw [color=c, fill=c] (14.3383,8.33558) rectangle (14.3781,8.44143);
\draw [color=c, fill=c] (14.3781,8.33558) rectangle (14.4179,8.44143);
\draw [color=c, fill=c] (14.4179,8.33558) rectangle (14.4577,8.44143);
\draw [color=c, fill=c] (14.4577,8.33558) rectangle (14.4975,8.44143);
\draw [color=c, fill=c] (14.4975,8.33558) rectangle (14.5373,8.44143);
\draw [color=c, fill=c] (14.5373,8.33558) rectangle (14.5771,8.44143);
\draw [color=c, fill=c] (14.5771,8.33558) rectangle (14.6169,8.44143);
\draw [color=c, fill=c] (14.6169,8.33558) rectangle (14.6567,8.44143);
\draw [color=c, fill=c] (14.6567,8.33558) rectangle (14.6965,8.44143);
\draw [color=c, fill=c] (14.6965,8.33558) rectangle (14.7363,8.44143);
\draw [color=c, fill=c] (14.7363,8.33558) rectangle (14.7761,8.44143);
\draw [color=c, fill=c] (14.7761,8.33558) rectangle (14.8159,8.44143);
\draw [color=c, fill=c] (14.8159,8.33558) rectangle (14.8557,8.44143);
\draw [color=c, fill=c] (14.8557,8.33558) rectangle (14.8955,8.44143);
\draw [color=c, fill=c] (14.8955,8.33558) rectangle (14.9353,8.44143);
\draw [color=c, fill=c] (14.9353,8.33558) rectangle (14.9751,8.44143);
\draw [color=c, fill=c] (14.9751,8.33558) rectangle (15.0149,8.44143);
\draw [color=c, fill=c] (15.0149,8.33558) rectangle (15.0547,8.44143);
\draw [color=c, fill=c] (15.0547,8.33558) rectangle (15.0945,8.44143);
\draw [color=c, fill=c] (15.0945,8.33558) rectangle (15.1343,8.44143);
\draw [color=c, fill=c] (15.1343,8.33558) rectangle (15.1741,8.44143);
\draw [color=c, fill=c] (15.1741,8.33558) rectangle (15.2139,8.44143);
\draw [color=c, fill=c] (15.2139,8.33558) rectangle (15.2537,8.44143);
\draw [color=c, fill=c] (15.2537,8.33558) rectangle (15.2935,8.44143);
\draw [color=c, fill=c] (15.2935,8.33558) rectangle (15.3333,8.44143);
\draw [color=c, fill=c] (15.3333,8.33558) rectangle (15.3731,8.44143);
\draw [color=c, fill=c] (15.3731,8.33558) rectangle (15.4129,8.44143);
\draw [color=c, fill=c] (15.4129,8.33558) rectangle (15.4527,8.44143);
\draw [color=c, fill=c] (15.4527,8.33558) rectangle (15.4925,8.44143);
\draw [color=c, fill=c] (15.4925,8.33558) rectangle (15.5323,8.44143);
\draw [color=c, fill=c] (15.5323,8.33558) rectangle (15.5721,8.44143);
\draw [color=c, fill=c] (15.5721,8.33558) rectangle (15.6119,8.44143);
\draw [color=c, fill=c] (15.6119,8.33558) rectangle (15.6517,8.44143);
\draw [color=c, fill=c] (15.6517,8.33558) rectangle (15.6915,8.44143);
\draw [color=c, fill=c] (15.6915,8.33558) rectangle (15.7313,8.44143);
\draw [color=c, fill=c] (15.7313,8.33558) rectangle (15.7711,8.44143);
\draw [color=c, fill=c] (15.7711,8.33558) rectangle (15.8109,8.44143);
\draw [color=c, fill=c] (15.8109,8.33558) rectangle (15.8507,8.44143);
\draw [color=c, fill=c] (15.8507,8.33558) rectangle (15.8905,8.44143);
\draw [color=c, fill=c] (15.8905,8.33558) rectangle (15.9303,8.44143);
\draw [color=c, fill=c] (15.9303,8.33558) rectangle (15.9701,8.44143);
\draw [color=c, fill=c] (15.9701,8.33558) rectangle (16.01,8.44143);
\draw [color=c, fill=c] (16.01,8.33558) rectangle (16.0498,8.44143);
\draw [color=c, fill=c] (16.0498,8.33558) rectangle (16.0896,8.44143);
\draw [color=c, fill=c] (16.0896,8.33558) rectangle (16.1294,8.44143);
\draw [color=c, fill=c] (16.1294,8.33558) rectangle (16.1692,8.44143);
\draw [color=c, fill=c] (16.1692,8.33558) rectangle (16.209,8.44143);
\draw [color=c, fill=c] (16.209,8.33558) rectangle (16.2488,8.44143);
\draw [color=c, fill=c] (16.2488,8.33558) rectangle (16.2886,8.44143);
\draw [color=c, fill=c] (16.2886,8.33558) rectangle (16.3284,8.44143);
\draw [color=c, fill=c] (16.3284,8.33558) rectangle (16.3682,8.44143);
\draw [color=c, fill=c] (16.3682,8.33558) rectangle (16.408,8.44143);
\draw [color=c, fill=c] (16.408,8.33558) rectangle (16.4478,8.44143);
\draw [color=c, fill=c] (16.4478,8.33558) rectangle (16.4876,8.44143);
\draw [color=c, fill=c] (16.4876,8.33558) rectangle (16.5274,8.44143);
\draw [color=c, fill=c] (16.5274,8.33558) rectangle (16.5672,8.44143);
\draw [color=c, fill=c] (16.5672,8.33558) rectangle (16.607,8.44143);
\draw [color=c, fill=c] (16.607,8.33558) rectangle (16.6468,8.44143);
\draw [color=c, fill=c] (16.6468,8.33558) rectangle (16.6866,8.44143);
\draw [color=c, fill=c] (16.6866,8.33558) rectangle (16.7264,8.44143);
\draw [color=c, fill=c] (16.7264,8.33558) rectangle (16.7662,8.44143);
\draw [color=c, fill=c] (16.7662,8.33558) rectangle (16.806,8.44143);
\draw [color=c, fill=c] (16.806,8.33558) rectangle (16.8458,8.44143);
\draw [color=c, fill=c] (16.8458,8.33558) rectangle (16.8856,8.44143);
\draw [color=c, fill=c] (16.8856,8.33558) rectangle (16.9254,8.44143);
\draw [color=c, fill=c] (16.9254,8.33558) rectangle (16.9652,8.44143);
\draw [color=c, fill=c] (16.9652,8.33558) rectangle (17.005,8.44143);
\draw [color=c, fill=c] (17.005,8.33558) rectangle (17.0448,8.44143);
\draw [color=c, fill=c] (17.0448,8.33558) rectangle (17.0846,8.44143);
\draw [color=c, fill=c] (17.0846,8.33558) rectangle (17.1244,8.44143);
\draw [color=c, fill=c] (17.1244,8.33558) rectangle (17.1642,8.44143);
\draw [color=c, fill=c] (17.1642,8.33558) rectangle (17.204,8.44143);
\draw [color=c, fill=c] (17.204,8.33558) rectangle (17.2438,8.44143);
\draw [color=c, fill=c] (17.2438,8.33558) rectangle (17.2836,8.44143);
\draw [color=c, fill=c] (17.2836,8.33558) rectangle (17.3234,8.44143);
\draw [color=c, fill=c] (17.3234,8.33558) rectangle (17.3632,8.44143);
\draw [color=c, fill=c] (17.3632,8.33558) rectangle (17.403,8.44143);
\draw [color=c, fill=c] (17.403,8.33558) rectangle (17.4428,8.44143);
\draw [color=c, fill=c] (17.4428,8.33558) rectangle (17.4826,8.44143);
\draw [color=c, fill=c] (17.4826,8.33558) rectangle (17.5224,8.44143);
\draw [color=c, fill=c] (17.5224,8.33558) rectangle (17.5622,8.44143);
\draw [color=c, fill=c] (17.5622,8.33558) rectangle (17.602,8.44143);
\draw [color=c, fill=c] (17.602,8.33558) rectangle (17.6418,8.44143);
\draw [color=c, fill=c] (17.6418,8.33558) rectangle (17.6816,8.44143);
\draw [color=c, fill=c] (17.6816,8.33558) rectangle (17.7214,8.44143);
\draw [color=c, fill=c] (17.7214,8.33558) rectangle (17.7612,8.44143);
\draw [color=c, fill=c] (17.7612,8.33558) rectangle (17.801,8.44143);
\draw [color=c, fill=c] (17.801,8.33558) rectangle (17.8408,8.44143);
\draw [color=c, fill=c] (17.8408,8.33558) rectangle (17.8806,8.44143);
\draw [color=c, fill=c] (17.8806,8.33558) rectangle (17.9204,8.44143);
\draw [color=c, fill=c] (17.9204,8.33558) rectangle (17.9602,8.44143);
\draw [color=c, fill=c] (17.9602,8.33558) rectangle (18,8.44143);
\definecolor{c}{rgb}{0.2,0,1};
\draw [color=c, fill=c] (2,8.44143) rectangle (2.0398,8.54728);
\draw [color=c, fill=c] (2.0398,8.44143) rectangle (2.0796,8.54728);
\draw [color=c, fill=c] (2.0796,8.44143) rectangle (2.1194,8.54728);
\draw [color=c, fill=c] (2.1194,8.44143) rectangle (2.1592,8.54728);
\draw [color=c, fill=c] (2.1592,8.44143) rectangle (2.19901,8.54728);
\draw [color=c, fill=c] (2.19901,8.44143) rectangle (2.23881,8.54728);
\draw [color=c, fill=c] (2.23881,8.44143) rectangle (2.27861,8.54728);
\draw [color=c, fill=c] (2.27861,8.44143) rectangle (2.31841,8.54728);
\draw [color=c, fill=c] (2.31841,8.44143) rectangle (2.35821,8.54728);
\draw [color=c, fill=c] (2.35821,8.44143) rectangle (2.39801,8.54728);
\draw [color=c, fill=c] (2.39801,8.44143) rectangle (2.43781,8.54728);
\draw [color=c, fill=c] (2.43781,8.44143) rectangle (2.47761,8.54728);
\draw [color=c, fill=c] (2.47761,8.44143) rectangle (2.51741,8.54728);
\draw [color=c, fill=c] (2.51741,8.44143) rectangle (2.55721,8.54728);
\draw [color=c, fill=c] (2.55721,8.44143) rectangle (2.59702,8.54728);
\draw [color=c, fill=c] (2.59702,8.44143) rectangle (2.63682,8.54728);
\draw [color=c, fill=c] (2.63682,8.44143) rectangle (2.67662,8.54728);
\draw [color=c, fill=c] (2.67662,8.44143) rectangle (2.71642,8.54728);
\draw [color=c, fill=c] (2.71642,8.44143) rectangle (2.75622,8.54728);
\draw [color=c, fill=c] (2.75622,8.44143) rectangle (2.79602,8.54728);
\draw [color=c, fill=c] (2.79602,8.44143) rectangle (2.83582,8.54728);
\draw [color=c, fill=c] (2.83582,8.44143) rectangle (2.87562,8.54728);
\draw [color=c, fill=c] (2.87562,8.44143) rectangle (2.91542,8.54728);
\draw [color=c, fill=c] (2.91542,8.44143) rectangle (2.95522,8.54728);
\draw [color=c, fill=c] (2.95522,8.44143) rectangle (2.99502,8.54728);
\draw [color=c, fill=c] (2.99502,8.44143) rectangle (3.03483,8.54728);
\draw [color=c, fill=c] (3.03483,8.44143) rectangle (3.07463,8.54728);
\draw [color=c, fill=c] (3.07463,8.44143) rectangle (3.11443,8.54728);
\draw [color=c, fill=c] (3.11443,8.44143) rectangle (3.15423,8.54728);
\draw [color=c, fill=c] (3.15423,8.44143) rectangle (3.19403,8.54728);
\draw [color=c, fill=c] (3.19403,8.44143) rectangle (3.23383,8.54728);
\draw [color=c, fill=c] (3.23383,8.44143) rectangle (3.27363,8.54728);
\draw [color=c, fill=c] (3.27363,8.44143) rectangle (3.31343,8.54728);
\draw [color=c, fill=c] (3.31343,8.44143) rectangle (3.35323,8.54728);
\draw [color=c, fill=c] (3.35323,8.44143) rectangle (3.39303,8.54728);
\draw [color=c, fill=c] (3.39303,8.44143) rectangle (3.43284,8.54728);
\draw [color=c, fill=c] (3.43284,8.44143) rectangle (3.47264,8.54728);
\draw [color=c, fill=c] (3.47264,8.44143) rectangle (3.51244,8.54728);
\draw [color=c, fill=c] (3.51244,8.44143) rectangle (3.55224,8.54728);
\draw [color=c, fill=c] (3.55224,8.44143) rectangle (3.59204,8.54728);
\draw [color=c, fill=c] (3.59204,8.44143) rectangle (3.63184,8.54728);
\draw [color=c, fill=c] (3.63184,8.44143) rectangle (3.67164,8.54728);
\draw [color=c, fill=c] (3.67164,8.44143) rectangle (3.71144,8.54728);
\draw [color=c, fill=c] (3.71144,8.44143) rectangle (3.75124,8.54728);
\draw [color=c, fill=c] (3.75124,8.44143) rectangle (3.79104,8.54728);
\draw [color=c, fill=c] (3.79104,8.44143) rectangle (3.83085,8.54728);
\draw [color=c, fill=c] (3.83085,8.44143) rectangle (3.87065,8.54728);
\draw [color=c, fill=c] (3.87065,8.44143) rectangle (3.91045,8.54728);
\draw [color=c, fill=c] (3.91045,8.44143) rectangle (3.95025,8.54728);
\draw [color=c, fill=c] (3.95025,8.44143) rectangle (3.99005,8.54728);
\draw [color=c, fill=c] (3.99005,8.44143) rectangle (4.02985,8.54728);
\draw [color=c, fill=c] (4.02985,8.44143) rectangle (4.06965,8.54728);
\draw [color=c, fill=c] (4.06965,8.44143) rectangle (4.10945,8.54728);
\draw [color=c, fill=c] (4.10945,8.44143) rectangle (4.14925,8.54728);
\draw [color=c, fill=c] (4.14925,8.44143) rectangle (4.18905,8.54728);
\draw [color=c, fill=c] (4.18905,8.44143) rectangle (4.22886,8.54728);
\draw [color=c, fill=c] (4.22886,8.44143) rectangle (4.26866,8.54728);
\draw [color=c, fill=c] (4.26866,8.44143) rectangle (4.30846,8.54728);
\draw [color=c, fill=c] (4.30846,8.44143) rectangle (4.34826,8.54728);
\draw [color=c, fill=c] (4.34826,8.44143) rectangle (4.38806,8.54728);
\draw [color=c, fill=c] (4.38806,8.44143) rectangle (4.42786,8.54728);
\draw [color=c, fill=c] (4.42786,8.44143) rectangle (4.46766,8.54728);
\draw [color=c, fill=c] (4.46766,8.44143) rectangle (4.50746,8.54728);
\draw [color=c, fill=c] (4.50746,8.44143) rectangle (4.54726,8.54728);
\draw [color=c, fill=c] (4.54726,8.44143) rectangle (4.58706,8.54728);
\draw [color=c, fill=c] (4.58706,8.44143) rectangle (4.62687,8.54728);
\draw [color=c, fill=c] (4.62687,8.44143) rectangle (4.66667,8.54728);
\draw [color=c, fill=c] (4.66667,8.44143) rectangle (4.70647,8.54728);
\draw [color=c, fill=c] (4.70647,8.44143) rectangle (4.74627,8.54728);
\draw [color=c, fill=c] (4.74627,8.44143) rectangle (4.78607,8.54728);
\draw [color=c, fill=c] (4.78607,8.44143) rectangle (4.82587,8.54728);
\draw [color=c, fill=c] (4.82587,8.44143) rectangle (4.86567,8.54728);
\draw [color=c, fill=c] (4.86567,8.44143) rectangle (4.90547,8.54728);
\draw [color=c, fill=c] (4.90547,8.44143) rectangle (4.94527,8.54728);
\draw [color=c, fill=c] (4.94527,8.44143) rectangle (4.98507,8.54728);
\draw [color=c, fill=c] (4.98507,8.44143) rectangle (5.02488,8.54728);
\draw [color=c, fill=c] (5.02488,8.44143) rectangle (5.06468,8.54728);
\draw [color=c, fill=c] (5.06468,8.44143) rectangle (5.10448,8.54728);
\draw [color=c, fill=c] (5.10448,8.44143) rectangle (5.14428,8.54728);
\draw [color=c, fill=c] (5.14428,8.44143) rectangle (5.18408,8.54728);
\draw [color=c, fill=c] (5.18408,8.44143) rectangle (5.22388,8.54728);
\draw [color=c, fill=c] (5.22388,8.44143) rectangle (5.26368,8.54728);
\draw [color=c, fill=c] (5.26368,8.44143) rectangle (5.30348,8.54728);
\draw [color=c, fill=c] (5.30348,8.44143) rectangle (5.34328,8.54728);
\draw [color=c, fill=c] (5.34328,8.44143) rectangle (5.38308,8.54728);
\draw [color=c, fill=c] (5.38308,8.44143) rectangle (5.42289,8.54728);
\draw [color=c, fill=c] (5.42289,8.44143) rectangle (5.46269,8.54728);
\draw [color=c, fill=c] (5.46269,8.44143) rectangle (5.50249,8.54728);
\draw [color=c, fill=c] (5.50249,8.44143) rectangle (5.54229,8.54728);
\draw [color=c, fill=c] (5.54229,8.44143) rectangle (5.58209,8.54728);
\draw [color=c, fill=c] (5.58209,8.44143) rectangle (5.62189,8.54728);
\draw [color=c, fill=c] (5.62189,8.44143) rectangle (5.66169,8.54728);
\draw [color=c, fill=c] (5.66169,8.44143) rectangle (5.70149,8.54728);
\draw [color=c, fill=c] (5.70149,8.44143) rectangle (5.74129,8.54728);
\draw [color=c, fill=c] (5.74129,8.44143) rectangle (5.78109,8.54728);
\draw [color=c, fill=c] (5.78109,8.44143) rectangle (5.8209,8.54728);
\draw [color=c, fill=c] (5.8209,8.44143) rectangle (5.8607,8.54728);
\draw [color=c, fill=c] (5.8607,8.44143) rectangle (5.9005,8.54728);
\draw [color=c, fill=c] (5.9005,8.44143) rectangle (5.9403,8.54728);
\draw [color=c, fill=c] (5.9403,8.44143) rectangle (5.9801,8.54728);
\draw [color=c, fill=c] (5.9801,8.44143) rectangle (6.0199,8.54728);
\draw [color=c, fill=c] (6.0199,8.44143) rectangle (6.0597,8.54728);
\draw [color=c, fill=c] (6.0597,8.44143) rectangle (6.0995,8.54728);
\draw [color=c, fill=c] (6.0995,8.44143) rectangle (6.1393,8.54728);
\draw [color=c, fill=c] (6.1393,8.44143) rectangle (6.1791,8.54728);
\draw [color=c, fill=c] (6.1791,8.44143) rectangle (6.21891,8.54728);
\draw [color=c, fill=c] (6.21891,8.44143) rectangle (6.25871,8.54728);
\draw [color=c, fill=c] (6.25871,8.44143) rectangle (6.29851,8.54728);
\draw [color=c, fill=c] (6.29851,8.44143) rectangle (6.33831,8.54728);
\draw [color=c, fill=c] (6.33831,8.44143) rectangle (6.37811,8.54728);
\draw [color=c, fill=c] (6.37811,8.44143) rectangle (6.41791,8.54728);
\draw [color=c, fill=c] (6.41791,8.44143) rectangle (6.45771,8.54728);
\draw [color=c, fill=c] (6.45771,8.44143) rectangle (6.49751,8.54728);
\draw [color=c, fill=c] (6.49751,8.44143) rectangle (6.53731,8.54728);
\draw [color=c, fill=c] (6.53731,8.44143) rectangle (6.57711,8.54728);
\draw [color=c, fill=c] (6.57711,8.44143) rectangle (6.61692,8.54728);
\draw [color=c, fill=c] (6.61692,8.44143) rectangle (6.65672,8.54728);
\draw [color=c, fill=c] (6.65672,8.44143) rectangle (6.69652,8.54728);
\draw [color=c, fill=c] (6.69652,8.44143) rectangle (6.73632,8.54728);
\draw [color=c, fill=c] (6.73632,8.44143) rectangle (6.77612,8.54728);
\draw [color=c, fill=c] (6.77612,8.44143) rectangle (6.81592,8.54728);
\draw [color=c, fill=c] (6.81592,8.44143) rectangle (6.85572,8.54728);
\draw [color=c, fill=c] (6.85572,8.44143) rectangle (6.89552,8.54728);
\draw [color=c, fill=c] (6.89552,8.44143) rectangle (6.93532,8.54728);
\draw [color=c, fill=c] (6.93532,8.44143) rectangle (6.97512,8.54728);
\draw [color=c, fill=c] (6.97512,8.44143) rectangle (7.01493,8.54728);
\draw [color=c, fill=c] (7.01493,8.44143) rectangle (7.05473,8.54728);
\draw [color=c, fill=c] (7.05473,8.44143) rectangle (7.09453,8.54728);
\draw [color=c, fill=c] (7.09453,8.44143) rectangle (7.13433,8.54728);
\draw [color=c, fill=c] (7.13433,8.44143) rectangle (7.17413,8.54728);
\draw [color=c, fill=c] (7.17413,8.44143) rectangle (7.21393,8.54728);
\draw [color=c, fill=c] (7.21393,8.44143) rectangle (7.25373,8.54728);
\draw [color=c, fill=c] (7.25373,8.44143) rectangle (7.29353,8.54728);
\draw [color=c, fill=c] (7.29353,8.44143) rectangle (7.33333,8.54728);
\draw [color=c, fill=c] (7.33333,8.44143) rectangle (7.37313,8.54728);
\draw [color=c, fill=c] (7.37313,8.44143) rectangle (7.41294,8.54728);
\draw [color=c, fill=c] (7.41294,8.44143) rectangle (7.45274,8.54728);
\draw [color=c, fill=c] (7.45274,8.44143) rectangle (7.49254,8.54728);
\draw [color=c, fill=c] (7.49254,8.44143) rectangle (7.53234,8.54728);
\draw [color=c, fill=c] (7.53234,8.44143) rectangle (7.57214,8.54728);
\draw [color=c, fill=c] (7.57214,8.44143) rectangle (7.61194,8.54728);
\definecolor{c}{rgb}{0,0.0800001,1};
\draw [color=c, fill=c] (7.61194,8.44143) rectangle (7.65174,8.54728);
\draw [color=c, fill=c] (7.65174,8.44143) rectangle (7.69154,8.54728);
\draw [color=c, fill=c] (7.69154,8.44143) rectangle (7.73134,8.54728);
\draw [color=c, fill=c] (7.73134,8.44143) rectangle (7.77114,8.54728);
\draw [color=c, fill=c] (7.77114,8.44143) rectangle (7.81095,8.54728);
\draw [color=c, fill=c] (7.81095,8.44143) rectangle (7.85075,8.54728);
\draw [color=c, fill=c] (7.85075,8.44143) rectangle (7.89055,8.54728);
\draw [color=c, fill=c] (7.89055,8.44143) rectangle (7.93035,8.54728);
\draw [color=c, fill=c] (7.93035,8.44143) rectangle (7.97015,8.54728);
\draw [color=c, fill=c] (7.97015,8.44143) rectangle (8.00995,8.54728);
\draw [color=c, fill=c] (8.00995,8.44143) rectangle (8.04975,8.54728);
\draw [color=c, fill=c] (8.04975,8.44143) rectangle (8.08955,8.54728);
\draw [color=c, fill=c] (8.08955,8.44143) rectangle (8.12935,8.54728);
\draw [color=c, fill=c] (8.12935,8.44143) rectangle (8.16915,8.54728);
\draw [color=c, fill=c] (8.16915,8.44143) rectangle (8.20895,8.54728);
\draw [color=c, fill=c] (8.20895,8.44143) rectangle (8.24876,8.54728);
\draw [color=c, fill=c] (8.24876,8.44143) rectangle (8.28856,8.54728);
\draw [color=c, fill=c] (8.28856,8.44143) rectangle (8.32836,8.54728);
\draw [color=c, fill=c] (8.32836,8.44143) rectangle (8.36816,8.54728);
\draw [color=c, fill=c] (8.36816,8.44143) rectangle (8.40796,8.54728);
\draw [color=c, fill=c] (8.40796,8.44143) rectangle (8.44776,8.54728);
\draw [color=c, fill=c] (8.44776,8.44143) rectangle (8.48756,8.54728);
\draw [color=c, fill=c] (8.48756,8.44143) rectangle (8.52736,8.54728);
\draw [color=c, fill=c] (8.52736,8.44143) rectangle (8.56716,8.54728);
\draw [color=c, fill=c] (8.56716,8.44143) rectangle (8.60697,8.54728);
\draw [color=c, fill=c] (8.60697,8.44143) rectangle (8.64677,8.54728);
\draw [color=c, fill=c] (8.64677,8.44143) rectangle (8.68657,8.54728);
\draw [color=c, fill=c] (8.68657,8.44143) rectangle (8.72637,8.54728);
\draw [color=c, fill=c] (8.72637,8.44143) rectangle (8.76617,8.54728);
\draw [color=c, fill=c] (8.76617,8.44143) rectangle (8.80597,8.54728);
\draw [color=c, fill=c] (8.80597,8.44143) rectangle (8.84577,8.54728);
\draw [color=c, fill=c] (8.84577,8.44143) rectangle (8.88557,8.54728);
\draw [color=c, fill=c] (8.88557,8.44143) rectangle (8.92537,8.54728);
\draw [color=c, fill=c] (8.92537,8.44143) rectangle (8.96517,8.54728);
\draw [color=c, fill=c] (8.96517,8.44143) rectangle (9.00498,8.54728);
\draw [color=c, fill=c] (9.00498,8.44143) rectangle (9.04478,8.54728);
\draw [color=c, fill=c] (9.04478,8.44143) rectangle (9.08458,8.54728);
\draw [color=c, fill=c] (9.08458,8.44143) rectangle (9.12438,8.54728);
\draw [color=c, fill=c] (9.12438,8.44143) rectangle (9.16418,8.54728);
\draw [color=c, fill=c] (9.16418,8.44143) rectangle (9.20398,8.54728);
\draw [color=c, fill=c] (9.20398,8.44143) rectangle (9.24378,8.54728);
\draw [color=c, fill=c] (9.24378,8.44143) rectangle (9.28358,8.54728);
\draw [color=c, fill=c] (9.28358,8.44143) rectangle (9.32338,8.54728);
\draw [color=c, fill=c] (9.32338,8.44143) rectangle (9.36318,8.54728);
\draw [color=c, fill=c] (9.36318,8.44143) rectangle (9.40298,8.54728);
\draw [color=c, fill=c] (9.40298,8.44143) rectangle (9.44279,8.54728);
\draw [color=c, fill=c] (9.44279,8.44143) rectangle (9.48259,8.54728);
\definecolor{c}{rgb}{0,0.266667,1};
\draw [color=c, fill=c] (9.48259,8.44143) rectangle (9.52239,8.54728);
\draw [color=c, fill=c] (9.52239,8.44143) rectangle (9.56219,8.54728);
\draw [color=c, fill=c] (9.56219,8.44143) rectangle (9.60199,8.54728);
\draw [color=c, fill=c] (9.60199,8.44143) rectangle (9.64179,8.54728);
\draw [color=c, fill=c] (9.64179,8.44143) rectangle (9.68159,8.54728);
\draw [color=c, fill=c] (9.68159,8.44143) rectangle (9.72139,8.54728);
\draw [color=c, fill=c] (9.72139,8.44143) rectangle (9.76119,8.54728);
\draw [color=c, fill=c] (9.76119,8.44143) rectangle (9.80099,8.54728);
\draw [color=c, fill=c] (9.80099,8.44143) rectangle (9.8408,8.54728);
\draw [color=c, fill=c] (9.8408,8.44143) rectangle (9.8806,8.54728);
\draw [color=c, fill=c] (9.8806,8.44143) rectangle (9.9204,8.54728);
\draw [color=c, fill=c] (9.9204,8.44143) rectangle (9.9602,8.54728);
\draw [color=c, fill=c] (9.9602,8.44143) rectangle (10,8.54728);
\draw [color=c, fill=c] (10,8.44143) rectangle (10.0398,8.54728);
\draw [color=c, fill=c] (10.0398,8.44143) rectangle (10.0796,8.54728);
\draw [color=c, fill=c] (10.0796,8.44143) rectangle (10.1194,8.54728);
\draw [color=c, fill=c] (10.1194,8.44143) rectangle (10.1592,8.54728);
\draw [color=c, fill=c] (10.1592,8.44143) rectangle (10.199,8.54728);
\draw [color=c, fill=c] (10.199,8.44143) rectangle (10.2388,8.54728);
\draw [color=c, fill=c] (10.2388,8.44143) rectangle (10.2786,8.54728);
\draw [color=c, fill=c] (10.2786,8.44143) rectangle (10.3184,8.54728);
\draw [color=c, fill=c] (10.3184,8.44143) rectangle (10.3582,8.54728);
\draw [color=c, fill=c] (10.3582,8.44143) rectangle (10.398,8.54728);
\draw [color=c, fill=c] (10.398,8.44143) rectangle (10.4378,8.54728);
\draw [color=c, fill=c] (10.4378,8.44143) rectangle (10.4776,8.54728);
\definecolor{c}{rgb}{0,0.546666,1};
\draw [color=c, fill=c] (10.4776,8.44143) rectangle (10.5174,8.54728);
\draw [color=c, fill=c] (10.5174,8.44143) rectangle (10.5572,8.54728);
\draw [color=c, fill=c] (10.5572,8.44143) rectangle (10.597,8.54728);
\draw [color=c, fill=c] (10.597,8.44143) rectangle (10.6368,8.54728);
\draw [color=c, fill=c] (10.6368,8.44143) rectangle (10.6766,8.54728);
\draw [color=c, fill=c] (10.6766,8.44143) rectangle (10.7164,8.54728);
\draw [color=c, fill=c] (10.7164,8.44143) rectangle (10.7562,8.54728);
\draw [color=c, fill=c] (10.7562,8.44143) rectangle (10.796,8.54728);
\draw [color=c, fill=c] (10.796,8.44143) rectangle (10.8358,8.54728);
\draw [color=c, fill=c] (10.8358,8.44143) rectangle (10.8756,8.54728);
\draw [color=c, fill=c] (10.8756,8.44143) rectangle (10.9154,8.54728);
\draw [color=c, fill=c] (10.9154,8.44143) rectangle (10.9552,8.54728);
\draw [color=c, fill=c] (10.9552,8.44143) rectangle (10.995,8.54728);
\draw [color=c, fill=c] (10.995,8.44143) rectangle (11.0348,8.54728);
\draw [color=c, fill=c] (11.0348,8.44143) rectangle (11.0746,8.54728);
\draw [color=c, fill=c] (11.0746,8.44143) rectangle (11.1144,8.54728);
\draw [color=c, fill=c] (11.1144,8.44143) rectangle (11.1542,8.54728);
\draw [color=c, fill=c] (11.1542,8.44143) rectangle (11.194,8.54728);
\draw [color=c, fill=c] (11.194,8.44143) rectangle (11.2338,8.54728);
\draw [color=c, fill=c] (11.2338,8.44143) rectangle (11.2736,8.54728);
\draw [color=c, fill=c] (11.2736,8.44143) rectangle (11.3134,8.54728);
\draw [color=c, fill=c] (11.3134,8.44143) rectangle (11.3532,8.54728);
\draw [color=c, fill=c] (11.3532,8.44143) rectangle (11.393,8.54728);
\draw [color=c, fill=c] (11.393,8.44143) rectangle (11.4328,8.54728);
\draw [color=c, fill=c] (11.4328,8.44143) rectangle (11.4726,8.54728);
\draw [color=c, fill=c] (11.4726,8.44143) rectangle (11.5124,8.54728);
\draw [color=c, fill=c] (11.5124,8.44143) rectangle (11.5522,8.54728);
\draw [color=c, fill=c] (11.5522,8.44143) rectangle (11.592,8.54728);
\draw [color=c, fill=c] (11.592,8.44143) rectangle (11.6318,8.54728);
\draw [color=c, fill=c] (11.6318,8.44143) rectangle (11.6716,8.54728);
\draw [color=c, fill=c] (11.6716,8.44143) rectangle (11.7114,8.54728);
\draw [color=c, fill=c] (11.7114,8.44143) rectangle (11.7512,8.54728);
\draw [color=c, fill=c] (11.7512,8.44143) rectangle (11.791,8.54728);
\draw [color=c, fill=c] (11.791,8.44143) rectangle (11.8308,8.54728);
\draw [color=c, fill=c] (11.8308,8.44143) rectangle (11.8706,8.54728);
\draw [color=c, fill=c] (11.8706,8.44143) rectangle (11.9104,8.54728);
\draw [color=c, fill=c] (11.9104,8.44143) rectangle (11.9502,8.54728);
\draw [color=c, fill=c] (11.9502,8.44143) rectangle (11.99,8.54728);
\draw [color=c, fill=c] (11.99,8.44143) rectangle (12.0299,8.54728);
\draw [color=c, fill=c] (12.0299,8.44143) rectangle (12.0697,8.54728);
\draw [color=c, fill=c] (12.0697,8.44143) rectangle (12.1095,8.54728);
\draw [color=c, fill=c] (12.1095,8.44143) rectangle (12.1493,8.54728);
\draw [color=c, fill=c] (12.1493,8.44143) rectangle (12.1891,8.54728);
\draw [color=c, fill=c] (12.1891,8.44143) rectangle (12.2289,8.54728);
\draw [color=c, fill=c] (12.2289,8.44143) rectangle (12.2687,8.54728);
\draw [color=c, fill=c] (12.2687,8.44143) rectangle (12.3085,8.54728);
\draw [color=c, fill=c] (12.3085,8.44143) rectangle (12.3483,8.54728);
\draw [color=c, fill=c] (12.3483,8.44143) rectangle (12.3881,8.54728);
\draw [color=c, fill=c] (12.3881,8.44143) rectangle (12.4279,8.54728);
\draw [color=c, fill=c] (12.4279,8.44143) rectangle (12.4677,8.54728);
\draw [color=c, fill=c] (12.4677,8.44143) rectangle (12.5075,8.54728);
\draw [color=c, fill=c] (12.5075,8.44143) rectangle (12.5473,8.54728);
\draw [color=c, fill=c] (12.5473,8.44143) rectangle (12.5871,8.54728);
\draw [color=c, fill=c] (12.5871,8.44143) rectangle (12.6269,8.54728);
\draw [color=c, fill=c] (12.6269,8.44143) rectangle (12.6667,8.54728);
\draw [color=c, fill=c] (12.6667,8.44143) rectangle (12.7065,8.54728);
\definecolor{c}{rgb}{0,0.733333,1};
\draw [color=c, fill=c] (12.7065,8.44143) rectangle (12.7463,8.54728);
\draw [color=c, fill=c] (12.7463,8.44143) rectangle (12.7861,8.54728);
\draw [color=c, fill=c] (12.7861,8.44143) rectangle (12.8259,8.54728);
\draw [color=c, fill=c] (12.8259,8.44143) rectangle (12.8657,8.54728);
\draw [color=c, fill=c] (12.8657,8.44143) rectangle (12.9055,8.54728);
\draw [color=c, fill=c] (12.9055,8.44143) rectangle (12.9453,8.54728);
\draw [color=c, fill=c] (12.9453,8.44143) rectangle (12.9851,8.54728);
\draw [color=c, fill=c] (12.9851,8.44143) rectangle (13.0249,8.54728);
\draw [color=c, fill=c] (13.0249,8.44143) rectangle (13.0647,8.54728);
\draw [color=c, fill=c] (13.0647,8.44143) rectangle (13.1045,8.54728);
\draw [color=c, fill=c] (13.1045,8.44143) rectangle (13.1443,8.54728);
\draw [color=c, fill=c] (13.1443,8.44143) rectangle (13.1841,8.54728);
\draw [color=c, fill=c] (13.1841,8.44143) rectangle (13.2239,8.54728);
\draw [color=c, fill=c] (13.2239,8.44143) rectangle (13.2637,8.54728);
\draw [color=c, fill=c] (13.2637,8.44143) rectangle (13.3035,8.54728);
\draw [color=c, fill=c] (13.3035,8.44143) rectangle (13.3433,8.54728);
\draw [color=c, fill=c] (13.3433,8.44143) rectangle (13.3831,8.54728);
\draw [color=c, fill=c] (13.3831,8.44143) rectangle (13.4229,8.54728);
\draw [color=c, fill=c] (13.4229,8.44143) rectangle (13.4627,8.54728);
\draw [color=c, fill=c] (13.4627,8.44143) rectangle (13.5025,8.54728);
\draw [color=c, fill=c] (13.5025,8.44143) rectangle (13.5423,8.54728);
\draw [color=c, fill=c] (13.5423,8.44143) rectangle (13.5821,8.54728);
\draw [color=c, fill=c] (13.5821,8.44143) rectangle (13.6219,8.54728);
\draw [color=c, fill=c] (13.6219,8.44143) rectangle (13.6617,8.54728);
\draw [color=c, fill=c] (13.6617,8.44143) rectangle (13.7015,8.54728);
\draw [color=c, fill=c] (13.7015,8.44143) rectangle (13.7413,8.54728);
\draw [color=c, fill=c] (13.7413,8.44143) rectangle (13.7811,8.54728);
\draw [color=c, fill=c] (13.7811,8.44143) rectangle (13.8209,8.54728);
\draw [color=c, fill=c] (13.8209,8.44143) rectangle (13.8607,8.54728);
\draw [color=c, fill=c] (13.8607,8.44143) rectangle (13.9005,8.54728);
\draw [color=c, fill=c] (13.9005,8.44143) rectangle (13.9403,8.54728);
\draw [color=c, fill=c] (13.9403,8.44143) rectangle (13.9801,8.54728);
\draw [color=c, fill=c] (13.9801,8.44143) rectangle (14.0199,8.54728);
\draw [color=c, fill=c] (14.0199,8.44143) rectangle (14.0597,8.54728);
\draw [color=c, fill=c] (14.0597,8.44143) rectangle (14.0995,8.54728);
\draw [color=c, fill=c] (14.0995,8.44143) rectangle (14.1393,8.54728);
\draw [color=c, fill=c] (14.1393,8.44143) rectangle (14.1791,8.54728);
\draw [color=c, fill=c] (14.1791,8.44143) rectangle (14.2189,8.54728);
\draw [color=c, fill=c] (14.2189,8.44143) rectangle (14.2587,8.54728);
\draw [color=c, fill=c] (14.2587,8.44143) rectangle (14.2985,8.54728);
\draw [color=c, fill=c] (14.2985,8.44143) rectangle (14.3383,8.54728);
\draw [color=c, fill=c] (14.3383,8.44143) rectangle (14.3781,8.54728);
\draw [color=c, fill=c] (14.3781,8.44143) rectangle (14.4179,8.54728);
\draw [color=c, fill=c] (14.4179,8.44143) rectangle (14.4577,8.54728);
\draw [color=c, fill=c] (14.4577,8.44143) rectangle (14.4975,8.54728);
\draw [color=c, fill=c] (14.4975,8.44143) rectangle (14.5373,8.54728);
\draw [color=c, fill=c] (14.5373,8.44143) rectangle (14.5771,8.54728);
\draw [color=c, fill=c] (14.5771,8.44143) rectangle (14.6169,8.54728);
\draw [color=c, fill=c] (14.6169,8.44143) rectangle (14.6567,8.54728);
\draw [color=c, fill=c] (14.6567,8.44143) rectangle (14.6965,8.54728);
\draw [color=c, fill=c] (14.6965,8.44143) rectangle (14.7363,8.54728);
\draw [color=c, fill=c] (14.7363,8.44143) rectangle (14.7761,8.54728);
\draw [color=c, fill=c] (14.7761,8.44143) rectangle (14.8159,8.54728);
\draw [color=c, fill=c] (14.8159,8.44143) rectangle (14.8557,8.54728);
\draw [color=c, fill=c] (14.8557,8.44143) rectangle (14.8955,8.54728);
\draw [color=c, fill=c] (14.8955,8.44143) rectangle (14.9353,8.54728);
\draw [color=c, fill=c] (14.9353,8.44143) rectangle (14.9751,8.54728);
\draw [color=c, fill=c] (14.9751,8.44143) rectangle (15.0149,8.54728);
\draw [color=c, fill=c] (15.0149,8.44143) rectangle (15.0547,8.54728);
\draw [color=c, fill=c] (15.0547,8.44143) rectangle (15.0945,8.54728);
\draw [color=c, fill=c] (15.0945,8.44143) rectangle (15.1343,8.54728);
\draw [color=c, fill=c] (15.1343,8.44143) rectangle (15.1741,8.54728);
\draw [color=c, fill=c] (15.1741,8.44143) rectangle (15.2139,8.54728);
\draw [color=c, fill=c] (15.2139,8.44143) rectangle (15.2537,8.54728);
\draw [color=c, fill=c] (15.2537,8.44143) rectangle (15.2935,8.54728);
\draw [color=c, fill=c] (15.2935,8.44143) rectangle (15.3333,8.54728);
\draw [color=c, fill=c] (15.3333,8.44143) rectangle (15.3731,8.54728);
\draw [color=c, fill=c] (15.3731,8.44143) rectangle (15.4129,8.54728);
\draw [color=c, fill=c] (15.4129,8.44143) rectangle (15.4527,8.54728);
\draw [color=c, fill=c] (15.4527,8.44143) rectangle (15.4925,8.54728);
\draw [color=c, fill=c] (15.4925,8.44143) rectangle (15.5323,8.54728);
\draw [color=c, fill=c] (15.5323,8.44143) rectangle (15.5721,8.54728);
\draw [color=c, fill=c] (15.5721,8.44143) rectangle (15.6119,8.54728);
\draw [color=c, fill=c] (15.6119,8.44143) rectangle (15.6517,8.54728);
\draw [color=c, fill=c] (15.6517,8.44143) rectangle (15.6915,8.54728);
\draw [color=c, fill=c] (15.6915,8.44143) rectangle (15.7313,8.54728);
\draw [color=c, fill=c] (15.7313,8.44143) rectangle (15.7711,8.54728);
\draw [color=c, fill=c] (15.7711,8.44143) rectangle (15.8109,8.54728);
\draw [color=c, fill=c] (15.8109,8.44143) rectangle (15.8507,8.54728);
\draw [color=c, fill=c] (15.8507,8.44143) rectangle (15.8905,8.54728);
\draw [color=c, fill=c] (15.8905,8.44143) rectangle (15.9303,8.54728);
\draw [color=c, fill=c] (15.9303,8.44143) rectangle (15.9701,8.54728);
\draw [color=c, fill=c] (15.9701,8.44143) rectangle (16.01,8.54728);
\draw [color=c, fill=c] (16.01,8.44143) rectangle (16.0498,8.54728);
\draw [color=c, fill=c] (16.0498,8.44143) rectangle (16.0896,8.54728);
\draw [color=c, fill=c] (16.0896,8.44143) rectangle (16.1294,8.54728);
\draw [color=c, fill=c] (16.1294,8.44143) rectangle (16.1692,8.54728);
\draw [color=c, fill=c] (16.1692,8.44143) rectangle (16.209,8.54728);
\draw [color=c, fill=c] (16.209,8.44143) rectangle (16.2488,8.54728);
\draw [color=c, fill=c] (16.2488,8.44143) rectangle (16.2886,8.54728);
\draw [color=c, fill=c] (16.2886,8.44143) rectangle (16.3284,8.54728);
\draw [color=c, fill=c] (16.3284,8.44143) rectangle (16.3682,8.54728);
\draw [color=c, fill=c] (16.3682,8.44143) rectangle (16.408,8.54728);
\draw [color=c, fill=c] (16.408,8.44143) rectangle (16.4478,8.54728);
\draw [color=c, fill=c] (16.4478,8.44143) rectangle (16.4876,8.54728);
\draw [color=c, fill=c] (16.4876,8.44143) rectangle (16.5274,8.54728);
\draw [color=c, fill=c] (16.5274,8.44143) rectangle (16.5672,8.54728);
\draw [color=c, fill=c] (16.5672,8.44143) rectangle (16.607,8.54728);
\draw [color=c, fill=c] (16.607,8.44143) rectangle (16.6468,8.54728);
\draw [color=c, fill=c] (16.6468,8.44143) rectangle (16.6866,8.54728);
\draw [color=c, fill=c] (16.6866,8.44143) rectangle (16.7264,8.54728);
\draw [color=c, fill=c] (16.7264,8.44143) rectangle (16.7662,8.54728);
\draw [color=c, fill=c] (16.7662,8.44143) rectangle (16.806,8.54728);
\draw [color=c, fill=c] (16.806,8.44143) rectangle (16.8458,8.54728);
\draw [color=c, fill=c] (16.8458,8.44143) rectangle (16.8856,8.54728);
\draw [color=c, fill=c] (16.8856,8.44143) rectangle (16.9254,8.54728);
\draw [color=c, fill=c] (16.9254,8.44143) rectangle (16.9652,8.54728);
\draw [color=c, fill=c] (16.9652,8.44143) rectangle (17.005,8.54728);
\draw [color=c, fill=c] (17.005,8.44143) rectangle (17.0448,8.54728);
\draw [color=c, fill=c] (17.0448,8.44143) rectangle (17.0846,8.54728);
\draw [color=c, fill=c] (17.0846,8.44143) rectangle (17.1244,8.54728);
\draw [color=c, fill=c] (17.1244,8.44143) rectangle (17.1642,8.54728);
\draw [color=c, fill=c] (17.1642,8.44143) rectangle (17.204,8.54728);
\draw [color=c, fill=c] (17.204,8.44143) rectangle (17.2438,8.54728);
\draw [color=c, fill=c] (17.2438,8.44143) rectangle (17.2836,8.54728);
\draw [color=c, fill=c] (17.2836,8.44143) rectangle (17.3234,8.54728);
\draw [color=c, fill=c] (17.3234,8.44143) rectangle (17.3632,8.54728);
\draw [color=c, fill=c] (17.3632,8.44143) rectangle (17.403,8.54728);
\draw [color=c, fill=c] (17.403,8.44143) rectangle (17.4428,8.54728);
\draw [color=c, fill=c] (17.4428,8.44143) rectangle (17.4826,8.54728);
\draw [color=c, fill=c] (17.4826,8.44143) rectangle (17.5224,8.54728);
\draw [color=c, fill=c] (17.5224,8.44143) rectangle (17.5622,8.54728);
\draw [color=c, fill=c] (17.5622,8.44143) rectangle (17.602,8.54728);
\draw [color=c, fill=c] (17.602,8.44143) rectangle (17.6418,8.54728);
\draw [color=c, fill=c] (17.6418,8.44143) rectangle (17.6816,8.54728);
\draw [color=c, fill=c] (17.6816,8.44143) rectangle (17.7214,8.54728);
\draw [color=c, fill=c] (17.7214,8.44143) rectangle (17.7612,8.54728);
\draw [color=c, fill=c] (17.7612,8.44143) rectangle (17.801,8.54728);
\draw [color=c, fill=c] (17.801,8.44143) rectangle (17.8408,8.54728);
\draw [color=c, fill=c] (17.8408,8.44143) rectangle (17.8806,8.54728);
\draw [color=c, fill=c] (17.8806,8.44143) rectangle (17.9204,8.54728);
\draw [color=c, fill=c] (17.9204,8.44143) rectangle (17.9602,8.54728);
\draw [color=c, fill=c] (17.9602,8.44143) rectangle (18,8.54728);
\definecolor{c}{rgb}{0.2,0,1};
\draw [color=c, fill=c] (2,8.54728) rectangle (2.0398,8.65313);
\draw [color=c, fill=c] (2.0398,8.54728) rectangle (2.0796,8.65313);
\draw [color=c, fill=c] (2.0796,8.54728) rectangle (2.1194,8.65313);
\draw [color=c, fill=c] (2.1194,8.54728) rectangle (2.1592,8.65313);
\draw [color=c, fill=c] (2.1592,8.54728) rectangle (2.19901,8.65313);
\draw [color=c, fill=c] (2.19901,8.54728) rectangle (2.23881,8.65313);
\draw [color=c, fill=c] (2.23881,8.54728) rectangle (2.27861,8.65313);
\draw [color=c, fill=c] (2.27861,8.54728) rectangle (2.31841,8.65313);
\draw [color=c, fill=c] (2.31841,8.54728) rectangle (2.35821,8.65313);
\draw [color=c, fill=c] (2.35821,8.54728) rectangle (2.39801,8.65313);
\draw [color=c, fill=c] (2.39801,8.54728) rectangle (2.43781,8.65313);
\draw [color=c, fill=c] (2.43781,8.54728) rectangle (2.47761,8.65313);
\draw [color=c, fill=c] (2.47761,8.54728) rectangle (2.51741,8.65313);
\draw [color=c, fill=c] (2.51741,8.54728) rectangle (2.55721,8.65313);
\draw [color=c, fill=c] (2.55721,8.54728) rectangle (2.59702,8.65313);
\draw [color=c, fill=c] (2.59702,8.54728) rectangle (2.63682,8.65313);
\draw [color=c, fill=c] (2.63682,8.54728) rectangle (2.67662,8.65313);
\draw [color=c, fill=c] (2.67662,8.54728) rectangle (2.71642,8.65313);
\draw [color=c, fill=c] (2.71642,8.54728) rectangle (2.75622,8.65313);
\draw [color=c, fill=c] (2.75622,8.54728) rectangle (2.79602,8.65313);
\draw [color=c, fill=c] (2.79602,8.54728) rectangle (2.83582,8.65313);
\draw [color=c, fill=c] (2.83582,8.54728) rectangle (2.87562,8.65313);
\draw [color=c, fill=c] (2.87562,8.54728) rectangle (2.91542,8.65313);
\draw [color=c, fill=c] (2.91542,8.54728) rectangle (2.95522,8.65313);
\draw [color=c, fill=c] (2.95522,8.54728) rectangle (2.99502,8.65313);
\draw [color=c, fill=c] (2.99502,8.54728) rectangle (3.03483,8.65313);
\draw [color=c, fill=c] (3.03483,8.54728) rectangle (3.07463,8.65313);
\draw [color=c, fill=c] (3.07463,8.54728) rectangle (3.11443,8.65313);
\draw [color=c, fill=c] (3.11443,8.54728) rectangle (3.15423,8.65313);
\draw [color=c, fill=c] (3.15423,8.54728) rectangle (3.19403,8.65313);
\draw [color=c, fill=c] (3.19403,8.54728) rectangle (3.23383,8.65313);
\draw [color=c, fill=c] (3.23383,8.54728) rectangle (3.27363,8.65313);
\draw [color=c, fill=c] (3.27363,8.54728) rectangle (3.31343,8.65313);
\draw [color=c, fill=c] (3.31343,8.54728) rectangle (3.35323,8.65313);
\draw [color=c, fill=c] (3.35323,8.54728) rectangle (3.39303,8.65313);
\draw [color=c, fill=c] (3.39303,8.54728) rectangle (3.43284,8.65313);
\draw [color=c, fill=c] (3.43284,8.54728) rectangle (3.47264,8.65313);
\draw [color=c, fill=c] (3.47264,8.54728) rectangle (3.51244,8.65313);
\draw [color=c, fill=c] (3.51244,8.54728) rectangle (3.55224,8.65313);
\draw [color=c, fill=c] (3.55224,8.54728) rectangle (3.59204,8.65313);
\draw [color=c, fill=c] (3.59204,8.54728) rectangle (3.63184,8.65313);
\draw [color=c, fill=c] (3.63184,8.54728) rectangle (3.67164,8.65313);
\draw [color=c, fill=c] (3.67164,8.54728) rectangle (3.71144,8.65313);
\draw [color=c, fill=c] (3.71144,8.54728) rectangle (3.75124,8.65313);
\draw [color=c, fill=c] (3.75124,8.54728) rectangle (3.79104,8.65313);
\draw [color=c, fill=c] (3.79104,8.54728) rectangle (3.83085,8.65313);
\draw [color=c, fill=c] (3.83085,8.54728) rectangle (3.87065,8.65313);
\draw [color=c, fill=c] (3.87065,8.54728) rectangle (3.91045,8.65313);
\draw [color=c, fill=c] (3.91045,8.54728) rectangle (3.95025,8.65313);
\draw [color=c, fill=c] (3.95025,8.54728) rectangle (3.99005,8.65313);
\draw [color=c, fill=c] (3.99005,8.54728) rectangle (4.02985,8.65313);
\draw [color=c, fill=c] (4.02985,8.54728) rectangle (4.06965,8.65313);
\draw [color=c, fill=c] (4.06965,8.54728) rectangle (4.10945,8.65313);
\draw [color=c, fill=c] (4.10945,8.54728) rectangle (4.14925,8.65313);
\draw [color=c, fill=c] (4.14925,8.54728) rectangle (4.18905,8.65313);
\draw [color=c, fill=c] (4.18905,8.54728) rectangle (4.22886,8.65313);
\draw [color=c, fill=c] (4.22886,8.54728) rectangle (4.26866,8.65313);
\draw [color=c, fill=c] (4.26866,8.54728) rectangle (4.30846,8.65313);
\draw [color=c, fill=c] (4.30846,8.54728) rectangle (4.34826,8.65313);
\draw [color=c, fill=c] (4.34826,8.54728) rectangle (4.38806,8.65313);
\draw [color=c, fill=c] (4.38806,8.54728) rectangle (4.42786,8.65313);
\draw [color=c, fill=c] (4.42786,8.54728) rectangle (4.46766,8.65313);
\draw [color=c, fill=c] (4.46766,8.54728) rectangle (4.50746,8.65313);
\draw [color=c, fill=c] (4.50746,8.54728) rectangle (4.54726,8.65313);
\draw [color=c, fill=c] (4.54726,8.54728) rectangle (4.58706,8.65313);
\draw [color=c, fill=c] (4.58706,8.54728) rectangle (4.62687,8.65313);
\draw [color=c, fill=c] (4.62687,8.54728) rectangle (4.66667,8.65313);
\draw [color=c, fill=c] (4.66667,8.54728) rectangle (4.70647,8.65313);
\draw [color=c, fill=c] (4.70647,8.54728) rectangle (4.74627,8.65313);
\draw [color=c, fill=c] (4.74627,8.54728) rectangle (4.78607,8.65313);
\draw [color=c, fill=c] (4.78607,8.54728) rectangle (4.82587,8.65313);
\draw [color=c, fill=c] (4.82587,8.54728) rectangle (4.86567,8.65313);
\draw [color=c, fill=c] (4.86567,8.54728) rectangle (4.90547,8.65313);
\draw [color=c, fill=c] (4.90547,8.54728) rectangle (4.94527,8.65313);
\draw [color=c, fill=c] (4.94527,8.54728) rectangle (4.98507,8.65313);
\draw [color=c, fill=c] (4.98507,8.54728) rectangle (5.02488,8.65313);
\draw [color=c, fill=c] (5.02488,8.54728) rectangle (5.06468,8.65313);
\draw [color=c, fill=c] (5.06468,8.54728) rectangle (5.10448,8.65313);
\draw [color=c, fill=c] (5.10448,8.54728) rectangle (5.14428,8.65313);
\draw [color=c, fill=c] (5.14428,8.54728) rectangle (5.18408,8.65313);
\draw [color=c, fill=c] (5.18408,8.54728) rectangle (5.22388,8.65313);
\draw [color=c, fill=c] (5.22388,8.54728) rectangle (5.26368,8.65313);
\draw [color=c, fill=c] (5.26368,8.54728) rectangle (5.30348,8.65313);
\draw [color=c, fill=c] (5.30348,8.54728) rectangle (5.34328,8.65313);
\draw [color=c, fill=c] (5.34328,8.54728) rectangle (5.38308,8.65313);
\draw [color=c, fill=c] (5.38308,8.54728) rectangle (5.42289,8.65313);
\draw [color=c, fill=c] (5.42289,8.54728) rectangle (5.46269,8.65313);
\draw [color=c, fill=c] (5.46269,8.54728) rectangle (5.50249,8.65313);
\draw [color=c, fill=c] (5.50249,8.54728) rectangle (5.54229,8.65313);
\draw [color=c, fill=c] (5.54229,8.54728) rectangle (5.58209,8.65313);
\draw [color=c, fill=c] (5.58209,8.54728) rectangle (5.62189,8.65313);
\draw [color=c, fill=c] (5.62189,8.54728) rectangle (5.66169,8.65313);
\draw [color=c, fill=c] (5.66169,8.54728) rectangle (5.70149,8.65313);
\draw [color=c, fill=c] (5.70149,8.54728) rectangle (5.74129,8.65313);
\draw [color=c, fill=c] (5.74129,8.54728) rectangle (5.78109,8.65313);
\draw [color=c, fill=c] (5.78109,8.54728) rectangle (5.8209,8.65313);
\draw [color=c, fill=c] (5.8209,8.54728) rectangle (5.8607,8.65313);
\draw [color=c, fill=c] (5.8607,8.54728) rectangle (5.9005,8.65313);
\draw [color=c, fill=c] (5.9005,8.54728) rectangle (5.9403,8.65313);
\draw [color=c, fill=c] (5.9403,8.54728) rectangle (5.9801,8.65313);
\draw [color=c, fill=c] (5.9801,8.54728) rectangle (6.0199,8.65313);
\draw [color=c, fill=c] (6.0199,8.54728) rectangle (6.0597,8.65313);
\draw [color=c, fill=c] (6.0597,8.54728) rectangle (6.0995,8.65313);
\draw [color=c, fill=c] (6.0995,8.54728) rectangle (6.1393,8.65313);
\draw [color=c, fill=c] (6.1393,8.54728) rectangle (6.1791,8.65313);
\draw [color=c, fill=c] (6.1791,8.54728) rectangle (6.21891,8.65313);
\draw [color=c, fill=c] (6.21891,8.54728) rectangle (6.25871,8.65313);
\draw [color=c, fill=c] (6.25871,8.54728) rectangle (6.29851,8.65313);
\draw [color=c, fill=c] (6.29851,8.54728) rectangle (6.33831,8.65313);
\draw [color=c, fill=c] (6.33831,8.54728) rectangle (6.37811,8.65313);
\draw [color=c, fill=c] (6.37811,8.54728) rectangle (6.41791,8.65313);
\draw [color=c, fill=c] (6.41791,8.54728) rectangle (6.45771,8.65313);
\draw [color=c, fill=c] (6.45771,8.54728) rectangle (6.49751,8.65313);
\draw [color=c, fill=c] (6.49751,8.54728) rectangle (6.53731,8.65313);
\draw [color=c, fill=c] (6.53731,8.54728) rectangle (6.57711,8.65313);
\draw [color=c, fill=c] (6.57711,8.54728) rectangle (6.61692,8.65313);
\draw [color=c, fill=c] (6.61692,8.54728) rectangle (6.65672,8.65313);
\draw [color=c, fill=c] (6.65672,8.54728) rectangle (6.69652,8.65313);
\draw [color=c, fill=c] (6.69652,8.54728) rectangle (6.73632,8.65313);
\draw [color=c, fill=c] (6.73632,8.54728) rectangle (6.77612,8.65313);
\draw [color=c, fill=c] (6.77612,8.54728) rectangle (6.81592,8.65313);
\draw [color=c, fill=c] (6.81592,8.54728) rectangle (6.85572,8.65313);
\draw [color=c, fill=c] (6.85572,8.54728) rectangle (6.89552,8.65313);
\draw [color=c, fill=c] (6.89552,8.54728) rectangle (6.93532,8.65313);
\draw [color=c, fill=c] (6.93532,8.54728) rectangle (6.97512,8.65313);
\draw [color=c, fill=c] (6.97512,8.54728) rectangle (7.01493,8.65313);
\draw [color=c, fill=c] (7.01493,8.54728) rectangle (7.05473,8.65313);
\draw [color=c, fill=c] (7.05473,8.54728) rectangle (7.09453,8.65313);
\draw [color=c, fill=c] (7.09453,8.54728) rectangle (7.13433,8.65313);
\draw [color=c, fill=c] (7.13433,8.54728) rectangle (7.17413,8.65313);
\draw [color=c, fill=c] (7.17413,8.54728) rectangle (7.21393,8.65313);
\draw [color=c, fill=c] (7.21393,8.54728) rectangle (7.25373,8.65313);
\draw [color=c, fill=c] (7.25373,8.54728) rectangle (7.29353,8.65313);
\draw [color=c, fill=c] (7.29353,8.54728) rectangle (7.33333,8.65313);
\draw [color=c, fill=c] (7.33333,8.54728) rectangle (7.37313,8.65313);
\draw [color=c, fill=c] (7.37313,8.54728) rectangle (7.41294,8.65313);
\draw [color=c, fill=c] (7.41294,8.54728) rectangle (7.45274,8.65313);
\draw [color=c, fill=c] (7.45274,8.54728) rectangle (7.49254,8.65313);
\draw [color=c, fill=c] (7.49254,8.54728) rectangle (7.53234,8.65313);
\draw [color=c, fill=c] (7.53234,8.54728) rectangle (7.57214,8.65313);
\draw [color=c, fill=c] (7.57214,8.54728) rectangle (7.61194,8.65313);
\definecolor{c}{rgb}{0,0.0800001,1};
\draw [color=c, fill=c] (7.61194,8.54728) rectangle (7.65174,8.65313);
\draw [color=c, fill=c] (7.65174,8.54728) rectangle (7.69154,8.65313);
\draw [color=c, fill=c] (7.69154,8.54728) rectangle (7.73134,8.65313);
\draw [color=c, fill=c] (7.73134,8.54728) rectangle (7.77114,8.65313);
\draw [color=c, fill=c] (7.77114,8.54728) rectangle (7.81095,8.65313);
\draw [color=c, fill=c] (7.81095,8.54728) rectangle (7.85075,8.65313);
\draw [color=c, fill=c] (7.85075,8.54728) rectangle (7.89055,8.65313);
\draw [color=c, fill=c] (7.89055,8.54728) rectangle (7.93035,8.65313);
\draw [color=c, fill=c] (7.93035,8.54728) rectangle (7.97015,8.65313);
\draw [color=c, fill=c] (7.97015,8.54728) rectangle (8.00995,8.65313);
\draw [color=c, fill=c] (8.00995,8.54728) rectangle (8.04975,8.65313);
\draw [color=c, fill=c] (8.04975,8.54728) rectangle (8.08955,8.65313);
\draw [color=c, fill=c] (8.08955,8.54728) rectangle (8.12935,8.65313);
\draw [color=c, fill=c] (8.12935,8.54728) rectangle (8.16915,8.65313);
\draw [color=c, fill=c] (8.16915,8.54728) rectangle (8.20895,8.65313);
\draw [color=c, fill=c] (8.20895,8.54728) rectangle (8.24876,8.65313);
\draw [color=c, fill=c] (8.24876,8.54728) rectangle (8.28856,8.65313);
\draw [color=c, fill=c] (8.28856,8.54728) rectangle (8.32836,8.65313);
\draw [color=c, fill=c] (8.32836,8.54728) rectangle (8.36816,8.65313);
\draw [color=c, fill=c] (8.36816,8.54728) rectangle (8.40796,8.65313);
\draw [color=c, fill=c] (8.40796,8.54728) rectangle (8.44776,8.65313);
\draw [color=c, fill=c] (8.44776,8.54728) rectangle (8.48756,8.65313);
\draw [color=c, fill=c] (8.48756,8.54728) rectangle (8.52736,8.65313);
\draw [color=c, fill=c] (8.52736,8.54728) rectangle (8.56716,8.65313);
\draw [color=c, fill=c] (8.56716,8.54728) rectangle (8.60697,8.65313);
\draw [color=c, fill=c] (8.60697,8.54728) rectangle (8.64677,8.65313);
\draw [color=c, fill=c] (8.64677,8.54728) rectangle (8.68657,8.65313);
\draw [color=c, fill=c] (8.68657,8.54728) rectangle (8.72637,8.65313);
\draw [color=c, fill=c] (8.72637,8.54728) rectangle (8.76617,8.65313);
\draw [color=c, fill=c] (8.76617,8.54728) rectangle (8.80597,8.65313);
\draw [color=c, fill=c] (8.80597,8.54728) rectangle (8.84577,8.65313);
\draw [color=c, fill=c] (8.84577,8.54728) rectangle (8.88557,8.65313);
\draw [color=c, fill=c] (8.88557,8.54728) rectangle (8.92537,8.65313);
\draw [color=c, fill=c] (8.92537,8.54728) rectangle (8.96517,8.65313);
\draw [color=c, fill=c] (8.96517,8.54728) rectangle (9.00498,8.65313);
\draw [color=c, fill=c] (9.00498,8.54728) rectangle (9.04478,8.65313);
\draw [color=c, fill=c] (9.04478,8.54728) rectangle (9.08458,8.65313);
\draw [color=c, fill=c] (9.08458,8.54728) rectangle (9.12438,8.65313);
\draw [color=c, fill=c] (9.12438,8.54728) rectangle (9.16418,8.65313);
\draw [color=c, fill=c] (9.16418,8.54728) rectangle (9.20398,8.65313);
\draw [color=c, fill=c] (9.20398,8.54728) rectangle (9.24378,8.65313);
\draw [color=c, fill=c] (9.24378,8.54728) rectangle (9.28358,8.65313);
\draw [color=c, fill=c] (9.28358,8.54728) rectangle (9.32338,8.65313);
\draw [color=c, fill=c] (9.32338,8.54728) rectangle (9.36318,8.65313);
\draw [color=c, fill=c] (9.36318,8.54728) rectangle (9.40298,8.65313);
\draw [color=c, fill=c] (9.40298,8.54728) rectangle (9.44279,8.65313);
\draw [color=c, fill=c] (9.44279,8.54728) rectangle (9.48259,8.65313);
\definecolor{c}{rgb}{0,0.266667,1};
\draw [color=c, fill=c] (9.48259,8.54728) rectangle (9.52239,8.65313);
\draw [color=c, fill=c] (9.52239,8.54728) rectangle (9.56219,8.65313);
\draw [color=c, fill=c] (9.56219,8.54728) rectangle (9.60199,8.65313);
\draw [color=c, fill=c] (9.60199,8.54728) rectangle (9.64179,8.65313);
\draw [color=c, fill=c] (9.64179,8.54728) rectangle (9.68159,8.65313);
\draw [color=c, fill=c] (9.68159,8.54728) rectangle (9.72139,8.65313);
\draw [color=c, fill=c] (9.72139,8.54728) rectangle (9.76119,8.65313);
\draw [color=c, fill=c] (9.76119,8.54728) rectangle (9.80099,8.65313);
\draw [color=c, fill=c] (9.80099,8.54728) rectangle (9.8408,8.65313);
\draw [color=c, fill=c] (9.8408,8.54728) rectangle (9.8806,8.65313);
\draw [color=c, fill=c] (9.8806,8.54728) rectangle (9.9204,8.65313);
\draw [color=c, fill=c] (9.9204,8.54728) rectangle (9.9602,8.65313);
\draw [color=c, fill=c] (9.9602,8.54728) rectangle (10,8.65313);
\draw [color=c, fill=c] (10,8.54728) rectangle (10.0398,8.65313);
\draw [color=c, fill=c] (10.0398,8.54728) rectangle (10.0796,8.65313);
\draw [color=c, fill=c] (10.0796,8.54728) rectangle (10.1194,8.65313);
\draw [color=c, fill=c] (10.1194,8.54728) rectangle (10.1592,8.65313);
\draw [color=c, fill=c] (10.1592,8.54728) rectangle (10.199,8.65313);
\draw [color=c, fill=c] (10.199,8.54728) rectangle (10.2388,8.65313);
\draw [color=c, fill=c] (10.2388,8.54728) rectangle (10.2786,8.65313);
\draw [color=c, fill=c] (10.2786,8.54728) rectangle (10.3184,8.65313);
\draw [color=c, fill=c] (10.3184,8.54728) rectangle (10.3582,8.65313);
\draw [color=c, fill=c] (10.3582,8.54728) rectangle (10.398,8.65313);
\draw [color=c, fill=c] (10.398,8.54728) rectangle (10.4378,8.65313);
\draw [color=c, fill=c] (10.4378,8.54728) rectangle (10.4776,8.65313);
\draw [color=c, fill=c] (10.4776,8.54728) rectangle (10.5174,8.65313);
\definecolor{c}{rgb}{0,0.546666,1};
\draw [color=c, fill=c] (10.5174,8.54728) rectangle (10.5572,8.65313);
\draw [color=c, fill=c] (10.5572,8.54728) rectangle (10.597,8.65313);
\draw [color=c, fill=c] (10.597,8.54728) rectangle (10.6368,8.65313);
\draw [color=c, fill=c] (10.6368,8.54728) rectangle (10.6766,8.65313);
\draw [color=c, fill=c] (10.6766,8.54728) rectangle (10.7164,8.65313);
\draw [color=c, fill=c] (10.7164,8.54728) rectangle (10.7562,8.65313);
\draw [color=c, fill=c] (10.7562,8.54728) rectangle (10.796,8.65313);
\draw [color=c, fill=c] (10.796,8.54728) rectangle (10.8358,8.65313);
\draw [color=c, fill=c] (10.8358,8.54728) rectangle (10.8756,8.65313);
\draw [color=c, fill=c] (10.8756,8.54728) rectangle (10.9154,8.65313);
\draw [color=c, fill=c] (10.9154,8.54728) rectangle (10.9552,8.65313);
\draw [color=c, fill=c] (10.9552,8.54728) rectangle (10.995,8.65313);
\draw [color=c, fill=c] (10.995,8.54728) rectangle (11.0348,8.65313);
\draw [color=c, fill=c] (11.0348,8.54728) rectangle (11.0746,8.65313);
\draw [color=c, fill=c] (11.0746,8.54728) rectangle (11.1144,8.65313);
\draw [color=c, fill=c] (11.1144,8.54728) rectangle (11.1542,8.65313);
\draw [color=c, fill=c] (11.1542,8.54728) rectangle (11.194,8.65313);
\draw [color=c, fill=c] (11.194,8.54728) rectangle (11.2338,8.65313);
\draw [color=c, fill=c] (11.2338,8.54728) rectangle (11.2736,8.65313);
\draw [color=c, fill=c] (11.2736,8.54728) rectangle (11.3134,8.65313);
\draw [color=c, fill=c] (11.3134,8.54728) rectangle (11.3532,8.65313);
\draw [color=c, fill=c] (11.3532,8.54728) rectangle (11.393,8.65313);
\draw [color=c, fill=c] (11.393,8.54728) rectangle (11.4328,8.65313);
\draw [color=c, fill=c] (11.4328,8.54728) rectangle (11.4726,8.65313);
\draw [color=c, fill=c] (11.4726,8.54728) rectangle (11.5124,8.65313);
\draw [color=c, fill=c] (11.5124,8.54728) rectangle (11.5522,8.65313);
\draw [color=c, fill=c] (11.5522,8.54728) rectangle (11.592,8.65313);
\draw [color=c, fill=c] (11.592,8.54728) rectangle (11.6318,8.65313);
\draw [color=c, fill=c] (11.6318,8.54728) rectangle (11.6716,8.65313);
\draw [color=c, fill=c] (11.6716,8.54728) rectangle (11.7114,8.65313);
\draw [color=c, fill=c] (11.7114,8.54728) rectangle (11.7512,8.65313);
\draw [color=c, fill=c] (11.7512,8.54728) rectangle (11.791,8.65313);
\draw [color=c, fill=c] (11.791,8.54728) rectangle (11.8308,8.65313);
\draw [color=c, fill=c] (11.8308,8.54728) rectangle (11.8706,8.65313);
\draw [color=c, fill=c] (11.8706,8.54728) rectangle (11.9104,8.65313);
\draw [color=c, fill=c] (11.9104,8.54728) rectangle (11.9502,8.65313);
\draw [color=c, fill=c] (11.9502,8.54728) rectangle (11.99,8.65313);
\draw [color=c, fill=c] (11.99,8.54728) rectangle (12.0299,8.65313);
\draw [color=c, fill=c] (12.0299,8.54728) rectangle (12.0697,8.65313);
\draw [color=c, fill=c] (12.0697,8.54728) rectangle (12.1095,8.65313);
\draw [color=c, fill=c] (12.1095,8.54728) rectangle (12.1493,8.65313);
\draw [color=c, fill=c] (12.1493,8.54728) rectangle (12.1891,8.65313);
\draw [color=c, fill=c] (12.1891,8.54728) rectangle (12.2289,8.65313);
\draw [color=c, fill=c] (12.2289,8.54728) rectangle (12.2687,8.65313);
\draw [color=c, fill=c] (12.2687,8.54728) rectangle (12.3085,8.65313);
\draw [color=c, fill=c] (12.3085,8.54728) rectangle (12.3483,8.65313);
\draw [color=c, fill=c] (12.3483,8.54728) rectangle (12.3881,8.65313);
\draw [color=c, fill=c] (12.3881,8.54728) rectangle (12.4279,8.65313);
\draw [color=c, fill=c] (12.4279,8.54728) rectangle (12.4677,8.65313);
\draw [color=c, fill=c] (12.4677,8.54728) rectangle (12.5075,8.65313);
\draw [color=c, fill=c] (12.5075,8.54728) rectangle (12.5473,8.65313);
\draw [color=c, fill=c] (12.5473,8.54728) rectangle (12.5871,8.65313);
\draw [color=c, fill=c] (12.5871,8.54728) rectangle (12.6269,8.65313);
\draw [color=c, fill=c] (12.6269,8.54728) rectangle (12.6667,8.65313);
\draw [color=c, fill=c] (12.6667,8.54728) rectangle (12.7065,8.65313);
\draw [color=c, fill=c] (12.7065,8.54728) rectangle (12.7463,8.65313);
\draw [color=c, fill=c] (12.7463,8.54728) rectangle (12.7861,8.65313);
\definecolor{c}{rgb}{0,0.733333,1};
\draw [color=c, fill=c] (12.7861,8.54728) rectangle (12.8259,8.65313);
\draw [color=c, fill=c] (12.8259,8.54728) rectangle (12.8657,8.65313);
\draw [color=c, fill=c] (12.8657,8.54728) rectangle (12.9055,8.65313);
\draw [color=c, fill=c] (12.9055,8.54728) rectangle (12.9453,8.65313);
\draw [color=c, fill=c] (12.9453,8.54728) rectangle (12.9851,8.65313);
\draw [color=c, fill=c] (12.9851,8.54728) rectangle (13.0249,8.65313);
\draw [color=c, fill=c] (13.0249,8.54728) rectangle (13.0647,8.65313);
\draw [color=c, fill=c] (13.0647,8.54728) rectangle (13.1045,8.65313);
\draw [color=c, fill=c] (13.1045,8.54728) rectangle (13.1443,8.65313);
\draw [color=c, fill=c] (13.1443,8.54728) rectangle (13.1841,8.65313);
\draw [color=c, fill=c] (13.1841,8.54728) rectangle (13.2239,8.65313);
\draw [color=c, fill=c] (13.2239,8.54728) rectangle (13.2637,8.65313);
\draw [color=c, fill=c] (13.2637,8.54728) rectangle (13.3035,8.65313);
\draw [color=c, fill=c] (13.3035,8.54728) rectangle (13.3433,8.65313);
\draw [color=c, fill=c] (13.3433,8.54728) rectangle (13.3831,8.65313);
\draw [color=c, fill=c] (13.3831,8.54728) rectangle (13.4229,8.65313);
\draw [color=c, fill=c] (13.4229,8.54728) rectangle (13.4627,8.65313);
\draw [color=c, fill=c] (13.4627,8.54728) rectangle (13.5025,8.65313);
\draw [color=c, fill=c] (13.5025,8.54728) rectangle (13.5423,8.65313);
\draw [color=c, fill=c] (13.5423,8.54728) rectangle (13.5821,8.65313);
\draw [color=c, fill=c] (13.5821,8.54728) rectangle (13.6219,8.65313);
\draw [color=c, fill=c] (13.6219,8.54728) rectangle (13.6617,8.65313);
\draw [color=c, fill=c] (13.6617,8.54728) rectangle (13.7015,8.65313);
\draw [color=c, fill=c] (13.7015,8.54728) rectangle (13.7413,8.65313);
\draw [color=c, fill=c] (13.7413,8.54728) rectangle (13.7811,8.65313);
\draw [color=c, fill=c] (13.7811,8.54728) rectangle (13.8209,8.65313);
\draw [color=c, fill=c] (13.8209,8.54728) rectangle (13.8607,8.65313);
\draw [color=c, fill=c] (13.8607,8.54728) rectangle (13.9005,8.65313);
\draw [color=c, fill=c] (13.9005,8.54728) rectangle (13.9403,8.65313);
\draw [color=c, fill=c] (13.9403,8.54728) rectangle (13.9801,8.65313);
\draw [color=c, fill=c] (13.9801,8.54728) rectangle (14.0199,8.65313);
\draw [color=c, fill=c] (14.0199,8.54728) rectangle (14.0597,8.65313);
\draw [color=c, fill=c] (14.0597,8.54728) rectangle (14.0995,8.65313);
\draw [color=c, fill=c] (14.0995,8.54728) rectangle (14.1393,8.65313);
\draw [color=c, fill=c] (14.1393,8.54728) rectangle (14.1791,8.65313);
\draw [color=c, fill=c] (14.1791,8.54728) rectangle (14.2189,8.65313);
\draw [color=c, fill=c] (14.2189,8.54728) rectangle (14.2587,8.65313);
\draw [color=c, fill=c] (14.2587,8.54728) rectangle (14.2985,8.65313);
\draw [color=c, fill=c] (14.2985,8.54728) rectangle (14.3383,8.65313);
\draw [color=c, fill=c] (14.3383,8.54728) rectangle (14.3781,8.65313);
\draw [color=c, fill=c] (14.3781,8.54728) rectangle (14.4179,8.65313);
\draw [color=c, fill=c] (14.4179,8.54728) rectangle (14.4577,8.65313);
\draw [color=c, fill=c] (14.4577,8.54728) rectangle (14.4975,8.65313);
\draw [color=c, fill=c] (14.4975,8.54728) rectangle (14.5373,8.65313);
\draw [color=c, fill=c] (14.5373,8.54728) rectangle (14.5771,8.65313);
\draw [color=c, fill=c] (14.5771,8.54728) rectangle (14.6169,8.65313);
\draw [color=c, fill=c] (14.6169,8.54728) rectangle (14.6567,8.65313);
\draw [color=c, fill=c] (14.6567,8.54728) rectangle (14.6965,8.65313);
\draw [color=c, fill=c] (14.6965,8.54728) rectangle (14.7363,8.65313);
\draw [color=c, fill=c] (14.7363,8.54728) rectangle (14.7761,8.65313);
\draw [color=c, fill=c] (14.7761,8.54728) rectangle (14.8159,8.65313);
\draw [color=c, fill=c] (14.8159,8.54728) rectangle (14.8557,8.65313);
\draw [color=c, fill=c] (14.8557,8.54728) rectangle (14.8955,8.65313);
\draw [color=c, fill=c] (14.8955,8.54728) rectangle (14.9353,8.65313);
\draw [color=c, fill=c] (14.9353,8.54728) rectangle (14.9751,8.65313);
\draw [color=c, fill=c] (14.9751,8.54728) rectangle (15.0149,8.65313);
\draw [color=c, fill=c] (15.0149,8.54728) rectangle (15.0547,8.65313);
\draw [color=c, fill=c] (15.0547,8.54728) rectangle (15.0945,8.65313);
\draw [color=c, fill=c] (15.0945,8.54728) rectangle (15.1343,8.65313);
\draw [color=c, fill=c] (15.1343,8.54728) rectangle (15.1741,8.65313);
\draw [color=c, fill=c] (15.1741,8.54728) rectangle (15.2139,8.65313);
\draw [color=c, fill=c] (15.2139,8.54728) rectangle (15.2537,8.65313);
\draw [color=c, fill=c] (15.2537,8.54728) rectangle (15.2935,8.65313);
\draw [color=c, fill=c] (15.2935,8.54728) rectangle (15.3333,8.65313);
\draw [color=c, fill=c] (15.3333,8.54728) rectangle (15.3731,8.65313);
\draw [color=c, fill=c] (15.3731,8.54728) rectangle (15.4129,8.65313);
\draw [color=c, fill=c] (15.4129,8.54728) rectangle (15.4527,8.65313);
\draw [color=c, fill=c] (15.4527,8.54728) rectangle (15.4925,8.65313);
\draw [color=c, fill=c] (15.4925,8.54728) rectangle (15.5323,8.65313);
\draw [color=c, fill=c] (15.5323,8.54728) rectangle (15.5721,8.65313);
\draw [color=c, fill=c] (15.5721,8.54728) rectangle (15.6119,8.65313);
\draw [color=c, fill=c] (15.6119,8.54728) rectangle (15.6517,8.65313);
\draw [color=c, fill=c] (15.6517,8.54728) rectangle (15.6915,8.65313);
\draw [color=c, fill=c] (15.6915,8.54728) rectangle (15.7313,8.65313);
\draw [color=c, fill=c] (15.7313,8.54728) rectangle (15.7711,8.65313);
\draw [color=c, fill=c] (15.7711,8.54728) rectangle (15.8109,8.65313);
\draw [color=c, fill=c] (15.8109,8.54728) rectangle (15.8507,8.65313);
\draw [color=c, fill=c] (15.8507,8.54728) rectangle (15.8905,8.65313);
\draw [color=c, fill=c] (15.8905,8.54728) rectangle (15.9303,8.65313);
\draw [color=c, fill=c] (15.9303,8.54728) rectangle (15.9701,8.65313);
\draw [color=c, fill=c] (15.9701,8.54728) rectangle (16.01,8.65313);
\draw [color=c, fill=c] (16.01,8.54728) rectangle (16.0498,8.65313);
\draw [color=c, fill=c] (16.0498,8.54728) rectangle (16.0896,8.65313);
\draw [color=c, fill=c] (16.0896,8.54728) rectangle (16.1294,8.65313);
\draw [color=c, fill=c] (16.1294,8.54728) rectangle (16.1692,8.65313);
\draw [color=c, fill=c] (16.1692,8.54728) rectangle (16.209,8.65313);
\draw [color=c, fill=c] (16.209,8.54728) rectangle (16.2488,8.65313);
\draw [color=c, fill=c] (16.2488,8.54728) rectangle (16.2886,8.65313);
\draw [color=c, fill=c] (16.2886,8.54728) rectangle (16.3284,8.65313);
\draw [color=c, fill=c] (16.3284,8.54728) rectangle (16.3682,8.65313);
\draw [color=c, fill=c] (16.3682,8.54728) rectangle (16.408,8.65313);
\draw [color=c, fill=c] (16.408,8.54728) rectangle (16.4478,8.65313);
\draw [color=c, fill=c] (16.4478,8.54728) rectangle (16.4876,8.65313);
\draw [color=c, fill=c] (16.4876,8.54728) rectangle (16.5274,8.65313);
\draw [color=c, fill=c] (16.5274,8.54728) rectangle (16.5672,8.65313);
\draw [color=c, fill=c] (16.5672,8.54728) rectangle (16.607,8.65313);
\draw [color=c, fill=c] (16.607,8.54728) rectangle (16.6468,8.65313);
\draw [color=c, fill=c] (16.6468,8.54728) rectangle (16.6866,8.65313);
\draw [color=c, fill=c] (16.6866,8.54728) rectangle (16.7264,8.65313);
\draw [color=c, fill=c] (16.7264,8.54728) rectangle (16.7662,8.65313);
\draw [color=c, fill=c] (16.7662,8.54728) rectangle (16.806,8.65313);
\draw [color=c, fill=c] (16.806,8.54728) rectangle (16.8458,8.65313);
\draw [color=c, fill=c] (16.8458,8.54728) rectangle (16.8856,8.65313);
\draw [color=c, fill=c] (16.8856,8.54728) rectangle (16.9254,8.65313);
\draw [color=c, fill=c] (16.9254,8.54728) rectangle (16.9652,8.65313);
\draw [color=c, fill=c] (16.9652,8.54728) rectangle (17.005,8.65313);
\draw [color=c, fill=c] (17.005,8.54728) rectangle (17.0448,8.65313);
\draw [color=c, fill=c] (17.0448,8.54728) rectangle (17.0846,8.65313);
\draw [color=c, fill=c] (17.0846,8.54728) rectangle (17.1244,8.65313);
\draw [color=c, fill=c] (17.1244,8.54728) rectangle (17.1642,8.65313);
\draw [color=c, fill=c] (17.1642,8.54728) rectangle (17.204,8.65313);
\draw [color=c, fill=c] (17.204,8.54728) rectangle (17.2438,8.65313);
\draw [color=c, fill=c] (17.2438,8.54728) rectangle (17.2836,8.65313);
\draw [color=c, fill=c] (17.2836,8.54728) rectangle (17.3234,8.65313);
\draw [color=c, fill=c] (17.3234,8.54728) rectangle (17.3632,8.65313);
\draw [color=c, fill=c] (17.3632,8.54728) rectangle (17.403,8.65313);
\draw [color=c, fill=c] (17.403,8.54728) rectangle (17.4428,8.65313);
\draw [color=c, fill=c] (17.4428,8.54728) rectangle (17.4826,8.65313);
\draw [color=c, fill=c] (17.4826,8.54728) rectangle (17.5224,8.65313);
\draw [color=c, fill=c] (17.5224,8.54728) rectangle (17.5622,8.65313);
\draw [color=c, fill=c] (17.5622,8.54728) rectangle (17.602,8.65313);
\draw [color=c, fill=c] (17.602,8.54728) rectangle (17.6418,8.65313);
\draw [color=c, fill=c] (17.6418,8.54728) rectangle (17.6816,8.65313);
\draw [color=c, fill=c] (17.6816,8.54728) rectangle (17.7214,8.65313);
\draw [color=c, fill=c] (17.7214,8.54728) rectangle (17.7612,8.65313);
\draw [color=c, fill=c] (17.7612,8.54728) rectangle (17.801,8.65313);
\draw [color=c, fill=c] (17.801,8.54728) rectangle (17.8408,8.65313);
\draw [color=c, fill=c] (17.8408,8.54728) rectangle (17.8806,8.65313);
\draw [color=c, fill=c] (17.8806,8.54728) rectangle (17.9204,8.65313);
\draw [color=c, fill=c] (17.9204,8.54728) rectangle (17.9602,8.65313);
\draw [color=c, fill=c] (17.9602,8.54728) rectangle (18,8.65313);
\definecolor{c}{rgb}{0.2,0,1};
\draw [color=c, fill=c] (2,8.65313) rectangle (2.0398,8.75898);
\draw [color=c, fill=c] (2.0398,8.65313) rectangle (2.0796,8.75898);
\draw [color=c, fill=c] (2.0796,8.65313) rectangle (2.1194,8.75898);
\draw [color=c, fill=c] (2.1194,8.65313) rectangle (2.1592,8.75898);
\draw [color=c, fill=c] (2.1592,8.65313) rectangle (2.19901,8.75898);
\draw [color=c, fill=c] (2.19901,8.65313) rectangle (2.23881,8.75898);
\draw [color=c, fill=c] (2.23881,8.65313) rectangle (2.27861,8.75898);
\draw [color=c, fill=c] (2.27861,8.65313) rectangle (2.31841,8.75898);
\draw [color=c, fill=c] (2.31841,8.65313) rectangle (2.35821,8.75898);
\draw [color=c, fill=c] (2.35821,8.65313) rectangle (2.39801,8.75898);
\draw [color=c, fill=c] (2.39801,8.65313) rectangle (2.43781,8.75898);
\draw [color=c, fill=c] (2.43781,8.65313) rectangle (2.47761,8.75898);
\draw [color=c, fill=c] (2.47761,8.65313) rectangle (2.51741,8.75898);
\draw [color=c, fill=c] (2.51741,8.65313) rectangle (2.55721,8.75898);
\draw [color=c, fill=c] (2.55721,8.65313) rectangle (2.59702,8.75898);
\draw [color=c, fill=c] (2.59702,8.65313) rectangle (2.63682,8.75898);
\draw [color=c, fill=c] (2.63682,8.65313) rectangle (2.67662,8.75898);
\draw [color=c, fill=c] (2.67662,8.65313) rectangle (2.71642,8.75898);
\draw [color=c, fill=c] (2.71642,8.65313) rectangle (2.75622,8.75898);
\draw [color=c, fill=c] (2.75622,8.65313) rectangle (2.79602,8.75898);
\draw [color=c, fill=c] (2.79602,8.65313) rectangle (2.83582,8.75898);
\draw [color=c, fill=c] (2.83582,8.65313) rectangle (2.87562,8.75898);
\draw [color=c, fill=c] (2.87562,8.65313) rectangle (2.91542,8.75898);
\draw [color=c, fill=c] (2.91542,8.65313) rectangle (2.95522,8.75898);
\draw [color=c, fill=c] (2.95522,8.65313) rectangle (2.99502,8.75898);
\draw [color=c, fill=c] (2.99502,8.65313) rectangle (3.03483,8.75898);
\draw [color=c, fill=c] (3.03483,8.65313) rectangle (3.07463,8.75898);
\draw [color=c, fill=c] (3.07463,8.65313) rectangle (3.11443,8.75898);
\draw [color=c, fill=c] (3.11443,8.65313) rectangle (3.15423,8.75898);
\draw [color=c, fill=c] (3.15423,8.65313) rectangle (3.19403,8.75898);
\draw [color=c, fill=c] (3.19403,8.65313) rectangle (3.23383,8.75898);
\draw [color=c, fill=c] (3.23383,8.65313) rectangle (3.27363,8.75898);
\draw [color=c, fill=c] (3.27363,8.65313) rectangle (3.31343,8.75898);
\draw [color=c, fill=c] (3.31343,8.65313) rectangle (3.35323,8.75898);
\draw [color=c, fill=c] (3.35323,8.65313) rectangle (3.39303,8.75898);
\draw [color=c, fill=c] (3.39303,8.65313) rectangle (3.43284,8.75898);
\draw [color=c, fill=c] (3.43284,8.65313) rectangle (3.47264,8.75898);
\draw [color=c, fill=c] (3.47264,8.65313) rectangle (3.51244,8.75898);
\draw [color=c, fill=c] (3.51244,8.65313) rectangle (3.55224,8.75898);
\draw [color=c, fill=c] (3.55224,8.65313) rectangle (3.59204,8.75898);
\draw [color=c, fill=c] (3.59204,8.65313) rectangle (3.63184,8.75898);
\draw [color=c, fill=c] (3.63184,8.65313) rectangle (3.67164,8.75898);
\draw [color=c, fill=c] (3.67164,8.65313) rectangle (3.71144,8.75898);
\draw [color=c, fill=c] (3.71144,8.65313) rectangle (3.75124,8.75898);
\draw [color=c, fill=c] (3.75124,8.65313) rectangle (3.79104,8.75898);
\draw [color=c, fill=c] (3.79104,8.65313) rectangle (3.83085,8.75898);
\draw [color=c, fill=c] (3.83085,8.65313) rectangle (3.87065,8.75898);
\draw [color=c, fill=c] (3.87065,8.65313) rectangle (3.91045,8.75898);
\draw [color=c, fill=c] (3.91045,8.65313) rectangle (3.95025,8.75898);
\draw [color=c, fill=c] (3.95025,8.65313) rectangle (3.99005,8.75898);
\draw [color=c, fill=c] (3.99005,8.65313) rectangle (4.02985,8.75898);
\draw [color=c, fill=c] (4.02985,8.65313) rectangle (4.06965,8.75898);
\draw [color=c, fill=c] (4.06965,8.65313) rectangle (4.10945,8.75898);
\draw [color=c, fill=c] (4.10945,8.65313) rectangle (4.14925,8.75898);
\draw [color=c, fill=c] (4.14925,8.65313) rectangle (4.18905,8.75898);
\draw [color=c, fill=c] (4.18905,8.65313) rectangle (4.22886,8.75898);
\draw [color=c, fill=c] (4.22886,8.65313) rectangle (4.26866,8.75898);
\draw [color=c, fill=c] (4.26866,8.65313) rectangle (4.30846,8.75898);
\draw [color=c, fill=c] (4.30846,8.65313) rectangle (4.34826,8.75898);
\draw [color=c, fill=c] (4.34826,8.65313) rectangle (4.38806,8.75898);
\draw [color=c, fill=c] (4.38806,8.65313) rectangle (4.42786,8.75898);
\draw [color=c, fill=c] (4.42786,8.65313) rectangle (4.46766,8.75898);
\draw [color=c, fill=c] (4.46766,8.65313) rectangle (4.50746,8.75898);
\draw [color=c, fill=c] (4.50746,8.65313) rectangle (4.54726,8.75898);
\draw [color=c, fill=c] (4.54726,8.65313) rectangle (4.58706,8.75898);
\draw [color=c, fill=c] (4.58706,8.65313) rectangle (4.62687,8.75898);
\draw [color=c, fill=c] (4.62687,8.65313) rectangle (4.66667,8.75898);
\draw [color=c, fill=c] (4.66667,8.65313) rectangle (4.70647,8.75898);
\draw [color=c, fill=c] (4.70647,8.65313) rectangle (4.74627,8.75898);
\draw [color=c, fill=c] (4.74627,8.65313) rectangle (4.78607,8.75898);
\draw [color=c, fill=c] (4.78607,8.65313) rectangle (4.82587,8.75898);
\draw [color=c, fill=c] (4.82587,8.65313) rectangle (4.86567,8.75898);
\draw [color=c, fill=c] (4.86567,8.65313) rectangle (4.90547,8.75898);
\draw [color=c, fill=c] (4.90547,8.65313) rectangle (4.94527,8.75898);
\draw [color=c, fill=c] (4.94527,8.65313) rectangle (4.98507,8.75898);
\draw [color=c, fill=c] (4.98507,8.65313) rectangle (5.02488,8.75898);
\draw [color=c, fill=c] (5.02488,8.65313) rectangle (5.06468,8.75898);
\draw [color=c, fill=c] (5.06468,8.65313) rectangle (5.10448,8.75898);
\draw [color=c, fill=c] (5.10448,8.65313) rectangle (5.14428,8.75898);
\draw [color=c, fill=c] (5.14428,8.65313) rectangle (5.18408,8.75898);
\draw [color=c, fill=c] (5.18408,8.65313) rectangle (5.22388,8.75898);
\draw [color=c, fill=c] (5.22388,8.65313) rectangle (5.26368,8.75898);
\draw [color=c, fill=c] (5.26368,8.65313) rectangle (5.30348,8.75898);
\draw [color=c, fill=c] (5.30348,8.65313) rectangle (5.34328,8.75898);
\draw [color=c, fill=c] (5.34328,8.65313) rectangle (5.38308,8.75898);
\draw [color=c, fill=c] (5.38308,8.65313) rectangle (5.42289,8.75898);
\draw [color=c, fill=c] (5.42289,8.65313) rectangle (5.46269,8.75898);
\draw [color=c, fill=c] (5.46269,8.65313) rectangle (5.50249,8.75898);
\draw [color=c, fill=c] (5.50249,8.65313) rectangle (5.54229,8.75898);
\draw [color=c, fill=c] (5.54229,8.65313) rectangle (5.58209,8.75898);
\draw [color=c, fill=c] (5.58209,8.65313) rectangle (5.62189,8.75898);
\draw [color=c, fill=c] (5.62189,8.65313) rectangle (5.66169,8.75898);
\draw [color=c, fill=c] (5.66169,8.65313) rectangle (5.70149,8.75898);
\draw [color=c, fill=c] (5.70149,8.65313) rectangle (5.74129,8.75898);
\draw [color=c, fill=c] (5.74129,8.65313) rectangle (5.78109,8.75898);
\draw [color=c, fill=c] (5.78109,8.65313) rectangle (5.8209,8.75898);
\draw [color=c, fill=c] (5.8209,8.65313) rectangle (5.8607,8.75898);
\draw [color=c, fill=c] (5.8607,8.65313) rectangle (5.9005,8.75898);
\draw [color=c, fill=c] (5.9005,8.65313) rectangle (5.9403,8.75898);
\draw [color=c, fill=c] (5.9403,8.65313) rectangle (5.9801,8.75898);
\draw [color=c, fill=c] (5.9801,8.65313) rectangle (6.0199,8.75898);
\draw [color=c, fill=c] (6.0199,8.65313) rectangle (6.0597,8.75898);
\draw [color=c, fill=c] (6.0597,8.65313) rectangle (6.0995,8.75898);
\draw [color=c, fill=c] (6.0995,8.65313) rectangle (6.1393,8.75898);
\draw [color=c, fill=c] (6.1393,8.65313) rectangle (6.1791,8.75898);
\draw [color=c, fill=c] (6.1791,8.65313) rectangle (6.21891,8.75898);
\draw [color=c, fill=c] (6.21891,8.65313) rectangle (6.25871,8.75898);
\draw [color=c, fill=c] (6.25871,8.65313) rectangle (6.29851,8.75898);
\draw [color=c, fill=c] (6.29851,8.65313) rectangle (6.33831,8.75898);
\draw [color=c, fill=c] (6.33831,8.65313) rectangle (6.37811,8.75898);
\draw [color=c, fill=c] (6.37811,8.65313) rectangle (6.41791,8.75898);
\draw [color=c, fill=c] (6.41791,8.65313) rectangle (6.45771,8.75898);
\draw [color=c, fill=c] (6.45771,8.65313) rectangle (6.49751,8.75898);
\draw [color=c, fill=c] (6.49751,8.65313) rectangle (6.53731,8.75898);
\draw [color=c, fill=c] (6.53731,8.65313) rectangle (6.57711,8.75898);
\draw [color=c, fill=c] (6.57711,8.65313) rectangle (6.61692,8.75898);
\draw [color=c, fill=c] (6.61692,8.65313) rectangle (6.65672,8.75898);
\draw [color=c, fill=c] (6.65672,8.65313) rectangle (6.69652,8.75898);
\draw [color=c, fill=c] (6.69652,8.65313) rectangle (6.73632,8.75898);
\draw [color=c, fill=c] (6.73632,8.65313) rectangle (6.77612,8.75898);
\draw [color=c, fill=c] (6.77612,8.65313) rectangle (6.81592,8.75898);
\draw [color=c, fill=c] (6.81592,8.65313) rectangle (6.85572,8.75898);
\draw [color=c, fill=c] (6.85572,8.65313) rectangle (6.89552,8.75898);
\draw [color=c, fill=c] (6.89552,8.65313) rectangle (6.93532,8.75898);
\draw [color=c, fill=c] (6.93532,8.65313) rectangle (6.97512,8.75898);
\draw [color=c, fill=c] (6.97512,8.65313) rectangle (7.01493,8.75898);
\draw [color=c, fill=c] (7.01493,8.65313) rectangle (7.05473,8.75898);
\draw [color=c, fill=c] (7.05473,8.65313) rectangle (7.09453,8.75898);
\draw [color=c, fill=c] (7.09453,8.65313) rectangle (7.13433,8.75898);
\draw [color=c, fill=c] (7.13433,8.65313) rectangle (7.17413,8.75898);
\draw [color=c, fill=c] (7.17413,8.65313) rectangle (7.21393,8.75898);
\draw [color=c, fill=c] (7.21393,8.65313) rectangle (7.25373,8.75898);
\draw [color=c, fill=c] (7.25373,8.65313) rectangle (7.29353,8.75898);
\draw [color=c, fill=c] (7.29353,8.65313) rectangle (7.33333,8.75898);
\draw [color=c, fill=c] (7.33333,8.65313) rectangle (7.37313,8.75898);
\draw [color=c, fill=c] (7.37313,8.65313) rectangle (7.41294,8.75898);
\draw [color=c, fill=c] (7.41294,8.65313) rectangle (7.45274,8.75898);
\draw [color=c, fill=c] (7.45274,8.65313) rectangle (7.49254,8.75898);
\draw [color=c, fill=c] (7.49254,8.65313) rectangle (7.53234,8.75898);
\draw [color=c, fill=c] (7.53234,8.65313) rectangle (7.57214,8.75898);
\draw [color=c, fill=c] (7.57214,8.65313) rectangle (7.61194,8.75898);
\definecolor{c}{rgb}{0,0.0800001,1};
\draw [color=c, fill=c] (7.61194,8.65313) rectangle (7.65174,8.75898);
\draw [color=c, fill=c] (7.65174,8.65313) rectangle (7.69154,8.75898);
\draw [color=c, fill=c] (7.69154,8.65313) rectangle (7.73134,8.75898);
\draw [color=c, fill=c] (7.73134,8.65313) rectangle (7.77114,8.75898);
\draw [color=c, fill=c] (7.77114,8.65313) rectangle (7.81095,8.75898);
\draw [color=c, fill=c] (7.81095,8.65313) rectangle (7.85075,8.75898);
\draw [color=c, fill=c] (7.85075,8.65313) rectangle (7.89055,8.75898);
\draw [color=c, fill=c] (7.89055,8.65313) rectangle (7.93035,8.75898);
\draw [color=c, fill=c] (7.93035,8.65313) rectangle (7.97015,8.75898);
\draw [color=c, fill=c] (7.97015,8.65313) rectangle (8.00995,8.75898);
\draw [color=c, fill=c] (8.00995,8.65313) rectangle (8.04975,8.75898);
\draw [color=c, fill=c] (8.04975,8.65313) rectangle (8.08955,8.75898);
\draw [color=c, fill=c] (8.08955,8.65313) rectangle (8.12935,8.75898);
\draw [color=c, fill=c] (8.12935,8.65313) rectangle (8.16915,8.75898);
\draw [color=c, fill=c] (8.16915,8.65313) rectangle (8.20895,8.75898);
\draw [color=c, fill=c] (8.20895,8.65313) rectangle (8.24876,8.75898);
\draw [color=c, fill=c] (8.24876,8.65313) rectangle (8.28856,8.75898);
\draw [color=c, fill=c] (8.28856,8.65313) rectangle (8.32836,8.75898);
\draw [color=c, fill=c] (8.32836,8.65313) rectangle (8.36816,8.75898);
\draw [color=c, fill=c] (8.36816,8.65313) rectangle (8.40796,8.75898);
\draw [color=c, fill=c] (8.40796,8.65313) rectangle (8.44776,8.75898);
\draw [color=c, fill=c] (8.44776,8.65313) rectangle (8.48756,8.75898);
\draw [color=c, fill=c] (8.48756,8.65313) rectangle (8.52736,8.75898);
\draw [color=c, fill=c] (8.52736,8.65313) rectangle (8.56716,8.75898);
\draw [color=c, fill=c] (8.56716,8.65313) rectangle (8.60697,8.75898);
\draw [color=c, fill=c] (8.60697,8.65313) rectangle (8.64677,8.75898);
\draw [color=c, fill=c] (8.64677,8.65313) rectangle (8.68657,8.75898);
\draw [color=c, fill=c] (8.68657,8.65313) rectangle (8.72637,8.75898);
\draw [color=c, fill=c] (8.72637,8.65313) rectangle (8.76617,8.75898);
\draw [color=c, fill=c] (8.76617,8.65313) rectangle (8.80597,8.75898);
\draw [color=c, fill=c] (8.80597,8.65313) rectangle (8.84577,8.75898);
\draw [color=c, fill=c] (8.84577,8.65313) rectangle (8.88557,8.75898);
\draw [color=c, fill=c] (8.88557,8.65313) rectangle (8.92537,8.75898);
\draw [color=c, fill=c] (8.92537,8.65313) rectangle (8.96517,8.75898);
\draw [color=c, fill=c] (8.96517,8.65313) rectangle (9.00498,8.75898);
\draw [color=c, fill=c] (9.00498,8.65313) rectangle (9.04478,8.75898);
\draw [color=c, fill=c] (9.04478,8.65313) rectangle (9.08458,8.75898);
\draw [color=c, fill=c] (9.08458,8.65313) rectangle (9.12438,8.75898);
\draw [color=c, fill=c] (9.12438,8.65313) rectangle (9.16418,8.75898);
\draw [color=c, fill=c] (9.16418,8.65313) rectangle (9.20398,8.75898);
\draw [color=c, fill=c] (9.20398,8.65313) rectangle (9.24378,8.75898);
\draw [color=c, fill=c] (9.24378,8.65313) rectangle (9.28358,8.75898);
\draw [color=c, fill=c] (9.28358,8.65313) rectangle (9.32338,8.75898);
\draw [color=c, fill=c] (9.32338,8.65313) rectangle (9.36318,8.75898);
\draw [color=c, fill=c] (9.36318,8.65313) rectangle (9.40298,8.75898);
\draw [color=c, fill=c] (9.40298,8.65313) rectangle (9.44279,8.75898);
\draw [color=c, fill=c] (9.44279,8.65313) rectangle (9.48259,8.75898);
\definecolor{c}{rgb}{0,0.266667,1};
\draw [color=c, fill=c] (9.48259,8.65313) rectangle (9.52239,8.75898);
\draw [color=c, fill=c] (9.52239,8.65313) rectangle (9.56219,8.75898);
\draw [color=c, fill=c] (9.56219,8.65313) rectangle (9.60199,8.75898);
\draw [color=c, fill=c] (9.60199,8.65313) rectangle (9.64179,8.75898);
\draw [color=c, fill=c] (9.64179,8.65313) rectangle (9.68159,8.75898);
\draw [color=c, fill=c] (9.68159,8.65313) rectangle (9.72139,8.75898);
\draw [color=c, fill=c] (9.72139,8.65313) rectangle (9.76119,8.75898);
\draw [color=c, fill=c] (9.76119,8.65313) rectangle (9.80099,8.75898);
\draw [color=c, fill=c] (9.80099,8.65313) rectangle (9.8408,8.75898);
\draw [color=c, fill=c] (9.8408,8.65313) rectangle (9.8806,8.75898);
\draw [color=c, fill=c] (9.8806,8.65313) rectangle (9.9204,8.75898);
\draw [color=c, fill=c] (9.9204,8.65313) rectangle (9.9602,8.75898);
\draw [color=c, fill=c] (9.9602,8.65313) rectangle (10,8.75898);
\draw [color=c, fill=c] (10,8.65313) rectangle (10.0398,8.75898);
\draw [color=c, fill=c] (10.0398,8.65313) rectangle (10.0796,8.75898);
\draw [color=c, fill=c] (10.0796,8.65313) rectangle (10.1194,8.75898);
\draw [color=c, fill=c] (10.1194,8.65313) rectangle (10.1592,8.75898);
\draw [color=c, fill=c] (10.1592,8.65313) rectangle (10.199,8.75898);
\draw [color=c, fill=c] (10.199,8.65313) rectangle (10.2388,8.75898);
\draw [color=c, fill=c] (10.2388,8.65313) rectangle (10.2786,8.75898);
\draw [color=c, fill=c] (10.2786,8.65313) rectangle (10.3184,8.75898);
\draw [color=c, fill=c] (10.3184,8.65313) rectangle (10.3582,8.75898);
\draw [color=c, fill=c] (10.3582,8.65313) rectangle (10.398,8.75898);
\draw [color=c, fill=c] (10.398,8.65313) rectangle (10.4378,8.75898);
\draw [color=c, fill=c] (10.4378,8.65313) rectangle (10.4776,8.75898);
\draw [color=c, fill=c] (10.4776,8.65313) rectangle (10.5174,8.75898);
\definecolor{c}{rgb}{0,0.546666,1};
\draw [color=c, fill=c] (10.5174,8.65313) rectangle (10.5572,8.75898);
\draw [color=c, fill=c] (10.5572,8.65313) rectangle (10.597,8.75898);
\draw [color=c, fill=c] (10.597,8.65313) rectangle (10.6368,8.75898);
\draw [color=c, fill=c] (10.6368,8.65313) rectangle (10.6766,8.75898);
\draw [color=c, fill=c] (10.6766,8.65313) rectangle (10.7164,8.75898);
\draw [color=c, fill=c] (10.7164,8.65313) rectangle (10.7562,8.75898);
\draw [color=c, fill=c] (10.7562,8.65313) rectangle (10.796,8.75898);
\draw [color=c, fill=c] (10.796,8.65313) rectangle (10.8358,8.75898);
\draw [color=c, fill=c] (10.8358,8.65313) rectangle (10.8756,8.75898);
\draw [color=c, fill=c] (10.8756,8.65313) rectangle (10.9154,8.75898);
\draw [color=c, fill=c] (10.9154,8.65313) rectangle (10.9552,8.75898);
\draw [color=c, fill=c] (10.9552,8.65313) rectangle (10.995,8.75898);
\draw [color=c, fill=c] (10.995,8.65313) rectangle (11.0348,8.75898);
\draw [color=c, fill=c] (11.0348,8.65313) rectangle (11.0746,8.75898);
\draw [color=c, fill=c] (11.0746,8.65313) rectangle (11.1144,8.75898);
\draw [color=c, fill=c] (11.1144,8.65313) rectangle (11.1542,8.75898);
\draw [color=c, fill=c] (11.1542,8.65313) rectangle (11.194,8.75898);
\draw [color=c, fill=c] (11.194,8.65313) rectangle (11.2338,8.75898);
\draw [color=c, fill=c] (11.2338,8.65313) rectangle (11.2736,8.75898);
\draw [color=c, fill=c] (11.2736,8.65313) rectangle (11.3134,8.75898);
\draw [color=c, fill=c] (11.3134,8.65313) rectangle (11.3532,8.75898);
\draw [color=c, fill=c] (11.3532,8.65313) rectangle (11.393,8.75898);
\draw [color=c, fill=c] (11.393,8.65313) rectangle (11.4328,8.75898);
\draw [color=c, fill=c] (11.4328,8.65313) rectangle (11.4726,8.75898);
\draw [color=c, fill=c] (11.4726,8.65313) rectangle (11.5124,8.75898);
\draw [color=c, fill=c] (11.5124,8.65313) rectangle (11.5522,8.75898);
\draw [color=c, fill=c] (11.5522,8.65313) rectangle (11.592,8.75898);
\draw [color=c, fill=c] (11.592,8.65313) rectangle (11.6318,8.75898);
\draw [color=c, fill=c] (11.6318,8.65313) rectangle (11.6716,8.75898);
\draw [color=c, fill=c] (11.6716,8.65313) rectangle (11.7114,8.75898);
\draw [color=c, fill=c] (11.7114,8.65313) rectangle (11.7512,8.75898);
\draw [color=c, fill=c] (11.7512,8.65313) rectangle (11.791,8.75898);
\draw [color=c, fill=c] (11.791,8.65313) rectangle (11.8308,8.75898);
\draw [color=c, fill=c] (11.8308,8.65313) rectangle (11.8706,8.75898);
\draw [color=c, fill=c] (11.8706,8.65313) rectangle (11.9104,8.75898);
\draw [color=c, fill=c] (11.9104,8.65313) rectangle (11.9502,8.75898);
\draw [color=c, fill=c] (11.9502,8.65313) rectangle (11.99,8.75898);
\draw [color=c, fill=c] (11.99,8.65313) rectangle (12.0299,8.75898);
\draw [color=c, fill=c] (12.0299,8.65313) rectangle (12.0697,8.75898);
\draw [color=c, fill=c] (12.0697,8.65313) rectangle (12.1095,8.75898);
\draw [color=c, fill=c] (12.1095,8.65313) rectangle (12.1493,8.75898);
\draw [color=c, fill=c] (12.1493,8.65313) rectangle (12.1891,8.75898);
\draw [color=c, fill=c] (12.1891,8.65313) rectangle (12.2289,8.75898);
\draw [color=c, fill=c] (12.2289,8.65313) rectangle (12.2687,8.75898);
\draw [color=c, fill=c] (12.2687,8.65313) rectangle (12.3085,8.75898);
\draw [color=c, fill=c] (12.3085,8.65313) rectangle (12.3483,8.75898);
\draw [color=c, fill=c] (12.3483,8.65313) rectangle (12.3881,8.75898);
\draw [color=c, fill=c] (12.3881,8.65313) rectangle (12.4279,8.75898);
\draw [color=c, fill=c] (12.4279,8.65313) rectangle (12.4677,8.75898);
\draw [color=c, fill=c] (12.4677,8.65313) rectangle (12.5075,8.75898);
\draw [color=c, fill=c] (12.5075,8.65313) rectangle (12.5473,8.75898);
\draw [color=c, fill=c] (12.5473,8.65313) rectangle (12.5871,8.75898);
\draw [color=c, fill=c] (12.5871,8.65313) rectangle (12.6269,8.75898);
\draw [color=c, fill=c] (12.6269,8.65313) rectangle (12.6667,8.75898);
\draw [color=c, fill=c] (12.6667,8.65313) rectangle (12.7065,8.75898);
\draw [color=c, fill=c] (12.7065,8.65313) rectangle (12.7463,8.75898);
\draw [color=c, fill=c] (12.7463,8.65313) rectangle (12.7861,8.75898);
\draw [color=c, fill=c] (12.7861,8.65313) rectangle (12.8259,8.75898);
\draw [color=c, fill=c] (12.8259,8.65313) rectangle (12.8657,8.75898);
\draw [color=c, fill=c] (12.8657,8.65313) rectangle (12.9055,8.75898);
\definecolor{c}{rgb}{0,0.733333,1};
\draw [color=c, fill=c] (12.9055,8.65313) rectangle (12.9453,8.75898);
\draw [color=c, fill=c] (12.9453,8.65313) rectangle (12.9851,8.75898);
\draw [color=c, fill=c] (12.9851,8.65313) rectangle (13.0249,8.75898);
\draw [color=c, fill=c] (13.0249,8.65313) rectangle (13.0647,8.75898);
\draw [color=c, fill=c] (13.0647,8.65313) rectangle (13.1045,8.75898);
\draw [color=c, fill=c] (13.1045,8.65313) rectangle (13.1443,8.75898);
\draw [color=c, fill=c] (13.1443,8.65313) rectangle (13.1841,8.75898);
\draw [color=c, fill=c] (13.1841,8.65313) rectangle (13.2239,8.75898);
\draw [color=c, fill=c] (13.2239,8.65313) rectangle (13.2637,8.75898);
\draw [color=c, fill=c] (13.2637,8.65313) rectangle (13.3035,8.75898);
\draw [color=c, fill=c] (13.3035,8.65313) rectangle (13.3433,8.75898);
\draw [color=c, fill=c] (13.3433,8.65313) rectangle (13.3831,8.75898);
\draw [color=c, fill=c] (13.3831,8.65313) rectangle (13.4229,8.75898);
\draw [color=c, fill=c] (13.4229,8.65313) rectangle (13.4627,8.75898);
\draw [color=c, fill=c] (13.4627,8.65313) rectangle (13.5025,8.75898);
\draw [color=c, fill=c] (13.5025,8.65313) rectangle (13.5423,8.75898);
\draw [color=c, fill=c] (13.5423,8.65313) rectangle (13.5821,8.75898);
\draw [color=c, fill=c] (13.5821,8.65313) rectangle (13.6219,8.75898);
\draw [color=c, fill=c] (13.6219,8.65313) rectangle (13.6617,8.75898);
\draw [color=c, fill=c] (13.6617,8.65313) rectangle (13.7015,8.75898);
\draw [color=c, fill=c] (13.7015,8.65313) rectangle (13.7413,8.75898);
\draw [color=c, fill=c] (13.7413,8.65313) rectangle (13.7811,8.75898);
\draw [color=c, fill=c] (13.7811,8.65313) rectangle (13.8209,8.75898);
\draw [color=c, fill=c] (13.8209,8.65313) rectangle (13.8607,8.75898);
\draw [color=c, fill=c] (13.8607,8.65313) rectangle (13.9005,8.75898);
\draw [color=c, fill=c] (13.9005,8.65313) rectangle (13.9403,8.75898);
\draw [color=c, fill=c] (13.9403,8.65313) rectangle (13.9801,8.75898);
\draw [color=c, fill=c] (13.9801,8.65313) rectangle (14.0199,8.75898);
\draw [color=c, fill=c] (14.0199,8.65313) rectangle (14.0597,8.75898);
\draw [color=c, fill=c] (14.0597,8.65313) rectangle (14.0995,8.75898);
\draw [color=c, fill=c] (14.0995,8.65313) rectangle (14.1393,8.75898);
\draw [color=c, fill=c] (14.1393,8.65313) rectangle (14.1791,8.75898);
\draw [color=c, fill=c] (14.1791,8.65313) rectangle (14.2189,8.75898);
\draw [color=c, fill=c] (14.2189,8.65313) rectangle (14.2587,8.75898);
\draw [color=c, fill=c] (14.2587,8.65313) rectangle (14.2985,8.75898);
\draw [color=c, fill=c] (14.2985,8.65313) rectangle (14.3383,8.75898);
\draw [color=c, fill=c] (14.3383,8.65313) rectangle (14.3781,8.75898);
\draw [color=c, fill=c] (14.3781,8.65313) rectangle (14.4179,8.75898);
\draw [color=c, fill=c] (14.4179,8.65313) rectangle (14.4577,8.75898);
\draw [color=c, fill=c] (14.4577,8.65313) rectangle (14.4975,8.75898);
\draw [color=c, fill=c] (14.4975,8.65313) rectangle (14.5373,8.75898);
\draw [color=c, fill=c] (14.5373,8.65313) rectangle (14.5771,8.75898);
\draw [color=c, fill=c] (14.5771,8.65313) rectangle (14.6169,8.75898);
\draw [color=c, fill=c] (14.6169,8.65313) rectangle (14.6567,8.75898);
\draw [color=c, fill=c] (14.6567,8.65313) rectangle (14.6965,8.75898);
\draw [color=c, fill=c] (14.6965,8.65313) rectangle (14.7363,8.75898);
\draw [color=c, fill=c] (14.7363,8.65313) rectangle (14.7761,8.75898);
\draw [color=c, fill=c] (14.7761,8.65313) rectangle (14.8159,8.75898);
\draw [color=c, fill=c] (14.8159,8.65313) rectangle (14.8557,8.75898);
\draw [color=c, fill=c] (14.8557,8.65313) rectangle (14.8955,8.75898);
\draw [color=c, fill=c] (14.8955,8.65313) rectangle (14.9353,8.75898);
\draw [color=c, fill=c] (14.9353,8.65313) rectangle (14.9751,8.75898);
\draw [color=c, fill=c] (14.9751,8.65313) rectangle (15.0149,8.75898);
\draw [color=c, fill=c] (15.0149,8.65313) rectangle (15.0547,8.75898);
\draw [color=c, fill=c] (15.0547,8.65313) rectangle (15.0945,8.75898);
\draw [color=c, fill=c] (15.0945,8.65313) rectangle (15.1343,8.75898);
\draw [color=c, fill=c] (15.1343,8.65313) rectangle (15.1741,8.75898);
\draw [color=c, fill=c] (15.1741,8.65313) rectangle (15.2139,8.75898);
\draw [color=c, fill=c] (15.2139,8.65313) rectangle (15.2537,8.75898);
\draw [color=c, fill=c] (15.2537,8.65313) rectangle (15.2935,8.75898);
\draw [color=c, fill=c] (15.2935,8.65313) rectangle (15.3333,8.75898);
\draw [color=c, fill=c] (15.3333,8.65313) rectangle (15.3731,8.75898);
\draw [color=c, fill=c] (15.3731,8.65313) rectangle (15.4129,8.75898);
\draw [color=c, fill=c] (15.4129,8.65313) rectangle (15.4527,8.75898);
\draw [color=c, fill=c] (15.4527,8.65313) rectangle (15.4925,8.75898);
\draw [color=c, fill=c] (15.4925,8.65313) rectangle (15.5323,8.75898);
\draw [color=c, fill=c] (15.5323,8.65313) rectangle (15.5721,8.75898);
\draw [color=c, fill=c] (15.5721,8.65313) rectangle (15.6119,8.75898);
\draw [color=c, fill=c] (15.6119,8.65313) rectangle (15.6517,8.75898);
\draw [color=c, fill=c] (15.6517,8.65313) rectangle (15.6915,8.75898);
\draw [color=c, fill=c] (15.6915,8.65313) rectangle (15.7313,8.75898);
\draw [color=c, fill=c] (15.7313,8.65313) rectangle (15.7711,8.75898);
\draw [color=c, fill=c] (15.7711,8.65313) rectangle (15.8109,8.75898);
\draw [color=c, fill=c] (15.8109,8.65313) rectangle (15.8507,8.75898);
\draw [color=c, fill=c] (15.8507,8.65313) rectangle (15.8905,8.75898);
\draw [color=c, fill=c] (15.8905,8.65313) rectangle (15.9303,8.75898);
\draw [color=c, fill=c] (15.9303,8.65313) rectangle (15.9701,8.75898);
\draw [color=c, fill=c] (15.9701,8.65313) rectangle (16.01,8.75898);
\draw [color=c, fill=c] (16.01,8.65313) rectangle (16.0498,8.75898);
\draw [color=c, fill=c] (16.0498,8.65313) rectangle (16.0896,8.75898);
\draw [color=c, fill=c] (16.0896,8.65313) rectangle (16.1294,8.75898);
\draw [color=c, fill=c] (16.1294,8.65313) rectangle (16.1692,8.75898);
\draw [color=c, fill=c] (16.1692,8.65313) rectangle (16.209,8.75898);
\draw [color=c, fill=c] (16.209,8.65313) rectangle (16.2488,8.75898);
\draw [color=c, fill=c] (16.2488,8.65313) rectangle (16.2886,8.75898);
\draw [color=c, fill=c] (16.2886,8.65313) rectangle (16.3284,8.75898);
\draw [color=c, fill=c] (16.3284,8.65313) rectangle (16.3682,8.75898);
\draw [color=c, fill=c] (16.3682,8.65313) rectangle (16.408,8.75898);
\draw [color=c, fill=c] (16.408,8.65313) rectangle (16.4478,8.75898);
\draw [color=c, fill=c] (16.4478,8.65313) rectangle (16.4876,8.75898);
\draw [color=c, fill=c] (16.4876,8.65313) rectangle (16.5274,8.75898);
\draw [color=c, fill=c] (16.5274,8.65313) rectangle (16.5672,8.75898);
\draw [color=c, fill=c] (16.5672,8.65313) rectangle (16.607,8.75898);
\draw [color=c, fill=c] (16.607,8.65313) rectangle (16.6468,8.75898);
\draw [color=c, fill=c] (16.6468,8.65313) rectangle (16.6866,8.75898);
\draw [color=c, fill=c] (16.6866,8.65313) rectangle (16.7264,8.75898);
\draw [color=c, fill=c] (16.7264,8.65313) rectangle (16.7662,8.75898);
\draw [color=c, fill=c] (16.7662,8.65313) rectangle (16.806,8.75898);
\draw [color=c, fill=c] (16.806,8.65313) rectangle (16.8458,8.75898);
\draw [color=c, fill=c] (16.8458,8.65313) rectangle (16.8856,8.75898);
\draw [color=c, fill=c] (16.8856,8.65313) rectangle (16.9254,8.75898);
\draw [color=c, fill=c] (16.9254,8.65313) rectangle (16.9652,8.75898);
\draw [color=c, fill=c] (16.9652,8.65313) rectangle (17.005,8.75898);
\draw [color=c, fill=c] (17.005,8.65313) rectangle (17.0448,8.75898);
\draw [color=c, fill=c] (17.0448,8.65313) rectangle (17.0846,8.75898);
\draw [color=c, fill=c] (17.0846,8.65313) rectangle (17.1244,8.75898);
\draw [color=c, fill=c] (17.1244,8.65313) rectangle (17.1642,8.75898);
\draw [color=c, fill=c] (17.1642,8.65313) rectangle (17.204,8.75898);
\draw [color=c, fill=c] (17.204,8.65313) rectangle (17.2438,8.75898);
\draw [color=c, fill=c] (17.2438,8.65313) rectangle (17.2836,8.75898);
\draw [color=c, fill=c] (17.2836,8.65313) rectangle (17.3234,8.75898);
\draw [color=c, fill=c] (17.3234,8.65313) rectangle (17.3632,8.75898);
\draw [color=c, fill=c] (17.3632,8.65313) rectangle (17.403,8.75898);
\draw [color=c, fill=c] (17.403,8.65313) rectangle (17.4428,8.75898);
\draw [color=c, fill=c] (17.4428,8.65313) rectangle (17.4826,8.75898);
\draw [color=c, fill=c] (17.4826,8.65313) rectangle (17.5224,8.75898);
\draw [color=c, fill=c] (17.5224,8.65313) rectangle (17.5622,8.75898);
\draw [color=c, fill=c] (17.5622,8.65313) rectangle (17.602,8.75898);
\draw [color=c, fill=c] (17.602,8.65313) rectangle (17.6418,8.75898);
\draw [color=c, fill=c] (17.6418,8.65313) rectangle (17.6816,8.75898);
\draw [color=c, fill=c] (17.6816,8.65313) rectangle (17.7214,8.75898);
\draw [color=c, fill=c] (17.7214,8.65313) rectangle (17.7612,8.75898);
\draw [color=c, fill=c] (17.7612,8.65313) rectangle (17.801,8.75898);
\draw [color=c, fill=c] (17.801,8.65313) rectangle (17.8408,8.75898);
\draw [color=c, fill=c] (17.8408,8.65313) rectangle (17.8806,8.75898);
\draw [color=c, fill=c] (17.8806,8.65313) rectangle (17.9204,8.75898);
\draw [color=c, fill=c] (17.9204,8.65313) rectangle (17.9602,8.75898);
\draw [color=c, fill=c] (17.9602,8.65313) rectangle (18,8.75898);
\definecolor{c}{rgb}{0.2,0,1};
\draw [color=c, fill=c] (2,8.75898) rectangle (2.0398,8.86482);
\draw [color=c, fill=c] (2.0398,8.75898) rectangle (2.0796,8.86482);
\draw [color=c, fill=c] (2.0796,8.75898) rectangle (2.1194,8.86482);
\draw [color=c, fill=c] (2.1194,8.75898) rectangle (2.1592,8.86482);
\draw [color=c, fill=c] (2.1592,8.75898) rectangle (2.19901,8.86482);
\draw [color=c, fill=c] (2.19901,8.75898) rectangle (2.23881,8.86482);
\draw [color=c, fill=c] (2.23881,8.75898) rectangle (2.27861,8.86482);
\draw [color=c, fill=c] (2.27861,8.75898) rectangle (2.31841,8.86482);
\draw [color=c, fill=c] (2.31841,8.75898) rectangle (2.35821,8.86482);
\draw [color=c, fill=c] (2.35821,8.75898) rectangle (2.39801,8.86482);
\draw [color=c, fill=c] (2.39801,8.75898) rectangle (2.43781,8.86482);
\draw [color=c, fill=c] (2.43781,8.75898) rectangle (2.47761,8.86482);
\draw [color=c, fill=c] (2.47761,8.75898) rectangle (2.51741,8.86482);
\draw [color=c, fill=c] (2.51741,8.75898) rectangle (2.55721,8.86482);
\draw [color=c, fill=c] (2.55721,8.75898) rectangle (2.59702,8.86482);
\draw [color=c, fill=c] (2.59702,8.75898) rectangle (2.63682,8.86482);
\draw [color=c, fill=c] (2.63682,8.75898) rectangle (2.67662,8.86482);
\draw [color=c, fill=c] (2.67662,8.75898) rectangle (2.71642,8.86482);
\draw [color=c, fill=c] (2.71642,8.75898) rectangle (2.75622,8.86482);
\draw [color=c, fill=c] (2.75622,8.75898) rectangle (2.79602,8.86482);
\draw [color=c, fill=c] (2.79602,8.75898) rectangle (2.83582,8.86482);
\draw [color=c, fill=c] (2.83582,8.75898) rectangle (2.87562,8.86482);
\draw [color=c, fill=c] (2.87562,8.75898) rectangle (2.91542,8.86482);
\draw [color=c, fill=c] (2.91542,8.75898) rectangle (2.95522,8.86482);
\draw [color=c, fill=c] (2.95522,8.75898) rectangle (2.99502,8.86482);
\draw [color=c, fill=c] (2.99502,8.75898) rectangle (3.03483,8.86482);
\draw [color=c, fill=c] (3.03483,8.75898) rectangle (3.07463,8.86482);
\draw [color=c, fill=c] (3.07463,8.75898) rectangle (3.11443,8.86482);
\draw [color=c, fill=c] (3.11443,8.75898) rectangle (3.15423,8.86482);
\draw [color=c, fill=c] (3.15423,8.75898) rectangle (3.19403,8.86482);
\draw [color=c, fill=c] (3.19403,8.75898) rectangle (3.23383,8.86482);
\draw [color=c, fill=c] (3.23383,8.75898) rectangle (3.27363,8.86482);
\draw [color=c, fill=c] (3.27363,8.75898) rectangle (3.31343,8.86482);
\draw [color=c, fill=c] (3.31343,8.75898) rectangle (3.35323,8.86482);
\draw [color=c, fill=c] (3.35323,8.75898) rectangle (3.39303,8.86482);
\draw [color=c, fill=c] (3.39303,8.75898) rectangle (3.43284,8.86482);
\draw [color=c, fill=c] (3.43284,8.75898) rectangle (3.47264,8.86482);
\draw [color=c, fill=c] (3.47264,8.75898) rectangle (3.51244,8.86482);
\draw [color=c, fill=c] (3.51244,8.75898) rectangle (3.55224,8.86482);
\draw [color=c, fill=c] (3.55224,8.75898) rectangle (3.59204,8.86482);
\draw [color=c, fill=c] (3.59204,8.75898) rectangle (3.63184,8.86482);
\draw [color=c, fill=c] (3.63184,8.75898) rectangle (3.67164,8.86482);
\draw [color=c, fill=c] (3.67164,8.75898) rectangle (3.71144,8.86482);
\draw [color=c, fill=c] (3.71144,8.75898) rectangle (3.75124,8.86482);
\draw [color=c, fill=c] (3.75124,8.75898) rectangle (3.79104,8.86482);
\draw [color=c, fill=c] (3.79104,8.75898) rectangle (3.83085,8.86482);
\draw [color=c, fill=c] (3.83085,8.75898) rectangle (3.87065,8.86482);
\draw [color=c, fill=c] (3.87065,8.75898) rectangle (3.91045,8.86482);
\draw [color=c, fill=c] (3.91045,8.75898) rectangle (3.95025,8.86482);
\draw [color=c, fill=c] (3.95025,8.75898) rectangle (3.99005,8.86482);
\draw [color=c, fill=c] (3.99005,8.75898) rectangle (4.02985,8.86482);
\draw [color=c, fill=c] (4.02985,8.75898) rectangle (4.06965,8.86482);
\draw [color=c, fill=c] (4.06965,8.75898) rectangle (4.10945,8.86482);
\draw [color=c, fill=c] (4.10945,8.75898) rectangle (4.14925,8.86482);
\draw [color=c, fill=c] (4.14925,8.75898) rectangle (4.18905,8.86482);
\draw [color=c, fill=c] (4.18905,8.75898) rectangle (4.22886,8.86482);
\draw [color=c, fill=c] (4.22886,8.75898) rectangle (4.26866,8.86482);
\draw [color=c, fill=c] (4.26866,8.75898) rectangle (4.30846,8.86482);
\draw [color=c, fill=c] (4.30846,8.75898) rectangle (4.34826,8.86482);
\draw [color=c, fill=c] (4.34826,8.75898) rectangle (4.38806,8.86482);
\draw [color=c, fill=c] (4.38806,8.75898) rectangle (4.42786,8.86482);
\draw [color=c, fill=c] (4.42786,8.75898) rectangle (4.46766,8.86482);
\draw [color=c, fill=c] (4.46766,8.75898) rectangle (4.50746,8.86482);
\draw [color=c, fill=c] (4.50746,8.75898) rectangle (4.54726,8.86482);
\draw [color=c, fill=c] (4.54726,8.75898) rectangle (4.58706,8.86482);
\draw [color=c, fill=c] (4.58706,8.75898) rectangle (4.62687,8.86482);
\draw [color=c, fill=c] (4.62687,8.75898) rectangle (4.66667,8.86482);
\draw [color=c, fill=c] (4.66667,8.75898) rectangle (4.70647,8.86482);
\draw [color=c, fill=c] (4.70647,8.75898) rectangle (4.74627,8.86482);
\draw [color=c, fill=c] (4.74627,8.75898) rectangle (4.78607,8.86482);
\draw [color=c, fill=c] (4.78607,8.75898) rectangle (4.82587,8.86482);
\draw [color=c, fill=c] (4.82587,8.75898) rectangle (4.86567,8.86482);
\draw [color=c, fill=c] (4.86567,8.75898) rectangle (4.90547,8.86482);
\draw [color=c, fill=c] (4.90547,8.75898) rectangle (4.94527,8.86482);
\draw [color=c, fill=c] (4.94527,8.75898) rectangle (4.98507,8.86482);
\draw [color=c, fill=c] (4.98507,8.75898) rectangle (5.02488,8.86482);
\draw [color=c, fill=c] (5.02488,8.75898) rectangle (5.06468,8.86482);
\draw [color=c, fill=c] (5.06468,8.75898) rectangle (5.10448,8.86482);
\draw [color=c, fill=c] (5.10448,8.75898) rectangle (5.14428,8.86482);
\draw [color=c, fill=c] (5.14428,8.75898) rectangle (5.18408,8.86482);
\draw [color=c, fill=c] (5.18408,8.75898) rectangle (5.22388,8.86482);
\draw [color=c, fill=c] (5.22388,8.75898) rectangle (5.26368,8.86482);
\draw [color=c, fill=c] (5.26368,8.75898) rectangle (5.30348,8.86482);
\draw [color=c, fill=c] (5.30348,8.75898) rectangle (5.34328,8.86482);
\draw [color=c, fill=c] (5.34328,8.75898) rectangle (5.38308,8.86482);
\draw [color=c, fill=c] (5.38308,8.75898) rectangle (5.42289,8.86482);
\draw [color=c, fill=c] (5.42289,8.75898) rectangle (5.46269,8.86482);
\draw [color=c, fill=c] (5.46269,8.75898) rectangle (5.50249,8.86482);
\draw [color=c, fill=c] (5.50249,8.75898) rectangle (5.54229,8.86482);
\draw [color=c, fill=c] (5.54229,8.75898) rectangle (5.58209,8.86482);
\draw [color=c, fill=c] (5.58209,8.75898) rectangle (5.62189,8.86482);
\draw [color=c, fill=c] (5.62189,8.75898) rectangle (5.66169,8.86482);
\draw [color=c, fill=c] (5.66169,8.75898) rectangle (5.70149,8.86482);
\draw [color=c, fill=c] (5.70149,8.75898) rectangle (5.74129,8.86482);
\draw [color=c, fill=c] (5.74129,8.75898) rectangle (5.78109,8.86482);
\draw [color=c, fill=c] (5.78109,8.75898) rectangle (5.8209,8.86482);
\draw [color=c, fill=c] (5.8209,8.75898) rectangle (5.8607,8.86482);
\draw [color=c, fill=c] (5.8607,8.75898) rectangle (5.9005,8.86482);
\draw [color=c, fill=c] (5.9005,8.75898) rectangle (5.9403,8.86482);
\draw [color=c, fill=c] (5.9403,8.75898) rectangle (5.9801,8.86482);
\draw [color=c, fill=c] (5.9801,8.75898) rectangle (6.0199,8.86482);
\draw [color=c, fill=c] (6.0199,8.75898) rectangle (6.0597,8.86482);
\draw [color=c, fill=c] (6.0597,8.75898) rectangle (6.0995,8.86482);
\draw [color=c, fill=c] (6.0995,8.75898) rectangle (6.1393,8.86482);
\draw [color=c, fill=c] (6.1393,8.75898) rectangle (6.1791,8.86482);
\draw [color=c, fill=c] (6.1791,8.75898) rectangle (6.21891,8.86482);
\draw [color=c, fill=c] (6.21891,8.75898) rectangle (6.25871,8.86482);
\draw [color=c, fill=c] (6.25871,8.75898) rectangle (6.29851,8.86482);
\draw [color=c, fill=c] (6.29851,8.75898) rectangle (6.33831,8.86482);
\draw [color=c, fill=c] (6.33831,8.75898) rectangle (6.37811,8.86482);
\draw [color=c, fill=c] (6.37811,8.75898) rectangle (6.41791,8.86482);
\draw [color=c, fill=c] (6.41791,8.75898) rectangle (6.45771,8.86482);
\draw [color=c, fill=c] (6.45771,8.75898) rectangle (6.49751,8.86482);
\draw [color=c, fill=c] (6.49751,8.75898) rectangle (6.53731,8.86482);
\draw [color=c, fill=c] (6.53731,8.75898) rectangle (6.57711,8.86482);
\draw [color=c, fill=c] (6.57711,8.75898) rectangle (6.61692,8.86482);
\draw [color=c, fill=c] (6.61692,8.75898) rectangle (6.65672,8.86482);
\draw [color=c, fill=c] (6.65672,8.75898) rectangle (6.69652,8.86482);
\draw [color=c, fill=c] (6.69652,8.75898) rectangle (6.73632,8.86482);
\draw [color=c, fill=c] (6.73632,8.75898) rectangle (6.77612,8.86482);
\draw [color=c, fill=c] (6.77612,8.75898) rectangle (6.81592,8.86482);
\draw [color=c, fill=c] (6.81592,8.75898) rectangle (6.85572,8.86482);
\draw [color=c, fill=c] (6.85572,8.75898) rectangle (6.89552,8.86482);
\draw [color=c, fill=c] (6.89552,8.75898) rectangle (6.93532,8.86482);
\draw [color=c, fill=c] (6.93532,8.75898) rectangle (6.97512,8.86482);
\draw [color=c, fill=c] (6.97512,8.75898) rectangle (7.01493,8.86482);
\draw [color=c, fill=c] (7.01493,8.75898) rectangle (7.05473,8.86482);
\draw [color=c, fill=c] (7.05473,8.75898) rectangle (7.09453,8.86482);
\draw [color=c, fill=c] (7.09453,8.75898) rectangle (7.13433,8.86482);
\draw [color=c, fill=c] (7.13433,8.75898) rectangle (7.17413,8.86482);
\draw [color=c, fill=c] (7.17413,8.75898) rectangle (7.21393,8.86482);
\draw [color=c, fill=c] (7.21393,8.75898) rectangle (7.25373,8.86482);
\draw [color=c, fill=c] (7.25373,8.75898) rectangle (7.29353,8.86482);
\draw [color=c, fill=c] (7.29353,8.75898) rectangle (7.33333,8.86482);
\draw [color=c, fill=c] (7.33333,8.75898) rectangle (7.37313,8.86482);
\draw [color=c, fill=c] (7.37313,8.75898) rectangle (7.41294,8.86482);
\draw [color=c, fill=c] (7.41294,8.75898) rectangle (7.45274,8.86482);
\draw [color=c, fill=c] (7.45274,8.75898) rectangle (7.49254,8.86482);
\draw [color=c, fill=c] (7.49254,8.75898) rectangle (7.53234,8.86482);
\draw [color=c, fill=c] (7.53234,8.75898) rectangle (7.57214,8.86482);
\draw [color=c, fill=c] (7.57214,8.75898) rectangle (7.61194,8.86482);
\definecolor{c}{rgb}{0,0.0800001,1};
\draw [color=c, fill=c] (7.61194,8.75898) rectangle (7.65174,8.86482);
\draw [color=c, fill=c] (7.65174,8.75898) rectangle (7.69154,8.86482);
\draw [color=c, fill=c] (7.69154,8.75898) rectangle (7.73134,8.86482);
\draw [color=c, fill=c] (7.73134,8.75898) rectangle (7.77114,8.86482);
\draw [color=c, fill=c] (7.77114,8.75898) rectangle (7.81095,8.86482);
\draw [color=c, fill=c] (7.81095,8.75898) rectangle (7.85075,8.86482);
\draw [color=c, fill=c] (7.85075,8.75898) rectangle (7.89055,8.86482);
\draw [color=c, fill=c] (7.89055,8.75898) rectangle (7.93035,8.86482);
\draw [color=c, fill=c] (7.93035,8.75898) rectangle (7.97015,8.86482);
\draw [color=c, fill=c] (7.97015,8.75898) rectangle (8.00995,8.86482);
\draw [color=c, fill=c] (8.00995,8.75898) rectangle (8.04975,8.86482);
\draw [color=c, fill=c] (8.04975,8.75898) rectangle (8.08955,8.86482);
\draw [color=c, fill=c] (8.08955,8.75898) rectangle (8.12935,8.86482);
\draw [color=c, fill=c] (8.12935,8.75898) rectangle (8.16915,8.86482);
\draw [color=c, fill=c] (8.16915,8.75898) rectangle (8.20895,8.86482);
\draw [color=c, fill=c] (8.20895,8.75898) rectangle (8.24876,8.86482);
\draw [color=c, fill=c] (8.24876,8.75898) rectangle (8.28856,8.86482);
\draw [color=c, fill=c] (8.28856,8.75898) rectangle (8.32836,8.86482);
\draw [color=c, fill=c] (8.32836,8.75898) rectangle (8.36816,8.86482);
\draw [color=c, fill=c] (8.36816,8.75898) rectangle (8.40796,8.86482);
\draw [color=c, fill=c] (8.40796,8.75898) rectangle (8.44776,8.86482);
\draw [color=c, fill=c] (8.44776,8.75898) rectangle (8.48756,8.86482);
\draw [color=c, fill=c] (8.48756,8.75898) rectangle (8.52736,8.86482);
\draw [color=c, fill=c] (8.52736,8.75898) rectangle (8.56716,8.86482);
\draw [color=c, fill=c] (8.56716,8.75898) rectangle (8.60697,8.86482);
\draw [color=c, fill=c] (8.60697,8.75898) rectangle (8.64677,8.86482);
\draw [color=c, fill=c] (8.64677,8.75898) rectangle (8.68657,8.86482);
\draw [color=c, fill=c] (8.68657,8.75898) rectangle (8.72637,8.86482);
\draw [color=c, fill=c] (8.72637,8.75898) rectangle (8.76617,8.86482);
\draw [color=c, fill=c] (8.76617,8.75898) rectangle (8.80597,8.86482);
\draw [color=c, fill=c] (8.80597,8.75898) rectangle (8.84577,8.86482);
\draw [color=c, fill=c] (8.84577,8.75898) rectangle (8.88557,8.86482);
\draw [color=c, fill=c] (8.88557,8.75898) rectangle (8.92537,8.86482);
\draw [color=c, fill=c] (8.92537,8.75898) rectangle (8.96517,8.86482);
\draw [color=c, fill=c] (8.96517,8.75898) rectangle (9.00498,8.86482);
\draw [color=c, fill=c] (9.00498,8.75898) rectangle (9.04478,8.86482);
\draw [color=c, fill=c] (9.04478,8.75898) rectangle (9.08458,8.86482);
\draw [color=c, fill=c] (9.08458,8.75898) rectangle (9.12438,8.86482);
\draw [color=c, fill=c] (9.12438,8.75898) rectangle (9.16418,8.86482);
\draw [color=c, fill=c] (9.16418,8.75898) rectangle (9.20398,8.86482);
\draw [color=c, fill=c] (9.20398,8.75898) rectangle (9.24378,8.86482);
\draw [color=c, fill=c] (9.24378,8.75898) rectangle (9.28358,8.86482);
\draw [color=c, fill=c] (9.28358,8.75898) rectangle (9.32338,8.86482);
\draw [color=c, fill=c] (9.32338,8.75898) rectangle (9.36318,8.86482);
\draw [color=c, fill=c] (9.36318,8.75898) rectangle (9.40298,8.86482);
\draw [color=c, fill=c] (9.40298,8.75898) rectangle (9.44279,8.86482);
\draw [color=c, fill=c] (9.44279,8.75898) rectangle (9.48259,8.86482);
\definecolor{c}{rgb}{0,0.266667,1};
\draw [color=c, fill=c] (9.48259,8.75898) rectangle (9.52239,8.86482);
\draw [color=c, fill=c] (9.52239,8.75898) rectangle (9.56219,8.86482);
\draw [color=c, fill=c] (9.56219,8.75898) rectangle (9.60199,8.86482);
\draw [color=c, fill=c] (9.60199,8.75898) rectangle (9.64179,8.86482);
\draw [color=c, fill=c] (9.64179,8.75898) rectangle (9.68159,8.86482);
\draw [color=c, fill=c] (9.68159,8.75898) rectangle (9.72139,8.86482);
\draw [color=c, fill=c] (9.72139,8.75898) rectangle (9.76119,8.86482);
\draw [color=c, fill=c] (9.76119,8.75898) rectangle (9.80099,8.86482);
\draw [color=c, fill=c] (9.80099,8.75898) rectangle (9.8408,8.86482);
\draw [color=c, fill=c] (9.8408,8.75898) rectangle (9.8806,8.86482);
\draw [color=c, fill=c] (9.8806,8.75898) rectangle (9.9204,8.86482);
\draw [color=c, fill=c] (9.9204,8.75898) rectangle (9.9602,8.86482);
\draw [color=c, fill=c] (9.9602,8.75898) rectangle (10,8.86482);
\draw [color=c, fill=c] (10,8.75898) rectangle (10.0398,8.86482);
\draw [color=c, fill=c] (10.0398,8.75898) rectangle (10.0796,8.86482);
\draw [color=c, fill=c] (10.0796,8.75898) rectangle (10.1194,8.86482);
\draw [color=c, fill=c] (10.1194,8.75898) rectangle (10.1592,8.86482);
\draw [color=c, fill=c] (10.1592,8.75898) rectangle (10.199,8.86482);
\draw [color=c, fill=c] (10.199,8.75898) rectangle (10.2388,8.86482);
\draw [color=c, fill=c] (10.2388,8.75898) rectangle (10.2786,8.86482);
\draw [color=c, fill=c] (10.2786,8.75898) rectangle (10.3184,8.86482);
\draw [color=c, fill=c] (10.3184,8.75898) rectangle (10.3582,8.86482);
\draw [color=c, fill=c] (10.3582,8.75898) rectangle (10.398,8.86482);
\draw [color=c, fill=c] (10.398,8.75898) rectangle (10.4378,8.86482);
\draw [color=c, fill=c] (10.4378,8.75898) rectangle (10.4776,8.86482);
\draw [color=c, fill=c] (10.4776,8.75898) rectangle (10.5174,8.86482);
\draw [color=c, fill=c] (10.5174,8.75898) rectangle (10.5572,8.86482);
\definecolor{c}{rgb}{0,0.546666,1};
\draw [color=c, fill=c] (10.5572,8.75898) rectangle (10.597,8.86482);
\draw [color=c, fill=c] (10.597,8.75898) rectangle (10.6368,8.86482);
\draw [color=c, fill=c] (10.6368,8.75898) rectangle (10.6766,8.86482);
\draw [color=c, fill=c] (10.6766,8.75898) rectangle (10.7164,8.86482);
\draw [color=c, fill=c] (10.7164,8.75898) rectangle (10.7562,8.86482);
\draw [color=c, fill=c] (10.7562,8.75898) rectangle (10.796,8.86482);
\draw [color=c, fill=c] (10.796,8.75898) rectangle (10.8358,8.86482);
\draw [color=c, fill=c] (10.8358,8.75898) rectangle (10.8756,8.86482);
\draw [color=c, fill=c] (10.8756,8.75898) rectangle (10.9154,8.86482);
\draw [color=c, fill=c] (10.9154,8.75898) rectangle (10.9552,8.86482);
\draw [color=c, fill=c] (10.9552,8.75898) rectangle (10.995,8.86482);
\draw [color=c, fill=c] (10.995,8.75898) rectangle (11.0348,8.86482);
\draw [color=c, fill=c] (11.0348,8.75898) rectangle (11.0746,8.86482);
\draw [color=c, fill=c] (11.0746,8.75898) rectangle (11.1144,8.86482);
\draw [color=c, fill=c] (11.1144,8.75898) rectangle (11.1542,8.86482);
\draw [color=c, fill=c] (11.1542,8.75898) rectangle (11.194,8.86482);
\draw [color=c, fill=c] (11.194,8.75898) rectangle (11.2338,8.86482);
\draw [color=c, fill=c] (11.2338,8.75898) rectangle (11.2736,8.86482);
\draw [color=c, fill=c] (11.2736,8.75898) rectangle (11.3134,8.86482);
\draw [color=c, fill=c] (11.3134,8.75898) rectangle (11.3532,8.86482);
\draw [color=c, fill=c] (11.3532,8.75898) rectangle (11.393,8.86482);
\draw [color=c, fill=c] (11.393,8.75898) rectangle (11.4328,8.86482);
\draw [color=c, fill=c] (11.4328,8.75898) rectangle (11.4726,8.86482);
\draw [color=c, fill=c] (11.4726,8.75898) rectangle (11.5124,8.86482);
\draw [color=c, fill=c] (11.5124,8.75898) rectangle (11.5522,8.86482);
\draw [color=c, fill=c] (11.5522,8.75898) rectangle (11.592,8.86482);
\draw [color=c, fill=c] (11.592,8.75898) rectangle (11.6318,8.86482);
\draw [color=c, fill=c] (11.6318,8.75898) rectangle (11.6716,8.86482);
\draw [color=c, fill=c] (11.6716,8.75898) rectangle (11.7114,8.86482);
\draw [color=c, fill=c] (11.7114,8.75898) rectangle (11.7512,8.86482);
\draw [color=c, fill=c] (11.7512,8.75898) rectangle (11.791,8.86482);
\draw [color=c, fill=c] (11.791,8.75898) rectangle (11.8308,8.86482);
\draw [color=c, fill=c] (11.8308,8.75898) rectangle (11.8706,8.86482);
\draw [color=c, fill=c] (11.8706,8.75898) rectangle (11.9104,8.86482);
\draw [color=c, fill=c] (11.9104,8.75898) rectangle (11.9502,8.86482);
\draw [color=c, fill=c] (11.9502,8.75898) rectangle (11.99,8.86482);
\draw [color=c, fill=c] (11.99,8.75898) rectangle (12.0299,8.86482);
\draw [color=c, fill=c] (12.0299,8.75898) rectangle (12.0697,8.86482);
\draw [color=c, fill=c] (12.0697,8.75898) rectangle (12.1095,8.86482);
\draw [color=c, fill=c] (12.1095,8.75898) rectangle (12.1493,8.86482);
\draw [color=c, fill=c] (12.1493,8.75898) rectangle (12.1891,8.86482);
\draw [color=c, fill=c] (12.1891,8.75898) rectangle (12.2289,8.86482);
\draw [color=c, fill=c] (12.2289,8.75898) rectangle (12.2687,8.86482);
\draw [color=c, fill=c] (12.2687,8.75898) rectangle (12.3085,8.86482);
\draw [color=c, fill=c] (12.3085,8.75898) rectangle (12.3483,8.86482);
\draw [color=c, fill=c] (12.3483,8.75898) rectangle (12.3881,8.86482);
\draw [color=c, fill=c] (12.3881,8.75898) rectangle (12.4279,8.86482);
\draw [color=c, fill=c] (12.4279,8.75898) rectangle (12.4677,8.86482);
\draw [color=c, fill=c] (12.4677,8.75898) rectangle (12.5075,8.86482);
\draw [color=c, fill=c] (12.5075,8.75898) rectangle (12.5473,8.86482);
\draw [color=c, fill=c] (12.5473,8.75898) rectangle (12.5871,8.86482);
\draw [color=c, fill=c] (12.5871,8.75898) rectangle (12.6269,8.86482);
\draw [color=c, fill=c] (12.6269,8.75898) rectangle (12.6667,8.86482);
\draw [color=c, fill=c] (12.6667,8.75898) rectangle (12.7065,8.86482);
\draw [color=c, fill=c] (12.7065,8.75898) rectangle (12.7463,8.86482);
\draw [color=c, fill=c] (12.7463,8.75898) rectangle (12.7861,8.86482);
\draw [color=c, fill=c] (12.7861,8.75898) rectangle (12.8259,8.86482);
\draw [color=c, fill=c] (12.8259,8.75898) rectangle (12.8657,8.86482);
\draw [color=c, fill=c] (12.8657,8.75898) rectangle (12.9055,8.86482);
\draw [color=c, fill=c] (12.9055,8.75898) rectangle (12.9453,8.86482);
\draw [color=c, fill=c] (12.9453,8.75898) rectangle (12.9851,8.86482);
\definecolor{c}{rgb}{0,0.733333,1};
\draw [color=c, fill=c] (12.9851,8.75898) rectangle (13.0249,8.86482);
\draw [color=c, fill=c] (13.0249,8.75898) rectangle (13.0647,8.86482);
\draw [color=c, fill=c] (13.0647,8.75898) rectangle (13.1045,8.86482);
\draw [color=c, fill=c] (13.1045,8.75898) rectangle (13.1443,8.86482);
\draw [color=c, fill=c] (13.1443,8.75898) rectangle (13.1841,8.86482);
\draw [color=c, fill=c] (13.1841,8.75898) rectangle (13.2239,8.86482);
\draw [color=c, fill=c] (13.2239,8.75898) rectangle (13.2637,8.86482);
\draw [color=c, fill=c] (13.2637,8.75898) rectangle (13.3035,8.86482);
\draw [color=c, fill=c] (13.3035,8.75898) rectangle (13.3433,8.86482);
\draw [color=c, fill=c] (13.3433,8.75898) rectangle (13.3831,8.86482);
\draw [color=c, fill=c] (13.3831,8.75898) rectangle (13.4229,8.86482);
\draw [color=c, fill=c] (13.4229,8.75898) rectangle (13.4627,8.86482);
\draw [color=c, fill=c] (13.4627,8.75898) rectangle (13.5025,8.86482);
\draw [color=c, fill=c] (13.5025,8.75898) rectangle (13.5423,8.86482);
\draw [color=c, fill=c] (13.5423,8.75898) rectangle (13.5821,8.86482);
\draw [color=c, fill=c] (13.5821,8.75898) rectangle (13.6219,8.86482);
\draw [color=c, fill=c] (13.6219,8.75898) rectangle (13.6617,8.86482);
\draw [color=c, fill=c] (13.6617,8.75898) rectangle (13.7015,8.86482);
\draw [color=c, fill=c] (13.7015,8.75898) rectangle (13.7413,8.86482);
\draw [color=c, fill=c] (13.7413,8.75898) rectangle (13.7811,8.86482);
\draw [color=c, fill=c] (13.7811,8.75898) rectangle (13.8209,8.86482);
\draw [color=c, fill=c] (13.8209,8.75898) rectangle (13.8607,8.86482);
\draw [color=c, fill=c] (13.8607,8.75898) rectangle (13.9005,8.86482);
\draw [color=c, fill=c] (13.9005,8.75898) rectangle (13.9403,8.86482);
\draw [color=c, fill=c] (13.9403,8.75898) rectangle (13.9801,8.86482);
\draw [color=c, fill=c] (13.9801,8.75898) rectangle (14.0199,8.86482);
\draw [color=c, fill=c] (14.0199,8.75898) rectangle (14.0597,8.86482);
\draw [color=c, fill=c] (14.0597,8.75898) rectangle (14.0995,8.86482);
\draw [color=c, fill=c] (14.0995,8.75898) rectangle (14.1393,8.86482);
\draw [color=c, fill=c] (14.1393,8.75898) rectangle (14.1791,8.86482);
\draw [color=c, fill=c] (14.1791,8.75898) rectangle (14.2189,8.86482);
\draw [color=c, fill=c] (14.2189,8.75898) rectangle (14.2587,8.86482);
\draw [color=c, fill=c] (14.2587,8.75898) rectangle (14.2985,8.86482);
\draw [color=c, fill=c] (14.2985,8.75898) rectangle (14.3383,8.86482);
\draw [color=c, fill=c] (14.3383,8.75898) rectangle (14.3781,8.86482);
\draw [color=c, fill=c] (14.3781,8.75898) rectangle (14.4179,8.86482);
\draw [color=c, fill=c] (14.4179,8.75898) rectangle (14.4577,8.86482);
\draw [color=c, fill=c] (14.4577,8.75898) rectangle (14.4975,8.86482);
\draw [color=c, fill=c] (14.4975,8.75898) rectangle (14.5373,8.86482);
\draw [color=c, fill=c] (14.5373,8.75898) rectangle (14.5771,8.86482);
\draw [color=c, fill=c] (14.5771,8.75898) rectangle (14.6169,8.86482);
\draw [color=c, fill=c] (14.6169,8.75898) rectangle (14.6567,8.86482);
\draw [color=c, fill=c] (14.6567,8.75898) rectangle (14.6965,8.86482);
\draw [color=c, fill=c] (14.6965,8.75898) rectangle (14.7363,8.86482);
\draw [color=c, fill=c] (14.7363,8.75898) rectangle (14.7761,8.86482);
\draw [color=c, fill=c] (14.7761,8.75898) rectangle (14.8159,8.86482);
\draw [color=c, fill=c] (14.8159,8.75898) rectangle (14.8557,8.86482);
\draw [color=c, fill=c] (14.8557,8.75898) rectangle (14.8955,8.86482);
\draw [color=c, fill=c] (14.8955,8.75898) rectangle (14.9353,8.86482);
\draw [color=c, fill=c] (14.9353,8.75898) rectangle (14.9751,8.86482);
\draw [color=c, fill=c] (14.9751,8.75898) rectangle (15.0149,8.86482);
\draw [color=c, fill=c] (15.0149,8.75898) rectangle (15.0547,8.86482);
\draw [color=c, fill=c] (15.0547,8.75898) rectangle (15.0945,8.86482);
\draw [color=c, fill=c] (15.0945,8.75898) rectangle (15.1343,8.86482);
\draw [color=c, fill=c] (15.1343,8.75898) rectangle (15.1741,8.86482);
\draw [color=c, fill=c] (15.1741,8.75898) rectangle (15.2139,8.86482);
\draw [color=c, fill=c] (15.2139,8.75898) rectangle (15.2537,8.86482);
\draw [color=c, fill=c] (15.2537,8.75898) rectangle (15.2935,8.86482);
\draw [color=c, fill=c] (15.2935,8.75898) rectangle (15.3333,8.86482);
\draw [color=c, fill=c] (15.3333,8.75898) rectangle (15.3731,8.86482);
\draw [color=c, fill=c] (15.3731,8.75898) rectangle (15.4129,8.86482);
\draw [color=c, fill=c] (15.4129,8.75898) rectangle (15.4527,8.86482);
\draw [color=c, fill=c] (15.4527,8.75898) rectangle (15.4925,8.86482);
\draw [color=c, fill=c] (15.4925,8.75898) rectangle (15.5323,8.86482);
\draw [color=c, fill=c] (15.5323,8.75898) rectangle (15.5721,8.86482);
\draw [color=c, fill=c] (15.5721,8.75898) rectangle (15.6119,8.86482);
\draw [color=c, fill=c] (15.6119,8.75898) rectangle (15.6517,8.86482);
\draw [color=c, fill=c] (15.6517,8.75898) rectangle (15.6915,8.86482);
\draw [color=c, fill=c] (15.6915,8.75898) rectangle (15.7313,8.86482);
\draw [color=c, fill=c] (15.7313,8.75898) rectangle (15.7711,8.86482);
\draw [color=c, fill=c] (15.7711,8.75898) rectangle (15.8109,8.86482);
\draw [color=c, fill=c] (15.8109,8.75898) rectangle (15.8507,8.86482);
\draw [color=c, fill=c] (15.8507,8.75898) rectangle (15.8905,8.86482);
\draw [color=c, fill=c] (15.8905,8.75898) rectangle (15.9303,8.86482);
\draw [color=c, fill=c] (15.9303,8.75898) rectangle (15.9701,8.86482);
\draw [color=c, fill=c] (15.9701,8.75898) rectangle (16.01,8.86482);
\draw [color=c, fill=c] (16.01,8.75898) rectangle (16.0498,8.86482);
\draw [color=c, fill=c] (16.0498,8.75898) rectangle (16.0896,8.86482);
\draw [color=c, fill=c] (16.0896,8.75898) rectangle (16.1294,8.86482);
\draw [color=c, fill=c] (16.1294,8.75898) rectangle (16.1692,8.86482);
\draw [color=c, fill=c] (16.1692,8.75898) rectangle (16.209,8.86482);
\draw [color=c, fill=c] (16.209,8.75898) rectangle (16.2488,8.86482);
\draw [color=c, fill=c] (16.2488,8.75898) rectangle (16.2886,8.86482);
\draw [color=c, fill=c] (16.2886,8.75898) rectangle (16.3284,8.86482);
\draw [color=c, fill=c] (16.3284,8.75898) rectangle (16.3682,8.86482);
\draw [color=c, fill=c] (16.3682,8.75898) rectangle (16.408,8.86482);
\draw [color=c, fill=c] (16.408,8.75898) rectangle (16.4478,8.86482);
\draw [color=c, fill=c] (16.4478,8.75898) rectangle (16.4876,8.86482);
\draw [color=c, fill=c] (16.4876,8.75898) rectangle (16.5274,8.86482);
\draw [color=c, fill=c] (16.5274,8.75898) rectangle (16.5672,8.86482);
\draw [color=c, fill=c] (16.5672,8.75898) rectangle (16.607,8.86482);
\draw [color=c, fill=c] (16.607,8.75898) rectangle (16.6468,8.86482);
\draw [color=c, fill=c] (16.6468,8.75898) rectangle (16.6866,8.86482);
\draw [color=c, fill=c] (16.6866,8.75898) rectangle (16.7264,8.86482);
\draw [color=c, fill=c] (16.7264,8.75898) rectangle (16.7662,8.86482);
\draw [color=c, fill=c] (16.7662,8.75898) rectangle (16.806,8.86482);
\draw [color=c, fill=c] (16.806,8.75898) rectangle (16.8458,8.86482);
\draw [color=c, fill=c] (16.8458,8.75898) rectangle (16.8856,8.86482);
\draw [color=c, fill=c] (16.8856,8.75898) rectangle (16.9254,8.86482);
\draw [color=c, fill=c] (16.9254,8.75898) rectangle (16.9652,8.86482);
\draw [color=c, fill=c] (16.9652,8.75898) rectangle (17.005,8.86482);
\draw [color=c, fill=c] (17.005,8.75898) rectangle (17.0448,8.86482);
\draw [color=c, fill=c] (17.0448,8.75898) rectangle (17.0846,8.86482);
\draw [color=c, fill=c] (17.0846,8.75898) rectangle (17.1244,8.86482);
\draw [color=c, fill=c] (17.1244,8.75898) rectangle (17.1642,8.86482);
\draw [color=c, fill=c] (17.1642,8.75898) rectangle (17.204,8.86482);
\draw [color=c, fill=c] (17.204,8.75898) rectangle (17.2438,8.86482);
\draw [color=c, fill=c] (17.2438,8.75898) rectangle (17.2836,8.86482);
\draw [color=c, fill=c] (17.2836,8.75898) rectangle (17.3234,8.86482);
\draw [color=c, fill=c] (17.3234,8.75898) rectangle (17.3632,8.86482);
\draw [color=c, fill=c] (17.3632,8.75898) rectangle (17.403,8.86482);
\draw [color=c, fill=c] (17.403,8.75898) rectangle (17.4428,8.86482);
\draw [color=c, fill=c] (17.4428,8.75898) rectangle (17.4826,8.86482);
\draw [color=c, fill=c] (17.4826,8.75898) rectangle (17.5224,8.86482);
\draw [color=c, fill=c] (17.5224,8.75898) rectangle (17.5622,8.86482);
\draw [color=c, fill=c] (17.5622,8.75898) rectangle (17.602,8.86482);
\draw [color=c, fill=c] (17.602,8.75898) rectangle (17.6418,8.86482);
\draw [color=c, fill=c] (17.6418,8.75898) rectangle (17.6816,8.86482);
\draw [color=c, fill=c] (17.6816,8.75898) rectangle (17.7214,8.86482);
\draw [color=c, fill=c] (17.7214,8.75898) rectangle (17.7612,8.86482);
\draw [color=c, fill=c] (17.7612,8.75898) rectangle (17.801,8.86482);
\draw [color=c, fill=c] (17.801,8.75898) rectangle (17.8408,8.86482);
\draw [color=c, fill=c] (17.8408,8.75898) rectangle (17.8806,8.86482);
\draw [color=c, fill=c] (17.8806,8.75898) rectangle (17.9204,8.86482);
\draw [color=c, fill=c] (17.9204,8.75898) rectangle (17.9602,8.86482);
\draw [color=c, fill=c] (17.9602,8.75898) rectangle (18,8.86482);
\definecolor{c}{rgb}{0.2,0,1};
\draw [color=c, fill=c] (2,8.86482) rectangle (2.0398,8.97067);
\draw [color=c, fill=c] (2.0398,8.86482) rectangle (2.0796,8.97067);
\draw [color=c, fill=c] (2.0796,8.86482) rectangle (2.1194,8.97067);
\draw [color=c, fill=c] (2.1194,8.86482) rectangle (2.1592,8.97067);
\draw [color=c, fill=c] (2.1592,8.86482) rectangle (2.19901,8.97067);
\draw [color=c, fill=c] (2.19901,8.86482) rectangle (2.23881,8.97067);
\draw [color=c, fill=c] (2.23881,8.86482) rectangle (2.27861,8.97067);
\draw [color=c, fill=c] (2.27861,8.86482) rectangle (2.31841,8.97067);
\draw [color=c, fill=c] (2.31841,8.86482) rectangle (2.35821,8.97067);
\draw [color=c, fill=c] (2.35821,8.86482) rectangle (2.39801,8.97067);
\draw [color=c, fill=c] (2.39801,8.86482) rectangle (2.43781,8.97067);
\draw [color=c, fill=c] (2.43781,8.86482) rectangle (2.47761,8.97067);
\draw [color=c, fill=c] (2.47761,8.86482) rectangle (2.51741,8.97067);
\draw [color=c, fill=c] (2.51741,8.86482) rectangle (2.55721,8.97067);
\draw [color=c, fill=c] (2.55721,8.86482) rectangle (2.59702,8.97067);
\draw [color=c, fill=c] (2.59702,8.86482) rectangle (2.63682,8.97067);
\draw [color=c, fill=c] (2.63682,8.86482) rectangle (2.67662,8.97067);
\draw [color=c, fill=c] (2.67662,8.86482) rectangle (2.71642,8.97067);
\draw [color=c, fill=c] (2.71642,8.86482) rectangle (2.75622,8.97067);
\draw [color=c, fill=c] (2.75622,8.86482) rectangle (2.79602,8.97067);
\draw [color=c, fill=c] (2.79602,8.86482) rectangle (2.83582,8.97067);
\draw [color=c, fill=c] (2.83582,8.86482) rectangle (2.87562,8.97067);
\draw [color=c, fill=c] (2.87562,8.86482) rectangle (2.91542,8.97067);
\draw [color=c, fill=c] (2.91542,8.86482) rectangle (2.95522,8.97067);
\draw [color=c, fill=c] (2.95522,8.86482) rectangle (2.99502,8.97067);
\draw [color=c, fill=c] (2.99502,8.86482) rectangle (3.03483,8.97067);
\draw [color=c, fill=c] (3.03483,8.86482) rectangle (3.07463,8.97067);
\draw [color=c, fill=c] (3.07463,8.86482) rectangle (3.11443,8.97067);
\draw [color=c, fill=c] (3.11443,8.86482) rectangle (3.15423,8.97067);
\draw [color=c, fill=c] (3.15423,8.86482) rectangle (3.19403,8.97067);
\draw [color=c, fill=c] (3.19403,8.86482) rectangle (3.23383,8.97067);
\draw [color=c, fill=c] (3.23383,8.86482) rectangle (3.27363,8.97067);
\draw [color=c, fill=c] (3.27363,8.86482) rectangle (3.31343,8.97067);
\draw [color=c, fill=c] (3.31343,8.86482) rectangle (3.35323,8.97067);
\draw [color=c, fill=c] (3.35323,8.86482) rectangle (3.39303,8.97067);
\draw [color=c, fill=c] (3.39303,8.86482) rectangle (3.43284,8.97067);
\draw [color=c, fill=c] (3.43284,8.86482) rectangle (3.47264,8.97067);
\draw [color=c, fill=c] (3.47264,8.86482) rectangle (3.51244,8.97067);
\draw [color=c, fill=c] (3.51244,8.86482) rectangle (3.55224,8.97067);
\draw [color=c, fill=c] (3.55224,8.86482) rectangle (3.59204,8.97067);
\draw [color=c, fill=c] (3.59204,8.86482) rectangle (3.63184,8.97067);
\draw [color=c, fill=c] (3.63184,8.86482) rectangle (3.67164,8.97067);
\draw [color=c, fill=c] (3.67164,8.86482) rectangle (3.71144,8.97067);
\draw [color=c, fill=c] (3.71144,8.86482) rectangle (3.75124,8.97067);
\draw [color=c, fill=c] (3.75124,8.86482) rectangle (3.79104,8.97067);
\draw [color=c, fill=c] (3.79104,8.86482) rectangle (3.83085,8.97067);
\draw [color=c, fill=c] (3.83085,8.86482) rectangle (3.87065,8.97067);
\draw [color=c, fill=c] (3.87065,8.86482) rectangle (3.91045,8.97067);
\draw [color=c, fill=c] (3.91045,8.86482) rectangle (3.95025,8.97067);
\draw [color=c, fill=c] (3.95025,8.86482) rectangle (3.99005,8.97067);
\draw [color=c, fill=c] (3.99005,8.86482) rectangle (4.02985,8.97067);
\draw [color=c, fill=c] (4.02985,8.86482) rectangle (4.06965,8.97067);
\draw [color=c, fill=c] (4.06965,8.86482) rectangle (4.10945,8.97067);
\draw [color=c, fill=c] (4.10945,8.86482) rectangle (4.14925,8.97067);
\draw [color=c, fill=c] (4.14925,8.86482) rectangle (4.18905,8.97067);
\draw [color=c, fill=c] (4.18905,8.86482) rectangle (4.22886,8.97067);
\draw [color=c, fill=c] (4.22886,8.86482) rectangle (4.26866,8.97067);
\draw [color=c, fill=c] (4.26866,8.86482) rectangle (4.30846,8.97067);
\draw [color=c, fill=c] (4.30846,8.86482) rectangle (4.34826,8.97067);
\draw [color=c, fill=c] (4.34826,8.86482) rectangle (4.38806,8.97067);
\draw [color=c, fill=c] (4.38806,8.86482) rectangle (4.42786,8.97067);
\draw [color=c, fill=c] (4.42786,8.86482) rectangle (4.46766,8.97067);
\draw [color=c, fill=c] (4.46766,8.86482) rectangle (4.50746,8.97067);
\draw [color=c, fill=c] (4.50746,8.86482) rectangle (4.54726,8.97067);
\draw [color=c, fill=c] (4.54726,8.86482) rectangle (4.58706,8.97067);
\draw [color=c, fill=c] (4.58706,8.86482) rectangle (4.62687,8.97067);
\draw [color=c, fill=c] (4.62687,8.86482) rectangle (4.66667,8.97067);
\draw [color=c, fill=c] (4.66667,8.86482) rectangle (4.70647,8.97067);
\draw [color=c, fill=c] (4.70647,8.86482) rectangle (4.74627,8.97067);
\draw [color=c, fill=c] (4.74627,8.86482) rectangle (4.78607,8.97067);
\draw [color=c, fill=c] (4.78607,8.86482) rectangle (4.82587,8.97067);
\draw [color=c, fill=c] (4.82587,8.86482) rectangle (4.86567,8.97067);
\draw [color=c, fill=c] (4.86567,8.86482) rectangle (4.90547,8.97067);
\draw [color=c, fill=c] (4.90547,8.86482) rectangle (4.94527,8.97067);
\draw [color=c, fill=c] (4.94527,8.86482) rectangle (4.98507,8.97067);
\draw [color=c, fill=c] (4.98507,8.86482) rectangle (5.02488,8.97067);
\draw [color=c, fill=c] (5.02488,8.86482) rectangle (5.06468,8.97067);
\draw [color=c, fill=c] (5.06468,8.86482) rectangle (5.10448,8.97067);
\draw [color=c, fill=c] (5.10448,8.86482) rectangle (5.14428,8.97067);
\draw [color=c, fill=c] (5.14428,8.86482) rectangle (5.18408,8.97067);
\draw [color=c, fill=c] (5.18408,8.86482) rectangle (5.22388,8.97067);
\draw [color=c, fill=c] (5.22388,8.86482) rectangle (5.26368,8.97067);
\draw [color=c, fill=c] (5.26368,8.86482) rectangle (5.30348,8.97067);
\draw [color=c, fill=c] (5.30348,8.86482) rectangle (5.34328,8.97067);
\draw [color=c, fill=c] (5.34328,8.86482) rectangle (5.38308,8.97067);
\draw [color=c, fill=c] (5.38308,8.86482) rectangle (5.42289,8.97067);
\draw [color=c, fill=c] (5.42289,8.86482) rectangle (5.46269,8.97067);
\draw [color=c, fill=c] (5.46269,8.86482) rectangle (5.50249,8.97067);
\draw [color=c, fill=c] (5.50249,8.86482) rectangle (5.54229,8.97067);
\draw [color=c, fill=c] (5.54229,8.86482) rectangle (5.58209,8.97067);
\draw [color=c, fill=c] (5.58209,8.86482) rectangle (5.62189,8.97067);
\draw [color=c, fill=c] (5.62189,8.86482) rectangle (5.66169,8.97067);
\draw [color=c, fill=c] (5.66169,8.86482) rectangle (5.70149,8.97067);
\draw [color=c, fill=c] (5.70149,8.86482) rectangle (5.74129,8.97067);
\draw [color=c, fill=c] (5.74129,8.86482) rectangle (5.78109,8.97067);
\draw [color=c, fill=c] (5.78109,8.86482) rectangle (5.8209,8.97067);
\draw [color=c, fill=c] (5.8209,8.86482) rectangle (5.8607,8.97067);
\draw [color=c, fill=c] (5.8607,8.86482) rectangle (5.9005,8.97067);
\draw [color=c, fill=c] (5.9005,8.86482) rectangle (5.9403,8.97067);
\draw [color=c, fill=c] (5.9403,8.86482) rectangle (5.9801,8.97067);
\draw [color=c, fill=c] (5.9801,8.86482) rectangle (6.0199,8.97067);
\draw [color=c, fill=c] (6.0199,8.86482) rectangle (6.0597,8.97067);
\draw [color=c, fill=c] (6.0597,8.86482) rectangle (6.0995,8.97067);
\draw [color=c, fill=c] (6.0995,8.86482) rectangle (6.1393,8.97067);
\draw [color=c, fill=c] (6.1393,8.86482) rectangle (6.1791,8.97067);
\draw [color=c, fill=c] (6.1791,8.86482) rectangle (6.21891,8.97067);
\draw [color=c, fill=c] (6.21891,8.86482) rectangle (6.25871,8.97067);
\draw [color=c, fill=c] (6.25871,8.86482) rectangle (6.29851,8.97067);
\draw [color=c, fill=c] (6.29851,8.86482) rectangle (6.33831,8.97067);
\draw [color=c, fill=c] (6.33831,8.86482) rectangle (6.37811,8.97067);
\draw [color=c, fill=c] (6.37811,8.86482) rectangle (6.41791,8.97067);
\draw [color=c, fill=c] (6.41791,8.86482) rectangle (6.45771,8.97067);
\draw [color=c, fill=c] (6.45771,8.86482) rectangle (6.49751,8.97067);
\draw [color=c, fill=c] (6.49751,8.86482) rectangle (6.53731,8.97067);
\draw [color=c, fill=c] (6.53731,8.86482) rectangle (6.57711,8.97067);
\draw [color=c, fill=c] (6.57711,8.86482) rectangle (6.61692,8.97067);
\draw [color=c, fill=c] (6.61692,8.86482) rectangle (6.65672,8.97067);
\draw [color=c, fill=c] (6.65672,8.86482) rectangle (6.69652,8.97067);
\draw [color=c, fill=c] (6.69652,8.86482) rectangle (6.73632,8.97067);
\draw [color=c, fill=c] (6.73632,8.86482) rectangle (6.77612,8.97067);
\draw [color=c, fill=c] (6.77612,8.86482) rectangle (6.81592,8.97067);
\draw [color=c, fill=c] (6.81592,8.86482) rectangle (6.85572,8.97067);
\draw [color=c, fill=c] (6.85572,8.86482) rectangle (6.89552,8.97067);
\draw [color=c, fill=c] (6.89552,8.86482) rectangle (6.93532,8.97067);
\draw [color=c, fill=c] (6.93532,8.86482) rectangle (6.97512,8.97067);
\draw [color=c, fill=c] (6.97512,8.86482) rectangle (7.01493,8.97067);
\draw [color=c, fill=c] (7.01493,8.86482) rectangle (7.05473,8.97067);
\draw [color=c, fill=c] (7.05473,8.86482) rectangle (7.09453,8.97067);
\draw [color=c, fill=c] (7.09453,8.86482) rectangle (7.13433,8.97067);
\draw [color=c, fill=c] (7.13433,8.86482) rectangle (7.17413,8.97067);
\draw [color=c, fill=c] (7.17413,8.86482) rectangle (7.21393,8.97067);
\draw [color=c, fill=c] (7.21393,8.86482) rectangle (7.25373,8.97067);
\draw [color=c, fill=c] (7.25373,8.86482) rectangle (7.29353,8.97067);
\draw [color=c, fill=c] (7.29353,8.86482) rectangle (7.33333,8.97067);
\draw [color=c, fill=c] (7.33333,8.86482) rectangle (7.37313,8.97067);
\draw [color=c, fill=c] (7.37313,8.86482) rectangle (7.41294,8.97067);
\draw [color=c, fill=c] (7.41294,8.86482) rectangle (7.45274,8.97067);
\draw [color=c, fill=c] (7.45274,8.86482) rectangle (7.49254,8.97067);
\draw [color=c, fill=c] (7.49254,8.86482) rectangle (7.53234,8.97067);
\draw [color=c, fill=c] (7.53234,8.86482) rectangle (7.57214,8.97067);
\draw [color=c, fill=c] (7.57214,8.86482) rectangle (7.61194,8.97067);
\definecolor{c}{rgb}{0,0.0800001,1};
\draw [color=c, fill=c] (7.61194,8.86482) rectangle (7.65174,8.97067);
\draw [color=c, fill=c] (7.65174,8.86482) rectangle (7.69154,8.97067);
\draw [color=c, fill=c] (7.69154,8.86482) rectangle (7.73134,8.97067);
\draw [color=c, fill=c] (7.73134,8.86482) rectangle (7.77114,8.97067);
\draw [color=c, fill=c] (7.77114,8.86482) rectangle (7.81095,8.97067);
\draw [color=c, fill=c] (7.81095,8.86482) rectangle (7.85075,8.97067);
\draw [color=c, fill=c] (7.85075,8.86482) rectangle (7.89055,8.97067);
\draw [color=c, fill=c] (7.89055,8.86482) rectangle (7.93035,8.97067);
\draw [color=c, fill=c] (7.93035,8.86482) rectangle (7.97015,8.97067);
\draw [color=c, fill=c] (7.97015,8.86482) rectangle (8.00995,8.97067);
\draw [color=c, fill=c] (8.00995,8.86482) rectangle (8.04975,8.97067);
\draw [color=c, fill=c] (8.04975,8.86482) rectangle (8.08955,8.97067);
\draw [color=c, fill=c] (8.08955,8.86482) rectangle (8.12935,8.97067);
\draw [color=c, fill=c] (8.12935,8.86482) rectangle (8.16915,8.97067);
\draw [color=c, fill=c] (8.16915,8.86482) rectangle (8.20895,8.97067);
\draw [color=c, fill=c] (8.20895,8.86482) rectangle (8.24876,8.97067);
\draw [color=c, fill=c] (8.24876,8.86482) rectangle (8.28856,8.97067);
\draw [color=c, fill=c] (8.28856,8.86482) rectangle (8.32836,8.97067);
\draw [color=c, fill=c] (8.32836,8.86482) rectangle (8.36816,8.97067);
\draw [color=c, fill=c] (8.36816,8.86482) rectangle (8.40796,8.97067);
\draw [color=c, fill=c] (8.40796,8.86482) rectangle (8.44776,8.97067);
\draw [color=c, fill=c] (8.44776,8.86482) rectangle (8.48756,8.97067);
\draw [color=c, fill=c] (8.48756,8.86482) rectangle (8.52736,8.97067);
\draw [color=c, fill=c] (8.52736,8.86482) rectangle (8.56716,8.97067);
\draw [color=c, fill=c] (8.56716,8.86482) rectangle (8.60697,8.97067);
\draw [color=c, fill=c] (8.60697,8.86482) rectangle (8.64677,8.97067);
\draw [color=c, fill=c] (8.64677,8.86482) rectangle (8.68657,8.97067);
\draw [color=c, fill=c] (8.68657,8.86482) rectangle (8.72637,8.97067);
\draw [color=c, fill=c] (8.72637,8.86482) rectangle (8.76617,8.97067);
\draw [color=c, fill=c] (8.76617,8.86482) rectangle (8.80597,8.97067);
\draw [color=c, fill=c] (8.80597,8.86482) rectangle (8.84577,8.97067);
\draw [color=c, fill=c] (8.84577,8.86482) rectangle (8.88557,8.97067);
\draw [color=c, fill=c] (8.88557,8.86482) rectangle (8.92537,8.97067);
\draw [color=c, fill=c] (8.92537,8.86482) rectangle (8.96517,8.97067);
\draw [color=c, fill=c] (8.96517,8.86482) rectangle (9.00498,8.97067);
\draw [color=c, fill=c] (9.00498,8.86482) rectangle (9.04478,8.97067);
\draw [color=c, fill=c] (9.04478,8.86482) rectangle (9.08458,8.97067);
\draw [color=c, fill=c] (9.08458,8.86482) rectangle (9.12438,8.97067);
\draw [color=c, fill=c] (9.12438,8.86482) rectangle (9.16418,8.97067);
\draw [color=c, fill=c] (9.16418,8.86482) rectangle (9.20398,8.97067);
\draw [color=c, fill=c] (9.20398,8.86482) rectangle (9.24378,8.97067);
\draw [color=c, fill=c] (9.24378,8.86482) rectangle (9.28358,8.97067);
\draw [color=c, fill=c] (9.28358,8.86482) rectangle (9.32338,8.97067);
\draw [color=c, fill=c] (9.32338,8.86482) rectangle (9.36318,8.97067);
\draw [color=c, fill=c] (9.36318,8.86482) rectangle (9.40298,8.97067);
\draw [color=c, fill=c] (9.40298,8.86482) rectangle (9.44279,8.97067);
\draw [color=c, fill=c] (9.44279,8.86482) rectangle (9.48259,8.97067);
\definecolor{c}{rgb}{0,0.266667,1};
\draw [color=c, fill=c] (9.48259,8.86482) rectangle (9.52239,8.97067);
\draw [color=c, fill=c] (9.52239,8.86482) rectangle (9.56219,8.97067);
\draw [color=c, fill=c] (9.56219,8.86482) rectangle (9.60199,8.97067);
\draw [color=c, fill=c] (9.60199,8.86482) rectangle (9.64179,8.97067);
\draw [color=c, fill=c] (9.64179,8.86482) rectangle (9.68159,8.97067);
\draw [color=c, fill=c] (9.68159,8.86482) rectangle (9.72139,8.97067);
\draw [color=c, fill=c] (9.72139,8.86482) rectangle (9.76119,8.97067);
\draw [color=c, fill=c] (9.76119,8.86482) rectangle (9.80099,8.97067);
\draw [color=c, fill=c] (9.80099,8.86482) rectangle (9.8408,8.97067);
\draw [color=c, fill=c] (9.8408,8.86482) rectangle (9.8806,8.97067);
\draw [color=c, fill=c] (9.8806,8.86482) rectangle (9.9204,8.97067);
\draw [color=c, fill=c] (9.9204,8.86482) rectangle (9.9602,8.97067);
\draw [color=c, fill=c] (9.9602,8.86482) rectangle (10,8.97067);
\draw [color=c, fill=c] (10,8.86482) rectangle (10.0398,8.97067);
\draw [color=c, fill=c] (10.0398,8.86482) rectangle (10.0796,8.97067);
\draw [color=c, fill=c] (10.0796,8.86482) rectangle (10.1194,8.97067);
\draw [color=c, fill=c] (10.1194,8.86482) rectangle (10.1592,8.97067);
\draw [color=c, fill=c] (10.1592,8.86482) rectangle (10.199,8.97067);
\draw [color=c, fill=c] (10.199,8.86482) rectangle (10.2388,8.97067);
\draw [color=c, fill=c] (10.2388,8.86482) rectangle (10.2786,8.97067);
\draw [color=c, fill=c] (10.2786,8.86482) rectangle (10.3184,8.97067);
\draw [color=c, fill=c] (10.3184,8.86482) rectangle (10.3582,8.97067);
\draw [color=c, fill=c] (10.3582,8.86482) rectangle (10.398,8.97067);
\draw [color=c, fill=c] (10.398,8.86482) rectangle (10.4378,8.97067);
\draw [color=c, fill=c] (10.4378,8.86482) rectangle (10.4776,8.97067);
\draw [color=c, fill=c] (10.4776,8.86482) rectangle (10.5174,8.97067);
\draw [color=c, fill=c] (10.5174,8.86482) rectangle (10.5572,8.97067);
\definecolor{c}{rgb}{0,0.546666,1};
\draw [color=c, fill=c] (10.5572,8.86482) rectangle (10.597,8.97067);
\draw [color=c, fill=c] (10.597,8.86482) rectangle (10.6368,8.97067);
\draw [color=c, fill=c] (10.6368,8.86482) rectangle (10.6766,8.97067);
\draw [color=c, fill=c] (10.6766,8.86482) rectangle (10.7164,8.97067);
\draw [color=c, fill=c] (10.7164,8.86482) rectangle (10.7562,8.97067);
\draw [color=c, fill=c] (10.7562,8.86482) rectangle (10.796,8.97067);
\draw [color=c, fill=c] (10.796,8.86482) rectangle (10.8358,8.97067);
\draw [color=c, fill=c] (10.8358,8.86482) rectangle (10.8756,8.97067);
\draw [color=c, fill=c] (10.8756,8.86482) rectangle (10.9154,8.97067);
\draw [color=c, fill=c] (10.9154,8.86482) rectangle (10.9552,8.97067);
\draw [color=c, fill=c] (10.9552,8.86482) rectangle (10.995,8.97067);
\draw [color=c, fill=c] (10.995,8.86482) rectangle (11.0348,8.97067);
\draw [color=c, fill=c] (11.0348,8.86482) rectangle (11.0746,8.97067);
\draw [color=c, fill=c] (11.0746,8.86482) rectangle (11.1144,8.97067);
\draw [color=c, fill=c] (11.1144,8.86482) rectangle (11.1542,8.97067);
\draw [color=c, fill=c] (11.1542,8.86482) rectangle (11.194,8.97067);
\draw [color=c, fill=c] (11.194,8.86482) rectangle (11.2338,8.97067);
\draw [color=c, fill=c] (11.2338,8.86482) rectangle (11.2736,8.97067);
\draw [color=c, fill=c] (11.2736,8.86482) rectangle (11.3134,8.97067);
\draw [color=c, fill=c] (11.3134,8.86482) rectangle (11.3532,8.97067);
\draw [color=c, fill=c] (11.3532,8.86482) rectangle (11.393,8.97067);
\draw [color=c, fill=c] (11.393,8.86482) rectangle (11.4328,8.97067);
\draw [color=c, fill=c] (11.4328,8.86482) rectangle (11.4726,8.97067);
\draw [color=c, fill=c] (11.4726,8.86482) rectangle (11.5124,8.97067);
\draw [color=c, fill=c] (11.5124,8.86482) rectangle (11.5522,8.97067);
\draw [color=c, fill=c] (11.5522,8.86482) rectangle (11.592,8.97067);
\draw [color=c, fill=c] (11.592,8.86482) rectangle (11.6318,8.97067);
\draw [color=c, fill=c] (11.6318,8.86482) rectangle (11.6716,8.97067);
\draw [color=c, fill=c] (11.6716,8.86482) rectangle (11.7114,8.97067);
\draw [color=c, fill=c] (11.7114,8.86482) rectangle (11.7512,8.97067);
\draw [color=c, fill=c] (11.7512,8.86482) rectangle (11.791,8.97067);
\draw [color=c, fill=c] (11.791,8.86482) rectangle (11.8308,8.97067);
\draw [color=c, fill=c] (11.8308,8.86482) rectangle (11.8706,8.97067);
\draw [color=c, fill=c] (11.8706,8.86482) rectangle (11.9104,8.97067);
\draw [color=c, fill=c] (11.9104,8.86482) rectangle (11.9502,8.97067);
\draw [color=c, fill=c] (11.9502,8.86482) rectangle (11.99,8.97067);
\draw [color=c, fill=c] (11.99,8.86482) rectangle (12.0299,8.97067);
\draw [color=c, fill=c] (12.0299,8.86482) rectangle (12.0697,8.97067);
\draw [color=c, fill=c] (12.0697,8.86482) rectangle (12.1095,8.97067);
\draw [color=c, fill=c] (12.1095,8.86482) rectangle (12.1493,8.97067);
\draw [color=c, fill=c] (12.1493,8.86482) rectangle (12.1891,8.97067);
\draw [color=c, fill=c] (12.1891,8.86482) rectangle (12.2289,8.97067);
\draw [color=c, fill=c] (12.2289,8.86482) rectangle (12.2687,8.97067);
\draw [color=c, fill=c] (12.2687,8.86482) rectangle (12.3085,8.97067);
\draw [color=c, fill=c] (12.3085,8.86482) rectangle (12.3483,8.97067);
\draw [color=c, fill=c] (12.3483,8.86482) rectangle (12.3881,8.97067);
\draw [color=c, fill=c] (12.3881,8.86482) rectangle (12.4279,8.97067);
\draw [color=c, fill=c] (12.4279,8.86482) rectangle (12.4677,8.97067);
\draw [color=c, fill=c] (12.4677,8.86482) rectangle (12.5075,8.97067);
\draw [color=c, fill=c] (12.5075,8.86482) rectangle (12.5473,8.97067);
\draw [color=c, fill=c] (12.5473,8.86482) rectangle (12.5871,8.97067);
\draw [color=c, fill=c] (12.5871,8.86482) rectangle (12.6269,8.97067);
\draw [color=c, fill=c] (12.6269,8.86482) rectangle (12.6667,8.97067);
\draw [color=c, fill=c] (12.6667,8.86482) rectangle (12.7065,8.97067);
\draw [color=c, fill=c] (12.7065,8.86482) rectangle (12.7463,8.97067);
\draw [color=c, fill=c] (12.7463,8.86482) rectangle (12.7861,8.97067);
\draw [color=c, fill=c] (12.7861,8.86482) rectangle (12.8259,8.97067);
\draw [color=c, fill=c] (12.8259,8.86482) rectangle (12.8657,8.97067);
\draw [color=c, fill=c] (12.8657,8.86482) rectangle (12.9055,8.97067);
\draw [color=c, fill=c] (12.9055,8.86482) rectangle (12.9453,8.97067);
\draw [color=c, fill=c] (12.9453,8.86482) rectangle (12.9851,8.97067);
\draw [color=c, fill=c] (12.9851,8.86482) rectangle (13.0249,8.97067);
\draw [color=c, fill=c] (13.0249,8.86482) rectangle (13.0647,8.97067);
\definecolor{c}{rgb}{0,0.733333,1};
\draw [color=c, fill=c] (13.0647,8.86482) rectangle (13.1045,8.97067);
\draw [color=c, fill=c] (13.1045,8.86482) rectangle (13.1443,8.97067);
\draw [color=c, fill=c] (13.1443,8.86482) rectangle (13.1841,8.97067);
\draw [color=c, fill=c] (13.1841,8.86482) rectangle (13.2239,8.97067);
\draw [color=c, fill=c] (13.2239,8.86482) rectangle (13.2637,8.97067);
\draw [color=c, fill=c] (13.2637,8.86482) rectangle (13.3035,8.97067);
\draw [color=c, fill=c] (13.3035,8.86482) rectangle (13.3433,8.97067);
\draw [color=c, fill=c] (13.3433,8.86482) rectangle (13.3831,8.97067);
\draw [color=c, fill=c] (13.3831,8.86482) rectangle (13.4229,8.97067);
\draw [color=c, fill=c] (13.4229,8.86482) rectangle (13.4627,8.97067);
\draw [color=c, fill=c] (13.4627,8.86482) rectangle (13.5025,8.97067);
\draw [color=c, fill=c] (13.5025,8.86482) rectangle (13.5423,8.97067);
\draw [color=c, fill=c] (13.5423,8.86482) rectangle (13.5821,8.97067);
\draw [color=c, fill=c] (13.5821,8.86482) rectangle (13.6219,8.97067);
\draw [color=c, fill=c] (13.6219,8.86482) rectangle (13.6617,8.97067);
\draw [color=c, fill=c] (13.6617,8.86482) rectangle (13.7015,8.97067);
\draw [color=c, fill=c] (13.7015,8.86482) rectangle (13.7413,8.97067);
\draw [color=c, fill=c] (13.7413,8.86482) rectangle (13.7811,8.97067);
\draw [color=c, fill=c] (13.7811,8.86482) rectangle (13.8209,8.97067);
\draw [color=c, fill=c] (13.8209,8.86482) rectangle (13.8607,8.97067);
\draw [color=c, fill=c] (13.8607,8.86482) rectangle (13.9005,8.97067);
\draw [color=c, fill=c] (13.9005,8.86482) rectangle (13.9403,8.97067);
\draw [color=c, fill=c] (13.9403,8.86482) rectangle (13.9801,8.97067);
\draw [color=c, fill=c] (13.9801,8.86482) rectangle (14.0199,8.97067);
\draw [color=c, fill=c] (14.0199,8.86482) rectangle (14.0597,8.97067);
\draw [color=c, fill=c] (14.0597,8.86482) rectangle (14.0995,8.97067);
\draw [color=c, fill=c] (14.0995,8.86482) rectangle (14.1393,8.97067);
\draw [color=c, fill=c] (14.1393,8.86482) rectangle (14.1791,8.97067);
\draw [color=c, fill=c] (14.1791,8.86482) rectangle (14.2189,8.97067);
\draw [color=c, fill=c] (14.2189,8.86482) rectangle (14.2587,8.97067);
\draw [color=c, fill=c] (14.2587,8.86482) rectangle (14.2985,8.97067);
\draw [color=c, fill=c] (14.2985,8.86482) rectangle (14.3383,8.97067);
\draw [color=c, fill=c] (14.3383,8.86482) rectangle (14.3781,8.97067);
\draw [color=c, fill=c] (14.3781,8.86482) rectangle (14.4179,8.97067);
\draw [color=c, fill=c] (14.4179,8.86482) rectangle (14.4577,8.97067);
\draw [color=c, fill=c] (14.4577,8.86482) rectangle (14.4975,8.97067);
\draw [color=c, fill=c] (14.4975,8.86482) rectangle (14.5373,8.97067);
\draw [color=c, fill=c] (14.5373,8.86482) rectangle (14.5771,8.97067);
\draw [color=c, fill=c] (14.5771,8.86482) rectangle (14.6169,8.97067);
\draw [color=c, fill=c] (14.6169,8.86482) rectangle (14.6567,8.97067);
\draw [color=c, fill=c] (14.6567,8.86482) rectangle (14.6965,8.97067);
\draw [color=c, fill=c] (14.6965,8.86482) rectangle (14.7363,8.97067);
\draw [color=c, fill=c] (14.7363,8.86482) rectangle (14.7761,8.97067);
\draw [color=c, fill=c] (14.7761,8.86482) rectangle (14.8159,8.97067);
\draw [color=c, fill=c] (14.8159,8.86482) rectangle (14.8557,8.97067);
\draw [color=c, fill=c] (14.8557,8.86482) rectangle (14.8955,8.97067);
\draw [color=c, fill=c] (14.8955,8.86482) rectangle (14.9353,8.97067);
\draw [color=c, fill=c] (14.9353,8.86482) rectangle (14.9751,8.97067);
\draw [color=c, fill=c] (14.9751,8.86482) rectangle (15.0149,8.97067);
\draw [color=c, fill=c] (15.0149,8.86482) rectangle (15.0547,8.97067);
\draw [color=c, fill=c] (15.0547,8.86482) rectangle (15.0945,8.97067);
\draw [color=c, fill=c] (15.0945,8.86482) rectangle (15.1343,8.97067);
\draw [color=c, fill=c] (15.1343,8.86482) rectangle (15.1741,8.97067);
\draw [color=c, fill=c] (15.1741,8.86482) rectangle (15.2139,8.97067);
\draw [color=c, fill=c] (15.2139,8.86482) rectangle (15.2537,8.97067);
\draw [color=c, fill=c] (15.2537,8.86482) rectangle (15.2935,8.97067);
\draw [color=c, fill=c] (15.2935,8.86482) rectangle (15.3333,8.97067);
\draw [color=c, fill=c] (15.3333,8.86482) rectangle (15.3731,8.97067);
\draw [color=c, fill=c] (15.3731,8.86482) rectangle (15.4129,8.97067);
\draw [color=c, fill=c] (15.4129,8.86482) rectangle (15.4527,8.97067);
\draw [color=c, fill=c] (15.4527,8.86482) rectangle (15.4925,8.97067);
\draw [color=c, fill=c] (15.4925,8.86482) rectangle (15.5323,8.97067);
\draw [color=c, fill=c] (15.5323,8.86482) rectangle (15.5721,8.97067);
\draw [color=c, fill=c] (15.5721,8.86482) rectangle (15.6119,8.97067);
\draw [color=c, fill=c] (15.6119,8.86482) rectangle (15.6517,8.97067);
\draw [color=c, fill=c] (15.6517,8.86482) rectangle (15.6915,8.97067);
\draw [color=c, fill=c] (15.6915,8.86482) rectangle (15.7313,8.97067);
\draw [color=c, fill=c] (15.7313,8.86482) rectangle (15.7711,8.97067);
\draw [color=c, fill=c] (15.7711,8.86482) rectangle (15.8109,8.97067);
\draw [color=c, fill=c] (15.8109,8.86482) rectangle (15.8507,8.97067);
\draw [color=c, fill=c] (15.8507,8.86482) rectangle (15.8905,8.97067);
\draw [color=c, fill=c] (15.8905,8.86482) rectangle (15.9303,8.97067);
\draw [color=c, fill=c] (15.9303,8.86482) rectangle (15.9701,8.97067);
\draw [color=c, fill=c] (15.9701,8.86482) rectangle (16.01,8.97067);
\draw [color=c, fill=c] (16.01,8.86482) rectangle (16.0498,8.97067);
\draw [color=c, fill=c] (16.0498,8.86482) rectangle (16.0896,8.97067);
\draw [color=c, fill=c] (16.0896,8.86482) rectangle (16.1294,8.97067);
\draw [color=c, fill=c] (16.1294,8.86482) rectangle (16.1692,8.97067);
\draw [color=c, fill=c] (16.1692,8.86482) rectangle (16.209,8.97067);
\draw [color=c, fill=c] (16.209,8.86482) rectangle (16.2488,8.97067);
\draw [color=c, fill=c] (16.2488,8.86482) rectangle (16.2886,8.97067);
\draw [color=c, fill=c] (16.2886,8.86482) rectangle (16.3284,8.97067);
\draw [color=c, fill=c] (16.3284,8.86482) rectangle (16.3682,8.97067);
\draw [color=c, fill=c] (16.3682,8.86482) rectangle (16.408,8.97067);
\draw [color=c, fill=c] (16.408,8.86482) rectangle (16.4478,8.97067);
\draw [color=c, fill=c] (16.4478,8.86482) rectangle (16.4876,8.97067);
\draw [color=c, fill=c] (16.4876,8.86482) rectangle (16.5274,8.97067);
\draw [color=c, fill=c] (16.5274,8.86482) rectangle (16.5672,8.97067);
\draw [color=c, fill=c] (16.5672,8.86482) rectangle (16.607,8.97067);
\draw [color=c, fill=c] (16.607,8.86482) rectangle (16.6468,8.97067);
\draw [color=c, fill=c] (16.6468,8.86482) rectangle (16.6866,8.97067);
\draw [color=c, fill=c] (16.6866,8.86482) rectangle (16.7264,8.97067);
\draw [color=c, fill=c] (16.7264,8.86482) rectangle (16.7662,8.97067);
\draw [color=c, fill=c] (16.7662,8.86482) rectangle (16.806,8.97067);
\draw [color=c, fill=c] (16.806,8.86482) rectangle (16.8458,8.97067);
\draw [color=c, fill=c] (16.8458,8.86482) rectangle (16.8856,8.97067);
\draw [color=c, fill=c] (16.8856,8.86482) rectangle (16.9254,8.97067);
\draw [color=c, fill=c] (16.9254,8.86482) rectangle (16.9652,8.97067);
\draw [color=c, fill=c] (16.9652,8.86482) rectangle (17.005,8.97067);
\draw [color=c, fill=c] (17.005,8.86482) rectangle (17.0448,8.97067);
\draw [color=c, fill=c] (17.0448,8.86482) rectangle (17.0846,8.97067);
\draw [color=c, fill=c] (17.0846,8.86482) rectangle (17.1244,8.97067);
\draw [color=c, fill=c] (17.1244,8.86482) rectangle (17.1642,8.97067);
\draw [color=c, fill=c] (17.1642,8.86482) rectangle (17.204,8.97067);
\draw [color=c, fill=c] (17.204,8.86482) rectangle (17.2438,8.97067);
\draw [color=c, fill=c] (17.2438,8.86482) rectangle (17.2836,8.97067);
\draw [color=c, fill=c] (17.2836,8.86482) rectangle (17.3234,8.97067);
\draw [color=c, fill=c] (17.3234,8.86482) rectangle (17.3632,8.97067);
\draw [color=c, fill=c] (17.3632,8.86482) rectangle (17.403,8.97067);
\draw [color=c, fill=c] (17.403,8.86482) rectangle (17.4428,8.97067);
\draw [color=c, fill=c] (17.4428,8.86482) rectangle (17.4826,8.97067);
\draw [color=c, fill=c] (17.4826,8.86482) rectangle (17.5224,8.97067);
\draw [color=c, fill=c] (17.5224,8.86482) rectangle (17.5622,8.97067);
\draw [color=c, fill=c] (17.5622,8.86482) rectangle (17.602,8.97067);
\draw [color=c, fill=c] (17.602,8.86482) rectangle (17.6418,8.97067);
\draw [color=c, fill=c] (17.6418,8.86482) rectangle (17.6816,8.97067);
\draw [color=c, fill=c] (17.6816,8.86482) rectangle (17.7214,8.97067);
\draw [color=c, fill=c] (17.7214,8.86482) rectangle (17.7612,8.97067);
\draw [color=c, fill=c] (17.7612,8.86482) rectangle (17.801,8.97067);
\draw [color=c, fill=c] (17.801,8.86482) rectangle (17.8408,8.97067);
\draw [color=c, fill=c] (17.8408,8.86482) rectangle (17.8806,8.97067);
\draw [color=c, fill=c] (17.8806,8.86482) rectangle (17.9204,8.97067);
\draw [color=c, fill=c] (17.9204,8.86482) rectangle (17.9602,8.97067);
\draw [color=c, fill=c] (17.9602,8.86482) rectangle (18,8.97067);
\definecolor{c}{rgb}{0.2,0,1};
\draw [color=c, fill=c] (2,8.97067) rectangle (2.0398,9.07652);
\draw [color=c, fill=c] (2.0398,8.97067) rectangle (2.0796,9.07652);
\draw [color=c, fill=c] (2.0796,8.97067) rectangle (2.1194,9.07652);
\draw [color=c, fill=c] (2.1194,8.97067) rectangle (2.1592,9.07652);
\draw [color=c, fill=c] (2.1592,8.97067) rectangle (2.19901,9.07652);
\draw [color=c, fill=c] (2.19901,8.97067) rectangle (2.23881,9.07652);
\draw [color=c, fill=c] (2.23881,8.97067) rectangle (2.27861,9.07652);
\draw [color=c, fill=c] (2.27861,8.97067) rectangle (2.31841,9.07652);
\draw [color=c, fill=c] (2.31841,8.97067) rectangle (2.35821,9.07652);
\draw [color=c, fill=c] (2.35821,8.97067) rectangle (2.39801,9.07652);
\draw [color=c, fill=c] (2.39801,8.97067) rectangle (2.43781,9.07652);
\draw [color=c, fill=c] (2.43781,8.97067) rectangle (2.47761,9.07652);
\draw [color=c, fill=c] (2.47761,8.97067) rectangle (2.51741,9.07652);
\draw [color=c, fill=c] (2.51741,8.97067) rectangle (2.55721,9.07652);
\draw [color=c, fill=c] (2.55721,8.97067) rectangle (2.59702,9.07652);
\draw [color=c, fill=c] (2.59702,8.97067) rectangle (2.63682,9.07652);
\draw [color=c, fill=c] (2.63682,8.97067) rectangle (2.67662,9.07652);
\draw [color=c, fill=c] (2.67662,8.97067) rectangle (2.71642,9.07652);
\draw [color=c, fill=c] (2.71642,8.97067) rectangle (2.75622,9.07652);
\draw [color=c, fill=c] (2.75622,8.97067) rectangle (2.79602,9.07652);
\draw [color=c, fill=c] (2.79602,8.97067) rectangle (2.83582,9.07652);
\draw [color=c, fill=c] (2.83582,8.97067) rectangle (2.87562,9.07652);
\draw [color=c, fill=c] (2.87562,8.97067) rectangle (2.91542,9.07652);
\draw [color=c, fill=c] (2.91542,8.97067) rectangle (2.95522,9.07652);
\draw [color=c, fill=c] (2.95522,8.97067) rectangle (2.99502,9.07652);
\draw [color=c, fill=c] (2.99502,8.97067) rectangle (3.03483,9.07652);
\draw [color=c, fill=c] (3.03483,8.97067) rectangle (3.07463,9.07652);
\draw [color=c, fill=c] (3.07463,8.97067) rectangle (3.11443,9.07652);
\draw [color=c, fill=c] (3.11443,8.97067) rectangle (3.15423,9.07652);
\draw [color=c, fill=c] (3.15423,8.97067) rectangle (3.19403,9.07652);
\draw [color=c, fill=c] (3.19403,8.97067) rectangle (3.23383,9.07652);
\draw [color=c, fill=c] (3.23383,8.97067) rectangle (3.27363,9.07652);
\draw [color=c, fill=c] (3.27363,8.97067) rectangle (3.31343,9.07652);
\draw [color=c, fill=c] (3.31343,8.97067) rectangle (3.35323,9.07652);
\draw [color=c, fill=c] (3.35323,8.97067) rectangle (3.39303,9.07652);
\draw [color=c, fill=c] (3.39303,8.97067) rectangle (3.43284,9.07652);
\draw [color=c, fill=c] (3.43284,8.97067) rectangle (3.47264,9.07652);
\draw [color=c, fill=c] (3.47264,8.97067) rectangle (3.51244,9.07652);
\draw [color=c, fill=c] (3.51244,8.97067) rectangle (3.55224,9.07652);
\draw [color=c, fill=c] (3.55224,8.97067) rectangle (3.59204,9.07652);
\draw [color=c, fill=c] (3.59204,8.97067) rectangle (3.63184,9.07652);
\draw [color=c, fill=c] (3.63184,8.97067) rectangle (3.67164,9.07652);
\draw [color=c, fill=c] (3.67164,8.97067) rectangle (3.71144,9.07652);
\draw [color=c, fill=c] (3.71144,8.97067) rectangle (3.75124,9.07652);
\draw [color=c, fill=c] (3.75124,8.97067) rectangle (3.79104,9.07652);
\draw [color=c, fill=c] (3.79104,8.97067) rectangle (3.83085,9.07652);
\draw [color=c, fill=c] (3.83085,8.97067) rectangle (3.87065,9.07652);
\draw [color=c, fill=c] (3.87065,8.97067) rectangle (3.91045,9.07652);
\draw [color=c, fill=c] (3.91045,8.97067) rectangle (3.95025,9.07652);
\draw [color=c, fill=c] (3.95025,8.97067) rectangle (3.99005,9.07652);
\draw [color=c, fill=c] (3.99005,8.97067) rectangle (4.02985,9.07652);
\draw [color=c, fill=c] (4.02985,8.97067) rectangle (4.06965,9.07652);
\draw [color=c, fill=c] (4.06965,8.97067) rectangle (4.10945,9.07652);
\draw [color=c, fill=c] (4.10945,8.97067) rectangle (4.14925,9.07652);
\draw [color=c, fill=c] (4.14925,8.97067) rectangle (4.18905,9.07652);
\draw [color=c, fill=c] (4.18905,8.97067) rectangle (4.22886,9.07652);
\draw [color=c, fill=c] (4.22886,8.97067) rectangle (4.26866,9.07652);
\draw [color=c, fill=c] (4.26866,8.97067) rectangle (4.30846,9.07652);
\draw [color=c, fill=c] (4.30846,8.97067) rectangle (4.34826,9.07652);
\draw [color=c, fill=c] (4.34826,8.97067) rectangle (4.38806,9.07652);
\draw [color=c, fill=c] (4.38806,8.97067) rectangle (4.42786,9.07652);
\draw [color=c, fill=c] (4.42786,8.97067) rectangle (4.46766,9.07652);
\draw [color=c, fill=c] (4.46766,8.97067) rectangle (4.50746,9.07652);
\draw [color=c, fill=c] (4.50746,8.97067) rectangle (4.54726,9.07652);
\draw [color=c, fill=c] (4.54726,8.97067) rectangle (4.58706,9.07652);
\draw [color=c, fill=c] (4.58706,8.97067) rectangle (4.62687,9.07652);
\draw [color=c, fill=c] (4.62687,8.97067) rectangle (4.66667,9.07652);
\draw [color=c, fill=c] (4.66667,8.97067) rectangle (4.70647,9.07652);
\draw [color=c, fill=c] (4.70647,8.97067) rectangle (4.74627,9.07652);
\draw [color=c, fill=c] (4.74627,8.97067) rectangle (4.78607,9.07652);
\draw [color=c, fill=c] (4.78607,8.97067) rectangle (4.82587,9.07652);
\draw [color=c, fill=c] (4.82587,8.97067) rectangle (4.86567,9.07652);
\draw [color=c, fill=c] (4.86567,8.97067) rectangle (4.90547,9.07652);
\draw [color=c, fill=c] (4.90547,8.97067) rectangle (4.94527,9.07652);
\draw [color=c, fill=c] (4.94527,8.97067) rectangle (4.98507,9.07652);
\draw [color=c, fill=c] (4.98507,8.97067) rectangle (5.02488,9.07652);
\draw [color=c, fill=c] (5.02488,8.97067) rectangle (5.06468,9.07652);
\draw [color=c, fill=c] (5.06468,8.97067) rectangle (5.10448,9.07652);
\draw [color=c, fill=c] (5.10448,8.97067) rectangle (5.14428,9.07652);
\draw [color=c, fill=c] (5.14428,8.97067) rectangle (5.18408,9.07652);
\draw [color=c, fill=c] (5.18408,8.97067) rectangle (5.22388,9.07652);
\draw [color=c, fill=c] (5.22388,8.97067) rectangle (5.26368,9.07652);
\draw [color=c, fill=c] (5.26368,8.97067) rectangle (5.30348,9.07652);
\draw [color=c, fill=c] (5.30348,8.97067) rectangle (5.34328,9.07652);
\draw [color=c, fill=c] (5.34328,8.97067) rectangle (5.38308,9.07652);
\draw [color=c, fill=c] (5.38308,8.97067) rectangle (5.42289,9.07652);
\draw [color=c, fill=c] (5.42289,8.97067) rectangle (5.46269,9.07652);
\draw [color=c, fill=c] (5.46269,8.97067) rectangle (5.50249,9.07652);
\draw [color=c, fill=c] (5.50249,8.97067) rectangle (5.54229,9.07652);
\draw [color=c, fill=c] (5.54229,8.97067) rectangle (5.58209,9.07652);
\draw [color=c, fill=c] (5.58209,8.97067) rectangle (5.62189,9.07652);
\draw [color=c, fill=c] (5.62189,8.97067) rectangle (5.66169,9.07652);
\draw [color=c, fill=c] (5.66169,8.97067) rectangle (5.70149,9.07652);
\draw [color=c, fill=c] (5.70149,8.97067) rectangle (5.74129,9.07652);
\draw [color=c, fill=c] (5.74129,8.97067) rectangle (5.78109,9.07652);
\draw [color=c, fill=c] (5.78109,8.97067) rectangle (5.8209,9.07652);
\draw [color=c, fill=c] (5.8209,8.97067) rectangle (5.8607,9.07652);
\draw [color=c, fill=c] (5.8607,8.97067) rectangle (5.9005,9.07652);
\draw [color=c, fill=c] (5.9005,8.97067) rectangle (5.9403,9.07652);
\draw [color=c, fill=c] (5.9403,8.97067) rectangle (5.9801,9.07652);
\draw [color=c, fill=c] (5.9801,8.97067) rectangle (6.0199,9.07652);
\draw [color=c, fill=c] (6.0199,8.97067) rectangle (6.0597,9.07652);
\draw [color=c, fill=c] (6.0597,8.97067) rectangle (6.0995,9.07652);
\draw [color=c, fill=c] (6.0995,8.97067) rectangle (6.1393,9.07652);
\draw [color=c, fill=c] (6.1393,8.97067) rectangle (6.1791,9.07652);
\draw [color=c, fill=c] (6.1791,8.97067) rectangle (6.21891,9.07652);
\draw [color=c, fill=c] (6.21891,8.97067) rectangle (6.25871,9.07652);
\draw [color=c, fill=c] (6.25871,8.97067) rectangle (6.29851,9.07652);
\draw [color=c, fill=c] (6.29851,8.97067) rectangle (6.33831,9.07652);
\draw [color=c, fill=c] (6.33831,8.97067) rectangle (6.37811,9.07652);
\draw [color=c, fill=c] (6.37811,8.97067) rectangle (6.41791,9.07652);
\draw [color=c, fill=c] (6.41791,8.97067) rectangle (6.45771,9.07652);
\draw [color=c, fill=c] (6.45771,8.97067) rectangle (6.49751,9.07652);
\draw [color=c, fill=c] (6.49751,8.97067) rectangle (6.53731,9.07652);
\draw [color=c, fill=c] (6.53731,8.97067) rectangle (6.57711,9.07652);
\draw [color=c, fill=c] (6.57711,8.97067) rectangle (6.61692,9.07652);
\draw [color=c, fill=c] (6.61692,8.97067) rectangle (6.65672,9.07652);
\draw [color=c, fill=c] (6.65672,8.97067) rectangle (6.69652,9.07652);
\draw [color=c, fill=c] (6.69652,8.97067) rectangle (6.73632,9.07652);
\draw [color=c, fill=c] (6.73632,8.97067) rectangle (6.77612,9.07652);
\draw [color=c, fill=c] (6.77612,8.97067) rectangle (6.81592,9.07652);
\draw [color=c, fill=c] (6.81592,8.97067) rectangle (6.85572,9.07652);
\draw [color=c, fill=c] (6.85572,8.97067) rectangle (6.89552,9.07652);
\draw [color=c, fill=c] (6.89552,8.97067) rectangle (6.93532,9.07652);
\draw [color=c, fill=c] (6.93532,8.97067) rectangle (6.97512,9.07652);
\draw [color=c, fill=c] (6.97512,8.97067) rectangle (7.01493,9.07652);
\draw [color=c, fill=c] (7.01493,8.97067) rectangle (7.05473,9.07652);
\draw [color=c, fill=c] (7.05473,8.97067) rectangle (7.09453,9.07652);
\draw [color=c, fill=c] (7.09453,8.97067) rectangle (7.13433,9.07652);
\draw [color=c, fill=c] (7.13433,8.97067) rectangle (7.17413,9.07652);
\draw [color=c, fill=c] (7.17413,8.97067) rectangle (7.21393,9.07652);
\draw [color=c, fill=c] (7.21393,8.97067) rectangle (7.25373,9.07652);
\draw [color=c, fill=c] (7.25373,8.97067) rectangle (7.29353,9.07652);
\draw [color=c, fill=c] (7.29353,8.97067) rectangle (7.33333,9.07652);
\draw [color=c, fill=c] (7.33333,8.97067) rectangle (7.37313,9.07652);
\draw [color=c, fill=c] (7.37313,8.97067) rectangle (7.41294,9.07652);
\draw [color=c, fill=c] (7.41294,8.97067) rectangle (7.45274,9.07652);
\draw [color=c, fill=c] (7.45274,8.97067) rectangle (7.49254,9.07652);
\draw [color=c, fill=c] (7.49254,8.97067) rectangle (7.53234,9.07652);
\draw [color=c, fill=c] (7.53234,8.97067) rectangle (7.57214,9.07652);
\draw [color=c, fill=c] (7.57214,8.97067) rectangle (7.61194,9.07652);
\definecolor{c}{rgb}{0,0.0800001,1};
\draw [color=c, fill=c] (7.61194,8.97067) rectangle (7.65174,9.07652);
\draw [color=c, fill=c] (7.65174,8.97067) rectangle (7.69154,9.07652);
\draw [color=c, fill=c] (7.69154,8.97067) rectangle (7.73134,9.07652);
\draw [color=c, fill=c] (7.73134,8.97067) rectangle (7.77114,9.07652);
\draw [color=c, fill=c] (7.77114,8.97067) rectangle (7.81095,9.07652);
\draw [color=c, fill=c] (7.81095,8.97067) rectangle (7.85075,9.07652);
\draw [color=c, fill=c] (7.85075,8.97067) rectangle (7.89055,9.07652);
\draw [color=c, fill=c] (7.89055,8.97067) rectangle (7.93035,9.07652);
\draw [color=c, fill=c] (7.93035,8.97067) rectangle (7.97015,9.07652);
\draw [color=c, fill=c] (7.97015,8.97067) rectangle (8.00995,9.07652);
\draw [color=c, fill=c] (8.00995,8.97067) rectangle (8.04975,9.07652);
\draw [color=c, fill=c] (8.04975,8.97067) rectangle (8.08955,9.07652);
\draw [color=c, fill=c] (8.08955,8.97067) rectangle (8.12935,9.07652);
\draw [color=c, fill=c] (8.12935,8.97067) rectangle (8.16915,9.07652);
\draw [color=c, fill=c] (8.16915,8.97067) rectangle (8.20895,9.07652);
\draw [color=c, fill=c] (8.20895,8.97067) rectangle (8.24876,9.07652);
\draw [color=c, fill=c] (8.24876,8.97067) rectangle (8.28856,9.07652);
\draw [color=c, fill=c] (8.28856,8.97067) rectangle (8.32836,9.07652);
\draw [color=c, fill=c] (8.32836,8.97067) rectangle (8.36816,9.07652);
\draw [color=c, fill=c] (8.36816,8.97067) rectangle (8.40796,9.07652);
\draw [color=c, fill=c] (8.40796,8.97067) rectangle (8.44776,9.07652);
\draw [color=c, fill=c] (8.44776,8.97067) rectangle (8.48756,9.07652);
\draw [color=c, fill=c] (8.48756,8.97067) rectangle (8.52736,9.07652);
\draw [color=c, fill=c] (8.52736,8.97067) rectangle (8.56716,9.07652);
\draw [color=c, fill=c] (8.56716,8.97067) rectangle (8.60697,9.07652);
\draw [color=c, fill=c] (8.60697,8.97067) rectangle (8.64677,9.07652);
\draw [color=c, fill=c] (8.64677,8.97067) rectangle (8.68657,9.07652);
\draw [color=c, fill=c] (8.68657,8.97067) rectangle (8.72637,9.07652);
\draw [color=c, fill=c] (8.72637,8.97067) rectangle (8.76617,9.07652);
\draw [color=c, fill=c] (8.76617,8.97067) rectangle (8.80597,9.07652);
\draw [color=c, fill=c] (8.80597,8.97067) rectangle (8.84577,9.07652);
\draw [color=c, fill=c] (8.84577,8.97067) rectangle (8.88557,9.07652);
\draw [color=c, fill=c] (8.88557,8.97067) rectangle (8.92537,9.07652);
\draw [color=c, fill=c] (8.92537,8.97067) rectangle (8.96517,9.07652);
\draw [color=c, fill=c] (8.96517,8.97067) rectangle (9.00498,9.07652);
\draw [color=c, fill=c] (9.00498,8.97067) rectangle (9.04478,9.07652);
\draw [color=c, fill=c] (9.04478,8.97067) rectangle (9.08458,9.07652);
\draw [color=c, fill=c] (9.08458,8.97067) rectangle (9.12438,9.07652);
\draw [color=c, fill=c] (9.12438,8.97067) rectangle (9.16418,9.07652);
\draw [color=c, fill=c] (9.16418,8.97067) rectangle (9.20398,9.07652);
\draw [color=c, fill=c] (9.20398,8.97067) rectangle (9.24378,9.07652);
\draw [color=c, fill=c] (9.24378,8.97067) rectangle (9.28358,9.07652);
\draw [color=c, fill=c] (9.28358,8.97067) rectangle (9.32338,9.07652);
\draw [color=c, fill=c] (9.32338,8.97067) rectangle (9.36318,9.07652);
\draw [color=c, fill=c] (9.36318,8.97067) rectangle (9.40298,9.07652);
\draw [color=c, fill=c] (9.40298,8.97067) rectangle (9.44279,9.07652);
\draw [color=c, fill=c] (9.44279,8.97067) rectangle (9.48259,9.07652);
\definecolor{c}{rgb}{0,0.266667,1};
\draw [color=c, fill=c] (9.48259,8.97067) rectangle (9.52239,9.07652);
\draw [color=c, fill=c] (9.52239,8.97067) rectangle (9.56219,9.07652);
\draw [color=c, fill=c] (9.56219,8.97067) rectangle (9.60199,9.07652);
\draw [color=c, fill=c] (9.60199,8.97067) rectangle (9.64179,9.07652);
\draw [color=c, fill=c] (9.64179,8.97067) rectangle (9.68159,9.07652);
\draw [color=c, fill=c] (9.68159,8.97067) rectangle (9.72139,9.07652);
\draw [color=c, fill=c] (9.72139,8.97067) rectangle (9.76119,9.07652);
\draw [color=c, fill=c] (9.76119,8.97067) rectangle (9.80099,9.07652);
\draw [color=c, fill=c] (9.80099,8.97067) rectangle (9.8408,9.07652);
\draw [color=c, fill=c] (9.8408,8.97067) rectangle (9.8806,9.07652);
\draw [color=c, fill=c] (9.8806,8.97067) rectangle (9.9204,9.07652);
\draw [color=c, fill=c] (9.9204,8.97067) rectangle (9.9602,9.07652);
\draw [color=c, fill=c] (9.9602,8.97067) rectangle (10,9.07652);
\draw [color=c, fill=c] (10,8.97067) rectangle (10.0398,9.07652);
\draw [color=c, fill=c] (10.0398,8.97067) rectangle (10.0796,9.07652);
\draw [color=c, fill=c] (10.0796,8.97067) rectangle (10.1194,9.07652);
\draw [color=c, fill=c] (10.1194,8.97067) rectangle (10.1592,9.07652);
\draw [color=c, fill=c] (10.1592,8.97067) rectangle (10.199,9.07652);
\draw [color=c, fill=c] (10.199,8.97067) rectangle (10.2388,9.07652);
\draw [color=c, fill=c] (10.2388,8.97067) rectangle (10.2786,9.07652);
\draw [color=c, fill=c] (10.2786,8.97067) rectangle (10.3184,9.07652);
\draw [color=c, fill=c] (10.3184,8.97067) rectangle (10.3582,9.07652);
\draw [color=c, fill=c] (10.3582,8.97067) rectangle (10.398,9.07652);
\draw [color=c, fill=c] (10.398,8.97067) rectangle (10.4378,9.07652);
\draw [color=c, fill=c] (10.4378,8.97067) rectangle (10.4776,9.07652);
\draw [color=c, fill=c] (10.4776,8.97067) rectangle (10.5174,9.07652);
\draw [color=c, fill=c] (10.5174,8.97067) rectangle (10.5572,9.07652);
\draw [color=c, fill=c] (10.5572,8.97067) rectangle (10.597,9.07652);
\definecolor{c}{rgb}{0,0.546666,1};
\draw [color=c, fill=c] (10.597,8.97067) rectangle (10.6368,9.07652);
\draw [color=c, fill=c] (10.6368,8.97067) rectangle (10.6766,9.07652);
\draw [color=c, fill=c] (10.6766,8.97067) rectangle (10.7164,9.07652);
\draw [color=c, fill=c] (10.7164,8.97067) rectangle (10.7562,9.07652);
\draw [color=c, fill=c] (10.7562,8.97067) rectangle (10.796,9.07652);
\draw [color=c, fill=c] (10.796,8.97067) rectangle (10.8358,9.07652);
\draw [color=c, fill=c] (10.8358,8.97067) rectangle (10.8756,9.07652);
\draw [color=c, fill=c] (10.8756,8.97067) rectangle (10.9154,9.07652);
\draw [color=c, fill=c] (10.9154,8.97067) rectangle (10.9552,9.07652);
\draw [color=c, fill=c] (10.9552,8.97067) rectangle (10.995,9.07652);
\draw [color=c, fill=c] (10.995,8.97067) rectangle (11.0348,9.07652);
\draw [color=c, fill=c] (11.0348,8.97067) rectangle (11.0746,9.07652);
\draw [color=c, fill=c] (11.0746,8.97067) rectangle (11.1144,9.07652);
\draw [color=c, fill=c] (11.1144,8.97067) rectangle (11.1542,9.07652);
\draw [color=c, fill=c] (11.1542,8.97067) rectangle (11.194,9.07652);
\draw [color=c, fill=c] (11.194,8.97067) rectangle (11.2338,9.07652);
\draw [color=c, fill=c] (11.2338,8.97067) rectangle (11.2736,9.07652);
\draw [color=c, fill=c] (11.2736,8.97067) rectangle (11.3134,9.07652);
\draw [color=c, fill=c] (11.3134,8.97067) rectangle (11.3532,9.07652);
\draw [color=c, fill=c] (11.3532,8.97067) rectangle (11.393,9.07652);
\draw [color=c, fill=c] (11.393,8.97067) rectangle (11.4328,9.07652);
\draw [color=c, fill=c] (11.4328,8.97067) rectangle (11.4726,9.07652);
\draw [color=c, fill=c] (11.4726,8.97067) rectangle (11.5124,9.07652);
\draw [color=c, fill=c] (11.5124,8.97067) rectangle (11.5522,9.07652);
\draw [color=c, fill=c] (11.5522,8.97067) rectangle (11.592,9.07652);
\draw [color=c, fill=c] (11.592,8.97067) rectangle (11.6318,9.07652);
\draw [color=c, fill=c] (11.6318,8.97067) rectangle (11.6716,9.07652);
\draw [color=c, fill=c] (11.6716,8.97067) rectangle (11.7114,9.07652);
\draw [color=c, fill=c] (11.7114,8.97067) rectangle (11.7512,9.07652);
\draw [color=c, fill=c] (11.7512,8.97067) rectangle (11.791,9.07652);
\draw [color=c, fill=c] (11.791,8.97067) rectangle (11.8308,9.07652);
\draw [color=c, fill=c] (11.8308,8.97067) rectangle (11.8706,9.07652);
\draw [color=c, fill=c] (11.8706,8.97067) rectangle (11.9104,9.07652);
\draw [color=c, fill=c] (11.9104,8.97067) rectangle (11.9502,9.07652);
\draw [color=c, fill=c] (11.9502,8.97067) rectangle (11.99,9.07652);
\draw [color=c, fill=c] (11.99,8.97067) rectangle (12.0299,9.07652);
\draw [color=c, fill=c] (12.0299,8.97067) rectangle (12.0697,9.07652);
\draw [color=c, fill=c] (12.0697,8.97067) rectangle (12.1095,9.07652);
\draw [color=c, fill=c] (12.1095,8.97067) rectangle (12.1493,9.07652);
\draw [color=c, fill=c] (12.1493,8.97067) rectangle (12.1891,9.07652);
\draw [color=c, fill=c] (12.1891,8.97067) rectangle (12.2289,9.07652);
\draw [color=c, fill=c] (12.2289,8.97067) rectangle (12.2687,9.07652);
\draw [color=c, fill=c] (12.2687,8.97067) rectangle (12.3085,9.07652);
\draw [color=c, fill=c] (12.3085,8.97067) rectangle (12.3483,9.07652);
\draw [color=c, fill=c] (12.3483,8.97067) rectangle (12.3881,9.07652);
\draw [color=c, fill=c] (12.3881,8.97067) rectangle (12.4279,9.07652);
\draw [color=c, fill=c] (12.4279,8.97067) rectangle (12.4677,9.07652);
\draw [color=c, fill=c] (12.4677,8.97067) rectangle (12.5075,9.07652);
\draw [color=c, fill=c] (12.5075,8.97067) rectangle (12.5473,9.07652);
\draw [color=c, fill=c] (12.5473,8.97067) rectangle (12.5871,9.07652);
\draw [color=c, fill=c] (12.5871,8.97067) rectangle (12.6269,9.07652);
\draw [color=c, fill=c] (12.6269,8.97067) rectangle (12.6667,9.07652);
\draw [color=c, fill=c] (12.6667,8.97067) rectangle (12.7065,9.07652);
\draw [color=c, fill=c] (12.7065,8.97067) rectangle (12.7463,9.07652);
\draw [color=c, fill=c] (12.7463,8.97067) rectangle (12.7861,9.07652);
\draw [color=c, fill=c] (12.7861,8.97067) rectangle (12.8259,9.07652);
\draw [color=c, fill=c] (12.8259,8.97067) rectangle (12.8657,9.07652);
\draw [color=c, fill=c] (12.8657,8.97067) rectangle (12.9055,9.07652);
\draw [color=c, fill=c] (12.9055,8.97067) rectangle (12.9453,9.07652);
\draw [color=c, fill=c] (12.9453,8.97067) rectangle (12.9851,9.07652);
\draw [color=c, fill=c] (12.9851,8.97067) rectangle (13.0249,9.07652);
\draw [color=c, fill=c] (13.0249,8.97067) rectangle (13.0647,9.07652);
\draw [color=c, fill=c] (13.0647,8.97067) rectangle (13.1045,9.07652);
\draw [color=c, fill=c] (13.1045,8.97067) rectangle (13.1443,9.07652);
\definecolor{c}{rgb}{0,0.733333,1};
\draw [color=c, fill=c] (13.1443,8.97067) rectangle (13.1841,9.07652);
\draw [color=c, fill=c] (13.1841,8.97067) rectangle (13.2239,9.07652);
\draw [color=c, fill=c] (13.2239,8.97067) rectangle (13.2637,9.07652);
\draw [color=c, fill=c] (13.2637,8.97067) rectangle (13.3035,9.07652);
\draw [color=c, fill=c] (13.3035,8.97067) rectangle (13.3433,9.07652);
\draw [color=c, fill=c] (13.3433,8.97067) rectangle (13.3831,9.07652);
\draw [color=c, fill=c] (13.3831,8.97067) rectangle (13.4229,9.07652);
\draw [color=c, fill=c] (13.4229,8.97067) rectangle (13.4627,9.07652);
\draw [color=c, fill=c] (13.4627,8.97067) rectangle (13.5025,9.07652);
\draw [color=c, fill=c] (13.5025,8.97067) rectangle (13.5423,9.07652);
\draw [color=c, fill=c] (13.5423,8.97067) rectangle (13.5821,9.07652);
\draw [color=c, fill=c] (13.5821,8.97067) rectangle (13.6219,9.07652);
\draw [color=c, fill=c] (13.6219,8.97067) rectangle (13.6617,9.07652);
\draw [color=c, fill=c] (13.6617,8.97067) rectangle (13.7015,9.07652);
\draw [color=c, fill=c] (13.7015,8.97067) rectangle (13.7413,9.07652);
\draw [color=c, fill=c] (13.7413,8.97067) rectangle (13.7811,9.07652);
\draw [color=c, fill=c] (13.7811,8.97067) rectangle (13.8209,9.07652);
\draw [color=c, fill=c] (13.8209,8.97067) rectangle (13.8607,9.07652);
\draw [color=c, fill=c] (13.8607,8.97067) rectangle (13.9005,9.07652);
\draw [color=c, fill=c] (13.9005,8.97067) rectangle (13.9403,9.07652);
\draw [color=c, fill=c] (13.9403,8.97067) rectangle (13.9801,9.07652);
\draw [color=c, fill=c] (13.9801,8.97067) rectangle (14.0199,9.07652);
\draw [color=c, fill=c] (14.0199,8.97067) rectangle (14.0597,9.07652);
\draw [color=c, fill=c] (14.0597,8.97067) rectangle (14.0995,9.07652);
\draw [color=c, fill=c] (14.0995,8.97067) rectangle (14.1393,9.07652);
\draw [color=c, fill=c] (14.1393,8.97067) rectangle (14.1791,9.07652);
\draw [color=c, fill=c] (14.1791,8.97067) rectangle (14.2189,9.07652);
\draw [color=c, fill=c] (14.2189,8.97067) rectangle (14.2587,9.07652);
\draw [color=c, fill=c] (14.2587,8.97067) rectangle (14.2985,9.07652);
\draw [color=c, fill=c] (14.2985,8.97067) rectangle (14.3383,9.07652);
\draw [color=c, fill=c] (14.3383,8.97067) rectangle (14.3781,9.07652);
\draw [color=c, fill=c] (14.3781,8.97067) rectangle (14.4179,9.07652);
\draw [color=c, fill=c] (14.4179,8.97067) rectangle (14.4577,9.07652);
\draw [color=c, fill=c] (14.4577,8.97067) rectangle (14.4975,9.07652);
\draw [color=c, fill=c] (14.4975,8.97067) rectangle (14.5373,9.07652);
\draw [color=c, fill=c] (14.5373,8.97067) rectangle (14.5771,9.07652);
\draw [color=c, fill=c] (14.5771,8.97067) rectangle (14.6169,9.07652);
\draw [color=c, fill=c] (14.6169,8.97067) rectangle (14.6567,9.07652);
\draw [color=c, fill=c] (14.6567,8.97067) rectangle (14.6965,9.07652);
\draw [color=c, fill=c] (14.6965,8.97067) rectangle (14.7363,9.07652);
\draw [color=c, fill=c] (14.7363,8.97067) rectangle (14.7761,9.07652);
\draw [color=c, fill=c] (14.7761,8.97067) rectangle (14.8159,9.07652);
\draw [color=c, fill=c] (14.8159,8.97067) rectangle (14.8557,9.07652);
\draw [color=c, fill=c] (14.8557,8.97067) rectangle (14.8955,9.07652);
\draw [color=c, fill=c] (14.8955,8.97067) rectangle (14.9353,9.07652);
\draw [color=c, fill=c] (14.9353,8.97067) rectangle (14.9751,9.07652);
\draw [color=c, fill=c] (14.9751,8.97067) rectangle (15.0149,9.07652);
\draw [color=c, fill=c] (15.0149,8.97067) rectangle (15.0547,9.07652);
\draw [color=c, fill=c] (15.0547,8.97067) rectangle (15.0945,9.07652);
\draw [color=c, fill=c] (15.0945,8.97067) rectangle (15.1343,9.07652);
\draw [color=c, fill=c] (15.1343,8.97067) rectangle (15.1741,9.07652);
\draw [color=c, fill=c] (15.1741,8.97067) rectangle (15.2139,9.07652);
\draw [color=c, fill=c] (15.2139,8.97067) rectangle (15.2537,9.07652);
\draw [color=c, fill=c] (15.2537,8.97067) rectangle (15.2935,9.07652);
\draw [color=c, fill=c] (15.2935,8.97067) rectangle (15.3333,9.07652);
\draw [color=c, fill=c] (15.3333,8.97067) rectangle (15.3731,9.07652);
\draw [color=c, fill=c] (15.3731,8.97067) rectangle (15.4129,9.07652);
\draw [color=c, fill=c] (15.4129,8.97067) rectangle (15.4527,9.07652);
\draw [color=c, fill=c] (15.4527,8.97067) rectangle (15.4925,9.07652);
\draw [color=c, fill=c] (15.4925,8.97067) rectangle (15.5323,9.07652);
\draw [color=c, fill=c] (15.5323,8.97067) rectangle (15.5721,9.07652);
\draw [color=c, fill=c] (15.5721,8.97067) rectangle (15.6119,9.07652);
\draw [color=c, fill=c] (15.6119,8.97067) rectangle (15.6517,9.07652);
\draw [color=c, fill=c] (15.6517,8.97067) rectangle (15.6915,9.07652);
\draw [color=c, fill=c] (15.6915,8.97067) rectangle (15.7313,9.07652);
\draw [color=c, fill=c] (15.7313,8.97067) rectangle (15.7711,9.07652);
\draw [color=c, fill=c] (15.7711,8.97067) rectangle (15.8109,9.07652);
\draw [color=c, fill=c] (15.8109,8.97067) rectangle (15.8507,9.07652);
\draw [color=c, fill=c] (15.8507,8.97067) rectangle (15.8905,9.07652);
\draw [color=c, fill=c] (15.8905,8.97067) rectangle (15.9303,9.07652);
\draw [color=c, fill=c] (15.9303,8.97067) rectangle (15.9701,9.07652);
\draw [color=c, fill=c] (15.9701,8.97067) rectangle (16.01,9.07652);
\draw [color=c, fill=c] (16.01,8.97067) rectangle (16.0498,9.07652);
\draw [color=c, fill=c] (16.0498,8.97067) rectangle (16.0896,9.07652);
\draw [color=c, fill=c] (16.0896,8.97067) rectangle (16.1294,9.07652);
\draw [color=c, fill=c] (16.1294,8.97067) rectangle (16.1692,9.07652);
\draw [color=c, fill=c] (16.1692,8.97067) rectangle (16.209,9.07652);
\draw [color=c, fill=c] (16.209,8.97067) rectangle (16.2488,9.07652);
\draw [color=c, fill=c] (16.2488,8.97067) rectangle (16.2886,9.07652);
\draw [color=c, fill=c] (16.2886,8.97067) rectangle (16.3284,9.07652);
\draw [color=c, fill=c] (16.3284,8.97067) rectangle (16.3682,9.07652);
\draw [color=c, fill=c] (16.3682,8.97067) rectangle (16.408,9.07652);
\draw [color=c, fill=c] (16.408,8.97067) rectangle (16.4478,9.07652);
\draw [color=c, fill=c] (16.4478,8.97067) rectangle (16.4876,9.07652);
\draw [color=c, fill=c] (16.4876,8.97067) rectangle (16.5274,9.07652);
\draw [color=c, fill=c] (16.5274,8.97067) rectangle (16.5672,9.07652);
\draw [color=c, fill=c] (16.5672,8.97067) rectangle (16.607,9.07652);
\draw [color=c, fill=c] (16.607,8.97067) rectangle (16.6468,9.07652);
\draw [color=c, fill=c] (16.6468,8.97067) rectangle (16.6866,9.07652);
\draw [color=c, fill=c] (16.6866,8.97067) rectangle (16.7264,9.07652);
\draw [color=c, fill=c] (16.7264,8.97067) rectangle (16.7662,9.07652);
\draw [color=c, fill=c] (16.7662,8.97067) rectangle (16.806,9.07652);
\draw [color=c, fill=c] (16.806,8.97067) rectangle (16.8458,9.07652);
\draw [color=c, fill=c] (16.8458,8.97067) rectangle (16.8856,9.07652);
\draw [color=c, fill=c] (16.8856,8.97067) rectangle (16.9254,9.07652);
\draw [color=c, fill=c] (16.9254,8.97067) rectangle (16.9652,9.07652);
\draw [color=c, fill=c] (16.9652,8.97067) rectangle (17.005,9.07652);
\draw [color=c, fill=c] (17.005,8.97067) rectangle (17.0448,9.07652);
\draw [color=c, fill=c] (17.0448,8.97067) rectangle (17.0846,9.07652);
\draw [color=c, fill=c] (17.0846,8.97067) rectangle (17.1244,9.07652);
\draw [color=c, fill=c] (17.1244,8.97067) rectangle (17.1642,9.07652);
\draw [color=c, fill=c] (17.1642,8.97067) rectangle (17.204,9.07652);
\draw [color=c, fill=c] (17.204,8.97067) rectangle (17.2438,9.07652);
\draw [color=c, fill=c] (17.2438,8.97067) rectangle (17.2836,9.07652);
\draw [color=c, fill=c] (17.2836,8.97067) rectangle (17.3234,9.07652);
\draw [color=c, fill=c] (17.3234,8.97067) rectangle (17.3632,9.07652);
\draw [color=c, fill=c] (17.3632,8.97067) rectangle (17.403,9.07652);
\draw [color=c, fill=c] (17.403,8.97067) rectangle (17.4428,9.07652);
\draw [color=c, fill=c] (17.4428,8.97067) rectangle (17.4826,9.07652);
\draw [color=c, fill=c] (17.4826,8.97067) rectangle (17.5224,9.07652);
\draw [color=c, fill=c] (17.5224,8.97067) rectangle (17.5622,9.07652);
\draw [color=c, fill=c] (17.5622,8.97067) rectangle (17.602,9.07652);
\draw [color=c, fill=c] (17.602,8.97067) rectangle (17.6418,9.07652);
\draw [color=c, fill=c] (17.6418,8.97067) rectangle (17.6816,9.07652);
\draw [color=c, fill=c] (17.6816,8.97067) rectangle (17.7214,9.07652);
\draw [color=c, fill=c] (17.7214,8.97067) rectangle (17.7612,9.07652);
\draw [color=c, fill=c] (17.7612,8.97067) rectangle (17.801,9.07652);
\draw [color=c, fill=c] (17.801,8.97067) rectangle (17.8408,9.07652);
\draw [color=c, fill=c] (17.8408,8.97067) rectangle (17.8806,9.07652);
\draw [color=c, fill=c] (17.8806,8.97067) rectangle (17.9204,9.07652);
\draw [color=c, fill=c] (17.9204,8.97067) rectangle (17.9602,9.07652);
\draw [color=c, fill=c] (17.9602,8.97067) rectangle (18,9.07652);
\definecolor{c}{rgb}{0.2,0,1};
\draw [color=c, fill=c] (2,9.07652) rectangle (2.0398,9.18237);
\draw [color=c, fill=c] (2.0398,9.07652) rectangle (2.0796,9.18237);
\draw [color=c, fill=c] (2.0796,9.07652) rectangle (2.1194,9.18237);
\draw [color=c, fill=c] (2.1194,9.07652) rectangle (2.1592,9.18237);
\draw [color=c, fill=c] (2.1592,9.07652) rectangle (2.19901,9.18237);
\draw [color=c, fill=c] (2.19901,9.07652) rectangle (2.23881,9.18237);
\draw [color=c, fill=c] (2.23881,9.07652) rectangle (2.27861,9.18237);
\draw [color=c, fill=c] (2.27861,9.07652) rectangle (2.31841,9.18237);
\draw [color=c, fill=c] (2.31841,9.07652) rectangle (2.35821,9.18237);
\draw [color=c, fill=c] (2.35821,9.07652) rectangle (2.39801,9.18237);
\draw [color=c, fill=c] (2.39801,9.07652) rectangle (2.43781,9.18237);
\draw [color=c, fill=c] (2.43781,9.07652) rectangle (2.47761,9.18237);
\draw [color=c, fill=c] (2.47761,9.07652) rectangle (2.51741,9.18237);
\draw [color=c, fill=c] (2.51741,9.07652) rectangle (2.55721,9.18237);
\draw [color=c, fill=c] (2.55721,9.07652) rectangle (2.59702,9.18237);
\draw [color=c, fill=c] (2.59702,9.07652) rectangle (2.63682,9.18237);
\draw [color=c, fill=c] (2.63682,9.07652) rectangle (2.67662,9.18237);
\draw [color=c, fill=c] (2.67662,9.07652) rectangle (2.71642,9.18237);
\draw [color=c, fill=c] (2.71642,9.07652) rectangle (2.75622,9.18237);
\draw [color=c, fill=c] (2.75622,9.07652) rectangle (2.79602,9.18237);
\draw [color=c, fill=c] (2.79602,9.07652) rectangle (2.83582,9.18237);
\draw [color=c, fill=c] (2.83582,9.07652) rectangle (2.87562,9.18237);
\draw [color=c, fill=c] (2.87562,9.07652) rectangle (2.91542,9.18237);
\draw [color=c, fill=c] (2.91542,9.07652) rectangle (2.95522,9.18237);
\draw [color=c, fill=c] (2.95522,9.07652) rectangle (2.99502,9.18237);
\draw [color=c, fill=c] (2.99502,9.07652) rectangle (3.03483,9.18237);
\draw [color=c, fill=c] (3.03483,9.07652) rectangle (3.07463,9.18237);
\draw [color=c, fill=c] (3.07463,9.07652) rectangle (3.11443,9.18237);
\draw [color=c, fill=c] (3.11443,9.07652) rectangle (3.15423,9.18237);
\draw [color=c, fill=c] (3.15423,9.07652) rectangle (3.19403,9.18237);
\draw [color=c, fill=c] (3.19403,9.07652) rectangle (3.23383,9.18237);
\draw [color=c, fill=c] (3.23383,9.07652) rectangle (3.27363,9.18237);
\draw [color=c, fill=c] (3.27363,9.07652) rectangle (3.31343,9.18237);
\draw [color=c, fill=c] (3.31343,9.07652) rectangle (3.35323,9.18237);
\draw [color=c, fill=c] (3.35323,9.07652) rectangle (3.39303,9.18237);
\draw [color=c, fill=c] (3.39303,9.07652) rectangle (3.43284,9.18237);
\draw [color=c, fill=c] (3.43284,9.07652) rectangle (3.47264,9.18237);
\draw [color=c, fill=c] (3.47264,9.07652) rectangle (3.51244,9.18237);
\draw [color=c, fill=c] (3.51244,9.07652) rectangle (3.55224,9.18237);
\draw [color=c, fill=c] (3.55224,9.07652) rectangle (3.59204,9.18237);
\draw [color=c, fill=c] (3.59204,9.07652) rectangle (3.63184,9.18237);
\draw [color=c, fill=c] (3.63184,9.07652) rectangle (3.67164,9.18237);
\draw [color=c, fill=c] (3.67164,9.07652) rectangle (3.71144,9.18237);
\draw [color=c, fill=c] (3.71144,9.07652) rectangle (3.75124,9.18237);
\draw [color=c, fill=c] (3.75124,9.07652) rectangle (3.79104,9.18237);
\draw [color=c, fill=c] (3.79104,9.07652) rectangle (3.83085,9.18237);
\draw [color=c, fill=c] (3.83085,9.07652) rectangle (3.87065,9.18237);
\draw [color=c, fill=c] (3.87065,9.07652) rectangle (3.91045,9.18237);
\draw [color=c, fill=c] (3.91045,9.07652) rectangle (3.95025,9.18237);
\draw [color=c, fill=c] (3.95025,9.07652) rectangle (3.99005,9.18237);
\draw [color=c, fill=c] (3.99005,9.07652) rectangle (4.02985,9.18237);
\draw [color=c, fill=c] (4.02985,9.07652) rectangle (4.06965,9.18237);
\draw [color=c, fill=c] (4.06965,9.07652) rectangle (4.10945,9.18237);
\draw [color=c, fill=c] (4.10945,9.07652) rectangle (4.14925,9.18237);
\draw [color=c, fill=c] (4.14925,9.07652) rectangle (4.18905,9.18237);
\draw [color=c, fill=c] (4.18905,9.07652) rectangle (4.22886,9.18237);
\draw [color=c, fill=c] (4.22886,9.07652) rectangle (4.26866,9.18237);
\draw [color=c, fill=c] (4.26866,9.07652) rectangle (4.30846,9.18237);
\draw [color=c, fill=c] (4.30846,9.07652) rectangle (4.34826,9.18237);
\draw [color=c, fill=c] (4.34826,9.07652) rectangle (4.38806,9.18237);
\draw [color=c, fill=c] (4.38806,9.07652) rectangle (4.42786,9.18237);
\draw [color=c, fill=c] (4.42786,9.07652) rectangle (4.46766,9.18237);
\draw [color=c, fill=c] (4.46766,9.07652) rectangle (4.50746,9.18237);
\draw [color=c, fill=c] (4.50746,9.07652) rectangle (4.54726,9.18237);
\draw [color=c, fill=c] (4.54726,9.07652) rectangle (4.58706,9.18237);
\draw [color=c, fill=c] (4.58706,9.07652) rectangle (4.62687,9.18237);
\draw [color=c, fill=c] (4.62687,9.07652) rectangle (4.66667,9.18237);
\draw [color=c, fill=c] (4.66667,9.07652) rectangle (4.70647,9.18237);
\draw [color=c, fill=c] (4.70647,9.07652) rectangle (4.74627,9.18237);
\draw [color=c, fill=c] (4.74627,9.07652) rectangle (4.78607,9.18237);
\draw [color=c, fill=c] (4.78607,9.07652) rectangle (4.82587,9.18237);
\draw [color=c, fill=c] (4.82587,9.07652) rectangle (4.86567,9.18237);
\draw [color=c, fill=c] (4.86567,9.07652) rectangle (4.90547,9.18237);
\draw [color=c, fill=c] (4.90547,9.07652) rectangle (4.94527,9.18237);
\draw [color=c, fill=c] (4.94527,9.07652) rectangle (4.98507,9.18237);
\draw [color=c, fill=c] (4.98507,9.07652) rectangle (5.02488,9.18237);
\draw [color=c, fill=c] (5.02488,9.07652) rectangle (5.06468,9.18237);
\draw [color=c, fill=c] (5.06468,9.07652) rectangle (5.10448,9.18237);
\draw [color=c, fill=c] (5.10448,9.07652) rectangle (5.14428,9.18237);
\draw [color=c, fill=c] (5.14428,9.07652) rectangle (5.18408,9.18237);
\draw [color=c, fill=c] (5.18408,9.07652) rectangle (5.22388,9.18237);
\draw [color=c, fill=c] (5.22388,9.07652) rectangle (5.26368,9.18237);
\draw [color=c, fill=c] (5.26368,9.07652) rectangle (5.30348,9.18237);
\draw [color=c, fill=c] (5.30348,9.07652) rectangle (5.34328,9.18237);
\draw [color=c, fill=c] (5.34328,9.07652) rectangle (5.38308,9.18237);
\draw [color=c, fill=c] (5.38308,9.07652) rectangle (5.42289,9.18237);
\draw [color=c, fill=c] (5.42289,9.07652) rectangle (5.46269,9.18237);
\draw [color=c, fill=c] (5.46269,9.07652) rectangle (5.50249,9.18237);
\draw [color=c, fill=c] (5.50249,9.07652) rectangle (5.54229,9.18237);
\draw [color=c, fill=c] (5.54229,9.07652) rectangle (5.58209,9.18237);
\draw [color=c, fill=c] (5.58209,9.07652) rectangle (5.62189,9.18237);
\draw [color=c, fill=c] (5.62189,9.07652) rectangle (5.66169,9.18237);
\draw [color=c, fill=c] (5.66169,9.07652) rectangle (5.70149,9.18237);
\draw [color=c, fill=c] (5.70149,9.07652) rectangle (5.74129,9.18237);
\draw [color=c, fill=c] (5.74129,9.07652) rectangle (5.78109,9.18237);
\draw [color=c, fill=c] (5.78109,9.07652) rectangle (5.8209,9.18237);
\draw [color=c, fill=c] (5.8209,9.07652) rectangle (5.8607,9.18237);
\draw [color=c, fill=c] (5.8607,9.07652) rectangle (5.9005,9.18237);
\draw [color=c, fill=c] (5.9005,9.07652) rectangle (5.9403,9.18237);
\draw [color=c, fill=c] (5.9403,9.07652) rectangle (5.9801,9.18237);
\draw [color=c, fill=c] (5.9801,9.07652) rectangle (6.0199,9.18237);
\draw [color=c, fill=c] (6.0199,9.07652) rectangle (6.0597,9.18237);
\draw [color=c, fill=c] (6.0597,9.07652) rectangle (6.0995,9.18237);
\draw [color=c, fill=c] (6.0995,9.07652) rectangle (6.1393,9.18237);
\draw [color=c, fill=c] (6.1393,9.07652) rectangle (6.1791,9.18237);
\draw [color=c, fill=c] (6.1791,9.07652) rectangle (6.21891,9.18237);
\draw [color=c, fill=c] (6.21891,9.07652) rectangle (6.25871,9.18237);
\draw [color=c, fill=c] (6.25871,9.07652) rectangle (6.29851,9.18237);
\draw [color=c, fill=c] (6.29851,9.07652) rectangle (6.33831,9.18237);
\draw [color=c, fill=c] (6.33831,9.07652) rectangle (6.37811,9.18237);
\draw [color=c, fill=c] (6.37811,9.07652) rectangle (6.41791,9.18237);
\draw [color=c, fill=c] (6.41791,9.07652) rectangle (6.45771,9.18237);
\draw [color=c, fill=c] (6.45771,9.07652) rectangle (6.49751,9.18237);
\draw [color=c, fill=c] (6.49751,9.07652) rectangle (6.53731,9.18237);
\draw [color=c, fill=c] (6.53731,9.07652) rectangle (6.57711,9.18237);
\draw [color=c, fill=c] (6.57711,9.07652) rectangle (6.61692,9.18237);
\draw [color=c, fill=c] (6.61692,9.07652) rectangle (6.65672,9.18237);
\draw [color=c, fill=c] (6.65672,9.07652) rectangle (6.69652,9.18237);
\draw [color=c, fill=c] (6.69652,9.07652) rectangle (6.73632,9.18237);
\draw [color=c, fill=c] (6.73632,9.07652) rectangle (6.77612,9.18237);
\draw [color=c, fill=c] (6.77612,9.07652) rectangle (6.81592,9.18237);
\draw [color=c, fill=c] (6.81592,9.07652) rectangle (6.85572,9.18237);
\draw [color=c, fill=c] (6.85572,9.07652) rectangle (6.89552,9.18237);
\draw [color=c, fill=c] (6.89552,9.07652) rectangle (6.93532,9.18237);
\draw [color=c, fill=c] (6.93532,9.07652) rectangle (6.97512,9.18237);
\draw [color=c, fill=c] (6.97512,9.07652) rectangle (7.01493,9.18237);
\draw [color=c, fill=c] (7.01493,9.07652) rectangle (7.05473,9.18237);
\draw [color=c, fill=c] (7.05473,9.07652) rectangle (7.09453,9.18237);
\draw [color=c, fill=c] (7.09453,9.07652) rectangle (7.13433,9.18237);
\draw [color=c, fill=c] (7.13433,9.07652) rectangle (7.17413,9.18237);
\draw [color=c, fill=c] (7.17413,9.07652) rectangle (7.21393,9.18237);
\draw [color=c, fill=c] (7.21393,9.07652) rectangle (7.25373,9.18237);
\draw [color=c, fill=c] (7.25373,9.07652) rectangle (7.29353,9.18237);
\draw [color=c, fill=c] (7.29353,9.07652) rectangle (7.33333,9.18237);
\draw [color=c, fill=c] (7.33333,9.07652) rectangle (7.37313,9.18237);
\draw [color=c, fill=c] (7.37313,9.07652) rectangle (7.41294,9.18237);
\draw [color=c, fill=c] (7.41294,9.07652) rectangle (7.45274,9.18237);
\draw [color=c, fill=c] (7.45274,9.07652) rectangle (7.49254,9.18237);
\draw [color=c, fill=c] (7.49254,9.07652) rectangle (7.53234,9.18237);
\draw [color=c, fill=c] (7.53234,9.07652) rectangle (7.57214,9.18237);
\draw [color=c, fill=c] (7.57214,9.07652) rectangle (7.61194,9.18237);
\definecolor{c}{rgb}{0,0.0800001,1};
\draw [color=c, fill=c] (7.61194,9.07652) rectangle (7.65174,9.18237);
\draw [color=c, fill=c] (7.65174,9.07652) rectangle (7.69154,9.18237);
\draw [color=c, fill=c] (7.69154,9.07652) rectangle (7.73134,9.18237);
\draw [color=c, fill=c] (7.73134,9.07652) rectangle (7.77114,9.18237);
\draw [color=c, fill=c] (7.77114,9.07652) rectangle (7.81095,9.18237);
\draw [color=c, fill=c] (7.81095,9.07652) rectangle (7.85075,9.18237);
\draw [color=c, fill=c] (7.85075,9.07652) rectangle (7.89055,9.18237);
\draw [color=c, fill=c] (7.89055,9.07652) rectangle (7.93035,9.18237);
\draw [color=c, fill=c] (7.93035,9.07652) rectangle (7.97015,9.18237);
\draw [color=c, fill=c] (7.97015,9.07652) rectangle (8.00995,9.18237);
\draw [color=c, fill=c] (8.00995,9.07652) rectangle (8.04975,9.18237);
\draw [color=c, fill=c] (8.04975,9.07652) rectangle (8.08955,9.18237);
\draw [color=c, fill=c] (8.08955,9.07652) rectangle (8.12935,9.18237);
\draw [color=c, fill=c] (8.12935,9.07652) rectangle (8.16915,9.18237);
\draw [color=c, fill=c] (8.16915,9.07652) rectangle (8.20895,9.18237);
\draw [color=c, fill=c] (8.20895,9.07652) rectangle (8.24876,9.18237);
\draw [color=c, fill=c] (8.24876,9.07652) rectangle (8.28856,9.18237);
\draw [color=c, fill=c] (8.28856,9.07652) rectangle (8.32836,9.18237);
\draw [color=c, fill=c] (8.32836,9.07652) rectangle (8.36816,9.18237);
\draw [color=c, fill=c] (8.36816,9.07652) rectangle (8.40796,9.18237);
\draw [color=c, fill=c] (8.40796,9.07652) rectangle (8.44776,9.18237);
\draw [color=c, fill=c] (8.44776,9.07652) rectangle (8.48756,9.18237);
\draw [color=c, fill=c] (8.48756,9.07652) rectangle (8.52736,9.18237);
\draw [color=c, fill=c] (8.52736,9.07652) rectangle (8.56716,9.18237);
\draw [color=c, fill=c] (8.56716,9.07652) rectangle (8.60697,9.18237);
\draw [color=c, fill=c] (8.60697,9.07652) rectangle (8.64677,9.18237);
\draw [color=c, fill=c] (8.64677,9.07652) rectangle (8.68657,9.18237);
\draw [color=c, fill=c] (8.68657,9.07652) rectangle (8.72637,9.18237);
\draw [color=c, fill=c] (8.72637,9.07652) rectangle (8.76617,9.18237);
\draw [color=c, fill=c] (8.76617,9.07652) rectangle (8.80597,9.18237);
\draw [color=c, fill=c] (8.80597,9.07652) rectangle (8.84577,9.18237);
\draw [color=c, fill=c] (8.84577,9.07652) rectangle (8.88557,9.18237);
\draw [color=c, fill=c] (8.88557,9.07652) rectangle (8.92537,9.18237);
\draw [color=c, fill=c] (8.92537,9.07652) rectangle (8.96517,9.18237);
\draw [color=c, fill=c] (8.96517,9.07652) rectangle (9.00498,9.18237);
\draw [color=c, fill=c] (9.00498,9.07652) rectangle (9.04478,9.18237);
\draw [color=c, fill=c] (9.04478,9.07652) rectangle (9.08458,9.18237);
\draw [color=c, fill=c] (9.08458,9.07652) rectangle (9.12438,9.18237);
\draw [color=c, fill=c] (9.12438,9.07652) rectangle (9.16418,9.18237);
\draw [color=c, fill=c] (9.16418,9.07652) rectangle (9.20398,9.18237);
\draw [color=c, fill=c] (9.20398,9.07652) rectangle (9.24378,9.18237);
\draw [color=c, fill=c] (9.24378,9.07652) rectangle (9.28358,9.18237);
\draw [color=c, fill=c] (9.28358,9.07652) rectangle (9.32338,9.18237);
\draw [color=c, fill=c] (9.32338,9.07652) rectangle (9.36318,9.18237);
\draw [color=c, fill=c] (9.36318,9.07652) rectangle (9.40298,9.18237);
\draw [color=c, fill=c] (9.40298,9.07652) rectangle (9.44279,9.18237);
\draw [color=c, fill=c] (9.44279,9.07652) rectangle (9.48259,9.18237);
\definecolor{c}{rgb}{0,0.266667,1};
\draw [color=c, fill=c] (9.48259,9.07652) rectangle (9.52239,9.18237);
\draw [color=c, fill=c] (9.52239,9.07652) rectangle (9.56219,9.18237);
\draw [color=c, fill=c] (9.56219,9.07652) rectangle (9.60199,9.18237);
\draw [color=c, fill=c] (9.60199,9.07652) rectangle (9.64179,9.18237);
\draw [color=c, fill=c] (9.64179,9.07652) rectangle (9.68159,9.18237);
\draw [color=c, fill=c] (9.68159,9.07652) rectangle (9.72139,9.18237);
\draw [color=c, fill=c] (9.72139,9.07652) rectangle (9.76119,9.18237);
\draw [color=c, fill=c] (9.76119,9.07652) rectangle (9.80099,9.18237);
\draw [color=c, fill=c] (9.80099,9.07652) rectangle (9.8408,9.18237);
\draw [color=c, fill=c] (9.8408,9.07652) rectangle (9.8806,9.18237);
\draw [color=c, fill=c] (9.8806,9.07652) rectangle (9.9204,9.18237);
\draw [color=c, fill=c] (9.9204,9.07652) rectangle (9.9602,9.18237);
\draw [color=c, fill=c] (9.9602,9.07652) rectangle (10,9.18237);
\draw [color=c, fill=c] (10,9.07652) rectangle (10.0398,9.18237);
\draw [color=c, fill=c] (10.0398,9.07652) rectangle (10.0796,9.18237);
\draw [color=c, fill=c] (10.0796,9.07652) rectangle (10.1194,9.18237);
\draw [color=c, fill=c] (10.1194,9.07652) rectangle (10.1592,9.18237);
\draw [color=c, fill=c] (10.1592,9.07652) rectangle (10.199,9.18237);
\draw [color=c, fill=c] (10.199,9.07652) rectangle (10.2388,9.18237);
\draw [color=c, fill=c] (10.2388,9.07652) rectangle (10.2786,9.18237);
\draw [color=c, fill=c] (10.2786,9.07652) rectangle (10.3184,9.18237);
\draw [color=c, fill=c] (10.3184,9.07652) rectangle (10.3582,9.18237);
\draw [color=c, fill=c] (10.3582,9.07652) rectangle (10.398,9.18237);
\draw [color=c, fill=c] (10.398,9.07652) rectangle (10.4378,9.18237);
\draw [color=c, fill=c] (10.4378,9.07652) rectangle (10.4776,9.18237);
\draw [color=c, fill=c] (10.4776,9.07652) rectangle (10.5174,9.18237);
\draw [color=c, fill=c] (10.5174,9.07652) rectangle (10.5572,9.18237);
\draw [color=c, fill=c] (10.5572,9.07652) rectangle (10.597,9.18237);
\definecolor{c}{rgb}{0,0.546666,1};
\draw [color=c, fill=c] (10.597,9.07652) rectangle (10.6368,9.18237);
\draw [color=c, fill=c] (10.6368,9.07652) rectangle (10.6766,9.18237);
\draw [color=c, fill=c] (10.6766,9.07652) rectangle (10.7164,9.18237);
\draw [color=c, fill=c] (10.7164,9.07652) rectangle (10.7562,9.18237);
\draw [color=c, fill=c] (10.7562,9.07652) rectangle (10.796,9.18237);
\draw [color=c, fill=c] (10.796,9.07652) rectangle (10.8358,9.18237);
\draw [color=c, fill=c] (10.8358,9.07652) rectangle (10.8756,9.18237);
\draw [color=c, fill=c] (10.8756,9.07652) rectangle (10.9154,9.18237);
\draw [color=c, fill=c] (10.9154,9.07652) rectangle (10.9552,9.18237);
\draw [color=c, fill=c] (10.9552,9.07652) rectangle (10.995,9.18237);
\draw [color=c, fill=c] (10.995,9.07652) rectangle (11.0348,9.18237);
\draw [color=c, fill=c] (11.0348,9.07652) rectangle (11.0746,9.18237);
\draw [color=c, fill=c] (11.0746,9.07652) rectangle (11.1144,9.18237);
\draw [color=c, fill=c] (11.1144,9.07652) rectangle (11.1542,9.18237);
\draw [color=c, fill=c] (11.1542,9.07652) rectangle (11.194,9.18237);
\draw [color=c, fill=c] (11.194,9.07652) rectangle (11.2338,9.18237);
\draw [color=c, fill=c] (11.2338,9.07652) rectangle (11.2736,9.18237);
\draw [color=c, fill=c] (11.2736,9.07652) rectangle (11.3134,9.18237);
\draw [color=c, fill=c] (11.3134,9.07652) rectangle (11.3532,9.18237);
\draw [color=c, fill=c] (11.3532,9.07652) rectangle (11.393,9.18237);
\draw [color=c, fill=c] (11.393,9.07652) rectangle (11.4328,9.18237);
\draw [color=c, fill=c] (11.4328,9.07652) rectangle (11.4726,9.18237);
\draw [color=c, fill=c] (11.4726,9.07652) rectangle (11.5124,9.18237);
\draw [color=c, fill=c] (11.5124,9.07652) rectangle (11.5522,9.18237);
\draw [color=c, fill=c] (11.5522,9.07652) rectangle (11.592,9.18237);
\draw [color=c, fill=c] (11.592,9.07652) rectangle (11.6318,9.18237);
\draw [color=c, fill=c] (11.6318,9.07652) rectangle (11.6716,9.18237);
\draw [color=c, fill=c] (11.6716,9.07652) rectangle (11.7114,9.18237);
\draw [color=c, fill=c] (11.7114,9.07652) rectangle (11.7512,9.18237);
\draw [color=c, fill=c] (11.7512,9.07652) rectangle (11.791,9.18237);
\draw [color=c, fill=c] (11.791,9.07652) rectangle (11.8308,9.18237);
\draw [color=c, fill=c] (11.8308,9.07652) rectangle (11.8706,9.18237);
\draw [color=c, fill=c] (11.8706,9.07652) rectangle (11.9104,9.18237);
\draw [color=c, fill=c] (11.9104,9.07652) rectangle (11.9502,9.18237);
\draw [color=c, fill=c] (11.9502,9.07652) rectangle (11.99,9.18237);
\draw [color=c, fill=c] (11.99,9.07652) rectangle (12.0299,9.18237);
\draw [color=c, fill=c] (12.0299,9.07652) rectangle (12.0697,9.18237);
\draw [color=c, fill=c] (12.0697,9.07652) rectangle (12.1095,9.18237);
\draw [color=c, fill=c] (12.1095,9.07652) rectangle (12.1493,9.18237);
\draw [color=c, fill=c] (12.1493,9.07652) rectangle (12.1891,9.18237);
\draw [color=c, fill=c] (12.1891,9.07652) rectangle (12.2289,9.18237);
\draw [color=c, fill=c] (12.2289,9.07652) rectangle (12.2687,9.18237);
\draw [color=c, fill=c] (12.2687,9.07652) rectangle (12.3085,9.18237);
\draw [color=c, fill=c] (12.3085,9.07652) rectangle (12.3483,9.18237);
\draw [color=c, fill=c] (12.3483,9.07652) rectangle (12.3881,9.18237);
\draw [color=c, fill=c] (12.3881,9.07652) rectangle (12.4279,9.18237);
\draw [color=c, fill=c] (12.4279,9.07652) rectangle (12.4677,9.18237);
\draw [color=c, fill=c] (12.4677,9.07652) rectangle (12.5075,9.18237);
\draw [color=c, fill=c] (12.5075,9.07652) rectangle (12.5473,9.18237);
\draw [color=c, fill=c] (12.5473,9.07652) rectangle (12.5871,9.18237);
\draw [color=c, fill=c] (12.5871,9.07652) rectangle (12.6269,9.18237);
\draw [color=c, fill=c] (12.6269,9.07652) rectangle (12.6667,9.18237);
\draw [color=c, fill=c] (12.6667,9.07652) rectangle (12.7065,9.18237);
\draw [color=c, fill=c] (12.7065,9.07652) rectangle (12.7463,9.18237);
\draw [color=c, fill=c] (12.7463,9.07652) rectangle (12.7861,9.18237);
\draw [color=c, fill=c] (12.7861,9.07652) rectangle (12.8259,9.18237);
\draw [color=c, fill=c] (12.8259,9.07652) rectangle (12.8657,9.18237);
\draw [color=c, fill=c] (12.8657,9.07652) rectangle (12.9055,9.18237);
\draw [color=c, fill=c] (12.9055,9.07652) rectangle (12.9453,9.18237);
\draw [color=c, fill=c] (12.9453,9.07652) rectangle (12.9851,9.18237);
\draw [color=c, fill=c] (12.9851,9.07652) rectangle (13.0249,9.18237);
\draw [color=c, fill=c] (13.0249,9.07652) rectangle (13.0647,9.18237);
\draw [color=c, fill=c] (13.0647,9.07652) rectangle (13.1045,9.18237);
\draw [color=c, fill=c] (13.1045,9.07652) rectangle (13.1443,9.18237);
\draw [color=c, fill=c] (13.1443,9.07652) rectangle (13.1841,9.18237);
\draw [color=c, fill=c] (13.1841,9.07652) rectangle (13.2239,9.18237);
\definecolor{c}{rgb}{0,0.733333,1};
\draw [color=c, fill=c] (13.2239,9.07652) rectangle (13.2637,9.18237);
\draw [color=c, fill=c] (13.2637,9.07652) rectangle (13.3035,9.18237);
\draw [color=c, fill=c] (13.3035,9.07652) rectangle (13.3433,9.18237);
\draw [color=c, fill=c] (13.3433,9.07652) rectangle (13.3831,9.18237);
\draw [color=c, fill=c] (13.3831,9.07652) rectangle (13.4229,9.18237);
\draw [color=c, fill=c] (13.4229,9.07652) rectangle (13.4627,9.18237);
\draw [color=c, fill=c] (13.4627,9.07652) rectangle (13.5025,9.18237);
\draw [color=c, fill=c] (13.5025,9.07652) rectangle (13.5423,9.18237);
\draw [color=c, fill=c] (13.5423,9.07652) rectangle (13.5821,9.18237);
\draw [color=c, fill=c] (13.5821,9.07652) rectangle (13.6219,9.18237);
\draw [color=c, fill=c] (13.6219,9.07652) rectangle (13.6617,9.18237);
\draw [color=c, fill=c] (13.6617,9.07652) rectangle (13.7015,9.18237);
\draw [color=c, fill=c] (13.7015,9.07652) rectangle (13.7413,9.18237);
\draw [color=c, fill=c] (13.7413,9.07652) rectangle (13.7811,9.18237);
\draw [color=c, fill=c] (13.7811,9.07652) rectangle (13.8209,9.18237);
\draw [color=c, fill=c] (13.8209,9.07652) rectangle (13.8607,9.18237);
\draw [color=c, fill=c] (13.8607,9.07652) rectangle (13.9005,9.18237);
\draw [color=c, fill=c] (13.9005,9.07652) rectangle (13.9403,9.18237);
\draw [color=c, fill=c] (13.9403,9.07652) rectangle (13.9801,9.18237);
\draw [color=c, fill=c] (13.9801,9.07652) rectangle (14.0199,9.18237);
\draw [color=c, fill=c] (14.0199,9.07652) rectangle (14.0597,9.18237);
\draw [color=c, fill=c] (14.0597,9.07652) rectangle (14.0995,9.18237);
\draw [color=c, fill=c] (14.0995,9.07652) rectangle (14.1393,9.18237);
\draw [color=c, fill=c] (14.1393,9.07652) rectangle (14.1791,9.18237);
\draw [color=c, fill=c] (14.1791,9.07652) rectangle (14.2189,9.18237);
\draw [color=c, fill=c] (14.2189,9.07652) rectangle (14.2587,9.18237);
\draw [color=c, fill=c] (14.2587,9.07652) rectangle (14.2985,9.18237);
\draw [color=c, fill=c] (14.2985,9.07652) rectangle (14.3383,9.18237);
\draw [color=c, fill=c] (14.3383,9.07652) rectangle (14.3781,9.18237);
\draw [color=c, fill=c] (14.3781,9.07652) rectangle (14.4179,9.18237);
\draw [color=c, fill=c] (14.4179,9.07652) rectangle (14.4577,9.18237);
\draw [color=c, fill=c] (14.4577,9.07652) rectangle (14.4975,9.18237);
\draw [color=c, fill=c] (14.4975,9.07652) rectangle (14.5373,9.18237);
\draw [color=c, fill=c] (14.5373,9.07652) rectangle (14.5771,9.18237);
\draw [color=c, fill=c] (14.5771,9.07652) rectangle (14.6169,9.18237);
\draw [color=c, fill=c] (14.6169,9.07652) rectangle (14.6567,9.18237);
\draw [color=c, fill=c] (14.6567,9.07652) rectangle (14.6965,9.18237);
\draw [color=c, fill=c] (14.6965,9.07652) rectangle (14.7363,9.18237);
\draw [color=c, fill=c] (14.7363,9.07652) rectangle (14.7761,9.18237);
\draw [color=c, fill=c] (14.7761,9.07652) rectangle (14.8159,9.18237);
\draw [color=c, fill=c] (14.8159,9.07652) rectangle (14.8557,9.18237);
\draw [color=c, fill=c] (14.8557,9.07652) rectangle (14.8955,9.18237);
\draw [color=c, fill=c] (14.8955,9.07652) rectangle (14.9353,9.18237);
\draw [color=c, fill=c] (14.9353,9.07652) rectangle (14.9751,9.18237);
\draw [color=c, fill=c] (14.9751,9.07652) rectangle (15.0149,9.18237);
\draw [color=c, fill=c] (15.0149,9.07652) rectangle (15.0547,9.18237);
\draw [color=c, fill=c] (15.0547,9.07652) rectangle (15.0945,9.18237);
\draw [color=c, fill=c] (15.0945,9.07652) rectangle (15.1343,9.18237);
\draw [color=c, fill=c] (15.1343,9.07652) rectangle (15.1741,9.18237);
\draw [color=c, fill=c] (15.1741,9.07652) rectangle (15.2139,9.18237);
\draw [color=c, fill=c] (15.2139,9.07652) rectangle (15.2537,9.18237);
\draw [color=c, fill=c] (15.2537,9.07652) rectangle (15.2935,9.18237);
\draw [color=c, fill=c] (15.2935,9.07652) rectangle (15.3333,9.18237);
\draw [color=c, fill=c] (15.3333,9.07652) rectangle (15.3731,9.18237);
\draw [color=c, fill=c] (15.3731,9.07652) rectangle (15.4129,9.18237);
\draw [color=c, fill=c] (15.4129,9.07652) rectangle (15.4527,9.18237);
\draw [color=c, fill=c] (15.4527,9.07652) rectangle (15.4925,9.18237);
\draw [color=c, fill=c] (15.4925,9.07652) rectangle (15.5323,9.18237);
\draw [color=c, fill=c] (15.5323,9.07652) rectangle (15.5721,9.18237);
\draw [color=c, fill=c] (15.5721,9.07652) rectangle (15.6119,9.18237);
\draw [color=c, fill=c] (15.6119,9.07652) rectangle (15.6517,9.18237);
\draw [color=c, fill=c] (15.6517,9.07652) rectangle (15.6915,9.18237);
\draw [color=c, fill=c] (15.6915,9.07652) rectangle (15.7313,9.18237);
\draw [color=c, fill=c] (15.7313,9.07652) rectangle (15.7711,9.18237);
\draw [color=c, fill=c] (15.7711,9.07652) rectangle (15.8109,9.18237);
\draw [color=c, fill=c] (15.8109,9.07652) rectangle (15.8507,9.18237);
\draw [color=c, fill=c] (15.8507,9.07652) rectangle (15.8905,9.18237);
\draw [color=c, fill=c] (15.8905,9.07652) rectangle (15.9303,9.18237);
\draw [color=c, fill=c] (15.9303,9.07652) rectangle (15.9701,9.18237);
\draw [color=c, fill=c] (15.9701,9.07652) rectangle (16.01,9.18237);
\draw [color=c, fill=c] (16.01,9.07652) rectangle (16.0498,9.18237);
\draw [color=c, fill=c] (16.0498,9.07652) rectangle (16.0896,9.18237);
\draw [color=c, fill=c] (16.0896,9.07652) rectangle (16.1294,9.18237);
\draw [color=c, fill=c] (16.1294,9.07652) rectangle (16.1692,9.18237);
\draw [color=c, fill=c] (16.1692,9.07652) rectangle (16.209,9.18237);
\draw [color=c, fill=c] (16.209,9.07652) rectangle (16.2488,9.18237);
\draw [color=c, fill=c] (16.2488,9.07652) rectangle (16.2886,9.18237);
\draw [color=c, fill=c] (16.2886,9.07652) rectangle (16.3284,9.18237);
\draw [color=c, fill=c] (16.3284,9.07652) rectangle (16.3682,9.18237);
\draw [color=c, fill=c] (16.3682,9.07652) rectangle (16.408,9.18237);
\draw [color=c, fill=c] (16.408,9.07652) rectangle (16.4478,9.18237);
\draw [color=c, fill=c] (16.4478,9.07652) rectangle (16.4876,9.18237);
\draw [color=c, fill=c] (16.4876,9.07652) rectangle (16.5274,9.18237);
\draw [color=c, fill=c] (16.5274,9.07652) rectangle (16.5672,9.18237);
\draw [color=c, fill=c] (16.5672,9.07652) rectangle (16.607,9.18237);
\draw [color=c, fill=c] (16.607,9.07652) rectangle (16.6468,9.18237);
\draw [color=c, fill=c] (16.6468,9.07652) rectangle (16.6866,9.18237);
\draw [color=c, fill=c] (16.6866,9.07652) rectangle (16.7264,9.18237);
\draw [color=c, fill=c] (16.7264,9.07652) rectangle (16.7662,9.18237);
\draw [color=c, fill=c] (16.7662,9.07652) rectangle (16.806,9.18237);
\draw [color=c, fill=c] (16.806,9.07652) rectangle (16.8458,9.18237);
\draw [color=c, fill=c] (16.8458,9.07652) rectangle (16.8856,9.18237);
\draw [color=c, fill=c] (16.8856,9.07652) rectangle (16.9254,9.18237);
\draw [color=c, fill=c] (16.9254,9.07652) rectangle (16.9652,9.18237);
\draw [color=c, fill=c] (16.9652,9.07652) rectangle (17.005,9.18237);
\draw [color=c, fill=c] (17.005,9.07652) rectangle (17.0448,9.18237);
\draw [color=c, fill=c] (17.0448,9.07652) rectangle (17.0846,9.18237);
\draw [color=c, fill=c] (17.0846,9.07652) rectangle (17.1244,9.18237);
\draw [color=c, fill=c] (17.1244,9.07652) rectangle (17.1642,9.18237);
\draw [color=c, fill=c] (17.1642,9.07652) rectangle (17.204,9.18237);
\draw [color=c, fill=c] (17.204,9.07652) rectangle (17.2438,9.18237);
\draw [color=c, fill=c] (17.2438,9.07652) rectangle (17.2836,9.18237);
\draw [color=c, fill=c] (17.2836,9.07652) rectangle (17.3234,9.18237);
\draw [color=c, fill=c] (17.3234,9.07652) rectangle (17.3632,9.18237);
\draw [color=c, fill=c] (17.3632,9.07652) rectangle (17.403,9.18237);
\draw [color=c, fill=c] (17.403,9.07652) rectangle (17.4428,9.18237);
\draw [color=c, fill=c] (17.4428,9.07652) rectangle (17.4826,9.18237);
\draw [color=c, fill=c] (17.4826,9.07652) rectangle (17.5224,9.18237);
\draw [color=c, fill=c] (17.5224,9.07652) rectangle (17.5622,9.18237);
\draw [color=c, fill=c] (17.5622,9.07652) rectangle (17.602,9.18237);
\draw [color=c, fill=c] (17.602,9.07652) rectangle (17.6418,9.18237);
\draw [color=c, fill=c] (17.6418,9.07652) rectangle (17.6816,9.18237);
\draw [color=c, fill=c] (17.6816,9.07652) rectangle (17.7214,9.18237);
\draw [color=c, fill=c] (17.7214,9.07652) rectangle (17.7612,9.18237);
\draw [color=c, fill=c] (17.7612,9.07652) rectangle (17.801,9.18237);
\draw [color=c, fill=c] (17.801,9.07652) rectangle (17.8408,9.18237);
\draw [color=c, fill=c] (17.8408,9.07652) rectangle (17.8806,9.18237);
\draw [color=c, fill=c] (17.8806,9.07652) rectangle (17.9204,9.18237);
\draw [color=c, fill=c] (17.9204,9.07652) rectangle (17.9602,9.18237);
\draw [color=c, fill=c] (17.9602,9.07652) rectangle (18,9.18237);
\definecolor{c}{rgb}{0.2,0,1};
\draw [color=c, fill=c] (2,9.18237) rectangle (2.0398,9.28822);
\draw [color=c, fill=c] (2.0398,9.18237) rectangle (2.0796,9.28822);
\draw [color=c, fill=c] (2.0796,9.18237) rectangle (2.1194,9.28822);
\draw [color=c, fill=c] (2.1194,9.18237) rectangle (2.1592,9.28822);
\draw [color=c, fill=c] (2.1592,9.18237) rectangle (2.19901,9.28822);
\draw [color=c, fill=c] (2.19901,9.18237) rectangle (2.23881,9.28822);
\draw [color=c, fill=c] (2.23881,9.18237) rectangle (2.27861,9.28822);
\draw [color=c, fill=c] (2.27861,9.18237) rectangle (2.31841,9.28822);
\draw [color=c, fill=c] (2.31841,9.18237) rectangle (2.35821,9.28822);
\draw [color=c, fill=c] (2.35821,9.18237) rectangle (2.39801,9.28822);
\draw [color=c, fill=c] (2.39801,9.18237) rectangle (2.43781,9.28822);
\draw [color=c, fill=c] (2.43781,9.18237) rectangle (2.47761,9.28822);
\draw [color=c, fill=c] (2.47761,9.18237) rectangle (2.51741,9.28822);
\draw [color=c, fill=c] (2.51741,9.18237) rectangle (2.55721,9.28822);
\draw [color=c, fill=c] (2.55721,9.18237) rectangle (2.59702,9.28822);
\draw [color=c, fill=c] (2.59702,9.18237) rectangle (2.63682,9.28822);
\draw [color=c, fill=c] (2.63682,9.18237) rectangle (2.67662,9.28822);
\draw [color=c, fill=c] (2.67662,9.18237) rectangle (2.71642,9.28822);
\draw [color=c, fill=c] (2.71642,9.18237) rectangle (2.75622,9.28822);
\draw [color=c, fill=c] (2.75622,9.18237) rectangle (2.79602,9.28822);
\draw [color=c, fill=c] (2.79602,9.18237) rectangle (2.83582,9.28822);
\draw [color=c, fill=c] (2.83582,9.18237) rectangle (2.87562,9.28822);
\draw [color=c, fill=c] (2.87562,9.18237) rectangle (2.91542,9.28822);
\draw [color=c, fill=c] (2.91542,9.18237) rectangle (2.95522,9.28822);
\draw [color=c, fill=c] (2.95522,9.18237) rectangle (2.99502,9.28822);
\draw [color=c, fill=c] (2.99502,9.18237) rectangle (3.03483,9.28822);
\draw [color=c, fill=c] (3.03483,9.18237) rectangle (3.07463,9.28822);
\draw [color=c, fill=c] (3.07463,9.18237) rectangle (3.11443,9.28822);
\draw [color=c, fill=c] (3.11443,9.18237) rectangle (3.15423,9.28822);
\draw [color=c, fill=c] (3.15423,9.18237) rectangle (3.19403,9.28822);
\draw [color=c, fill=c] (3.19403,9.18237) rectangle (3.23383,9.28822);
\draw [color=c, fill=c] (3.23383,9.18237) rectangle (3.27363,9.28822);
\draw [color=c, fill=c] (3.27363,9.18237) rectangle (3.31343,9.28822);
\draw [color=c, fill=c] (3.31343,9.18237) rectangle (3.35323,9.28822);
\draw [color=c, fill=c] (3.35323,9.18237) rectangle (3.39303,9.28822);
\draw [color=c, fill=c] (3.39303,9.18237) rectangle (3.43284,9.28822);
\draw [color=c, fill=c] (3.43284,9.18237) rectangle (3.47264,9.28822);
\draw [color=c, fill=c] (3.47264,9.18237) rectangle (3.51244,9.28822);
\draw [color=c, fill=c] (3.51244,9.18237) rectangle (3.55224,9.28822);
\draw [color=c, fill=c] (3.55224,9.18237) rectangle (3.59204,9.28822);
\draw [color=c, fill=c] (3.59204,9.18237) rectangle (3.63184,9.28822);
\draw [color=c, fill=c] (3.63184,9.18237) rectangle (3.67164,9.28822);
\draw [color=c, fill=c] (3.67164,9.18237) rectangle (3.71144,9.28822);
\draw [color=c, fill=c] (3.71144,9.18237) rectangle (3.75124,9.28822);
\draw [color=c, fill=c] (3.75124,9.18237) rectangle (3.79104,9.28822);
\draw [color=c, fill=c] (3.79104,9.18237) rectangle (3.83085,9.28822);
\draw [color=c, fill=c] (3.83085,9.18237) rectangle (3.87065,9.28822);
\draw [color=c, fill=c] (3.87065,9.18237) rectangle (3.91045,9.28822);
\draw [color=c, fill=c] (3.91045,9.18237) rectangle (3.95025,9.28822);
\draw [color=c, fill=c] (3.95025,9.18237) rectangle (3.99005,9.28822);
\draw [color=c, fill=c] (3.99005,9.18237) rectangle (4.02985,9.28822);
\draw [color=c, fill=c] (4.02985,9.18237) rectangle (4.06965,9.28822);
\draw [color=c, fill=c] (4.06965,9.18237) rectangle (4.10945,9.28822);
\draw [color=c, fill=c] (4.10945,9.18237) rectangle (4.14925,9.28822);
\draw [color=c, fill=c] (4.14925,9.18237) rectangle (4.18905,9.28822);
\draw [color=c, fill=c] (4.18905,9.18237) rectangle (4.22886,9.28822);
\draw [color=c, fill=c] (4.22886,9.18237) rectangle (4.26866,9.28822);
\draw [color=c, fill=c] (4.26866,9.18237) rectangle (4.30846,9.28822);
\draw [color=c, fill=c] (4.30846,9.18237) rectangle (4.34826,9.28822);
\draw [color=c, fill=c] (4.34826,9.18237) rectangle (4.38806,9.28822);
\draw [color=c, fill=c] (4.38806,9.18237) rectangle (4.42786,9.28822);
\draw [color=c, fill=c] (4.42786,9.18237) rectangle (4.46766,9.28822);
\draw [color=c, fill=c] (4.46766,9.18237) rectangle (4.50746,9.28822);
\draw [color=c, fill=c] (4.50746,9.18237) rectangle (4.54726,9.28822);
\draw [color=c, fill=c] (4.54726,9.18237) rectangle (4.58706,9.28822);
\draw [color=c, fill=c] (4.58706,9.18237) rectangle (4.62687,9.28822);
\draw [color=c, fill=c] (4.62687,9.18237) rectangle (4.66667,9.28822);
\draw [color=c, fill=c] (4.66667,9.18237) rectangle (4.70647,9.28822);
\draw [color=c, fill=c] (4.70647,9.18237) rectangle (4.74627,9.28822);
\draw [color=c, fill=c] (4.74627,9.18237) rectangle (4.78607,9.28822);
\draw [color=c, fill=c] (4.78607,9.18237) rectangle (4.82587,9.28822);
\draw [color=c, fill=c] (4.82587,9.18237) rectangle (4.86567,9.28822);
\draw [color=c, fill=c] (4.86567,9.18237) rectangle (4.90547,9.28822);
\draw [color=c, fill=c] (4.90547,9.18237) rectangle (4.94527,9.28822);
\draw [color=c, fill=c] (4.94527,9.18237) rectangle (4.98507,9.28822);
\draw [color=c, fill=c] (4.98507,9.18237) rectangle (5.02488,9.28822);
\draw [color=c, fill=c] (5.02488,9.18237) rectangle (5.06468,9.28822);
\draw [color=c, fill=c] (5.06468,9.18237) rectangle (5.10448,9.28822);
\draw [color=c, fill=c] (5.10448,9.18237) rectangle (5.14428,9.28822);
\draw [color=c, fill=c] (5.14428,9.18237) rectangle (5.18408,9.28822);
\draw [color=c, fill=c] (5.18408,9.18237) rectangle (5.22388,9.28822);
\draw [color=c, fill=c] (5.22388,9.18237) rectangle (5.26368,9.28822);
\draw [color=c, fill=c] (5.26368,9.18237) rectangle (5.30348,9.28822);
\draw [color=c, fill=c] (5.30348,9.18237) rectangle (5.34328,9.28822);
\draw [color=c, fill=c] (5.34328,9.18237) rectangle (5.38308,9.28822);
\draw [color=c, fill=c] (5.38308,9.18237) rectangle (5.42289,9.28822);
\draw [color=c, fill=c] (5.42289,9.18237) rectangle (5.46269,9.28822);
\draw [color=c, fill=c] (5.46269,9.18237) rectangle (5.50249,9.28822);
\draw [color=c, fill=c] (5.50249,9.18237) rectangle (5.54229,9.28822);
\draw [color=c, fill=c] (5.54229,9.18237) rectangle (5.58209,9.28822);
\draw [color=c, fill=c] (5.58209,9.18237) rectangle (5.62189,9.28822);
\draw [color=c, fill=c] (5.62189,9.18237) rectangle (5.66169,9.28822);
\draw [color=c, fill=c] (5.66169,9.18237) rectangle (5.70149,9.28822);
\draw [color=c, fill=c] (5.70149,9.18237) rectangle (5.74129,9.28822);
\draw [color=c, fill=c] (5.74129,9.18237) rectangle (5.78109,9.28822);
\draw [color=c, fill=c] (5.78109,9.18237) rectangle (5.8209,9.28822);
\draw [color=c, fill=c] (5.8209,9.18237) rectangle (5.8607,9.28822);
\draw [color=c, fill=c] (5.8607,9.18237) rectangle (5.9005,9.28822);
\draw [color=c, fill=c] (5.9005,9.18237) rectangle (5.9403,9.28822);
\draw [color=c, fill=c] (5.9403,9.18237) rectangle (5.9801,9.28822);
\draw [color=c, fill=c] (5.9801,9.18237) rectangle (6.0199,9.28822);
\draw [color=c, fill=c] (6.0199,9.18237) rectangle (6.0597,9.28822);
\draw [color=c, fill=c] (6.0597,9.18237) rectangle (6.0995,9.28822);
\draw [color=c, fill=c] (6.0995,9.18237) rectangle (6.1393,9.28822);
\draw [color=c, fill=c] (6.1393,9.18237) rectangle (6.1791,9.28822);
\draw [color=c, fill=c] (6.1791,9.18237) rectangle (6.21891,9.28822);
\draw [color=c, fill=c] (6.21891,9.18237) rectangle (6.25871,9.28822);
\draw [color=c, fill=c] (6.25871,9.18237) rectangle (6.29851,9.28822);
\draw [color=c, fill=c] (6.29851,9.18237) rectangle (6.33831,9.28822);
\draw [color=c, fill=c] (6.33831,9.18237) rectangle (6.37811,9.28822);
\draw [color=c, fill=c] (6.37811,9.18237) rectangle (6.41791,9.28822);
\draw [color=c, fill=c] (6.41791,9.18237) rectangle (6.45771,9.28822);
\draw [color=c, fill=c] (6.45771,9.18237) rectangle (6.49751,9.28822);
\draw [color=c, fill=c] (6.49751,9.18237) rectangle (6.53731,9.28822);
\draw [color=c, fill=c] (6.53731,9.18237) rectangle (6.57711,9.28822);
\draw [color=c, fill=c] (6.57711,9.18237) rectangle (6.61692,9.28822);
\draw [color=c, fill=c] (6.61692,9.18237) rectangle (6.65672,9.28822);
\draw [color=c, fill=c] (6.65672,9.18237) rectangle (6.69652,9.28822);
\draw [color=c, fill=c] (6.69652,9.18237) rectangle (6.73632,9.28822);
\draw [color=c, fill=c] (6.73632,9.18237) rectangle (6.77612,9.28822);
\draw [color=c, fill=c] (6.77612,9.18237) rectangle (6.81592,9.28822);
\draw [color=c, fill=c] (6.81592,9.18237) rectangle (6.85572,9.28822);
\draw [color=c, fill=c] (6.85572,9.18237) rectangle (6.89552,9.28822);
\draw [color=c, fill=c] (6.89552,9.18237) rectangle (6.93532,9.28822);
\draw [color=c, fill=c] (6.93532,9.18237) rectangle (6.97512,9.28822);
\draw [color=c, fill=c] (6.97512,9.18237) rectangle (7.01493,9.28822);
\draw [color=c, fill=c] (7.01493,9.18237) rectangle (7.05473,9.28822);
\draw [color=c, fill=c] (7.05473,9.18237) rectangle (7.09453,9.28822);
\draw [color=c, fill=c] (7.09453,9.18237) rectangle (7.13433,9.28822);
\draw [color=c, fill=c] (7.13433,9.18237) rectangle (7.17413,9.28822);
\draw [color=c, fill=c] (7.17413,9.18237) rectangle (7.21393,9.28822);
\draw [color=c, fill=c] (7.21393,9.18237) rectangle (7.25373,9.28822);
\draw [color=c, fill=c] (7.25373,9.18237) rectangle (7.29353,9.28822);
\draw [color=c, fill=c] (7.29353,9.18237) rectangle (7.33333,9.28822);
\draw [color=c, fill=c] (7.33333,9.18237) rectangle (7.37313,9.28822);
\draw [color=c, fill=c] (7.37313,9.18237) rectangle (7.41294,9.28822);
\draw [color=c, fill=c] (7.41294,9.18237) rectangle (7.45274,9.28822);
\draw [color=c, fill=c] (7.45274,9.18237) rectangle (7.49254,9.28822);
\draw [color=c, fill=c] (7.49254,9.18237) rectangle (7.53234,9.28822);
\draw [color=c, fill=c] (7.53234,9.18237) rectangle (7.57214,9.28822);
\draw [color=c, fill=c] (7.57214,9.18237) rectangle (7.61194,9.28822);
\definecolor{c}{rgb}{0,0.0800001,1};
\draw [color=c, fill=c] (7.61194,9.18237) rectangle (7.65174,9.28822);
\draw [color=c, fill=c] (7.65174,9.18237) rectangle (7.69154,9.28822);
\draw [color=c, fill=c] (7.69154,9.18237) rectangle (7.73134,9.28822);
\draw [color=c, fill=c] (7.73134,9.18237) rectangle (7.77114,9.28822);
\draw [color=c, fill=c] (7.77114,9.18237) rectangle (7.81095,9.28822);
\draw [color=c, fill=c] (7.81095,9.18237) rectangle (7.85075,9.28822);
\draw [color=c, fill=c] (7.85075,9.18237) rectangle (7.89055,9.28822);
\draw [color=c, fill=c] (7.89055,9.18237) rectangle (7.93035,9.28822);
\draw [color=c, fill=c] (7.93035,9.18237) rectangle (7.97015,9.28822);
\draw [color=c, fill=c] (7.97015,9.18237) rectangle (8.00995,9.28822);
\draw [color=c, fill=c] (8.00995,9.18237) rectangle (8.04975,9.28822);
\draw [color=c, fill=c] (8.04975,9.18237) rectangle (8.08955,9.28822);
\draw [color=c, fill=c] (8.08955,9.18237) rectangle (8.12935,9.28822);
\draw [color=c, fill=c] (8.12935,9.18237) rectangle (8.16915,9.28822);
\draw [color=c, fill=c] (8.16915,9.18237) rectangle (8.20895,9.28822);
\draw [color=c, fill=c] (8.20895,9.18237) rectangle (8.24876,9.28822);
\draw [color=c, fill=c] (8.24876,9.18237) rectangle (8.28856,9.28822);
\draw [color=c, fill=c] (8.28856,9.18237) rectangle (8.32836,9.28822);
\draw [color=c, fill=c] (8.32836,9.18237) rectangle (8.36816,9.28822);
\draw [color=c, fill=c] (8.36816,9.18237) rectangle (8.40796,9.28822);
\draw [color=c, fill=c] (8.40796,9.18237) rectangle (8.44776,9.28822);
\draw [color=c, fill=c] (8.44776,9.18237) rectangle (8.48756,9.28822);
\draw [color=c, fill=c] (8.48756,9.18237) rectangle (8.52736,9.28822);
\draw [color=c, fill=c] (8.52736,9.18237) rectangle (8.56716,9.28822);
\draw [color=c, fill=c] (8.56716,9.18237) rectangle (8.60697,9.28822);
\draw [color=c, fill=c] (8.60697,9.18237) rectangle (8.64677,9.28822);
\draw [color=c, fill=c] (8.64677,9.18237) rectangle (8.68657,9.28822);
\draw [color=c, fill=c] (8.68657,9.18237) rectangle (8.72637,9.28822);
\draw [color=c, fill=c] (8.72637,9.18237) rectangle (8.76617,9.28822);
\draw [color=c, fill=c] (8.76617,9.18237) rectangle (8.80597,9.28822);
\draw [color=c, fill=c] (8.80597,9.18237) rectangle (8.84577,9.28822);
\draw [color=c, fill=c] (8.84577,9.18237) rectangle (8.88557,9.28822);
\draw [color=c, fill=c] (8.88557,9.18237) rectangle (8.92537,9.28822);
\draw [color=c, fill=c] (8.92537,9.18237) rectangle (8.96517,9.28822);
\draw [color=c, fill=c] (8.96517,9.18237) rectangle (9.00498,9.28822);
\draw [color=c, fill=c] (9.00498,9.18237) rectangle (9.04478,9.28822);
\draw [color=c, fill=c] (9.04478,9.18237) rectangle (9.08458,9.28822);
\draw [color=c, fill=c] (9.08458,9.18237) rectangle (9.12438,9.28822);
\draw [color=c, fill=c] (9.12438,9.18237) rectangle (9.16418,9.28822);
\draw [color=c, fill=c] (9.16418,9.18237) rectangle (9.20398,9.28822);
\draw [color=c, fill=c] (9.20398,9.18237) rectangle (9.24378,9.28822);
\draw [color=c, fill=c] (9.24378,9.18237) rectangle (9.28358,9.28822);
\draw [color=c, fill=c] (9.28358,9.18237) rectangle (9.32338,9.28822);
\draw [color=c, fill=c] (9.32338,9.18237) rectangle (9.36318,9.28822);
\draw [color=c, fill=c] (9.36318,9.18237) rectangle (9.40298,9.28822);
\draw [color=c, fill=c] (9.40298,9.18237) rectangle (9.44279,9.28822);
\draw [color=c, fill=c] (9.44279,9.18237) rectangle (9.48259,9.28822);
\definecolor{c}{rgb}{0,0.266667,1};
\draw [color=c, fill=c] (9.48259,9.18237) rectangle (9.52239,9.28822);
\draw [color=c, fill=c] (9.52239,9.18237) rectangle (9.56219,9.28822);
\draw [color=c, fill=c] (9.56219,9.18237) rectangle (9.60199,9.28822);
\draw [color=c, fill=c] (9.60199,9.18237) rectangle (9.64179,9.28822);
\draw [color=c, fill=c] (9.64179,9.18237) rectangle (9.68159,9.28822);
\draw [color=c, fill=c] (9.68159,9.18237) rectangle (9.72139,9.28822);
\draw [color=c, fill=c] (9.72139,9.18237) rectangle (9.76119,9.28822);
\draw [color=c, fill=c] (9.76119,9.18237) rectangle (9.80099,9.28822);
\draw [color=c, fill=c] (9.80099,9.18237) rectangle (9.8408,9.28822);
\draw [color=c, fill=c] (9.8408,9.18237) rectangle (9.8806,9.28822);
\draw [color=c, fill=c] (9.8806,9.18237) rectangle (9.9204,9.28822);
\draw [color=c, fill=c] (9.9204,9.18237) rectangle (9.9602,9.28822);
\draw [color=c, fill=c] (9.9602,9.18237) rectangle (10,9.28822);
\draw [color=c, fill=c] (10,9.18237) rectangle (10.0398,9.28822);
\draw [color=c, fill=c] (10.0398,9.18237) rectangle (10.0796,9.28822);
\draw [color=c, fill=c] (10.0796,9.18237) rectangle (10.1194,9.28822);
\draw [color=c, fill=c] (10.1194,9.18237) rectangle (10.1592,9.28822);
\draw [color=c, fill=c] (10.1592,9.18237) rectangle (10.199,9.28822);
\draw [color=c, fill=c] (10.199,9.18237) rectangle (10.2388,9.28822);
\draw [color=c, fill=c] (10.2388,9.18237) rectangle (10.2786,9.28822);
\draw [color=c, fill=c] (10.2786,9.18237) rectangle (10.3184,9.28822);
\draw [color=c, fill=c] (10.3184,9.18237) rectangle (10.3582,9.28822);
\draw [color=c, fill=c] (10.3582,9.18237) rectangle (10.398,9.28822);
\draw [color=c, fill=c] (10.398,9.18237) rectangle (10.4378,9.28822);
\draw [color=c, fill=c] (10.4378,9.18237) rectangle (10.4776,9.28822);
\draw [color=c, fill=c] (10.4776,9.18237) rectangle (10.5174,9.28822);
\draw [color=c, fill=c] (10.5174,9.18237) rectangle (10.5572,9.28822);
\draw [color=c, fill=c] (10.5572,9.18237) rectangle (10.597,9.28822);
\draw [color=c, fill=c] (10.597,9.18237) rectangle (10.6368,9.28822);
\definecolor{c}{rgb}{0,0.546666,1};
\draw [color=c, fill=c] (10.6368,9.18237) rectangle (10.6766,9.28822);
\draw [color=c, fill=c] (10.6766,9.18237) rectangle (10.7164,9.28822);
\draw [color=c, fill=c] (10.7164,9.18237) rectangle (10.7562,9.28822);
\draw [color=c, fill=c] (10.7562,9.18237) rectangle (10.796,9.28822);
\draw [color=c, fill=c] (10.796,9.18237) rectangle (10.8358,9.28822);
\draw [color=c, fill=c] (10.8358,9.18237) rectangle (10.8756,9.28822);
\draw [color=c, fill=c] (10.8756,9.18237) rectangle (10.9154,9.28822);
\draw [color=c, fill=c] (10.9154,9.18237) rectangle (10.9552,9.28822);
\draw [color=c, fill=c] (10.9552,9.18237) rectangle (10.995,9.28822);
\draw [color=c, fill=c] (10.995,9.18237) rectangle (11.0348,9.28822);
\draw [color=c, fill=c] (11.0348,9.18237) rectangle (11.0746,9.28822);
\draw [color=c, fill=c] (11.0746,9.18237) rectangle (11.1144,9.28822);
\draw [color=c, fill=c] (11.1144,9.18237) rectangle (11.1542,9.28822);
\draw [color=c, fill=c] (11.1542,9.18237) rectangle (11.194,9.28822);
\draw [color=c, fill=c] (11.194,9.18237) rectangle (11.2338,9.28822);
\draw [color=c, fill=c] (11.2338,9.18237) rectangle (11.2736,9.28822);
\draw [color=c, fill=c] (11.2736,9.18237) rectangle (11.3134,9.28822);
\draw [color=c, fill=c] (11.3134,9.18237) rectangle (11.3532,9.28822);
\draw [color=c, fill=c] (11.3532,9.18237) rectangle (11.393,9.28822);
\draw [color=c, fill=c] (11.393,9.18237) rectangle (11.4328,9.28822);
\draw [color=c, fill=c] (11.4328,9.18237) rectangle (11.4726,9.28822);
\draw [color=c, fill=c] (11.4726,9.18237) rectangle (11.5124,9.28822);
\draw [color=c, fill=c] (11.5124,9.18237) rectangle (11.5522,9.28822);
\draw [color=c, fill=c] (11.5522,9.18237) rectangle (11.592,9.28822);
\draw [color=c, fill=c] (11.592,9.18237) rectangle (11.6318,9.28822);
\draw [color=c, fill=c] (11.6318,9.18237) rectangle (11.6716,9.28822);
\draw [color=c, fill=c] (11.6716,9.18237) rectangle (11.7114,9.28822);
\draw [color=c, fill=c] (11.7114,9.18237) rectangle (11.7512,9.28822);
\draw [color=c, fill=c] (11.7512,9.18237) rectangle (11.791,9.28822);
\draw [color=c, fill=c] (11.791,9.18237) rectangle (11.8308,9.28822);
\draw [color=c, fill=c] (11.8308,9.18237) rectangle (11.8706,9.28822);
\draw [color=c, fill=c] (11.8706,9.18237) rectangle (11.9104,9.28822);
\draw [color=c, fill=c] (11.9104,9.18237) rectangle (11.9502,9.28822);
\draw [color=c, fill=c] (11.9502,9.18237) rectangle (11.99,9.28822);
\draw [color=c, fill=c] (11.99,9.18237) rectangle (12.0299,9.28822);
\draw [color=c, fill=c] (12.0299,9.18237) rectangle (12.0697,9.28822);
\draw [color=c, fill=c] (12.0697,9.18237) rectangle (12.1095,9.28822);
\draw [color=c, fill=c] (12.1095,9.18237) rectangle (12.1493,9.28822);
\draw [color=c, fill=c] (12.1493,9.18237) rectangle (12.1891,9.28822);
\draw [color=c, fill=c] (12.1891,9.18237) rectangle (12.2289,9.28822);
\draw [color=c, fill=c] (12.2289,9.18237) rectangle (12.2687,9.28822);
\draw [color=c, fill=c] (12.2687,9.18237) rectangle (12.3085,9.28822);
\draw [color=c, fill=c] (12.3085,9.18237) rectangle (12.3483,9.28822);
\draw [color=c, fill=c] (12.3483,9.18237) rectangle (12.3881,9.28822);
\draw [color=c, fill=c] (12.3881,9.18237) rectangle (12.4279,9.28822);
\draw [color=c, fill=c] (12.4279,9.18237) rectangle (12.4677,9.28822);
\draw [color=c, fill=c] (12.4677,9.18237) rectangle (12.5075,9.28822);
\draw [color=c, fill=c] (12.5075,9.18237) rectangle (12.5473,9.28822);
\draw [color=c, fill=c] (12.5473,9.18237) rectangle (12.5871,9.28822);
\draw [color=c, fill=c] (12.5871,9.18237) rectangle (12.6269,9.28822);
\draw [color=c, fill=c] (12.6269,9.18237) rectangle (12.6667,9.28822);
\draw [color=c, fill=c] (12.6667,9.18237) rectangle (12.7065,9.28822);
\draw [color=c, fill=c] (12.7065,9.18237) rectangle (12.7463,9.28822);
\draw [color=c, fill=c] (12.7463,9.18237) rectangle (12.7861,9.28822);
\draw [color=c, fill=c] (12.7861,9.18237) rectangle (12.8259,9.28822);
\draw [color=c, fill=c] (12.8259,9.18237) rectangle (12.8657,9.28822);
\draw [color=c, fill=c] (12.8657,9.18237) rectangle (12.9055,9.28822);
\draw [color=c, fill=c] (12.9055,9.18237) rectangle (12.9453,9.28822);
\draw [color=c, fill=c] (12.9453,9.18237) rectangle (12.9851,9.28822);
\draw [color=c, fill=c] (12.9851,9.18237) rectangle (13.0249,9.28822);
\draw [color=c, fill=c] (13.0249,9.18237) rectangle (13.0647,9.28822);
\draw [color=c, fill=c] (13.0647,9.18237) rectangle (13.1045,9.28822);
\draw [color=c, fill=c] (13.1045,9.18237) rectangle (13.1443,9.28822);
\draw [color=c, fill=c] (13.1443,9.18237) rectangle (13.1841,9.28822);
\draw [color=c, fill=c] (13.1841,9.18237) rectangle (13.2239,9.28822);
\draw [color=c, fill=c] (13.2239,9.18237) rectangle (13.2637,9.28822);
\draw [color=c, fill=c] (13.2637,9.18237) rectangle (13.3035,9.28822);
\definecolor{c}{rgb}{0,0.733333,1};
\draw [color=c, fill=c] (13.3035,9.18237) rectangle (13.3433,9.28822);
\draw [color=c, fill=c] (13.3433,9.18237) rectangle (13.3831,9.28822);
\draw [color=c, fill=c] (13.3831,9.18237) rectangle (13.4229,9.28822);
\draw [color=c, fill=c] (13.4229,9.18237) rectangle (13.4627,9.28822);
\draw [color=c, fill=c] (13.4627,9.18237) rectangle (13.5025,9.28822);
\draw [color=c, fill=c] (13.5025,9.18237) rectangle (13.5423,9.28822);
\draw [color=c, fill=c] (13.5423,9.18237) rectangle (13.5821,9.28822);
\draw [color=c, fill=c] (13.5821,9.18237) rectangle (13.6219,9.28822);
\draw [color=c, fill=c] (13.6219,9.18237) rectangle (13.6617,9.28822);
\draw [color=c, fill=c] (13.6617,9.18237) rectangle (13.7015,9.28822);
\draw [color=c, fill=c] (13.7015,9.18237) rectangle (13.7413,9.28822);
\draw [color=c, fill=c] (13.7413,9.18237) rectangle (13.7811,9.28822);
\draw [color=c, fill=c] (13.7811,9.18237) rectangle (13.8209,9.28822);
\draw [color=c, fill=c] (13.8209,9.18237) rectangle (13.8607,9.28822);
\draw [color=c, fill=c] (13.8607,9.18237) rectangle (13.9005,9.28822);
\draw [color=c, fill=c] (13.9005,9.18237) rectangle (13.9403,9.28822);
\draw [color=c, fill=c] (13.9403,9.18237) rectangle (13.9801,9.28822);
\draw [color=c, fill=c] (13.9801,9.18237) rectangle (14.0199,9.28822);
\draw [color=c, fill=c] (14.0199,9.18237) rectangle (14.0597,9.28822);
\draw [color=c, fill=c] (14.0597,9.18237) rectangle (14.0995,9.28822);
\draw [color=c, fill=c] (14.0995,9.18237) rectangle (14.1393,9.28822);
\draw [color=c, fill=c] (14.1393,9.18237) rectangle (14.1791,9.28822);
\draw [color=c, fill=c] (14.1791,9.18237) rectangle (14.2189,9.28822);
\draw [color=c, fill=c] (14.2189,9.18237) rectangle (14.2587,9.28822);
\draw [color=c, fill=c] (14.2587,9.18237) rectangle (14.2985,9.28822);
\draw [color=c, fill=c] (14.2985,9.18237) rectangle (14.3383,9.28822);
\draw [color=c, fill=c] (14.3383,9.18237) rectangle (14.3781,9.28822);
\draw [color=c, fill=c] (14.3781,9.18237) rectangle (14.4179,9.28822);
\draw [color=c, fill=c] (14.4179,9.18237) rectangle (14.4577,9.28822);
\draw [color=c, fill=c] (14.4577,9.18237) rectangle (14.4975,9.28822);
\draw [color=c, fill=c] (14.4975,9.18237) rectangle (14.5373,9.28822);
\draw [color=c, fill=c] (14.5373,9.18237) rectangle (14.5771,9.28822);
\draw [color=c, fill=c] (14.5771,9.18237) rectangle (14.6169,9.28822);
\draw [color=c, fill=c] (14.6169,9.18237) rectangle (14.6567,9.28822);
\draw [color=c, fill=c] (14.6567,9.18237) rectangle (14.6965,9.28822);
\draw [color=c, fill=c] (14.6965,9.18237) rectangle (14.7363,9.28822);
\draw [color=c, fill=c] (14.7363,9.18237) rectangle (14.7761,9.28822);
\draw [color=c, fill=c] (14.7761,9.18237) rectangle (14.8159,9.28822);
\draw [color=c, fill=c] (14.8159,9.18237) rectangle (14.8557,9.28822);
\draw [color=c, fill=c] (14.8557,9.18237) rectangle (14.8955,9.28822);
\draw [color=c, fill=c] (14.8955,9.18237) rectangle (14.9353,9.28822);
\draw [color=c, fill=c] (14.9353,9.18237) rectangle (14.9751,9.28822);
\draw [color=c, fill=c] (14.9751,9.18237) rectangle (15.0149,9.28822);
\draw [color=c, fill=c] (15.0149,9.18237) rectangle (15.0547,9.28822);
\draw [color=c, fill=c] (15.0547,9.18237) rectangle (15.0945,9.28822);
\draw [color=c, fill=c] (15.0945,9.18237) rectangle (15.1343,9.28822);
\draw [color=c, fill=c] (15.1343,9.18237) rectangle (15.1741,9.28822);
\draw [color=c, fill=c] (15.1741,9.18237) rectangle (15.2139,9.28822);
\draw [color=c, fill=c] (15.2139,9.18237) rectangle (15.2537,9.28822);
\draw [color=c, fill=c] (15.2537,9.18237) rectangle (15.2935,9.28822);
\draw [color=c, fill=c] (15.2935,9.18237) rectangle (15.3333,9.28822);
\draw [color=c, fill=c] (15.3333,9.18237) rectangle (15.3731,9.28822);
\draw [color=c, fill=c] (15.3731,9.18237) rectangle (15.4129,9.28822);
\draw [color=c, fill=c] (15.4129,9.18237) rectangle (15.4527,9.28822);
\draw [color=c, fill=c] (15.4527,9.18237) rectangle (15.4925,9.28822);
\draw [color=c, fill=c] (15.4925,9.18237) rectangle (15.5323,9.28822);
\draw [color=c, fill=c] (15.5323,9.18237) rectangle (15.5721,9.28822);
\draw [color=c, fill=c] (15.5721,9.18237) rectangle (15.6119,9.28822);
\draw [color=c, fill=c] (15.6119,9.18237) rectangle (15.6517,9.28822);
\draw [color=c, fill=c] (15.6517,9.18237) rectangle (15.6915,9.28822);
\draw [color=c, fill=c] (15.6915,9.18237) rectangle (15.7313,9.28822);
\draw [color=c, fill=c] (15.7313,9.18237) rectangle (15.7711,9.28822);
\draw [color=c, fill=c] (15.7711,9.18237) rectangle (15.8109,9.28822);
\draw [color=c, fill=c] (15.8109,9.18237) rectangle (15.8507,9.28822);
\draw [color=c, fill=c] (15.8507,9.18237) rectangle (15.8905,9.28822);
\draw [color=c, fill=c] (15.8905,9.18237) rectangle (15.9303,9.28822);
\draw [color=c, fill=c] (15.9303,9.18237) rectangle (15.9701,9.28822);
\draw [color=c, fill=c] (15.9701,9.18237) rectangle (16.01,9.28822);
\draw [color=c, fill=c] (16.01,9.18237) rectangle (16.0498,9.28822);
\draw [color=c, fill=c] (16.0498,9.18237) rectangle (16.0896,9.28822);
\draw [color=c, fill=c] (16.0896,9.18237) rectangle (16.1294,9.28822);
\draw [color=c, fill=c] (16.1294,9.18237) rectangle (16.1692,9.28822);
\draw [color=c, fill=c] (16.1692,9.18237) rectangle (16.209,9.28822);
\draw [color=c, fill=c] (16.209,9.18237) rectangle (16.2488,9.28822);
\draw [color=c, fill=c] (16.2488,9.18237) rectangle (16.2886,9.28822);
\draw [color=c, fill=c] (16.2886,9.18237) rectangle (16.3284,9.28822);
\draw [color=c, fill=c] (16.3284,9.18237) rectangle (16.3682,9.28822);
\draw [color=c, fill=c] (16.3682,9.18237) rectangle (16.408,9.28822);
\draw [color=c, fill=c] (16.408,9.18237) rectangle (16.4478,9.28822);
\draw [color=c, fill=c] (16.4478,9.18237) rectangle (16.4876,9.28822);
\draw [color=c, fill=c] (16.4876,9.18237) rectangle (16.5274,9.28822);
\draw [color=c, fill=c] (16.5274,9.18237) rectangle (16.5672,9.28822);
\draw [color=c, fill=c] (16.5672,9.18237) rectangle (16.607,9.28822);
\draw [color=c, fill=c] (16.607,9.18237) rectangle (16.6468,9.28822);
\draw [color=c, fill=c] (16.6468,9.18237) rectangle (16.6866,9.28822);
\draw [color=c, fill=c] (16.6866,9.18237) rectangle (16.7264,9.28822);
\draw [color=c, fill=c] (16.7264,9.18237) rectangle (16.7662,9.28822);
\draw [color=c, fill=c] (16.7662,9.18237) rectangle (16.806,9.28822);
\draw [color=c, fill=c] (16.806,9.18237) rectangle (16.8458,9.28822);
\draw [color=c, fill=c] (16.8458,9.18237) rectangle (16.8856,9.28822);
\draw [color=c, fill=c] (16.8856,9.18237) rectangle (16.9254,9.28822);
\draw [color=c, fill=c] (16.9254,9.18237) rectangle (16.9652,9.28822);
\draw [color=c, fill=c] (16.9652,9.18237) rectangle (17.005,9.28822);
\draw [color=c, fill=c] (17.005,9.18237) rectangle (17.0448,9.28822);
\draw [color=c, fill=c] (17.0448,9.18237) rectangle (17.0846,9.28822);
\draw [color=c, fill=c] (17.0846,9.18237) rectangle (17.1244,9.28822);
\draw [color=c, fill=c] (17.1244,9.18237) rectangle (17.1642,9.28822);
\draw [color=c, fill=c] (17.1642,9.18237) rectangle (17.204,9.28822);
\draw [color=c, fill=c] (17.204,9.18237) rectangle (17.2438,9.28822);
\draw [color=c, fill=c] (17.2438,9.18237) rectangle (17.2836,9.28822);
\draw [color=c, fill=c] (17.2836,9.18237) rectangle (17.3234,9.28822);
\draw [color=c, fill=c] (17.3234,9.18237) rectangle (17.3632,9.28822);
\draw [color=c, fill=c] (17.3632,9.18237) rectangle (17.403,9.28822);
\draw [color=c, fill=c] (17.403,9.18237) rectangle (17.4428,9.28822);
\draw [color=c, fill=c] (17.4428,9.18237) rectangle (17.4826,9.28822);
\draw [color=c, fill=c] (17.4826,9.18237) rectangle (17.5224,9.28822);
\draw [color=c, fill=c] (17.5224,9.18237) rectangle (17.5622,9.28822);
\draw [color=c, fill=c] (17.5622,9.18237) rectangle (17.602,9.28822);
\draw [color=c, fill=c] (17.602,9.18237) rectangle (17.6418,9.28822);
\draw [color=c, fill=c] (17.6418,9.18237) rectangle (17.6816,9.28822);
\draw [color=c, fill=c] (17.6816,9.18237) rectangle (17.7214,9.28822);
\draw [color=c, fill=c] (17.7214,9.18237) rectangle (17.7612,9.28822);
\draw [color=c, fill=c] (17.7612,9.18237) rectangle (17.801,9.28822);
\draw [color=c, fill=c] (17.801,9.18237) rectangle (17.8408,9.28822);
\draw [color=c, fill=c] (17.8408,9.18237) rectangle (17.8806,9.28822);
\draw [color=c, fill=c] (17.8806,9.18237) rectangle (17.9204,9.28822);
\draw [color=c, fill=c] (17.9204,9.18237) rectangle (17.9602,9.28822);
\draw [color=c, fill=c] (17.9602,9.18237) rectangle (18,9.28822);
\definecolor{c}{rgb}{0.2,0,1};
\draw [color=c, fill=c] (2,9.28822) rectangle (2.0398,9.39407);
\draw [color=c, fill=c] (2.0398,9.28822) rectangle (2.0796,9.39407);
\draw [color=c, fill=c] (2.0796,9.28822) rectangle (2.1194,9.39407);
\draw [color=c, fill=c] (2.1194,9.28822) rectangle (2.1592,9.39407);
\draw [color=c, fill=c] (2.1592,9.28822) rectangle (2.19901,9.39407);
\draw [color=c, fill=c] (2.19901,9.28822) rectangle (2.23881,9.39407);
\draw [color=c, fill=c] (2.23881,9.28822) rectangle (2.27861,9.39407);
\draw [color=c, fill=c] (2.27861,9.28822) rectangle (2.31841,9.39407);
\draw [color=c, fill=c] (2.31841,9.28822) rectangle (2.35821,9.39407);
\draw [color=c, fill=c] (2.35821,9.28822) rectangle (2.39801,9.39407);
\draw [color=c, fill=c] (2.39801,9.28822) rectangle (2.43781,9.39407);
\draw [color=c, fill=c] (2.43781,9.28822) rectangle (2.47761,9.39407);
\draw [color=c, fill=c] (2.47761,9.28822) rectangle (2.51741,9.39407);
\draw [color=c, fill=c] (2.51741,9.28822) rectangle (2.55721,9.39407);
\draw [color=c, fill=c] (2.55721,9.28822) rectangle (2.59702,9.39407);
\draw [color=c, fill=c] (2.59702,9.28822) rectangle (2.63682,9.39407);
\draw [color=c, fill=c] (2.63682,9.28822) rectangle (2.67662,9.39407);
\draw [color=c, fill=c] (2.67662,9.28822) rectangle (2.71642,9.39407);
\draw [color=c, fill=c] (2.71642,9.28822) rectangle (2.75622,9.39407);
\draw [color=c, fill=c] (2.75622,9.28822) rectangle (2.79602,9.39407);
\draw [color=c, fill=c] (2.79602,9.28822) rectangle (2.83582,9.39407);
\draw [color=c, fill=c] (2.83582,9.28822) rectangle (2.87562,9.39407);
\draw [color=c, fill=c] (2.87562,9.28822) rectangle (2.91542,9.39407);
\draw [color=c, fill=c] (2.91542,9.28822) rectangle (2.95522,9.39407);
\draw [color=c, fill=c] (2.95522,9.28822) rectangle (2.99502,9.39407);
\draw [color=c, fill=c] (2.99502,9.28822) rectangle (3.03483,9.39407);
\draw [color=c, fill=c] (3.03483,9.28822) rectangle (3.07463,9.39407);
\draw [color=c, fill=c] (3.07463,9.28822) rectangle (3.11443,9.39407);
\draw [color=c, fill=c] (3.11443,9.28822) rectangle (3.15423,9.39407);
\draw [color=c, fill=c] (3.15423,9.28822) rectangle (3.19403,9.39407);
\draw [color=c, fill=c] (3.19403,9.28822) rectangle (3.23383,9.39407);
\draw [color=c, fill=c] (3.23383,9.28822) rectangle (3.27363,9.39407);
\draw [color=c, fill=c] (3.27363,9.28822) rectangle (3.31343,9.39407);
\draw [color=c, fill=c] (3.31343,9.28822) rectangle (3.35323,9.39407);
\draw [color=c, fill=c] (3.35323,9.28822) rectangle (3.39303,9.39407);
\draw [color=c, fill=c] (3.39303,9.28822) rectangle (3.43284,9.39407);
\draw [color=c, fill=c] (3.43284,9.28822) rectangle (3.47264,9.39407);
\draw [color=c, fill=c] (3.47264,9.28822) rectangle (3.51244,9.39407);
\draw [color=c, fill=c] (3.51244,9.28822) rectangle (3.55224,9.39407);
\draw [color=c, fill=c] (3.55224,9.28822) rectangle (3.59204,9.39407);
\draw [color=c, fill=c] (3.59204,9.28822) rectangle (3.63184,9.39407);
\draw [color=c, fill=c] (3.63184,9.28822) rectangle (3.67164,9.39407);
\draw [color=c, fill=c] (3.67164,9.28822) rectangle (3.71144,9.39407);
\draw [color=c, fill=c] (3.71144,9.28822) rectangle (3.75124,9.39407);
\draw [color=c, fill=c] (3.75124,9.28822) rectangle (3.79104,9.39407);
\draw [color=c, fill=c] (3.79104,9.28822) rectangle (3.83085,9.39407);
\draw [color=c, fill=c] (3.83085,9.28822) rectangle (3.87065,9.39407);
\draw [color=c, fill=c] (3.87065,9.28822) rectangle (3.91045,9.39407);
\draw [color=c, fill=c] (3.91045,9.28822) rectangle (3.95025,9.39407);
\draw [color=c, fill=c] (3.95025,9.28822) rectangle (3.99005,9.39407);
\draw [color=c, fill=c] (3.99005,9.28822) rectangle (4.02985,9.39407);
\draw [color=c, fill=c] (4.02985,9.28822) rectangle (4.06965,9.39407);
\draw [color=c, fill=c] (4.06965,9.28822) rectangle (4.10945,9.39407);
\draw [color=c, fill=c] (4.10945,9.28822) rectangle (4.14925,9.39407);
\draw [color=c, fill=c] (4.14925,9.28822) rectangle (4.18905,9.39407);
\draw [color=c, fill=c] (4.18905,9.28822) rectangle (4.22886,9.39407);
\draw [color=c, fill=c] (4.22886,9.28822) rectangle (4.26866,9.39407);
\draw [color=c, fill=c] (4.26866,9.28822) rectangle (4.30846,9.39407);
\draw [color=c, fill=c] (4.30846,9.28822) rectangle (4.34826,9.39407);
\draw [color=c, fill=c] (4.34826,9.28822) rectangle (4.38806,9.39407);
\draw [color=c, fill=c] (4.38806,9.28822) rectangle (4.42786,9.39407);
\draw [color=c, fill=c] (4.42786,9.28822) rectangle (4.46766,9.39407);
\draw [color=c, fill=c] (4.46766,9.28822) rectangle (4.50746,9.39407);
\draw [color=c, fill=c] (4.50746,9.28822) rectangle (4.54726,9.39407);
\draw [color=c, fill=c] (4.54726,9.28822) rectangle (4.58706,9.39407);
\draw [color=c, fill=c] (4.58706,9.28822) rectangle (4.62687,9.39407);
\draw [color=c, fill=c] (4.62687,9.28822) rectangle (4.66667,9.39407);
\draw [color=c, fill=c] (4.66667,9.28822) rectangle (4.70647,9.39407);
\draw [color=c, fill=c] (4.70647,9.28822) rectangle (4.74627,9.39407);
\draw [color=c, fill=c] (4.74627,9.28822) rectangle (4.78607,9.39407);
\draw [color=c, fill=c] (4.78607,9.28822) rectangle (4.82587,9.39407);
\draw [color=c, fill=c] (4.82587,9.28822) rectangle (4.86567,9.39407);
\draw [color=c, fill=c] (4.86567,9.28822) rectangle (4.90547,9.39407);
\draw [color=c, fill=c] (4.90547,9.28822) rectangle (4.94527,9.39407);
\draw [color=c, fill=c] (4.94527,9.28822) rectangle (4.98507,9.39407);
\draw [color=c, fill=c] (4.98507,9.28822) rectangle (5.02488,9.39407);
\draw [color=c, fill=c] (5.02488,9.28822) rectangle (5.06468,9.39407);
\draw [color=c, fill=c] (5.06468,9.28822) rectangle (5.10448,9.39407);
\draw [color=c, fill=c] (5.10448,9.28822) rectangle (5.14428,9.39407);
\draw [color=c, fill=c] (5.14428,9.28822) rectangle (5.18408,9.39407);
\draw [color=c, fill=c] (5.18408,9.28822) rectangle (5.22388,9.39407);
\draw [color=c, fill=c] (5.22388,9.28822) rectangle (5.26368,9.39407);
\draw [color=c, fill=c] (5.26368,9.28822) rectangle (5.30348,9.39407);
\draw [color=c, fill=c] (5.30348,9.28822) rectangle (5.34328,9.39407);
\draw [color=c, fill=c] (5.34328,9.28822) rectangle (5.38308,9.39407);
\draw [color=c, fill=c] (5.38308,9.28822) rectangle (5.42289,9.39407);
\draw [color=c, fill=c] (5.42289,9.28822) rectangle (5.46269,9.39407);
\draw [color=c, fill=c] (5.46269,9.28822) rectangle (5.50249,9.39407);
\draw [color=c, fill=c] (5.50249,9.28822) rectangle (5.54229,9.39407);
\draw [color=c, fill=c] (5.54229,9.28822) rectangle (5.58209,9.39407);
\draw [color=c, fill=c] (5.58209,9.28822) rectangle (5.62189,9.39407);
\draw [color=c, fill=c] (5.62189,9.28822) rectangle (5.66169,9.39407);
\draw [color=c, fill=c] (5.66169,9.28822) rectangle (5.70149,9.39407);
\draw [color=c, fill=c] (5.70149,9.28822) rectangle (5.74129,9.39407);
\draw [color=c, fill=c] (5.74129,9.28822) rectangle (5.78109,9.39407);
\draw [color=c, fill=c] (5.78109,9.28822) rectangle (5.8209,9.39407);
\draw [color=c, fill=c] (5.8209,9.28822) rectangle (5.8607,9.39407);
\draw [color=c, fill=c] (5.8607,9.28822) rectangle (5.9005,9.39407);
\draw [color=c, fill=c] (5.9005,9.28822) rectangle (5.9403,9.39407);
\draw [color=c, fill=c] (5.9403,9.28822) rectangle (5.9801,9.39407);
\draw [color=c, fill=c] (5.9801,9.28822) rectangle (6.0199,9.39407);
\draw [color=c, fill=c] (6.0199,9.28822) rectangle (6.0597,9.39407);
\draw [color=c, fill=c] (6.0597,9.28822) rectangle (6.0995,9.39407);
\draw [color=c, fill=c] (6.0995,9.28822) rectangle (6.1393,9.39407);
\draw [color=c, fill=c] (6.1393,9.28822) rectangle (6.1791,9.39407);
\draw [color=c, fill=c] (6.1791,9.28822) rectangle (6.21891,9.39407);
\draw [color=c, fill=c] (6.21891,9.28822) rectangle (6.25871,9.39407);
\draw [color=c, fill=c] (6.25871,9.28822) rectangle (6.29851,9.39407);
\draw [color=c, fill=c] (6.29851,9.28822) rectangle (6.33831,9.39407);
\draw [color=c, fill=c] (6.33831,9.28822) rectangle (6.37811,9.39407);
\draw [color=c, fill=c] (6.37811,9.28822) rectangle (6.41791,9.39407);
\draw [color=c, fill=c] (6.41791,9.28822) rectangle (6.45771,9.39407);
\draw [color=c, fill=c] (6.45771,9.28822) rectangle (6.49751,9.39407);
\draw [color=c, fill=c] (6.49751,9.28822) rectangle (6.53731,9.39407);
\draw [color=c, fill=c] (6.53731,9.28822) rectangle (6.57711,9.39407);
\draw [color=c, fill=c] (6.57711,9.28822) rectangle (6.61692,9.39407);
\draw [color=c, fill=c] (6.61692,9.28822) rectangle (6.65672,9.39407);
\draw [color=c, fill=c] (6.65672,9.28822) rectangle (6.69652,9.39407);
\draw [color=c, fill=c] (6.69652,9.28822) rectangle (6.73632,9.39407);
\draw [color=c, fill=c] (6.73632,9.28822) rectangle (6.77612,9.39407);
\draw [color=c, fill=c] (6.77612,9.28822) rectangle (6.81592,9.39407);
\draw [color=c, fill=c] (6.81592,9.28822) rectangle (6.85572,9.39407);
\draw [color=c, fill=c] (6.85572,9.28822) rectangle (6.89552,9.39407);
\draw [color=c, fill=c] (6.89552,9.28822) rectangle (6.93532,9.39407);
\draw [color=c, fill=c] (6.93532,9.28822) rectangle (6.97512,9.39407);
\draw [color=c, fill=c] (6.97512,9.28822) rectangle (7.01493,9.39407);
\draw [color=c, fill=c] (7.01493,9.28822) rectangle (7.05473,9.39407);
\draw [color=c, fill=c] (7.05473,9.28822) rectangle (7.09453,9.39407);
\draw [color=c, fill=c] (7.09453,9.28822) rectangle (7.13433,9.39407);
\draw [color=c, fill=c] (7.13433,9.28822) rectangle (7.17413,9.39407);
\draw [color=c, fill=c] (7.17413,9.28822) rectangle (7.21393,9.39407);
\draw [color=c, fill=c] (7.21393,9.28822) rectangle (7.25373,9.39407);
\draw [color=c, fill=c] (7.25373,9.28822) rectangle (7.29353,9.39407);
\draw [color=c, fill=c] (7.29353,9.28822) rectangle (7.33333,9.39407);
\draw [color=c, fill=c] (7.33333,9.28822) rectangle (7.37313,9.39407);
\draw [color=c, fill=c] (7.37313,9.28822) rectangle (7.41294,9.39407);
\draw [color=c, fill=c] (7.41294,9.28822) rectangle (7.45274,9.39407);
\draw [color=c, fill=c] (7.45274,9.28822) rectangle (7.49254,9.39407);
\draw [color=c, fill=c] (7.49254,9.28822) rectangle (7.53234,9.39407);
\draw [color=c, fill=c] (7.53234,9.28822) rectangle (7.57214,9.39407);
\draw [color=c, fill=c] (7.57214,9.28822) rectangle (7.61194,9.39407);
\definecolor{c}{rgb}{0,0.0800001,1};
\draw [color=c, fill=c] (7.61194,9.28822) rectangle (7.65174,9.39407);
\draw [color=c, fill=c] (7.65174,9.28822) rectangle (7.69154,9.39407);
\draw [color=c, fill=c] (7.69154,9.28822) rectangle (7.73134,9.39407);
\draw [color=c, fill=c] (7.73134,9.28822) rectangle (7.77114,9.39407);
\draw [color=c, fill=c] (7.77114,9.28822) rectangle (7.81095,9.39407);
\draw [color=c, fill=c] (7.81095,9.28822) rectangle (7.85075,9.39407);
\draw [color=c, fill=c] (7.85075,9.28822) rectangle (7.89055,9.39407);
\draw [color=c, fill=c] (7.89055,9.28822) rectangle (7.93035,9.39407);
\draw [color=c, fill=c] (7.93035,9.28822) rectangle (7.97015,9.39407);
\draw [color=c, fill=c] (7.97015,9.28822) rectangle (8.00995,9.39407);
\draw [color=c, fill=c] (8.00995,9.28822) rectangle (8.04975,9.39407);
\draw [color=c, fill=c] (8.04975,9.28822) rectangle (8.08955,9.39407);
\draw [color=c, fill=c] (8.08955,9.28822) rectangle (8.12935,9.39407);
\draw [color=c, fill=c] (8.12935,9.28822) rectangle (8.16915,9.39407);
\draw [color=c, fill=c] (8.16915,9.28822) rectangle (8.20895,9.39407);
\draw [color=c, fill=c] (8.20895,9.28822) rectangle (8.24876,9.39407);
\draw [color=c, fill=c] (8.24876,9.28822) rectangle (8.28856,9.39407);
\draw [color=c, fill=c] (8.28856,9.28822) rectangle (8.32836,9.39407);
\draw [color=c, fill=c] (8.32836,9.28822) rectangle (8.36816,9.39407);
\draw [color=c, fill=c] (8.36816,9.28822) rectangle (8.40796,9.39407);
\draw [color=c, fill=c] (8.40796,9.28822) rectangle (8.44776,9.39407);
\draw [color=c, fill=c] (8.44776,9.28822) rectangle (8.48756,9.39407);
\draw [color=c, fill=c] (8.48756,9.28822) rectangle (8.52736,9.39407);
\draw [color=c, fill=c] (8.52736,9.28822) rectangle (8.56716,9.39407);
\draw [color=c, fill=c] (8.56716,9.28822) rectangle (8.60697,9.39407);
\draw [color=c, fill=c] (8.60697,9.28822) rectangle (8.64677,9.39407);
\draw [color=c, fill=c] (8.64677,9.28822) rectangle (8.68657,9.39407);
\draw [color=c, fill=c] (8.68657,9.28822) rectangle (8.72637,9.39407);
\draw [color=c, fill=c] (8.72637,9.28822) rectangle (8.76617,9.39407);
\draw [color=c, fill=c] (8.76617,9.28822) rectangle (8.80597,9.39407);
\draw [color=c, fill=c] (8.80597,9.28822) rectangle (8.84577,9.39407);
\draw [color=c, fill=c] (8.84577,9.28822) rectangle (8.88557,9.39407);
\draw [color=c, fill=c] (8.88557,9.28822) rectangle (8.92537,9.39407);
\draw [color=c, fill=c] (8.92537,9.28822) rectangle (8.96517,9.39407);
\draw [color=c, fill=c] (8.96517,9.28822) rectangle (9.00498,9.39407);
\draw [color=c, fill=c] (9.00498,9.28822) rectangle (9.04478,9.39407);
\draw [color=c, fill=c] (9.04478,9.28822) rectangle (9.08458,9.39407);
\draw [color=c, fill=c] (9.08458,9.28822) rectangle (9.12438,9.39407);
\draw [color=c, fill=c] (9.12438,9.28822) rectangle (9.16418,9.39407);
\draw [color=c, fill=c] (9.16418,9.28822) rectangle (9.20398,9.39407);
\draw [color=c, fill=c] (9.20398,9.28822) rectangle (9.24378,9.39407);
\draw [color=c, fill=c] (9.24378,9.28822) rectangle (9.28358,9.39407);
\draw [color=c, fill=c] (9.28358,9.28822) rectangle (9.32338,9.39407);
\draw [color=c, fill=c] (9.32338,9.28822) rectangle (9.36318,9.39407);
\draw [color=c, fill=c] (9.36318,9.28822) rectangle (9.40298,9.39407);
\draw [color=c, fill=c] (9.40298,9.28822) rectangle (9.44279,9.39407);
\draw [color=c, fill=c] (9.44279,9.28822) rectangle (9.48259,9.39407);
\definecolor{c}{rgb}{0,0.266667,1};
\draw [color=c, fill=c] (9.48259,9.28822) rectangle (9.52239,9.39407);
\draw [color=c, fill=c] (9.52239,9.28822) rectangle (9.56219,9.39407);
\draw [color=c, fill=c] (9.56219,9.28822) rectangle (9.60199,9.39407);
\draw [color=c, fill=c] (9.60199,9.28822) rectangle (9.64179,9.39407);
\draw [color=c, fill=c] (9.64179,9.28822) rectangle (9.68159,9.39407);
\draw [color=c, fill=c] (9.68159,9.28822) rectangle (9.72139,9.39407);
\draw [color=c, fill=c] (9.72139,9.28822) rectangle (9.76119,9.39407);
\draw [color=c, fill=c] (9.76119,9.28822) rectangle (9.80099,9.39407);
\draw [color=c, fill=c] (9.80099,9.28822) rectangle (9.8408,9.39407);
\draw [color=c, fill=c] (9.8408,9.28822) rectangle (9.8806,9.39407);
\draw [color=c, fill=c] (9.8806,9.28822) rectangle (9.9204,9.39407);
\draw [color=c, fill=c] (9.9204,9.28822) rectangle (9.9602,9.39407);
\draw [color=c, fill=c] (9.9602,9.28822) rectangle (10,9.39407);
\draw [color=c, fill=c] (10,9.28822) rectangle (10.0398,9.39407);
\draw [color=c, fill=c] (10.0398,9.28822) rectangle (10.0796,9.39407);
\draw [color=c, fill=c] (10.0796,9.28822) rectangle (10.1194,9.39407);
\draw [color=c, fill=c] (10.1194,9.28822) rectangle (10.1592,9.39407);
\draw [color=c, fill=c] (10.1592,9.28822) rectangle (10.199,9.39407);
\draw [color=c, fill=c] (10.199,9.28822) rectangle (10.2388,9.39407);
\draw [color=c, fill=c] (10.2388,9.28822) rectangle (10.2786,9.39407);
\draw [color=c, fill=c] (10.2786,9.28822) rectangle (10.3184,9.39407);
\draw [color=c, fill=c] (10.3184,9.28822) rectangle (10.3582,9.39407);
\draw [color=c, fill=c] (10.3582,9.28822) rectangle (10.398,9.39407);
\draw [color=c, fill=c] (10.398,9.28822) rectangle (10.4378,9.39407);
\draw [color=c, fill=c] (10.4378,9.28822) rectangle (10.4776,9.39407);
\draw [color=c, fill=c] (10.4776,9.28822) rectangle (10.5174,9.39407);
\draw [color=c, fill=c] (10.5174,9.28822) rectangle (10.5572,9.39407);
\draw [color=c, fill=c] (10.5572,9.28822) rectangle (10.597,9.39407);
\draw [color=c, fill=c] (10.597,9.28822) rectangle (10.6368,9.39407);
\definecolor{c}{rgb}{0,0.546666,1};
\draw [color=c, fill=c] (10.6368,9.28822) rectangle (10.6766,9.39407);
\draw [color=c, fill=c] (10.6766,9.28822) rectangle (10.7164,9.39407);
\draw [color=c, fill=c] (10.7164,9.28822) rectangle (10.7562,9.39407);
\draw [color=c, fill=c] (10.7562,9.28822) rectangle (10.796,9.39407);
\draw [color=c, fill=c] (10.796,9.28822) rectangle (10.8358,9.39407);
\draw [color=c, fill=c] (10.8358,9.28822) rectangle (10.8756,9.39407);
\draw [color=c, fill=c] (10.8756,9.28822) rectangle (10.9154,9.39407);
\draw [color=c, fill=c] (10.9154,9.28822) rectangle (10.9552,9.39407);
\draw [color=c, fill=c] (10.9552,9.28822) rectangle (10.995,9.39407);
\draw [color=c, fill=c] (10.995,9.28822) rectangle (11.0348,9.39407);
\draw [color=c, fill=c] (11.0348,9.28822) rectangle (11.0746,9.39407);
\draw [color=c, fill=c] (11.0746,9.28822) rectangle (11.1144,9.39407);
\draw [color=c, fill=c] (11.1144,9.28822) rectangle (11.1542,9.39407);
\draw [color=c, fill=c] (11.1542,9.28822) rectangle (11.194,9.39407);
\draw [color=c, fill=c] (11.194,9.28822) rectangle (11.2338,9.39407);
\draw [color=c, fill=c] (11.2338,9.28822) rectangle (11.2736,9.39407);
\draw [color=c, fill=c] (11.2736,9.28822) rectangle (11.3134,9.39407);
\draw [color=c, fill=c] (11.3134,9.28822) rectangle (11.3532,9.39407);
\draw [color=c, fill=c] (11.3532,9.28822) rectangle (11.393,9.39407);
\draw [color=c, fill=c] (11.393,9.28822) rectangle (11.4328,9.39407);
\draw [color=c, fill=c] (11.4328,9.28822) rectangle (11.4726,9.39407);
\draw [color=c, fill=c] (11.4726,9.28822) rectangle (11.5124,9.39407);
\draw [color=c, fill=c] (11.5124,9.28822) rectangle (11.5522,9.39407);
\draw [color=c, fill=c] (11.5522,9.28822) rectangle (11.592,9.39407);
\draw [color=c, fill=c] (11.592,9.28822) rectangle (11.6318,9.39407);
\draw [color=c, fill=c] (11.6318,9.28822) rectangle (11.6716,9.39407);
\draw [color=c, fill=c] (11.6716,9.28822) rectangle (11.7114,9.39407);
\draw [color=c, fill=c] (11.7114,9.28822) rectangle (11.7512,9.39407);
\draw [color=c, fill=c] (11.7512,9.28822) rectangle (11.791,9.39407);
\draw [color=c, fill=c] (11.791,9.28822) rectangle (11.8308,9.39407);
\draw [color=c, fill=c] (11.8308,9.28822) rectangle (11.8706,9.39407);
\draw [color=c, fill=c] (11.8706,9.28822) rectangle (11.9104,9.39407);
\draw [color=c, fill=c] (11.9104,9.28822) rectangle (11.9502,9.39407);
\draw [color=c, fill=c] (11.9502,9.28822) rectangle (11.99,9.39407);
\draw [color=c, fill=c] (11.99,9.28822) rectangle (12.0299,9.39407);
\draw [color=c, fill=c] (12.0299,9.28822) rectangle (12.0697,9.39407);
\draw [color=c, fill=c] (12.0697,9.28822) rectangle (12.1095,9.39407);
\draw [color=c, fill=c] (12.1095,9.28822) rectangle (12.1493,9.39407);
\draw [color=c, fill=c] (12.1493,9.28822) rectangle (12.1891,9.39407);
\draw [color=c, fill=c] (12.1891,9.28822) rectangle (12.2289,9.39407);
\draw [color=c, fill=c] (12.2289,9.28822) rectangle (12.2687,9.39407);
\draw [color=c, fill=c] (12.2687,9.28822) rectangle (12.3085,9.39407);
\draw [color=c, fill=c] (12.3085,9.28822) rectangle (12.3483,9.39407);
\draw [color=c, fill=c] (12.3483,9.28822) rectangle (12.3881,9.39407);
\draw [color=c, fill=c] (12.3881,9.28822) rectangle (12.4279,9.39407);
\draw [color=c, fill=c] (12.4279,9.28822) rectangle (12.4677,9.39407);
\draw [color=c, fill=c] (12.4677,9.28822) rectangle (12.5075,9.39407);
\draw [color=c, fill=c] (12.5075,9.28822) rectangle (12.5473,9.39407);
\draw [color=c, fill=c] (12.5473,9.28822) rectangle (12.5871,9.39407);
\draw [color=c, fill=c] (12.5871,9.28822) rectangle (12.6269,9.39407);
\draw [color=c, fill=c] (12.6269,9.28822) rectangle (12.6667,9.39407);
\draw [color=c, fill=c] (12.6667,9.28822) rectangle (12.7065,9.39407);
\draw [color=c, fill=c] (12.7065,9.28822) rectangle (12.7463,9.39407);
\draw [color=c, fill=c] (12.7463,9.28822) rectangle (12.7861,9.39407);
\draw [color=c, fill=c] (12.7861,9.28822) rectangle (12.8259,9.39407);
\draw [color=c, fill=c] (12.8259,9.28822) rectangle (12.8657,9.39407);
\draw [color=c, fill=c] (12.8657,9.28822) rectangle (12.9055,9.39407);
\draw [color=c, fill=c] (12.9055,9.28822) rectangle (12.9453,9.39407);
\draw [color=c, fill=c] (12.9453,9.28822) rectangle (12.9851,9.39407);
\draw [color=c, fill=c] (12.9851,9.28822) rectangle (13.0249,9.39407);
\draw [color=c, fill=c] (13.0249,9.28822) rectangle (13.0647,9.39407);
\draw [color=c, fill=c] (13.0647,9.28822) rectangle (13.1045,9.39407);
\draw [color=c, fill=c] (13.1045,9.28822) rectangle (13.1443,9.39407);
\draw [color=c, fill=c] (13.1443,9.28822) rectangle (13.1841,9.39407);
\draw [color=c, fill=c] (13.1841,9.28822) rectangle (13.2239,9.39407);
\draw [color=c, fill=c] (13.2239,9.28822) rectangle (13.2637,9.39407);
\draw [color=c, fill=c] (13.2637,9.28822) rectangle (13.3035,9.39407);
\draw [color=c, fill=c] (13.3035,9.28822) rectangle (13.3433,9.39407);
\draw [color=c, fill=c] (13.3433,9.28822) rectangle (13.3831,9.39407);
\definecolor{c}{rgb}{0,0.733333,1};
\draw [color=c, fill=c] (13.3831,9.28822) rectangle (13.4229,9.39407);
\draw [color=c, fill=c] (13.4229,9.28822) rectangle (13.4627,9.39407);
\draw [color=c, fill=c] (13.4627,9.28822) rectangle (13.5025,9.39407);
\draw [color=c, fill=c] (13.5025,9.28822) rectangle (13.5423,9.39407);
\draw [color=c, fill=c] (13.5423,9.28822) rectangle (13.5821,9.39407);
\draw [color=c, fill=c] (13.5821,9.28822) rectangle (13.6219,9.39407);
\draw [color=c, fill=c] (13.6219,9.28822) rectangle (13.6617,9.39407);
\draw [color=c, fill=c] (13.6617,9.28822) rectangle (13.7015,9.39407);
\draw [color=c, fill=c] (13.7015,9.28822) rectangle (13.7413,9.39407);
\draw [color=c, fill=c] (13.7413,9.28822) rectangle (13.7811,9.39407);
\draw [color=c, fill=c] (13.7811,9.28822) rectangle (13.8209,9.39407);
\draw [color=c, fill=c] (13.8209,9.28822) rectangle (13.8607,9.39407);
\draw [color=c, fill=c] (13.8607,9.28822) rectangle (13.9005,9.39407);
\draw [color=c, fill=c] (13.9005,9.28822) rectangle (13.9403,9.39407);
\draw [color=c, fill=c] (13.9403,9.28822) rectangle (13.9801,9.39407);
\draw [color=c, fill=c] (13.9801,9.28822) rectangle (14.0199,9.39407);
\draw [color=c, fill=c] (14.0199,9.28822) rectangle (14.0597,9.39407);
\draw [color=c, fill=c] (14.0597,9.28822) rectangle (14.0995,9.39407);
\draw [color=c, fill=c] (14.0995,9.28822) rectangle (14.1393,9.39407);
\draw [color=c, fill=c] (14.1393,9.28822) rectangle (14.1791,9.39407);
\draw [color=c, fill=c] (14.1791,9.28822) rectangle (14.2189,9.39407);
\draw [color=c, fill=c] (14.2189,9.28822) rectangle (14.2587,9.39407);
\draw [color=c, fill=c] (14.2587,9.28822) rectangle (14.2985,9.39407);
\draw [color=c, fill=c] (14.2985,9.28822) rectangle (14.3383,9.39407);
\draw [color=c, fill=c] (14.3383,9.28822) rectangle (14.3781,9.39407);
\draw [color=c, fill=c] (14.3781,9.28822) rectangle (14.4179,9.39407);
\draw [color=c, fill=c] (14.4179,9.28822) rectangle (14.4577,9.39407);
\draw [color=c, fill=c] (14.4577,9.28822) rectangle (14.4975,9.39407);
\draw [color=c, fill=c] (14.4975,9.28822) rectangle (14.5373,9.39407);
\draw [color=c, fill=c] (14.5373,9.28822) rectangle (14.5771,9.39407);
\draw [color=c, fill=c] (14.5771,9.28822) rectangle (14.6169,9.39407);
\draw [color=c, fill=c] (14.6169,9.28822) rectangle (14.6567,9.39407);
\draw [color=c, fill=c] (14.6567,9.28822) rectangle (14.6965,9.39407);
\draw [color=c, fill=c] (14.6965,9.28822) rectangle (14.7363,9.39407);
\draw [color=c, fill=c] (14.7363,9.28822) rectangle (14.7761,9.39407);
\draw [color=c, fill=c] (14.7761,9.28822) rectangle (14.8159,9.39407);
\draw [color=c, fill=c] (14.8159,9.28822) rectangle (14.8557,9.39407);
\draw [color=c, fill=c] (14.8557,9.28822) rectangle (14.8955,9.39407);
\draw [color=c, fill=c] (14.8955,9.28822) rectangle (14.9353,9.39407);
\draw [color=c, fill=c] (14.9353,9.28822) rectangle (14.9751,9.39407);
\draw [color=c, fill=c] (14.9751,9.28822) rectangle (15.0149,9.39407);
\draw [color=c, fill=c] (15.0149,9.28822) rectangle (15.0547,9.39407);
\draw [color=c, fill=c] (15.0547,9.28822) rectangle (15.0945,9.39407);
\draw [color=c, fill=c] (15.0945,9.28822) rectangle (15.1343,9.39407);
\draw [color=c, fill=c] (15.1343,9.28822) rectangle (15.1741,9.39407);
\draw [color=c, fill=c] (15.1741,9.28822) rectangle (15.2139,9.39407);
\draw [color=c, fill=c] (15.2139,9.28822) rectangle (15.2537,9.39407);
\draw [color=c, fill=c] (15.2537,9.28822) rectangle (15.2935,9.39407);
\draw [color=c, fill=c] (15.2935,9.28822) rectangle (15.3333,9.39407);
\draw [color=c, fill=c] (15.3333,9.28822) rectangle (15.3731,9.39407);
\draw [color=c, fill=c] (15.3731,9.28822) rectangle (15.4129,9.39407);
\draw [color=c, fill=c] (15.4129,9.28822) rectangle (15.4527,9.39407);
\draw [color=c, fill=c] (15.4527,9.28822) rectangle (15.4925,9.39407);
\draw [color=c, fill=c] (15.4925,9.28822) rectangle (15.5323,9.39407);
\draw [color=c, fill=c] (15.5323,9.28822) rectangle (15.5721,9.39407);
\draw [color=c, fill=c] (15.5721,9.28822) rectangle (15.6119,9.39407);
\draw [color=c, fill=c] (15.6119,9.28822) rectangle (15.6517,9.39407);
\draw [color=c, fill=c] (15.6517,9.28822) rectangle (15.6915,9.39407);
\draw [color=c, fill=c] (15.6915,9.28822) rectangle (15.7313,9.39407);
\draw [color=c, fill=c] (15.7313,9.28822) rectangle (15.7711,9.39407);
\draw [color=c, fill=c] (15.7711,9.28822) rectangle (15.8109,9.39407);
\draw [color=c, fill=c] (15.8109,9.28822) rectangle (15.8507,9.39407);
\draw [color=c, fill=c] (15.8507,9.28822) rectangle (15.8905,9.39407);
\draw [color=c, fill=c] (15.8905,9.28822) rectangle (15.9303,9.39407);
\draw [color=c, fill=c] (15.9303,9.28822) rectangle (15.9701,9.39407);
\draw [color=c, fill=c] (15.9701,9.28822) rectangle (16.01,9.39407);
\draw [color=c, fill=c] (16.01,9.28822) rectangle (16.0498,9.39407);
\draw [color=c, fill=c] (16.0498,9.28822) rectangle (16.0896,9.39407);
\draw [color=c, fill=c] (16.0896,9.28822) rectangle (16.1294,9.39407);
\draw [color=c, fill=c] (16.1294,9.28822) rectangle (16.1692,9.39407);
\draw [color=c, fill=c] (16.1692,9.28822) rectangle (16.209,9.39407);
\draw [color=c, fill=c] (16.209,9.28822) rectangle (16.2488,9.39407);
\draw [color=c, fill=c] (16.2488,9.28822) rectangle (16.2886,9.39407);
\draw [color=c, fill=c] (16.2886,9.28822) rectangle (16.3284,9.39407);
\draw [color=c, fill=c] (16.3284,9.28822) rectangle (16.3682,9.39407);
\draw [color=c, fill=c] (16.3682,9.28822) rectangle (16.408,9.39407);
\draw [color=c, fill=c] (16.408,9.28822) rectangle (16.4478,9.39407);
\draw [color=c, fill=c] (16.4478,9.28822) rectangle (16.4876,9.39407);
\draw [color=c, fill=c] (16.4876,9.28822) rectangle (16.5274,9.39407);
\draw [color=c, fill=c] (16.5274,9.28822) rectangle (16.5672,9.39407);
\draw [color=c, fill=c] (16.5672,9.28822) rectangle (16.607,9.39407);
\draw [color=c, fill=c] (16.607,9.28822) rectangle (16.6468,9.39407);
\draw [color=c, fill=c] (16.6468,9.28822) rectangle (16.6866,9.39407);
\draw [color=c, fill=c] (16.6866,9.28822) rectangle (16.7264,9.39407);
\draw [color=c, fill=c] (16.7264,9.28822) rectangle (16.7662,9.39407);
\draw [color=c, fill=c] (16.7662,9.28822) rectangle (16.806,9.39407);
\draw [color=c, fill=c] (16.806,9.28822) rectangle (16.8458,9.39407);
\draw [color=c, fill=c] (16.8458,9.28822) rectangle (16.8856,9.39407);
\draw [color=c, fill=c] (16.8856,9.28822) rectangle (16.9254,9.39407);
\draw [color=c, fill=c] (16.9254,9.28822) rectangle (16.9652,9.39407);
\draw [color=c, fill=c] (16.9652,9.28822) rectangle (17.005,9.39407);
\draw [color=c, fill=c] (17.005,9.28822) rectangle (17.0448,9.39407);
\draw [color=c, fill=c] (17.0448,9.28822) rectangle (17.0846,9.39407);
\draw [color=c, fill=c] (17.0846,9.28822) rectangle (17.1244,9.39407);
\draw [color=c, fill=c] (17.1244,9.28822) rectangle (17.1642,9.39407);
\draw [color=c, fill=c] (17.1642,9.28822) rectangle (17.204,9.39407);
\draw [color=c, fill=c] (17.204,9.28822) rectangle (17.2438,9.39407);
\draw [color=c, fill=c] (17.2438,9.28822) rectangle (17.2836,9.39407);
\draw [color=c, fill=c] (17.2836,9.28822) rectangle (17.3234,9.39407);
\draw [color=c, fill=c] (17.3234,9.28822) rectangle (17.3632,9.39407);
\draw [color=c, fill=c] (17.3632,9.28822) rectangle (17.403,9.39407);
\draw [color=c, fill=c] (17.403,9.28822) rectangle (17.4428,9.39407);
\draw [color=c, fill=c] (17.4428,9.28822) rectangle (17.4826,9.39407);
\draw [color=c, fill=c] (17.4826,9.28822) rectangle (17.5224,9.39407);
\draw [color=c, fill=c] (17.5224,9.28822) rectangle (17.5622,9.39407);
\draw [color=c, fill=c] (17.5622,9.28822) rectangle (17.602,9.39407);
\draw [color=c, fill=c] (17.602,9.28822) rectangle (17.6418,9.39407);
\draw [color=c, fill=c] (17.6418,9.28822) rectangle (17.6816,9.39407);
\draw [color=c, fill=c] (17.6816,9.28822) rectangle (17.7214,9.39407);
\draw [color=c, fill=c] (17.7214,9.28822) rectangle (17.7612,9.39407);
\draw [color=c, fill=c] (17.7612,9.28822) rectangle (17.801,9.39407);
\draw [color=c, fill=c] (17.801,9.28822) rectangle (17.8408,9.39407);
\draw [color=c, fill=c] (17.8408,9.28822) rectangle (17.8806,9.39407);
\draw [color=c, fill=c] (17.8806,9.28822) rectangle (17.9204,9.39407);
\draw [color=c, fill=c] (17.9204,9.28822) rectangle (17.9602,9.39407);
\draw [color=c, fill=c] (17.9602,9.28822) rectangle (18,9.39407);
\definecolor{c}{rgb}{0.2,0,1};
\draw [color=c, fill=c] (2,9.39407) rectangle (2.0398,9.49992);
\draw [color=c, fill=c] (2.0398,9.39407) rectangle (2.0796,9.49992);
\draw [color=c, fill=c] (2.0796,9.39407) rectangle (2.1194,9.49992);
\draw [color=c, fill=c] (2.1194,9.39407) rectangle (2.1592,9.49992);
\draw [color=c, fill=c] (2.1592,9.39407) rectangle (2.19901,9.49992);
\draw [color=c, fill=c] (2.19901,9.39407) rectangle (2.23881,9.49992);
\draw [color=c, fill=c] (2.23881,9.39407) rectangle (2.27861,9.49992);
\draw [color=c, fill=c] (2.27861,9.39407) rectangle (2.31841,9.49992);
\draw [color=c, fill=c] (2.31841,9.39407) rectangle (2.35821,9.49992);
\draw [color=c, fill=c] (2.35821,9.39407) rectangle (2.39801,9.49992);
\draw [color=c, fill=c] (2.39801,9.39407) rectangle (2.43781,9.49992);
\draw [color=c, fill=c] (2.43781,9.39407) rectangle (2.47761,9.49992);
\draw [color=c, fill=c] (2.47761,9.39407) rectangle (2.51741,9.49992);
\draw [color=c, fill=c] (2.51741,9.39407) rectangle (2.55721,9.49992);
\draw [color=c, fill=c] (2.55721,9.39407) rectangle (2.59702,9.49992);
\draw [color=c, fill=c] (2.59702,9.39407) rectangle (2.63682,9.49992);
\draw [color=c, fill=c] (2.63682,9.39407) rectangle (2.67662,9.49992);
\draw [color=c, fill=c] (2.67662,9.39407) rectangle (2.71642,9.49992);
\draw [color=c, fill=c] (2.71642,9.39407) rectangle (2.75622,9.49992);
\draw [color=c, fill=c] (2.75622,9.39407) rectangle (2.79602,9.49992);
\draw [color=c, fill=c] (2.79602,9.39407) rectangle (2.83582,9.49992);
\draw [color=c, fill=c] (2.83582,9.39407) rectangle (2.87562,9.49992);
\draw [color=c, fill=c] (2.87562,9.39407) rectangle (2.91542,9.49992);
\draw [color=c, fill=c] (2.91542,9.39407) rectangle (2.95522,9.49992);
\draw [color=c, fill=c] (2.95522,9.39407) rectangle (2.99502,9.49992);
\draw [color=c, fill=c] (2.99502,9.39407) rectangle (3.03483,9.49992);
\draw [color=c, fill=c] (3.03483,9.39407) rectangle (3.07463,9.49992);
\draw [color=c, fill=c] (3.07463,9.39407) rectangle (3.11443,9.49992);
\draw [color=c, fill=c] (3.11443,9.39407) rectangle (3.15423,9.49992);
\draw [color=c, fill=c] (3.15423,9.39407) rectangle (3.19403,9.49992);
\draw [color=c, fill=c] (3.19403,9.39407) rectangle (3.23383,9.49992);
\draw [color=c, fill=c] (3.23383,9.39407) rectangle (3.27363,9.49992);
\draw [color=c, fill=c] (3.27363,9.39407) rectangle (3.31343,9.49992);
\draw [color=c, fill=c] (3.31343,9.39407) rectangle (3.35323,9.49992);
\draw [color=c, fill=c] (3.35323,9.39407) rectangle (3.39303,9.49992);
\draw [color=c, fill=c] (3.39303,9.39407) rectangle (3.43284,9.49992);
\draw [color=c, fill=c] (3.43284,9.39407) rectangle (3.47264,9.49992);
\draw [color=c, fill=c] (3.47264,9.39407) rectangle (3.51244,9.49992);
\draw [color=c, fill=c] (3.51244,9.39407) rectangle (3.55224,9.49992);
\draw [color=c, fill=c] (3.55224,9.39407) rectangle (3.59204,9.49992);
\draw [color=c, fill=c] (3.59204,9.39407) rectangle (3.63184,9.49992);
\draw [color=c, fill=c] (3.63184,9.39407) rectangle (3.67164,9.49992);
\draw [color=c, fill=c] (3.67164,9.39407) rectangle (3.71144,9.49992);
\draw [color=c, fill=c] (3.71144,9.39407) rectangle (3.75124,9.49992);
\draw [color=c, fill=c] (3.75124,9.39407) rectangle (3.79104,9.49992);
\draw [color=c, fill=c] (3.79104,9.39407) rectangle (3.83085,9.49992);
\draw [color=c, fill=c] (3.83085,9.39407) rectangle (3.87065,9.49992);
\draw [color=c, fill=c] (3.87065,9.39407) rectangle (3.91045,9.49992);
\draw [color=c, fill=c] (3.91045,9.39407) rectangle (3.95025,9.49992);
\draw [color=c, fill=c] (3.95025,9.39407) rectangle (3.99005,9.49992);
\draw [color=c, fill=c] (3.99005,9.39407) rectangle (4.02985,9.49992);
\draw [color=c, fill=c] (4.02985,9.39407) rectangle (4.06965,9.49992);
\draw [color=c, fill=c] (4.06965,9.39407) rectangle (4.10945,9.49992);
\draw [color=c, fill=c] (4.10945,9.39407) rectangle (4.14925,9.49992);
\draw [color=c, fill=c] (4.14925,9.39407) rectangle (4.18905,9.49992);
\draw [color=c, fill=c] (4.18905,9.39407) rectangle (4.22886,9.49992);
\draw [color=c, fill=c] (4.22886,9.39407) rectangle (4.26866,9.49992);
\draw [color=c, fill=c] (4.26866,9.39407) rectangle (4.30846,9.49992);
\draw [color=c, fill=c] (4.30846,9.39407) rectangle (4.34826,9.49992);
\draw [color=c, fill=c] (4.34826,9.39407) rectangle (4.38806,9.49992);
\draw [color=c, fill=c] (4.38806,9.39407) rectangle (4.42786,9.49992);
\draw [color=c, fill=c] (4.42786,9.39407) rectangle (4.46766,9.49992);
\draw [color=c, fill=c] (4.46766,9.39407) rectangle (4.50746,9.49992);
\draw [color=c, fill=c] (4.50746,9.39407) rectangle (4.54726,9.49992);
\draw [color=c, fill=c] (4.54726,9.39407) rectangle (4.58706,9.49992);
\draw [color=c, fill=c] (4.58706,9.39407) rectangle (4.62687,9.49992);
\draw [color=c, fill=c] (4.62687,9.39407) rectangle (4.66667,9.49992);
\draw [color=c, fill=c] (4.66667,9.39407) rectangle (4.70647,9.49992);
\draw [color=c, fill=c] (4.70647,9.39407) rectangle (4.74627,9.49992);
\draw [color=c, fill=c] (4.74627,9.39407) rectangle (4.78607,9.49992);
\draw [color=c, fill=c] (4.78607,9.39407) rectangle (4.82587,9.49992);
\draw [color=c, fill=c] (4.82587,9.39407) rectangle (4.86567,9.49992);
\draw [color=c, fill=c] (4.86567,9.39407) rectangle (4.90547,9.49992);
\draw [color=c, fill=c] (4.90547,9.39407) rectangle (4.94527,9.49992);
\draw [color=c, fill=c] (4.94527,9.39407) rectangle (4.98507,9.49992);
\draw [color=c, fill=c] (4.98507,9.39407) rectangle (5.02488,9.49992);
\draw [color=c, fill=c] (5.02488,9.39407) rectangle (5.06468,9.49992);
\draw [color=c, fill=c] (5.06468,9.39407) rectangle (5.10448,9.49992);
\draw [color=c, fill=c] (5.10448,9.39407) rectangle (5.14428,9.49992);
\draw [color=c, fill=c] (5.14428,9.39407) rectangle (5.18408,9.49992);
\draw [color=c, fill=c] (5.18408,9.39407) rectangle (5.22388,9.49992);
\draw [color=c, fill=c] (5.22388,9.39407) rectangle (5.26368,9.49992);
\draw [color=c, fill=c] (5.26368,9.39407) rectangle (5.30348,9.49992);
\draw [color=c, fill=c] (5.30348,9.39407) rectangle (5.34328,9.49992);
\draw [color=c, fill=c] (5.34328,9.39407) rectangle (5.38308,9.49992);
\draw [color=c, fill=c] (5.38308,9.39407) rectangle (5.42289,9.49992);
\draw [color=c, fill=c] (5.42289,9.39407) rectangle (5.46269,9.49992);
\draw [color=c, fill=c] (5.46269,9.39407) rectangle (5.50249,9.49992);
\draw [color=c, fill=c] (5.50249,9.39407) rectangle (5.54229,9.49992);
\draw [color=c, fill=c] (5.54229,9.39407) rectangle (5.58209,9.49992);
\draw [color=c, fill=c] (5.58209,9.39407) rectangle (5.62189,9.49992);
\draw [color=c, fill=c] (5.62189,9.39407) rectangle (5.66169,9.49992);
\draw [color=c, fill=c] (5.66169,9.39407) rectangle (5.70149,9.49992);
\draw [color=c, fill=c] (5.70149,9.39407) rectangle (5.74129,9.49992);
\draw [color=c, fill=c] (5.74129,9.39407) rectangle (5.78109,9.49992);
\draw [color=c, fill=c] (5.78109,9.39407) rectangle (5.8209,9.49992);
\draw [color=c, fill=c] (5.8209,9.39407) rectangle (5.8607,9.49992);
\draw [color=c, fill=c] (5.8607,9.39407) rectangle (5.9005,9.49992);
\draw [color=c, fill=c] (5.9005,9.39407) rectangle (5.9403,9.49992);
\draw [color=c, fill=c] (5.9403,9.39407) rectangle (5.9801,9.49992);
\draw [color=c, fill=c] (5.9801,9.39407) rectangle (6.0199,9.49992);
\draw [color=c, fill=c] (6.0199,9.39407) rectangle (6.0597,9.49992);
\draw [color=c, fill=c] (6.0597,9.39407) rectangle (6.0995,9.49992);
\draw [color=c, fill=c] (6.0995,9.39407) rectangle (6.1393,9.49992);
\draw [color=c, fill=c] (6.1393,9.39407) rectangle (6.1791,9.49992);
\draw [color=c, fill=c] (6.1791,9.39407) rectangle (6.21891,9.49992);
\draw [color=c, fill=c] (6.21891,9.39407) rectangle (6.25871,9.49992);
\draw [color=c, fill=c] (6.25871,9.39407) rectangle (6.29851,9.49992);
\draw [color=c, fill=c] (6.29851,9.39407) rectangle (6.33831,9.49992);
\draw [color=c, fill=c] (6.33831,9.39407) rectangle (6.37811,9.49992);
\draw [color=c, fill=c] (6.37811,9.39407) rectangle (6.41791,9.49992);
\draw [color=c, fill=c] (6.41791,9.39407) rectangle (6.45771,9.49992);
\draw [color=c, fill=c] (6.45771,9.39407) rectangle (6.49751,9.49992);
\draw [color=c, fill=c] (6.49751,9.39407) rectangle (6.53731,9.49992);
\draw [color=c, fill=c] (6.53731,9.39407) rectangle (6.57711,9.49992);
\draw [color=c, fill=c] (6.57711,9.39407) rectangle (6.61692,9.49992);
\draw [color=c, fill=c] (6.61692,9.39407) rectangle (6.65672,9.49992);
\draw [color=c, fill=c] (6.65672,9.39407) rectangle (6.69652,9.49992);
\draw [color=c, fill=c] (6.69652,9.39407) rectangle (6.73632,9.49992);
\draw [color=c, fill=c] (6.73632,9.39407) rectangle (6.77612,9.49992);
\draw [color=c, fill=c] (6.77612,9.39407) rectangle (6.81592,9.49992);
\draw [color=c, fill=c] (6.81592,9.39407) rectangle (6.85572,9.49992);
\draw [color=c, fill=c] (6.85572,9.39407) rectangle (6.89552,9.49992);
\draw [color=c, fill=c] (6.89552,9.39407) rectangle (6.93532,9.49992);
\draw [color=c, fill=c] (6.93532,9.39407) rectangle (6.97512,9.49992);
\draw [color=c, fill=c] (6.97512,9.39407) rectangle (7.01493,9.49992);
\draw [color=c, fill=c] (7.01493,9.39407) rectangle (7.05473,9.49992);
\draw [color=c, fill=c] (7.05473,9.39407) rectangle (7.09453,9.49992);
\draw [color=c, fill=c] (7.09453,9.39407) rectangle (7.13433,9.49992);
\draw [color=c, fill=c] (7.13433,9.39407) rectangle (7.17413,9.49992);
\draw [color=c, fill=c] (7.17413,9.39407) rectangle (7.21393,9.49992);
\draw [color=c, fill=c] (7.21393,9.39407) rectangle (7.25373,9.49992);
\draw [color=c, fill=c] (7.25373,9.39407) rectangle (7.29353,9.49992);
\draw [color=c, fill=c] (7.29353,9.39407) rectangle (7.33333,9.49992);
\draw [color=c, fill=c] (7.33333,9.39407) rectangle (7.37313,9.49992);
\draw [color=c, fill=c] (7.37313,9.39407) rectangle (7.41294,9.49992);
\draw [color=c, fill=c] (7.41294,9.39407) rectangle (7.45274,9.49992);
\draw [color=c, fill=c] (7.45274,9.39407) rectangle (7.49254,9.49992);
\draw [color=c, fill=c] (7.49254,9.39407) rectangle (7.53234,9.49992);
\draw [color=c, fill=c] (7.53234,9.39407) rectangle (7.57214,9.49992);
\draw [color=c, fill=c] (7.57214,9.39407) rectangle (7.61194,9.49992);
\draw [color=c, fill=c] (7.61194,9.39407) rectangle (7.65174,9.49992);
\definecolor{c}{rgb}{0,0.0800001,1};
\draw [color=c, fill=c] (7.65174,9.39407) rectangle (7.69154,9.49992);
\draw [color=c, fill=c] (7.69154,9.39407) rectangle (7.73134,9.49992);
\draw [color=c, fill=c] (7.73134,9.39407) rectangle (7.77114,9.49992);
\draw [color=c, fill=c] (7.77114,9.39407) rectangle (7.81095,9.49992);
\draw [color=c, fill=c] (7.81095,9.39407) rectangle (7.85075,9.49992);
\draw [color=c, fill=c] (7.85075,9.39407) rectangle (7.89055,9.49992);
\draw [color=c, fill=c] (7.89055,9.39407) rectangle (7.93035,9.49992);
\draw [color=c, fill=c] (7.93035,9.39407) rectangle (7.97015,9.49992);
\draw [color=c, fill=c] (7.97015,9.39407) rectangle (8.00995,9.49992);
\draw [color=c, fill=c] (8.00995,9.39407) rectangle (8.04975,9.49992);
\draw [color=c, fill=c] (8.04975,9.39407) rectangle (8.08955,9.49992);
\draw [color=c, fill=c] (8.08955,9.39407) rectangle (8.12935,9.49992);
\draw [color=c, fill=c] (8.12935,9.39407) rectangle (8.16915,9.49992);
\draw [color=c, fill=c] (8.16915,9.39407) rectangle (8.20895,9.49992);
\draw [color=c, fill=c] (8.20895,9.39407) rectangle (8.24876,9.49992);
\draw [color=c, fill=c] (8.24876,9.39407) rectangle (8.28856,9.49992);
\draw [color=c, fill=c] (8.28856,9.39407) rectangle (8.32836,9.49992);
\draw [color=c, fill=c] (8.32836,9.39407) rectangle (8.36816,9.49992);
\draw [color=c, fill=c] (8.36816,9.39407) rectangle (8.40796,9.49992);
\draw [color=c, fill=c] (8.40796,9.39407) rectangle (8.44776,9.49992);
\draw [color=c, fill=c] (8.44776,9.39407) rectangle (8.48756,9.49992);
\draw [color=c, fill=c] (8.48756,9.39407) rectangle (8.52736,9.49992);
\draw [color=c, fill=c] (8.52736,9.39407) rectangle (8.56716,9.49992);
\draw [color=c, fill=c] (8.56716,9.39407) rectangle (8.60697,9.49992);
\draw [color=c, fill=c] (8.60697,9.39407) rectangle (8.64677,9.49992);
\draw [color=c, fill=c] (8.64677,9.39407) rectangle (8.68657,9.49992);
\draw [color=c, fill=c] (8.68657,9.39407) rectangle (8.72637,9.49992);
\draw [color=c, fill=c] (8.72637,9.39407) rectangle (8.76617,9.49992);
\draw [color=c, fill=c] (8.76617,9.39407) rectangle (8.80597,9.49992);
\draw [color=c, fill=c] (8.80597,9.39407) rectangle (8.84577,9.49992);
\draw [color=c, fill=c] (8.84577,9.39407) rectangle (8.88557,9.49992);
\draw [color=c, fill=c] (8.88557,9.39407) rectangle (8.92537,9.49992);
\draw [color=c, fill=c] (8.92537,9.39407) rectangle (8.96517,9.49992);
\draw [color=c, fill=c] (8.96517,9.39407) rectangle (9.00498,9.49992);
\draw [color=c, fill=c] (9.00498,9.39407) rectangle (9.04478,9.49992);
\draw [color=c, fill=c] (9.04478,9.39407) rectangle (9.08458,9.49992);
\draw [color=c, fill=c] (9.08458,9.39407) rectangle (9.12438,9.49992);
\draw [color=c, fill=c] (9.12438,9.39407) rectangle (9.16418,9.49992);
\draw [color=c, fill=c] (9.16418,9.39407) rectangle (9.20398,9.49992);
\draw [color=c, fill=c] (9.20398,9.39407) rectangle (9.24378,9.49992);
\draw [color=c, fill=c] (9.24378,9.39407) rectangle (9.28358,9.49992);
\draw [color=c, fill=c] (9.28358,9.39407) rectangle (9.32338,9.49992);
\draw [color=c, fill=c] (9.32338,9.39407) rectangle (9.36318,9.49992);
\draw [color=c, fill=c] (9.36318,9.39407) rectangle (9.40298,9.49992);
\draw [color=c, fill=c] (9.40298,9.39407) rectangle (9.44279,9.49992);
\draw [color=c, fill=c] (9.44279,9.39407) rectangle (9.48259,9.49992);
\definecolor{c}{rgb}{0,0.266667,1};
\draw [color=c, fill=c] (9.48259,9.39407) rectangle (9.52239,9.49992);
\draw [color=c, fill=c] (9.52239,9.39407) rectangle (9.56219,9.49992);
\draw [color=c, fill=c] (9.56219,9.39407) rectangle (9.60199,9.49992);
\draw [color=c, fill=c] (9.60199,9.39407) rectangle (9.64179,9.49992);
\draw [color=c, fill=c] (9.64179,9.39407) rectangle (9.68159,9.49992);
\draw [color=c, fill=c] (9.68159,9.39407) rectangle (9.72139,9.49992);
\draw [color=c, fill=c] (9.72139,9.39407) rectangle (9.76119,9.49992);
\draw [color=c, fill=c] (9.76119,9.39407) rectangle (9.80099,9.49992);
\draw [color=c, fill=c] (9.80099,9.39407) rectangle (9.8408,9.49992);
\draw [color=c, fill=c] (9.8408,9.39407) rectangle (9.8806,9.49992);
\draw [color=c, fill=c] (9.8806,9.39407) rectangle (9.9204,9.49992);
\draw [color=c, fill=c] (9.9204,9.39407) rectangle (9.9602,9.49992);
\draw [color=c, fill=c] (9.9602,9.39407) rectangle (10,9.49992);
\draw [color=c, fill=c] (10,9.39407) rectangle (10.0398,9.49992);
\draw [color=c, fill=c] (10.0398,9.39407) rectangle (10.0796,9.49992);
\draw [color=c, fill=c] (10.0796,9.39407) rectangle (10.1194,9.49992);
\draw [color=c, fill=c] (10.1194,9.39407) rectangle (10.1592,9.49992);
\draw [color=c, fill=c] (10.1592,9.39407) rectangle (10.199,9.49992);
\draw [color=c, fill=c] (10.199,9.39407) rectangle (10.2388,9.49992);
\draw [color=c, fill=c] (10.2388,9.39407) rectangle (10.2786,9.49992);
\draw [color=c, fill=c] (10.2786,9.39407) rectangle (10.3184,9.49992);
\draw [color=c, fill=c] (10.3184,9.39407) rectangle (10.3582,9.49992);
\draw [color=c, fill=c] (10.3582,9.39407) rectangle (10.398,9.49992);
\draw [color=c, fill=c] (10.398,9.39407) rectangle (10.4378,9.49992);
\draw [color=c, fill=c] (10.4378,9.39407) rectangle (10.4776,9.49992);
\draw [color=c, fill=c] (10.4776,9.39407) rectangle (10.5174,9.49992);
\draw [color=c, fill=c] (10.5174,9.39407) rectangle (10.5572,9.49992);
\draw [color=c, fill=c] (10.5572,9.39407) rectangle (10.597,9.49992);
\draw [color=c, fill=c] (10.597,9.39407) rectangle (10.6368,9.49992);
\draw [color=c, fill=c] (10.6368,9.39407) rectangle (10.6766,9.49992);
\definecolor{c}{rgb}{0,0.546666,1};
\draw [color=c, fill=c] (10.6766,9.39407) rectangle (10.7164,9.49992);
\draw [color=c, fill=c] (10.7164,9.39407) rectangle (10.7562,9.49992);
\draw [color=c, fill=c] (10.7562,9.39407) rectangle (10.796,9.49992);
\draw [color=c, fill=c] (10.796,9.39407) rectangle (10.8358,9.49992);
\draw [color=c, fill=c] (10.8358,9.39407) rectangle (10.8756,9.49992);
\draw [color=c, fill=c] (10.8756,9.39407) rectangle (10.9154,9.49992);
\draw [color=c, fill=c] (10.9154,9.39407) rectangle (10.9552,9.49992);
\draw [color=c, fill=c] (10.9552,9.39407) rectangle (10.995,9.49992);
\draw [color=c, fill=c] (10.995,9.39407) rectangle (11.0348,9.49992);
\draw [color=c, fill=c] (11.0348,9.39407) rectangle (11.0746,9.49992);
\draw [color=c, fill=c] (11.0746,9.39407) rectangle (11.1144,9.49992);
\draw [color=c, fill=c] (11.1144,9.39407) rectangle (11.1542,9.49992);
\draw [color=c, fill=c] (11.1542,9.39407) rectangle (11.194,9.49992);
\draw [color=c, fill=c] (11.194,9.39407) rectangle (11.2338,9.49992);
\draw [color=c, fill=c] (11.2338,9.39407) rectangle (11.2736,9.49992);
\draw [color=c, fill=c] (11.2736,9.39407) rectangle (11.3134,9.49992);
\draw [color=c, fill=c] (11.3134,9.39407) rectangle (11.3532,9.49992);
\draw [color=c, fill=c] (11.3532,9.39407) rectangle (11.393,9.49992);
\draw [color=c, fill=c] (11.393,9.39407) rectangle (11.4328,9.49992);
\draw [color=c, fill=c] (11.4328,9.39407) rectangle (11.4726,9.49992);
\draw [color=c, fill=c] (11.4726,9.39407) rectangle (11.5124,9.49992);
\draw [color=c, fill=c] (11.5124,9.39407) rectangle (11.5522,9.49992);
\draw [color=c, fill=c] (11.5522,9.39407) rectangle (11.592,9.49992);
\draw [color=c, fill=c] (11.592,9.39407) rectangle (11.6318,9.49992);
\draw [color=c, fill=c] (11.6318,9.39407) rectangle (11.6716,9.49992);
\draw [color=c, fill=c] (11.6716,9.39407) rectangle (11.7114,9.49992);
\draw [color=c, fill=c] (11.7114,9.39407) rectangle (11.7512,9.49992);
\draw [color=c, fill=c] (11.7512,9.39407) rectangle (11.791,9.49992);
\draw [color=c, fill=c] (11.791,9.39407) rectangle (11.8308,9.49992);
\draw [color=c, fill=c] (11.8308,9.39407) rectangle (11.8706,9.49992);
\draw [color=c, fill=c] (11.8706,9.39407) rectangle (11.9104,9.49992);
\draw [color=c, fill=c] (11.9104,9.39407) rectangle (11.9502,9.49992);
\draw [color=c, fill=c] (11.9502,9.39407) rectangle (11.99,9.49992);
\draw [color=c, fill=c] (11.99,9.39407) rectangle (12.0299,9.49992);
\draw [color=c, fill=c] (12.0299,9.39407) rectangle (12.0697,9.49992);
\draw [color=c, fill=c] (12.0697,9.39407) rectangle (12.1095,9.49992);
\draw [color=c, fill=c] (12.1095,9.39407) rectangle (12.1493,9.49992);
\draw [color=c, fill=c] (12.1493,9.39407) rectangle (12.1891,9.49992);
\draw [color=c, fill=c] (12.1891,9.39407) rectangle (12.2289,9.49992);
\draw [color=c, fill=c] (12.2289,9.39407) rectangle (12.2687,9.49992);
\draw [color=c, fill=c] (12.2687,9.39407) rectangle (12.3085,9.49992);
\draw [color=c, fill=c] (12.3085,9.39407) rectangle (12.3483,9.49992);
\draw [color=c, fill=c] (12.3483,9.39407) rectangle (12.3881,9.49992);
\draw [color=c, fill=c] (12.3881,9.39407) rectangle (12.4279,9.49992);
\draw [color=c, fill=c] (12.4279,9.39407) rectangle (12.4677,9.49992);
\draw [color=c, fill=c] (12.4677,9.39407) rectangle (12.5075,9.49992);
\draw [color=c, fill=c] (12.5075,9.39407) rectangle (12.5473,9.49992);
\draw [color=c, fill=c] (12.5473,9.39407) rectangle (12.5871,9.49992);
\draw [color=c, fill=c] (12.5871,9.39407) rectangle (12.6269,9.49992);
\draw [color=c, fill=c] (12.6269,9.39407) rectangle (12.6667,9.49992);
\draw [color=c, fill=c] (12.6667,9.39407) rectangle (12.7065,9.49992);
\draw [color=c, fill=c] (12.7065,9.39407) rectangle (12.7463,9.49992);
\draw [color=c, fill=c] (12.7463,9.39407) rectangle (12.7861,9.49992);
\draw [color=c, fill=c] (12.7861,9.39407) rectangle (12.8259,9.49992);
\draw [color=c, fill=c] (12.8259,9.39407) rectangle (12.8657,9.49992);
\draw [color=c, fill=c] (12.8657,9.39407) rectangle (12.9055,9.49992);
\draw [color=c, fill=c] (12.9055,9.39407) rectangle (12.9453,9.49992);
\draw [color=c, fill=c] (12.9453,9.39407) rectangle (12.9851,9.49992);
\draw [color=c, fill=c] (12.9851,9.39407) rectangle (13.0249,9.49992);
\draw [color=c, fill=c] (13.0249,9.39407) rectangle (13.0647,9.49992);
\draw [color=c, fill=c] (13.0647,9.39407) rectangle (13.1045,9.49992);
\draw [color=c, fill=c] (13.1045,9.39407) rectangle (13.1443,9.49992);
\draw [color=c, fill=c] (13.1443,9.39407) rectangle (13.1841,9.49992);
\draw [color=c, fill=c] (13.1841,9.39407) rectangle (13.2239,9.49992);
\draw [color=c, fill=c] (13.2239,9.39407) rectangle (13.2637,9.49992);
\draw [color=c, fill=c] (13.2637,9.39407) rectangle (13.3035,9.49992);
\draw [color=c, fill=c] (13.3035,9.39407) rectangle (13.3433,9.49992);
\draw [color=c, fill=c] (13.3433,9.39407) rectangle (13.3831,9.49992);
\draw [color=c, fill=c] (13.3831,9.39407) rectangle (13.4229,9.49992);
\draw [color=c, fill=c] (13.4229,9.39407) rectangle (13.4627,9.49992);
\definecolor{c}{rgb}{0,0.733333,1};
\draw [color=c, fill=c] (13.4627,9.39407) rectangle (13.5025,9.49992);
\draw [color=c, fill=c] (13.5025,9.39407) rectangle (13.5423,9.49992);
\draw [color=c, fill=c] (13.5423,9.39407) rectangle (13.5821,9.49992);
\draw [color=c, fill=c] (13.5821,9.39407) rectangle (13.6219,9.49992);
\draw [color=c, fill=c] (13.6219,9.39407) rectangle (13.6617,9.49992);
\draw [color=c, fill=c] (13.6617,9.39407) rectangle (13.7015,9.49992);
\draw [color=c, fill=c] (13.7015,9.39407) rectangle (13.7413,9.49992);
\draw [color=c, fill=c] (13.7413,9.39407) rectangle (13.7811,9.49992);
\draw [color=c, fill=c] (13.7811,9.39407) rectangle (13.8209,9.49992);
\draw [color=c, fill=c] (13.8209,9.39407) rectangle (13.8607,9.49992);
\draw [color=c, fill=c] (13.8607,9.39407) rectangle (13.9005,9.49992);
\draw [color=c, fill=c] (13.9005,9.39407) rectangle (13.9403,9.49992);
\draw [color=c, fill=c] (13.9403,9.39407) rectangle (13.9801,9.49992);
\draw [color=c, fill=c] (13.9801,9.39407) rectangle (14.0199,9.49992);
\draw [color=c, fill=c] (14.0199,9.39407) rectangle (14.0597,9.49992);
\draw [color=c, fill=c] (14.0597,9.39407) rectangle (14.0995,9.49992);
\draw [color=c, fill=c] (14.0995,9.39407) rectangle (14.1393,9.49992);
\draw [color=c, fill=c] (14.1393,9.39407) rectangle (14.1791,9.49992);
\draw [color=c, fill=c] (14.1791,9.39407) rectangle (14.2189,9.49992);
\draw [color=c, fill=c] (14.2189,9.39407) rectangle (14.2587,9.49992);
\draw [color=c, fill=c] (14.2587,9.39407) rectangle (14.2985,9.49992);
\draw [color=c, fill=c] (14.2985,9.39407) rectangle (14.3383,9.49992);
\draw [color=c, fill=c] (14.3383,9.39407) rectangle (14.3781,9.49992);
\draw [color=c, fill=c] (14.3781,9.39407) rectangle (14.4179,9.49992);
\draw [color=c, fill=c] (14.4179,9.39407) rectangle (14.4577,9.49992);
\draw [color=c, fill=c] (14.4577,9.39407) rectangle (14.4975,9.49992);
\draw [color=c, fill=c] (14.4975,9.39407) rectangle (14.5373,9.49992);
\draw [color=c, fill=c] (14.5373,9.39407) rectangle (14.5771,9.49992);
\draw [color=c, fill=c] (14.5771,9.39407) rectangle (14.6169,9.49992);
\draw [color=c, fill=c] (14.6169,9.39407) rectangle (14.6567,9.49992);
\draw [color=c, fill=c] (14.6567,9.39407) rectangle (14.6965,9.49992);
\draw [color=c, fill=c] (14.6965,9.39407) rectangle (14.7363,9.49992);
\draw [color=c, fill=c] (14.7363,9.39407) rectangle (14.7761,9.49992);
\draw [color=c, fill=c] (14.7761,9.39407) rectangle (14.8159,9.49992);
\draw [color=c, fill=c] (14.8159,9.39407) rectangle (14.8557,9.49992);
\draw [color=c, fill=c] (14.8557,9.39407) rectangle (14.8955,9.49992);
\draw [color=c, fill=c] (14.8955,9.39407) rectangle (14.9353,9.49992);
\draw [color=c, fill=c] (14.9353,9.39407) rectangle (14.9751,9.49992);
\draw [color=c, fill=c] (14.9751,9.39407) rectangle (15.0149,9.49992);
\draw [color=c, fill=c] (15.0149,9.39407) rectangle (15.0547,9.49992);
\draw [color=c, fill=c] (15.0547,9.39407) rectangle (15.0945,9.49992);
\draw [color=c, fill=c] (15.0945,9.39407) rectangle (15.1343,9.49992);
\draw [color=c, fill=c] (15.1343,9.39407) rectangle (15.1741,9.49992);
\draw [color=c, fill=c] (15.1741,9.39407) rectangle (15.2139,9.49992);
\draw [color=c, fill=c] (15.2139,9.39407) rectangle (15.2537,9.49992);
\draw [color=c, fill=c] (15.2537,9.39407) rectangle (15.2935,9.49992);
\draw [color=c, fill=c] (15.2935,9.39407) rectangle (15.3333,9.49992);
\draw [color=c, fill=c] (15.3333,9.39407) rectangle (15.3731,9.49992);
\draw [color=c, fill=c] (15.3731,9.39407) rectangle (15.4129,9.49992);
\draw [color=c, fill=c] (15.4129,9.39407) rectangle (15.4527,9.49992);
\draw [color=c, fill=c] (15.4527,9.39407) rectangle (15.4925,9.49992);
\draw [color=c, fill=c] (15.4925,9.39407) rectangle (15.5323,9.49992);
\draw [color=c, fill=c] (15.5323,9.39407) rectangle (15.5721,9.49992);
\draw [color=c, fill=c] (15.5721,9.39407) rectangle (15.6119,9.49992);
\draw [color=c, fill=c] (15.6119,9.39407) rectangle (15.6517,9.49992);
\draw [color=c, fill=c] (15.6517,9.39407) rectangle (15.6915,9.49992);
\draw [color=c, fill=c] (15.6915,9.39407) rectangle (15.7313,9.49992);
\draw [color=c, fill=c] (15.7313,9.39407) rectangle (15.7711,9.49992);
\draw [color=c, fill=c] (15.7711,9.39407) rectangle (15.8109,9.49992);
\draw [color=c, fill=c] (15.8109,9.39407) rectangle (15.8507,9.49992);
\draw [color=c, fill=c] (15.8507,9.39407) rectangle (15.8905,9.49992);
\draw [color=c, fill=c] (15.8905,9.39407) rectangle (15.9303,9.49992);
\draw [color=c, fill=c] (15.9303,9.39407) rectangle (15.9701,9.49992);
\draw [color=c, fill=c] (15.9701,9.39407) rectangle (16.01,9.49992);
\draw [color=c, fill=c] (16.01,9.39407) rectangle (16.0498,9.49992);
\draw [color=c, fill=c] (16.0498,9.39407) rectangle (16.0896,9.49992);
\draw [color=c, fill=c] (16.0896,9.39407) rectangle (16.1294,9.49992);
\draw [color=c, fill=c] (16.1294,9.39407) rectangle (16.1692,9.49992);
\draw [color=c, fill=c] (16.1692,9.39407) rectangle (16.209,9.49992);
\draw [color=c, fill=c] (16.209,9.39407) rectangle (16.2488,9.49992);
\draw [color=c, fill=c] (16.2488,9.39407) rectangle (16.2886,9.49992);
\draw [color=c, fill=c] (16.2886,9.39407) rectangle (16.3284,9.49992);
\draw [color=c, fill=c] (16.3284,9.39407) rectangle (16.3682,9.49992);
\draw [color=c, fill=c] (16.3682,9.39407) rectangle (16.408,9.49992);
\draw [color=c, fill=c] (16.408,9.39407) rectangle (16.4478,9.49992);
\draw [color=c, fill=c] (16.4478,9.39407) rectangle (16.4876,9.49992);
\draw [color=c, fill=c] (16.4876,9.39407) rectangle (16.5274,9.49992);
\draw [color=c, fill=c] (16.5274,9.39407) rectangle (16.5672,9.49992);
\draw [color=c, fill=c] (16.5672,9.39407) rectangle (16.607,9.49992);
\draw [color=c, fill=c] (16.607,9.39407) rectangle (16.6468,9.49992);
\draw [color=c, fill=c] (16.6468,9.39407) rectangle (16.6866,9.49992);
\draw [color=c, fill=c] (16.6866,9.39407) rectangle (16.7264,9.49992);
\draw [color=c, fill=c] (16.7264,9.39407) rectangle (16.7662,9.49992);
\draw [color=c, fill=c] (16.7662,9.39407) rectangle (16.806,9.49992);
\draw [color=c, fill=c] (16.806,9.39407) rectangle (16.8458,9.49992);
\draw [color=c, fill=c] (16.8458,9.39407) rectangle (16.8856,9.49992);
\draw [color=c, fill=c] (16.8856,9.39407) rectangle (16.9254,9.49992);
\draw [color=c, fill=c] (16.9254,9.39407) rectangle (16.9652,9.49992);
\draw [color=c, fill=c] (16.9652,9.39407) rectangle (17.005,9.49992);
\draw [color=c, fill=c] (17.005,9.39407) rectangle (17.0448,9.49992);
\draw [color=c, fill=c] (17.0448,9.39407) rectangle (17.0846,9.49992);
\draw [color=c, fill=c] (17.0846,9.39407) rectangle (17.1244,9.49992);
\draw [color=c, fill=c] (17.1244,9.39407) rectangle (17.1642,9.49992);
\draw [color=c, fill=c] (17.1642,9.39407) rectangle (17.204,9.49992);
\draw [color=c, fill=c] (17.204,9.39407) rectangle (17.2438,9.49992);
\draw [color=c, fill=c] (17.2438,9.39407) rectangle (17.2836,9.49992);
\draw [color=c, fill=c] (17.2836,9.39407) rectangle (17.3234,9.49992);
\draw [color=c, fill=c] (17.3234,9.39407) rectangle (17.3632,9.49992);
\draw [color=c, fill=c] (17.3632,9.39407) rectangle (17.403,9.49992);
\draw [color=c, fill=c] (17.403,9.39407) rectangle (17.4428,9.49992);
\draw [color=c, fill=c] (17.4428,9.39407) rectangle (17.4826,9.49992);
\draw [color=c, fill=c] (17.4826,9.39407) rectangle (17.5224,9.49992);
\draw [color=c, fill=c] (17.5224,9.39407) rectangle (17.5622,9.49992);
\draw [color=c, fill=c] (17.5622,9.39407) rectangle (17.602,9.49992);
\draw [color=c, fill=c] (17.602,9.39407) rectangle (17.6418,9.49992);
\draw [color=c, fill=c] (17.6418,9.39407) rectangle (17.6816,9.49992);
\draw [color=c, fill=c] (17.6816,9.39407) rectangle (17.7214,9.49992);
\draw [color=c, fill=c] (17.7214,9.39407) rectangle (17.7612,9.49992);
\draw [color=c, fill=c] (17.7612,9.39407) rectangle (17.801,9.49992);
\draw [color=c, fill=c] (17.801,9.39407) rectangle (17.8408,9.49992);
\draw [color=c, fill=c] (17.8408,9.39407) rectangle (17.8806,9.49992);
\draw [color=c, fill=c] (17.8806,9.39407) rectangle (17.9204,9.49992);
\draw [color=c, fill=c] (17.9204,9.39407) rectangle (17.9602,9.49992);
\draw [color=c, fill=c] (17.9602,9.39407) rectangle (18,9.49992);
\definecolor{c}{rgb}{0.2,0,1};
\draw [color=c, fill=c] (2,9.49992) rectangle (2.0398,9.60576);
\draw [color=c, fill=c] (2.0398,9.49992) rectangle (2.0796,9.60576);
\draw [color=c, fill=c] (2.0796,9.49992) rectangle (2.1194,9.60576);
\draw [color=c, fill=c] (2.1194,9.49992) rectangle (2.1592,9.60576);
\draw [color=c, fill=c] (2.1592,9.49992) rectangle (2.19901,9.60576);
\draw [color=c, fill=c] (2.19901,9.49992) rectangle (2.23881,9.60576);
\draw [color=c, fill=c] (2.23881,9.49992) rectangle (2.27861,9.60576);
\draw [color=c, fill=c] (2.27861,9.49992) rectangle (2.31841,9.60576);
\draw [color=c, fill=c] (2.31841,9.49992) rectangle (2.35821,9.60576);
\draw [color=c, fill=c] (2.35821,9.49992) rectangle (2.39801,9.60576);
\draw [color=c, fill=c] (2.39801,9.49992) rectangle (2.43781,9.60576);
\draw [color=c, fill=c] (2.43781,9.49992) rectangle (2.47761,9.60576);
\draw [color=c, fill=c] (2.47761,9.49992) rectangle (2.51741,9.60576);
\draw [color=c, fill=c] (2.51741,9.49992) rectangle (2.55721,9.60576);
\draw [color=c, fill=c] (2.55721,9.49992) rectangle (2.59702,9.60576);
\draw [color=c, fill=c] (2.59702,9.49992) rectangle (2.63682,9.60576);
\draw [color=c, fill=c] (2.63682,9.49992) rectangle (2.67662,9.60576);
\draw [color=c, fill=c] (2.67662,9.49992) rectangle (2.71642,9.60576);
\draw [color=c, fill=c] (2.71642,9.49992) rectangle (2.75622,9.60576);
\draw [color=c, fill=c] (2.75622,9.49992) rectangle (2.79602,9.60576);
\draw [color=c, fill=c] (2.79602,9.49992) rectangle (2.83582,9.60576);
\draw [color=c, fill=c] (2.83582,9.49992) rectangle (2.87562,9.60576);
\draw [color=c, fill=c] (2.87562,9.49992) rectangle (2.91542,9.60576);
\draw [color=c, fill=c] (2.91542,9.49992) rectangle (2.95522,9.60576);
\draw [color=c, fill=c] (2.95522,9.49992) rectangle (2.99502,9.60576);
\draw [color=c, fill=c] (2.99502,9.49992) rectangle (3.03483,9.60576);
\draw [color=c, fill=c] (3.03483,9.49992) rectangle (3.07463,9.60576);
\draw [color=c, fill=c] (3.07463,9.49992) rectangle (3.11443,9.60576);
\draw [color=c, fill=c] (3.11443,9.49992) rectangle (3.15423,9.60576);
\draw [color=c, fill=c] (3.15423,9.49992) rectangle (3.19403,9.60576);
\draw [color=c, fill=c] (3.19403,9.49992) rectangle (3.23383,9.60576);
\draw [color=c, fill=c] (3.23383,9.49992) rectangle (3.27363,9.60576);
\draw [color=c, fill=c] (3.27363,9.49992) rectangle (3.31343,9.60576);
\draw [color=c, fill=c] (3.31343,9.49992) rectangle (3.35323,9.60576);
\draw [color=c, fill=c] (3.35323,9.49992) rectangle (3.39303,9.60576);
\draw [color=c, fill=c] (3.39303,9.49992) rectangle (3.43284,9.60576);
\draw [color=c, fill=c] (3.43284,9.49992) rectangle (3.47264,9.60576);
\draw [color=c, fill=c] (3.47264,9.49992) rectangle (3.51244,9.60576);
\draw [color=c, fill=c] (3.51244,9.49992) rectangle (3.55224,9.60576);
\draw [color=c, fill=c] (3.55224,9.49992) rectangle (3.59204,9.60576);
\draw [color=c, fill=c] (3.59204,9.49992) rectangle (3.63184,9.60576);
\draw [color=c, fill=c] (3.63184,9.49992) rectangle (3.67164,9.60576);
\draw [color=c, fill=c] (3.67164,9.49992) rectangle (3.71144,9.60576);
\draw [color=c, fill=c] (3.71144,9.49992) rectangle (3.75124,9.60576);
\draw [color=c, fill=c] (3.75124,9.49992) rectangle (3.79104,9.60576);
\draw [color=c, fill=c] (3.79104,9.49992) rectangle (3.83085,9.60576);
\draw [color=c, fill=c] (3.83085,9.49992) rectangle (3.87065,9.60576);
\draw [color=c, fill=c] (3.87065,9.49992) rectangle (3.91045,9.60576);
\draw [color=c, fill=c] (3.91045,9.49992) rectangle (3.95025,9.60576);
\draw [color=c, fill=c] (3.95025,9.49992) rectangle (3.99005,9.60576);
\draw [color=c, fill=c] (3.99005,9.49992) rectangle (4.02985,9.60576);
\draw [color=c, fill=c] (4.02985,9.49992) rectangle (4.06965,9.60576);
\draw [color=c, fill=c] (4.06965,9.49992) rectangle (4.10945,9.60576);
\draw [color=c, fill=c] (4.10945,9.49992) rectangle (4.14925,9.60576);
\draw [color=c, fill=c] (4.14925,9.49992) rectangle (4.18905,9.60576);
\draw [color=c, fill=c] (4.18905,9.49992) rectangle (4.22886,9.60576);
\draw [color=c, fill=c] (4.22886,9.49992) rectangle (4.26866,9.60576);
\draw [color=c, fill=c] (4.26866,9.49992) rectangle (4.30846,9.60576);
\draw [color=c, fill=c] (4.30846,9.49992) rectangle (4.34826,9.60576);
\draw [color=c, fill=c] (4.34826,9.49992) rectangle (4.38806,9.60576);
\draw [color=c, fill=c] (4.38806,9.49992) rectangle (4.42786,9.60576);
\draw [color=c, fill=c] (4.42786,9.49992) rectangle (4.46766,9.60576);
\draw [color=c, fill=c] (4.46766,9.49992) rectangle (4.50746,9.60576);
\draw [color=c, fill=c] (4.50746,9.49992) rectangle (4.54726,9.60576);
\draw [color=c, fill=c] (4.54726,9.49992) rectangle (4.58706,9.60576);
\draw [color=c, fill=c] (4.58706,9.49992) rectangle (4.62687,9.60576);
\draw [color=c, fill=c] (4.62687,9.49992) rectangle (4.66667,9.60576);
\draw [color=c, fill=c] (4.66667,9.49992) rectangle (4.70647,9.60576);
\draw [color=c, fill=c] (4.70647,9.49992) rectangle (4.74627,9.60576);
\draw [color=c, fill=c] (4.74627,9.49992) rectangle (4.78607,9.60576);
\draw [color=c, fill=c] (4.78607,9.49992) rectangle (4.82587,9.60576);
\draw [color=c, fill=c] (4.82587,9.49992) rectangle (4.86567,9.60576);
\draw [color=c, fill=c] (4.86567,9.49992) rectangle (4.90547,9.60576);
\draw [color=c, fill=c] (4.90547,9.49992) rectangle (4.94527,9.60576);
\draw [color=c, fill=c] (4.94527,9.49992) rectangle (4.98507,9.60576);
\draw [color=c, fill=c] (4.98507,9.49992) rectangle (5.02488,9.60576);
\draw [color=c, fill=c] (5.02488,9.49992) rectangle (5.06468,9.60576);
\draw [color=c, fill=c] (5.06468,9.49992) rectangle (5.10448,9.60576);
\draw [color=c, fill=c] (5.10448,9.49992) rectangle (5.14428,9.60576);
\draw [color=c, fill=c] (5.14428,9.49992) rectangle (5.18408,9.60576);
\draw [color=c, fill=c] (5.18408,9.49992) rectangle (5.22388,9.60576);
\draw [color=c, fill=c] (5.22388,9.49992) rectangle (5.26368,9.60576);
\draw [color=c, fill=c] (5.26368,9.49992) rectangle (5.30348,9.60576);
\draw [color=c, fill=c] (5.30348,9.49992) rectangle (5.34328,9.60576);
\draw [color=c, fill=c] (5.34328,9.49992) rectangle (5.38308,9.60576);
\draw [color=c, fill=c] (5.38308,9.49992) rectangle (5.42289,9.60576);
\draw [color=c, fill=c] (5.42289,9.49992) rectangle (5.46269,9.60576);
\draw [color=c, fill=c] (5.46269,9.49992) rectangle (5.50249,9.60576);
\draw [color=c, fill=c] (5.50249,9.49992) rectangle (5.54229,9.60576);
\draw [color=c, fill=c] (5.54229,9.49992) rectangle (5.58209,9.60576);
\draw [color=c, fill=c] (5.58209,9.49992) rectangle (5.62189,9.60576);
\draw [color=c, fill=c] (5.62189,9.49992) rectangle (5.66169,9.60576);
\draw [color=c, fill=c] (5.66169,9.49992) rectangle (5.70149,9.60576);
\draw [color=c, fill=c] (5.70149,9.49992) rectangle (5.74129,9.60576);
\draw [color=c, fill=c] (5.74129,9.49992) rectangle (5.78109,9.60576);
\draw [color=c, fill=c] (5.78109,9.49992) rectangle (5.8209,9.60576);
\draw [color=c, fill=c] (5.8209,9.49992) rectangle (5.8607,9.60576);
\draw [color=c, fill=c] (5.8607,9.49992) rectangle (5.9005,9.60576);
\draw [color=c, fill=c] (5.9005,9.49992) rectangle (5.9403,9.60576);
\draw [color=c, fill=c] (5.9403,9.49992) rectangle (5.9801,9.60576);
\draw [color=c, fill=c] (5.9801,9.49992) rectangle (6.0199,9.60576);
\draw [color=c, fill=c] (6.0199,9.49992) rectangle (6.0597,9.60576);
\draw [color=c, fill=c] (6.0597,9.49992) rectangle (6.0995,9.60576);
\draw [color=c, fill=c] (6.0995,9.49992) rectangle (6.1393,9.60576);
\draw [color=c, fill=c] (6.1393,9.49992) rectangle (6.1791,9.60576);
\draw [color=c, fill=c] (6.1791,9.49992) rectangle (6.21891,9.60576);
\draw [color=c, fill=c] (6.21891,9.49992) rectangle (6.25871,9.60576);
\draw [color=c, fill=c] (6.25871,9.49992) rectangle (6.29851,9.60576);
\draw [color=c, fill=c] (6.29851,9.49992) rectangle (6.33831,9.60576);
\draw [color=c, fill=c] (6.33831,9.49992) rectangle (6.37811,9.60576);
\draw [color=c, fill=c] (6.37811,9.49992) rectangle (6.41791,9.60576);
\draw [color=c, fill=c] (6.41791,9.49992) rectangle (6.45771,9.60576);
\draw [color=c, fill=c] (6.45771,9.49992) rectangle (6.49751,9.60576);
\draw [color=c, fill=c] (6.49751,9.49992) rectangle (6.53731,9.60576);
\draw [color=c, fill=c] (6.53731,9.49992) rectangle (6.57711,9.60576);
\draw [color=c, fill=c] (6.57711,9.49992) rectangle (6.61692,9.60576);
\draw [color=c, fill=c] (6.61692,9.49992) rectangle (6.65672,9.60576);
\draw [color=c, fill=c] (6.65672,9.49992) rectangle (6.69652,9.60576);
\draw [color=c, fill=c] (6.69652,9.49992) rectangle (6.73632,9.60576);
\draw [color=c, fill=c] (6.73632,9.49992) rectangle (6.77612,9.60576);
\draw [color=c, fill=c] (6.77612,9.49992) rectangle (6.81592,9.60576);
\draw [color=c, fill=c] (6.81592,9.49992) rectangle (6.85572,9.60576);
\draw [color=c, fill=c] (6.85572,9.49992) rectangle (6.89552,9.60576);
\draw [color=c, fill=c] (6.89552,9.49992) rectangle (6.93532,9.60576);
\draw [color=c, fill=c] (6.93532,9.49992) rectangle (6.97512,9.60576);
\draw [color=c, fill=c] (6.97512,9.49992) rectangle (7.01493,9.60576);
\draw [color=c, fill=c] (7.01493,9.49992) rectangle (7.05473,9.60576);
\draw [color=c, fill=c] (7.05473,9.49992) rectangle (7.09453,9.60576);
\draw [color=c, fill=c] (7.09453,9.49992) rectangle (7.13433,9.60576);
\draw [color=c, fill=c] (7.13433,9.49992) rectangle (7.17413,9.60576);
\draw [color=c, fill=c] (7.17413,9.49992) rectangle (7.21393,9.60576);
\draw [color=c, fill=c] (7.21393,9.49992) rectangle (7.25373,9.60576);
\draw [color=c, fill=c] (7.25373,9.49992) rectangle (7.29353,9.60576);
\draw [color=c, fill=c] (7.29353,9.49992) rectangle (7.33333,9.60576);
\draw [color=c, fill=c] (7.33333,9.49992) rectangle (7.37313,9.60576);
\draw [color=c, fill=c] (7.37313,9.49992) rectangle (7.41294,9.60576);
\draw [color=c, fill=c] (7.41294,9.49992) rectangle (7.45274,9.60576);
\draw [color=c, fill=c] (7.45274,9.49992) rectangle (7.49254,9.60576);
\draw [color=c, fill=c] (7.49254,9.49992) rectangle (7.53234,9.60576);
\draw [color=c, fill=c] (7.53234,9.49992) rectangle (7.57214,9.60576);
\draw [color=c, fill=c] (7.57214,9.49992) rectangle (7.61194,9.60576);
\draw [color=c, fill=c] (7.61194,9.49992) rectangle (7.65174,9.60576);
\definecolor{c}{rgb}{0,0.0800001,1};
\draw [color=c, fill=c] (7.65174,9.49992) rectangle (7.69154,9.60576);
\draw [color=c, fill=c] (7.69154,9.49992) rectangle (7.73134,9.60576);
\draw [color=c, fill=c] (7.73134,9.49992) rectangle (7.77114,9.60576);
\draw [color=c, fill=c] (7.77114,9.49992) rectangle (7.81095,9.60576);
\draw [color=c, fill=c] (7.81095,9.49992) rectangle (7.85075,9.60576);
\draw [color=c, fill=c] (7.85075,9.49992) rectangle (7.89055,9.60576);
\draw [color=c, fill=c] (7.89055,9.49992) rectangle (7.93035,9.60576);
\draw [color=c, fill=c] (7.93035,9.49992) rectangle (7.97015,9.60576);
\draw [color=c, fill=c] (7.97015,9.49992) rectangle (8.00995,9.60576);
\draw [color=c, fill=c] (8.00995,9.49992) rectangle (8.04975,9.60576);
\draw [color=c, fill=c] (8.04975,9.49992) rectangle (8.08955,9.60576);
\draw [color=c, fill=c] (8.08955,9.49992) rectangle (8.12935,9.60576);
\draw [color=c, fill=c] (8.12935,9.49992) rectangle (8.16915,9.60576);
\draw [color=c, fill=c] (8.16915,9.49992) rectangle (8.20895,9.60576);
\draw [color=c, fill=c] (8.20895,9.49992) rectangle (8.24876,9.60576);
\draw [color=c, fill=c] (8.24876,9.49992) rectangle (8.28856,9.60576);
\draw [color=c, fill=c] (8.28856,9.49992) rectangle (8.32836,9.60576);
\draw [color=c, fill=c] (8.32836,9.49992) rectangle (8.36816,9.60576);
\draw [color=c, fill=c] (8.36816,9.49992) rectangle (8.40796,9.60576);
\draw [color=c, fill=c] (8.40796,9.49992) rectangle (8.44776,9.60576);
\draw [color=c, fill=c] (8.44776,9.49992) rectangle (8.48756,9.60576);
\draw [color=c, fill=c] (8.48756,9.49992) rectangle (8.52736,9.60576);
\draw [color=c, fill=c] (8.52736,9.49992) rectangle (8.56716,9.60576);
\draw [color=c, fill=c] (8.56716,9.49992) rectangle (8.60697,9.60576);
\draw [color=c, fill=c] (8.60697,9.49992) rectangle (8.64677,9.60576);
\draw [color=c, fill=c] (8.64677,9.49992) rectangle (8.68657,9.60576);
\draw [color=c, fill=c] (8.68657,9.49992) rectangle (8.72637,9.60576);
\draw [color=c, fill=c] (8.72637,9.49992) rectangle (8.76617,9.60576);
\draw [color=c, fill=c] (8.76617,9.49992) rectangle (8.80597,9.60576);
\draw [color=c, fill=c] (8.80597,9.49992) rectangle (8.84577,9.60576);
\draw [color=c, fill=c] (8.84577,9.49992) rectangle (8.88557,9.60576);
\draw [color=c, fill=c] (8.88557,9.49992) rectangle (8.92537,9.60576);
\draw [color=c, fill=c] (8.92537,9.49992) rectangle (8.96517,9.60576);
\draw [color=c, fill=c] (8.96517,9.49992) rectangle (9.00498,9.60576);
\draw [color=c, fill=c] (9.00498,9.49992) rectangle (9.04478,9.60576);
\draw [color=c, fill=c] (9.04478,9.49992) rectangle (9.08458,9.60576);
\draw [color=c, fill=c] (9.08458,9.49992) rectangle (9.12438,9.60576);
\draw [color=c, fill=c] (9.12438,9.49992) rectangle (9.16418,9.60576);
\draw [color=c, fill=c] (9.16418,9.49992) rectangle (9.20398,9.60576);
\draw [color=c, fill=c] (9.20398,9.49992) rectangle (9.24378,9.60576);
\draw [color=c, fill=c] (9.24378,9.49992) rectangle (9.28358,9.60576);
\draw [color=c, fill=c] (9.28358,9.49992) rectangle (9.32338,9.60576);
\draw [color=c, fill=c] (9.32338,9.49992) rectangle (9.36318,9.60576);
\draw [color=c, fill=c] (9.36318,9.49992) rectangle (9.40298,9.60576);
\draw [color=c, fill=c] (9.40298,9.49992) rectangle (9.44279,9.60576);
\draw [color=c, fill=c] (9.44279,9.49992) rectangle (9.48259,9.60576);
\definecolor{c}{rgb}{0,0.266667,1};
\draw [color=c, fill=c] (9.48259,9.49992) rectangle (9.52239,9.60576);
\draw [color=c, fill=c] (9.52239,9.49992) rectangle (9.56219,9.60576);
\draw [color=c, fill=c] (9.56219,9.49992) rectangle (9.60199,9.60576);
\draw [color=c, fill=c] (9.60199,9.49992) rectangle (9.64179,9.60576);
\draw [color=c, fill=c] (9.64179,9.49992) rectangle (9.68159,9.60576);
\draw [color=c, fill=c] (9.68159,9.49992) rectangle (9.72139,9.60576);
\draw [color=c, fill=c] (9.72139,9.49992) rectangle (9.76119,9.60576);
\draw [color=c, fill=c] (9.76119,9.49992) rectangle (9.80099,9.60576);
\draw [color=c, fill=c] (9.80099,9.49992) rectangle (9.8408,9.60576);
\draw [color=c, fill=c] (9.8408,9.49992) rectangle (9.8806,9.60576);
\draw [color=c, fill=c] (9.8806,9.49992) rectangle (9.9204,9.60576);
\draw [color=c, fill=c] (9.9204,9.49992) rectangle (9.9602,9.60576);
\draw [color=c, fill=c] (9.9602,9.49992) rectangle (10,9.60576);
\draw [color=c, fill=c] (10,9.49992) rectangle (10.0398,9.60576);
\draw [color=c, fill=c] (10.0398,9.49992) rectangle (10.0796,9.60576);
\draw [color=c, fill=c] (10.0796,9.49992) rectangle (10.1194,9.60576);
\draw [color=c, fill=c] (10.1194,9.49992) rectangle (10.1592,9.60576);
\draw [color=c, fill=c] (10.1592,9.49992) rectangle (10.199,9.60576);
\draw [color=c, fill=c] (10.199,9.49992) rectangle (10.2388,9.60576);
\draw [color=c, fill=c] (10.2388,9.49992) rectangle (10.2786,9.60576);
\draw [color=c, fill=c] (10.2786,9.49992) rectangle (10.3184,9.60576);
\draw [color=c, fill=c] (10.3184,9.49992) rectangle (10.3582,9.60576);
\draw [color=c, fill=c] (10.3582,9.49992) rectangle (10.398,9.60576);
\draw [color=c, fill=c] (10.398,9.49992) rectangle (10.4378,9.60576);
\draw [color=c, fill=c] (10.4378,9.49992) rectangle (10.4776,9.60576);
\draw [color=c, fill=c] (10.4776,9.49992) rectangle (10.5174,9.60576);
\draw [color=c, fill=c] (10.5174,9.49992) rectangle (10.5572,9.60576);
\draw [color=c, fill=c] (10.5572,9.49992) rectangle (10.597,9.60576);
\draw [color=c, fill=c] (10.597,9.49992) rectangle (10.6368,9.60576);
\draw [color=c, fill=c] (10.6368,9.49992) rectangle (10.6766,9.60576);
\definecolor{c}{rgb}{0,0.546666,1};
\draw [color=c, fill=c] (10.6766,9.49992) rectangle (10.7164,9.60576);
\draw [color=c, fill=c] (10.7164,9.49992) rectangle (10.7562,9.60576);
\draw [color=c, fill=c] (10.7562,9.49992) rectangle (10.796,9.60576);
\draw [color=c, fill=c] (10.796,9.49992) rectangle (10.8358,9.60576);
\draw [color=c, fill=c] (10.8358,9.49992) rectangle (10.8756,9.60576);
\draw [color=c, fill=c] (10.8756,9.49992) rectangle (10.9154,9.60576);
\draw [color=c, fill=c] (10.9154,9.49992) rectangle (10.9552,9.60576);
\draw [color=c, fill=c] (10.9552,9.49992) rectangle (10.995,9.60576);
\draw [color=c, fill=c] (10.995,9.49992) rectangle (11.0348,9.60576);
\draw [color=c, fill=c] (11.0348,9.49992) rectangle (11.0746,9.60576);
\draw [color=c, fill=c] (11.0746,9.49992) rectangle (11.1144,9.60576);
\draw [color=c, fill=c] (11.1144,9.49992) rectangle (11.1542,9.60576);
\draw [color=c, fill=c] (11.1542,9.49992) rectangle (11.194,9.60576);
\draw [color=c, fill=c] (11.194,9.49992) rectangle (11.2338,9.60576);
\draw [color=c, fill=c] (11.2338,9.49992) rectangle (11.2736,9.60576);
\draw [color=c, fill=c] (11.2736,9.49992) rectangle (11.3134,9.60576);
\draw [color=c, fill=c] (11.3134,9.49992) rectangle (11.3532,9.60576);
\draw [color=c, fill=c] (11.3532,9.49992) rectangle (11.393,9.60576);
\draw [color=c, fill=c] (11.393,9.49992) rectangle (11.4328,9.60576);
\draw [color=c, fill=c] (11.4328,9.49992) rectangle (11.4726,9.60576);
\draw [color=c, fill=c] (11.4726,9.49992) rectangle (11.5124,9.60576);
\draw [color=c, fill=c] (11.5124,9.49992) rectangle (11.5522,9.60576);
\draw [color=c, fill=c] (11.5522,9.49992) rectangle (11.592,9.60576);
\draw [color=c, fill=c] (11.592,9.49992) rectangle (11.6318,9.60576);
\draw [color=c, fill=c] (11.6318,9.49992) rectangle (11.6716,9.60576);
\draw [color=c, fill=c] (11.6716,9.49992) rectangle (11.7114,9.60576);
\draw [color=c, fill=c] (11.7114,9.49992) rectangle (11.7512,9.60576);
\draw [color=c, fill=c] (11.7512,9.49992) rectangle (11.791,9.60576);
\draw [color=c, fill=c] (11.791,9.49992) rectangle (11.8308,9.60576);
\draw [color=c, fill=c] (11.8308,9.49992) rectangle (11.8706,9.60576);
\draw [color=c, fill=c] (11.8706,9.49992) rectangle (11.9104,9.60576);
\draw [color=c, fill=c] (11.9104,9.49992) rectangle (11.9502,9.60576);
\draw [color=c, fill=c] (11.9502,9.49992) rectangle (11.99,9.60576);
\draw [color=c, fill=c] (11.99,9.49992) rectangle (12.0299,9.60576);
\draw [color=c, fill=c] (12.0299,9.49992) rectangle (12.0697,9.60576);
\draw [color=c, fill=c] (12.0697,9.49992) rectangle (12.1095,9.60576);
\draw [color=c, fill=c] (12.1095,9.49992) rectangle (12.1493,9.60576);
\draw [color=c, fill=c] (12.1493,9.49992) rectangle (12.1891,9.60576);
\draw [color=c, fill=c] (12.1891,9.49992) rectangle (12.2289,9.60576);
\draw [color=c, fill=c] (12.2289,9.49992) rectangle (12.2687,9.60576);
\draw [color=c, fill=c] (12.2687,9.49992) rectangle (12.3085,9.60576);
\draw [color=c, fill=c] (12.3085,9.49992) rectangle (12.3483,9.60576);
\draw [color=c, fill=c] (12.3483,9.49992) rectangle (12.3881,9.60576);
\draw [color=c, fill=c] (12.3881,9.49992) rectangle (12.4279,9.60576);
\draw [color=c, fill=c] (12.4279,9.49992) rectangle (12.4677,9.60576);
\draw [color=c, fill=c] (12.4677,9.49992) rectangle (12.5075,9.60576);
\draw [color=c, fill=c] (12.5075,9.49992) rectangle (12.5473,9.60576);
\draw [color=c, fill=c] (12.5473,9.49992) rectangle (12.5871,9.60576);
\draw [color=c, fill=c] (12.5871,9.49992) rectangle (12.6269,9.60576);
\draw [color=c, fill=c] (12.6269,9.49992) rectangle (12.6667,9.60576);
\draw [color=c, fill=c] (12.6667,9.49992) rectangle (12.7065,9.60576);
\draw [color=c, fill=c] (12.7065,9.49992) rectangle (12.7463,9.60576);
\draw [color=c, fill=c] (12.7463,9.49992) rectangle (12.7861,9.60576);
\draw [color=c, fill=c] (12.7861,9.49992) rectangle (12.8259,9.60576);
\draw [color=c, fill=c] (12.8259,9.49992) rectangle (12.8657,9.60576);
\draw [color=c, fill=c] (12.8657,9.49992) rectangle (12.9055,9.60576);
\draw [color=c, fill=c] (12.9055,9.49992) rectangle (12.9453,9.60576);
\draw [color=c, fill=c] (12.9453,9.49992) rectangle (12.9851,9.60576);
\draw [color=c, fill=c] (12.9851,9.49992) rectangle (13.0249,9.60576);
\draw [color=c, fill=c] (13.0249,9.49992) rectangle (13.0647,9.60576);
\draw [color=c, fill=c] (13.0647,9.49992) rectangle (13.1045,9.60576);
\draw [color=c, fill=c] (13.1045,9.49992) rectangle (13.1443,9.60576);
\draw [color=c, fill=c] (13.1443,9.49992) rectangle (13.1841,9.60576);
\draw [color=c, fill=c] (13.1841,9.49992) rectangle (13.2239,9.60576);
\draw [color=c, fill=c] (13.2239,9.49992) rectangle (13.2637,9.60576);
\draw [color=c, fill=c] (13.2637,9.49992) rectangle (13.3035,9.60576);
\draw [color=c, fill=c] (13.3035,9.49992) rectangle (13.3433,9.60576);
\draw [color=c, fill=c] (13.3433,9.49992) rectangle (13.3831,9.60576);
\draw [color=c, fill=c] (13.3831,9.49992) rectangle (13.4229,9.60576);
\draw [color=c, fill=c] (13.4229,9.49992) rectangle (13.4627,9.60576);
\draw [color=c, fill=c] (13.4627,9.49992) rectangle (13.5025,9.60576);
\draw [color=c, fill=c] (13.5025,9.49992) rectangle (13.5423,9.60576);
\definecolor{c}{rgb}{0,0.733333,1};
\draw [color=c, fill=c] (13.5423,9.49992) rectangle (13.5821,9.60576);
\draw [color=c, fill=c] (13.5821,9.49992) rectangle (13.6219,9.60576);
\draw [color=c, fill=c] (13.6219,9.49992) rectangle (13.6617,9.60576);
\draw [color=c, fill=c] (13.6617,9.49992) rectangle (13.7015,9.60576);
\draw [color=c, fill=c] (13.7015,9.49992) rectangle (13.7413,9.60576);
\draw [color=c, fill=c] (13.7413,9.49992) rectangle (13.7811,9.60576);
\draw [color=c, fill=c] (13.7811,9.49992) rectangle (13.8209,9.60576);
\draw [color=c, fill=c] (13.8209,9.49992) rectangle (13.8607,9.60576);
\draw [color=c, fill=c] (13.8607,9.49992) rectangle (13.9005,9.60576);
\draw [color=c, fill=c] (13.9005,9.49992) rectangle (13.9403,9.60576);
\draw [color=c, fill=c] (13.9403,9.49992) rectangle (13.9801,9.60576);
\draw [color=c, fill=c] (13.9801,9.49992) rectangle (14.0199,9.60576);
\draw [color=c, fill=c] (14.0199,9.49992) rectangle (14.0597,9.60576);
\draw [color=c, fill=c] (14.0597,9.49992) rectangle (14.0995,9.60576);
\draw [color=c, fill=c] (14.0995,9.49992) rectangle (14.1393,9.60576);
\draw [color=c, fill=c] (14.1393,9.49992) rectangle (14.1791,9.60576);
\draw [color=c, fill=c] (14.1791,9.49992) rectangle (14.2189,9.60576);
\draw [color=c, fill=c] (14.2189,9.49992) rectangle (14.2587,9.60576);
\draw [color=c, fill=c] (14.2587,9.49992) rectangle (14.2985,9.60576);
\draw [color=c, fill=c] (14.2985,9.49992) rectangle (14.3383,9.60576);
\draw [color=c, fill=c] (14.3383,9.49992) rectangle (14.3781,9.60576);
\draw [color=c, fill=c] (14.3781,9.49992) rectangle (14.4179,9.60576);
\draw [color=c, fill=c] (14.4179,9.49992) rectangle (14.4577,9.60576);
\draw [color=c, fill=c] (14.4577,9.49992) rectangle (14.4975,9.60576);
\draw [color=c, fill=c] (14.4975,9.49992) rectangle (14.5373,9.60576);
\draw [color=c, fill=c] (14.5373,9.49992) rectangle (14.5771,9.60576);
\draw [color=c, fill=c] (14.5771,9.49992) rectangle (14.6169,9.60576);
\draw [color=c, fill=c] (14.6169,9.49992) rectangle (14.6567,9.60576);
\draw [color=c, fill=c] (14.6567,9.49992) rectangle (14.6965,9.60576);
\draw [color=c, fill=c] (14.6965,9.49992) rectangle (14.7363,9.60576);
\draw [color=c, fill=c] (14.7363,9.49992) rectangle (14.7761,9.60576);
\draw [color=c, fill=c] (14.7761,9.49992) rectangle (14.8159,9.60576);
\draw [color=c, fill=c] (14.8159,9.49992) rectangle (14.8557,9.60576);
\draw [color=c, fill=c] (14.8557,9.49992) rectangle (14.8955,9.60576);
\draw [color=c, fill=c] (14.8955,9.49992) rectangle (14.9353,9.60576);
\draw [color=c, fill=c] (14.9353,9.49992) rectangle (14.9751,9.60576);
\draw [color=c, fill=c] (14.9751,9.49992) rectangle (15.0149,9.60576);
\draw [color=c, fill=c] (15.0149,9.49992) rectangle (15.0547,9.60576);
\draw [color=c, fill=c] (15.0547,9.49992) rectangle (15.0945,9.60576);
\draw [color=c, fill=c] (15.0945,9.49992) rectangle (15.1343,9.60576);
\draw [color=c, fill=c] (15.1343,9.49992) rectangle (15.1741,9.60576);
\draw [color=c, fill=c] (15.1741,9.49992) rectangle (15.2139,9.60576);
\draw [color=c, fill=c] (15.2139,9.49992) rectangle (15.2537,9.60576);
\draw [color=c, fill=c] (15.2537,9.49992) rectangle (15.2935,9.60576);
\draw [color=c, fill=c] (15.2935,9.49992) rectangle (15.3333,9.60576);
\draw [color=c, fill=c] (15.3333,9.49992) rectangle (15.3731,9.60576);
\draw [color=c, fill=c] (15.3731,9.49992) rectangle (15.4129,9.60576);
\draw [color=c, fill=c] (15.4129,9.49992) rectangle (15.4527,9.60576);
\draw [color=c, fill=c] (15.4527,9.49992) rectangle (15.4925,9.60576);
\draw [color=c, fill=c] (15.4925,9.49992) rectangle (15.5323,9.60576);
\draw [color=c, fill=c] (15.5323,9.49992) rectangle (15.5721,9.60576);
\draw [color=c, fill=c] (15.5721,9.49992) rectangle (15.6119,9.60576);
\draw [color=c, fill=c] (15.6119,9.49992) rectangle (15.6517,9.60576);
\draw [color=c, fill=c] (15.6517,9.49992) rectangle (15.6915,9.60576);
\draw [color=c, fill=c] (15.6915,9.49992) rectangle (15.7313,9.60576);
\draw [color=c, fill=c] (15.7313,9.49992) rectangle (15.7711,9.60576);
\draw [color=c, fill=c] (15.7711,9.49992) rectangle (15.8109,9.60576);
\draw [color=c, fill=c] (15.8109,9.49992) rectangle (15.8507,9.60576);
\draw [color=c, fill=c] (15.8507,9.49992) rectangle (15.8905,9.60576);
\draw [color=c, fill=c] (15.8905,9.49992) rectangle (15.9303,9.60576);
\draw [color=c, fill=c] (15.9303,9.49992) rectangle (15.9701,9.60576);
\draw [color=c, fill=c] (15.9701,9.49992) rectangle (16.01,9.60576);
\draw [color=c, fill=c] (16.01,9.49992) rectangle (16.0498,9.60576);
\draw [color=c, fill=c] (16.0498,9.49992) rectangle (16.0896,9.60576);
\draw [color=c, fill=c] (16.0896,9.49992) rectangle (16.1294,9.60576);
\draw [color=c, fill=c] (16.1294,9.49992) rectangle (16.1692,9.60576);
\draw [color=c, fill=c] (16.1692,9.49992) rectangle (16.209,9.60576);
\draw [color=c, fill=c] (16.209,9.49992) rectangle (16.2488,9.60576);
\draw [color=c, fill=c] (16.2488,9.49992) rectangle (16.2886,9.60576);
\draw [color=c, fill=c] (16.2886,9.49992) rectangle (16.3284,9.60576);
\draw [color=c, fill=c] (16.3284,9.49992) rectangle (16.3682,9.60576);
\draw [color=c, fill=c] (16.3682,9.49992) rectangle (16.408,9.60576);
\draw [color=c, fill=c] (16.408,9.49992) rectangle (16.4478,9.60576);
\draw [color=c, fill=c] (16.4478,9.49992) rectangle (16.4876,9.60576);
\draw [color=c, fill=c] (16.4876,9.49992) rectangle (16.5274,9.60576);
\draw [color=c, fill=c] (16.5274,9.49992) rectangle (16.5672,9.60576);
\draw [color=c, fill=c] (16.5672,9.49992) rectangle (16.607,9.60576);
\draw [color=c, fill=c] (16.607,9.49992) rectangle (16.6468,9.60576);
\draw [color=c, fill=c] (16.6468,9.49992) rectangle (16.6866,9.60576);
\draw [color=c, fill=c] (16.6866,9.49992) rectangle (16.7264,9.60576);
\draw [color=c, fill=c] (16.7264,9.49992) rectangle (16.7662,9.60576);
\draw [color=c, fill=c] (16.7662,9.49992) rectangle (16.806,9.60576);
\draw [color=c, fill=c] (16.806,9.49992) rectangle (16.8458,9.60576);
\draw [color=c, fill=c] (16.8458,9.49992) rectangle (16.8856,9.60576);
\draw [color=c, fill=c] (16.8856,9.49992) rectangle (16.9254,9.60576);
\draw [color=c, fill=c] (16.9254,9.49992) rectangle (16.9652,9.60576);
\draw [color=c, fill=c] (16.9652,9.49992) rectangle (17.005,9.60576);
\draw [color=c, fill=c] (17.005,9.49992) rectangle (17.0448,9.60576);
\draw [color=c, fill=c] (17.0448,9.49992) rectangle (17.0846,9.60576);
\draw [color=c, fill=c] (17.0846,9.49992) rectangle (17.1244,9.60576);
\draw [color=c, fill=c] (17.1244,9.49992) rectangle (17.1642,9.60576);
\draw [color=c, fill=c] (17.1642,9.49992) rectangle (17.204,9.60576);
\draw [color=c, fill=c] (17.204,9.49992) rectangle (17.2438,9.60576);
\draw [color=c, fill=c] (17.2438,9.49992) rectangle (17.2836,9.60576);
\draw [color=c, fill=c] (17.2836,9.49992) rectangle (17.3234,9.60576);
\draw [color=c, fill=c] (17.3234,9.49992) rectangle (17.3632,9.60576);
\draw [color=c, fill=c] (17.3632,9.49992) rectangle (17.403,9.60576);
\draw [color=c, fill=c] (17.403,9.49992) rectangle (17.4428,9.60576);
\draw [color=c, fill=c] (17.4428,9.49992) rectangle (17.4826,9.60576);
\draw [color=c, fill=c] (17.4826,9.49992) rectangle (17.5224,9.60576);
\draw [color=c, fill=c] (17.5224,9.49992) rectangle (17.5622,9.60576);
\draw [color=c, fill=c] (17.5622,9.49992) rectangle (17.602,9.60576);
\draw [color=c, fill=c] (17.602,9.49992) rectangle (17.6418,9.60576);
\draw [color=c, fill=c] (17.6418,9.49992) rectangle (17.6816,9.60576);
\draw [color=c, fill=c] (17.6816,9.49992) rectangle (17.7214,9.60576);
\draw [color=c, fill=c] (17.7214,9.49992) rectangle (17.7612,9.60576);
\draw [color=c, fill=c] (17.7612,9.49992) rectangle (17.801,9.60576);
\draw [color=c, fill=c] (17.801,9.49992) rectangle (17.8408,9.60576);
\draw [color=c, fill=c] (17.8408,9.49992) rectangle (17.8806,9.60576);
\draw [color=c, fill=c] (17.8806,9.49992) rectangle (17.9204,9.60576);
\draw [color=c, fill=c] (17.9204,9.49992) rectangle (17.9602,9.60576);
\draw [color=c, fill=c] (17.9602,9.49992) rectangle (18,9.60576);
\definecolor{c}{rgb}{0.2,0,1};
\draw [color=c, fill=c] (2,9.60576) rectangle (2.0398,9.71161);
\draw [color=c, fill=c] (2.0398,9.60576) rectangle (2.0796,9.71161);
\draw [color=c, fill=c] (2.0796,9.60576) rectangle (2.1194,9.71161);
\draw [color=c, fill=c] (2.1194,9.60576) rectangle (2.1592,9.71161);
\draw [color=c, fill=c] (2.1592,9.60576) rectangle (2.19901,9.71161);
\draw [color=c, fill=c] (2.19901,9.60576) rectangle (2.23881,9.71161);
\draw [color=c, fill=c] (2.23881,9.60576) rectangle (2.27861,9.71161);
\draw [color=c, fill=c] (2.27861,9.60576) rectangle (2.31841,9.71161);
\draw [color=c, fill=c] (2.31841,9.60576) rectangle (2.35821,9.71161);
\draw [color=c, fill=c] (2.35821,9.60576) rectangle (2.39801,9.71161);
\draw [color=c, fill=c] (2.39801,9.60576) rectangle (2.43781,9.71161);
\draw [color=c, fill=c] (2.43781,9.60576) rectangle (2.47761,9.71161);
\draw [color=c, fill=c] (2.47761,9.60576) rectangle (2.51741,9.71161);
\draw [color=c, fill=c] (2.51741,9.60576) rectangle (2.55721,9.71161);
\draw [color=c, fill=c] (2.55721,9.60576) rectangle (2.59702,9.71161);
\draw [color=c, fill=c] (2.59702,9.60576) rectangle (2.63682,9.71161);
\draw [color=c, fill=c] (2.63682,9.60576) rectangle (2.67662,9.71161);
\draw [color=c, fill=c] (2.67662,9.60576) rectangle (2.71642,9.71161);
\draw [color=c, fill=c] (2.71642,9.60576) rectangle (2.75622,9.71161);
\draw [color=c, fill=c] (2.75622,9.60576) rectangle (2.79602,9.71161);
\draw [color=c, fill=c] (2.79602,9.60576) rectangle (2.83582,9.71161);
\draw [color=c, fill=c] (2.83582,9.60576) rectangle (2.87562,9.71161);
\draw [color=c, fill=c] (2.87562,9.60576) rectangle (2.91542,9.71161);
\draw [color=c, fill=c] (2.91542,9.60576) rectangle (2.95522,9.71161);
\draw [color=c, fill=c] (2.95522,9.60576) rectangle (2.99502,9.71161);
\draw [color=c, fill=c] (2.99502,9.60576) rectangle (3.03483,9.71161);
\draw [color=c, fill=c] (3.03483,9.60576) rectangle (3.07463,9.71161);
\draw [color=c, fill=c] (3.07463,9.60576) rectangle (3.11443,9.71161);
\draw [color=c, fill=c] (3.11443,9.60576) rectangle (3.15423,9.71161);
\draw [color=c, fill=c] (3.15423,9.60576) rectangle (3.19403,9.71161);
\draw [color=c, fill=c] (3.19403,9.60576) rectangle (3.23383,9.71161);
\draw [color=c, fill=c] (3.23383,9.60576) rectangle (3.27363,9.71161);
\draw [color=c, fill=c] (3.27363,9.60576) rectangle (3.31343,9.71161);
\draw [color=c, fill=c] (3.31343,9.60576) rectangle (3.35323,9.71161);
\draw [color=c, fill=c] (3.35323,9.60576) rectangle (3.39303,9.71161);
\draw [color=c, fill=c] (3.39303,9.60576) rectangle (3.43284,9.71161);
\draw [color=c, fill=c] (3.43284,9.60576) rectangle (3.47264,9.71161);
\draw [color=c, fill=c] (3.47264,9.60576) rectangle (3.51244,9.71161);
\draw [color=c, fill=c] (3.51244,9.60576) rectangle (3.55224,9.71161);
\draw [color=c, fill=c] (3.55224,9.60576) rectangle (3.59204,9.71161);
\draw [color=c, fill=c] (3.59204,9.60576) rectangle (3.63184,9.71161);
\draw [color=c, fill=c] (3.63184,9.60576) rectangle (3.67164,9.71161);
\draw [color=c, fill=c] (3.67164,9.60576) rectangle (3.71144,9.71161);
\draw [color=c, fill=c] (3.71144,9.60576) rectangle (3.75124,9.71161);
\draw [color=c, fill=c] (3.75124,9.60576) rectangle (3.79104,9.71161);
\draw [color=c, fill=c] (3.79104,9.60576) rectangle (3.83085,9.71161);
\draw [color=c, fill=c] (3.83085,9.60576) rectangle (3.87065,9.71161);
\draw [color=c, fill=c] (3.87065,9.60576) rectangle (3.91045,9.71161);
\draw [color=c, fill=c] (3.91045,9.60576) rectangle (3.95025,9.71161);
\draw [color=c, fill=c] (3.95025,9.60576) rectangle (3.99005,9.71161);
\draw [color=c, fill=c] (3.99005,9.60576) rectangle (4.02985,9.71161);
\draw [color=c, fill=c] (4.02985,9.60576) rectangle (4.06965,9.71161);
\draw [color=c, fill=c] (4.06965,9.60576) rectangle (4.10945,9.71161);
\draw [color=c, fill=c] (4.10945,9.60576) rectangle (4.14925,9.71161);
\draw [color=c, fill=c] (4.14925,9.60576) rectangle (4.18905,9.71161);
\draw [color=c, fill=c] (4.18905,9.60576) rectangle (4.22886,9.71161);
\draw [color=c, fill=c] (4.22886,9.60576) rectangle (4.26866,9.71161);
\draw [color=c, fill=c] (4.26866,9.60576) rectangle (4.30846,9.71161);
\draw [color=c, fill=c] (4.30846,9.60576) rectangle (4.34826,9.71161);
\draw [color=c, fill=c] (4.34826,9.60576) rectangle (4.38806,9.71161);
\draw [color=c, fill=c] (4.38806,9.60576) rectangle (4.42786,9.71161);
\draw [color=c, fill=c] (4.42786,9.60576) rectangle (4.46766,9.71161);
\draw [color=c, fill=c] (4.46766,9.60576) rectangle (4.50746,9.71161);
\draw [color=c, fill=c] (4.50746,9.60576) rectangle (4.54726,9.71161);
\draw [color=c, fill=c] (4.54726,9.60576) rectangle (4.58706,9.71161);
\draw [color=c, fill=c] (4.58706,9.60576) rectangle (4.62687,9.71161);
\draw [color=c, fill=c] (4.62687,9.60576) rectangle (4.66667,9.71161);
\draw [color=c, fill=c] (4.66667,9.60576) rectangle (4.70647,9.71161);
\draw [color=c, fill=c] (4.70647,9.60576) rectangle (4.74627,9.71161);
\draw [color=c, fill=c] (4.74627,9.60576) rectangle (4.78607,9.71161);
\draw [color=c, fill=c] (4.78607,9.60576) rectangle (4.82587,9.71161);
\draw [color=c, fill=c] (4.82587,9.60576) rectangle (4.86567,9.71161);
\draw [color=c, fill=c] (4.86567,9.60576) rectangle (4.90547,9.71161);
\draw [color=c, fill=c] (4.90547,9.60576) rectangle (4.94527,9.71161);
\draw [color=c, fill=c] (4.94527,9.60576) rectangle (4.98507,9.71161);
\draw [color=c, fill=c] (4.98507,9.60576) rectangle (5.02488,9.71161);
\draw [color=c, fill=c] (5.02488,9.60576) rectangle (5.06468,9.71161);
\draw [color=c, fill=c] (5.06468,9.60576) rectangle (5.10448,9.71161);
\draw [color=c, fill=c] (5.10448,9.60576) rectangle (5.14428,9.71161);
\draw [color=c, fill=c] (5.14428,9.60576) rectangle (5.18408,9.71161);
\draw [color=c, fill=c] (5.18408,9.60576) rectangle (5.22388,9.71161);
\draw [color=c, fill=c] (5.22388,9.60576) rectangle (5.26368,9.71161);
\draw [color=c, fill=c] (5.26368,9.60576) rectangle (5.30348,9.71161);
\draw [color=c, fill=c] (5.30348,9.60576) rectangle (5.34328,9.71161);
\draw [color=c, fill=c] (5.34328,9.60576) rectangle (5.38308,9.71161);
\draw [color=c, fill=c] (5.38308,9.60576) rectangle (5.42289,9.71161);
\draw [color=c, fill=c] (5.42289,9.60576) rectangle (5.46269,9.71161);
\draw [color=c, fill=c] (5.46269,9.60576) rectangle (5.50249,9.71161);
\draw [color=c, fill=c] (5.50249,9.60576) rectangle (5.54229,9.71161);
\draw [color=c, fill=c] (5.54229,9.60576) rectangle (5.58209,9.71161);
\draw [color=c, fill=c] (5.58209,9.60576) rectangle (5.62189,9.71161);
\draw [color=c, fill=c] (5.62189,9.60576) rectangle (5.66169,9.71161);
\draw [color=c, fill=c] (5.66169,9.60576) rectangle (5.70149,9.71161);
\draw [color=c, fill=c] (5.70149,9.60576) rectangle (5.74129,9.71161);
\draw [color=c, fill=c] (5.74129,9.60576) rectangle (5.78109,9.71161);
\draw [color=c, fill=c] (5.78109,9.60576) rectangle (5.8209,9.71161);
\draw [color=c, fill=c] (5.8209,9.60576) rectangle (5.8607,9.71161);
\draw [color=c, fill=c] (5.8607,9.60576) rectangle (5.9005,9.71161);
\draw [color=c, fill=c] (5.9005,9.60576) rectangle (5.9403,9.71161);
\draw [color=c, fill=c] (5.9403,9.60576) rectangle (5.9801,9.71161);
\draw [color=c, fill=c] (5.9801,9.60576) rectangle (6.0199,9.71161);
\draw [color=c, fill=c] (6.0199,9.60576) rectangle (6.0597,9.71161);
\draw [color=c, fill=c] (6.0597,9.60576) rectangle (6.0995,9.71161);
\draw [color=c, fill=c] (6.0995,9.60576) rectangle (6.1393,9.71161);
\draw [color=c, fill=c] (6.1393,9.60576) rectangle (6.1791,9.71161);
\draw [color=c, fill=c] (6.1791,9.60576) rectangle (6.21891,9.71161);
\draw [color=c, fill=c] (6.21891,9.60576) rectangle (6.25871,9.71161);
\draw [color=c, fill=c] (6.25871,9.60576) rectangle (6.29851,9.71161);
\draw [color=c, fill=c] (6.29851,9.60576) rectangle (6.33831,9.71161);
\draw [color=c, fill=c] (6.33831,9.60576) rectangle (6.37811,9.71161);
\draw [color=c, fill=c] (6.37811,9.60576) rectangle (6.41791,9.71161);
\draw [color=c, fill=c] (6.41791,9.60576) rectangle (6.45771,9.71161);
\draw [color=c, fill=c] (6.45771,9.60576) rectangle (6.49751,9.71161);
\draw [color=c, fill=c] (6.49751,9.60576) rectangle (6.53731,9.71161);
\draw [color=c, fill=c] (6.53731,9.60576) rectangle (6.57711,9.71161);
\draw [color=c, fill=c] (6.57711,9.60576) rectangle (6.61692,9.71161);
\draw [color=c, fill=c] (6.61692,9.60576) rectangle (6.65672,9.71161);
\draw [color=c, fill=c] (6.65672,9.60576) rectangle (6.69652,9.71161);
\draw [color=c, fill=c] (6.69652,9.60576) rectangle (6.73632,9.71161);
\draw [color=c, fill=c] (6.73632,9.60576) rectangle (6.77612,9.71161);
\draw [color=c, fill=c] (6.77612,9.60576) rectangle (6.81592,9.71161);
\draw [color=c, fill=c] (6.81592,9.60576) rectangle (6.85572,9.71161);
\draw [color=c, fill=c] (6.85572,9.60576) rectangle (6.89552,9.71161);
\draw [color=c, fill=c] (6.89552,9.60576) rectangle (6.93532,9.71161);
\draw [color=c, fill=c] (6.93532,9.60576) rectangle (6.97512,9.71161);
\draw [color=c, fill=c] (6.97512,9.60576) rectangle (7.01493,9.71161);
\draw [color=c, fill=c] (7.01493,9.60576) rectangle (7.05473,9.71161);
\draw [color=c, fill=c] (7.05473,9.60576) rectangle (7.09453,9.71161);
\draw [color=c, fill=c] (7.09453,9.60576) rectangle (7.13433,9.71161);
\draw [color=c, fill=c] (7.13433,9.60576) rectangle (7.17413,9.71161);
\draw [color=c, fill=c] (7.17413,9.60576) rectangle (7.21393,9.71161);
\draw [color=c, fill=c] (7.21393,9.60576) rectangle (7.25373,9.71161);
\draw [color=c, fill=c] (7.25373,9.60576) rectangle (7.29353,9.71161);
\draw [color=c, fill=c] (7.29353,9.60576) rectangle (7.33333,9.71161);
\draw [color=c, fill=c] (7.33333,9.60576) rectangle (7.37313,9.71161);
\draw [color=c, fill=c] (7.37313,9.60576) rectangle (7.41294,9.71161);
\draw [color=c, fill=c] (7.41294,9.60576) rectangle (7.45274,9.71161);
\draw [color=c, fill=c] (7.45274,9.60576) rectangle (7.49254,9.71161);
\draw [color=c, fill=c] (7.49254,9.60576) rectangle (7.53234,9.71161);
\draw [color=c, fill=c] (7.53234,9.60576) rectangle (7.57214,9.71161);
\draw [color=c, fill=c] (7.57214,9.60576) rectangle (7.61194,9.71161);
\draw [color=c, fill=c] (7.61194,9.60576) rectangle (7.65174,9.71161);
\definecolor{c}{rgb}{0,0.0800001,1};
\draw [color=c, fill=c] (7.65174,9.60576) rectangle (7.69154,9.71161);
\draw [color=c, fill=c] (7.69154,9.60576) rectangle (7.73134,9.71161);
\draw [color=c, fill=c] (7.73134,9.60576) rectangle (7.77114,9.71161);
\draw [color=c, fill=c] (7.77114,9.60576) rectangle (7.81095,9.71161);
\draw [color=c, fill=c] (7.81095,9.60576) rectangle (7.85075,9.71161);
\draw [color=c, fill=c] (7.85075,9.60576) rectangle (7.89055,9.71161);
\draw [color=c, fill=c] (7.89055,9.60576) rectangle (7.93035,9.71161);
\draw [color=c, fill=c] (7.93035,9.60576) rectangle (7.97015,9.71161);
\draw [color=c, fill=c] (7.97015,9.60576) rectangle (8.00995,9.71161);
\draw [color=c, fill=c] (8.00995,9.60576) rectangle (8.04975,9.71161);
\draw [color=c, fill=c] (8.04975,9.60576) rectangle (8.08955,9.71161);
\draw [color=c, fill=c] (8.08955,9.60576) rectangle (8.12935,9.71161);
\draw [color=c, fill=c] (8.12935,9.60576) rectangle (8.16915,9.71161);
\draw [color=c, fill=c] (8.16915,9.60576) rectangle (8.20895,9.71161);
\draw [color=c, fill=c] (8.20895,9.60576) rectangle (8.24876,9.71161);
\draw [color=c, fill=c] (8.24876,9.60576) rectangle (8.28856,9.71161);
\draw [color=c, fill=c] (8.28856,9.60576) rectangle (8.32836,9.71161);
\draw [color=c, fill=c] (8.32836,9.60576) rectangle (8.36816,9.71161);
\draw [color=c, fill=c] (8.36816,9.60576) rectangle (8.40796,9.71161);
\draw [color=c, fill=c] (8.40796,9.60576) rectangle (8.44776,9.71161);
\draw [color=c, fill=c] (8.44776,9.60576) rectangle (8.48756,9.71161);
\draw [color=c, fill=c] (8.48756,9.60576) rectangle (8.52736,9.71161);
\draw [color=c, fill=c] (8.52736,9.60576) rectangle (8.56716,9.71161);
\draw [color=c, fill=c] (8.56716,9.60576) rectangle (8.60697,9.71161);
\draw [color=c, fill=c] (8.60697,9.60576) rectangle (8.64677,9.71161);
\draw [color=c, fill=c] (8.64677,9.60576) rectangle (8.68657,9.71161);
\draw [color=c, fill=c] (8.68657,9.60576) rectangle (8.72637,9.71161);
\draw [color=c, fill=c] (8.72637,9.60576) rectangle (8.76617,9.71161);
\draw [color=c, fill=c] (8.76617,9.60576) rectangle (8.80597,9.71161);
\draw [color=c, fill=c] (8.80597,9.60576) rectangle (8.84577,9.71161);
\draw [color=c, fill=c] (8.84577,9.60576) rectangle (8.88557,9.71161);
\draw [color=c, fill=c] (8.88557,9.60576) rectangle (8.92537,9.71161);
\draw [color=c, fill=c] (8.92537,9.60576) rectangle (8.96517,9.71161);
\draw [color=c, fill=c] (8.96517,9.60576) rectangle (9.00498,9.71161);
\draw [color=c, fill=c] (9.00498,9.60576) rectangle (9.04478,9.71161);
\draw [color=c, fill=c] (9.04478,9.60576) rectangle (9.08458,9.71161);
\draw [color=c, fill=c] (9.08458,9.60576) rectangle (9.12438,9.71161);
\draw [color=c, fill=c] (9.12438,9.60576) rectangle (9.16418,9.71161);
\draw [color=c, fill=c] (9.16418,9.60576) rectangle (9.20398,9.71161);
\draw [color=c, fill=c] (9.20398,9.60576) rectangle (9.24378,9.71161);
\draw [color=c, fill=c] (9.24378,9.60576) rectangle (9.28358,9.71161);
\draw [color=c, fill=c] (9.28358,9.60576) rectangle (9.32338,9.71161);
\draw [color=c, fill=c] (9.32338,9.60576) rectangle (9.36318,9.71161);
\draw [color=c, fill=c] (9.36318,9.60576) rectangle (9.40298,9.71161);
\draw [color=c, fill=c] (9.40298,9.60576) rectangle (9.44279,9.71161);
\draw [color=c, fill=c] (9.44279,9.60576) rectangle (9.48259,9.71161);
\draw [color=c, fill=c] (9.48259,9.60576) rectangle (9.52239,9.71161);
\definecolor{c}{rgb}{0,0.266667,1};
\draw [color=c, fill=c] (9.52239,9.60576) rectangle (9.56219,9.71161);
\draw [color=c, fill=c] (9.56219,9.60576) rectangle (9.60199,9.71161);
\draw [color=c, fill=c] (9.60199,9.60576) rectangle (9.64179,9.71161);
\draw [color=c, fill=c] (9.64179,9.60576) rectangle (9.68159,9.71161);
\draw [color=c, fill=c] (9.68159,9.60576) rectangle (9.72139,9.71161);
\draw [color=c, fill=c] (9.72139,9.60576) rectangle (9.76119,9.71161);
\draw [color=c, fill=c] (9.76119,9.60576) rectangle (9.80099,9.71161);
\draw [color=c, fill=c] (9.80099,9.60576) rectangle (9.8408,9.71161);
\draw [color=c, fill=c] (9.8408,9.60576) rectangle (9.8806,9.71161);
\draw [color=c, fill=c] (9.8806,9.60576) rectangle (9.9204,9.71161);
\draw [color=c, fill=c] (9.9204,9.60576) rectangle (9.9602,9.71161);
\draw [color=c, fill=c] (9.9602,9.60576) rectangle (10,9.71161);
\draw [color=c, fill=c] (10,9.60576) rectangle (10.0398,9.71161);
\draw [color=c, fill=c] (10.0398,9.60576) rectangle (10.0796,9.71161);
\draw [color=c, fill=c] (10.0796,9.60576) rectangle (10.1194,9.71161);
\draw [color=c, fill=c] (10.1194,9.60576) rectangle (10.1592,9.71161);
\draw [color=c, fill=c] (10.1592,9.60576) rectangle (10.199,9.71161);
\draw [color=c, fill=c] (10.199,9.60576) rectangle (10.2388,9.71161);
\draw [color=c, fill=c] (10.2388,9.60576) rectangle (10.2786,9.71161);
\draw [color=c, fill=c] (10.2786,9.60576) rectangle (10.3184,9.71161);
\draw [color=c, fill=c] (10.3184,9.60576) rectangle (10.3582,9.71161);
\draw [color=c, fill=c] (10.3582,9.60576) rectangle (10.398,9.71161);
\draw [color=c, fill=c] (10.398,9.60576) rectangle (10.4378,9.71161);
\draw [color=c, fill=c] (10.4378,9.60576) rectangle (10.4776,9.71161);
\draw [color=c, fill=c] (10.4776,9.60576) rectangle (10.5174,9.71161);
\draw [color=c, fill=c] (10.5174,9.60576) rectangle (10.5572,9.71161);
\draw [color=c, fill=c] (10.5572,9.60576) rectangle (10.597,9.71161);
\draw [color=c, fill=c] (10.597,9.60576) rectangle (10.6368,9.71161);
\draw [color=c, fill=c] (10.6368,9.60576) rectangle (10.6766,9.71161);
\draw [color=c, fill=c] (10.6766,9.60576) rectangle (10.7164,9.71161);
\definecolor{c}{rgb}{0,0.546666,1};
\draw [color=c, fill=c] (10.7164,9.60576) rectangle (10.7562,9.71161);
\draw [color=c, fill=c] (10.7562,9.60576) rectangle (10.796,9.71161);
\draw [color=c, fill=c] (10.796,9.60576) rectangle (10.8358,9.71161);
\draw [color=c, fill=c] (10.8358,9.60576) rectangle (10.8756,9.71161);
\draw [color=c, fill=c] (10.8756,9.60576) rectangle (10.9154,9.71161);
\draw [color=c, fill=c] (10.9154,9.60576) rectangle (10.9552,9.71161);
\draw [color=c, fill=c] (10.9552,9.60576) rectangle (10.995,9.71161);
\draw [color=c, fill=c] (10.995,9.60576) rectangle (11.0348,9.71161);
\draw [color=c, fill=c] (11.0348,9.60576) rectangle (11.0746,9.71161);
\draw [color=c, fill=c] (11.0746,9.60576) rectangle (11.1144,9.71161);
\draw [color=c, fill=c] (11.1144,9.60576) rectangle (11.1542,9.71161);
\draw [color=c, fill=c] (11.1542,9.60576) rectangle (11.194,9.71161);
\draw [color=c, fill=c] (11.194,9.60576) rectangle (11.2338,9.71161);
\draw [color=c, fill=c] (11.2338,9.60576) rectangle (11.2736,9.71161);
\draw [color=c, fill=c] (11.2736,9.60576) rectangle (11.3134,9.71161);
\draw [color=c, fill=c] (11.3134,9.60576) rectangle (11.3532,9.71161);
\draw [color=c, fill=c] (11.3532,9.60576) rectangle (11.393,9.71161);
\draw [color=c, fill=c] (11.393,9.60576) rectangle (11.4328,9.71161);
\draw [color=c, fill=c] (11.4328,9.60576) rectangle (11.4726,9.71161);
\draw [color=c, fill=c] (11.4726,9.60576) rectangle (11.5124,9.71161);
\draw [color=c, fill=c] (11.5124,9.60576) rectangle (11.5522,9.71161);
\draw [color=c, fill=c] (11.5522,9.60576) rectangle (11.592,9.71161);
\draw [color=c, fill=c] (11.592,9.60576) rectangle (11.6318,9.71161);
\draw [color=c, fill=c] (11.6318,9.60576) rectangle (11.6716,9.71161);
\draw [color=c, fill=c] (11.6716,9.60576) rectangle (11.7114,9.71161);
\draw [color=c, fill=c] (11.7114,9.60576) rectangle (11.7512,9.71161);
\draw [color=c, fill=c] (11.7512,9.60576) rectangle (11.791,9.71161);
\draw [color=c, fill=c] (11.791,9.60576) rectangle (11.8308,9.71161);
\draw [color=c, fill=c] (11.8308,9.60576) rectangle (11.8706,9.71161);
\draw [color=c, fill=c] (11.8706,9.60576) rectangle (11.9104,9.71161);
\draw [color=c, fill=c] (11.9104,9.60576) rectangle (11.9502,9.71161);
\draw [color=c, fill=c] (11.9502,9.60576) rectangle (11.99,9.71161);
\draw [color=c, fill=c] (11.99,9.60576) rectangle (12.0299,9.71161);
\draw [color=c, fill=c] (12.0299,9.60576) rectangle (12.0697,9.71161);
\draw [color=c, fill=c] (12.0697,9.60576) rectangle (12.1095,9.71161);
\draw [color=c, fill=c] (12.1095,9.60576) rectangle (12.1493,9.71161);
\draw [color=c, fill=c] (12.1493,9.60576) rectangle (12.1891,9.71161);
\draw [color=c, fill=c] (12.1891,9.60576) rectangle (12.2289,9.71161);
\draw [color=c, fill=c] (12.2289,9.60576) rectangle (12.2687,9.71161);
\draw [color=c, fill=c] (12.2687,9.60576) rectangle (12.3085,9.71161);
\draw [color=c, fill=c] (12.3085,9.60576) rectangle (12.3483,9.71161);
\draw [color=c, fill=c] (12.3483,9.60576) rectangle (12.3881,9.71161);
\draw [color=c, fill=c] (12.3881,9.60576) rectangle (12.4279,9.71161);
\draw [color=c, fill=c] (12.4279,9.60576) rectangle (12.4677,9.71161);
\draw [color=c, fill=c] (12.4677,9.60576) rectangle (12.5075,9.71161);
\draw [color=c, fill=c] (12.5075,9.60576) rectangle (12.5473,9.71161);
\draw [color=c, fill=c] (12.5473,9.60576) rectangle (12.5871,9.71161);
\draw [color=c, fill=c] (12.5871,9.60576) rectangle (12.6269,9.71161);
\draw [color=c, fill=c] (12.6269,9.60576) rectangle (12.6667,9.71161);
\draw [color=c, fill=c] (12.6667,9.60576) rectangle (12.7065,9.71161);
\draw [color=c, fill=c] (12.7065,9.60576) rectangle (12.7463,9.71161);
\draw [color=c, fill=c] (12.7463,9.60576) rectangle (12.7861,9.71161);
\draw [color=c, fill=c] (12.7861,9.60576) rectangle (12.8259,9.71161);
\draw [color=c, fill=c] (12.8259,9.60576) rectangle (12.8657,9.71161);
\draw [color=c, fill=c] (12.8657,9.60576) rectangle (12.9055,9.71161);
\draw [color=c, fill=c] (12.9055,9.60576) rectangle (12.9453,9.71161);
\draw [color=c, fill=c] (12.9453,9.60576) rectangle (12.9851,9.71161);
\draw [color=c, fill=c] (12.9851,9.60576) rectangle (13.0249,9.71161);
\draw [color=c, fill=c] (13.0249,9.60576) rectangle (13.0647,9.71161);
\draw [color=c, fill=c] (13.0647,9.60576) rectangle (13.1045,9.71161);
\draw [color=c, fill=c] (13.1045,9.60576) rectangle (13.1443,9.71161);
\draw [color=c, fill=c] (13.1443,9.60576) rectangle (13.1841,9.71161);
\draw [color=c, fill=c] (13.1841,9.60576) rectangle (13.2239,9.71161);
\draw [color=c, fill=c] (13.2239,9.60576) rectangle (13.2637,9.71161);
\draw [color=c, fill=c] (13.2637,9.60576) rectangle (13.3035,9.71161);
\draw [color=c, fill=c] (13.3035,9.60576) rectangle (13.3433,9.71161);
\draw [color=c, fill=c] (13.3433,9.60576) rectangle (13.3831,9.71161);
\draw [color=c, fill=c] (13.3831,9.60576) rectangle (13.4229,9.71161);
\draw [color=c, fill=c] (13.4229,9.60576) rectangle (13.4627,9.71161);
\draw [color=c, fill=c] (13.4627,9.60576) rectangle (13.5025,9.71161);
\draw [color=c, fill=c] (13.5025,9.60576) rectangle (13.5423,9.71161);
\draw [color=c, fill=c] (13.5423,9.60576) rectangle (13.5821,9.71161);
\draw [color=c, fill=c] (13.5821,9.60576) rectangle (13.6219,9.71161);
\definecolor{c}{rgb}{0,0.733333,1};
\draw [color=c, fill=c] (13.6219,9.60576) rectangle (13.6617,9.71161);
\draw [color=c, fill=c] (13.6617,9.60576) rectangle (13.7015,9.71161);
\draw [color=c, fill=c] (13.7015,9.60576) rectangle (13.7413,9.71161);
\draw [color=c, fill=c] (13.7413,9.60576) rectangle (13.7811,9.71161);
\draw [color=c, fill=c] (13.7811,9.60576) rectangle (13.8209,9.71161);
\draw [color=c, fill=c] (13.8209,9.60576) rectangle (13.8607,9.71161);
\draw [color=c, fill=c] (13.8607,9.60576) rectangle (13.9005,9.71161);
\draw [color=c, fill=c] (13.9005,9.60576) rectangle (13.9403,9.71161);
\draw [color=c, fill=c] (13.9403,9.60576) rectangle (13.9801,9.71161);
\draw [color=c, fill=c] (13.9801,9.60576) rectangle (14.0199,9.71161);
\draw [color=c, fill=c] (14.0199,9.60576) rectangle (14.0597,9.71161);
\draw [color=c, fill=c] (14.0597,9.60576) rectangle (14.0995,9.71161);
\draw [color=c, fill=c] (14.0995,9.60576) rectangle (14.1393,9.71161);
\draw [color=c, fill=c] (14.1393,9.60576) rectangle (14.1791,9.71161);
\draw [color=c, fill=c] (14.1791,9.60576) rectangle (14.2189,9.71161);
\draw [color=c, fill=c] (14.2189,9.60576) rectangle (14.2587,9.71161);
\draw [color=c, fill=c] (14.2587,9.60576) rectangle (14.2985,9.71161);
\draw [color=c, fill=c] (14.2985,9.60576) rectangle (14.3383,9.71161);
\draw [color=c, fill=c] (14.3383,9.60576) rectangle (14.3781,9.71161);
\draw [color=c, fill=c] (14.3781,9.60576) rectangle (14.4179,9.71161);
\draw [color=c, fill=c] (14.4179,9.60576) rectangle (14.4577,9.71161);
\draw [color=c, fill=c] (14.4577,9.60576) rectangle (14.4975,9.71161);
\draw [color=c, fill=c] (14.4975,9.60576) rectangle (14.5373,9.71161);
\draw [color=c, fill=c] (14.5373,9.60576) rectangle (14.5771,9.71161);
\draw [color=c, fill=c] (14.5771,9.60576) rectangle (14.6169,9.71161);
\draw [color=c, fill=c] (14.6169,9.60576) rectangle (14.6567,9.71161);
\draw [color=c, fill=c] (14.6567,9.60576) rectangle (14.6965,9.71161);
\draw [color=c, fill=c] (14.6965,9.60576) rectangle (14.7363,9.71161);
\draw [color=c, fill=c] (14.7363,9.60576) rectangle (14.7761,9.71161);
\draw [color=c, fill=c] (14.7761,9.60576) rectangle (14.8159,9.71161);
\draw [color=c, fill=c] (14.8159,9.60576) rectangle (14.8557,9.71161);
\draw [color=c, fill=c] (14.8557,9.60576) rectangle (14.8955,9.71161);
\draw [color=c, fill=c] (14.8955,9.60576) rectangle (14.9353,9.71161);
\draw [color=c, fill=c] (14.9353,9.60576) rectangle (14.9751,9.71161);
\draw [color=c, fill=c] (14.9751,9.60576) rectangle (15.0149,9.71161);
\draw [color=c, fill=c] (15.0149,9.60576) rectangle (15.0547,9.71161);
\draw [color=c, fill=c] (15.0547,9.60576) rectangle (15.0945,9.71161);
\draw [color=c, fill=c] (15.0945,9.60576) rectangle (15.1343,9.71161);
\draw [color=c, fill=c] (15.1343,9.60576) rectangle (15.1741,9.71161);
\draw [color=c, fill=c] (15.1741,9.60576) rectangle (15.2139,9.71161);
\draw [color=c, fill=c] (15.2139,9.60576) rectangle (15.2537,9.71161);
\draw [color=c, fill=c] (15.2537,9.60576) rectangle (15.2935,9.71161);
\draw [color=c, fill=c] (15.2935,9.60576) rectangle (15.3333,9.71161);
\draw [color=c, fill=c] (15.3333,9.60576) rectangle (15.3731,9.71161);
\draw [color=c, fill=c] (15.3731,9.60576) rectangle (15.4129,9.71161);
\draw [color=c, fill=c] (15.4129,9.60576) rectangle (15.4527,9.71161);
\draw [color=c, fill=c] (15.4527,9.60576) rectangle (15.4925,9.71161);
\draw [color=c, fill=c] (15.4925,9.60576) rectangle (15.5323,9.71161);
\draw [color=c, fill=c] (15.5323,9.60576) rectangle (15.5721,9.71161);
\draw [color=c, fill=c] (15.5721,9.60576) rectangle (15.6119,9.71161);
\draw [color=c, fill=c] (15.6119,9.60576) rectangle (15.6517,9.71161);
\draw [color=c, fill=c] (15.6517,9.60576) rectangle (15.6915,9.71161);
\draw [color=c, fill=c] (15.6915,9.60576) rectangle (15.7313,9.71161);
\draw [color=c, fill=c] (15.7313,9.60576) rectangle (15.7711,9.71161);
\draw [color=c, fill=c] (15.7711,9.60576) rectangle (15.8109,9.71161);
\draw [color=c, fill=c] (15.8109,9.60576) rectangle (15.8507,9.71161);
\draw [color=c, fill=c] (15.8507,9.60576) rectangle (15.8905,9.71161);
\draw [color=c, fill=c] (15.8905,9.60576) rectangle (15.9303,9.71161);
\draw [color=c, fill=c] (15.9303,9.60576) rectangle (15.9701,9.71161);
\draw [color=c, fill=c] (15.9701,9.60576) rectangle (16.01,9.71161);
\draw [color=c, fill=c] (16.01,9.60576) rectangle (16.0498,9.71161);
\draw [color=c, fill=c] (16.0498,9.60576) rectangle (16.0896,9.71161);
\draw [color=c, fill=c] (16.0896,9.60576) rectangle (16.1294,9.71161);
\draw [color=c, fill=c] (16.1294,9.60576) rectangle (16.1692,9.71161);
\draw [color=c, fill=c] (16.1692,9.60576) rectangle (16.209,9.71161);
\draw [color=c, fill=c] (16.209,9.60576) rectangle (16.2488,9.71161);
\draw [color=c, fill=c] (16.2488,9.60576) rectangle (16.2886,9.71161);
\draw [color=c, fill=c] (16.2886,9.60576) rectangle (16.3284,9.71161);
\draw [color=c, fill=c] (16.3284,9.60576) rectangle (16.3682,9.71161);
\draw [color=c, fill=c] (16.3682,9.60576) rectangle (16.408,9.71161);
\draw [color=c, fill=c] (16.408,9.60576) rectangle (16.4478,9.71161);
\draw [color=c, fill=c] (16.4478,9.60576) rectangle (16.4876,9.71161);
\draw [color=c, fill=c] (16.4876,9.60576) rectangle (16.5274,9.71161);
\draw [color=c, fill=c] (16.5274,9.60576) rectangle (16.5672,9.71161);
\draw [color=c, fill=c] (16.5672,9.60576) rectangle (16.607,9.71161);
\draw [color=c, fill=c] (16.607,9.60576) rectangle (16.6468,9.71161);
\draw [color=c, fill=c] (16.6468,9.60576) rectangle (16.6866,9.71161);
\draw [color=c, fill=c] (16.6866,9.60576) rectangle (16.7264,9.71161);
\draw [color=c, fill=c] (16.7264,9.60576) rectangle (16.7662,9.71161);
\draw [color=c, fill=c] (16.7662,9.60576) rectangle (16.806,9.71161);
\draw [color=c, fill=c] (16.806,9.60576) rectangle (16.8458,9.71161);
\draw [color=c, fill=c] (16.8458,9.60576) rectangle (16.8856,9.71161);
\draw [color=c, fill=c] (16.8856,9.60576) rectangle (16.9254,9.71161);
\draw [color=c, fill=c] (16.9254,9.60576) rectangle (16.9652,9.71161);
\draw [color=c, fill=c] (16.9652,9.60576) rectangle (17.005,9.71161);
\draw [color=c, fill=c] (17.005,9.60576) rectangle (17.0448,9.71161);
\draw [color=c, fill=c] (17.0448,9.60576) rectangle (17.0846,9.71161);
\draw [color=c, fill=c] (17.0846,9.60576) rectangle (17.1244,9.71161);
\draw [color=c, fill=c] (17.1244,9.60576) rectangle (17.1642,9.71161);
\draw [color=c, fill=c] (17.1642,9.60576) rectangle (17.204,9.71161);
\draw [color=c, fill=c] (17.204,9.60576) rectangle (17.2438,9.71161);
\draw [color=c, fill=c] (17.2438,9.60576) rectangle (17.2836,9.71161);
\draw [color=c, fill=c] (17.2836,9.60576) rectangle (17.3234,9.71161);
\draw [color=c, fill=c] (17.3234,9.60576) rectangle (17.3632,9.71161);
\draw [color=c, fill=c] (17.3632,9.60576) rectangle (17.403,9.71161);
\draw [color=c, fill=c] (17.403,9.60576) rectangle (17.4428,9.71161);
\draw [color=c, fill=c] (17.4428,9.60576) rectangle (17.4826,9.71161);
\draw [color=c, fill=c] (17.4826,9.60576) rectangle (17.5224,9.71161);
\draw [color=c, fill=c] (17.5224,9.60576) rectangle (17.5622,9.71161);
\draw [color=c, fill=c] (17.5622,9.60576) rectangle (17.602,9.71161);
\draw [color=c, fill=c] (17.602,9.60576) rectangle (17.6418,9.71161);
\draw [color=c, fill=c] (17.6418,9.60576) rectangle (17.6816,9.71161);
\draw [color=c, fill=c] (17.6816,9.60576) rectangle (17.7214,9.71161);
\draw [color=c, fill=c] (17.7214,9.60576) rectangle (17.7612,9.71161);
\draw [color=c, fill=c] (17.7612,9.60576) rectangle (17.801,9.71161);
\draw [color=c, fill=c] (17.801,9.60576) rectangle (17.8408,9.71161);
\draw [color=c, fill=c] (17.8408,9.60576) rectangle (17.8806,9.71161);
\draw [color=c, fill=c] (17.8806,9.60576) rectangle (17.9204,9.71161);
\draw [color=c, fill=c] (17.9204,9.60576) rectangle (17.9602,9.71161);
\draw [color=c, fill=c] (17.9602,9.60576) rectangle (18,9.71161);
\definecolor{c}{rgb}{0.2,0,1};
\draw [color=c, fill=c] (2,9.71161) rectangle (2.0398,9.81746);
\draw [color=c, fill=c] (2.0398,9.71161) rectangle (2.0796,9.81746);
\draw [color=c, fill=c] (2.0796,9.71161) rectangle (2.1194,9.81746);
\draw [color=c, fill=c] (2.1194,9.71161) rectangle (2.1592,9.81746);
\draw [color=c, fill=c] (2.1592,9.71161) rectangle (2.19901,9.81746);
\draw [color=c, fill=c] (2.19901,9.71161) rectangle (2.23881,9.81746);
\draw [color=c, fill=c] (2.23881,9.71161) rectangle (2.27861,9.81746);
\draw [color=c, fill=c] (2.27861,9.71161) rectangle (2.31841,9.81746);
\draw [color=c, fill=c] (2.31841,9.71161) rectangle (2.35821,9.81746);
\draw [color=c, fill=c] (2.35821,9.71161) rectangle (2.39801,9.81746);
\draw [color=c, fill=c] (2.39801,9.71161) rectangle (2.43781,9.81746);
\draw [color=c, fill=c] (2.43781,9.71161) rectangle (2.47761,9.81746);
\draw [color=c, fill=c] (2.47761,9.71161) rectangle (2.51741,9.81746);
\draw [color=c, fill=c] (2.51741,9.71161) rectangle (2.55721,9.81746);
\draw [color=c, fill=c] (2.55721,9.71161) rectangle (2.59702,9.81746);
\draw [color=c, fill=c] (2.59702,9.71161) rectangle (2.63682,9.81746);
\draw [color=c, fill=c] (2.63682,9.71161) rectangle (2.67662,9.81746);
\draw [color=c, fill=c] (2.67662,9.71161) rectangle (2.71642,9.81746);
\draw [color=c, fill=c] (2.71642,9.71161) rectangle (2.75622,9.81746);
\draw [color=c, fill=c] (2.75622,9.71161) rectangle (2.79602,9.81746);
\draw [color=c, fill=c] (2.79602,9.71161) rectangle (2.83582,9.81746);
\draw [color=c, fill=c] (2.83582,9.71161) rectangle (2.87562,9.81746);
\draw [color=c, fill=c] (2.87562,9.71161) rectangle (2.91542,9.81746);
\draw [color=c, fill=c] (2.91542,9.71161) rectangle (2.95522,9.81746);
\draw [color=c, fill=c] (2.95522,9.71161) rectangle (2.99502,9.81746);
\draw [color=c, fill=c] (2.99502,9.71161) rectangle (3.03483,9.81746);
\draw [color=c, fill=c] (3.03483,9.71161) rectangle (3.07463,9.81746);
\draw [color=c, fill=c] (3.07463,9.71161) rectangle (3.11443,9.81746);
\draw [color=c, fill=c] (3.11443,9.71161) rectangle (3.15423,9.81746);
\draw [color=c, fill=c] (3.15423,9.71161) rectangle (3.19403,9.81746);
\draw [color=c, fill=c] (3.19403,9.71161) rectangle (3.23383,9.81746);
\draw [color=c, fill=c] (3.23383,9.71161) rectangle (3.27363,9.81746);
\draw [color=c, fill=c] (3.27363,9.71161) rectangle (3.31343,9.81746);
\draw [color=c, fill=c] (3.31343,9.71161) rectangle (3.35323,9.81746);
\draw [color=c, fill=c] (3.35323,9.71161) rectangle (3.39303,9.81746);
\draw [color=c, fill=c] (3.39303,9.71161) rectangle (3.43284,9.81746);
\draw [color=c, fill=c] (3.43284,9.71161) rectangle (3.47264,9.81746);
\draw [color=c, fill=c] (3.47264,9.71161) rectangle (3.51244,9.81746);
\draw [color=c, fill=c] (3.51244,9.71161) rectangle (3.55224,9.81746);
\draw [color=c, fill=c] (3.55224,9.71161) rectangle (3.59204,9.81746);
\draw [color=c, fill=c] (3.59204,9.71161) rectangle (3.63184,9.81746);
\draw [color=c, fill=c] (3.63184,9.71161) rectangle (3.67164,9.81746);
\draw [color=c, fill=c] (3.67164,9.71161) rectangle (3.71144,9.81746);
\draw [color=c, fill=c] (3.71144,9.71161) rectangle (3.75124,9.81746);
\draw [color=c, fill=c] (3.75124,9.71161) rectangle (3.79104,9.81746);
\draw [color=c, fill=c] (3.79104,9.71161) rectangle (3.83085,9.81746);
\draw [color=c, fill=c] (3.83085,9.71161) rectangle (3.87065,9.81746);
\draw [color=c, fill=c] (3.87065,9.71161) rectangle (3.91045,9.81746);
\draw [color=c, fill=c] (3.91045,9.71161) rectangle (3.95025,9.81746);
\draw [color=c, fill=c] (3.95025,9.71161) rectangle (3.99005,9.81746);
\draw [color=c, fill=c] (3.99005,9.71161) rectangle (4.02985,9.81746);
\draw [color=c, fill=c] (4.02985,9.71161) rectangle (4.06965,9.81746);
\draw [color=c, fill=c] (4.06965,9.71161) rectangle (4.10945,9.81746);
\draw [color=c, fill=c] (4.10945,9.71161) rectangle (4.14925,9.81746);
\draw [color=c, fill=c] (4.14925,9.71161) rectangle (4.18905,9.81746);
\draw [color=c, fill=c] (4.18905,9.71161) rectangle (4.22886,9.81746);
\draw [color=c, fill=c] (4.22886,9.71161) rectangle (4.26866,9.81746);
\draw [color=c, fill=c] (4.26866,9.71161) rectangle (4.30846,9.81746);
\draw [color=c, fill=c] (4.30846,9.71161) rectangle (4.34826,9.81746);
\draw [color=c, fill=c] (4.34826,9.71161) rectangle (4.38806,9.81746);
\draw [color=c, fill=c] (4.38806,9.71161) rectangle (4.42786,9.81746);
\draw [color=c, fill=c] (4.42786,9.71161) rectangle (4.46766,9.81746);
\draw [color=c, fill=c] (4.46766,9.71161) rectangle (4.50746,9.81746);
\draw [color=c, fill=c] (4.50746,9.71161) rectangle (4.54726,9.81746);
\draw [color=c, fill=c] (4.54726,9.71161) rectangle (4.58706,9.81746);
\draw [color=c, fill=c] (4.58706,9.71161) rectangle (4.62687,9.81746);
\draw [color=c, fill=c] (4.62687,9.71161) rectangle (4.66667,9.81746);
\draw [color=c, fill=c] (4.66667,9.71161) rectangle (4.70647,9.81746);
\draw [color=c, fill=c] (4.70647,9.71161) rectangle (4.74627,9.81746);
\draw [color=c, fill=c] (4.74627,9.71161) rectangle (4.78607,9.81746);
\draw [color=c, fill=c] (4.78607,9.71161) rectangle (4.82587,9.81746);
\draw [color=c, fill=c] (4.82587,9.71161) rectangle (4.86567,9.81746);
\draw [color=c, fill=c] (4.86567,9.71161) rectangle (4.90547,9.81746);
\draw [color=c, fill=c] (4.90547,9.71161) rectangle (4.94527,9.81746);
\draw [color=c, fill=c] (4.94527,9.71161) rectangle (4.98507,9.81746);
\draw [color=c, fill=c] (4.98507,9.71161) rectangle (5.02488,9.81746);
\draw [color=c, fill=c] (5.02488,9.71161) rectangle (5.06468,9.81746);
\draw [color=c, fill=c] (5.06468,9.71161) rectangle (5.10448,9.81746);
\draw [color=c, fill=c] (5.10448,9.71161) rectangle (5.14428,9.81746);
\draw [color=c, fill=c] (5.14428,9.71161) rectangle (5.18408,9.81746);
\draw [color=c, fill=c] (5.18408,9.71161) rectangle (5.22388,9.81746);
\draw [color=c, fill=c] (5.22388,9.71161) rectangle (5.26368,9.81746);
\draw [color=c, fill=c] (5.26368,9.71161) rectangle (5.30348,9.81746);
\draw [color=c, fill=c] (5.30348,9.71161) rectangle (5.34328,9.81746);
\draw [color=c, fill=c] (5.34328,9.71161) rectangle (5.38308,9.81746);
\draw [color=c, fill=c] (5.38308,9.71161) rectangle (5.42289,9.81746);
\draw [color=c, fill=c] (5.42289,9.71161) rectangle (5.46269,9.81746);
\draw [color=c, fill=c] (5.46269,9.71161) rectangle (5.50249,9.81746);
\draw [color=c, fill=c] (5.50249,9.71161) rectangle (5.54229,9.81746);
\draw [color=c, fill=c] (5.54229,9.71161) rectangle (5.58209,9.81746);
\draw [color=c, fill=c] (5.58209,9.71161) rectangle (5.62189,9.81746);
\draw [color=c, fill=c] (5.62189,9.71161) rectangle (5.66169,9.81746);
\draw [color=c, fill=c] (5.66169,9.71161) rectangle (5.70149,9.81746);
\draw [color=c, fill=c] (5.70149,9.71161) rectangle (5.74129,9.81746);
\draw [color=c, fill=c] (5.74129,9.71161) rectangle (5.78109,9.81746);
\draw [color=c, fill=c] (5.78109,9.71161) rectangle (5.8209,9.81746);
\draw [color=c, fill=c] (5.8209,9.71161) rectangle (5.8607,9.81746);
\draw [color=c, fill=c] (5.8607,9.71161) rectangle (5.9005,9.81746);
\draw [color=c, fill=c] (5.9005,9.71161) rectangle (5.9403,9.81746);
\draw [color=c, fill=c] (5.9403,9.71161) rectangle (5.9801,9.81746);
\draw [color=c, fill=c] (5.9801,9.71161) rectangle (6.0199,9.81746);
\draw [color=c, fill=c] (6.0199,9.71161) rectangle (6.0597,9.81746);
\draw [color=c, fill=c] (6.0597,9.71161) rectangle (6.0995,9.81746);
\draw [color=c, fill=c] (6.0995,9.71161) rectangle (6.1393,9.81746);
\draw [color=c, fill=c] (6.1393,9.71161) rectangle (6.1791,9.81746);
\draw [color=c, fill=c] (6.1791,9.71161) rectangle (6.21891,9.81746);
\draw [color=c, fill=c] (6.21891,9.71161) rectangle (6.25871,9.81746);
\draw [color=c, fill=c] (6.25871,9.71161) rectangle (6.29851,9.81746);
\draw [color=c, fill=c] (6.29851,9.71161) rectangle (6.33831,9.81746);
\draw [color=c, fill=c] (6.33831,9.71161) rectangle (6.37811,9.81746);
\draw [color=c, fill=c] (6.37811,9.71161) rectangle (6.41791,9.81746);
\draw [color=c, fill=c] (6.41791,9.71161) rectangle (6.45771,9.81746);
\draw [color=c, fill=c] (6.45771,9.71161) rectangle (6.49751,9.81746);
\draw [color=c, fill=c] (6.49751,9.71161) rectangle (6.53731,9.81746);
\draw [color=c, fill=c] (6.53731,9.71161) rectangle (6.57711,9.81746);
\draw [color=c, fill=c] (6.57711,9.71161) rectangle (6.61692,9.81746);
\draw [color=c, fill=c] (6.61692,9.71161) rectangle (6.65672,9.81746);
\draw [color=c, fill=c] (6.65672,9.71161) rectangle (6.69652,9.81746);
\draw [color=c, fill=c] (6.69652,9.71161) rectangle (6.73632,9.81746);
\draw [color=c, fill=c] (6.73632,9.71161) rectangle (6.77612,9.81746);
\draw [color=c, fill=c] (6.77612,9.71161) rectangle (6.81592,9.81746);
\draw [color=c, fill=c] (6.81592,9.71161) rectangle (6.85572,9.81746);
\draw [color=c, fill=c] (6.85572,9.71161) rectangle (6.89552,9.81746);
\draw [color=c, fill=c] (6.89552,9.71161) rectangle (6.93532,9.81746);
\draw [color=c, fill=c] (6.93532,9.71161) rectangle (6.97512,9.81746);
\draw [color=c, fill=c] (6.97512,9.71161) rectangle (7.01493,9.81746);
\draw [color=c, fill=c] (7.01493,9.71161) rectangle (7.05473,9.81746);
\draw [color=c, fill=c] (7.05473,9.71161) rectangle (7.09453,9.81746);
\draw [color=c, fill=c] (7.09453,9.71161) rectangle (7.13433,9.81746);
\draw [color=c, fill=c] (7.13433,9.71161) rectangle (7.17413,9.81746);
\draw [color=c, fill=c] (7.17413,9.71161) rectangle (7.21393,9.81746);
\draw [color=c, fill=c] (7.21393,9.71161) rectangle (7.25373,9.81746);
\draw [color=c, fill=c] (7.25373,9.71161) rectangle (7.29353,9.81746);
\draw [color=c, fill=c] (7.29353,9.71161) rectangle (7.33333,9.81746);
\draw [color=c, fill=c] (7.33333,9.71161) rectangle (7.37313,9.81746);
\draw [color=c, fill=c] (7.37313,9.71161) rectangle (7.41294,9.81746);
\draw [color=c, fill=c] (7.41294,9.71161) rectangle (7.45274,9.81746);
\draw [color=c, fill=c] (7.45274,9.71161) rectangle (7.49254,9.81746);
\draw [color=c, fill=c] (7.49254,9.71161) rectangle (7.53234,9.81746);
\draw [color=c, fill=c] (7.53234,9.71161) rectangle (7.57214,9.81746);
\draw [color=c, fill=c] (7.57214,9.71161) rectangle (7.61194,9.81746);
\draw [color=c, fill=c] (7.61194,9.71161) rectangle (7.65174,9.81746);
\draw [color=c, fill=c] (7.65174,9.71161) rectangle (7.69154,9.81746);
\definecolor{c}{rgb}{0,0.0800001,1};
\draw [color=c, fill=c] (7.69154,9.71161) rectangle (7.73134,9.81746);
\draw [color=c, fill=c] (7.73134,9.71161) rectangle (7.77114,9.81746);
\draw [color=c, fill=c] (7.77114,9.71161) rectangle (7.81095,9.81746);
\draw [color=c, fill=c] (7.81095,9.71161) rectangle (7.85075,9.81746);
\draw [color=c, fill=c] (7.85075,9.71161) rectangle (7.89055,9.81746);
\draw [color=c, fill=c] (7.89055,9.71161) rectangle (7.93035,9.81746);
\draw [color=c, fill=c] (7.93035,9.71161) rectangle (7.97015,9.81746);
\draw [color=c, fill=c] (7.97015,9.71161) rectangle (8.00995,9.81746);
\draw [color=c, fill=c] (8.00995,9.71161) rectangle (8.04975,9.81746);
\draw [color=c, fill=c] (8.04975,9.71161) rectangle (8.08955,9.81746);
\draw [color=c, fill=c] (8.08955,9.71161) rectangle (8.12935,9.81746);
\draw [color=c, fill=c] (8.12935,9.71161) rectangle (8.16915,9.81746);
\draw [color=c, fill=c] (8.16915,9.71161) rectangle (8.20895,9.81746);
\draw [color=c, fill=c] (8.20895,9.71161) rectangle (8.24876,9.81746);
\draw [color=c, fill=c] (8.24876,9.71161) rectangle (8.28856,9.81746);
\draw [color=c, fill=c] (8.28856,9.71161) rectangle (8.32836,9.81746);
\draw [color=c, fill=c] (8.32836,9.71161) rectangle (8.36816,9.81746);
\draw [color=c, fill=c] (8.36816,9.71161) rectangle (8.40796,9.81746);
\draw [color=c, fill=c] (8.40796,9.71161) rectangle (8.44776,9.81746);
\draw [color=c, fill=c] (8.44776,9.71161) rectangle (8.48756,9.81746);
\draw [color=c, fill=c] (8.48756,9.71161) rectangle (8.52736,9.81746);
\draw [color=c, fill=c] (8.52736,9.71161) rectangle (8.56716,9.81746);
\draw [color=c, fill=c] (8.56716,9.71161) rectangle (8.60697,9.81746);
\draw [color=c, fill=c] (8.60697,9.71161) rectangle (8.64677,9.81746);
\draw [color=c, fill=c] (8.64677,9.71161) rectangle (8.68657,9.81746);
\draw [color=c, fill=c] (8.68657,9.71161) rectangle (8.72637,9.81746);
\draw [color=c, fill=c] (8.72637,9.71161) rectangle (8.76617,9.81746);
\draw [color=c, fill=c] (8.76617,9.71161) rectangle (8.80597,9.81746);
\draw [color=c, fill=c] (8.80597,9.71161) rectangle (8.84577,9.81746);
\draw [color=c, fill=c] (8.84577,9.71161) rectangle (8.88557,9.81746);
\draw [color=c, fill=c] (8.88557,9.71161) rectangle (8.92537,9.81746);
\draw [color=c, fill=c] (8.92537,9.71161) rectangle (8.96517,9.81746);
\draw [color=c, fill=c] (8.96517,9.71161) rectangle (9.00498,9.81746);
\draw [color=c, fill=c] (9.00498,9.71161) rectangle (9.04478,9.81746);
\draw [color=c, fill=c] (9.04478,9.71161) rectangle (9.08458,9.81746);
\draw [color=c, fill=c] (9.08458,9.71161) rectangle (9.12438,9.81746);
\draw [color=c, fill=c] (9.12438,9.71161) rectangle (9.16418,9.81746);
\draw [color=c, fill=c] (9.16418,9.71161) rectangle (9.20398,9.81746);
\draw [color=c, fill=c] (9.20398,9.71161) rectangle (9.24378,9.81746);
\draw [color=c, fill=c] (9.24378,9.71161) rectangle (9.28358,9.81746);
\draw [color=c, fill=c] (9.28358,9.71161) rectangle (9.32338,9.81746);
\draw [color=c, fill=c] (9.32338,9.71161) rectangle (9.36318,9.81746);
\draw [color=c, fill=c] (9.36318,9.71161) rectangle (9.40298,9.81746);
\draw [color=c, fill=c] (9.40298,9.71161) rectangle (9.44279,9.81746);
\draw [color=c, fill=c] (9.44279,9.71161) rectangle (9.48259,9.81746);
\draw [color=c, fill=c] (9.48259,9.71161) rectangle (9.52239,9.81746);
\definecolor{c}{rgb}{0,0.266667,1};
\draw [color=c, fill=c] (9.52239,9.71161) rectangle (9.56219,9.81746);
\draw [color=c, fill=c] (9.56219,9.71161) rectangle (9.60199,9.81746);
\draw [color=c, fill=c] (9.60199,9.71161) rectangle (9.64179,9.81746);
\draw [color=c, fill=c] (9.64179,9.71161) rectangle (9.68159,9.81746);
\draw [color=c, fill=c] (9.68159,9.71161) rectangle (9.72139,9.81746);
\draw [color=c, fill=c] (9.72139,9.71161) rectangle (9.76119,9.81746);
\draw [color=c, fill=c] (9.76119,9.71161) rectangle (9.80099,9.81746);
\draw [color=c, fill=c] (9.80099,9.71161) rectangle (9.8408,9.81746);
\draw [color=c, fill=c] (9.8408,9.71161) rectangle (9.8806,9.81746);
\draw [color=c, fill=c] (9.8806,9.71161) rectangle (9.9204,9.81746);
\draw [color=c, fill=c] (9.9204,9.71161) rectangle (9.9602,9.81746);
\draw [color=c, fill=c] (9.9602,9.71161) rectangle (10,9.81746);
\draw [color=c, fill=c] (10,9.71161) rectangle (10.0398,9.81746);
\draw [color=c, fill=c] (10.0398,9.71161) rectangle (10.0796,9.81746);
\draw [color=c, fill=c] (10.0796,9.71161) rectangle (10.1194,9.81746);
\draw [color=c, fill=c] (10.1194,9.71161) rectangle (10.1592,9.81746);
\draw [color=c, fill=c] (10.1592,9.71161) rectangle (10.199,9.81746);
\draw [color=c, fill=c] (10.199,9.71161) rectangle (10.2388,9.81746);
\draw [color=c, fill=c] (10.2388,9.71161) rectangle (10.2786,9.81746);
\draw [color=c, fill=c] (10.2786,9.71161) rectangle (10.3184,9.81746);
\draw [color=c, fill=c] (10.3184,9.71161) rectangle (10.3582,9.81746);
\draw [color=c, fill=c] (10.3582,9.71161) rectangle (10.398,9.81746);
\draw [color=c, fill=c] (10.398,9.71161) rectangle (10.4378,9.81746);
\draw [color=c, fill=c] (10.4378,9.71161) rectangle (10.4776,9.81746);
\draw [color=c, fill=c] (10.4776,9.71161) rectangle (10.5174,9.81746);
\draw [color=c, fill=c] (10.5174,9.71161) rectangle (10.5572,9.81746);
\draw [color=c, fill=c] (10.5572,9.71161) rectangle (10.597,9.81746);
\draw [color=c, fill=c] (10.597,9.71161) rectangle (10.6368,9.81746);
\draw [color=c, fill=c] (10.6368,9.71161) rectangle (10.6766,9.81746);
\draw [color=c, fill=c] (10.6766,9.71161) rectangle (10.7164,9.81746);
\definecolor{c}{rgb}{0,0.546666,1};
\draw [color=c, fill=c] (10.7164,9.71161) rectangle (10.7562,9.81746);
\draw [color=c, fill=c] (10.7562,9.71161) rectangle (10.796,9.81746);
\draw [color=c, fill=c] (10.796,9.71161) rectangle (10.8358,9.81746);
\draw [color=c, fill=c] (10.8358,9.71161) rectangle (10.8756,9.81746);
\draw [color=c, fill=c] (10.8756,9.71161) rectangle (10.9154,9.81746);
\draw [color=c, fill=c] (10.9154,9.71161) rectangle (10.9552,9.81746);
\draw [color=c, fill=c] (10.9552,9.71161) rectangle (10.995,9.81746);
\draw [color=c, fill=c] (10.995,9.71161) rectangle (11.0348,9.81746);
\draw [color=c, fill=c] (11.0348,9.71161) rectangle (11.0746,9.81746);
\draw [color=c, fill=c] (11.0746,9.71161) rectangle (11.1144,9.81746);
\draw [color=c, fill=c] (11.1144,9.71161) rectangle (11.1542,9.81746);
\draw [color=c, fill=c] (11.1542,9.71161) rectangle (11.194,9.81746);
\draw [color=c, fill=c] (11.194,9.71161) rectangle (11.2338,9.81746);
\draw [color=c, fill=c] (11.2338,9.71161) rectangle (11.2736,9.81746);
\draw [color=c, fill=c] (11.2736,9.71161) rectangle (11.3134,9.81746);
\draw [color=c, fill=c] (11.3134,9.71161) rectangle (11.3532,9.81746);
\draw [color=c, fill=c] (11.3532,9.71161) rectangle (11.393,9.81746);
\draw [color=c, fill=c] (11.393,9.71161) rectangle (11.4328,9.81746);
\draw [color=c, fill=c] (11.4328,9.71161) rectangle (11.4726,9.81746);
\draw [color=c, fill=c] (11.4726,9.71161) rectangle (11.5124,9.81746);
\draw [color=c, fill=c] (11.5124,9.71161) rectangle (11.5522,9.81746);
\draw [color=c, fill=c] (11.5522,9.71161) rectangle (11.592,9.81746);
\draw [color=c, fill=c] (11.592,9.71161) rectangle (11.6318,9.81746);
\draw [color=c, fill=c] (11.6318,9.71161) rectangle (11.6716,9.81746);
\draw [color=c, fill=c] (11.6716,9.71161) rectangle (11.7114,9.81746);
\draw [color=c, fill=c] (11.7114,9.71161) rectangle (11.7512,9.81746);
\draw [color=c, fill=c] (11.7512,9.71161) rectangle (11.791,9.81746);
\draw [color=c, fill=c] (11.791,9.71161) rectangle (11.8308,9.81746);
\draw [color=c, fill=c] (11.8308,9.71161) rectangle (11.8706,9.81746);
\draw [color=c, fill=c] (11.8706,9.71161) rectangle (11.9104,9.81746);
\draw [color=c, fill=c] (11.9104,9.71161) rectangle (11.9502,9.81746);
\draw [color=c, fill=c] (11.9502,9.71161) rectangle (11.99,9.81746);
\draw [color=c, fill=c] (11.99,9.71161) rectangle (12.0299,9.81746);
\draw [color=c, fill=c] (12.0299,9.71161) rectangle (12.0697,9.81746);
\draw [color=c, fill=c] (12.0697,9.71161) rectangle (12.1095,9.81746);
\draw [color=c, fill=c] (12.1095,9.71161) rectangle (12.1493,9.81746);
\draw [color=c, fill=c] (12.1493,9.71161) rectangle (12.1891,9.81746);
\draw [color=c, fill=c] (12.1891,9.71161) rectangle (12.2289,9.81746);
\draw [color=c, fill=c] (12.2289,9.71161) rectangle (12.2687,9.81746);
\draw [color=c, fill=c] (12.2687,9.71161) rectangle (12.3085,9.81746);
\draw [color=c, fill=c] (12.3085,9.71161) rectangle (12.3483,9.81746);
\draw [color=c, fill=c] (12.3483,9.71161) rectangle (12.3881,9.81746);
\draw [color=c, fill=c] (12.3881,9.71161) rectangle (12.4279,9.81746);
\draw [color=c, fill=c] (12.4279,9.71161) rectangle (12.4677,9.81746);
\draw [color=c, fill=c] (12.4677,9.71161) rectangle (12.5075,9.81746);
\draw [color=c, fill=c] (12.5075,9.71161) rectangle (12.5473,9.81746);
\draw [color=c, fill=c] (12.5473,9.71161) rectangle (12.5871,9.81746);
\draw [color=c, fill=c] (12.5871,9.71161) rectangle (12.6269,9.81746);
\draw [color=c, fill=c] (12.6269,9.71161) rectangle (12.6667,9.81746);
\draw [color=c, fill=c] (12.6667,9.71161) rectangle (12.7065,9.81746);
\draw [color=c, fill=c] (12.7065,9.71161) rectangle (12.7463,9.81746);
\draw [color=c, fill=c] (12.7463,9.71161) rectangle (12.7861,9.81746);
\draw [color=c, fill=c] (12.7861,9.71161) rectangle (12.8259,9.81746);
\draw [color=c, fill=c] (12.8259,9.71161) rectangle (12.8657,9.81746);
\draw [color=c, fill=c] (12.8657,9.71161) rectangle (12.9055,9.81746);
\draw [color=c, fill=c] (12.9055,9.71161) rectangle (12.9453,9.81746);
\draw [color=c, fill=c] (12.9453,9.71161) rectangle (12.9851,9.81746);
\draw [color=c, fill=c] (12.9851,9.71161) rectangle (13.0249,9.81746);
\draw [color=c, fill=c] (13.0249,9.71161) rectangle (13.0647,9.81746);
\draw [color=c, fill=c] (13.0647,9.71161) rectangle (13.1045,9.81746);
\draw [color=c, fill=c] (13.1045,9.71161) rectangle (13.1443,9.81746);
\draw [color=c, fill=c] (13.1443,9.71161) rectangle (13.1841,9.81746);
\draw [color=c, fill=c] (13.1841,9.71161) rectangle (13.2239,9.81746);
\draw [color=c, fill=c] (13.2239,9.71161) rectangle (13.2637,9.81746);
\draw [color=c, fill=c] (13.2637,9.71161) rectangle (13.3035,9.81746);
\draw [color=c, fill=c] (13.3035,9.71161) rectangle (13.3433,9.81746);
\draw [color=c, fill=c] (13.3433,9.71161) rectangle (13.3831,9.81746);
\draw [color=c, fill=c] (13.3831,9.71161) rectangle (13.4229,9.81746);
\draw [color=c, fill=c] (13.4229,9.71161) rectangle (13.4627,9.81746);
\draw [color=c, fill=c] (13.4627,9.71161) rectangle (13.5025,9.81746);
\draw [color=c, fill=c] (13.5025,9.71161) rectangle (13.5423,9.81746);
\draw [color=c, fill=c] (13.5423,9.71161) rectangle (13.5821,9.81746);
\draw [color=c, fill=c] (13.5821,9.71161) rectangle (13.6219,9.81746);
\draw [color=c, fill=c] (13.6219,9.71161) rectangle (13.6617,9.81746);
\draw [color=c, fill=c] (13.6617,9.71161) rectangle (13.7015,9.81746);
\definecolor{c}{rgb}{0,0.733333,1};
\draw [color=c, fill=c] (13.7015,9.71161) rectangle (13.7413,9.81746);
\draw [color=c, fill=c] (13.7413,9.71161) rectangle (13.7811,9.81746);
\draw [color=c, fill=c] (13.7811,9.71161) rectangle (13.8209,9.81746);
\draw [color=c, fill=c] (13.8209,9.71161) rectangle (13.8607,9.81746);
\draw [color=c, fill=c] (13.8607,9.71161) rectangle (13.9005,9.81746);
\draw [color=c, fill=c] (13.9005,9.71161) rectangle (13.9403,9.81746);
\draw [color=c, fill=c] (13.9403,9.71161) rectangle (13.9801,9.81746);
\draw [color=c, fill=c] (13.9801,9.71161) rectangle (14.0199,9.81746);
\draw [color=c, fill=c] (14.0199,9.71161) rectangle (14.0597,9.81746);
\draw [color=c, fill=c] (14.0597,9.71161) rectangle (14.0995,9.81746);
\draw [color=c, fill=c] (14.0995,9.71161) rectangle (14.1393,9.81746);
\draw [color=c, fill=c] (14.1393,9.71161) rectangle (14.1791,9.81746);
\draw [color=c, fill=c] (14.1791,9.71161) rectangle (14.2189,9.81746);
\draw [color=c, fill=c] (14.2189,9.71161) rectangle (14.2587,9.81746);
\draw [color=c, fill=c] (14.2587,9.71161) rectangle (14.2985,9.81746);
\draw [color=c, fill=c] (14.2985,9.71161) rectangle (14.3383,9.81746);
\draw [color=c, fill=c] (14.3383,9.71161) rectangle (14.3781,9.81746);
\draw [color=c, fill=c] (14.3781,9.71161) rectangle (14.4179,9.81746);
\draw [color=c, fill=c] (14.4179,9.71161) rectangle (14.4577,9.81746);
\draw [color=c, fill=c] (14.4577,9.71161) rectangle (14.4975,9.81746);
\draw [color=c, fill=c] (14.4975,9.71161) rectangle (14.5373,9.81746);
\draw [color=c, fill=c] (14.5373,9.71161) rectangle (14.5771,9.81746);
\draw [color=c, fill=c] (14.5771,9.71161) rectangle (14.6169,9.81746);
\draw [color=c, fill=c] (14.6169,9.71161) rectangle (14.6567,9.81746);
\draw [color=c, fill=c] (14.6567,9.71161) rectangle (14.6965,9.81746);
\draw [color=c, fill=c] (14.6965,9.71161) rectangle (14.7363,9.81746);
\draw [color=c, fill=c] (14.7363,9.71161) rectangle (14.7761,9.81746);
\draw [color=c, fill=c] (14.7761,9.71161) rectangle (14.8159,9.81746);
\draw [color=c, fill=c] (14.8159,9.71161) rectangle (14.8557,9.81746);
\draw [color=c, fill=c] (14.8557,9.71161) rectangle (14.8955,9.81746);
\draw [color=c, fill=c] (14.8955,9.71161) rectangle (14.9353,9.81746);
\draw [color=c, fill=c] (14.9353,9.71161) rectangle (14.9751,9.81746);
\draw [color=c, fill=c] (14.9751,9.71161) rectangle (15.0149,9.81746);
\draw [color=c, fill=c] (15.0149,9.71161) rectangle (15.0547,9.81746);
\draw [color=c, fill=c] (15.0547,9.71161) rectangle (15.0945,9.81746);
\draw [color=c, fill=c] (15.0945,9.71161) rectangle (15.1343,9.81746);
\draw [color=c, fill=c] (15.1343,9.71161) rectangle (15.1741,9.81746);
\draw [color=c, fill=c] (15.1741,9.71161) rectangle (15.2139,9.81746);
\draw [color=c, fill=c] (15.2139,9.71161) rectangle (15.2537,9.81746);
\draw [color=c, fill=c] (15.2537,9.71161) rectangle (15.2935,9.81746);
\draw [color=c, fill=c] (15.2935,9.71161) rectangle (15.3333,9.81746);
\draw [color=c, fill=c] (15.3333,9.71161) rectangle (15.3731,9.81746);
\draw [color=c, fill=c] (15.3731,9.71161) rectangle (15.4129,9.81746);
\draw [color=c, fill=c] (15.4129,9.71161) rectangle (15.4527,9.81746);
\draw [color=c, fill=c] (15.4527,9.71161) rectangle (15.4925,9.81746);
\draw [color=c, fill=c] (15.4925,9.71161) rectangle (15.5323,9.81746);
\draw [color=c, fill=c] (15.5323,9.71161) rectangle (15.5721,9.81746);
\draw [color=c, fill=c] (15.5721,9.71161) rectangle (15.6119,9.81746);
\draw [color=c, fill=c] (15.6119,9.71161) rectangle (15.6517,9.81746);
\draw [color=c, fill=c] (15.6517,9.71161) rectangle (15.6915,9.81746);
\draw [color=c, fill=c] (15.6915,9.71161) rectangle (15.7313,9.81746);
\draw [color=c, fill=c] (15.7313,9.71161) rectangle (15.7711,9.81746);
\draw [color=c, fill=c] (15.7711,9.71161) rectangle (15.8109,9.81746);
\draw [color=c, fill=c] (15.8109,9.71161) rectangle (15.8507,9.81746);
\draw [color=c, fill=c] (15.8507,9.71161) rectangle (15.8905,9.81746);
\draw [color=c, fill=c] (15.8905,9.71161) rectangle (15.9303,9.81746);
\draw [color=c, fill=c] (15.9303,9.71161) rectangle (15.9701,9.81746);
\draw [color=c, fill=c] (15.9701,9.71161) rectangle (16.01,9.81746);
\draw [color=c, fill=c] (16.01,9.71161) rectangle (16.0498,9.81746);
\draw [color=c, fill=c] (16.0498,9.71161) rectangle (16.0896,9.81746);
\draw [color=c, fill=c] (16.0896,9.71161) rectangle (16.1294,9.81746);
\draw [color=c, fill=c] (16.1294,9.71161) rectangle (16.1692,9.81746);
\draw [color=c, fill=c] (16.1692,9.71161) rectangle (16.209,9.81746);
\draw [color=c, fill=c] (16.209,9.71161) rectangle (16.2488,9.81746);
\draw [color=c, fill=c] (16.2488,9.71161) rectangle (16.2886,9.81746);
\draw [color=c, fill=c] (16.2886,9.71161) rectangle (16.3284,9.81746);
\draw [color=c, fill=c] (16.3284,9.71161) rectangle (16.3682,9.81746);
\draw [color=c, fill=c] (16.3682,9.71161) rectangle (16.408,9.81746);
\draw [color=c, fill=c] (16.408,9.71161) rectangle (16.4478,9.81746);
\draw [color=c, fill=c] (16.4478,9.71161) rectangle (16.4876,9.81746);
\draw [color=c, fill=c] (16.4876,9.71161) rectangle (16.5274,9.81746);
\draw [color=c, fill=c] (16.5274,9.71161) rectangle (16.5672,9.81746);
\draw [color=c, fill=c] (16.5672,9.71161) rectangle (16.607,9.81746);
\draw [color=c, fill=c] (16.607,9.71161) rectangle (16.6468,9.81746);
\draw [color=c, fill=c] (16.6468,9.71161) rectangle (16.6866,9.81746);
\draw [color=c, fill=c] (16.6866,9.71161) rectangle (16.7264,9.81746);
\draw [color=c, fill=c] (16.7264,9.71161) rectangle (16.7662,9.81746);
\draw [color=c, fill=c] (16.7662,9.71161) rectangle (16.806,9.81746);
\draw [color=c, fill=c] (16.806,9.71161) rectangle (16.8458,9.81746);
\draw [color=c, fill=c] (16.8458,9.71161) rectangle (16.8856,9.81746);
\draw [color=c, fill=c] (16.8856,9.71161) rectangle (16.9254,9.81746);
\draw [color=c, fill=c] (16.9254,9.71161) rectangle (16.9652,9.81746);
\draw [color=c, fill=c] (16.9652,9.71161) rectangle (17.005,9.81746);
\draw [color=c, fill=c] (17.005,9.71161) rectangle (17.0448,9.81746);
\draw [color=c, fill=c] (17.0448,9.71161) rectangle (17.0846,9.81746);
\draw [color=c, fill=c] (17.0846,9.71161) rectangle (17.1244,9.81746);
\draw [color=c, fill=c] (17.1244,9.71161) rectangle (17.1642,9.81746);
\draw [color=c, fill=c] (17.1642,9.71161) rectangle (17.204,9.81746);
\draw [color=c, fill=c] (17.204,9.71161) rectangle (17.2438,9.81746);
\draw [color=c, fill=c] (17.2438,9.71161) rectangle (17.2836,9.81746);
\draw [color=c, fill=c] (17.2836,9.71161) rectangle (17.3234,9.81746);
\draw [color=c, fill=c] (17.3234,9.71161) rectangle (17.3632,9.81746);
\draw [color=c, fill=c] (17.3632,9.71161) rectangle (17.403,9.81746);
\draw [color=c, fill=c] (17.403,9.71161) rectangle (17.4428,9.81746);
\draw [color=c, fill=c] (17.4428,9.71161) rectangle (17.4826,9.81746);
\draw [color=c, fill=c] (17.4826,9.71161) rectangle (17.5224,9.81746);
\draw [color=c, fill=c] (17.5224,9.71161) rectangle (17.5622,9.81746);
\draw [color=c, fill=c] (17.5622,9.71161) rectangle (17.602,9.81746);
\draw [color=c, fill=c] (17.602,9.71161) rectangle (17.6418,9.81746);
\draw [color=c, fill=c] (17.6418,9.71161) rectangle (17.6816,9.81746);
\draw [color=c, fill=c] (17.6816,9.71161) rectangle (17.7214,9.81746);
\draw [color=c, fill=c] (17.7214,9.71161) rectangle (17.7612,9.81746);
\draw [color=c, fill=c] (17.7612,9.71161) rectangle (17.801,9.81746);
\draw [color=c, fill=c] (17.801,9.71161) rectangle (17.8408,9.81746);
\draw [color=c, fill=c] (17.8408,9.71161) rectangle (17.8806,9.81746);
\draw [color=c, fill=c] (17.8806,9.71161) rectangle (17.9204,9.81746);
\draw [color=c, fill=c] (17.9204,9.71161) rectangle (17.9602,9.81746);
\draw [color=c, fill=c] (17.9602,9.71161) rectangle (18,9.81746);
\definecolor{c}{rgb}{0.2,0,1};
\draw [color=c, fill=c] (2,9.81746) rectangle (2.0398,9.92331);
\draw [color=c, fill=c] (2.0398,9.81746) rectangle (2.0796,9.92331);
\draw [color=c, fill=c] (2.0796,9.81746) rectangle (2.1194,9.92331);
\draw [color=c, fill=c] (2.1194,9.81746) rectangle (2.1592,9.92331);
\draw [color=c, fill=c] (2.1592,9.81746) rectangle (2.19901,9.92331);
\draw [color=c, fill=c] (2.19901,9.81746) rectangle (2.23881,9.92331);
\draw [color=c, fill=c] (2.23881,9.81746) rectangle (2.27861,9.92331);
\draw [color=c, fill=c] (2.27861,9.81746) rectangle (2.31841,9.92331);
\draw [color=c, fill=c] (2.31841,9.81746) rectangle (2.35821,9.92331);
\draw [color=c, fill=c] (2.35821,9.81746) rectangle (2.39801,9.92331);
\draw [color=c, fill=c] (2.39801,9.81746) rectangle (2.43781,9.92331);
\draw [color=c, fill=c] (2.43781,9.81746) rectangle (2.47761,9.92331);
\draw [color=c, fill=c] (2.47761,9.81746) rectangle (2.51741,9.92331);
\draw [color=c, fill=c] (2.51741,9.81746) rectangle (2.55721,9.92331);
\draw [color=c, fill=c] (2.55721,9.81746) rectangle (2.59702,9.92331);
\draw [color=c, fill=c] (2.59702,9.81746) rectangle (2.63682,9.92331);
\draw [color=c, fill=c] (2.63682,9.81746) rectangle (2.67662,9.92331);
\draw [color=c, fill=c] (2.67662,9.81746) rectangle (2.71642,9.92331);
\draw [color=c, fill=c] (2.71642,9.81746) rectangle (2.75622,9.92331);
\draw [color=c, fill=c] (2.75622,9.81746) rectangle (2.79602,9.92331);
\draw [color=c, fill=c] (2.79602,9.81746) rectangle (2.83582,9.92331);
\draw [color=c, fill=c] (2.83582,9.81746) rectangle (2.87562,9.92331);
\draw [color=c, fill=c] (2.87562,9.81746) rectangle (2.91542,9.92331);
\draw [color=c, fill=c] (2.91542,9.81746) rectangle (2.95522,9.92331);
\draw [color=c, fill=c] (2.95522,9.81746) rectangle (2.99502,9.92331);
\draw [color=c, fill=c] (2.99502,9.81746) rectangle (3.03483,9.92331);
\draw [color=c, fill=c] (3.03483,9.81746) rectangle (3.07463,9.92331);
\draw [color=c, fill=c] (3.07463,9.81746) rectangle (3.11443,9.92331);
\draw [color=c, fill=c] (3.11443,9.81746) rectangle (3.15423,9.92331);
\draw [color=c, fill=c] (3.15423,9.81746) rectangle (3.19403,9.92331);
\draw [color=c, fill=c] (3.19403,9.81746) rectangle (3.23383,9.92331);
\draw [color=c, fill=c] (3.23383,9.81746) rectangle (3.27363,9.92331);
\draw [color=c, fill=c] (3.27363,9.81746) rectangle (3.31343,9.92331);
\draw [color=c, fill=c] (3.31343,9.81746) rectangle (3.35323,9.92331);
\draw [color=c, fill=c] (3.35323,9.81746) rectangle (3.39303,9.92331);
\draw [color=c, fill=c] (3.39303,9.81746) rectangle (3.43284,9.92331);
\draw [color=c, fill=c] (3.43284,9.81746) rectangle (3.47264,9.92331);
\draw [color=c, fill=c] (3.47264,9.81746) rectangle (3.51244,9.92331);
\draw [color=c, fill=c] (3.51244,9.81746) rectangle (3.55224,9.92331);
\draw [color=c, fill=c] (3.55224,9.81746) rectangle (3.59204,9.92331);
\draw [color=c, fill=c] (3.59204,9.81746) rectangle (3.63184,9.92331);
\draw [color=c, fill=c] (3.63184,9.81746) rectangle (3.67164,9.92331);
\draw [color=c, fill=c] (3.67164,9.81746) rectangle (3.71144,9.92331);
\draw [color=c, fill=c] (3.71144,9.81746) rectangle (3.75124,9.92331);
\draw [color=c, fill=c] (3.75124,9.81746) rectangle (3.79104,9.92331);
\draw [color=c, fill=c] (3.79104,9.81746) rectangle (3.83085,9.92331);
\draw [color=c, fill=c] (3.83085,9.81746) rectangle (3.87065,9.92331);
\draw [color=c, fill=c] (3.87065,9.81746) rectangle (3.91045,9.92331);
\draw [color=c, fill=c] (3.91045,9.81746) rectangle (3.95025,9.92331);
\draw [color=c, fill=c] (3.95025,9.81746) rectangle (3.99005,9.92331);
\draw [color=c, fill=c] (3.99005,9.81746) rectangle (4.02985,9.92331);
\draw [color=c, fill=c] (4.02985,9.81746) rectangle (4.06965,9.92331);
\draw [color=c, fill=c] (4.06965,9.81746) rectangle (4.10945,9.92331);
\draw [color=c, fill=c] (4.10945,9.81746) rectangle (4.14925,9.92331);
\draw [color=c, fill=c] (4.14925,9.81746) rectangle (4.18905,9.92331);
\draw [color=c, fill=c] (4.18905,9.81746) rectangle (4.22886,9.92331);
\draw [color=c, fill=c] (4.22886,9.81746) rectangle (4.26866,9.92331);
\draw [color=c, fill=c] (4.26866,9.81746) rectangle (4.30846,9.92331);
\draw [color=c, fill=c] (4.30846,9.81746) rectangle (4.34826,9.92331);
\draw [color=c, fill=c] (4.34826,9.81746) rectangle (4.38806,9.92331);
\draw [color=c, fill=c] (4.38806,9.81746) rectangle (4.42786,9.92331);
\draw [color=c, fill=c] (4.42786,9.81746) rectangle (4.46766,9.92331);
\draw [color=c, fill=c] (4.46766,9.81746) rectangle (4.50746,9.92331);
\draw [color=c, fill=c] (4.50746,9.81746) rectangle (4.54726,9.92331);
\draw [color=c, fill=c] (4.54726,9.81746) rectangle (4.58706,9.92331);
\draw [color=c, fill=c] (4.58706,9.81746) rectangle (4.62687,9.92331);
\draw [color=c, fill=c] (4.62687,9.81746) rectangle (4.66667,9.92331);
\draw [color=c, fill=c] (4.66667,9.81746) rectangle (4.70647,9.92331);
\draw [color=c, fill=c] (4.70647,9.81746) rectangle (4.74627,9.92331);
\draw [color=c, fill=c] (4.74627,9.81746) rectangle (4.78607,9.92331);
\draw [color=c, fill=c] (4.78607,9.81746) rectangle (4.82587,9.92331);
\draw [color=c, fill=c] (4.82587,9.81746) rectangle (4.86567,9.92331);
\draw [color=c, fill=c] (4.86567,9.81746) rectangle (4.90547,9.92331);
\draw [color=c, fill=c] (4.90547,9.81746) rectangle (4.94527,9.92331);
\draw [color=c, fill=c] (4.94527,9.81746) rectangle (4.98507,9.92331);
\draw [color=c, fill=c] (4.98507,9.81746) rectangle (5.02488,9.92331);
\draw [color=c, fill=c] (5.02488,9.81746) rectangle (5.06468,9.92331);
\draw [color=c, fill=c] (5.06468,9.81746) rectangle (5.10448,9.92331);
\draw [color=c, fill=c] (5.10448,9.81746) rectangle (5.14428,9.92331);
\draw [color=c, fill=c] (5.14428,9.81746) rectangle (5.18408,9.92331);
\draw [color=c, fill=c] (5.18408,9.81746) rectangle (5.22388,9.92331);
\draw [color=c, fill=c] (5.22388,9.81746) rectangle (5.26368,9.92331);
\draw [color=c, fill=c] (5.26368,9.81746) rectangle (5.30348,9.92331);
\draw [color=c, fill=c] (5.30348,9.81746) rectangle (5.34328,9.92331);
\draw [color=c, fill=c] (5.34328,9.81746) rectangle (5.38308,9.92331);
\draw [color=c, fill=c] (5.38308,9.81746) rectangle (5.42289,9.92331);
\draw [color=c, fill=c] (5.42289,9.81746) rectangle (5.46269,9.92331);
\draw [color=c, fill=c] (5.46269,9.81746) rectangle (5.50249,9.92331);
\draw [color=c, fill=c] (5.50249,9.81746) rectangle (5.54229,9.92331);
\draw [color=c, fill=c] (5.54229,9.81746) rectangle (5.58209,9.92331);
\draw [color=c, fill=c] (5.58209,9.81746) rectangle (5.62189,9.92331);
\draw [color=c, fill=c] (5.62189,9.81746) rectangle (5.66169,9.92331);
\draw [color=c, fill=c] (5.66169,9.81746) rectangle (5.70149,9.92331);
\draw [color=c, fill=c] (5.70149,9.81746) rectangle (5.74129,9.92331);
\draw [color=c, fill=c] (5.74129,9.81746) rectangle (5.78109,9.92331);
\draw [color=c, fill=c] (5.78109,9.81746) rectangle (5.8209,9.92331);
\draw [color=c, fill=c] (5.8209,9.81746) rectangle (5.8607,9.92331);
\draw [color=c, fill=c] (5.8607,9.81746) rectangle (5.9005,9.92331);
\draw [color=c, fill=c] (5.9005,9.81746) rectangle (5.9403,9.92331);
\draw [color=c, fill=c] (5.9403,9.81746) rectangle (5.9801,9.92331);
\draw [color=c, fill=c] (5.9801,9.81746) rectangle (6.0199,9.92331);
\draw [color=c, fill=c] (6.0199,9.81746) rectangle (6.0597,9.92331);
\draw [color=c, fill=c] (6.0597,9.81746) rectangle (6.0995,9.92331);
\draw [color=c, fill=c] (6.0995,9.81746) rectangle (6.1393,9.92331);
\draw [color=c, fill=c] (6.1393,9.81746) rectangle (6.1791,9.92331);
\draw [color=c, fill=c] (6.1791,9.81746) rectangle (6.21891,9.92331);
\draw [color=c, fill=c] (6.21891,9.81746) rectangle (6.25871,9.92331);
\draw [color=c, fill=c] (6.25871,9.81746) rectangle (6.29851,9.92331);
\draw [color=c, fill=c] (6.29851,9.81746) rectangle (6.33831,9.92331);
\draw [color=c, fill=c] (6.33831,9.81746) rectangle (6.37811,9.92331);
\draw [color=c, fill=c] (6.37811,9.81746) rectangle (6.41791,9.92331);
\draw [color=c, fill=c] (6.41791,9.81746) rectangle (6.45771,9.92331);
\draw [color=c, fill=c] (6.45771,9.81746) rectangle (6.49751,9.92331);
\draw [color=c, fill=c] (6.49751,9.81746) rectangle (6.53731,9.92331);
\draw [color=c, fill=c] (6.53731,9.81746) rectangle (6.57711,9.92331);
\draw [color=c, fill=c] (6.57711,9.81746) rectangle (6.61692,9.92331);
\draw [color=c, fill=c] (6.61692,9.81746) rectangle (6.65672,9.92331);
\draw [color=c, fill=c] (6.65672,9.81746) rectangle (6.69652,9.92331);
\draw [color=c, fill=c] (6.69652,9.81746) rectangle (6.73632,9.92331);
\draw [color=c, fill=c] (6.73632,9.81746) rectangle (6.77612,9.92331);
\draw [color=c, fill=c] (6.77612,9.81746) rectangle (6.81592,9.92331);
\draw [color=c, fill=c] (6.81592,9.81746) rectangle (6.85572,9.92331);
\draw [color=c, fill=c] (6.85572,9.81746) rectangle (6.89552,9.92331);
\draw [color=c, fill=c] (6.89552,9.81746) rectangle (6.93532,9.92331);
\draw [color=c, fill=c] (6.93532,9.81746) rectangle (6.97512,9.92331);
\draw [color=c, fill=c] (6.97512,9.81746) rectangle (7.01493,9.92331);
\draw [color=c, fill=c] (7.01493,9.81746) rectangle (7.05473,9.92331);
\draw [color=c, fill=c] (7.05473,9.81746) rectangle (7.09453,9.92331);
\draw [color=c, fill=c] (7.09453,9.81746) rectangle (7.13433,9.92331);
\draw [color=c, fill=c] (7.13433,9.81746) rectangle (7.17413,9.92331);
\draw [color=c, fill=c] (7.17413,9.81746) rectangle (7.21393,9.92331);
\draw [color=c, fill=c] (7.21393,9.81746) rectangle (7.25373,9.92331);
\draw [color=c, fill=c] (7.25373,9.81746) rectangle (7.29353,9.92331);
\draw [color=c, fill=c] (7.29353,9.81746) rectangle (7.33333,9.92331);
\draw [color=c, fill=c] (7.33333,9.81746) rectangle (7.37313,9.92331);
\draw [color=c, fill=c] (7.37313,9.81746) rectangle (7.41294,9.92331);
\draw [color=c, fill=c] (7.41294,9.81746) rectangle (7.45274,9.92331);
\draw [color=c, fill=c] (7.45274,9.81746) rectangle (7.49254,9.92331);
\draw [color=c, fill=c] (7.49254,9.81746) rectangle (7.53234,9.92331);
\draw [color=c, fill=c] (7.53234,9.81746) rectangle (7.57214,9.92331);
\draw [color=c, fill=c] (7.57214,9.81746) rectangle (7.61194,9.92331);
\draw [color=c, fill=c] (7.61194,9.81746) rectangle (7.65174,9.92331);
\draw [color=c, fill=c] (7.65174,9.81746) rectangle (7.69154,9.92331);
\definecolor{c}{rgb}{0,0.0800001,1};
\draw [color=c, fill=c] (7.69154,9.81746) rectangle (7.73134,9.92331);
\draw [color=c, fill=c] (7.73134,9.81746) rectangle (7.77114,9.92331);
\draw [color=c, fill=c] (7.77114,9.81746) rectangle (7.81095,9.92331);
\draw [color=c, fill=c] (7.81095,9.81746) rectangle (7.85075,9.92331);
\draw [color=c, fill=c] (7.85075,9.81746) rectangle (7.89055,9.92331);
\draw [color=c, fill=c] (7.89055,9.81746) rectangle (7.93035,9.92331);
\draw [color=c, fill=c] (7.93035,9.81746) rectangle (7.97015,9.92331);
\draw [color=c, fill=c] (7.97015,9.81746) rectangle (8.00995,9.92331);
\draw [color=c, fill=c] (8.00995,9.81746) rectangle (8.04975,9.92331);
\draw [color=c, fill=c] (8.04975,9.81746) rectangle (8.08955,9.92331);
\draw [color=c, fill=c] (8.08955,9.81746) rectangle (8.12935,9.92331);
\draw [color=c, fill=c] (8.12935,9.81746) rectangle (8.16915,9.92331);
\draw [color=c, fill=c] (8.16915,9.81746) rectangle (8.20895,9.92331);
\draw [color=c, fill=c] (8.20895,9.81746) rectangle (8.24876,9.92331);
\draw [color=c, fill=c] (8.24876,9.81746) rectangle (8.28856,9.92331);
\draw [color=c, fill=c] (8.28856,9.81746) rectangle (8.32836,9.92331);
\draw [color=c, fill=c] (8.32836,9.81746) rectangle (8.36816,9.92331);
\draw [color=c, fill=c] (8.36816,9.81746) rectangle (8.40796,9.92331);
\draw [color=c, fill=c] (8.40796,9.81746) rectangle (8.44776,9.92331);
\draw [color=c, fill=c] (8.44776,9.81746) rectangle (8.48756,9.92331);
\draw [color=c, fill=c] (8.48756,9.81746) rectangle (8.52736,9.92331);
\draw [color=c, fill=c] (8.52736,9.81746) rectangle (8.56716,9.92331);
\draw [color=c, fill=c] (8.56716,9.81746) rectangle (8.60697,9.92331);
\draw [color=c, fill=c] (8.60697,9.81746) rectangle (8.64677,9.92331);
\draw [color=c, fill=c] (8.64677,9.81746) rectangle (8.68657,9.92331);
\draw [color=c, fill=c] (8.68657,9.81746) rectangle (8.72637,9.92331);
\draw [color=c, fill=c] (8.72637,9.81746) rectangle (8.76617,9.92331);
\draw [color=c, fill=c] (8.76617,9.81746) rectangle (8.80597,9.92331);
\draw [color=c, fill=c] (8.80597,9.81746) rectangle (8.84577,9.92331);
\draw [color=c, fill=c] (8.84577,9.81746) rectangle (8.88557,9.92331);
\draw [color=c, fill=c] (8.88557,9.81746) rectangle (8.92537,9.92331);
\draw [color=c, fill=c] (8.92537,9.81746) rectangle (8.96517,9.92331);
\draw [color=c, fill=c] (8.96517,9.81746) rectangle (9.00498,9.92331);
\draw [color=c, fill=c] (9.00498,9.81746) rectangle (9.04478,9.92331);
\draw [color=c, fill=c] (9.04478,9.81746) rectangle (9.08458,9.92331);
\draw [color=c, fill=c] (9.08458,9.81746) rectangle (9.12438,9.92331);
\draw [color=c, fill=c] (9.12438,9.81746) rectangle (9.16418,9.92331);
\draw [color=c, fill=c] (9.16418,9.81746) rectangle (9.20398,9.92331);
\draw [color=c, fill=c] (9.20398,9.81746) rectangle (9.24378,9.92331);
\draw [color=c, fill=c] (9.24378,9.81746) rectangle (9.28358,9.92331);
\draw [color=c, fill=c] (9.28358,9.81746) rectangle (9.32338,9.92331);
\draw [color=c, fill=c] (9.32338,9.81746) rectangle (9.36318,9.92331);
\draw [color=c, fill=c] (9.36318,9.81746) rectangle (9.40298,9.92331);
\draw [color=c, fill=c] (9.40298,9.81746) rectangle (9.44279,9.92331);
\draw [color=c, fill=c] (9.44279,9.81746) rectangle (9.48259,9.92331);
\draw [color=c, fill=c] (9.48259,9.81746) rectangle (9.52239,9.92331);
\definecolor{c}{rgb}{0,0.266667,1};
\draw [color=c, fill=c] (9.52239,9.81746) rectangle (9.56219,9.92331);
\draw [color=c, fill=c] (9.56219,9.81746) rectangle (9.60199,9.92331);
\draw [color=c, fill=c] (9.60199,9.81746) rectangle (9.64179,9.92331);
\draw [color=c, fill=c] (9.64179,9.81746) rectangle (9.68159,9.92331);
\draw [color=c, fill=c] (9.68159,9.81746) rectangle (9.72139,9.92331);
\draw [color=c, fill=c] (9.72139,9.81746) rectangle (9.76119,9.92331);
\draw [color=c, fill=c] (9.76119,9.81746) rectangle (9.80099,9.92331);
\draw [color=c, fill=c] (9.80099,9.81746) rectangle (9.8408,9.92331);
\draw [color=c, fill=c] (9.8408,9.81746) rectangle (9.8806,9.92331);
\draw [color=c, fill=c] (9.8806,9.81746) rectangle (9.9204,9.92331);
\draw [color=c, fill=c] (9.9204,9.81746) rectangle (9.9602,9.92331);
\draw [color=c, fill=c] (9.9602,9.81746) rectangle (10,9.92331);
\draw [color=c, fill=c] (10,9.81746) rectangle (10.0398,9.92331);
\draw [color=c, fill=c] (10.0398,9.81746) rectangle (10.0796,9.92331);
\draw [color=c, fill=c] (10.0796,9.81746) rectangle (10.1194,9.92331);
\draw [color=c, fill=c] (10.1194,9.81746) rectangle (10.1592,9.92331);
\draw [color=c, fill=c] (10.1592,9.81746) rectangle (10.199,9.92331);
\draw [color=c, fill=c] (10.199,9.81746) rectangle (10.2388,9.92331);
\draw [color=c, fill=c] (10.2388,9.81746) rectangle (10.2786,9.92331);
\draw [color=c, fill=c] (10.2786,9.81746) rectangle (10.3184,9.92331);
\draw [color=c, fill=c] (10.3184,9.81746) rectangle (10.3582,9.92331);
\draw [color=c, fill=c] (10.3582,9.81746) rectangle (10.398,9.92331);
\draw [color=c, fill=c] (10.398,9.81746) rectangle (10.4378,9.92331);
\draw [color=c, fill=c] (10.4378,9.81746) rectangle (10.4776,9.92331);
\draw [color=c, fill=c] (10.4776,9.81746) rectangle (10.5174,9.92331);
\draw [color=c, fill=c] (10.5174,9.81746) rectangle (10.5572,9.92331);
\draw [color=c, fill=c] (10.5572,9.81746) rectangle (10.597,9.92331);
\draw [color=c, fill=c] (10.597,9.81746) rectangle (10.6368,9.92331);
\draw [color=c, fill=c] (10.6368,9.81746) rectangle (10.6766,9.92331);
\draw [color=c, fill=c] (10.6766,9.81746) rectangle (10.7164,9.92331);
\draw [color=c, fill=c] (10.7164,9.81746) rectangle (10.7562,9.92331);
\definecolor{c}{rgb}{0,0.546666,1};
\draw [color=c, fill=c] (10.7562,9.81746) rectangle (10.796,9.92331);
\draw [color=c, fill=c] (10.796,9.81746) rectangle (10.8358,9.92331);
\draw [color=c, fill=c] (10.8358,9.81746) rectangle (10.8756,9.92331);
\draw [color=c, fill=c] (10.8756,9.81746) rectangle (10.9154,9.92331);
\draw [color=c, fill=c] (10.9154,9.81746) rectangle (10.9552,9.92331);
\draw [color=c, fill=c] (10.9552,9.81746) rectangle (10.995,9.92331);
\draw [color=c, fill=c] (10.995,9.81746) rectangle (11.0348,9.92331);
\draw [color=c, fill=c] (11.0348,9.81746) rectangle (11.0746,9.92331);
\draw [color=c, fill=c] (11.0746,9.81746) rectangle (11.1144,9.92331);
\draw [color=c, fill=c] (11.1144,9.81746) rectangle (11.1542,9.92331);
\draw [color=c, fill=c] (11.1542,9.81746) rectangle (11.194,9.92331);
\draw [color=c, fill=c] (11.194,9.81746) rectangle (11.2338,9.92331);
\draw [color=c, fill=c] (11.2338,9.81746) rectangle (11.2736,9.92331);
\draw [color=c, fill=c] (11.2736,9.81746) rectangle (11.3134,9.92331);
\draw [color=c, fill=c] (11.3134,9.81746) rectangle (11.3532,9.92331);
\draw [color=c, fill=c] (11.3532,9.81746) rectangle (11.393,9.92331);
\draw [color=c, fill=c] (11.393,9.81746) rectangle (11.4328,9.92331);
\draw [color=c, fill=c] (11.4328,9.81746) rectangle (11.4726,9.92331);
\draw [color=c, fill=c] (11.4726,9.81746) rectangle (11.5124,9.92331);
\draw [color=c, fill=c] (11.5124,9.81746) rectangle (11.5522,9.92331);
\draw [color=c, fill=c] (11.5522,9.81746) rectangle (11.592,9.92331);
\draw [color=c, fill=c] (11.592,9.81746) rectangle (11.6318,9.92331);
\draw [color=c, fill=c] (11.6318,9.81746) rectangle (11.6716,9.92331);
\draw [color=c, fill=c] (11.6716,9.81746) rectangle (11.7114,9.92331);
\draw [color=c, fill=c] (11.7114,9.81746) rectangle (11.7512,9.92331);
\draw [color=c, fill=c] (11.7512,9.81746) rectangle (11.791,9.92331);
\draw [color=c, fill=c] (11.791,9.81746) rectangle (11.8308,9.92331);
\draw [color=c, fill=c] (11.8308,9.81746) rectangle (11.8706,9.92331);
\draw [color=c, fill=c] (11.8706,9.81746) rectangle (11.9104,9.92331);
\draw [color=c, fill=c] (11.9104,9.81746) rectangle (11.9502,9.92331);
\draw [color=c, fill=c] (11.9502,9.81746) rectangle (11.99,9.92331);
\draw [color=c, fill=c] (11.99,9.81746) rectangle (12.0299,9.92331);
\draw [color=c, fill=c] (12.0299,9.81746) rectangle (12.0697,9.92331);
\draw [color=c, fill=c] (12.0697,9.81746) rectangle (12.1095,9.92331);
\draw [color=c, fill=c] (12.1095,9.81746) rectangle (12.1493,9.92331);
\draw [color=c, fill=c] (12.1493,9.81746) rectangle (12.1891,9.92331);
\draw [color=c, fill=c] (12.1891,9.81746) rectangle (12.2289,9.92331);
\draw [color=c, fill=c] (12.2289,9.81746) rectangle (12.2687,9.92331);
\draw [color=c, fill=c] (12.2687,9.81746) rectangle (12.3085,9.92331);
\draw [color=c, fill=c] (12.3085,9.81746) rectangle (12.3483,9.92331);
\draw [color=c, fill=c] (12.3483,9.81746) rectangle (12.3881,9.92331);
\draw [color=c, fill=c] (12.3881,9.81746) rectangle (12.4279,9.92331);
\draw [color=c, fill=c] (12.4279,9.81746) rectangle (12.4677,9.92331);
\draw [color=c, fill=c] (12.4677,9.81746) rectangle (12.5075,9.92331);
\draw [color=c, fill=c] (12.5075,9.81746) rectangle (12.5473,9.92331);
\draw [color=c, fill=c] (12.5473,9.81746) rectangle (12.5871,9.92331);
\draw [color=c, fill=c] (12.5871,9.81746) rectangle (12.6269,9.92331);
\draw [color=c, fill=c] (12.6269,9.81746) rectangle (12.6667,9.92331);
\draw [color=c, fill=c] (12.6667,9.81746) rectangle (12.7065,9.92331);
\draw [color=c, fill=c] (12.7065,9.81746) rectangle (12.7463,9.92331);
\draw [color=c, fill=c] (12.7463,9.81746) rectangle (12.7861,9.92331);
\draw [color=c, fill=c] (12.7861,9.81746) rectangle (12.8259,9.92331);
\draw [color=c, fill=c] (12.8259,9.81746) rectangle (12.8657,9.92331);
\draw [color=c, fill=c] (12.8657,9.81746) rectangle (12.9055,9.92331);
\draw [color=c, fill=c] (12.9055,9.81746) rectangle (12.9453,9.92331);
\draw [color=c, fill=c] (12.9453,9.81746) rectangle (12.9851,9.92331);
\draw [color=c, fill=c] (12.9851,9.81746) rectangle (13.0249,9.92331);
\draw [color=c, fill=c] (13.0249,9.81746) rectangle (13.0647,9.92331);
\draw [color=c, fill=c] (13.0647,9.81746) rectangle (13.1045,9.92331);
\draw [color=c, fill=c] (13.1045,9.81746) rectangle (13.1443,9.92331);
\draw [color=c, fill=c] (13.1443,9.81746) rectangle (13.1841,9.92331);
\draw [color=c, fill=c] (13.1841,9.81746) rectangle (13.2239,9.92331);
\draw [color=c, fill=c] (13.2239,9.81746) rectangle (13.2637,9.92331);
\draw [color=c, fill=c] (13.2637,9.81746) rectangle (13.3035,9.92331);
\draw [color=c, fill=c] (13.3035,9.81746) rectangle (13.3433,9.92331);
\draw [color=c, fill=c] (13.3433,9.81746) rectangle (13.3831,9.92331);
\draw [color=c, fill=c] (13.3831,9.81746) rectangle (13.4229,9.92331);
\draw [color=c, fill=c] (13.4229,9.81746) rectangle (13.4627,9.92331);
\draw [color=c, fill=c] (13.4627,9.81746) rectangle (13.5025,9.92331);
\draw [color=c, fill=c] (13.5025,9.81746) rectangle (13.5423,9.92331);
\draw [color=c, fill=c] (13.5423,9.81746) rectangle (13.5821,9.92331);
\draw [color=c, fill=c] (13.5821,9.81746) rectangle (13.6219,9.92331);
\draw [color=c, fill=c] (13.6219,9.81746) rectangle (13.6617,9.92331);
\draw [color=c, fill=c] (13.6617,9.81746) rectangle (13.7015,9.92331);
\draw [color=c, fill=c] (13.7015,9.81746) rectangle (13.7413,9.92331);
\definecolor{c}{rgb}{0,0.733333,1};
\draw [color=c, fill=c] (13.7413,9.81746) rectangle (13.7811,9.92331);
\draw [color=c, fill=c] (13.7811,9.81746) rectangle (13.8209,9.92331);
\draw [color=c, fill=c] (13.8209,9.81746) rectangle (13.8607,9.92331);
\draw [color=c, fill=c] (13.8607,9.81746) rectangle (13.9005,9.92331);
\draw [color=c, fill=c] (13.9005,9.81746) rectangle (13.9403,9.92331);
\draw [color=c, fill=c] (13.9403,9.81746) rectangle (13.9801,9.92331);
\draw [color=c, fill=c] (13.9801,9.81746) rectangle (14.0199,9.92331);
\draw [color=c, fill=c] (14.0199,9.81746) rectangle (14.0597,9.92331);
\draw [color=c, fill=c] (14.0597,9.81746) rectangle (14.0995,9.92331);
\draw [color=c, fill=c] (14.0995,9.81746) rectangle (14.1393,9.92331);
\draw [color=c, fill=c] (14.1393,9.81746) rectangle (14.1791,9.92331);
\draw [color=c, fill=c] (14.1791,9.81746) rectangle (14.2189,9.92331);
\draw [color=c, fill=c] (14.2189,9.81746) rectangle (14.2587,9.92331);
\draw [color=c, fill=c] (14.2587,9.81746) rectangle (14.2985,9.92331);
\draw [color=c, fill=c] (14.2985,9.81746) rectangle (14.3383,9.92331);
\draw [color=c, fill=c] (14.3383,9.81746) rectangle (14.3781,9.92331);
\draw [color=c, fill=c] (14.3781,9.81746) rectangle (14.4179,9.92331);
\draw [color=c, fill=c] (14.4179,9.81746) rectangle (14.4577,9.92331);
\draw [color=c, fill=c] (14.4577,9.81746) rectangle (14.4975,9.92331);
\draw [color=c, fill=c] (14.4975,9.81746) rectangle (14.5373,9.92331);
\draw [color=c, fill=c] (14.5373,9.81746) rectangle (14.5771,9.92331);
\draw [color=c, fill=c] (14.5771,9.81746) rectangle (14.6169,9.92331);
\draw [color=c, fill=c] (14.6169,9.81746) rectangle (14.6567,9.92331);
\draw [color=c, fill=c] (14.6567,9.81746) rectangle (14.6965,9.92331);
\draw [color=c, fill=c] (14.6965,9.81746) rectangle (14.7363,9.92331);
\draw [color=c, fill=c] (14.7363,9.81746) rectangle (14.7761,9.92331);
\draw [color=c, fill=c] (14.7761,9.81746) rectangle (14.8159,9.92331);
\draw [color=c, fill=c] (14.8159,9.81746) rectangle (14.8557,9.92331);
\draw [color=c, fill=c] (14.8557,9.81746) rectangle (14.8955,9.92331);
\draw [color=c, fill=c] (14.8955,9.81746) rectangle (14.9353,9.92331);
\draw [color=c, fill=c] (14.9353,9.81746) rectangle (14.9751,9.92331);
\draw [color=c, fill=c] (14.9751,9.81746) rectangle (15.0149,9.92331);
\draw [color=c, fill=c] (15.0149,9.81746) rectangle (15.0547,9.92331);
\draw [color=c, fill=c] (15.0547,9.81746) rectangle (15.0945,9.92331);
\draw [color=c, fill=c] (15.0945,9.81746) rectangle (15.1343,9.92331);
\draw [color=c, fill=c] (15.1343,9.81746) rectangle (15.1741,9.92331);
\draw [color=c, fill=c] (15.1741,9.81746) rectangle (15.2139,9.92331);
\draw [color=c, fill=c] (15.2139,9.81746) rectangle (15.2537,9.92331);
\draw [color=c, fill=c] (15.2537,9.81746) rectangle (15.2935,9.92331);
\draw [color=c, fill=c] (15.2935,9.81746) rectangle (15.3333,9.92331);
\draw [color=c, fill=c] (15.3333,9.81746) rectangle (15.3731,9.92331);
\draw [color=c, fill=c] (15.3731,9.81746) rectangle (15.4129,9.92331);
\draw [color=c, fill=c] (15.4129,9.81746) rectangle (15.4527,9.92331);
\draw [color=c, fill=c] (15.4527,9.81746) rectangle (15.4925,9.92331);
\draw [color=c, fill=c] (15.4925,9.81746) rectangle (15.5323,9.92331);
\draw [color=c, fill=c] (15.5323,9.81746) rectangle (15.5721,9.92331);
\draw [color=c, fill=c] (15.5721,9.81746) rectangle (15.6119,9.92331);
\draw [color=c, fill=c] (15.6119,9.81746) rectangle (15.6517,9.92331);
\draw [color=c, fill=c] (15.6517,9.81746) rectangle (15.6915,9.92331);
\draw [color=c, fill=c] (15.6915,9.81746) rectangle (15.7313,9.92331);
\draw [color=c, fill=c] (15.7313,9.81746) rectangle (15.7711,9.92331);
\draw [color=c, fill=c] (15.7711,9.81746) rectangle (15.8109,9.92331);
\draw [color=c, fill=c] (15.8109,9.81746) rectangle (15.8507,9.92331);
\draw [color=c, fill=c] (15.8507,9.81746) rectangle (15.8905,9.92331);
\draw [color=c, fill=c] (15.8905,9.81746) rectangle (15.9303,9.92331);
\draw [color=c, fill=c] (15.9303,9.81746) rectangle (15.9701,9.92331);
\draw [color=c, fill=c] (15.9701,9.81746) rectangle (16.01,9.92331);
\draw [color=c, fill=c] (16.01,9.81746) rectangle (16.0498,9.92331);
\draw [color=c, fill=c] (16.0498,9.81746) rectangle (16.0896,9.92331);
\draw [color=c, fill=c] (16.0896,9.81746) rectangle (16.1294,9.92331);
\draw [color=c, fill=c] (16.1294,9.81746) rectangle (16.1692,9.92331);
\draw [color=c, fill=c] (16.1692,9.81746) rectangle (16.209,9.92331);
\draw [color=c, fill=c] (16.209,9.81746) rectangle (16.2488,9.92331);
\draw [color=c, fill=c] (16.2488,9.81746) rectangle (16.2886,9.92331);
\draw [color=c, fill=c] (16.2886,9.81746) rectangle (16.3284,9.92331);
\draw [color=c, fill=c] (16.3284,9.81746) rectangle (16.3682,9.92331);
\draw [color=c, fill=c] (16.3682,9.81746) rectangle (16.408,9.92331);
\draw [color=c, fill=c] (16.408,9.81746) rectangle (16.4478,9.92331);
\draw [color=c, fill=c] (16.4478,9.81746) rectangle (16.4876,9.92331);
\draw [color=c, fill=c] (16.4876,9.81746) rectangle (16.5274,9.92331);
\draw [color=c, fill=c] (16.5274,9.81746) rectangle (16.5672,9.92331);
\draw [color=c, fill=c] (16.5672,9.81746) rectangle (16.607,9.92331);
\draw [color=c, fill=c] (16.607,9.81746) rectangle (16.6468,9.92331);
\draw [color=c, fill=c] (16.6468,9.81746) rectangle (16.6866,9.92331);
\draw [color=c, fill=c] (16.6866,9.81746) rectangle (16.7264,9.92331);
\draw [color=c, fill=c] (16.7264,9.81746) rectangle (16.7662,9.92331);
\draw [color=c, fill=c] (16.7662,9.81746) rectangle (16.806,9.92331);
\draw [color=c, fill=c] (16.806,9.81746) rectangle (16.8458,9.92331);
\draw [color=c, fill=c] (16.8458,9.81746) rectangle (16.8856,9.92331);
\draw [color=c, fill=c] (16.8856,9.81746) rectangle (16.9254,9.92331);
\draw [color=c, fill=c] (16.9254,9.81746) rectangle (16.9652,9.92331);
\draw [color=c, fill=c] (16.9652,9.81746) rectangle (17.005,9.92331);
\draw [color=c, fill=c] (17.005,9.81746) rectangle (17.0448,9.92331);
\draw [color=c, fill=c] (17.0448,9.81746) rectangle (17.0846,9.92331);
\draw [color=c, fill=c] (17.0846,9.81746) rectangle (17.1244,9.92331);
\draw [color=c, fill=c] (17.1244,9.81746) rectangle (17.1642,9.92331);
\draw [color=c, fill=c] (17.1642,9.81746) rectangle (17.204,9.92331);
\draw [color=c, fill=c] (17.204,9.81746) rectangle (17.2438,9.92331);
\draw [color=c, fill=c] (17.2438,9.81746) rectangle (17.2836,9.92331);
\draw [color=c, fill=c] (17.2836,9.81746) rectangle (17.3234,9.92331);
\draw [color=c, fill=c] (17.3234,9.81746) rectangle (17.3632,9.92331);
\draw [color=c, fill=c] (17.3632,9.81746) rectangle (17.403,9.92331);
\draw [color=c, fill=c] (17.403,9.81746) rectangle (17.4428,9.92331);
\draw [color=c, fill=c] (17.4428,9.81746) rectangle (17.4826,9.92331);
\draw [color=c, fill=c] (17.4826,9.81746) rectangle (17.5224,9.92331);
\draw [color=c, fill=c] (17.5224,9.81746) rectangle (17.5622,9.92331);
\draw [color=c, fill=c] (17.5622,9.81746) rectangle (17.602,9.92331);
\draw [color=c, fill=c] (17.602,9.81746) rectangle (17.6418,9.92331);
\draw [color=c, fill=c] (17.6418,9.81746) rectangle (17.6816,9.92331);
\draw [color=c, fill=c] (17.6816,9.81746) rectangle (17.7214,9.92331);
\draw [color=c, fill=c] (17.7214,9.81746) rectangle (17.7612,9.92331);
\draw [color=c, fill=c] (17.7612,9.81746) rectangle (17.801,9.92331);
\draw [color=c, fill=c] (17.801,9.81746) rectangle (17.8408,9.92331);
\draw [color=c, fill=c] (17.8408,9.81746) rectangle (17.8806,9.92331);
\draw [color=c, fill=c] (17.8806,9.81746) rectangle (17.9204,9.92331);
\draw [color=c, fill=c] (17.9204,9.81746) rectangle (17.9602,9.92331);
\draw [color=c, fill=c] (17.9602,9.81746) rectangle (18,9.92331);
\definecolor{c}{rgb}{0.2,0,1};
\draw [color=c, fill=c] (2,9.92331) rectangle (2.0398,10.0292);
\draw [color=c, fill=c] (2.0398,9.92331) rectangle (2.0796,10.0292);
\draw [color=c, fill=c] (2.0796,9.92331) rectangle (2.1194,10.0292);
\draw [color=c, fill=c] (2.1194,9.92331) rectangle (2.1592,10.0292);
\draw [color=c, fill=c] (2.1592,9.92331) rectangle (2.19901,10.0292);
\draw [color=c, fill=c] (2.19901,9.92331) rectangle (2.23881,10.0292);
\draw [color=c, fill=c] (2.23881,9.92331) rectangle (2.27861,10.0292);
\draw [color=c, fill=c] (2.27861,9.92331) rectangle (2.31841,10.0292);
\draw [color=c, fill=c] (2.31841,9.92331) rectangle (2.35821,10.0292);
\draw [color=c, fill=c] (2.35821,9.92331) rectangle (2.39801,10.0292);
\draw [color=c, fill=c] (2.39801,9.92331) rectangle (2.43781,10.0292);
\draw [color=c, fill=c] (2.43781,9.92331) rectangle (2.47761,10.0292);
\draw [color=c, fill=c] (2.47761,9.92331) rectangle (2.51741,10.0292);
\draw [color=c, fill=c] (2.51741,9.92331) rectangle (2.55721,10.0292);
\draw [color=c, fill=c] (2.55721,9.92331) rectangle (2.59702,10.0292);
\draw [color=c, fill=c] (2.59702,9.92331) rectangle (2.63682,10.0292);
\draw [color=c, fill=c] (2.63682,9.92331) rectangle (2.67662,10.0292);
\draw [color=c, fill=c] (2.67662,9.92331) rectangle (2.71642,10.0292);
\draw [color=c, fill=c] (2.71642,9.92331) rectangle (2.75622,10.0292);
\draw [color=c, fill=c] (2.75622,9.92331) rectangle (2.79602,10.0292);
\draw [color=c, fill=c] (2.79602,9.92331) rectangle (2.83582,10.0292);
\draw [color=c, fill=c] (2.83582,9.92331) rectangle (2.87562,10.0292);
\draw [color=c, fill=c] (2.87562,9.92331) rectangle (2.91542,10.0292);
\draw [color=c, fill=c] (2.91542,9.92331) rectangle (2.95522,10.0292);
\draw [color=c, fill=c] (2.95522,9.92331) rectangle (2.99502,10.0292);
\draw [color=c, fill=c] (2.99502,9.92331) rectangle (3.03483,10.0292);
\draw [color=c, fill=c] (3.03483,9.92331) rectangle (3.07463,10.0292);
\draw [color=c, fill=c] (3.07463,9.92331) rectangle (3.11443,10.0292);
\draw [color=c, fill=c] (3.11443,9.92331) rectangle (3.15423,10.0292);
\draw [color=c, fill=c] (3.15423,9.92331) rectangle (3.19403,10.0292);
\draw [color=c, fill=c] (3.19403,9.92331) rectangle (3.23383,10.0292);
\draw [color=c, fill=c] (3.23383,9.92331) rectangle (3.27363,10.0292);
\draw [color=c, fill=c] (3.27363,9.92331) rectangle (3.31343,10.0292);
\draw [color=c, fill=c] (3.31343,9.92331) rectangle (3.35323,10.0292);
\draw [color=c, fill=c] (3.35323,9.92331) rectangle (3.39303,10.0292);
\draw [color=c, fill=c] (3.39303,9.92331) rectangle (3.43284,10.0292);
\draw [color=c, fill=c] (3.43284,9.92331) rectangle (3.47264,10.0292);
\draw [color=c, fill=c] (3.47264,9.92331) rectangle (3.51244,10.0292);
\draw [color=c, fill=c] (3.51244,9.92331) rectangle (3.55224,10.0292);
\draw [color=c, fill=c] (3.55224,9.92331) rectangle (3.59204,10.0292);
\draw [color=c, fill=c] (3.59204,9.92331) rectangle (3.63184,10.0292);
\draw [color=c, fill=c] (3.63184,9.92331) rectangle (3.67164,10.0292);
\draw [color=c, fill=c] (3.67164,9.92331) rectangle (3.71144,10.0292);
\draw [color=c, fill=c] (3.71144,9.92331) rectangle (3.75124,10.0292);
\draw [color=c, fill=c] (3.75124,9.92331) rectangle (3.79104,10.0292);
\draw [color=c, fill=c] (3.79104,9.92331) rectangle (3.83085,10.0292);
\draw [color=c, fill=c] (3.83085,9.92331) rectangle (3.87065,10.0292);
\draw [color=c, fill=c] (3.87065,9.92331) rectangle (3.91045,10.0292);
\draw [color=c, fill=c] (3.91045,9.92331) rectangle (3.95025,10.0292);
\draw [color=c, fill=c] (3.95025,9.92331) rectangle (3.99005,10.0292);
\draw [color=c, fill=c] (3.99005,9.92331) rectangle (4.02985,10.0292);
\draw [color=c, fill=c] (4.02985,9.92331) rectangle (4.06965,10.0292);
\draw [color=c, fill=c] (4.06965,9.92331) rectangle (4.10945,10.0292);
\draw [color=c, fill=c] (4.10945,9.92331) rectangle (4.14925,10.0292);
\draw [color=c, fill=c] (4.14925,9.92331) rectangle (4.18905,10.0292);
\draw [color=c, fill=c] (4.18905,9.92331) rectangle (4.22886,10.0292);
\draw [color=c, fill=c] (4.22886,9.92331) rectangle (4.26866,10.0292);
\draw [color=c, fill=c] (4.26866,9.92331) rectangle (4.30846,10.0292);
\draw [color=c, fill=c] (4.30846,9.92331) rectangle (4.34826,10.0292);
\draw [color=c, fill=c] (4.34826,9.92331) rectangle (4.38806,10.0292);
\draw [color=c, fill=c] (4.38806,9.92331) rectangle (4.42786,10.0292);
\draw [color=c, fill=c] (4.42786,9.92331) rectangle (4.46766,10.0292);
\draw [color=c, fill=c] (4.46766,9.92331) rectangle (4.50746,10.0292);
\draw [color=c, fill=c] (4.50746,9.92331) rectangle (4.54726,10.0292);
\draw [color=c, fill=c] (4.54726,9.92331) rectangle (4.58706,10.0292);
\draw [color=c, fill=c] (4.58706,9.92331) rectangle (4.62687,10.0292);
\draw [color=c, fill=c] (4.62687,9.92331) rectangle (4.66667,10.0292);
\draw [color=c, fill=c] (4.66667,9.92331) rectangle (4.70647,10.0292);
\draw [color=c, fill=c] (4.70647,9.92331) rectangle (4.74627,10.0292);
\draw [color=c, fill=c] (4.74627,9.92331) rectangle (4.78607,10.0292);
\draw [color=c, fill=c] (4.78607,9.92331) rectangle (4.82587,10.0292);
\draw [color=c, fill=c] (4.82587,9.92331) rectangle (4.86567,10.0292);
\draw [color=c, fill=c] (4.86567,9.92331) rectangle (4.90547,10.0292);
\draw [color=c, fill=c] (4.90547,9.92331) rectangle (4.94527,10.0292);
\draw [color=c, fill=c] (4.94527,9.92331) rectangle (4.98507,10.0292);
\draw [color=c, fill=c] (4.98507,9.92331) rectangle (5.02488,10.0292);
\draw [color=c, fill=c] (5.02488,9.92331) rectangle (5.06468,10.0292);
\draw [color=c, fill=c] (5.06468,9.92331) rectangle (5.10448,10.0292);
\draw [color=c, fill=c] (5.10448,9.92331) rectangle (5.14428,10.0292);
\draw [color=c, fill=c] (5.14428,9.92331) rectangle (5.18408,10.0292);
\draw [color=c, fill=c] (5.18408,9.92331) rectangle (5.22388,10.0292);
\draw [color=c, fill=c] (5.22388,9.92331) rectangle (5.26368,10.0292);
\draw [color=c, fill=c] (5.26368,9.92331) rectangle (5.30348,10.0292);
\draw [color=c, fill=c] (5.30348,9.92331) rectangle (5.34328,10.0292);
\draw [color=c, fill=c] (5.34328,9.92331) rectangle (5.38308,10.0292);
\draw [color=c, fill=c] (5.38308,9.92331) rectangle (5.42289,10.0292);
\draw [color=c, fill=c] (5.42289,9.92331) rectangle (5.46269,10.0292);
\draw [color=c, fill=c] (5.46269,9.92331) rectangle (5.50249,10.0292);
\draw [color=c, fill=c] (5.50249,9.92331) rectangle (5.54229,10.0292);
\draw [color=c, fill=c] (5.54229,9.92331) rectangle (5.58209,10.0292);
\draw [color=c, fill=c] (5.58209,9.92331) rectangle (5.62189,10.0292);
\draw [color=c, fill=c] (5.62189,9.92331) rectangle (5.66169,10.0292);
\draw [color=c, fill=c] (5.66169,9.92331) rectangle (5.70149,10.0292);
\draw [color=c, fill=c] (5.70149,9.92331) rectangle (5.74129,10.0292);
\draw [color=c, fill=c] (5.74129,9.92331) rectangle (5.78109,10.0292);
\draw [color=c, fill=c] (5.78109,9.92331) rectangle (5.8209,10.0292);
\draw [color=c, fill=c] (5.8209,9.92331) rectangle (5.8607,10.0292);
\draw [color=c, fill=c] (5.8607,9.92331) rectangle (5.9005,10.0292);
\draw [color=c, fill=c] (5.9005,9.92331) rectangle (5.9403,10.0292);
\draw [color=c, fill=c] (5.9403,9.92331) rectangle (5.9801,10.0292);
\draw [color=c, fill=c] (5.9801,9.92331) rectangle (6.0199,10.0292);
\draw [color=c, fill=c] (6.0199,9.92331) rectangle (6.0597,10.0292);
\draw [color=c, fill=c] (6.0597,9.92331) rectangle (6.0995,10.0292);
\draw [color=c, fill=c] (6.0995,9.92331) rectangle (6.1393,10.0292);
\draw [color=c, fill=c] (6.1393,9.92331) rectangle (6.1791,10.0292);
\draw [color=c, fill=c] (6.1791,9.92331) rectangle (6.21891,10.0292);
\draw [color=c, fill=c] (6.21891,9.92331) rectangle (6.25871,10.0292);
\draw [color=c, fill=c] (6.25871,9.92331) rectangle (6.29851,10.0292);
\draw [color=c, fill=c] (6.29851,9.92331) rectangle (6.33831,10.0292);
\draw [color=c, fill=c] (6.33831,9.92331) rectangle (6.37811,10.0292);
\draw [color=c, fill=c] (6.37811,9.92331) rectangle (6.41791,10.0292);
\draw [color=c, fill=c] (6.41791,9.92331) rectangle (6.45771,10.0292);
\draw [color=c, fill=c] (6.45771,9.92331) rectangle (6.49751,10.0292);
\draw [color=c, fill=c] (6.49751,9.92331) rectangle (6.53731,10.0292);
\draw [color=c, fill=c] (6.53731,9.92331) rectangle (6.57711,10.0292);
\draw [color=c, fill=c] (6.57711,9.92331) rectangle (6.61692,10.0292);
\draw [color=c, fill=c] (6.61692,9.92331) rectangle (6.65672,10.0292);
\draw [color=c, fill=c] (6.65672,9.92331) rectangle (6.69652,10.0292);
\draw [color=c, fill=c] (6.69652,9.92331) rectangle (6.73632,10.0292);
\draw [color=c, fill=c] (6.73632,9.92331) rectangle (6.77612,10.0292);
\draw [color=c, fill=c] (6.77612,9.92331) rectangle (6.81592,10.0292);
\draw [color=c, fill=c] (6.81592,9.92331) rectangle (6.85572,10.0292);
\draw [color=c, fill=c] (6.85572,9.92331) rectangle (6.89552,10.0292);
\draw [color=c, fill=c] (6.89552,9.92331) rectangle (6.93532,10.0292);
\draw [color=c, fill=c] (6.93532,9.92331) rectangle (6.97512,10.0292);
\draw [color=c, fill=c] (6.97512,9.92331) rectangle (7.01493,10.0292);
\draw [color=c, fill=c] (7.01493,9.92331) rectangle (7.05473,10.0292);
\draw [color=c, fill=c] (7.05473,9.92331) rectangle (7.09453,10.0292);
\draw [color=c, fill=c] (7.09453,9.92331) rectangle (7.13433,10.0292);
\draw [color=c, fill=c] (7.13433,9.92331) rectangle (7.17413,10.0292);
\draw [color=c, fill=c] (7.17413,9.92331) rectangle (7.21393,10.0292);
\draw [color=c, fill=c] (7.21393,9.92331) rectangle (7.25373,10.0292);
\draw [color=c, fill=c] (7.25373,9.92331) rectangle (7.29353,10.0292);
\draw [color=c, fill=c] (7.29353,9.92331) rectangle (7.33333,10.0292);
\draw [color=c, fill=c] (7.33333,9.92331) rectangle (7.37313,10.0292);
\draw [color=c, fill=c] (7.37313,9.92331) rectangle (7.41294,10.0292);
\draw [color=c, fill=c] (7.41294,9.92331) rectangle (7.45274,10.0292);
\draw [color=c, fill=c] (7.45274,9.92331) rectangle (7.49254,10.0292);
\draw [color=c, fill=c] (7.49254,9.92331) rectangle (7.53234,10.0292);
\draw [color=c, fill=c] (7.53234,9.92331) rectangle (7.57214,10.0292);
\draw [color=c, fill=c] (7.57214,9.92331) rectangle (7.61194,10.0292);
\draw [color=c, fill=c] (7.61194,9.92331) rectangle (7.65174,10.0292);
\draw [color=c, fill=c] (7.65174,9.92331) rectangle (7.69154,10.0292);
\definecolor{c}{rgb}{0,0.0800001,1};
\draw [color=c, fill=c] (7.69154,9.92331) rectangle (7.73134,10.0292);
\draw [color=c, fill=c] (7.73134,9.92331) rectangle (7.77114,10.0292);
\draw [color=c, fill=c] (7.77114,9.92331) rectangle (7.81095,10.0292);
\draw [color=c, fill=c] (7.81095,9.92331) rectangle (7.85075,10.0292);
\draw [color=c, fill=c] (7.85075,9.92331) rectangle (7.89055,10.0292);
\draw [color=c, fill=c] (7.89055,9.92331) rectangle (7.93035,10.0292);
\draw [color=c, fill=c] (7.93035,9.92331) rectangle (7.97015,10.0292);
\draw [color=c, fill=c] (7.97015,9.92331) rectangle (8.00995,10.0292);
\draw [color=c, fill=c] (8.00995,9.92331) rectangle (8.04975,10.0292);
\draw [color=c, fill=c] (8.04975,9.92331) rectangle (8.08955,10.0292);
\draw [color=c, fill=c] (8.08955,9.92331) rectangle (8.12935,10.0292);
\draw [color=c, fill=c] (8.12935,9.92331) rectangle (8.16915,10.0292);
\draw [color=c, fill=c] (8.16915,9.92331) rectangle (8.20895,10.0292);
\draw [color=c, fill=c] (8.20895,9.92331) rectangle (8.24876,10.0292);
\draw [color=c, fill=c] (8.24876,9.92331) rectangle (8.28856,10.0292);
\draw [color=c, fill=c] (8.28856,9.92331) rectangle (8.32836,10.0292);
\draw [color=c, fill=c] (8.32836,9.92331) rectangle (8.36816,10.0292);
\draw [color=c, fill=c] (8.36816,9.92331) rectangle (8.40796,10.0292);
\draw [color=c, fill=c] (8.40796,9.92331) rectangle (8.44776,10.0292);
\draw [color=c, fill=c] (8.44776,9.92331) rectangle (8.48756,10.0292);
\draw [color=c, fill=c] (8.48756,9.92331) rectangle (8.52736,10.0292);
\draw [color=c, fill=c] (8.52736,9.92331) rectangle (8.56716,10.0292);
\draw [color=c, fill=c] (8.56716,9.92331) rectangle (8.60697,10.0292);
\draw [color=c, fill=c] (8.60697,9.92331) rectangle (8.64677,10.0292);
\draw [color=c, fill=c] (8.64677,9.92331) rectangle (8.68657,10.0292);
\draw [color=c, fill=c] (8.68657,9.92331) rectangle (8.72637,10.0292);
\draw [color=c, fill=c] (8.72637,9.92331) rectangle (8.76617,10.0292);
\draw [color=c, fill=c] (8.76617,9.92331) rectangle (8.80597,10.0292);
\draw [color=c, fill=c] (8.80597,9.92331) rectangle (8.84577,10.0292);
\draw [color=c, fill=c] (8.84577,9.92331) rectangle (8.88557,10.0292);
\draw [color=c, fill=c] (8.88557,9.92331) rectangle (8.92537,10.0292);
\draw [color=c, fill=c] (8.92537,9.92331) rectangle (8.96517,10.0292);
\draw [color=c, fill=c] (8.96517,9.92331) rectangle (9.00498,10.0292);
\draw [color=c, fill=c] (9.00498,9.92331) rectangle (9.04478,10.0292);
\draw [color=c, fill=c] (9.04478,9.92331) rectangle (9.08458,10.0292);
\draw [color=c, fill=c] (9.08458,9.92331) rectangle (9.12438,10.0292);
\draw [color=c, fill=c] (9.12438,9.92331) rectangle (9.16418,10.0292);
\draw [color=c, fill=c] (9.16418,9.92331) rectangle (9.20398,10.0292);
\draw [color=c, fill=c] (9.20398,9.92331) rectangle (9.24378,10.0292);
\draw [color=c, fill=c] (9.24378,9.92331) rectangle (9.28358,10.0292);
\draw [color=c, fill=c] (9.28358,9.92331) rectangle (9.32338,10.0292);
\draw [color=c, fill=c] (9.32338,9.92331) rectangle (9.36318,10.0292);
\draw [color=c, fill=c] (9.36318,9.92331) rectangle (9.40298,10.0292);
\draw [color=c, fill=c] (9.40298,9.92331) rectangle (9.44279,10.0292);
\draw [color=c, fill=c] (9.44279,9.92331) rectangle (9.48259,10.0292);
\draw [color=c, fill=c] (9.48259,9.92331) rectangle (9.52239,10.0292);
\definecolor{c}{rgb}{0,0.266667,1};
\draw [color=c, fill=c] (9.52239,9.92331) rectangle (9.56219,10.0292);
\draw [color=c, fill=c] (9.56219,9.92331) rectangle (9.60199,10.0292);
\draw [color=c, fill=c] (9.60199,9.92331) rectangle (9.64179,10.0292);
\draw [color=c, fill=c] (9.64179,9.92331) rectangle (9.68159,10.0292);
\draw [color=c, fill=c] (9.68159,9.92331) rectangle (9.72139,10.0292);
\draw [color=c, fill=c] (9.72139,9.92331) rectangle (9.76119,10.0292);
\draw [color=c, fill=c] (9.76119,9.92331) rectangle (9.80099,10.0292);
\draw [color=c, fill=c] (9.80099,9.92331) rectangle (9.8408,10.0292);
\draw [color=c, fill=c] (9.8408,9.92331) rectangle (9.8806,10.0292);
\draw [color=c, fill=c] (9.8806,9.92331) rectangle (9.9204,10.0292);
\draw [color=c, fill=c] (9.9204,9.92331) rectangle (9.9602,10.0292);
\draw [color=c, fill=c] (9.9602,9.92331) rectangle (10,10.0292);
\draw [color=c, fill=c] (10,9.92331) rectangle (10.0398,10.0292);
\draw [color=c, fill=c] (10.0398,9.92331) rectangle (10.0796,10.0292);
\draw [color=c, fill=c] (10.0796,9.92331) rectangle (10.1194,10.0292);
\draw [color=c, fill=c] (10.1194,9.92331) rectangle (10.1592,10.0292);
\draw [color=c, fill=c] (10.1592,9.92331) rectangle (10.199,10.0292);
\draw [color=c, fill=c] (10.199,9.92331) rectangle (10.2388,10.0292);
\draw [color=c, fill=c] (10.2388,9.92331) rectangle (10.2786,10.0292);
\draw [color=c, fill=c] (10.2786,9.92331) rectangle (10.3184,10.0292);
\draw [color=c, fill=c] (10.3184,9.92331) rectangle (10.3582,10.0292);
\draw [color=c, fill=c] (10.3582,9.92331) rectangle (10.398,10.0292);
\draw [color=c, fill=c] (10.398,9.92331) rectangle (10.4378,10.0292);
\draw [color=c, fill=c] (10.4378,9.92331) rectangle (10.4776,10.0292);
\draw [color=c, fill=c] (10.4776,9.92331) rectangle (10.5174,10.0292);
\draw [color=c, fill=c] (10.5174,9.92331) rectangle (10.5572,10.0292);
\draw [color=c, fill=c] (10.5572,9.92331) rectangle (10.597,10.0292);
\draw [color=c, fill=c] (10.597,9.92331) rectangle (10.6368,10.0292);
\draw [color=c, fill=c] (10.6368,9.92331) rectangle (10.6766,10.0292);
\draw [color=c, fill=c] (10.6766,9.92331) rectangle (10.7164,10.0292);
\draw [color=c, fill=c] (10.7164,9.92331) rectangle (10.7562,10.0292);
\definecolor{c}{rgb}{0,0.546666,1};
\draw [color=c, fill=c] (10.7562,9.92331) rectangle (10.796,10.0292);
\draw [color=c, fill=c] (10.796,9.92331) rectangle (10.8358,10.0292);
\draw [color=c, fill=c] (10.8358,9.92331) rectangle (10.8756,10.0292);
\draw [color=c, fill=c] (10.8756,9.92331) rectangle (10.9154,10.0292);
\draw [color=c, fill=c] (10.9154,9.92331) rectangle (10.9552,10.0292);
\draw [color=c, fill=c] (10.9552,9.92331) rectangle (10.995,10.0292);
\draw [color=c, fill=c] (10.995,9.92331) rectangle (11.0348,10.0292);
\draw [color=c, fill=c] (11.0348,9.92331) rectangle (11.0746,10.0292);
\draw [color=c, fill=c] (11.0746,9.92331) rectangle (11.1144,10.0292);
\draw [color=c, fill=c] (11.1144,9.92331) rectangle (11.1542,10.0292);
\draw [color=c, fill=c] (11.1542,9.92331) rectangle (11.194,10.0292);
\draw [color=c, fill=c] (11.194,9.92331) rectangle (11.2338,10.0292);
\draw [color=c, fill=c] (11.2338,9.92331) rectangle (11.2736,10.0292);
\draw [color=c, fill=c] (11.2736,9.92331) rectangle (11.3134,10.0292);
\draw [color=c, fill=c] (11.3134,9.92331) rectangle (11.3532,10.0292);
\draw [color=c, fill=c] (11.3532,9.92331) rectangle (11.393,10.0292);
\draw [color=c, fill=c] (11.393,9.92331) rectangle (11.4328,10.0292);
\draw [color=c, fill=c] (11.4328,9.92331) rectangle (11.4726,10.0292);
\draw [color=c, fill=c] (11.4726,9.92331) rectangle (11.5124,10.0292);
\draw [color=c, fill=c] (11.5124,9.92331) rectangle (11.5522,10.0292);
\draw [color=c, fill=c] (11.5522,9.92331) rectangle (11.592,10.0292);
\draw [color=c, fill=c] (11.592,9.92331) rectangle (11.6318,10.0292);
\draw [color=c, fill=c] (11.6318,9.92331) rectangle (11.6716,10.0292);
\draw [color=c, fill=c] (11.6716,9.92331) rectangle (11.7114,10.0292);
\draw [color=c, fill=c] (11.7114,9.92331) rectangle (11.7512,10.0292);
\draw [color=c, fill=c] (11.7512,9.92331) rectangle (11.791,10.0292);
\draw [color=c, fill=c] (11.791,9.92331) rectangle (11.8308,10.0292);
\draw [color=c, fill=c] (11.8308,9.92331) rectangle (11.8706,10.0292);
\draw [color=c, fill=c] (11.8706,9.92331) rectangle (11.9104,10.0292);
\draw [color=c, fill=c] (11.9104,9.92331) rectangle (11.9502,10.0292);
\draw [color=c, fill=c] (11.9502,9.92331) rectangle (11.99,10.0292);
\draw [color=c, fill=c] (11.99,9.92331) rectangle (12.0299,10.0292);
\draw [color=c, fill=c] (12.0299,9.92331) rectangle (12.0697,10.0292);
\draw [color=c, fill=c] (12.0697,9.92331) rectangle (12.1095,10.0292);
\draw [color=c, fill=c] (12.1095,9.92331) rectangle (12.1493,10.0292);
\draw [color=c, fill=c] (12.1493,9.92331) rectangle (12.1891,10.0292);
\draw [color=c, fill=c] (12.1891,9.92331) rectangle (12.2289,10.0292);
\draw [color=c, fill=c] (12.2289,9.92331) rectangle (12.2687,10.0292);
\draw [color=c, fill=c] (12.2687,9.92331) rectangle (12.3085,10.0292);
\draw [color=c, fill=c] (12.3085,9.92331) rectangle (12.3483,10.0292);
\draw [color=c, fill=c] (12.3483,9.92331) rectangle (12.3881,10.0292);
\draw [color=c, fill=c] (12.3881,9.92331) rectangle (12.4279,10.0292);
\draw [color=c, fill=c] (12.4279,9.92331) rectangle (12.4677,10.0292);
\draw [color=c, fill=c] (12.4677,9.92331) rectangle (12.5075,10.0292);
\draw [color=c, fill=c] (12.5075,9.92331) rectangle (12.5473,10.0292);
\draw [color=c, fill=c] (12.5473,9.92331) rectangle (12.5871,10.0292);
\draw [color=c, fill=c] (12.5871,9.92331) rectangle (12.6269,10.0292);
\draw [color=c, fill=c] (12.6269,9.92331) rectangle (12.6667,10.0292);
\draw [color=c, fill=c] (12.6667,9.92331) rectangle (12.7065,10.0292);
\draw [color=c, fill=c] (12.7065,9.92331) rectangle (12.7463,10.0292);
\draw [color=c, fill=c] (12.7463,9.92331) rectangle (12.7861,10.0292);
\draw [color=c, fill=c] (12.7861,9.92331) rectangle (12.8259,10.0292);
\draw [color=c, fill=c] (12.8259,9.92331) rectangle (12.8657,10.0292);
\draw [color=c, fill=c] (12.8657,9.92331) rectangle (12.9055,10.0292);
\draw [color=c, fill=c] (12.9055,9.92331) rectangle (12.9453,10.0292);
\draw [color=c, fill=c] (12.9453,9.92331) rectangle (12.9851,10.0292);
\draw [color=c, fill=c] (12.9851,9.92331) rectangle (13.0249,10.0292);
\draw [color=c, fill=c] (13.0249,9.92331) rectangle (13.0647,10.0292);
\draw [color=c, fill=c] (13.0647,9.92331) rectangle (13.1045,10.0292);
\draw [color=c, fill=c] (13.1045,9.92331) rectangle (13.1443,10.0292);
\draw [color=c, fill=c] (13.1443,9.92331) rectangle (13.1841,10.0292);
\draw [color=c, fill=c] (13.1841,9.92331) rectangle (13.2239,10.0292);
\draw [color=c, fill=c] (13.2239,9.92331) rectangle (13.2637,10.0292);
\draw [color=c, fill=c] (13.2637,9.92331) rectangle (13.3035,10.0292);
\draw [color=c, fill=c] (13.3035,9.92331) rectangle (13.3433,10.0292);
\draw [color=c, fill=c] (13.3433,9.92331) rectangle (13.3831,10.0292);
\draw [color=c, fill=c] (13.3831,9.92331) rectangle (13.4229,10.0292);
\draw [color=c, fill=c] (13.4229,9.92331) rectangle (13.4627,10.0292);
\draw [color=c, fill=c] (13.4627,9.92331) rectangle (13.5025,10.0292);
\draw [color=c, fill=c] (13.5025,9.92331) rectangle (13.5423,10.0292);
\draw [color=c, fill=c] (13.5423,9.92331) rectangle (13.5821,10.0292);
\draw [color=c, fill=c] (13.5821,9.92331) rectangle (13.6219,10.0292);
\draw [color=c, fill=c] (13.6219,9.92331) rectangle (13.6617,10.0292);
\draw [color=c, fill=c] (13.6617,9.92331) rectangle (13.7015,10.0292);
\draw [color=c, fill=c] (13.7015,9.92331) rectangle (13.7413,10.0292);
\draw [color=c, fill=c] (13.7413,9.92331) rectangle (13.7811,10.0292);
\draw [color=c, fill=c] (13.7811,9.92331) rectangle (13.8209,10.0292);
\definecolor{c}{rgb}{0,0.733333,1};
\draw [color=c, fill=c] (13.8209,9.92331) rectangle (13.8607,10.0292);
\draw [color=c, fill=c] (13.8607,9.92331) rectangle (13.9005,10.0292);
\draw [color=c, fill=c] (13.9005,9.92331) rectangle (13.9403,10.0292);
\draw [color=c, fill=c] (13.9403,9.92331) rectangle (13.9801,10.0292);
\draw [color=c, fill=c] (13.9801,9.92331) rectangle (14.0199,10.0292);
\draw [color=c, fill=c] (14.0199,9.92331) rectangle (14.0597,10.0292);
\draw [color=c, fill=c] (14.0597,9.92331) rectangle (14.0995,10.0292);
\draw [color=c, fill=c] (14.0995,9.92331) rectangle (14.1393,10.0292);
\draw [color=c, fill=c] (14.1393,9.92331) rectangle (14.1791,10.0292);
\draw [color=c, fill=c] (14.1791,9.92331) rectangle (14.2189,10.0292);
\draw [color=c, fill=c] (14.2189,9.92331) rectangle (14.2587,10.0292);
\draw [color=c, fill=c] (14.2587,9.92331) rectangle (14.2985,10.0292);
\draw [color=c, fill=c] (14.2985,9.92331) rectangle (14.3383,10.0292);
\draw [color=c, fill=c] (14.3383,9.92331) rectangle (14.3781,10.0292);
\draw [color=c, fill=c] (14.3781,9.92331) rectangle (14.4179,10.0292);
\draw [color=c, fill=c] (14.4179,9.92331) rectangle (14.4577,10.0292);
\draw [color=c, fill=c] (14.4577,9.92331) rectangle (14.4975,10.0292);
\draw [color=c, fill=c] (14.4975,9.92331) rectangle (14.5373,10.0292);
\draw [color=c, fill=c] (14.5373,9.92331) rectangle (14.5771,10.0292);
\draw [color=c, fill=c] (14.5771,9.92331) rectangle (14.6169,10.0292);
\draw [color=c, fill=c] (14.6169,9.92331) rectangle (14.6567,10.0292);
\draw [color=c, fill=c] (14.6567,9.92331) rectangle (14.6965,10.0292);
\draw [color=c, fill=c] (14.6965,9.92331) rectangle (14.7363,10.0292);
\draw [color=c, fill=c] (14.7363,9.92331) rectangle (14.7761,10.0292);
\draw [color=c, fill=c] (14.7761,9.92331) rectangle (14.8159,10.0292);
\draw [color=c, fill=c] (14.8159,9.92331) rectangle (14.8557,10.0292);
\draw [color=c, fill=c] (14.8557,9.92331) rectangle (14.8955,10.0292);
\draw [color=c, fill=c] (14.8955,9.92331) rectangle (14.9353,10.0292);
\draw [color=c, fill=c] (14.9353,9.92331) rectangle (14.9751,10.0292);
\draw [color=c, fill=c] (14.9751,9.92331) rectangle (15.0149,10.0292);
\draw [color=c, fill=c] (15.0149,9.92331) rectangle (15.0547,10.0292);
\draw [color=c, fill=c] (15.0547,9.92331) rectangle (15.0945,10.0292);
\draw [color=c, fill=c] (15.0945,9.92331) rectangle (15.1343,10.0292);
\draw [color=c, fill=c] (15.1343,9.92331) rectangle (15.1741,10.0292);
\draw [color=c, fill=c] (15.1741,9.92331) rectangle (15.2139,10.0292);
\draw [color=c, fill=c] (15.2139,9.92331) rectangle (15.2537,10.0292);
\draw [color=c, fill=c] (15.2537,9.92331) rectangle (15.2935,10.0292);
\draw [color=c, fill=c] (15.2935,9.92331) rectangle (15.3333,10.0292);
\draw [color=c, fill=c] (15.3333,9.92331) rectangle (15.3731,10.0292);
\draw [color=c, fill=c] (15.3731,9.92331) rectangle (15.4129,10.0292);
\draw [color=c, fill=c] (15.4129,9.92331) rectangle (15.4527,10.0292);
\draw [color=c, fill=c] (15.4527,9.92331) rectangle (15.4925,10.0292);
\draw [color=c, fill=c] (15.4925,9.92331) rectangle (15.5323,10.0292);
\draw [color=c, fill=c] (15.5323,9.92331) rectangle (15.5721,10.0292);
\draw [color=c, fill=c] (15.5721,9.92331) rectangle (15.6119,10.0292);
\draw [color=c, fill=c] (15.6119,9.92331) rectangle (15.6517,10.0292);
\draw [color=c, fill=c] (15.6517,9.92331) rectangle (15.6915,10.0292);
\draw [color=c, fill=c] (15.6915,9.92331) rectangle (15.7313,10.0292);
\draw [color=c, fill=c] (15.7313,9.92331) rectangle (15.7711,10.0292);
\draw [color=c, fill=c] (15.7711,9.92331) rectangle (15.8109,10.0292);
\draw [color=c, fill=c] (15.8109,9.92331) rectangle (15.8507,10.0292);
\draw [color=c, fill=c] (15.8507,9.92331) rectangle (15.8905,10.0292);
\draw [color=c, fill=c] (15.8905,9.92331) rectangle (15.9303,10.0292);
\draw [color=c, fill=c] (15.9303,9.92331) rectangle (15.9701,10.0292);
\draw [color=c, fill=c] (15.9701,9.92331) rectangle (16.01,10.0292);
\draw [color=c, fill=c] (16.01,9.92331) rectangle (16.0498,10.0292);
\draw [color=c, fill=c] (16.0498,9.92331) rectangle (16.0896,10.0292);
\draw [color=c, fill=c] (16.0896,9.92331) rectangle (16.1294,10.0292);
\draw [color=c, fill=c] (16.1294,9.92331) rectangle (16.1692,10.0292);
\draw [color=c, fill=c] (16.1692,9.92331) rectangle (16.209,10.0292);
\draw [color=c, fill=c] (16.209,9.92331) rectangle (16.2488,10.0292);
\draw [color=c, fill=c] (16.2488,9.92331) rectangle (16.2886,10.0292);
\draw [color=c, fill=c] (16.2886,9.92331) rectangle (16.3284,10.0292);
\draw [color=c, fill=c] (16.3284,9.92331) rectangle (16.3682,10.0292);
\draw [color=c, fill=c] (16.3682,9.92331) rectangle (16.408,10.0292);
\draw [color=c, fill=c] (16.408,9.92331) rectangle (16.4478,10.0292);
\draw [color=c, fill=c] (16.4478,9.92331) rectangle (16.4876,10.0292);
\draw [color=c, fill=c] (16.4876,9.92331) rectangle (16.5274,10.0292);
\draw [color=c, fill=c] (16.5274,9.92331) rectangle (16.5672,10.0292);
\draw [color=c, fill=c] (16.5672,9.92331) rectangle (16.607,10.0292);
\draw [color=c, fill=c] (16.607,9.92331) rectangle (16.6468,10.0292);
\draw [color=c, fill=c] (16.6468,9.92331) rectangle (16.6866,10.0292);
\draw [color=c, fill=c] (16.6866,9.92331) rectangle (16.7264,10.0292);
\draw [color=c, fill=c] (16.7264,9.92331) rectangle (16.7662,10.0292);
\draw [color=c, fill=c] (16.7662,9.92331) rectangle (16.806,10.0292);
\draw [color=c, fill=c] (16.806,9.92331) rectangle (16.8458,10.0292);
\draw [color=c, fill=c] (16.8458,9.92331) rectangle (16.8856,10.0292);
\draw [color=c, fill=c] (16.8856,9.92331) rectangle (16.9254,10.0292);
\draw [color=c, fill=c] (16.9254,9.92331) rectangle (16.9652,10.0292);
\draw [color=c, fill=c] (16.9652,9.92331) rectangle (17.005,10.0292);
\draw [color=c, fill=c] (17.005,9.92331) rectangle (17.0448,10.0292);
\draw [color=c, fill=c] (17.0448,9.92331) rectangle (17.0846,10.0292);
\draw [color=c, fill=c] (17.0846,9.92331) rectangle (17.1244,10.0292);
\draw [color=c, fill=c] (17.1244,9.92331) rectangle (17.1642,10.0292);
\draw [color=c, fill=c] (17.1642,9.92331) rectangle (17.204,10.0292);
\draw [color=c, fill=c] (17.204,9.92331) rectangle (17.2438,10.0292);
\draw [color=c, fill=c] (17.2438,9.92331) rectangle (17.2836,10.0292);
\draw [color=c, fill=c] (17.2836,9.92331) rectangle (17.3234,10.0292);
\draw [color=c, fill=c] (17.3234,9.92331) rectangle (17.3632,10.0292);
\draw [color=c, fill=c] (17.3632,9.92331) rectangle (17.403,10.0292);
\draw [color=c, fill=c] (17.403,9.92331) rectangle (17.4428,10.0292);
\draw [color=c, fill=c] (17.4428,9.92331) rectangle (17.4826,10.0292);
\draw [color=c, fill=c] (17.4826,9.92331) rectangle (17.5224,10.0292);
\draw [color=c, fill=c] (17.5224,9.92331) rectangle (17.5622,10.0292);
\draw [color=c, fill=c] (17.5622,9.92331) rectangle (17.602,10.0292);
\draw [color=c, fill=c] (17.602,9.92331) rectangle (17.6418,10.0292);
\draw [color=c, fill=c] (17.6418,9.92331) rectangle (17.6816,10.0292);
\draw [color=c, fill=c] (17.6816,9.92331) rectangle (17.7214,10.0292);
\draw [color=c, fill=c] (17.7214,9.92331) rectangle (17.7612,10.0292);
\draw [color=c, fill=c] (17.7612,9.92331) rectangle (17.801,10.0292);
\draw [color=c, fill=c] (17.801,9.92331) rectangle (17.8408,10.0292);
\draw [color=c, fill=c] (17.8408,9.92331) rectangle (17.8806,10.0292);
\draw [color=c, fill=c] (17.8806,9.92331) rectangle (17.9204,10.0292);
\draw [color=c, fill=c] (17.9204,9.92331) rectangle (17.9602,10.0292);
\draw [color=c, fill=c] (17.9602,9.92331) rectangle (18,10.0292);
\definecolor{c}{rgb}{0.2,0,1};
\draw [color=c, fill=c] (2,10.0292) rectangle (2.0398,10.135);
\draw [color=c, fill=c] (2.0398,10.0292) rectangle (2.0796,10.135);
\draw [color=c, fill=c] (2.0796,10.0292) rectangle (2.1194,10.135);
\draw [color=c, fill=c] (2.1194,10.0292) rectangle (2.1592,10.135);
\draw [color=c, fill=c] (2.1592,10.0292) rectangle (2.19901,10.135);
\draw [color=c, fill=c] (2.19901,10.0292) rectangle (2.23881,10.135);
\draw [color=c, fill=c] (2.23881,10.0292) rectangle (2.27861,10.135);
\draw [color=c, fill=c] (2.27861,10.0292) rectangle (2.31841,10.135);
\draw [color=c, fill=c] (2.31841,10.0292) rectangle (2.35821,10.135);
\draw [color=c, fill=c] (2.35821,10.0292) rectangle (2.39801,10.135);
\draw [color=c, fill=c] (2.39801,10.0292) rectangle (2.43781,10.135);
\draw [color=c, fill=c] (2.43781,10.0292) rectangle (2.47761,10.135);
\draw [color=c, fill=c] (2.47761,10.0292) rectangle (2.51741,10.135);
\draw [color=c, fill=c] (2.51741,10.0292) rectangle (2.55721,10.135);
\draw [color=c, fill=c] (2.55721,10.0292) rectangle (2.59702,10.135);
\draw [color=c, fill=c] (2.59702,10.0292) rectangle (2.63682,10.135);
\draw [color=c, fill=c] (2.63682,10.0292) rectangle (2.67662,10.135);
\draw [color=c, fill=c] (2.67662,10.0292) rectangle (2.71642,10.135);
\draw [color=c, fill=c] (2.71642,10.0292) rectangle (2.75622,10.135);
\draw [color=c, fill=c] (2.75622,10.0292) rectangle (2.79602,10.135);
\draw [color=c, fill=c] (2.79602,10.0292) rectangle (2.83582,10.135);
\draw [color=c, fill=c] (2.83582,10.0292) rectangle (2.87562,10.135);
\draw [color=c, fill=c] (2.87562,10.0292) rectangle (2.91542,10.135);
\draw [color=c, fill=c] (2.91542,10.0292) rectangle (2.95522,10.135);
\draw [color=c, fill=c] (2.95522,10.0292) rectangle (2.99502,10.135);
\draw [color=c, fill=c] (2.99502,10.0292) rectangle (3.03483,10.135);
\draw [color=c, fill=c] (3.03483,10.0292) rectangle (3.07463,10.135);
\draw [color=c, fill=c] (3.07463,10.0292) rectangle (3.11443,10.135);
\draw [color=c, fill=c] (3.11443,10.0292) rectangle (3.15423,10.135);
\draw [color=c, fill=c] (3.15423,10.0292) rectangle (3.19403,10.135);
\draw [color=c, fill=c] (3.19403,10.0292) rectangle (3.23383,10.135);
\draw [color=c, fill=c] (3.23383,10.0292) rectangle (3.27363,10.135);
\draw [color=c, fill=c] (3.27363,10.0292) rectangle (3.31343,10.135);
\draw [color=c, fill=c] (3.31343,10.0292) rectangle (3.35323,10.135);
\draw [color=c, fill=c] (3.35323,10.0292) rectangle (3.39303,10.135);
\draw [color=c, fill=c] (3.39303,10.0292) rectangle (3.43284,10.135);
\draw [color=c, fill=c] (3.43284,10.0292) rectangle (3.47264,10.135);
\draw [color=c, fill=c] (3.47264,10.0292) rectangle (3.51244,10.135);
\draw [color=c, fill=c] (3.51244,10.0292) rectangle (3.55224,10.135);
\draw [color=c, fill=c] (3.55224,10.0292) rectangle (3.59204,10.135);
\draw [color=c, fill=c] (3.59204,10.0292) rectangle (3.63184,10.135);
\draw [color=c, fill=c] (3.63184,10.0292) rectangle (3.67164,10.135);
\draw [color=c, fill=c] (3.67164,10.0292) rectangle (3.71144,10.135);
\draw [color=c, fill=c] (3.71144,10.0292) rectangle (3.75124,10.135);
\draw [color=c, fill=c] (3.75124,10.0292) rectangle (3.79104,10.135);
\draw [color=c, fill=c] (3.79104,10.0292) rectangle (3.83085,10.135);
\draw [color=c, fill=c] (3.83085,10.0292) rectangle (3.87065,10.135);
\draw [color=c, fill=c] (3.87065,10.0292) rectangle (3.91045,10.135);
\draw [color=c, fill=c] (3.91045,10.0292) rectangle (3.95025,10.135);
\draw [color=c, fill=c] (3.95025,10.0292) rectangle (3.99005,10.135);
\draw [color=c, fill=c] (3.99005,10.0292) rectangle (4.02985,10.135);
\draw [color=c, fill=c] (4.02985,10.0292) rectangle (4.06965,10.135);
\draw [color=c, fill=c] (4.06965,10.0292) rectangle (4.10945,10.135);
\draw [color=c, fill=c] (4.10945,10.0292) rectangle (4.14925,10.135);
\draw [color=c, fill=c] (4.14925,10.0292) rectangle (4.18905,10.135);
\draw [color=c, fill=c] (4.18905,10.0292) rectangle (4.22886,10.135);
\draw [color=c, fill=c] (4.22886,10.0292) rectangle (4.26866,10.135);
\draw [color=c, fill=c] (4.26866,10.0292) rectangle (4.30846,10.135);
\draw [color=c, fill=c] (4.30846,10.0292) rectangle (4.34826,10.135);
\draw [color=c, fill=c] (4.34826,10.0292) rectangle (4.38806,10.135);
\draw [color=c, fill=c] (4.38806,10.0292) rectangle (4.42786,10.135);
\draw [color=c, fill=c] (4.42786,10.0292) rectangle (4.46766,10.135);
\draw [color=c, fill=c] (4.46766,10.0292) rectangle (4.50746,10.135);
\draw [color=c, fill=c] (4.50746,10.0292) rectangle (4.54726,10.135);
\draw [color=c, fill=c] (4.54726,10.0292) rectangle (4.58706,10.135);
\draw [color=c, fill=c] (4.58706,10.0292) rectangle (4.62687,10.135);
\draw [color=c, fill=c] (4.62687,10.0292) rectangle (4.66667,10.135);
\draw [color=c, fill=c] (4.66667,10.0292) rectangle (4.70647,10.135);
\draw [color=c, fill=c] (4.70647,10.0292) rectangle (4.74627,10.135);
\draw [color=c, fill=c] (4.74627,10.0292) rectangle (4.78607,10.135);
\draw [color=c, fill=c] (4.78607,10.0292) rectangle (4.82587,10.135);
\draw [color=c, fill=c] (4.82587,10.0292) rectangle (4.86567,10.135);
\draw [color=c, fill=c] (4.86567,10.0292) rectangle (4.90547,10.135);
\draw [color=c, fill=c] (4.90547,10.0292) rectangle (4.94527,10.135);
\draw [color=c, fill=c] (4.94527,10.0292) rectangle (4.98507,10.135);
\draw [color=c, fill=c] (4.98507,10.0292) rectangle (5.02488,10.135);
\draw [color=c, fill=c] (5.02488,10.0292) rectangle (5.06468,10.135);
\draw [color=c, fill=c] (5.06468,10.0292) rectangle (5.10448,10.135);
\draw [color=c, fill=c] (5.10448,10.0292) rectangle (5.14428,10.135);
\draw [color=c, fill=c] (5.14428,10.0292) rectangle (5.18408,10.135);
\draw [color=c, fill=c] (5.18408,10.0292) rectangle (5.22388,10.135);
\draw [color=c, fill=c] (5.22388,10.0292) rectangle (5.26368,10.135);
\draw [color=c, fill=c] (5.26368,10.0292) rectangle (5.30348,10.135);
\draw [color=c, fill=c] (5.30348,10.0292) rectangle (5.34328,10.135);
\draw [color=c, fill=c] (5.34328,10.0292) rectangle (5.38308,10.135);
\draw [color=c, fill=c] (5.38308,10.0292) rectangle (5.42289,10.135);
\draw [color=c, fill=c] (5.42289,10.0292) rectangle (5.46269,10.135);
\draw [color=c, fill=c] (5.46269,10.0292) rectangle (5.50249,10.135);
\draw [color=c, fill=c] (5.50249,10.0292) rectangle (5.54229,10.135);
\draw [color=c, fill=c] (5.54229,10.0292) rectangle (5.58209,10.135);
\draw [color=c, fill=c] (5.58209,10.0292) rectangle (5.62189,10.135);
\draw [color=c, fill=c] (5.62189,10.0292) rectangle (5.66169,10.135);
\draw [color=c, fill=c] (5.66169,10.0292) rectangle (5.70149,10.135);
\draw [color=c, fill=c] (5.70149,10.0292) rectangle (5.74129,10.135);
\draw [color=c, fill=c] (5.74129,10.0292) rectangle (5.78109,10.135);
\draw [color=c, fill=c] (5.78109,10.0292) rectangle (5.8209,10.135);
\draw [color=c, fill=c] (5.8209,10.0292) rectangle (5.8607,10.135);
\draw [color=c, fill=c] (5.8607,10.0292) rectangle (5.9005,10.135);
\draw [color=c, fill=c] (5.9005,10.0292) rectangle (5.9403,10.135);
\draw [color=c, fill=c] (5.9403,10.0292) rectangle (5.9801,10.135);
\draw [color=c, fill=c] (5.9801,10.0292) rectangle (6.0199,10.135);
\draw [color=c, fill=c] (6.0199,10.0292) rectangle (6.0597,10.135);
\draw [color=c, fill=c] (6.0597,10.0292) rectangle (6.0995,10.135);
\draw [color=c, fill=c] (6.0995,10.0292) rectangle (6.1393,10.135);
\draw [color=c, fill=c] (6.1393,10.0292) rectangle (6.1791,10.135);
\draw [color=c, fill=c] (6.1791,10.0292) rectangle (6.21891,10.135);
\draw [color=c, fill=c] (6.21891,10.0292) rectangle (6.25871,10.135);
\draw [color=c, fill=c] (6.25871,10.0292) rectangle (6.29851,10.135);
\draw [color=c, fill=c] (6.29851,10.0292) rectangle (6.33831,10.135);
\draw [color=c, fill=c] (6.33831,10.0292) rectangle (6.37811,10.135);
\draw [color=c, fill=c] (6.37811,10.0292) rectangle (6.41791,10.135);
\draw [color=c, fill=c] (6.41791,10.0292) rectangle (6.45771,10.135);
\draw [color=c, fill=c] (6.45771,10.0292) rectangle (6.49751,10.135);
\draw [color=c, fill=c] (6.49751,10.0292) rectangle (6.53731,10.135);
\draw [color=c, fill=c] (6.53731,10.0292) rectangle (6.57711,10.135);
\draw [color=c, fill=c] (6.57711,10.0292) rectangle (6.61692,10.135);
\draw [color=c, fill=c] (6.61692,10.0292) rectangle (6.65672,10.135);
\draw [color=c, fill=c] (6.65672,10.0292) rectangle (6.69652,10.135);
\draw [color=c, fill=c] (6.69652,10.0292) rectangle (6.73632,10.135);
\draw [color=c, fill=c] (6.73632,10.0292) rectangle (6.77612,10.135);
\draw [color=c, fill=c] (6.77612,10.0292) rectangle (6.81592,10.135);
\draw [color=c, fill=c] (6.81592,10.0292) rectangle (6.85572,10.135);
\draw [color=c, fill=c] (6.85572,10.0292) rectangle (6.89552,10.135);
\draw [color=c, fill=c] (6.89552,10.0292) rectangle (6.93532,10.135);
\draw [color=c, fill=c] (6.93532,10.0292) rectangle (6.97512,10.135);
\draw [color=c, fill=c] (6.97512,10.0292) rectangle (7.01493,10.135);
\draw [color=c, fill=c] (7.01493,10.0292) rectangle (7.05473,10.135);
\draw [color=c, fill=c] (7.05473,10.0292) rectangle (7.09453,10.135);
\draw [color=c, fill=c] (7.09453,10.0292) rectangle (7.13433,10.135);
\draw [color=c, fill=c] (7.13433,10.0292) rectangle (7.17413,10.135);
\draw [color=c, fill=c] (7.17413,10.0292) rectangle (7.21393,10.135);
\draw [color=c, fill=c] (7.21393,10.0292) rectangle (7.25373,10.135);
\draw [color=c, fill=c] (7.25373,10.0292) rectangle (7.29353,10.135);
\draw [color=c, fill=c] (7.29353,10.0292) rectangle (7.33333,10.135);
\draw [color=c, fill=c] (7.33333,10.0292) rectangle (7.37313,10.135);
\draw [color=c, fill=c] (7.37313,10.0292) rectangle (7.41294,10.135);
\draw [color=c, fill=c] (7.41294,10.0292) rectangle (7.45274,10.135);
\draw [color=c, fill=c] (7.45274,10.0292) rectangle (7.49254,10.135);
\draw [color=c, fill=c] (7.49254,10.0292) rectangle (7.53234,10.135);
\draw [color=c, fill=c] (7.53234,10.0292) rectangle (7.57214,10.135);
\draw [color=c, fill=c] (7.57214,10.0292) rectangle (7.61194,10.135);
\draw [color=c, fill=c] (7.61194,10.0292) rectangle (7.65174,10.135);
\draw [color=c, fill=c] (7.65174,10.0292) rectangle (7.69154,10.135);
\draw [color=c, fill=c] (7.69154,10.0292) rectangle (7.73134,10.135);
\definecolor{c}{rgb}{0,0.0800001,1};
\draw [color=c, fill=c] (7.73134,10.0292) rectangle (7.77114,10.135);
\draw [color=c, fill=c] (7.77114,10.0292) rectangle (7.81095,10.135);
\draw [color=c, fill=c] (7.81095,10.0292) rectangle (7.85075,10.135);
\draw [color=c, fill=c] (7.85075,10.0292) rectangle (7.89055,10.135);
\draw [color=c, fill=c] (7.89055,10.0292) rectangle (7.93035,10.135);
\draw [color=c, fill=c] (7.93035,10.0292) rectangle (7.97015,10.135);
\draw [color=c, fill=c] (7.97015,10.0292) rectangle (8.00995,10.135);
\draw [color=c, fill=c] (8.00995,10.0292) rectangle (8.04975,10.135);
\draw [color=c, fill=c] (8.04975,10.0292) rectangle (8.08955,10.135);
\draw [color=c, fill=c] (8.08955,10.0292) rectangle (8.12935,10.135);
\draw [color=c, fill=c] (8.12935,10.0292) rectangle (8.16915,10.135);
\draw [color=c, fill=c] (8.16915,10.0292) rectangle (8.20895,10.135);
\draw [color=c, fill=c] (8.20895,10.0292) rectangle (8.24876,10.135);
\draw [color=c, fill=c] (8.24876,10.0292) rectangle (8.28856,10.135);
\draw [color=c, fill=c] (8.28856,10.0292) rectangle (8.32836,10.135);
\draw [color=c, fill=c] (8.32836,10.0292) rectangle (8.36816,10.135);
\draw [color=c, fill=c] (8.36816,10.0292) rectangle (8.40796,10.135);
\draw [color=c, fill=c] (8.40796,10.0292) rectangle (8.44776,10.135);
\draw [color=c, fill=c] (8.44776,10.0292) rectangle (8.48756,10.135);
\draw [color=c, fill=c] (8.48756,10.0292) rectangle (8.52736,10.135);
\draw [color=c, fill=c] (8.52736,10.0292) rectangle (8.56716,10.135);
\draw [color=c, fill=c] (8.56716,10.0292) rectangle (8.60697,10.135);
\draw [color=c, fill=c] (8.60697,10.0292) rectangle (8.64677,10.135);
\draw [color=c, fill=c] (8.64677,10.0292) rectangle (8.68657,10.135);
\draw [color=c, fill=c] (8.68657,10.0292) rectangle (8.72637,10.135);
\draw [color=c, fill=c] (8.72637,10.0292) rectangle (8.76617,10.135);
\draw [color=c, fill=c] (8.76617,10.0292) rectangle (8.80597,10.135);
\draw [color=c, fill=c] (8.80597,10.0292) rectangle (8.84577,10.135);
\draw [color=c, fill=c] (8.84577,10.0292) rectangle (8.88557,10.135);
\draw [color=c, fill=c] (8.88557,10.0292) rectangle (8.92537,10.135);
\draw [color=c, fill=c] (8.92537,10.0292) rectangle (8.96517,10.135);
\draw [color=c, fill=c] (8.96517,10.0292) rectangle (9.00498,10.135);
\draw [color=c, fill=c] (9.00498,10.0292) rectangle (9.04478,10.135);
\draw [color=c, fill=c] (9.04478,10.0292) rectangle (9.08458,10.135);
\draw [color=c, fill=c] (9.08458,10.0292) rectangle (9.12438,10.135);
\draw [color=c, fill=c] (9.12438,10.0292) rectangle (9.16418,10.135);
\draw [color=c, fill=c] (9.16418,10.0292) rectangle (9.20398,10.135);
\draw [color=c, fill=c] (9.20398,10.0292) rectangle (9.24378,10.135);
\draw [color=c, fill=c] (9.24378,10.0292) rectangle (9.28358,10.135);
\draw [color=c, fill=c] (9.28358,10.0292) rectangle (9.32338,10.135);
\draw [color=c, fill=c] (9.32338,10.0292) rectangle (9.36318,10.135);
\draw [color=c, fill=c] (9.36318,10.0292) rectangle (9.40298,10.135);
\draw [color=c, fill=c] (9.40298,10.0292) rectangle (9.44279,10.135);
\draw [color=c, fill=c] (9.44279,10.0292) rectangle (9.48259,10.135);
\draw [color=c, fill=c] (9.48259,10.0292) rectangle (9.52239,10.135);
\definecolor{c}{rgb}{0,0.266667,1};
\draw [color=c, fill=c] (9.52239,10.0292) rectangle (9.56219,10.135);
\draw [color=c, fill=c] (9.56219,10.0292) rectangle (9.60199,10.135);
\draw [color=c, fill=c] (9.60199,10.0292) rectangle (9.64179,10.135);
\draw [color=c, fill=c] (9.64179,10.0292) rectangle (9.68159,10.135);
\draw [color=c, fill=c] (9.68159,10.0292) rectangle (9.72139,10.135);
\draw [color=c, fill=c] (9.72139,10.0292) rectangle (9.76119,10.135);
\draw [color=c, fill=c] (9.76119,10.0292) rectangle (9.80099,10.135);
\draw [color=c, fill=c] (9.80099,10.0292) rectangle (9.8408,10.135);
\draw [color=c, fill=c] (9.8408,10.0292) rectangle (9.8806,10.135);
\draw [color=c, fill=c] (9.8806,10.0292) rectangle (9.9204,10.135);
\draw [color=c, fill=c] (9.9204,10.0292) rectangle (9.9602,10.135);
\draw [color=c, fill=c] (9.9602,10.0292) rectangle (10,10.135);
\draw [color=c, fill=c] (10,10.0292) rectangle (10.0398,10.135);
\draw [color=c, fill=c] (10.0398,10.0292) rectangle (10.0796,10.135);
\draw [color=c, fill=c] (10.0796,10.0292) rectangle (10.1194,10.135);
\draw [color=c, fill=c] (10.1194,10.0292) rectangle (10.1592,10.135);
\draw [color=c, fill=c] (10.1592,10.0292) rectangle (10.199,10.135);
\draw [color=c, fill=c] (10.199,10.0292) rectangle (10.2388,10.135);
\draw [color=c, fill=c] (10.2388,10.0292) rectangle (10.2786,10.135);
\draw [color=c, fill=c] (10.2786,10.0292) rectangle (10.3184,10.135);
\draw [color=c, fill=c] (10.3184,10.0292) rectangle (10.3582,10.135);
\draw [color=c, fill=c] (10.3582,10.0292) rectangle (10.398,10.135);
\draw [color=c, fill=c] (10.398,10.0292) rectangle (10.4378,10.135);
\draw [color=c, fill=c] (10.4378,10.0292) rectangle (10.4776,10.135);
\draw [color=c, fill=c] (10.4776,10.0292) rectangle (10.5174,10.135);
\draw [color=c, fill=c] (10.5174,10.0292) rectangle (10.5572,10.135);
\draw [color=c, fill=c] (10.5572,10.0292) rectangle (10.597,10.135);
\draw [color=c, fill=c] (10.597,10.0292) rectangle (10.6368,10.135);
\draw [color=c, fill=c] (10.6368,10.0292) rectangle (10.6766,10.135);
\draw [color=c, fill=c] (10.6766,10.0292) rectangle (10.7164,10.135);
\draw [color=c, fill=c] (10.7164,10.0292) rectangle (10.7562,10.135);
\draw [color=c, fill=c] (10.7562,10.0292) rectangle (10.796,10.135);
\definecolor{c}{rgb}{0,0.546666,1};
\draw [color=c, fill=c] (10.796,10.0292) rectangle (10.8358,10.135);
\draw [color=c, fill=c] (10.8358,10.0292) rectangle (10.8756,10.135);
\draw [color=c, fill=c] (10.8756,10.0292) rectangle (10.9154,10.135);
\draw [color=c, fill=c] (10.9154,10.0292) rectangle (10.9552,10.135);
\draw [color=c, fill=c] (10.9552,10.0292) rectangle (10.995,10.135);
\draw [color=c, fill=c] (10.995,10.0292) rectangle (11.0348,10.135);
\draw [color=c, fill=c] (11.0348,10.0292) rectangle (11.0746,10.135);
\draw [color=c, fill=c] (11.0746,10.0292) rectangle (11.1144,10.135);
\draw [color=c, fill=c] (11.1144,10.0292) rectangle (11.1542,10.135);
\draw [color=c, fill=c] (11.1542,10.0292) rectangle (11.194,10.135);
\draw [color=c, fill=c] (11.194,10.0292) rectangle (11.2338,10.135);
\draw [color=c, fill=c] (11.2338,10.0292) rectangle (11.2736,10.135);
\draw [color=c, fill=c] (11.2736,10.0292) rectangle (11.3134,10.135);
\draw [color=c, fill=c] (11.3134,10.0292) rectangle (11.3532,10.135);
\draw [color=c, fill=c] (11.3532,10.0292) rectangle (11.393,10.135);
\draw [color=c, fill=c] (11.393,10.0292) rectangle (11.4328,10.135);
\draw [color=c, fill=c] (11.4328,10.0292) rectangle (11.4726,10.135);
\draw [color=c, fill=c] (11.4726,10.0292) rectangle (11.5124,10.135);
\draw [color=c, fill=c] (11.5124,10.0292) rectangle (11.5522,10.135);
\draw [color=c, fill=c] (11.5522,10.0292) rectangle (11.592,10.135);
\draw [color=c, fill=c] (11.592,10.0292) rectangle (11.6318,10.135);
\draw [color=c, fill=c] (11.6318,10.0292) rectangle (11.6716,10.135);
\draw [color=c, fill=c] (11.6716,10.0292) rectangle (11.7114,10.135);
\draw [color=c, fill=c] (11.7114,10.0292) rectangle (11.7512,10.135);
\draw [color=c, fill=c] (11.7512,10.0292) rectangle (11.791,10.135);
\draw [color=c, fill=c] (11.791,10.0292) rectangle (11.8308,10.135);
\draw [color=c, fill=c] (11.8308,10.0292) rectangle (11.8706,10.135);
\draw [color=c, fill=c] (11.8706,10.0292) rectangle (11.9104,10.135);
\draw [color=c, fill=c] (11.9104,10.0292) rectangle (11.9502,10.135);
\draw [color=c, fill=c] (11.9502,10.0292) rectangle (11.99,10.135);
\draw [color=c, fill=c] (11.99,10.0292) rectangle (12.0299,10.135);
\draw [color=c, fill=c] (12.0299,10.0292) rectangle (12.0697,10.135);
\draw [color=c, fill=c] (12.0697,10.0292) rectangle (12.1095,10.135);
\draw [color=c, fill=c] (12.1095,10.0292) rectangle (12.1493,10.135);
\draw [color=c, fill=c] (12.1493,10.0292) rectangle (12.1891,10.135);
\draw [color=c, fill=c] (12.1891,10.0292) rectangle (12.2289,10.135);
\draw [color=c, fill=c] (12.2289,10.0292) rectangle (12.2687,10.135);
\draw [color=c, fill=c] (12.2687,10.0292) rectangle (12.3085,10.135);
\draw [color=c, fill=c] (12.3085,10.0292) rectangle (12.3483,10.135);
\draw [color=c, fill=c] (12.3483,10.0292) rectangle (12.3881,10.135);
\draw [color=c, fill=c] (12.3881,10.0292) rectangle (12.4279,10.135);
\draw [color=c, fill=c] (12.4279,10.0292) rectangle (12.4677,10.135);
\draw [color=c, fill=c] (12.4677,10.0292) rectangle (12.5075,10.135);
\draw [color=c, fill=c] (12.5075,10.0292) rectangle (12.5473,10.135);
\draw [color=c, fill=c] (12.5473,10.0292) rectangle (12.5871,10.135);
\draw [color=c, fill=c] (12.5871,10.0292) rectangle (12.6269,10.135);
\draw [color=c, fill=c] (12.6269,10.0292) rectangle (12.6667,10.135);
\draw [color=c, fill=c] (12.6667,10.0292) rectangle (12.7065,10.135);
\draw [color=c, fill=c] (12.7065,10.0292) rectangle (12.7463,10.135);
\draw [color=c, fill=c] (12.7463,10.0292) rectangle (12.7861,10.135);
\draw [color=c, fill=c] (12.7861,10.0292) rectangle (12.8259,10.135);
\draw [color=c, fill=c] (12.8259,10.0292) rectangle (12.8657,10.135);
\draw [color=c, fill=c] (12.8657,10.0292) rectangle (12.9055,10.135);
\draw [color=c, fill=c] (12.9055,10.0292) rectangle (12.9453,10.135);
\draw [color=c, fill=c] (12.9453,10.0292) rectangle (12.9851,10.135);
\draw [color=c, fill=c] (12.9851,10.0292) rectangle (13.0249,10.135);
\draw [color=c, fill=c] (13.0249,10.0292) rectangle (13.0647,10.135);
\draw [color=c, fill=c] (13.0647,10.0292) rectangle (13.1045,10.135);
\draw [color=c, fill=c] (13.1045,10.0292) rectangle (13.1443,10.135);
\draw [color=c, fill=c] (13.1443,10.0292) rectangle (13.1841,10.135);
\draw [color=c, fill=c] (13.1841,10.0292) rectangle (13.2239,10.135);
\draw [color=c, fill=c] (13.2239,10.0292) rectangle (13.2637,10.135);
\draw [color=c, fill=c] (13.2637,10.0292) rectangle (13.3035,10.135);
\draw [color=c, fill=c] (13.3035,10.0292) rectangle (13.3433,10.135);
\draw [color=c, fill=c] (13.3433,10.0292) rectangle (13.3831,10.135);
\draw [color=c, fill=c] (13.3831,10.0292) rectangle (13.4229,10.135);
\draw [color=c, fill=c] (13.4229,10.0292) rectangle (13.4627,10.135);
\draw [color=c, fill=c] (13.4627,10.0292) rectangle (13.5025,10.135);
\draw [color=c, fill=c] (13.5025,10.0292) rectangle (13.5423,10.135);
\draw [color=c, fill=c] (13.5423,10.0292) rectangle (13.5821,10.135);
\draw [color=c, fill=c] (13.5821,10.0292) rectangle (13.6219,10.135);
\draw [color=c, fill=c] (13.6219,10.0292) rectangle (13.6617,10.135);
\draw [color=c, fill=c] (13.6617,10.0292) rectangle (13.7015,10.135);
\draw [color=c, fill=c] (13.7015,10.0292) rectangle (13.7413,10.135);
\draw [color=c, fill=c] (13.7413,10.0292) rectangle (13.7811,10.135);
\draw [color=c, fill=c] (13.7811,10.0292) rectangle (13.8209,10.135);
\draw [color=c, fill=c] (13.8209,10.0292) rectangle (13.8607,10.135);
\definecolor{c}{rgb}{0,0.733333,1};
\draw [color=c, fill=c] (13.8607,10.0292) rectangle (13.9005,10.135);
\draw [color=c, fill=c] (13.9005,10.0292) rectangle (13.9403,10.135);
\draw [color=c, fill=c] (13.9403,10.0292) rectangle (13.9801,10.135);
\draw [color=c, fill=c] (13.9801,10.0292) rectangle (14.0199,10.135);
\draw [color=c, fill=c] (14.0199,10.0292) rectangle (14.0597,10.135);
\draw [color=c, fill=c] (14.0597,10.0292) rectangle (14.0995,10.135);
\draw [color=c, fill=c] (14.0995,10.0292) rectangle (14.1393,10.135);
\draw [color=c, fill=c] (14.1393,10.0292) rectangle (14.1791,10.135);
\draw [color=c, fill=c] (14.1791,10.0292) rectangle (14.2189,10.135);
\draw [color=c, fill=c] (14.2189,10.0292) rectangle (14.2587,10.135);
\draw [color=c, fill=c] (14.2587,10.0292) rectangle (14.2985,10.135);
\draw [color=c, fill=c] (14.2985,10.0292) rectangle (14.3383,10.135);
\draw [color=c, fill=c] (14.3383,10.0292) rectangle (14.3781,10.135);
\draw [color=c, fill=c] (14.3781,10.0292) rectangle (14.4179,10.135);
\draw [color=c, fill=c] (14.4179,10.0292) rectangle (14.4577,10.135);
\draw [color=c, fill=c] (14.4577,10.0292) rectangle (14.4975,10.135);
\draw [color=c, fill=c] (14.4975,10.0292) rectangle (14.5373,10.135);
\draw [color=c, fill=c] (14.5373,10.0292) rectangle (14.5771,10.135);
\draw [color=c, fill=c] (14.5771,10.0292) rectangle (14.6169,10.135);
\draw [color=c, fill=c] (14.6169,10.0292) rectangle (14.6567,10.135);
\draw [color=c, fill=c] (14.6567,10.0292) rectangle (14.6965,10.135);
\draw [color=c, fill=c] (14.6965,10.0292) rectangle (14.7363,10.135);
\draw [color=c, fill=c] (14.7363,10.0292) rectangle (14.7761,10.135);
\draw [color=c, fill=c] (14.7761,10.0292) rectangle (14.8159,10.135);
\draw [color=c, fill=c] (14.8159,10.0292) rectangle (14.8557,10.135);
\draw [color=c, fill=c] (14.8557,10.0292) rectangle (14.8955,10.135);
\draw [color=c, fill=c] (14.8955,10.0292) rectangle (14.9353,10.135);
\draw [color=c, fill=c] (14.9353,10.0292) rectangle (14.9751,10.135);
\draw [color=c, fill=c] (14.9751,10.0292) rectangle (15.0149,10.135);
\draw [color=c, fill=c] (15.0149,10.0292) rectangle (15.0547,10.135);
\draw [color=c, fill=c] (15.0547,10.0292) rectangle (15.0945,10.135);
\draw [color=c, fill=c] (15.0945,10.0292) rectangle (15.1343,10.135);
\draw [color=c, fill=c] (15.1343,10.0292) rectangle (15.1741,10.135);
\draw [color=c, fill=c] (15.1741,10.0292) rectangle (15.2139,10.135);
\draw [color=c, fill=c] (15.2139,10.0292) rectangle (15.2537,10.135);
\draw [color=c, fill=c] (15.2537,10.0292) rectangle (15.2935,10.135);
\draw [color=c, fill=c] (15.2935,10.0292) rectangle (15.3333,10.135);
\draw [color=c, fill=c] (15.3333,10.0292) rectangle (15.3731,10.135);
\draw [color=c, fill=c] (15.3731,10.0292) rectangle (15.4129,10.135);
\draw [color=c, fill=c] (15.4129,10.0292) rectangle (15.4527,10.135);
\draw [color=c, fill=c] (15.4527,10.0292) rectangle (15.4925,10.135);
\draw [color=c, fill=c] (15.4925,10.0292) rectangle (15.5323,10.135);
\draw [color=c, fill=c] (15.5323,10.0292) rectangle (15.5721,10.135);
\draw [color=c, fill=c] (15.5721,10.0292) rectangle (15.6119,10.135);
\draw [color=c, fill=c] (15.6119,10.0292) rectangle (15.6517,10.135);
\draw [color=c, fill=c] (15.6517,10.0292) rectangle (15.6915,10.135);
\draw [color=c, fill=c] (15.6915,10.0292) rectangle (15.7313,10.135);
\draw [color=c, fill=c] (15.7313,10.0292) rectangle (15.7711,10.135);
\draw [color=c, fill=c] (15.7711,10.0292) rectangle (15.8109,10.135);
\draw [color=c, fill=c] (15.8109,10.0292) rectangle (15.8507,10.135);
\draw [color=c, fill=c] (15.8507,10.0292) rectangle (15.8905,10.135);
\draw [color=c, fill=c] (15.8905,10.0292) rectangle (15.9303,10.135);
\draw [color=c, fill=c] (15.9303,10.0292) rectangle (15.9701,10.135);
\draw [color=c, fill=c] (15.9701,10.0292) rectangle (16.01,10.135);
\draw [color=c, fill=c] (16.01,10.0292) rectangle (16.0498,10.135);
\draw [color=c, fill=c] (16.0498,10.0292) rectangle (16.0896,10.135);
\draw [color=c, fill=c] (16.0896,10.0292) rectangle (16.1294,10.135);
\draw [color=c, fill=c] (16.1294,10.0292) rectangle (16.1692,10.135);
\draw [color=c, fill=c] (16.1692,10.0292) rectangle (16.209,10.135);
\draw [color=c, fill=c] (16.209,10.0292) rectangle (16.2488,10.135);
\draw [color=c, fill=c] (16.2488,10.0292) rectangle (16.2886,10.135);
\draw [color=c, fill=c] (16.2886,10.0292) rectangle (16.3284,10.135);
\draw [color=c, fill=c] (16.3284,10.0292) rectangle (16.3682,10.135);
\draw [color=c, fill=c] (16.3682,10.0292) rectangle (16.408,10.135);
\draw [color=c, fill=c] (16.408,10.0292) rectangle (16.4478,10.135);
\draw [color=c, fill=c] (16.4478,10.0292) rectangle (16.4876,10.135);
\draw [color=c, fill=c] (16.4876,10.0292) rectangle (16.5274,10.135);
\draw [color=c, fill=c] (16.5274,10.0292) rectangle (16.5672,10.135);
\draw [color=c, fill=c] (16.5672,10.0292) rectangle (16.607,10.135);
\draw [color=c, fill=c] (16.607,10.0292) rectangle (16.6468,10.135);
\draw [color=c, fill=c] (16.6468,10.0292) rectangle (16.6866,10.135);
\draw [color=c, fill=c] (16.6866,10.0292) rectangle (16.7264,10.135);
\draw [color=c, fill=c] (16.7264,10.0292) rectangle (16.7662,10.135);
\draw [color=c, fill=c] (16.7662,10.0292) rectangle (16.806,10.135);
\draw [color=c, fill=c] (16.806,10.0292) rectangle (16.8458,10.135);
\draw [color=c, fill=c] (16.8458,10.0292) rectangle (16.8856,10.135);
\draw [color=c, fill=c] (16.8856,10.0292) rectangle (16.9254,10.135);
\draw [color=c, fill=c] (16.9254,10.0292) rectangle (16.9652,10.135);
\draw [color=c, fill=c] (16.9652,10.0292) rectangle (17.005,10.135);
\draw [color=c, fill=c] (17.005,10.0292) rectangle (17.0448,10.135);
\draw [color=c, fill=c] (17.0448,10.0292) rectangle (17.0846,10.135);
\draw [color=c, fill=c] (17.0846,10.0292) rectangle (17.1244,10.135);
\draw [color=c, fill=c] (17.1244,10.0292) rectangle (17.1642,10.135);
\draw [color=c, fill=c] (17.1642,10.0292) rectangle (17.204,10.135);
\draw [color=c, fill=c] (17.204,10.0292) rectangle (17.2438,10.135);
\draw [color=c, fill=c] (17.2438,10.0292) rectangle (17.2836,10.135);
\draw [color=c, fill=c] (17.2836,10.0292) rectangle (17.3234,10.135);
\draw [color=c, fill=c] (17.3234,10.0292) rectangle (17.3632,10.135);
\draw [color=c, fill=c] (17.3632,10.0292) rectangle (17.403,10.135);
\draw [color=c, fill=c] (17.403,10.0292) rectangle (17.4428,10.135);
\draw [color=c, fill=c] (17.4428,10.0292) rectangle (17.4826,10.135);
\draw [color=c, fill=c] (17.4826,10.0292) rectangle (17.5224,10.135);
\draw [color=c, fill=c] (17.5224,10.0292) rectangle (17.5622,10.135);
\draw [color=c, fill=c] (17.5622,10.0292) rectangle (17.602,10.135);
\draw [color=c, fill=c] (17.602,10.0292) rectangle (17.6418,10.135);
\draw [color=c, fill=c] (17.6418,10.0292) rectangle (17.6816,10.135);
\draw [color=c, fill=c] (17.6816,10.0292) rectangle (17.7214,10.135);
\draw [color=c, fill=c] (17.7214,10.0292) rectangle (17.7612,10.135);
\draw [color=c, fill=c] (17.7612,10.0292) rectangle (17.801,10.135);
\draw [color=c, fill=c] (17.801,10.0292) rectangle (17.8408,10.135);
\draw [color=c, fill=c] (17.8408,10.0292) rectangle (17.8806,10.135);
\draw [color=c, fill=c] (17.8806,10.0292) rectangle (17.9204,10.135);
\draw [color=c, fill=c] (17.9204,10.0292) rectangle (17.9602,10.135);
\draw [color=c, fill=c] (17.9602,10.0292) rectangle (18,10.135);
\definecolor{c}{rgb}{0.2,0,1};
\draw [color=c, fill=c] (2,10.135) rectangle (2.0398,10.2409);
\draw [color=c, fill=c] (2.0398,10.135) rectangle (2.0796,10.2409);
\draw [color=c, fill=c] (2.0796,10.135) rectangle (2.1194,10.2409);
\draw [color=c, fill=c] (2.1194,10.135) rectangle (2.1592,10.2409);
\draw [color=c, fill=c] (2.1592,10.135) rectangle (2.19901,10.2409);
\draw [color=c, fill=c] (2.19901,10.135) rectangle (2.23881,10.2409);
\draw [color=c, fill=c] (2.23881,10.135) rectangle (2.27861,10.2409);
\draw [color=c, fill=c] (2.27861,10.135) rectangle (2.31841,10.2409);
\draw [color=c, fill=c] (2.31841,10.135) rectangle (2.35821,10.2409);
\draw [color=c, fill=c] (2.35821,10.135) rectangle (2.39801,10.2409);
\draw [color=c, fill=c] (2.39801,10.135) rectangle (2.43781,10.2409);
\draw [color=c, fill=c] (2.43781,10.135) rectangle (2.47761,10.2409);
\draw [color=c, fill=c] (2.47761,10.135) rectangle (2.51741,10.2409);
\draw [color=c, fill=c] (2.51741,10.135) rectangle (2.55721,10.2409);
\draw [color=c, fill=c] (2.55721,10.135) rectangle (2.59702,10.2409);
\draw [color=c, fill=c] (2.59702,10.135) rectangle (2.63682,10.2409);
\draw [color=c, fill=c] (2.63682,10.135) rectangle (2.67662,10.2409);
\draw [color=c, fill=c] (2.67662,10.135) rectangle (2.71642,10.2409);
\draw [color=c, fill=c] (2.71642,10.135) rectangle (2.75622,10.2409);
\draw [color=c, fill=c] (2.75622,10.135) rectangle (2.79602,10.2409);
\draw [color=c, fill=c] (2.79602,10.135) rectangle (2.83582,10.2409);
\draw [color=c, fill=c] (2.83582,10.135) rectangle (2.87562,10.2409);
\draw [color=c, fill=c] (2.87562,10.135) rectangle (2.91542,10.2409);
\draw [color=c, fill=c] (2.91542,10.135) rectangle (2.95522,10.2409);
\draw [color=c, fill=c] (2.95522,10.135) rectangle (2.99502,10.2409);
\draw [color=c, fill=c] (2.99502,10.135) rectangle (3.03483,10.2409);
\draw [color=c, fill=c] (3.03483,10.135) rectangle (3.07463,10.2409);
\draw [color=c, fill=c] (3.07463,10.135) rectangle (3.11443,10.2409);
\draw [color=c, fill=c] (3.11443,10.135) rectangle (3.15423,10.2409);
\draw [color=c, fill=c] (3.15423,10.135) rectangle (3.19403,10.2409);
\draw [color=c, fill=c] (3.19403,10.135) rectangle (3.23383,10.2409);
\draw [color=c, fill=c] (3.23383,10.135) rectangle (3.27363,10.2409);
\draw [color=c, fill=c] (3.27363,10.135) rectangle (3.31343,10.2409);
\draw [color=c, fill=c] (3.31343,10.135) rectangle (3.35323,10.2409);
\draw [color=c, fill=c] (3.35323,10.135) rectangle (3.39303,10.2409);
\draw [color=c, fill=c] (3.39303,10.135) rectangle (3.43284,10.2409);
\draw [color=c, fill=c] (3.43284,10.135) rectangle (3.47264,10.2409);
\draw [color=c, fill=c] (3.47264,10.135) rectangle (3.51244,10.2409);
\draw [color=c, fill=c] (3.51244,10.135) rectangle (3.55224,10.2409);
\draw [color=c, fill=c] (3.55224,10.135) rectangle (3.59204,10.2409);
\draw [color=c, fill=c] (3.59204,10.135) rectangle (3.63184,10.2409);
\draw [color=c, fill=c] (3.63184,10.135) rectangle (3.67164,10.2409);
\draw [color=c, fill=c] (3.67164,10.135) rectangle (3.71144,10.2409);
\draw [color=c, fill=c] (3.71144,10.135) rectangle (3.75124,10.2409);
\draw [color=c, fill=c] (3.75124,10.135) rectangle (3.79104,10.2409);
\draw [color=c, fill=c] (3.79104,10.135) rectangle (3.83085,10.2409);
\draw [color=c, fill=c] (3.83085,10.135) rectangle (3.87065,10.2409);
\draw [color=c, fill=c] (3.87065,10.135) rectangle (3.91045,10.2409);
\draw [color=c, fill=c] (3.91045,10.135) rectangle (3.95025,10.2409);
\draw [color=c, fill=c] (3.95025,10.135) rectangle (3.99005,10.2409);
\draw [color=c, fill=c] (3.99005,10.135) rectangle (4.02985,10.2409);
\draw [color=c, fill=c] (4.02985,10.135) rectangle (4.06965,10.2409);
\draw [color=c, fill=c] (4.06965,10.135) rectangle (4.10945,10.2409);
\draw [color=c, fill=c] (4.10945,10.135) rectangle (4.14925,10.2409);
\draw [color=c, fill=c] (4.14925,10.135) rectangle (4.18905,10.2409);
\draw [color=c, fill=c] (4.18905,10.135) rectangle (4.22886,10.2409);
\draw [color=c, fill=c] (4.22886,10.135) rectangle (4.26866,10.2409);
\draw [color=c, fill=c] (4.26866,10.135) rectangle (4.30846,10.2409);
\draw [color=c, fill=c] (4.30846,10.135) rectangle (4.34826,10.2409);
\draw [color=c, fill=c] (4.34826,10.135) rectangle (4.38806,10.2409);
\draw [color=c, fill=c] (4.38806,10.135) rectangle (4.42786,10.2409);
\draw [color=c, fill=c] (4.42786,10.135) rectangle (4.46766,10.2409);
\draw [color=c, fill=c] (4.46766,10.135) rectangle (4.50746,10.2409);
\draw [color=c, fill=c] (4.50746,10.135) rectangle (4.54726,10.2409);
\draw [color=c, fill=c] (4.54726,10.135) rectangle (4.58706,10.2409);
\draw [color=c, fill=c] (4.58706,10.135) rectangle (4.62687,10.2409);
\draw [color=c, fill=c] (4.62687,10.135) rectangle (4.66667,10.2409);
\draw [color=c, fill=c] (4.66667,10.135) rectangle (4.70647,10.2409);
\draw [color=c, fill=c] (4.70647,10.135) rectangle (4.74627,10.2409);
\draw [color=c, fill=c] (4.74627,10.135) rectangle (4.78607,10.2409);
\draw [color=c, fill=c] (4.78607,10.135) rectangle (4.82587,10.2409);
\draw [color=c, fill=c] (4.82587,10.135) rectangle (4.86567,10.2409);
\draw [color=c, fill=c] (4.86567,10.135) rectangle (4.90547,10.2409);
\draw [color=c, fill=c] (4.90547,10.135) rectangle (4.94527,10.2409);
\draw [color=c, fill=c] (4.94527,10.135) rectangle (4.98507,10.2409);
\draw [color=c, fill=c] (4.98507,10.135) rectangle (5.02488,10.2409);
\draw [color=c, fill=c] (5.02488,10.135) rectangle (5.06468,10.2409);
\draw [color=c, fill=c] (5.06468,10.135) rectangle (5.10448,10.2409);
\draw [color=c, fill=c] (5.10448,10.135) rectangle (5.14428,10.2409);
\draw [color=c, fill=c] (5.14428,10.135) rectangle (5.18408,10.2409);
\draw [color=c, fill=c] (5.18408,10.135) rectangle (5.22388,10.2409);
\draw [color=c, fill=c] (5.22388,10.135) rectangle (5.26368,10.2409);
\draw [color=c, fill=c] (5.26368,10.135) rectangle (5.30348,10.2409);
\draw [color=c, fill=c] (5.30348,10.135) rectangle (5.34328,10.2409);
\draw [color=c, fill=c] (5.34328,10.135) rectangle (5.38308,10.2409);
\draw [color=c, fill=c] (5.38308,10.135) rectangle (5.42289,10.2409);
\draw [color=c, fill=c] (5.42289,10.135) rectangle (5.46269,10.2409);
\draw [color=c, fill=c] (5.46269,10.135) rectangle (5.50249,10.2409);
\draw [color=c, fill=c] (5.50249,10.135) rectangle (5.54229,10.2409);
\draw [color=c, fill=c] (5.54229,10.135) rectangle (5.58209,10.2409);
\draw [color=c, fill=c] (5.58209,10.135) rectangle (5.62189,10.2409);
\draw [color=c, fill=c] (5.62189,10.135) rectangle (5.66169,10.2409);
\draw [color=c, fill=c] (5.66169,10.135) rectangle (5.70149,10.2409);
\draw [color=c, fill=c] (5.70149,10.135) rectangle (5.74129,10.2409);
\draw [color=c, fill=c] (5.74129,10.135) rectangle (5.78109,10.2409);
\draw [color=c, fill=c] (5.78109,10.135) rectangle (5.8209,10.2409);
\draw [color=c, fill=c] (5.8209,10.135) rectangle (5.8607,10.2409);
\draw [color=c, fill=c] (5.8607,10.135) rectangle (5.9005,10.2409);
\draw [color=c, fill=c] (5.9005,10.135) rectangle (5.9403,10.2409);
\draw [color=c, fill=c] (5.9403,10.135) rectangle (5.9801,10.2409);
\draw [color=c, fill=c] (5.9801,10.135) rectangle (6.0199,10.2409);
\draw [color=c, fill=c] (6.0199,10.135) rectangle (6.0597,10.2409);
\draw [color=c, fill=c] (6.0597,10.135) rectangle (6.0995,10.2409);
\draw [color=c, fill=c] (6.0995,10.135) rectangle (6.1393,10.2409);
\draw [color=c, fill=c] (6.1393,10.135) rectangle (6.1791,10.2409);
\draw [color=c, fill=c] (6.1791,10.135) rectangle (6.21891,10.2409);
\draw [color=c, fill=c] (6.21891,10.135) rectangle (6.25871,10.2409);
\draw [color=c, fill=c] (6.25871,10.135) rectangle (6.29851,10.2409);
\draw [color=c, fill=c] (6.29851,10.135) rectangle (6.33831,10.2409);
\draw [color=c, fill=c] (6.33831,10.135) rectangle (6.37811,10.2409);
\draw [color=c, fill=c] (6.37811,10.135) rectangle (6.41791,10.2409);
\draw [color=c, fill=c] (6.41791,10.135) rectangle (6.45771,10.2409);
\draw [color=c, fill=c] (6.45771,10.135) rectangle (6.49751,10.2409);
\draw [color=c, fill=c] (6.49751,10.135) rectangle (6.53731,10.2409);
\draw [color=c, fill=c] (6.53731,10.135) rectangle (6.57711,10.2409);
\draw [color=c, fill=c] (6.57711,10.135) rectangle (6.61692,10.2409);
\draw [color=c, fill=c] (6.61692,10.135) rectangle (6.65672,10.2409);
\draw [color=c, fill=c] (6.65672,10.135) rectangle (6.69652,10.2409);
\draw [color=c, fill=c] (6.69652,10.135) rectangle (6.73632,10.2409);
\draw [color=c, fill=c] (6.73632,10.135) rectangle (6.77612,10.2409);
\draw [color=c, fill=c] (6.77612,10.135) rectangle (6.81592,10.2409);
\draw [color=c, fill=c] (6.81592,10.135) rectangle (6.85572,10.2409);
\draw [color=c, fill=c] (6.85572,10.135) rectangle (6.89552,10.2409);
\draw [color=c, fill=c] (6.89552,10.135) rectangle (6.93532,10.2409);
\draw [color=c, fill=c] (6.93532,10.135) rectangle (6.97512,10.2409);
\draw [color=c, fill=c] (6.97512,10.135) rectangle (7.01493,10.2409);
\draw [color=c, fill=c] (7.01493,10.135) rectangle (7.05473,10.2409);
\draw [color=c, fill=c] (7.05473,10.135) rectangle (7.09453,10.2409);
\draw [color=c, fill=c] (7.09453,10.135) rectangle (7.13433,10.2409);
\draw [color=c, fill=c] (7.13433,10.135) rectangle (7.17413,10.2409);
\draw [color=c, fill=c] (7.17413,10.135) rectangle (7.21393,10.2409);
\draw [color=c, fill=c] (7.21393,10.135) rectangle (7.25373,10.2409);
\draw [color=c, fill=c] (7.25373,10.135) rectangle (7.29353,10.2409);
\draw [color=c, fill=c] (7.29353,10.135) rectangle (7.33333,10.2409);
\draw [color=c, fill=c] (7.33333,10.135) rectangle (7.37313,10.2409);
\draw [color=c, fill=c] (7.37313,10.135) rectangle (7.41294,10.2409);
\draw [color=c, fill=c] (7.41294,10.135) rectangle (7.45274,10.2409);
\draw [color=c, fill=c] (7.45274,10.135) rectangle (7.49254,10.2409);
\draw [color=c, fill=c] (7.49254,10.135) rectangle (7.53234,10.2409);
\draw [color=c, fill=c] (7.53234,10.135) rectangle (7.57214,10.2409);
\draw [color=c, fill=c] (7.57214,10.135) rectangle (7.61194,10.2409);
\draw [color=c, fill=c] (7.61194,10.135) rectangle (7.65174,10.2409);
\draw [color=c, fill=c] (7.65174,10.135) rectangle (7.69154,10.2409);
\draw [color=c, fill=c] (7.69154,10.135) rectangle (7.73134,10.2409);
\definecolor{c}{rgb}{0,0.0800001,1};
\draw [color=c, fill=c] (7.73134,10.135) rectangle (7.77114,10.2409);
\draw [color=c, fill=c] (7.77114,10.135) rectangle (7.81095,10.2409);
\draw [color=c, fill=c] (7.81095,10.135) rectangle (7.85075,10.2409);
\draw [color=c, fill=c] (7.85075,10.135) rectangle (7.89055,10.2409);
\draw [color=c, fill=c] (7.89055,10.135) rectangle (7.93035,10.2409);
\draw [color=c, fill=c] (7.93035,10.135) rectangle (7.97015,10.2409);
\draw [color=c, fill=c] (7.97015,10.135) rectangle (8.00995,10.2409);
\draw [color=c, fill=c] (8.00995,10.135) rectangle (8.04975,10.2409);
\draw [color=c, fill=c] (8.04975,10.135) rectangle (8.08955,10.2409);
\draw [color=c, fill=c] (8.08955,10.135) rectangle (8.12935,10.2409);
\draw [color=c, fill=c] (8.12935,10.135) rectangle (8.16915,10.2409);
\draw [color=c, fill=c] (8.16915,10.135) rectangle (8.20895,10.2409);
\draw [color=c, fill=c] (8.20895,10.135) rectangle (8.24876,10.2409);
\draw [color=c, fill=c] (8.24876,10.135) rectangle (8.28856,10.2409);
\draw [color=c, fill=c] (8.28856,10.135) rectangle (8.32836,10.2409);
\draw [color=c, fill=c] (8.32836,10.135) rectangle (8.36816,10.2409);
\draw [color=c, fill=c] (8.36816,10.135) rectangle (8.40796,10.2409);
\draw [color=c, fill=c] (8.40796,10.135) rectangle (8.44776,10.2409);
\draw [color=c, fill=c] (8.44776,10.135) rectangle (8.48756,10.2409);
\draw [color=c, fill=c] (8.48756,10.135) rectangle (8.52736,10.2409);
\draw [color=c, fill=c] (8.52736,10.135) rectangle (8.56716,10.2409);
\draw [color=c, fill=c] (8.56716,10.135) rectangle (8.60697,10.2409);
\draw [color=c, fill=c] (8.60697,10.135) rectangle (8.64677,10.2409);
\draw [color=c, fill=c] (8.64677,10.135) rectangle (8.68657,10.2409);
\draw [color=c, fill=c] (8.68657,10.135) rectangle (8.72637,10.2409);
\draw [color=c, fill=c] (8.72637,10.135) rectangle (8.76617,10.2409);
\draw [color=c, fill=c] (8.76617,10.135) rectangle (8.80597,10.2409);
\draw [color=c, fill=c] (8.80597,10.135) rectangle (8.84577,10.2409);
\draw [color=c, fill=c] (8.84577,10.135) rectangle (8.88557,10.2409);
\draw [color=c, fill=c] (8.88557,10.135) rectangle (8.92537,10.2409);
\draw [color=c, fill=c] (8.92537,10.135) rectangle (8.96517,10.2409);
\draw [color=c, fill=c] (8.96517,10.135) rectangle (9.00498,10.2409);
\draw [color=c, fill=c] (9.00498,10.135) rectangle (9.04478,10.2409);
\draw [color=c, fill=c] (9.04478,10.135) rectangle (9.08458,10.2409);
\draw [color=c, fill=c] (9.08458,10.135) rectangle (9.12438,10.2409);
\draw [color=c, fill=c] (9.12438,10.135) rectangle (9.16418,10.2409);
\draw [color=c, fill=c] (9.16418,10.135) rectangle (9.20398,10.2409);
\draw [color=c, fill=c] (9.20398,10.135) rectangle (9.24378,10.2409);
\draw [color=c, fill=c] (9.24378,10.135) rectangle (9.28358,10.2409);
\draw [color=c, fill=c] (9.28358,10.135) rectangle (9.32338,10.2409);
\draw [color=c, fill=c] (9.32338,10.135) rectangle (9.36318,10.2409);
\draw [color=c, fill=c] (9.36318,10.135) rectangle (9.40298,10.2409);
\draw [color=c, fill=c] (9.40298,10.135) rectangle (9.44279,10.2409);
\draw [color=c, fill=c] (9.44279,10.135) rectangle (9.48259,10.2409);
\draw [color=c, fill=c] (9.48259,10.135) rectangle (9.52239,10.2409);
\definecolor{c}{rgb}{0,0.266667,1};
\draw [color=c, fill=c] (9.52239,10.135) rectangle (9.56219,10.2409);
\draw [color=c, fill=c] (9.56219,10.135) rectangle (9.60199,10.2409);
\draw [color=c, fill=c] (9.60199,10.135) rectangle (9.64179,10.2409);
\draw [color=c, fill=c] (9.64179,10.135) rectangle (9.68159,10.2409);
\draw [color=c, fill=c] (9.68159,10.135) rectangle (9.72139,10.2409);
\draw [color=c, fill=c] (9.72139,10.135) rectangle (9.76119,10.2409);
\draw [color=c, fill=c] (9.76119,10.135) rectangle (9.80099,10.2409);
\draw [color=c, fill=c] (9.80099,10.135) rectangle (9.8408,10.2409);
\draw [color=c, fill=c] (9.8408,10.135) rectangle (9.8806,10.2409);
\draw [color=c, fill=c] (9.8806,10.135) rectangle (9.9204,10.2409);
\draw [color=c, fill=c] (9.9204,10.135) rectangle (9.9602,10.2409);
\draw [color=c, fill=c] (9.9602,10.135) rectangle (10,10.2409);
\draw [color=c, fill=c] (10,10.135) rectangle (10.0398,10.2409);
\draw [color=c, fill=c] (10.0398,10.135) rectangle (10.0796,10.2409);
\draw [color=c, fill=c] (10.0796,10.135) rectangle (10.1194,10.2409);
\draw [color=c, fill=c] (10.1194,10.135) rectangle (10.1592,10.2409);
\draw [color=c, fill=c] (10.1592,10.135) rectangle (10.199,10.2409);
\draw [color=c, fill=c] (10.199,10.135) rectangle (10.2388,10.2409);
\draw [color=c, fill=c] (10.2388,10.135) rectangle (10.2786,10.2409);
\draw [color=c, fill=c] (10.2786,10.135) rectangle (10.3184,10.2409);
\draw [color=c, fill=c] (10.3184,10.135) rectangle (10.3582,10.2409);
\draw [color=c, fill=c] (10.3582,10.135) rectangle (10.398,10.2409);
\draw [color=c, fill=c] (10.398,10.135) rectangle (10.4378,10.2409);
\draw [color=c, fill=c] (10.4378,10.135) rectangle (10.4776,10.2409);
\draw [color=c, fill=c] (10.4776,10.135) rectangle (10.5174,10.2409);
\draw [color=c, fill=c] (10.5174,10.135) rectangle (10.5572,10.2409);
\draw [color=c, fill=c] (10.5572,10.135) rectangle (10.597,10.2409);
\draw [color=c, fill=c] (10.597,10.135) rectangle (10.6368,10.2409);
\draw [color=c, fill=c] (10.6368,10.135) rectangle (10.6766,10.2409);
\draw [color=c, fill=c] (10.6766,10.135) rectangle (10.7164,10.2409);
\draw [color=c, fill=c] (10.7164,10.135) rectangle (10.7562,10.2409);
\draw [color=c, fill=c] (10.7562,10.135) rectangle (10.796,10.2409);
\definecolor{c}{rgb}{0,0.546666,1};
\draw [color=c, fill=c] (10.796,10.135) rectangle (10.8358,10.2409);
\draw [color=c, fill=c] (10.8358,10.135) rectangle (10.8756,10.2409);
\draw [color=c, fill=c] (10.8756,10.135) rectangle (10.9154,10.2409);
\draw [color=c, fill=c] (10.9154,10.135) rectangle (10.9552,10.2409);
\draw [color=c, fill=c] (10.9552,10.135) rectangle (10.995,10.2409);
\draw [color=c, fill=c] (10.995,10.135) rectangle (11.0348,10.2409);
\draw [color=c, fill=c] (11.0348,10.135) rectangle (11.0746,10.2409);
\draw [color=c, fill=c] (11.0746,10.135) rectangle (11.1144,10.2409);
\draw [color=c, fill=c] (11.1144,10.135) rectangle (11.1542,10.2409);
\draw [color=c, fill=c] (11.1542,10.135) rectangle (11.194,10.2409);
\draw [color=c, fill=c] (11.194,10.135) rectangle (11.2338,10.2409);
\draw [color=c, fill=c] (11.2338,10.135) rectangle (11.2736,10.2409);
\draw [color=c, fill=c] (11.2736,10.135) rectangle (11.3134,10.2409);
\draw [color=c, fill=c] (11.3134,10.135) rectangle (11.3532,10.2409);
\draw [color=c, fill=c] (11.3532,10.135) rectangle (11.393,10.2409);
\draw [color=c, fill=c] (11.393,10.135) rectangle (11.4328,10.2409);
\draw [color=c, fill=c] (11.4328,10.135) rectangle (11.4726,10.2409);
\draw [color=c, fill=c] (11.4726,10.135) rectangle (11.5124,10.2409);
\draw [color=c, fill=c] (11.5124,10.135) rectangle (11.5522,10.2409);
\draw [color=c, fill=c] (11.5522,10.135) rectangle (11.592,10.2409);
\draw [color=c, fill=c] (11.592,10.135) rectangle (11.6318,10.2409);
\draw [color=c, fill=c] (11.6318,10.135) rectangle (11.6716,10.2409);
\draw [color=c, fill=c] (11.6716,10.135) rectangle (11.7114,10.2409);
\draw [color=c, fill=c] (11.7114,10.135) rectangle (11.7512,10.2409);
\draw [color=c, fill=c] (11.7512,10.135) rectangle (11.791,10.2409);
\draw [color=c, fill=c] (11.791,10.135) rectangle (11.8308,10.2409);
\draw [color=c, fill=c] (11.8308,10.135) rectangle (11.8706,10.2409);
\draw [color=c, fill=c] (11.8706,10.135) rectangle (11.9104,10.2409);
\draw [color=c, fill=c] (11.9104,10.135) rectangle (11.9502,10.2409);
\draw [color=c, fill=c] (11.9502,10.135) rectangle (11.99,10.2409);
\draw [color=c, fill=c] (11.99,10.135) rectangle (12.0299,10.2409);
\draw [color=c, fill=c] (12.0299,10.135) rectangle (12.0697,10.2409);
\draw [color=c, fill=c] (12.0697,10.135) rectangle (12.1095,10.2409);
\draw [color=c, fill=c] (12.1095,10.135) rectangle (12.1493,10.2409);
\draw [color=c, fill=c] (12.1493,10.135) rectangle (12.1891,10.2409);
\draw [color=c, fill=c] (12.1891,10.135) rectangle (12.2289,10.2409);
\draw [color=c, fill=c] (12.2289,10.135) rectangle (12.2687,10.2409);
\draw [color=c, fill=c] (12.2687,10.135) rectangle (12.3085,10.2409);
\draw [color=c, fill=c] (12.3085,10.135) rectangle (12.3483,10.2409);
\draw [color=c, fill=c] (12.3483,10.135) rectangle (12.3881,10.2409);
\draw [color=c, fill=c] (12.3881,10.135) rectangle (12.4279,10.2409);
\draw [color=c, fill=c] (12.4279,10.135) rectangle (12.4677,10.2409);
\draw [color=c, fill=c] (12.4677,10.135) rectangle (12.5075,10.2409);
\draw [color=c, fill=c] (12.5075,10.135) rectangle (12.5473,10.2409);
\draw [color=c, fill=c] (12.5473,10.135) rectangle (12.5871,10.2409);
\draw [color=c, fill=c] (12.5871,10.135) rectangle (12.6269,10.2409);
\draw [color=c, fill=c] (12.6269,10.135) rectangle (12.6667,10.2409);
\draw [color=c, fill=c] (12.6667,10.135) rectangle (12.7065,10.2409);
\draw [color=c, fill=c] (12.7065,10.135) rectangle (12.7463,10.2409);
\draw [color=c, fill=c] (12.7463,10.135) rectangle (12.7861,10.2409);
\draw [color=c, fill=c] (12.7861,10.135) rectangle (12.8259,10.2409);
\draw [color=c, fill=c] (12.8259,10.135) rectangle (12.8657,10.2409);
\draw [color=c, fill=c] (12.8657,10.135) rectangle (12.9055,10.2409);
\draw [color=c, fill=c] (12.9055,10.135) rectangle (12.9453,10.2409);
\draw [color=c, fill=c] (12.9453,10.135) rectangle (12.9851,10.2409);
\draw [color=c, fill=c] (12.9851,10.135) rectangle (13.0249,10.2409);
\draw [color=c, fill=c] (13.0249,10.135) rectangle (13.0647,10.2409);
\draw [color=c, fill=c] (13.0647,10.135) rectangle (13.1045,10.2409);
\draw [color=c, fill=c] (13.1045,10.135) rectangle (13.1443,10.2409);
\draw [color=c, fill=c] (13.1443,10.135) rectangle (13.1841,10.2409);
\draw [color=c, fill=c] (13.1841,10.135) rectangle (13.2239,10.2409);
\draw [color=c, fill=c] (13.2239,10.135) rectangle (13.2637,10.2409);
\draw [color=c, fill=c] (13.2637,10.135) rectangle (13.3035,10.2409);
\draw [color=c, fill=c] (13.3035,10.135) rectangle (13.3433,10.2409);
\draw [color=c, fill=c] (13.3433,10.135) rectangle (13.3831,10.2409);
\draw [color=c, fill=c] (13.3831,10.135) rectangle (13.4229,10.2409);
\draw [color=c, fill=c] (13.4229,10.135) rectangle (13.4627,10.2409);
\draw [color=c, fill=c] (13.4627,10.135) rectangle (13.5025,10.2409);
\draw [color=c, fill=c] (13.5025,10.135) rectangle (13.5423,10.2409);
\draw [color=c, fill=c] (13.5423,10.135) rectangle (13.5821,10.2409);
\draw [color=c, fill=c] (13.5821,10.135) rectangle (13.6219,10.2409);
\draw [color=c, fill=c] (13.6219,10.135) rectangle (13.6617,10.2409);
\draw [color=c, fill=c] (13.6617,10.135) rectangle (13.7015,10.2409);
\draw [color=c, fill=c] (13.7015,10.135) rectangle (13.7413,10.2409);
\draw [color=c, fill=c] (13.7413,10.135) rectangle (13.7811,10.2409);
\draw [color=c, fill=c] (13.7811,10.135) rectangle (13.8209,10.2409);
\draw [color=c, fill=c] (13.8209,10.135) rectangle (13.8607,10.2409);
\draw [color=c, fill=c] (13.8607,10.135) rectangle (13.9005,10.2409);
\draw [color=c, fill=c] (13.9005,10.135) rectangle (13.9403,10.2409);
\definecolor{c}{rgb}{0,0.733333,1};
\draw [color=c, fill=c] (13.9403,10.135) rectangle (13.9801,10.2409);
\draw [color=c, fill=c] (13.9801,10.135) rectangle (14.0199,10.2409);
\draw [color=c, fill=c] (14.0199,10.135) rectangle (14.0597,10.2409);
\draw [color=c, fill=c] (14.0597,10.135) rectangle (14.0995,10.2409);
\draw [color=c, fill=c] (14.0995,10.135) rectangle (14.1393,10.2409);
\draw [color=c, fill=c] (14.1393,10.135) rectangle (14.1791,10.2409);
\draw [color=c, fill=c] (14.1791,10.135) rectangle (14.2189,10.2409);
\draw [color=c, fill=c] (14.2189,10.135) rectangle (14.2587,10.2409);
\draw [color=c, fill=c] (14.2587,10.135) rectangle (14.2985,10.2409);
\draw [color=c, fill=c] (14.2985,10.135) rectangle (14.3383,10.2409);
\draw [color=c, fill=c] (14.3383,10.135) rectangle (14.3781,10.2409);
\draw [color=c, fill=c] (14.3781,10.135) rectangle (14.4179,10.2409);
\draw [color=c, fill=c] (14.4179,10.135) rectangle (14.4577,10.2409);
\draw [color=c, fill=c] (14.4577,10.135) rectangle (14.4975,10.2409);
\draw [color=c, fill=c] (14.4975,10.135) rectangle (14.5373,10.2409);
\draw [color=c, fill=c] (14.5373,10.135) rectangle (14.5771,10.2409);
\draw [color=c, fill=c] (14.5771,10.135) rectangle (14.6169,10.2409);
\draw [color=c, fill=c] (14.6169,10.135) rectangle (14.6567,10.2409);
\draw [color=c, fill=c] (14.6567,10.135) rectangle (14.6965,10.2409);
\draw [color=c, fill=c] (14.6965,10.135) rectangle (14.7363,10.2409);
\draw [color=c, fill=c] (14.7363,10.135) rectangle (14.7761,10.2409);
\draw [color=c, fill=c] (14.7761,10.135) rectangle (14.8159,10.2409);
\draw [color=c, fill=c] (14.8159,10.135) rectangle (14.8557,10.2409);
\draw [color=c, fill=c] (14.8557,10.135) rectangle (14.8955,10.2409);
\draw [color=c, fill=c] (14.8955,10.135) rectangle (14.9353,10.2409);
\draw [color=c, fill=c] (14.9353,10.135) rectangle (14.9751,10.2409);
\draw [color=c, fill=c] (14.9751,10.135) rectangle (15.0149,10.2409);
\draw [color=c, fill=c] (15.0149,10.135) rectangle (15.0547,10.2409);
\draw [color=c, fill=c] (15.0547,10.135) rectangle (15.0945,10.2409);
\draw [color=c, fill=c] (15.0945,10.135) rectangle (15.1343,10.2409);
\draw [color=c, fill=c] (15.1343,10.135) rectangle (15.1741,10.2409);
\draw [color=c, fill=c] (15.1741,10.135) rectangle (15.2139,10.2409);
\draw [color=c, fill=c] (15.2139,10.135) rectangle (15.2537,10.2409);
\draw [color=c, fill=c] (15.2537,10.135) rectangle (15.2935,10.2409);
\draw [color=c, fill=c] (15.2935,10.135) rectangle (15.3333,10.2409);
\draw [color=c, fill=c] (15.3333,10.135) rectangle (15.3731,10.2409);
\draw [color=c, fill=c] (15.3731,10.135) rectangle (15.4129,10.2409);
\draw [color=c, fill=c] (15.4129,10.135) rectangle (15.4527,10.2409);
\draw [color=c, fill=c] (15.4527,10.135) rectangle (15.4925,10.2409);
\draw [color=c, fill=c] (15.4925,10.135) rectangle (15.5323,10.2409);
\draw [color=c, fill=c] (15.5323,10.135) rectangle (15.5721,10.2409);
\draw [color=c, fill=c] (15.5721,10.135) rectangle (15.6119,10.2409);
\draw [color=c, fill=c] (15.6119,10.135) rectangle (15.6517,10.2409);
\draw [color=c, fill=c] (15.6517,10.135) rectangle (15.6915,10.2409);
\draw [color=c, fill=c] (15.6915,10.135) rectangle (15.7313,10.2409);
\draw [color=c, fill=c] (15.7313,10.135) rectangle (15.7711,10.2409);
\draw [color=c, fill=c] (15.7711,10.135) rectangle (15.8109,10.2409);
\draw [color=c, fill=c] (15.8109,10.135) rectangle (15.8507,10.2409);
\draw [color=c, fill=c] (15.8507,10.135) rectangle (15.8905,10.2409);
\draw [color=c, fill=c] (15.8905,10.135) rectangle (15.9303,10.2409);
\draw [color=c, fill=c] (15.9303,10.135) rectangle (15.9701,10.2409);
\draw [color=c, fill=c] (15.9701,10.135) rectangle (16.01,10.2409);
\draw [color=c, fill=c] (16.01,10.135) rectangle (16.0498,10.2409);
\draw [color=c, fill=c] (16.0498,10.135) rectangle (16.0896,10.2409);
\draw [color=c, fill=c] (16.0896,10.135) rectangle (16.1294,10.2409);
\draw [color=c, fill=c] (16.1294,10.135) rectangle (16.1692,10.2409);
\draw [color=c, fill=c] (16.1692,10.135) rectangle (16.209,10.2409);
\draw [color=c, fill=c] (16.209,10.135) rectangle (16.2488,10.2409);
\draw [color=c, fill=c] (16.2488,10.135) rectangle (16.2886,10.2409);
\draw [color=c, fill=c] (16.2886,10.135) rectangle (16.3284,10.2409);
\draw [color=c, fill=c] (16.3284,10.135) rectangle (16.3682,10.2409);
\draw [color=c, fill=c] (16.3682,10.135) rectangle (16.408,10.2409);
\draw [color=c, fill=c] (16.408,10.135) rectangle (16.4478,10.2409);
\draw [color=c, fill=c] (16.4478,10.135) rectangle (16.4876,10.2409);
\draw [color=c, fill=c] (16.4876,10.135) rectangle (16.5274,10.2409);
\draw [color=c, fill=c] (16.5274,10.135) rectangle (16.5672,10.2409);
\draw [color=c, fill=c] (16.5672,10.135) rectangle (16.607,10.2409);
\draw [color=c, fill=c] (16.607,10.135) rectangle (16.6468,10.2409);
\draw [color=c, fill=c] (16.6468,10.135) rectangle (16.6866,10.2409);
\draw [color=c, fill=c] (16.6866,10.135) rectangle (16.7264,10.2409);
\draw [color=c, fill=c] (16.7264,10.135) rectangle (16.7662,10.2409);
\draw [color=c, fill=c] (16.7662,10.135) rectangle (16.806,10.2409);
\draw [color=c, fill=c] (16.806,10.135) rectangle (16.8458,10.2409);
\draw [color=c, fill=c] (16.8458,10.135) rectangle (16.8856,10.2409);
\draw [color=c, fill=c] (16.8856,10.135) rectangle (16.9254,10.2409);
\draw [color=c, fill=c] (16.9254,10.135) rectangle (16.9652,10.2409);
\draw [color=c, fill=c] (16.9652,10.135) rectangle (17.005,10.2409);
\draw [color=c, fill=c] (17.005,10.135) rectangle (17.0448,10.2409);
\draw [color=c, fill=c] (17.0448,10.135) rectangle (17.0846,10.2409);
\draw [color=c, fill=c] (17.0846,10.135) rectangle (17.1244,10.2409);
\draw [color=c, fill=c] (17.1244,10.135) rectangle (17.1642,10.2409);
\draw [color=c, fill=c] (17.1642,10.135) rectangle (17.204,10.2409);
\draw [color=c, fill=c] (17.204,10.135) rectangle (17.2438,10.2409);
\draw [color=c, fill=c] (17.2438,10.135) rectangle (17.2836,10.2409);
\draw [color=c, fill=c] (17.2836,10.135) rectangle (17.3234,10.2409);
\draw [color=c, fill=c] (17.3234,10.135) rectangle (17.3632,10.2409);
\draw [color=c, fill=c] (17.3632,10.135) rectangle (17.403,10.2409);
\draw [color=c, fill=c] (17.403,10.135) rectangle (17.4428,10.2409);
\draw [color=c, fill=c] (17.4428,10.135) rectangle (17.4826,10.2409);
\draw [color=c, fill=c] (17.4826,10.135) rectangle (17.5224,10.2409);
\draw [color=c, fill=c] (17.5224,10.135) rectangle (17.5622,10.2409);
\draw [color=c, fill=c] (17.5622,10.135) rectangle (17.602,10.2409);
\draw [color=c, fill=c] (17.602,10.135) rectangle (17.6418,10.2409);
\draw [color=c, fill=c] (17.6418,10.135) rectangle (17.6816,10.2409);
\draw [color=c, fill=c] (17.6816,10.135) rectangle (17.7214,10.2409);
\draw [color=c, fill=c] (17.7214,10.135) rectangle (17.7612,10.2409);
\draw [color=c, fill=c] (17.7612,10.135) rectangle (17.801,10.2409);
\draw [color=c, fill=c] (17.801,10.135) rectangle (17.8408,10.2409);
\draw [color=c, fill=c] (17.8408,10.135) rectangle (17.8806,10.2409);
\draw [color=c, fill=c] (17.8806,10.135) rectangle (17.9204,10.2409);
\draw [color=c, fill=c] (17.9204,10.135) rectangle (17.9602,10.2409);
\draw [color=c, fill=c] (17.9602,10.135) rectangle (18,10.2409);
\definecolor{c}{rgb}{0.2,0,1};
\draw [color=c, fill=c] (2,10.2409) rectangle (2.0398,10.3467);
\draw [color=c, fill=c] (2.0398,10.2409) rectangle (2.0796,10.3467);
\draw [color=c, fill=c] (2.0796,10.2409) rectangle (2.1194,10.3467);
\draw [color=c, fill=c] (2.1194,10.2409) rectangle (2.1592,10.3467);
\draw [color=c, fill=c] (2.1592,10.2409) rectangle (2.19901,10.3467);
\draw [color=c, fill=c] (2.19901,10.2409) rectangle (2.23881,10.3467);
\draw [color=c, fill=c] (2.23881,10.2409) rectangle (2.27861,10.3467);
\draw [color=c, fill=c] (2.27861,10.2409) rectangle (2.31841,10.3467);
\draw [color=c, fill=c] (2.31841,10.2409) rectangle (2.35821,10.3467);
\draw [color=c, fill=c] (2.35821,10.2409) rectangle (2.39801,10.3467);
\draw [color=c, fill=c] (2.39801,10.2409) rectangle (2.43781,10.3467);
\draw [color=c, fill=c] (2.43781,10.2409) rectangle (2.47761,10.3467);
\draw [color=c, fill=c] (2.47761,10.2409) rectangle (2.51741,10.3467);
\draw [color=c, fill=c] (2.51741,10.2409) rectangle (2.55721,10.3467);
\draw [color=c, fill=c] (2.55721,10.2409) rectangle (2.59702,10.3467);
\draw [color=c, fill=c] (2.59702,10.2409) rectangle (2.63682,10.3467);
\draw [color=c, fill=c] (2.63682,10.2409) rectangle (2.67662,10.3467);
\draw [color=c, fill=c] (2.67662,10.2409) rectangle (2.71642,10.3467);
\draw [color=c, fill=c] (2.71642,10.2409) rectangle (2.75622,10.3467);
\draw [color=c, fill=c] (2.75622,10.2409) rectangle (2.79602,10.3467);
\draw [color=c, fill=c] (2.79602,10.2409) rectangle (2.83582,10.3467);
\draw [color=c, fill=c] (2.83582,10.2409) rectangle (2.87562,10.3467);
\draw [color=c, fill=c] (2.87562,10.2409) rectangle (2.91542,10.3467);
\draw [color=c, fill=c] (2.91542,10.2409) rectangle (2.95522,10.3467);
\draw [color=c, fill=c] (2.95522,10.2409) rectangle (2.99502,10.3467);
\draw [color=c, fill=c] (2.99502,10.2409) rectangle (3.03483,10.3467);
\draw [color=c, fill=c] (3.03483,10.2409) rectangle (3.07463,10.3467);
\draw [color=c, fill=c] (3.07463,10.2409) rectangle (3.11443,10.3467);
\draw [color=c, fill=c] (3.11443,10.2409) rectangle (3.15423,10.3467);
\draw [color=c, fill=c] (3.15423,10.2409) rectangle (3.19403,10.3467);
\draw [color=c, fill=c] (3.19403,10.2409) rectangle (3.23383,10.3467);
\draw [color=c, fill=c] (3.23383,10.2409) rectangle (3.27363,10.3467);
\draw [color=c, fill=c] (3.27363,10.2409) rectangle (3.31343,10.3467);
\draw [color=c, fill=c] (3.31343,10.2409) rectangle (3.35323,10.3467);
\draw [color=c, fill=c] (3.35323,10.2409) rectangle (3.39303,10.3467);
\draw [color=c, fill=c] (3.39303,10.2409) rectangle (3.43284,10.3467);
\draw [color=c, fill=c] (3.43284,10.2409) rectangle (3.47264,10.3467);
\draw [color=c, fill=c] (3.47264,10.2409) rectangle (3.51244,10.3467);
\draw [color=c, fill=c] (3.51244,10.2409) rectangle (3.55224,10.3467);
\draw [color=c, fill=c] (3.55224,10.2409) rectangle (3.59204,10.3467);
\draw [color=c, fill=c] (3.59204,10.2409) rectangle (3.63184,10.3467);
\draw [color=c, fill=c] (3.63184,10.2409) rectangle (3.67164,10.3467);
\draw [color=c, fill=c] (3.67164,10.2409) rectangle (3.71144,10.3467);
\draw [color=c, fill=c] (3.71144,10.2409) rectangle (3.75124,10.3467);
\draw [color=c, fill=c] (3.75124,10.2409) rectangle (3.79104,10.3467);
\draw [color=c, fill=c] (3.79104,10.2409) rectangle (3.83085,10.3467);
\draw [color=c, fill=c] (3.83085,10.2409) rectangle (3.87065,10.3467);
\draw [color=c, fill=c] (3.87065,10.2409) rectangle (3.91045,10.3467);
\draw [color=c, fill=c] (3.91045,10.2409) rectangle (3.95025,10.3467);
\draw [color=c, fill=c] (3.95025,10.2409) rectangle (3.99005,10.3467);
\draw [color=c, fill=c] (3.99005,10.2409) rectangle (4.02985,10.3467);
\draw [color=c, fill=c] (4.02985,10.2409) rectangle (4.06965,10.3467);
\draw [color=c, fill=c] (4.06965,10.2409) rectangle (4.10945,10.3467);
\draw [color=c, fill=c] (4.10945,10.2409) rectangle (4.14925,10.3467);
\draw [color=c, fill=c] (4.14925,10.2409) rectangle (4.18905,10.3467);
\draw [color=c, fill=c] (4.18905,10.2409) rectangle (4.22886,10.3467);
\draw [color=c, fill=c] (4.22886,10.2409) rectangle (4.26866,10.3467);
\draw [color=c, fill=c] (4.26866,10.2409) rectangle (4.30846,10.3467);
\draw [color=c, fill=c] (4.30846,10.2409) rectangle (4.34826,10.3467);
\draw [color=c, fill=c] (4.34826,10.2409) rectangle (4.38806,10.3467);
\draw [color=c, fill=c] (4.38806,10.2409) rectangle (4.42786,10.3467);
\draw [color=c, fill=c] (4.42786,10.2409) rectangle (4.46766,10.3467);
\draw [color=c, fill=c] (4.46766,10.2409) rectangle (4.50746,10.3467);
\draw [color=c, fill=c] (4.50746,10.2409) rectangle (4.54726,10.3467);
\draw [color=c, fill=c] (4.54726,10.2409) rectangle (4.58706,10.3467);
\draw [color=c, fill=c] (4.58706,10.2409) rectangle (4.62687,10.3467);
\draw [color=c, fill=c] (4.62687,10.2409) rectangle (4.66667,10.3467);
\draw [color=c, fill=c] (4.66667,10.2409) rectangle (4.70647,10.3467);
\draw [color=c, fill=c] (4.70647,10.2409) rectangle (4.74627,10.3467);
\draw [color=c, fill=c] (4.74627,10.2409) rectangle (4.78607,10.3467);
\draw [color=c, fill=c] (4.78607,10.2409) rectangle (4.82587,10.3467);
\draw [color=c, fill=c] (4.82587,10.2409) rectangle (4.86567,10.3467);
\draw [color=c, fill=c] (4.86567,10.2409) rectangle (4.90547,10.3467);
\draw [color=c, fill=c] (4.90547,10.2409) rectangle (4.94527,10.3467);
\draw [color=c, fill=c] (4.94527,10.2409) rectangle (4.98507,10.3467);
\draw [color=c, fill=c] (4.98507,10.2409) rectangle (5.02488,10.3467);
\draw [color=c, fill=c] (5.02488,10.2409) rectangle (5.06468,10.3467);
\draw [color=c, fill=c] (5.06468,10.2409) rectangle (5.10448,10.3467);
\draw [color=c, fill=c] (5.10448,10.2409) rectangle (5.14428,10.3467);
\draw [color=c, fill=c] (5.14428,10.2409) rectangle (5.18408,10.3467);
\draw [color=c, fill=c] (5.18408,10.2409) rectangle (5.22388,10.3467);
\draw [color=c, fill=c] (5.22388,10.2409) rectangle (5.26368,10.3467);
\draw [color=c, fill=c] (5.26368,10.2409) rectangle (5.30348,10.3467);
\draw [color=c, fill=c] (5.30348,10.2409) rectangle (5.34328,10.3467);
\draw [color=c, fill=c] (5.34328,10.2409) rectangle (5.38308,10.3467);
\draw [color=c, fill=c] (5.38308,10.2409) rectangle (5.42289,10.3467);
\draw [color=c, fill=c] (5.42289,10.2409) rectangle (5.46269,10.3467);
\draw [color=c, fill=c] (5.46269,10.2409) rectangle (5.50249,10.3467);
\draw [color=c, fill=c] (5.50249,10.2409) rectangle (5.54229,10.3467);
\draw [color=c, fill=c] (5.54229,10.2409) rectangle (5.58209,10.3467);
\draw [color=c, fill=c] (5.58209,10.2409) rectangle (5.62189,10.3467);
\draw [color=c, fill=c] (5.62189,10.2409) rectangle (5.66169,10.3467);
\draw [color=c, fill=c] (5.66169,10.2409) rectangle (5.70149,10.3467);
\draw [color=c, fill=c] (5.70149,10.2409) rectangle (5.74129,10.3467);
\draw [color=c, fill=c] (5.74129,10.2409) rectangle (5.78109,10.3467);
\draw [color=c, fill=c] (5.78109,10.2409) rectangle (5.8209,10.3467);
\draw [color=c, fill=c] (5.8209,10.2409) rectangle (5.8607,10.3467);
\draw [color=c, fill=c] (5.8607,10.2409) rectangle (5.9005,10.3467);
\draw [color=c, fill=c] (5.9005,10.2409) rectangle (5.9403,10.3467);
\draw [color=c, fill=c] (5.9403,10.2409) rectangle (5.9801,10.3467);
\draw [color=c, fill=c] (5.9801,10.2409) rectangle (6.0199,10.3467);
\draw [color=c, fill=c] (6.0199,10.2409) rectangle (6.0597,10.3467);
\draw [color=c, fill=c] (6.0597,10.2409) rectangle (6.0995,10.3467);
\draw [color=c, fill=c] (6.0995,10.2409) rectangle (6.1393,10.3467);
\draw [color=c, fill=c] (6.1393,10.2409) rectangle (6.1791,10.3467);
\draw [color=c, fill=c] (6.1791,10.2409) rectangle (6.21891,10.3467);
\draw [color=c, fill=c] (6.21891,10.2409) rectangle (6.25871,10.3467);
\draw [color=c, fill=c] (6.25871,10.2409) rectangle (6.29851,10.3467);
\draw [color=c, fill=c] (6.29851,10.2409) rectangle (6.33831,10.3467);
\draw [color=c, fill=c] (6.33831,10.2409) rectangle (6.37811,10.3467);
\draw [color=c, fill=c] (6.37811,10.2409) rectangle (6.41791,10.3467);
\draw [color=c, fill=c] (6.41791,10.2409) rectangle (6.45771,10.3467);
\draw [color=c, fill=c] (6.45771,10.2409) rectangle (6.49751,10.3467);
\draw [color=c, fill=c] (6.49751,10.2409) rectangle (6.53731,10.3467);
\draw [color=c, fill=c] (6.53731,10.2409) rectangle (6.57711,10.3467);
\draw [color=c, fill=c] (6.57711,10.2409) rectangle (6.61692,10.3467);
\draw [color=c, fill=c] (6.61692,10.2409) rectangle (6.65672,10.3467);
\draw [color=c, fill=c] (6.65672,10.2409) rectangle (6.69652,10.3467);
\draw [color=c, fill=c] (6.69652,10.2409) rectangle (6.73632,10.3467);
\draw [color=c, fill=c] (6.73632,10.2409) rectangle (6.77612,10.3467);
\draw [color=c, fill=c] (6.77612,10.2409) rectangle (6.81592,10.3467);
\draw [color=c, fill=c] (6.81592,10.2409) rectangle (6.85572,10.3467);
\draw [color=c, fill=c] (6.85572,10.2409) rectangle (6.89552,10.3467);
\draw [color=c, fill=c] (6.89552,10.2409) rectangle (6.93532,10.3467);
\draw [color=c, fill=c] (6.93532,10.2409) rectangle (6.97512,10.3467);
\draw [color=c, fill=c] (6.97512,10.2409) rectangle (7.01493,10.3467);
\draw [color=c, fill=c] (7.01493,10.2409) rectangle (7.05473,10.3467);
\draw [color=c, fill=c] (7.05473,10.2409) rectangle (7.09453,10.3467);
\draw [color=c, fill=c] (7.09453,10.2409) rectangle (7.13433,10.3467);
\draw [color=c, fill=c] (7.13433,10.2409) rectangle (7.17413,10.3467);
\draw [color=c, fill=c] (7.17413,10.2409) rectangle (7.21393,10.3467);
\draw [color=c, fill=c] (7.21393,10.2409) rectangle (7.25373,10.3467);
\draw [color=c, fill=c] (7.25373,10.2409) rectangle (7.29353,10.3467);
\draw [color=c, fill=c] (7.29353,10.2409) rectangle (7.33333,10.3467);
\draw [color=c, fill=c] (7.33333,10.2409) rectangle (7.37313,10.3467);
\draw [color=c, fill=c] (7.37313,10.2409) rectangle (7.41294,10.3467);
\draw [color=c, fill=c] (7.41294,10.2409) rectangle (7.45274,10.3467);
\draw [color=c, fill=c] (7.45274,10.2409) rectangle (7.49254,10.3467);
\draw [color=c, fill=c] (7.49254,10.2409) rectangle (7.53234,10.3467);
\draw [color=c, fill=c] (7.53234,10.2409) rectangle (7.57214,10.3467);
\draw [color=c, fill=c] (7.57214,10.2409) rectangle (7.61194,10.3467);
\draw [color=c, fill=c] (7.61194,10.2409) rectangle (7.65174,10.3467);
\draw [color=c, fill=c] (7.65174,10.2409) rectangle (7.69154,10.3467);
\draw [color=c, fill=c] (7.69154,10.2409) rectangle (7.73134,10.3467);
\draw [color=c, fill=c] (7.73134,10.2409) rectangle (7.77114,10.3467);
\definecolor{c}{rgb}{0,0.0800001,1};
\draw [color=c, fill=c] (7.77114,10.2409) rectangle (7.81095,10.3467);
\draw [color=c, fill=c] (7.81095,10.2409) rectangle (7.85075,10.3467);
\draw [color=c, fill=c] (7.85075,10.2409) rectangle (7.89055,10.3467);
\draw [color=c, fill=c] (7.89055,10.2409) rectangle (7.93035,10.3467);
\draw [color=c, fill=c] (7.93035,10.2409) rectangle (7.97015,10.3467);
\draw [color=c, fill=c] (7.97015,10.2409) rectangle (8.00995,10.3467);
\draw [color=c, fill=c] (8.00995,10.2409) rectangle (8.04975,10.3467);
\draw [color=c, fill=c] (8.04975,10.2409) rectangle (8.08955,10.3467);
\draw [color=c, fill=c] (8.08955,10.2409) rectangle (8.12935,10.3467);
\draw [color=c, fill=c] (8.12935,10.2409) rectangle (8.16915,10.3467);
\draw [color=c, fill=c] (8.16915,10.2409) rectangle (8.20895,10.3467);
\draw [color=c, fill=c] (8.20895,10.2409) rectangle (8.24876,10.3467);
\draw [color=c, fill=c] (8.24876,10.2409) rectangle (8.28856,10.3467);
\draw [color=c, fill=c] (8.28856,10.2409) rectangle (8.32836,10.3467);
\draw [color=c, fill=c] (8.32836,10.2409) rectangle (8.36816,10.3467);
\draw [color=c, fill=c] (8.36816,10.2409) rectangle (8.40796,10.3467);
\draw [color=c, fill=c] (8.40796,10.2409) rectangle (8.44776,10.3467);
\draw [color=c, fill=c] (8.44776,10.2409) rectangle (8.48756,10.3467);
\draw [color=c, fill=c] (8.48756,10.2409) rectangle (8.52736,10.3467);
\draw [color=c, fill=c] (8.52736,10.2409) rectangle (8.56716,10.3467);
\draw [color=c, fill=c] (8.56716,10.2409) rectangle (8.60697,10.3467);
\draw [color=c, fill=c] (8.60697,10.2409) rectangle (8.64677,10.3467);
\draw [color=c, fill=c] (8.64677,10.2409) rectangle (8.68657,10.3467);
\draw [color=c, fill=c] (8.68657,10.2409) rectangle (8.72637,10.3467);
\draw [color=c, fill=c] (8.72637,10.2409) rectangle (8.76617,10.3467);
\draw [color=c, fill=c] (8.76617,10.2409) rectangle (8.80597,10.3467);
\draw [color=c, fill=c] (8.80597,10.2409) rectangle (8.84577,10.3467);
\draw [color=c, fill=c] (8.84577,10.2409) rectangle (8.88557,10.3467);
\draw [color=c, fill=c] (8.88557,10.2409) rectangle (8.92537,10.3467);
\draw [color=c, fill=c] (8.92537,10.2409) rectangle (8.96517,10.3467);
\draw [color=c, fill=c] (8.96517,10.2409) rectangle (9.00498,10.3467);
\draw [color=c, fill=c] (9.00498,10.2409) rectangle (9.04478,10.3467);
\draw [color=c, fill=c] (9.04478,10.2409) rectangle (9.08458,10.3467);
\draw [color=c, fill=c] (9.08458,10.2409) rectangle (9.12438,10.3467);
\draw [color=c, fill=c] (9.12438,10.2409) rectangle (9.16418,10.3467);
\draw [color=c, fill=c] (9.16418,10.2409) rectangle (9.20398,10.3467);
\draw [color=c, fill=c] (9.20398,10.2409) rectangle (9.24378,10.3467);
\draw [color=c, fill=c] (9.24378,10.2409) rectangle (9.28358,10.3467);
\draw [color=c, fill=c] (9.28358,10.2409) rectangle (9.32338,10.3467);
\draw [color=c, fill=c] (9.32338,10.2409) rectangle (9.36318,10.3467);
\draw [color=c, fill=c] (9.36318,10.2409) rectangle (9.40298,10.3467);
\draw [color=c, fill=c] (9.40298,10.2409) rectangle (9.44279,10.3467);
\draw [color=c, fill=c] (9.44279,10.2409) rectangle (9.48259,10.3467);
\draw [color=c, fill=c] (9.48259,10.2409) rectangle (9.52239,10.3467);
\definecolor{c}{rgb}{0,0.266667,1};
\draw [color=c, fill=c] (9.52239,10.2409) rectangle (9.56219,10.3467);
\draw [color=c, fill=c] (9.56219,10.2409) rectangle (9.60199,10.3467);
\draw [color=c, fill=c] (9.60199,10.2409) rectangle (9.64179,10.3467);
\draw [color=c, fill=c] (9.64179,10.2409) rectangle (9.68159,10.3467);
\draw [color=c, fill=c] (9.68159,10.2409) rectangle (9.72139,10.3467);
\draw [color=c, fill=c] (9.72139,10.2409) rectangle (9.76119,10.3467);
\draw [color=c, fill=c] (9.76119,10.2409) rectangle (9.80099,10.3467);
\draw [color=c, fill=c] (9.80099,10.2409) rectangle (9.8408,10.3467);
\draw [color=c, fill=c] (9.8408,10.2409) rectangle (9.8806,10.3467);
\draw [color=c, fill=c] (9.8806,10.2409) rectangle (9.9204,10.3467);
\draw [color=c, fill=c] (9.9204,10.2409) rectangle (9.9602,10.3467);
\draw [color=c, fill=c] (9.9602,10.2409) rectangle (10,10.3467);
\draw [color=c, fill=c] (10,10.2409) rectangle (10.0398,10.3467);
\draw [color=c, fill=c] (10.0398,10.2409) rectangle (10.0796,10.3467);
\draw [color=c, fill=c] (10.0796,10.2409) rectangle (10.1194,10.3467);
\draw [color=c, fill=c] (10.1194,10.2409) rectangle (10.1592,10.3467);
\draw [color=c, fill=c] (10.1592,10.2409) rectangle (10.199,10.3467);
\draw [color=c, fill=c] (10.199,10.2409) rectangle (10.2388,10.3467);
\draw [color=c, fill=c] (10.2388,10.2409) rectangle (10.2786,10.3467);
\draw [color=c, fill=c] (10.2786,10.2409) rectangle (10.3184,10.3467);
\draw [color=c, fill=c] (10.3184,10.2409) rectangle (10.3582,10.3467);
\draw [color=c, fill=c] (10.3582,10.2409) rectangle (10.398,10.3467);
\draw [color=c, fill=c] (10.398,10.2409) rectangle (10.4378,10.3467);
\draw [color=c, fill=c] (10.4378,10.2409) rectangle (10.4776,10.3467);
\draw [color=c, fill=c] (10.4776,10.2409) rectangle (10.5174,10.3467);
\draw [color=c, fill=c] (10.5174,10.2409) rectangle (10.5572,10.3467);
\draw [color=c, fill=c] (10.5572,10.2409) rectangle (10.597,10.3467);
\draw [color=c, fill=c] (10.597,10.2409) rectangle (10.6368,10.3467);
\draw [color=c, fill=c] (10.6368,10.2409) rectangle (10.6766,10.3467);
\draw [color=c, fill=c] (10.6766,10.2409) rectangle (10.7164,10.3467);
\draw [color=c, fill=c] (10.7164,10.2409) rectangle (10.7562,10.3467);
\draw [color=c, fill=c] (10.7562,10.2409) rectangle (10.796,10.3467);
\definecolor{c}{rgb}{0,0.546666,1};
\draw [color=c, fill=c] (10.796,10.2409) rectangle (10.8358,10.3467);
\draw [color=c, fill=c] (10.8358,10.2409) rectangle (10.8756,10.3467);
\draw [color=c, fill=c] (10.8756,10.2409) rectangle (10.9154,10.3467);
\draw [color=c, fill=c] (10.9154,10.2409) rectangle (10.9552,10.3467);
\draw [color=c, fill=c] (10.9552,10.2409) rectangle (10.995,10.3467);
\draw [color=c, fill=c] (10.995,10.2409) rectangle (11.0348,10.3467);
\draw [color=c, fill=c] (11.0348,10.2409) rectangle (11.0746,10.3467);
\draw [color=c, fill=c] (11.0746,10.2409) rectangle (11.1144,10.3467);
\draw [color=c, fill=c] (11.1144,10.2409) rectangle (11.1542,10.3467);
\draw [color=c, fill=c] (11.1542,10.2409) rectangle (11.194,10.3467);
\draw [color=c, fill=c] (11.194,10.2409) rectangle (11.2338,10.3467);
\draw [color=c, fill=c] (11.2338,10.2409) rectangle (11.2736,10.3467);
\draw [color=c, fill=c] (11.2736,10.2409) rectangle (11.3134,10.3467);
\draw [color=c, fill=c] (11.3134,10.2409) rectangle (11.3532,10.3467);
\draw [color=c, fill=c] (11.3532,10.2409) rectangle (11.393,10.3467);
\draw [color=c, fill=c] (11.393,10.2409) rectangle (11.4328,10.3467);
\draw [color=c, fill=c] (11.4328,10.2409) rectangle (11.4726,10.3467);
\draw [color=c, fill=c] (11.4726,10.2409) rectangle (11.5124,10.3467);
\draw [color=c, fill=c] (11.5124,10.2409) rectangle (11.5522,10.3467);
\draw [color=c, fill=c] (11.5522,10.2409) rectangle (11.592,10.3467);
\draw [color=c, fill=c] (11.592,10.2409) rectangle (11.6318,10.3467);
\draw [color=c, fill=c] (11.6318,10.2409) rectangle (11.6716,10.3467);
\draw [color=c, fill=c] (11.6716,10.2409) rectangle (11.7114,10.3467);
\draw [color=c, fill=c] (11.7114,10.2409) rectangle (11.7512,10.3467);
\draw [color=c, fill=c] (11.7512,10.2409) rectangle (11.791,10.3467);
\draw [color=c, fill=c] (11.791,10.2409) rectangle (11.8308,10.3467);
\draw [color=c, fill=c] (11.8308,10.2409) rectangle (11.8706,10.3467);
\draw [color=c, fill=c] (11.8706,10.2409) rectangle (11.9104,10.3467);
\draw [color=c, fill=c] (11.9104,10.2409) rectangle (11.9502,10.3467);
\draw [color=c, fill=c] (11.9502,10.2409) rectangle (11.99,10.3467);
\draw [color=c, fill=c] (11.99,10.2409) rectangle (12.0299,10.3467);
\draw [color=c, fill=c] (12.0299,10.2409) rectangle (12.0697,10.3467);
\draw [color=c, fill=c] (12.0697,10.2409) rectangle (12.1095,10.3467);
\draw [color=c, fill=c] (12.1095,10.2409) rectangle (12.1493,10.3467);
\draw [color=c, fill=c] (12.1493,10.2409) rectangle (12.1891,10.3467);
\draw [color=c, fill=c] (12.1891,10.2409) rectangle (12.2289,10.3467);
\draw [color=c, fill=c] (12.2289,10.2409) rectangle (12.2687,10.3467);
\draw [color=c, fill=c] (12.2687,10.2409) rectangle (12.3085,10.3467);
\draw [color=c, fill=c] (12.3085,10.2409) rectangle (12.3483,10.3467);
\draw [color=c, fill=c] (12.3483,10.2409) rectangle (12.3881,10.3467);
\draw [color=c, fill=c] (12.3881,10.2409) rectangle (12.4279,10.3467);
\draw [color=c, fill=c] (12.4279,10.2409) rectangle (12.4677,10.3467);
\draw [color=c, fill=c] (12.4677,10.2409) rectangle (12.5075,10.3467);
\draw [color=c, fill=c] (12.5075,10.2409) rectangle (12.5473,10.3467);
\draw [color=c, fill=c] (12.5473,10.2409) rectangle (12.5871,10.3467);
\draw [color=c, fill=c] (12.5871,10.2409) rectangle (12.6269,10.3467);
\draw [color=c, fill=c] (12.6269,10.2409) rectangle (12.6667,10.3467);
\draw [color=c, fill=c] (12.6667,10.2409) rectangle (12.7065,10.3467);
\draw [color=c, fill=c] (12.7065,10.2409) rectangle (12.7463,10.3467);
\draw [color=c, fill=c] (12.7463,10.2409) rectangle (12.7861,10.3467);
\draw [color=c, fill=c] (12.7861,10.2409) rectangle (12.8259,10.3467);
\draw [color=c, fill=c] (12.8259,10.2409) rectangle (12.8657,10.3467);
\draw [color=c, fill=c] (12.8657,10.2409) rectangle (12.9055,10.3467);
\draw [color=c, fill=c] (12.9055,10.2409) rectangle (12.9453,10.3467);
\draw [color=c, fill=c] (12.9453,10.2409) rectangle (12.9851,10.3467);
\draw [color=c, fill=c] (12.9851,10.2409) rectangle (13.0249,10.3467);
\draw [color=c, fill=c] (13.0249,10.2409) rectangle (13.0647,10.3467);
\draw [color=c, fill=c] (13.0647,10.2409) rectangle (13.1045,10.3467);
\draw [color=c, fill=c] (13.1045,10.2409) rectangle (13.1443,10.3467);
\draw [color=c, fill=c] (13.1443,10.2409) rectangle (13.1841,10.3467);
\draw [color=c, fill=c] (13.1841,10.2409) rectangle (13.2239,10.3467);
\draw [color=c, fill=c] (13.2239,10.2409) rectangle (13.2637,10.3467);
\draw [color=c, fill=c] (13.2637,10.2409) rectangle (13.3035,10.3467);
\draw [color=c, fill=c] (13.3035,10.2409) rectangle (13.3433,10.3467);
\draw [color=c, fill=c] (13.3433,10.2409) rectangle (13.3831,10.3467);
\draw [color=c, fill=c] (13.3831,10.2409) rectangle (13.4229,10.3467);
\draw [color=c, fill=c] (13.4229,10.2409) rectangle (13.4627,10.3467);
\draw [color=c, fill=c] (13.4627,10.2409) rectangle (13.5025,10.3467);
\draw [color=c, fill=c] (13.5025,10.2409) rectangle (13.5423,10.3467);
\draw [color=c, fill=c] (13.5423,10.2409) rectangle (13.5821,10.3467);
\draw [color=c, fill=c] (13.5821,10.2409) rectangle (13.6219,10.3467);
\draw [color=c, fill=c] (13.6219,10.2409) rectangle (13.6617,10.3467);
\draw [color=c, fill=c] (13.6617,10.2409) rectangle (13.7015,10.3467);
\draw [color=c, fill=c] (13.7015,10.2409) rectangle (13.7413,10.3467);
\draw [color=c, fill=c] (13.7413,10.2409) rectangle (13.7811,10.3467);
\draw [color=c, fill=c] (13.7811,10.2409) rectangle (13.8209,10.3467);
\draw [color=c, fill=c] (13.8209,10.2409) rectangle (13.8607,10.3467);
\draw [color=c, fill=c] (13.8607,10.2409) rectangle (13.9005,10.3467);
\draw [color=c, fill=c] (13.9005,10.2409) rectangle (13.9403,10.3467);
\draw [color=c, fill=c] (13.9403,10.2409) rectangle (13.9801,10.3467);
\definecolor{c}{rgb}{0,0.733333,1};
\draw [color=c, fill=c] (13.9801,10.2409) rectangle (14.0199,10.3467);
\draw [color=c, fill=c] (14.0199,10.2409) rectangle (14.0597,10.3467);
\draw [color=c, fill=c] (14.0597,10.2409) rectangle (14.0995,10.3467);
\draw [color=c, fill=c] (14.0995,10.2409) rectangle (14.1393,10.3467);
\draw [color=c, fill=c] (14.1393,10.2409) rectangle (14.1791,10.3467);
\draw [color=c, fill=c] (14.1791,10.2409) rectangle (14.2189,10.3467);
\draw [color=c, fill=c] (14.2189,10.2409) rectangle (14.2587,10.3467);
\draw [color=c, fill=c] (14.2587,10.2409) rectangle (14.2985,10.3467);
\draw [color=c, fill=c] (14.2985,10.2409) rectangle (14.3383,10.3467);
\draw [color=c, fill=c] (14.3383,10.2409) rectangle (14.3781,10.3467);
\draw [color=c, fill=c] (14.3781,10.2409) rectangle (14.4179,10.3467);
\draw [color=c, fill=c] (14.4179,10.2409) rectangle (14.4577,10.3467);
\draw [color=c, fill=c] (14.4577,10.2409) rectangle (14.4975,10.3467);
\draw [color=c, fill=c] (14.4975,10.2409) rectangle (14.5373,10.3467);
\draw [color=c, fill=c] (14.5373,10.2409) rectangle (14.5771,10.3467);
\draw [color=c, fill=c] (14.5771,10.2409) rectangle (14.6169,10.3467);
\draw [color=c, fill=c] (14.6169,10.2409) rectangle (14.6567,10.3467);
\draw [color=c, fill=c] (14.6567,10.2409) rectangle (14.6965,10.3467);
\draw [color=c, fill=c] (14.6965,10.2409) rectangle (14.7363,10.3467);
\draw [color=c, fill=c] (14.7363,10.2409) rectangle (14.7761,10.3467);
\draw [color=c, fill=c] (14.7761,10.2409) rectangle (14.8159,10.3467);
\draw [color=c, fill=c] (14.8159,10.2409) rectangle (14.8557,10.3467);
\draw [color=c, fill=c] (14.8557,10.2409) rectangle (14.8955,10.3467);
\draw [color=c, fill=c] (14.8955,10.2409) rectangle (14.9353,10.3467);
\draw [color=c, fill=c] (14.9353,10.2409) rectangle (14.9751,10.3467);
\draw [color=c, fill=c] (14.9751,10.2409) rectangle (15.0149,10.3467);
\draw [color=c, fill=c] (15.0149,10.2409) rectangle (15.0547,10.3467);
\draw [color=c, fill=c] (15.0547,10.2409) rectangle (15.0945,10.3467);
\draw [color=c, fill=c] (15.0945,10.2409) rectangle (15.1343,10.3467);
\draw [color=c, fill=c] (15.1343,10.2409) rectangle (15.1741,10.3467);
\draw [color=c, fill=c] (15.1741,10.2409) rectangle (15.2139,10.3467);
\draw [color=c, fill=c] (15.2139,10.2409) rectangle (15.2537,10.3467);
\draw [color=c, fill=c] (15.2537,10.2409) rectangle (15.2935,10.3467);
\draw [color=c, fill=c] (15.2935,10.2409) rectangle (15.3333,10.3467);
\draw [color=c, fill=c] (15.3333,10.2409) rectangle (15.3731,10.3467);
\draw [color=c, fill=c] (15.3731,10.2409) rectangle (15.4129,10.3467);
\draw [color=c, fill=c] (15.4129,10.2409) rectangle (15.4527,10.3467);
\draw [color=c, fill=c] (15.4527,10.2409) rectangle (15.4925,10.3467);
\draw [color=c, fill=c] (15.4925,10.2409) rectangle (15.5323,10.3467);
\draw [color=c, fill=c] (15.5323,10.2409) rectangle (15.5721,10.3467);
\draw [color=c, fill=c] (15.5721,10.2409) rectangle (15.6119,10.3467);
\draw [color=c, fill=c] (15.6119,10.2409) rectangle (15.6517,10.3467);
\draw [color=c, fill=c] (15.6517,10.2409) rectangle (15.6915,10.3467);
\draw [color=c, fill=c] (15.6915,10.2409) rectangle (15.7313,10.3467);
\draw [color=c, fill=c] (15.7313,10.2409) rectangle (15.7711,10.3467);
\draw [color=c, fill=c] (15.7711,10.2409) rectangle (15.8109,10.3467);
\draw [color=c, fill=c] (15.8109,10.2409) rectangle (15.8507,10.3467);
\draw [color=c, fill=c] (15.8507,10.2409) rectangle (15.8905,10.3467);
\draw [color=c, fill=c] (15.8905,10.2409) rectangle (15.9303,10.3467);
\draw [color=c, fill=c] (15.9303,10.2409) rectangle (15.9701,10.3467);
\draw [color=c, fill=c] (15.9701,10.2409) rectangle (16.01,10.3467);
\draw [color=c, fill=c] (16.01,10.2409) rectangle (16.0498,10.3467);
\draw [color=c, fill=c] (16.0498,10.2409) rectangle (16.0896,10.3467);
\draw [color=c, fill=c] (16.0896,10.2409) rectangle (16.1294,10.3467);
\draw [color=c, fill=c] (16.1294,10.2409) rectangle (16.1692,10.3467);
\draw [color=c, fill=c] (16.1692,10.2409) rectangle (16.209,10.3467);
\draw [color=c, fill=c] (16.209,10.2409) rectangle (16.2488,10.3467);
\draw [color=c, fill=c] (16.2488,10.2409) rectangle (16.2886,10.3467);
\draw [color=c, fill=c] (16.2886,10.2409) rectangle (16.3284,10.3467);
\draw [color=c, fill=c] (16.3284,10.2409) rectangle (16.3682,10.3467);
\draw [color=c, fill=c] (16.3682,10.2409) rectangle (16.408,10.3467);
\draw [color=c, fill=c] (16.408,10.2409) rectangle (16.4478,10.3467);
\draw [color=c, fill=c] (16.4478,10.2409) rectangle (16.4876,10.3467);
\draw [color=c, fill=c] (16.4876,10.2409) rectangle (16.5274,10.3467);
\draw [color=c, fill=c] (16.5274,10.2409) rectangle (16.5672,10.3467);
\draw [color=c, fill=c] (16.5672,10.2409) rectangle (16.607,10.3467);
\draw [color=c, fill=c] (16.607,10.2409) rectangle (16.6468,10.3467);
\draw [color=c, fill=c] (16.6468,10.2409) rectangle (16.6866,10.3467);
\draw [color=c, fill=c] (16.6866,10.2409) rectangle (16.7264,10.3467);
\draw [color=c, fill=c] (16.7264,10.2409) rectangle (16.7662,10.3467);
\draw [color=c, fill=c] (16.7662,10.2409) rectangle (16.806,10.3467);
\draw [color=c, fill=c] (16.806,10.2409) rectangle (16.8458,10.3467);
\draw [color=c, fill=c] (16.8458,10.2409) rectangle (16.8856,10.3467);
\draw [color=c, fill=c] (16.8856,10.2409) rectangle (16.9254,10.3467);
\draw [color=c, fill=c] (16.9254,10.2409) rectangle (16.9652,10.3467);
\draw [color=c, fill=c] (16.9652,10.2409) rectangle (17.005,10.3467);
\draw [color=c, fill=c] (17.005,10.2409) rectangle (17.0448,10.3467);
\draw [color=c, fill=c] (17.0448,10.2409) rectangle (17.0846,10.3467);
\draw [color=c, fill=c] (17.0846,10.2409) rectangle (17.1244,10.3467);
\draw [color=c, fill=c] (17.1244,10.2409) rectangle (17.1642,10.3467);
\draw [color=c, fill=c] (17.1642,10.2409) rectangle (17.204,10.3467);
\draw [color=c, fill=c] (17.204,10.2409) rectangle (17.2438,10.3467);
\draw [color=c, fill=c] (17.2438,10.2409) rectangle (17.2836,10.3467);
\draw [color=c, fill=c] (17.2836,10.2409) rectangle (17.3234,10.3467);
\draw [color=c, fill=c] (17.3234,10.2409) rectangle (17.3632,10.3467);
\draw [color=c, fill=c] (17.3632,10.2409) rectangle (17.403,10.3467);
\draw [color=c, fill=c] (17.403,10.2409) rectangle (17.4428,10.3467);
\draw [color=c, fill=c] (17.4428,10.2409) rectangle (17.4826,10.3467);
\draw [color=c, fill=c] (17.4826,10.2409) rectangle (17.5224,10.3467);
\draw [color=c, fill=c] (17.5224,10.2409) rectangle (17.5622,10.3467);
\draw [color=c, fill=c] (17.5622,10.2409) rectangle (17.602,10.3467);
\draw [color=c, fill=c] (17.602,10.2409) rectangle (17.6418,10.3467);
\draw [color=c, fill=c] (17.6418,10.2409) rectangle (17.6816,10.3467);
\draw [color=c, fill=c] (17.6816,10.2409) rectangle (17.7214,10.3467);
\draw [color=c, fill=c] (17.7214,10.2409) rectangle (17.7612,10.3467);
\draw [color=c, fill=c] (17.7612,10.2409) rectangle (17.801,10.3467);
\draw [color=c, fill=c] (17.801,10.2409) rectangle (17.8408,10.3467);
\draw [color=c, fill=c] (17.8408,10.2409) rectangle (17.8806,10.3467);
\draw [color=c, fill=c] (17.8806,10.2409) rectangle (17.9204,10.3467);
\draw [color=c, fill=c] (17.9204,10.2409) rectangle (17.9602,10.3467);
\draw [color=c, fill=c] (17.9602,10.2409) rectangle (18,10.3467);
\definecolor{c}{rgb}{0.2,0,1};
\draw [color=c, fill=c] (2,10.3467) rectangle (2.0398,10.4526);
\draw [color=c, fill=c] (2.0398,10.3467) rectangle (2.0796,10.4526);
\draw [color=c, fill=c] (2.0796,10.3467) rectangle (2.1194,10.4526);
\draw [color=c, fill=c] (2.1194,10.3467) rectangle (2.1592,10.4526);
\draw [color=c, fill=c] (2.1592,10.3467) rectangle (2.19901,10.4526);
\draw [color=c, fill=c] (2.19901,10.3467) rectangle (2.23881,10.4526);
\draw [color=c, fill=c] (2.23881,10.3467) rectangle (2.27861,10.4526);
\draw [color=c, fill=c] (2.27861,10.3467) rectangle (2.31841,10.4526);
\draw [color=c, fill=c] (2.31841,10.3467) rectangle (2.35821,10.4526);
\draw [color=c, fill=c] (2.35821,10.3467) rectangle (2.39801,10.4526);
\draw [color=c, fill=c] (2.39801,10.3467) rectangle (2.43781,10.4526);
\draw [color=c, fill=c] (2.43781,10.3467) rectangle (2.47761,10.4526);
\draw [color=c, fill=c] (2.47761,10.3467) rectangle (2.51741,10.4526);
\draw [color=c, fill=c] (2.51741,10.3467) rectangle (2.55721,10.4526);
\draw [color=c, fill=c] (2.55721,10.3467) rectangle (2.59702,10.4526);
\draw [color=c, fill=c] (2.59702,10.3467) rectangle (2.63682,10.4526);
\draw [color=c, fill=c] (2.63682,10.3467) rectangle (2.67662,10.4526);
\draw [color=c, fill=c] (2.67662,10.3467) rectangle (2.71642,10.4526);
\draw [color=c, fill=c] (2.71642,10.3467) rectangle (2.75622,10.4526);
\draw [color=c, fill=c] (2.75622,10.3467) rectangle (2.79602,10.4526);
\draw [color=c, fill=c] (2.79602,10.3467) rectangle (2.83582,10.4526);
\draw [color=c, fill=c] (2.83582,10.3467) rectangle (2.87562,10.4526);
\draw [color=c, fill=c] (2.87562,10.3467) rectangle (2.91542,10.4526);
\draw [color=c, fill=c] (2.91542,10.3467) rectangle (2.95522,10.4526);
\draw [color=c, fill=c] (2.95522,10.3467) rectangle (2.99502,10.4526);
\draw [color=c, fill=c] (2.99502,10.3467) rectangle (3.03483,10.4526);
\draw [color=c, fill=c] (3.03483,10.3467) rectangle (3.07463,10.4526);
\draw [color=c, fill=c] (3.07463,10.3467) rectangle (3.11443,10.4526);
\draw [color=c, fill=c] (3.11443,10.3467) rectangle (3.15423,10.4526);
\draw [color=c, fill=c] (3.15423,10.3467) rectangle (3.19403,10.4526);
\draw [color=c, fill=c] (3.19403,10.3467) rectangle (3.23383,10.4526);
\draw [color=c, fill=c] (3.23383,10.3467) rectangle (3.27363,10.4526);
\draw [color=c, fill=c] (3.27363,10.3467) rectangle (3.31343,10.4526);
\draw [color=c, fill=c] (3.31343,10.3467) rectangle (3.35323,10.4526);
\draw [color=c, fill=c] (3.35323,10.3467) rectangle (3.39303,10.4526);
\draw [color=c, fill=c] (3.39303,10.3467) rectangle (3.43284,10.4526);
\draw [color=c, fill=c] (3.43284,10.3467) rectangle (3.47264,10.4526);
\draw [color=c, fill=c] (3.47264,10.3467) rectangle (3.51244,10.4526);
\draw [color=c, fill=c] (3.51244,10.3467) rectangle (3.55224,10.4526);
\draw [color=c, fill=c] (3.55224,10.3467) rectangle (3.59204,10.4526);
\draw [color=c, fill=c] (3.59204,10.3467) rectangle (3.63184,10.4526);
\draw [color=c, fill=c] (3.63184,10.3467) rectangle (3.67164,10.4526);
\draw [color=c, fill=c] (3.67164,10.3467) rectangle (3.71144,10.4526);
\draw [color=c, fill=c] (3.71144,10.3467) rectangle (3.75124,10.4526);
\draw [color=c, fill=c] (3.75124,10.3467) rectangle (3.79104,10.4526);
\draw [color=c, fill=c] (3.79104,10.3467) rectangle (3.83085,10.4526);
\draw [color=c, fill=c] (3.83085,10.3467) rectangle (3.87065,10.4526);
\draw [color=c, fill=c] (3.87065,10.3467) rectangle (3.91045,10.4526);
\draw [color=c, fill=c] (3.91045,10.3467) rectangle (3.95025,10.4526);
\draw [color=c, fill=c] (3.95025,10.3467) rectangle (3.99005,10.4526);
\draw [color=c, fill=c] (3.99005,10.3467) rectangle (4.02985,10.4526);
\draw [color=c, fill=c] (4.02985,10.3467) rectangle (4.06965,10.4526);
\draw [color=c, fill=c] (4.06965,10.3467) rectangle (4.10945,10.4526);
\draw [color=c, fill=c] (4.10945,10.3467) rectangle (4.14925,10.4526);
\draw [color=c, fill=c] (4.14925,10.3467) rectangle (4.18905,10.4526);
\draw [color=c, fill=c] (4.18905,10.3467) rectangle (4.22886,10.4526);
\draw [color=c, fill=c] (4.22886,10.3467) rectangle (4.26866,10.4526);
\draw [color=c, fill=c] (4.26866,10.3467) rectangle (4.30846,10.4526);
\draw [color=c, fill=c] (4.30846,10.3467) rectangle (4.34826,10.4526);
\draw [color=c, fill=c] (4.34826,10.3467) rectangle (4.38806,10.4526);
\draw [color=c, fill=c] (4.38806,10.3467) rectangle (4.42786,10.4526);
\draw [color=c, fill=c] (4.42786,10.3467) rectangle (4.46766,10.4526);
\draw [color=c, fill=c] (4.46766,10.3467) rectangle (4.50746,10.4526);
\draw [color=c, fill=c] (4.50746,10.3467) rectangle (4.54726,10.4526);
\draw [color=c, fill=c] (4.54726,10.3467) rectangle (4.58706,10.4526);
\draw [color=c, fill=c] (4.58706,10.3467) rectangle (4.62687,10.4526);
\draw [color=c, fill=c] (4.62687,10.3467) rectangle (4.66667,10.4526);
\draw [color=c, fill=c] (4.66667,10.3467) rectangle (4.70647,10.4526);
\draw [color=c, fill=c] (4.70647,10.3467) rectangle (4.74627,10.4526);
\draw [color=c, fill=c] (4.74627,10.3467) rectangle (4.78607,10.4526);
\draw [color=c, fill=c] (4.78607,10.3467) rectangle (4.82587,10.4526);
\draw [color=c, fill=c] (4.82587,10.3467) rectangle (4.86567,10.4526);
\draw [color=c, fill=c] (4.86567,10.3467) rectangle (4.90547,10.4526);
\draw [color=c, fill=c] (4.90547,10.3467) rectangle (4.94527,10.4526);
\draw [color=c, fill=c] (4.94527,10.3467) rectangle (4.98507,10.4526);
\draw [color=c, fill=c] (4.98507,10.3467) rectangle (5.02488,10.4526);
\draw [color=c, fill=c] (5.02488,10.3467) rectangle (5.06468,10.4526);
\draw [color=c, fill=c] (5.06468,10.3467) rectangle (5.10448,10.4526);
\draw [color=c, fill=c] (5.10448,10.3467) rectangle (5.14428,10.4526);
\draw [color=c, fill=c] (5.14428,10.3467) rectangle (5.18408,10.4526);
\draw [color=c, fill=c] (5.18408,10.3467) rectangle (5.22388,10.4526);
\draw [color=c, fill=c] (5.22388,10.3467) rectangle (5.26368,10.4526);
\draw [color=c, fill=c] (5.26368,10.3467) rectangle (5.30348,10.4526);
\draw [color=c, fill=c] (5.30348,10.3467) rectangle (5.34328,10.4526);
\draw [color=c, fill=c] (5.34328,10.3467) rectangle (5.38308,10.4526);
\draw [color=c, fill=c] (5.38308,10.3467) rectangle (5.42289,10.4526);
\draw [color=c, fill=c] (5.42289,10.3467) rectangle (5.46269,10.4526);
\draw [color=c, fill=c] (5.46269,10.3467) rectangle (5.50249,10.4526);
\draw [color=c, fill=c] (5.50249,10.3467) rectangle (5.54229,10.4526);
\draw [color=c, fill=c] (5.54229,10.3467) rectangle (5.58209,10.4526);
\draw [color=c, fill=c] (5.58209,10.3467) rectangle (5.62189,10.4526);
\draw [color=c, fill=c] (5.62189,10.3467) rectangle (5.66169,10.4526);
\draw [color=c, fill=c] (5.66169,10.3467) rectangle (5.70149,10.4526);
\draw [color=c, fill=c] (5.70149,10.3467) rectangle (5.74129,10.4526);
\draw [color=c, fill=c] (5.74129,10.3467) rectangle (5.78109,10.4526);
\draw [color=c, fill=c] (5.78109,10.3467) rectangle (5.8209,10.4526);
\draw [color=c, fill=c] (5.8209,10.3467) rectangle (5.8607,10.4526);
\draw [color=c, fill=c] (5.8607,10.3467) rectangle (5.9005,10.4526);
\draw [color=c, fill=c] (5.9005,10.3467) rectangle (5.9403,10.4526);
\draw [color=c, fill=c] (5.9403,10.3467) rectangle (5.9801,10.4526);
\draw [color=c, fill=c] (5.9801,10.3467) rectangle (6.0199,10.4526);
\draw [color=c, fill=c] (6.0199,10.3467) rectangle (6.0597,10.4526);
\draw [color=c, fill=c] (6.0597,10.3467) rectangle (6.0995,10.4526);
\draw [color=c, fill=c] (6.0995,10.3467) rectangle (6.1393,10.4526);
\draw [color=c, fill=c] (6.1393,10.3467) rectangle (6.1791,10.4526);
\draw [color=c, fill=c] (6.1791,10.3467) rectangle (6.21891,10.4526);
\draw [color=c, fill=c] (6.21891,10.3467) rectangle (6.25871,10.4526);
\draw [color=c, fill=c] (6.25871,10.3467) rectangle (6.29851,10.4526);
\draw [color=c, fill=c] (6.29851,10.3467) rectangle (6.33831,10.4526);
\draw [color=c, fill=c] (6.33831,10.3467) rectangle (6.37811,10.4526);
\draw [color=c, fill=c] (6.37811,10.3467) rectangle (6.41791,10.4526);
\draw [color=c, fill=c] (6.41791,10.3467) rectangle (6.45771,10.4526);
\draw [color=c, fill=c] (6.45771,10.3467) rectangle (6.49751,10.4526);
\draw [color=c, fill=c] (6.49751,10.3467) rectangle (6.53731,10.4526);
\draw [color=c, fill=c] (6.53731,10.3467) rectangle (6.57711,10.4526);
\draw [color=c, fill=c] (6.57711,10.3467) rectangle (6.61692,10.4526);
\draw [color=c, fill=c] (6.61692,10.3467) rectangle (6.65672,10.4526);
\draw [color=c, fill=c] (6.65672,10.3467) rectangle (6.69652,10.4526);
\draw [color=c, fill=c] (6.69652,10.3467) rectangle (6.73632,10.4526);
\draw [color=c, fill=c] (6.73632,10.3467) rectangle (6.77612,10.4526);
\draw [color=c, fill=c] (6.77612,10.3467) rectangle (6.81592,10.4526);
\draw [color=c, fill=c] (6.81592,10.3467) rectangle (6.85572,10.4526);
\draw [color=c, fill=c] (6.85572,10.3467) rectangle (6.89552,10.4526);
\draw [color=c, fill=c] (6.89552,10.3467) rectangle (6.93532,10.4526);
\draw [color=c, fill=c] (6.93532,10.3467) rectangle (6.97512,10.4526);
\draw [color=c, fill=c] (6.97512,10.3467) rectangle (7.01493,10.4526);
\draw [color=c, fill=c] (7.01493,10.3467) rectangle (7.05473,10.4526);
\draw [color=c, fill=c] (7.05473,10.3467) rectangle (7.09453,10.4526);
\draw [color=c, fill=c] (7.09453,10.3467) rectangle (7.13433,10.4526);
\draw [color=c, fill=c] (7.13433,10.3467) rectangle (7.17413,10.4526);
\draw [color=c, fill=c] (7.17413,10.3467) rectangle (7.21393,10.4526);
\draw [color=c, fill=c] (7.21393,10.3467) rectangle (7.25373,10.4526);
\draw [color=c, fill=c] (7.25373,10.3467) rectangle (7.29353,10.4526);
\draw [color=c, fill=c] (7.29353,10.3467) rectangle (7.33333,10.4526);
\draw [color=c, fill=c] (7.33333,10.3467) rectangle (7.37313,10.4526);
\draw [color=c, fill=c] (7.37313,10.3467) rectangle (7.41294,10.4526);
\draw [color=c, fill=c] (7.41294,10.3467) rectangle (7.45274,10.4526);
\draw [color=c, fill=c] (7.45274,10.3467) rectangle (7.49254,10.4526);
\draw [color=c, fill=c] (7.49254,10.3467) rectangle (7.53234,10.4526);
\draw [color=c, fill=c] (7.53234,10.3467) rectangle (7.57214,10.4526);
\draw [color=c, fill=c] (7.57214,10.3467) rectangle (7.61194,10.4526);
\draw [color=c, fill=c] (7.61194,10.3467) rectangle (7.65174,10.4526);
\draw [color=c, fill=c] (7.65174,10.3467) rectangle (7.69154,10.4526);
\draw [color=c, fill=c] (7.69154,10.3467) rectangle (7.73134,10.4526);
\draw [color=c, fill=c] (7.73134,10.3467) rectangle (7.77114,10.4526);
\definecolor{c}{rgb}{0,0.0800001,1};
\draw [color=c, fill=c] (7.77114,10.3467) rectangle (7.81095,10.4526);
\draw [color=c, fill=c] (7.81095,10.3467) rectangle (7.85075,10.4526);
\draw [color=c, fill=c] (7.85075,10.3467) rectangle (7.89055,10.4526);
\draw [color=c, fill=c] (7.89055,10.3467) rectangle (7.93035,10.4526);
\draw [color=c, fill=c] (7.93035,10.3467) rectangle (7.97015,10.4526);
\draw [color=c, fill=c] (7.97015,10.3467) rectangle (8.00995,10.4526);
\draw [color=c, fill=c] (8.00995,10.3467) rectangle (8.04975,10.4526);
\draw [color=c, fill=c] (8.04975,10.3467) rectangle (8.08955,10.4526);
\draw [color=c, fill=c] (8.08955,10.3467) rectangle (8.12935,10.4526);
\draw [color=c, fill=c] (8.12935,10.3467) rectangle (8.16915,10.4526);
\draw [color=c, fill=c] (8.16915,10.3467) rectangle (8.20895,10.4526);
\draw [color=c, fill=c] (8.20895,10.3467) rectangle (8.24876,10.4526);
\draw [color=c, fill=c] (8.24876,10.3467) rectangle (8.28856,10.4526);
\draw [color=c, fill=c] (8.28856,10.3467) rectangle (8.32836,10.4526);
\draw [color=c, fill=c] (8.32836,10.3467) rectangle (8.36816,10.4526);
\draw [color=c, fill=c] (8.36816,10.3467) rectangle (8.40796,10.4526);
\draw [color=c, fill=c] (8.40796,10.3467) rectangle (8.44776,10.4526);
\draw [color=c, fill=c] (8.44776,10.3467) rectangle (8.48756,10.4526);
\draw [color=c, fill=c] (8.48756,10.3467) rectangle (8.52736,10.4526);
\draw [color=c, fill=c] (8.52736,10.3467) rectangle (8.56716,10.4526);
\draw [color=c, fill=c] (8.56716,10.3467) rectangle (8.60697,10.4526);
\draw [color=c, fill=c] (8.60697,10.3467) rectangle (8.64677,10.4526);
\draw [color=c, fill=c] (8.64677,10.3467) rectangle (8.68657,10.4526);
\draw [color=c, fill=c] (8.68657,10.3467) rectangle (8.72637,10.4526);
\draw [color=c, fill=c] (8.72637,10.3467) rectangle (8.76617,10.4526);
\draw [color=c, fill=c] (8.76617,10.3467) rectangle (8.80597,10.4526);
\draw [color=c, fill=c] (8.80597,10.3467) rectangle (8.84577,10.4526);
\draw [color=c, fill=c] (8.84577,10.3467) rectangle (8.88557,10.4526);
\draw [color=c, fill=c] (8.88557,10.3467) rectangle (8.92537,10.4526);
\draw [color=c, fill=c] (8.92537,10.3467) rectangle (8.96517,10.4526);
\draw [color=c, fill=c] (8.96517,10.3467) rectangle (9.00498,10.4526);
\draw [color=c, fill=c] (9.00498,10.3467) rectangle (9.04478,10.4526);
\draw [color=c, fill=c] (9.04478,10.3467) rectangle (9.08458,10.4526);
\draw [color=c, fill=c] (9.08458,10.3467) rectangle (9.12438,10.4526);
\draw [color=c, fill=c] (9.12438,10.3467) rectangle (9.16418,10.4526);
\draw [color=c, fill=c] (9.16418,10.3467) rectangle (9.20398,10.4526);
\draw [color=c, fill=c] (9.20398,10.3467) rectangle (9.24378,10.4526);
\draw [color=c, fill=c] (9.24378,10.3467) rectangle (9.28358,10.4526);
\draw [color=c, fill=c] (9.28358,10.3467) rectangle (9.32338,10.4526);
\draw [color=c, fill=c] (9.32338,10.3467) rectangle (9.36318,10.4526);
\draw [color=c, fill=c] (9.36318,10.3467) rectangle (9.40298,10.4526);
\draw [color=c, fill=c] (9.40298,10.3467) rectangle (9.44279,10.4526);
\draw [color=c, fill=c] (9.44279,10.3467) rectangle (9.48259,10.4526);
\draw [color=c, fill=c] (9.48259,10.3467) rectangle (9.52239,10.4526);
\draw [color=c, fill=c] (9.52239,10.3467) rectangle (9.56219,10.4526);
\definecolor{c}{rgb}{0,0.266667,1};
\draw [color=c, fill=c] (9.56219,10.3467) rectangle (9.60199,10.4526);
\draw [color=c, fill=c] (9.60199,10.3467) rectangle (9.64179,10.4526);
\draw [color=c, fill=c] (9.64179,10.3467) rectangle (9.68159,10.4526);
\draw [color=c, fill=c] (9.68159,10.3467) rectangle (9.72139,10.4526);
\draw [color=c, fill=c] (9.72139,10.3467) rectangle (9.76119,10.4526);
\draw [color=c, fill=c] (9.76119,10.3467) rectangle (9.80099,10.4526);
\draw [color=c, fill=c] (9.80099,10.3467) rectangle (9.8408,10.4526);
\draw [color=c, fill=c] (9.8408,10.3467) rectangle (9.8806,10.4526);
\draw [color=c, fill=c] (9.8806,10.3467) rectangle (9.9204,10.4526);
\draw [color=c, fill=c] (9.9204,10.3467) rectangle (9.9602,10.4526);
\draw [color=c, fill=c] (9.9602,10.3467) rectangle (10,10.4526);
\draw [color=c, fill=c] (10,10.3467) rectangle (10.0398,10.4526);
\draw [color=c, fill=c] (10.0398,10.3467) rectangle (10.0796,10.4526);
\draw [color=c, fill=c] (10.0796,10.3467) rectangle (10.1194,10.4526);
\draw [color=c, fill=c] (10.1194,10.3467) rectangle (10.1592,10.4526);
\draw [color=c, fill=c] (10.1592,10.3467) rectangle (10.199,10.4526);
\draw [color=c, fill=c] (10.199,10.3467) rectangle (10.2388,10.4526);
\draw [color=c, fill=c] (10.2388,10.3467) rectangle (10.2786,10.4526);
\draw [color=c, fill=c] (10.2786,10.3467) rectangle (10.3184,10.4526);
\draw [color=c, fill=c] (10.3184,10.3467) rectangle (10.3582,10.4526);
\draw [color=c, fill=c] (10.3582,10.3467) rectangle (10.398,10.4526);
\draw [color=c, fill=c] (10.398,10.3467) rectangle (10.4378,10.4526);
\draw [color=c, fill=c] (10.4378,10.3467) rectangle (10.4776,10.4526);
\draw [color=c, fill=c] (10.4776,10.3467) rectangle (10.5174,10.4526);
\draw [color=c, fill=c] (10.5174,10.3467) rectangle (10.5572,10.4526);
\draw [color=c, fill=c] (10.5572,10.3467) rectangle (10.597,10.4526);
\draw [color=c, fill=c] (10.597,10.3467) rectangle (10.6368,10.4526);
\draw [color=c, fill=c] (10.6368,10.3467) rectangle (10.6766,10.4526);
\draw [color=c, fill=c] (10.6766,10.3467) rectangle (10.7164,10.4526);
\draw [color=c, fill=c] (10.7164,10.3467) rectangle (10.7562,10.4526);
\draw [color=c, fill=c] (10.7562,10.3467) rectangle (10.796,10.4526);
\draw [color=c, fill=c] (10.796,10.3467) rectangle (10.8358,10.4526);
\definecolor{c}{rgb}{0,0.546666,1};
\draw [color=c, fill=c] (10.8358,10.3467) rectangle (10.8756,10.4526);
\draw [color=c, fill=c] (10.8756,10.3467) rectangle (10.9154,10.4526);
\draw [color=c, fill=c] (10.9154,10.3467) rectangle (10.9552,10.4526);
\draw [color=c, fill=c] (10.9552,10.3467) rectangle (10.995,10.4526);
\draw [color=c, fill=c] (10.995,10.3467) rectangle (11.0348,10.4526);
\draw [color=c, fill=c] (11.0348,10.3467) rectangle (11.0746,10.4526);
\draw [color=c, fill=c] (11.0746,10.3467) rectangle (11.1144,10.4526);
\draw [color=c, fill=c] (11.1144,10.3467) rectangle (11.1542,10.4526);
\draw [color=c, fill=c] (11.1542,10.3467) rectangle (11.194,10.4526);
\draw [color=c, fill=c] (11.194,10.3467) rectangle (11.2338,10.4526);
\draw [color=c, fill=c] (11.2338,10.3467) rectangle (11.2736,10.4526);
\draw [color=c, fill=c] (11.2736,10.3467) rectangle (11.3134,10.4526);
\draw [color=c, fill=c] (11.3134,10.3467) rectangle (11.3532,10.4526);
\draw [color=c, fill=c] (11.3532,10.3467) rectangle (11.393,10.4526);
\draw [color=c, fill=c] (11.393,10.3467) rectangle (11.4328,10.4526);
\draw [color=c, fill=c] (11.4328,10.3467) rectangle (11.4726,10.4526);
\draw [color=c, fill=c] (11.4726,10.3467) rectangle (11.5124,10.4526);
\draw [color=c, fill=c] (11.5124,10.3467) rectangle (11.5522,10.4526);
\draw [color=c, fill=c] (11.5522,10.3467) rectangle (11.592,10.4526);
\draw [color=c, fill=c] (11.592,10.3467) rectangle (11.6318,10.4526);
\draw [color=c, fill=c] (11.6318,10.3467) rectangle (11.6716,10.4526);
\draw [color=c, fill=c] (11.6716,10.3467) rectangle (11.7114,10.4526);
\draw [color=c, fill=c] (11.7114,10.3467) rectangle (11.7512,10.4526);
\draw [color=c, fill=c] (11.7512,10.3467) rectangle (11.791,10.4526);
\draw [color=c, fill=c] (11.791,10.3467) rectangle (11.8308,10.4526);
\draw [color=c, fill=c] (11.8308,10.3467) rectangle (11.8706,10.4526);
\draw [color=c, fill=c] (11.8706,10.3467) rectangle (11.9104,10.4526);
\draw [color=c, fill=c] (11.9104,10.3467) rectangle (11.9502,10.4526);
\draw [color=c, fill=c] (11.9502,10.3467) rectangle (11.99,10.4526);
\draw [color=c, fill=c] (11.99,10.3467) rectangle (12.0299,10.4526);
\draw [color=c, fill=c] (12.0299,10.3467) rectangle (12.0697,10.4526);
\draw [color=c, fill=c] (12.0697,10.3467) rectangle (12.1095,10.4526);
\draw [color=c, fill=c] (12.1095,10.3467) rectangle (12.1493,10.4526);
\draw [color=c, fill=c] (12.1493,10.3467) rectangle (12.1891,10.4526);
\draw [color=c, fill=c] (12.1891,10.3467) rectangle (12.2289,10.4526);
\draw [color=c, fill=c] (12.2289,10.3467) rectangle (12.2687,10.4526);
\draw [color=c, fill=c] (12.2687,10.3467) rectangle (12.3085,10.4526);
\draw [color=c, fill=c] (12.3085,10.3467) rectangle (12.3483,10.4526);
\draw [color=c, fill=c] (12.3483,10.3467) rectangle (12.3881,10.4526);
\draw [color=c, fill=c] (12.3881,10.3467) rectangle (12.4279,10.4526);
\draw [color=c, fill=c] (12.4279,10.3467) rectangle (12.4677,10.4526);
\draw [color=c, fill=c] (12.4677,10.3467) rectangle (12.5075,10.4526);
\draw [color=c, fill=c] (12.5075,10.3467) rectangle (12.5473,10.4526);
\draw [color=c, fill=c] (12.5473,10.3467) rectangle (12.5871,10.4526);
\draw [color=c, fill=c] (12.5871,10.3467) rectangle (12.6269,10.4526);
\draw [color=c, fill=c] (12.6269,10.3467) rectangle (12.6667,10.4526);
\draw [color=c, fill=c] (12.6667,10.3467) rectangle (12.7065,10.4526);
\draw [color=c, fill=c] (12.7065,10.3467) rectangle (12.7463,10.4526);
\draw [color=c, fill=c] (12.7463,10.3467) rectangle (12.7861,10.4526);
\draw [color=c, fill=c] (12.7861,10.3467) rectangle (12.8259,10.4526);
\draw [color=c, fill=c] (12.8259,10.3467) rectangle (12.8657,10.4526);
\draw [color=c, fill=c] (12.8657,10.3467) rectangle (12.9055,10.4526);
\draw [color=c, fill=c] (12.9055,10.3467) rectangle (12.9453,10.4526);
\draw [color=c, fill=c] (12.9453,10.3467) rectangle (12.9851,10.4526);
\draw [color=c, fill=c] (12.9851,10.3467) rectangle (13.0249,10.4526);
\draw [color=c, fill=c] (13.0249,10.3467) rectangle (13.0647,10.4526);
\draw [color=c, fill=c] (13.0647,10.3467) rectangle (13.1045,10.4526);
\draw [color=c, fill=c] (13.1045,10.3467) rectangle (13.1443,10.4526);
\draw [color=c, fill=c] (13.1443,10.3467) rectangle (13.1841,10.4526);
\draw [color=c, fill=c] (13.1841,10.3467) rectangle (13.2239,10.4526);
\draw [color=c, fill=c] (13.2239,10.3467) rectangle (13.2637,10.4526);
\draw [color=c, fill=c] (13.2637,10.3467) rectangle (13.3035,10.4526);
\draw [color=c, fill=c] (13.3035,10.3467) rectangle (13.3433,10.4526);
\draw [color=c, fill=c] (13.3433,10.3467) rectangle (13.3831,10.4526);
\draw [color=c, fill=c] (13.3831,10.3467) rectangle (13.4229,10.4526);
\draw [color=c, fill=c] (13.4229,10.3467) rectangle (13.4627,10.4526);
\draw [color=c, fill=c] (13.4627,10.3467) rectangle (13.5025,10.4526);
\draw [color=c, fill=c] (13.5025,10.3467) rectangle (13.5423,10.4526);
\draw [color=c, fill=c] (13.5423,10.3467) rectangle (13.5821,10.4526);
\draw [color=c, fill=c] (13.5821,10.3467) rectangle (13.6219,10.4526);
\draw [color=c, fill=c] (13.6219,10.3467) rectangle (13.6617,10.4526);
\draw [color=c, fill=c] (13.6617,10.3467) rectangle (13.7015,10.4526);
\draw [color=c, fill=c] (13.7015,10.3467) rectangle (13.7413,10.4526);
\draw [color=c, fill=c] (13.7413,10.3467) rectangle (13.7811,10.4526);
\draw [color=c, fill=c] (13.7811,10.3467) rectangle (13.8209,10.4526);
\draw [color=c, fill=c] (13.8209,10.3467) rectangle (13.8607,10.4526);
\draw [color=c, fill=c] (13.8607,10.3467) rectangle (13.9005,10.4526);
\draw [color=c, fill=c] (13.9005,10.3467) rectangle (13.9403,10.4526);
\draw [color=c, fill=c] (13.9403,10.3467) rectangle (13.9801,10.4526);
\draw [color=c, fill=c] (13.9801,10.3467) rectangle (14.0199,10.4526);
\definecolor{c}{rgb}{0,0.733333,1};
\draw [color=c, fill=c] (14.0199,10.3467) rectangle (14.0597,10.4526);
\draw [color=c, fill=c] (14.0597,10.3467) rectangle (14.0995,10.4526);
\draw [color=c, fill=c] (14.0995,10.3467) rectangle (14.1393,10.4526);
\draw [color=c, fill=c] (14.1393,10.3467) rectangle (14.1791,10.4526);
\draw [color=c, fill=c] (14.1791,10.3467) rectangle (14.2189,10.4526);
\draw [color=c, fill=c] (14.2189,10.3467) rectangle (14.2587,10.4526);
\draw [color=c, fill=c] (14.2587,10.3467) rectangle (14.2985,10.4526);
\draw [color=c, fill=c] (14.2985,10.3467) rectangle (14.3383,10.4526);
\draw [color=c, fill=c] (14.3383,10.3467) rectangle (14.3781,10.4526);
\draw [color=c, fill=c] (14.3781,10.3467) rectangle (14.4179,10.4526);
\draw [color=c, fill=c] (14.4179,10.3467) rectangle (14.4577,10.4526);
\draw [color=c, fill=c] (14.4577,10.3467) rectangle (14.4975,10.4526);
\draw [color=c, fill=c] (14.4975,10.3467) rectangle (14.5373,10.4526);
\draw [color=c, fill=c] (14.5373,10.3467) rectangle (14.5771,10.4526);
\draw [color=c, fill=c] (14.5771,10.3467) rectangle (14.6169,10.4526);
\draw [color=c, fill=c] (14.6169,10.3467) rectangle (14.6567,10.4526);
\draw [color=c, fill=c] (14.6567,10.3467) rectangle (14.6965,10.4526);
\draw [color=c, fill=c] (14.6965,10.3467) rectangle (14.7363,10.4526);
\draw [color=c, fill=c] (14.7363,10.3467) rectangle (14.7761,10.4526);
\draw [color=c, fill=c] (14.7761,10.3467) rectangle (14.8159,10.4526);
\draw [color=c, fill=c] (14.8159,10.3467) rectangle (14.8557,10.4526);
\draw [color=c, fill=c] (14.8557,10.3467) rectangle (14.8955,10.4526);
\draw [color=c, fill=c] (14.8955,10.3467) rectangle (14.9353,10.4526);
\draw [color=c, fill=c] (14.9353,10.3467) rectangle (14.9751,10.4526);
\draw [color=c, fill=c] (14.9751,10.3467) rectangle (15.0149,10.4526);
\draw [color=c, fill=c] (15.0149,10.3467) rectangle (15.0547,10.4526);
\draw [color=c, fill=c] (15.0547,10.3467) rectangle (15.0945,10.4526);
\draw [color=c, fill=c] (15.0945,10.3467) rectangle (15.1343,10.4526);
\draw [color=c, fill=c] (15.1343,10.3467) rectangle (15.1741,10.4526);
\draw [color=c, fill=c] (15.1741,10.3467) rectangle (15.2139,10.4526);
\draw [color=c, fill=c] (15.2139,10.3467) rectangle (15.2537,10.4526);
\draw [color=c, fill=c] (15.2537,10.3467) rectangle (15.2935,10.4526);
\draw [color=c, fill=c] (15.2935,10.3467) rectangle (15.3333,10.4526);
\draw [color=c, fill=c] (15.3333,10.3467) rectangle (15.3731,10.4526);
\draw [color=c, fill=c] (15.3731,10.3467) rectangle (15.4129,10.4526);
\draw [color=c, fill=c] (15.4129,10.3467) rectangle (15.4527,10.4526);
\draw [color=c, fill=c] (15.4527,10.3467) rectangle (15.4925,10.4526);
\draw [color=c, fill=c] (15.4925,10.3467) rectangle (15.5323,10.4526);
\draw [color=c, fill=c] (15.5323,10.3467) rectangle (15.5721,10.4526);
\draw [color=c, fill=c] (15.5721,10.3467) rectangle (15.6119,10.4526);
\draw [color=c, fill=c] (15.6119,10.3467) rectangle (15.6517,10.4526);
\draw [color=c, fill=c] (15.6517,10.3467) rectangle (15.6915,10.4526);
\draw [color=c, fill=c] (15.6915,10.3467) rectangle (15.7313,10.4526);
\draw [color=c, fill=c] (15.7313,10.3467) rectangle (15.7711,10.4526);
\draw [color=c, fill=c] (15.7711,10.3467) rectangle (15.8109,10.4526);
\draw [color=c, fill=c] (15.8109,10.3467) rectangle (15.8507,10.4526);
\draw [color=c, fill=c] (15.8507,10.3467) rectangle (15.8905,10.4526);
\draw [color=c, fill=c] (15.8905,10.3467) rectangle (15.9303,10.4526);
\draw [color=c, fill=c] (15.9303,10.3467) rectangle (15.9701,10.4526);
\draw [color=c, fill=c] (15.9701,10.3467) rectangle (16.01,10.4526);
\draw [color=c, fill=c] (16.01,10.3467) rectangle (16.0498,10.4526);
\draw [color=c, fill=c] (16.0498,10.3467) rectangle (16.0896,10.4526);
\draw [color=c, fill=c] (16.0896,10.3467) rectangle (16.1294,10.4526);
\draw [color=c, fill=c] (16.1294,10.3467) rectangle (16.1692,10.4526);
\draw [color=c, fill=c] (16.1692,10.3467) rectangle (16.209,10.4526);
\draw [color=c, fill=c] (16.209,10.3467) rectangle (16.2488,10.4526);
\draw [color=c, fill=c] (16.2488,10.3467) rectangle (16.2886,10.4526);
\draw [color=c, fill=c] (16.2886,10.3467) rectangle (16.3284,10.4526);
\draw [color=c, fill=c] (16.3284,10.3467) rectangle (16.3682,10.4526);
\draw [color=c, fill=c] (16.3682,10.3467) rectangle (16.408,10.4526);
\draw [color=c, fill=c] (16.408,10.3467) rectangle (16.4478,10.4526);
\draw [color=c, fill=c] (16.4478,10.3467) rectangle (16.4876,10.4526);
\draw [color=c, fill=c] (16.4876,10.3467) rectangle (16.5274,10.4526);
\draw [color=c, fill=c] (16.5274,10.3467) rectangle (16.5672,10.4526);
\draw [color=c, fill=c] (16.5672,10.3467) rectangle (16.607,10.4526);
\draw [color=c, fill=c] (16.607,10.3467) rectangle (16.6468,10.4526);
\draw [color=c, fill=c] (16.6468,10.3467) rectangle (16.6866,10.4526);
\draw [color=c, fill=c] (16.6866,10.3467) rectangle (16.7264,10.4526);
\draw [color=c, fill=c] (16.7264,10.3467) rectangle (16.7662,10.4526);
\draw [color=c, fill=c] (16.7662,10.3467) rectangle (16.806,10.4526);
\draw [color=c, fill=c] (16.806,10.3467) rectangle (16.8458,10.4526);
\draw [color=c, fill=c] (16.8458,10.3467) rectangle (16.8856,10.4526);
\draw [color=c, fill=c] (16.8856,10.3467) rectangle (16.9254,10.4526);
\draw [color=c, fill=c] (16.9254,10.3467) rectangle (16.9652,10.4526);
\draw [color=c, fill=c] (16.9652,10.3467) rectangle (17.005,10.4526);
\draw [color=c, fill=c] (17.005,10.3467) rectangle (17.0448,10.4526);
\draw [color=c, fill=c] (17.0448,10.3467) rectangle (17.0846,10.4526);
\draw [color=c, fill=c] (17.0846,10.3467) rectangle (17.1244,10.4526);
\draw [color=c, fill=c] (17.1244,10.3467) rectangle (17.1642,10.4526);
\draw [color=c, fill=c] (17.1642,10.3467) rectangle (17.204,10.4526);
\draw [color=c, fill=c] (17.204,10.3467) rectangle (17.2438,10.4526);
\draw [color=c, fill=c] (17.2438,10.3467) rectangle (17.2836,10.4526);
\draw [color=c, fill=c] (17.2836,10.3467) rectangle (17.3234,10.4526);
\draw [color=c, fill=c] (17.3234,10.3467) rectangle (17.3632,10.4526);
\draw [color=c, fill=c] (17.3632,10.3467) rectangle (17.403,10.4526);
\draw [color=c, fill=c] (17.403,10.3467) rectangle (17.4428,10.4526);
\draw [color=c, fill=c] (17.4428,10.3467) rectangle (17.4826,10.4526);
\draw [color=c, fill=c] (17.4826,10.3467) rectangle (17.5224,10.4526);
\draw [color=c, fill=c] (17.5224,10.3467) rectangle (17.5622,10.4526);
\draw [color=c, fill=c] (17.5622,10.3467) rectangle (17.602,10.4526);
\draw [color=c, fill=c] (17.602,10.3467) rectangle (17.6418,10.4526);
\draw [color=c, fill=c] (17.6418,10.3467) rectangle (17.6816,10.4526);
\draw [color=c, fill=c] (17.6816,10.3467) rectangle (17.7214,10.4526);
\draw [color=c, fill=c] (17.7214,10.3467) rectangle (17.7612,10.4526);
\draw [color=c, fill=c] (17.7612,10.3467) rectangle (17.801,10.4526);
\draw [color=c, fill=c] (17.801,10.3467) rectangle (17.8408,10.4526);
\draw [color=c, fill=c] (17.8408,10.3467) rectangle (17.8806,10.4526);
\draw [color=c, fill=c] (17.8806,10.3467) rectangle (17.9204,10.4526);
\draw [color=c, fill=c] (17.9204,10.3467) rectangle (17.9602,10.4526);
\draw [color=c, fill=c] (17.9602,10.3467) rectangle (18,10.4526);
\definecolor{c}{rgb}{0.2,0,1};
\draw [color=c, fill=c] (2,10.4526) rectangle (2.0398,10.5584);
\draw [color=c, fill=c] (2.0398,10.4526) rectangle (2.0796,10.5584);
\draw [color=c, fill=c] (2.0796,10.4526) rectangle (2.1194,10.5584);
\draw [color=c, fill=c] (2.1194,10.4526) rectangle (2.1592,10.5584);
\draw [color=c, fill=c] (2.1592,10.4526) rectangle (2.19901,10.5584);
\draw [color=c, fill=c] (2.19901,10.4526) rectangle (2.23881,10.5584);
\draw [color=c, fill=c] (2.23881,10.4526) rectangle (2.27861,10.5584);
\draw [color=c, fill=c] (2.27861,10.4526) rectangle (2.31841,10.5584);
\draw [color=c, fill=c] (2.31841,10.4526) rectangle (2.35821,10.5584);
\draw [color=c, fill=c] (2.35821,10.4526) rectangle (2.39801,10.5584);
\draw [color=c, fill=c] (2.39801,10.4526) rectangle (2.43781,10.5584);
\draw [color=c, fill=c] (2.43781,10.4526) rectangle (2.47761,10.5584);
\draw [color=c, fill=c] (2.47761,10.4526) rectangle (2.51741,10.5584);
\draw [color=c, fill=c] (2.51741,10.4526) rectangle (2.55721,10.5584);
\draw [color=c, fill=c] (2.55721,10.4526) rectangle (2.59702,10.5584);
\draw [color=c, fill=c] (2.59702,10.4526) rectangle (2.63682,10.5584);
\draw [color=c, fill=c] (2.63682,10.4526) rectangle (2.67662,10.5584);
\draw [color=c, fill=c] (2.67662,10.4526) rectangle (2.71642,10.5584);
\draw [color=c, fill=c] (2.71642,10.4526) rectangle (2.75622,10.5584);
\draw [color=c, fill=c] (2.75622,10.4526) rectangle (2.79602,10.5584);
\draw [color=c, fill=c] (2.79602,10.4526) rectangle (2.83582,10.5584);
\draw [color=c, fill=c] (2.83582,10.4526) rectangle (2.87562,10.5584);
\draw [color=c, fill=c] (2.87562,10.4526) rectangle (2.91542,10.5584);
\draw [color=c, fill=c] (2.91542,10.4526) rectangle (2.95522,10.5584);
\draw [color=c, fill=c] (2.95522,10.4526) rectangle (2.99502,10.5584);
\draw [color=c, fill=c] (2.99502,10.4526) rectangle (3.03483,10.5584);
\draw [color=c, fill=c] (3.03483,10.4526) rectangle (3.07463,10.5584);
\draw [color=c, fill=c] (3.07463,10.4526) rectangle (3.11443,10.5584);
\draw [color=c, fill=c] (3.11443,10.4526) rectangle (3.15423,10.5584);
\draw [color=c, fill=c] (3.15423,10.4526) rectangle (3.19403,10.5584);
\draw [color=c, fill=c] (3.19403,10.4526) rectangle (3.23383,10.5584);
\draw [color=c, fill=c] (3.23383,10.4526) rectangle (3.27363,10.5584);
\draw [color=c, fill=c] (3.27363,10.4526) rectangle (3.31343,10.5584);
\draw [color=c, fill=c] (3.31343,10.4526) rectangle (3.35323,10.5584);
\draw [color=c, fill=c] (3.35323,10.4526) rectangle (3.39303,10.5584);
\draw [color=c, fill=c] (3.39303,10.4526) rectangle (3.43284,10.5584);
\draw [color=c, fill=c] (3.43284,10.4526) rectangle (3.47264,10.5584);
\draw [color=c, fill=c] (3.47264,10.4526) rectangle (3.51244,10.5584);
\draw [color=c, fill=c] (3.51244,10.4526) rectangle (3.55224,10.5584);
\draw [color=c, fill=c] (3.55224,10.4526) rectangle (3.59204,10.5584);
\draw [color=c, fill=c] (3.59204,10.4526) rectangle (3.63184,10.5584);
\draw [color=c, fill=c] (3.63184,10.4526) rectangle (3.67164,10.5584);
\draw [color=c, fill=c] (3.67164,10.4526) rectangle (3.71144,10.5584);
\draw [color=c, fill=c] (3.71144,10.4526) rectangle (3.75124,10.5584);
\draw [color=c, fill=c] (3.75124,10.4526) rectangle (3.79104,10.5584);
\draw [color=c, fill=c] (3.79104,10.4526) rectangle (3.83085,10.5584);
\draw [color=c, fill=c] (3.83085,10.4526) rectangle (3.87065,10.5584);
\draw [color=c, fill=c] (3.87065,10.4526) rectangle (3.91045,10.5584);
\draw [color=c, fill=c] (3.91045,10.4526) rectangle (3.95025,10.5584);
\draw [color=c, fill=c] (3.95025,10.4526) rectangle (3.99005,10.5584);
\draw [color=c, fill=c] (3.99005,10.4526) rectangle (4.02985,10.5584);
\draw [color=c, fill=c] (4.02985,10.4526) rectangle (4.06965,10.5584);
\draw [color=c, fill=c] (4.06965,10.4526) rectangle (4.10945,10.5584);
\draw [color=c, fill=c] (4.10945,10.4526) rectangle (4.14925,10.5584);
\draw [color=c, fill=c] (4.14925,10.4526) rectangle (4.18905,10.5584);
\draw [color=c, fill=c] (4.18905,10.4526) rectangle (4.22886,10.5584);
\draw [color=c, fill=c] (4.22886,10.4526) rectangle (4.26866,10.5584);
\draw [color=c, fill=c] (4.26866,10.4526) rectangle (4.30846,10.5584);
\draw [color=c, fill=c] (4.30846,10.4526) rectangle (4.34826,10.5584);
\draw [color=c, fill=c] (4.34826,10.4526) rectangle (4.38806,10.5584);
\draw [color=c, fill=c] (4.38806,10.4526) rectangle (4.42786,10.5584);
\draw [color=c, fill=c] (4.42786,10.4526) rectangle (4.46766,10.5584);
\draw [color=c, fill=c] (4.46766,10.4526) rectangle (4.50746,10.5584);
\draw [color=c, fill=c] (4.50746,10.4526) rectangle (4.54726,10.5584);
\draw [color=c, fill=c] (4.54726,10.4526) rectangle (4.58706,10.5584);
\draw [color=c, fill=c] (4.58706,10.4526) rectangle (4.62687,10.5584);
\draw [color=c, fill=c] (4.62687,10.4526) rectangle (4.66667,10.5584);
\draw [color=c, fill=c] (4.66667,10.4526) rectangle (4.70647,10.5584);
\draw [color=c, fill=c] (4.70647,10.4526) rectangle (4.74627,10.5584);
\draw [color=c, fill=c] (4.74627,10.4526) rectangle (4.78607,10.5584);
\draw [color=c, fill=c] (4.78607,10.4526) rectangle (4.82587,10.5584);
\draw [color=c, fill=c] (4.82587,10.4526) rectangle (4.86567,10.5584);
\draw [color=c, fill=c] (4.86567,10.4526) rectangle (4.90547,10.5584);
\draw [color=c, fill=c] (4.90547,10.4526) rectangle (4.94527,10.5584);
\draw [color=c, fill=c] (4.94527,10.4526) rectangle (4.98507,10.5584);
\draw [color=c, fill=c] (4.98507,10.4526) rectangle (5.02488,10.5584);
\draw [color=c, fill=c] (5.02488,10.4526) rectangle (5.06468,10.5584);
\draw [color=c, fill=c] (5.06468,10.4526) rectangle (5.10448,10.5584);
\draw [color=c, fill=c] (5.10448,10.4526) rectangle (5.14428,10.5584);
\draw [color=c, fill=c] (5.14428,10.4526) rectangle (5.18408,10.5584);
\draw [color=c, fill=c] (5.18408,10.4526) rectangle (5.22388,10.5584);
\draw [color=c, fill=c] (5.22388,10.4526) rectangle (5.26368,10.5584);
\draw [color=c, fill=c] (5.26368,10.4526) rectangle (5.30348,10.5584);
\draw [color=c, fill=c] (5.30348,10.4526) rectangle (5.34328,10.5584);
\draw [color=c, fill=c] (5.34328,10.4526) rectangle (5.38308,10.5584);
\draw [color=c, fill=c] (5.38308,10.4526) rectangle (5.42289,10.5584);
\draw [color=c, fill=c] (5.42289,10.4526) rectangle (5.46269,10.5584);
\draw [color=c, fill=c] (5.46269,10.4526) rectangle (5.50249,10.5584);
\draw [color=c, fill=c] (5.50249,10.4526) rectangle (5.54229,10.5584);
\draw [color=c, fill=c] (5.54229,10.4526) rectangle (5.58209,10.5584);
\draw [color=c, fill=c] (5.58209,10.4526) rectangle (5.62189,10.5584);
\draw [color=c, fill=c] (5.62189,10.4526) rectangle (5.66169,10.5584);
\draw [color=c, fill=c] (5.66169,10.4526) rectangle (5.70149,10.5584);
\draw [color=c, fill=c] (5.70149,10.4526) rectangle (5.74129,10.5584);
\draw [color=c, fill=c] (5.74129,10.4526) rectangle (5.78109,10.5584);
\draw [color=c, fill=c] (5.78109,10.4526) rectangle (5.8209,10.5584);
\draw [color=c, fill=c] (5.8209,10.4526) rectangle (5.8607,10.5584);
\draw [color=c, fill=c] (5.8607,10.4526) rectangle (5.9005,10.5584);
\draw [color=c, fill=c] (5.9005,10.4526) rectangle (5.9403,10.5584);
\draw [color=c, fill=c] (5.9403,10.4526) rectangle (5.9801,10.5584);
\draw [color=c, fill=c] (5.9801,10.4526) rectangle (6.0199,10.5584);
\draw [color=c, fill=c] (6.0199,10.4526) rectangle (6.0597,10.5584);
\draw [color=c, fill=c] (6.0597,10.4526) rectangle (6.0995,10.5584);
\draw [color=c, fill=c] (6.0995,10.4526) rectangle (6.1393,10.5584);
\draw [color=c, fill=c] (6.1393,10.4526) rectangle (6.1791,10.5584);
\draw [color=c, fill=c] (6.1791,10.4526) rectangle (6.21891,10.5584);
\draw [color=c, fill=c] (6.21891,10.4526) rectangle (6.25871,10.5584);
\draw [color=c, fill=c] (6.25871,10.4526) rectangle (6.29851,10.5584);
\draw [color=c, fill=c] (6.29851,10.4526) rectangle (6.33831,10.5584);
\draw [color=c, fill=c] (6.33831,10.4526) rectangle (6.37811,10.5584);
\draw [color=c, fill=c] (6.37811,10.4526) rectangle (6.41791,10.5584);
\draw [color=c, fill=c] (6.41791,10.4526) rectangle (6.45771,10.5584);
\draw [color=c, fill=c] (6.45771,10.4526) rectangle (6.49751,10.5584);
\draw [color=c, fill=c] (6.49751,10.4526) rectangle (6.53731,10.5584);
\draw [color=c, fill=c] (6.53731,10.4526) rectangle (6.57711,10.5584);
\draw [color=c, fill=c] (6.57711,10.4526) rectangle (6.61692,10.5584);
\draw [color=c, fill=c] (6.61692,10.4526) rectangle (6.65672,10.5584);
\draw [color=c, fill=c] (6.65672,10.4526) rectangle (6.69652,10.5584);
\draw [color=c, fill=c] (6.69652,10.4526) rectangle (6.73632,10.5584);
\draw [color=c, fill=c] (6.73632,10.4526) rectangle (6.77612,10.5584);
\draw [color=c, fill=c] (6.77612,10.4526) rectangle (6.81592,10.5584);
\draw [color=c, fill=c] (6.81592,10.4526) rectangle (6.85572,10.5584);
\draw [color=c, fill=c] (6.85572,10.4526) rectangle (6.89552,10.5584);
\draw [color=c, fill=c] (6.89552,10.4526) rectangle (6.93532,10.5584);
\draw [color=c, fill=c] (6.93532,10.4526) rectangle (6.97512,10.5584);
\draw [color=c, fill=c] (6.97512,10.4526) rectangle (7.01493,10.5584);
\draw [color=c, fill=c] (7.01493,10.4526) rectangle (7.05473,10.5584);
\draw [color=c, fill=c] (7.05473,10.4526) rectangle (7.09453,10.5584);
\draw [color=c, fill=c] (7.09453,10.4526) rectangle (7.13433,10.5584);
\draw [color=c, fill=c] (7.13433,10.4526) rectangle (7.17413,10.5584);
\draw [color=c, fill=c] (7.17413,10.4526) rectangle (7.21393,10.5584);
\draw [color=c, fill=c] (7.21393,10.4526) rectangle (7.25373,10.5584);
\draw [color=c, fill=c] (7.25373,10.4526) rectangle (7.29353,10.5584);
\draw [color=c, fill=c] (7.29353,10.4526) rectangle (7.33333,10.5584);
\draw [color=c, fill=c] (7.33333,10.4526) rectangle (7.37313,10.5584);
\draw [color=c, fill=c] (7.37313,10.4526) rectangle (7.41294,10.5584);
\draw [color=c, fill=c] (7.41294,10.4526) rectangle (7.45274,10.5584);
\draw [color=c, fill=c] (7.45274,10.4526) rectangle (7.49254,10.5584);
\draw [color=c, fill=c] (7.49254,10.4526) rectangle (7.53234,10.5584);
\draw [color=c, fill=c] (7.53234,10.4526) rectangle (7.57214,10.5584);
\draw [color=c, fill=c] (7.57214,10.4526) rectangle (7.61194,10.5584);
\draw [color=c, fill=c] (7.61194,10.4526) rectangle (7.65174,10.5584);
\draw [color=c, fill=c] (7.65174,10.4526) rectangle (7.69154,10.5584);
\draw [color=c, fill=c] (7.69154,10.4526) rectangle (7.73134,10.5584);
\draw [color=c, fill=c] (7.73134,10.4526) rectangle (7.77114,10.5584);
\draw [color=c, fill=c] (7.77114,10.4526) rectangle (7.81095,10.5584);
\definecolor{c}{rgb}{0,0.0800001,1};
\draw [color=c, fill=c] (7.81095,10.4526) rectangle (7.85075,10.5584);
\draw [color=c, fill=c] (7.85075,10.4526) rectangle (7.89055,10.5584);
\draw [color=c, fill=c] (7.89055,10.4526) rectangle (7.93035,10.5584);
\draw [color=c, fill=c] (7.93035,10.4526) rectangle (7.97015,10.5584);
\draw [color=c, fill=c] (7.97015,10.4526) rectangle (8.00995,10.5584);
\draw [color=c, fill=c] (8.00995,10.4526) rectangle (8.04975,10.5584);
\draw [color=c, fill=c] (8.04975,10.4526) rectangle (8.08955,10.5584);
\draw [color=c, fill=c] (8.08955,10.4526) rectangle (8.12935,10.5584);
\draw [color=c, fill=c] (8.12935,10.4526) rectangle (8.16915,10.5584);
\draw [color=c, fill=c] (8.16915,10.4526) rectangle (8.20895,10.5584);
\draw [color=c, fill=c] (8.20895,10.4526) rectangle (8.24876,10.5584);
\draw [color=c, fill=c] (8.24876,10.4526) rectangle (8.28856,10.5584);
\draw [color=c, fill=c] (8.28856,10.4526) rectangle (8.32836,10.5584);
\draw [color=c, fill=c] (8.32836,10.4526) rectangle (8.36816,10.5584);
\draw [color=c, fill=c] (8.36816,10.4526) rectangle (8.40796,10.5584);
\draw [color=c, fill=c] (8.40796,10.4526) rectangle (8.44776,10.5584);
\draw [color=c, fill=c] (8.44776,10.4526) rectangle (8.48756,10.5584);
\draw [color=c, fill=c] (8.48756,10.4526) rectangle (8.52736,10.5584);
\draw [color=c, fill=c] (8.52736,10.4526) rectangle (8.56716,10.5584);
\draw [color=c, fill=c] (8.56716,10.4526) rectangle (8.60697,10.5584);
\draw [color=c, fill=c] (8.60697,10.4526) rectangle (8.64677,10.5584);
\draw [color=c, fill=c] (8.64677,10.4526) rectangle (8.68657,10.5584);
\draw [color=c, fill=c] (8.68657,10.4526) rectangle (8.72637,10.5584);
\draw [color=c, fill=c] (8.72637,10.4526) rectangle (8.76617,10.5584);
\draw [color=c, fill=c] (8.76617,10.4526) rectangle (8.80597,10.5584);
\draw [color=c, fill=c] (8.80597,10.4526) rectangle (8.84577,10.5584);
\draw [color=c, fill=c] (8.84577,10.4526) rectangle (8.88557,10.5584);
\draw [color=c, fill=c] (8.88557,10.4526) rectangle (8.92537,10.5584);
\draw [color=c, fill=c] (8.92537,10.4526) rectangle (8.96517,10.5584);
\draw [color=c, fill=c] (8.96517,10.4526) rectangle (9.00498,10.5584);
\draw [color=c, fill=c] (9.00498,10.4526) rectangle (9.04478,10.5584);
\draw [color=c, fill=c] (9.04478,10.4526) rectangle (9.08458,10.5584);
\draw [color=c, fill=c] (9.08458,10.4526) rectangle (9.12438,10.5584);
\draw [color=c, fill=c] (9.12438,10.4526) rectangle (9.16418,10.5584);
\draw [color=c, fill=c] (9.16418,10.4526) rectangle (9.20398,10.5584);
\draw [color=c, fill=c] (9.20398,10.4526) rectangle (9.24378,10.5584);
\draw [color=c, fill=c] (9.24378,10.4526) rectangle (9.28358,10.5584);
\draw [color=c, fill=c] (9.28358,10.4526) rectangle (9.32338,10.5584);
\draw [color=c, fill=c] (9.32338,10.4526) rectangle (9.36318,10.5584);
\draw [color=c, fill=c] (9.36318,10.4526) rectangle (9.40298,10.5584);
\draw [color=c, fill=c] (9.40298,10.4526) rectangle (9.44279,10.5584);
\draw [color=c, fill=c] (9.44279,10.4526) rectangle (9.48259,10.5584);
\draw [color=c, fill=c] (9.48259,10.4526) rectangle (9.52239,10.5584);
\draw [color=c, fill=c] (9.52239,10.4526) rectangle (9.56219,10.5584);
\definecolor{c}{rgb}{0,0.266667,1};
\draw [color=c, fill=c] (9.56219,10.4526) rectangle (9.60199,10.5584);
\draw [color=c, fill=c] (9.60199,10.4526) rectangle (9.64179,10.5584);
\draw [color=c, fill=c] (9.64179,10.4526) rectangle (9.68159,10.5584);
\draw [color=c, fill=c] (9.68159,10.4526) rectangle (9.72139,10.5584);
\draw [color=c, fill=c] (9.72139,10.4526) rectangle (9.76119,10.5584);
\draw [color=c, fill=c] (9.76119,10.4526) rectangle (9.80099,10.5584);
\draw [color=c, fill=c] (9.80099,10.4526) rectangle (9.8408,10.5584);
\draw [color=c, fill=c] (9.8408,10.4526) rectangle (9.8806,10.5584);
\draw [color=c, fill=c] (9.8806,10.4526) rectangle (9.9204,10.5584);
\draw [color=c, fill=c] (9.9204,10.4526) rectangle (9.9602,10.5584);
\draw [color=c, fill=c] (9.9602,10.4526) rectangle (10,10.5584);
\draw [color=c, fill=c] (10,10.4526) rectangle (10.0398,10.5584);
\draw [color=c, fill=c] (10.0398,10.4526) rectangle (10.0796,10.5584);
\draw [color=c, fill=c] (10.0796,10.4526) rectangle (10.1194,10.5584);
\draw [color=c, fill=c] (10.1194,10.4526) rectangle (10.1592,10.5584);
\draw [color=c, fill=c] (10.1592,10.4526) rectangle (10.199,10.5584);
\draw [color=c, fill=c] (10.199,10.4526) rectangle (10.2388,10.5584);
\draw [color=c, fill=c] (10.2388,10.4526) rectangle (10.2786,10.5584);
\draw [color=c, fill=c] (10.2786,10.4526) rectangle (10.3184,10.5584);
\draw [color=c, fill=c] (10.3184,10.4526) rectangle (10.3582,10.5584);
\draw [color=c, fill=c] (10.3582,10.4526) rectangle (10.398,10.5584);
\draw [color=c, fill=c] (10.398,10.4526) rectangle (10.4378,10.5584);
\draw [color=c, fill=c] (10.4378,10.4526) rectangle (10.4776,10.5584);
\draw [color=c, fill=c] (10.4776,10.4526) rectangle (10.5174,10.5584);
\draw [color=c, fill=c] (10.5174,10.4526) rectangle (10.5572,10.5584);
\draw [color=c, fill=c] (10.5572,10.4526) rectangle (10.597,10.5584);
\draw [color=c, fill=c] (10.597,10.4526) rectangle (10.6368,10.5584);
\draw [color=c, fill=c] (10.6368,10.4526) rectangle (10.6766,10.5584);
\draw [color=c, fill=c] (10.6766,10.4526) rectangle (10.7164,10.5584);
\draw [color=c, fill=c] (10.7164,10.4526) rectangle (10.7562,10.5584);
\draw [color=c, fill=c] (10.7562,10.4526) rectangle (10.796,10.5584);
\draw [color=c, fill=c] (10.796,10.4526) rectangle (10.8358,10.5584);
\definecolor{c}{rgb}{0,0.546666,1};
\draw [color=c, fill=c] (10.8358,10.4526) rectangle (10.8756,10.5584);
\draw [color=c, fill=c] (10.8756,10.4526) rectangle (10.9154,10.5584);
\draw [color=c, fill=c] (10.9154,10.4526) rectangle (10.9552,10.5584);
\draw [color=c, fill=c] (10.9552,10.4526) rectangle (10.995,10.5584);
\draw [color=c, fill=c] (10.995,10.4526) rectangle (11.0348,10.5584);
\draw [color=c, fill=c] (11.0348,10.4526) rectangle (11.0746,10.5584);
\draw [color=c, fill=c] (11.0746,10.4526) rectangle (11.1144,10.5584);
\draw [color=c, fill=c] (11.1144,10.4526) rectangle (11.1542,10.5584);
\draw [color=c, fill=c] (11.1542,10.4526) rectangle (11.194,10.5584);
\draw [color=c, fill=c] (11.194,10.4526) rectangle (11.2338,10.5584);
\draw [color=c, fill=c] (11.2338,10.4526) rectangle (11.2736,10.5584);
\draw [color=c, fill=c] (11.2736,10.4526) rectangle (11.3134,10.5584);
\draw [color=c, fill=c] (11.3134,10.4526) rectangle (11.3532,10.5584);
\draw [color=c, fill=c] (11.3532,10.4526) rectangle (11.393,10.5584);
\draw [color=c, fill=c] (11.393,10.4526) rectangle (11.4328,10.5584);
\draw [color=c, fill=c] (11.4328,10.4526) rectangle (11.4726,10.5584);
\draw [color=c, fill=c] (11.4726,10.4526) rectangle (11.5124,10.5584);
\draw [color=c, fill=c] (11.5124,10.4526) rectangle (11.5522,10.5584);
\draw [color=c, fill=c] (11.5522,10.4526) rectangle (11.592,10.5584);
\draw [color=c, fill=c] (11.592,10.4526) rectangle (11.6318,10.5584);
\draw [color=c, fill=c] (11.6318,10.4526) rectangle (11.6716,10.5584);
\draw [color=c, fill=c] (11.6716,10.4526) rectangle (11.7114,10.5584);
\draw [color=c, fill=c] (11.7114,10.4526) rectangle (11.7512,10.5584);
\draw [color=c, fill=c] (11.7512,10.4526) rectangle (11.791,10.5584);
\draw [color=c, fill=c] (11.791,10.4526) rectangle (11.8308,10.5584);
\draw [color=c, fill=c] (11.8308,10.4526) rectangle (11.8706,10.5584);
\draw [color=c, fill=c] (11.8706,10.4526) rectangle (11.9104,10.5584);
\draw [color=c, fill=c] (11.9104,10.4526) rectangle (11.9502,10.5584);
\draw [color=c, fill=c] (11.9502,10.4526) rectangle (11.99,10.5584);
\draw [color=c, fill=c] (11.99,10.4526) rectangle (12.0299,10.5584);
\draw [color=c, fill=c] (12.0299,10.4526) rectangle (12.0697,10.5584);
\draw [color=c, fill=c] (12.0697,10.4526) rectangle (12.1095,10.5584);
\draw [color=c, fill=c] (12.1095,10.4526) rectangle (12.1493,10.5584);
\draw [color=c, fill=c] (12.1493,10.4526) rectangle (12.1891,10.5584);
\draw [color=c, fill=c] (12.1891,10.4526) rectangle (12.2289,10.5584);
\draw [color=c, fill=c] (12.2289,10.4526) rectangle (12.2687,10.5584);
\draw [color=c, fill=c] (12.2687,10.4526) rectangle (12.3085,10.5584);
\draw [color=c, fill=c] (12.3085,10.4526) rectangle (12.3483,10.5584);
\draw [color=c, fill=c] (12.3483,10.4526) rectangle (12.3881,10.5584);
\draw [color=c, fill=c] (12.3881,10.4526) rectangle (12.4279,10.5584);
\draw [color=c, fill=c] (12.4279,10.4526) rectangle (12.4677,10.5584);
\draw [color=c, fill=c] (12.4677,10.4526) rectangle (12.5075,10.5584);
\draw [color=c, fill=c] (12.5075,10.4526) rectangle (12.5473,10.5584);
\draw [color=c, fill=c] (12.5473,10.4526) rectangle (12.5871,10.5584);
\draw [color=c, fill=c] (12.5871,10.4526) rectangle (12.6269,10.5584);
\draw [color=c, fill=c] (12.6269,10.4526) rectangle (12.6667,10.5584);
\draw [color=c, fill=c] (12.6667,10.4526) rectangle (12.7065,10.5584);
\draw [color=c, fill=c] (12.7065,10.4526) rectangle (12.7463,10.5584);
\draw [color=c, fill=c] (12.7463,10.4526) rectangle (12.7861,10.5584);
\draw [color=c, fill=c] (12.7861,10.4526) rectangle (12.8259,10.5584);
\draw [color=c, fill=c] (12.8259,10.4526) rectangle (12.8657,10.5584);
\draw [color=c, fill=c] (12.8657,10.4526) rectangle (12.9055,10.5584);
\draw [color=c, fill=c] (12.9055,10.4526) rectangle (12.9453,10.5584);
\draw [color=c, fill=c] (12.9453,10.4526) rectangle (12.9851,10.5584);
\draw [color=c, fill=c] (12.9851,10.4526) rectangle (13.0249,10.5584);
\draw [color=c, fill=c] (13.0249,10.4526) rectangle (13.0647,10.5584);
\draw [color=c, fill=c] (13.0647,10.4526) rectangle (13.1045,10.5584);
\draw [color=c, fill=c] (13.1045,10.4526) rectangle (13.1443,10.5584);
\draw [color=c, fill=c] (13.1443,10.4526) rectangle (13.1841,10.5584);
\draw [color=c, fill=c] (13.1841,10.4526) rectangle (13.2239,10.5584);
\draw [color=c, fill=c] (13.2239,10.4526) rectangle (13.2637,10.5584);
\draw [color=c, fill=c] (13.2637,10.4526) rectangle (13.3035,10.5584);
\draw [color=c, fill=c] (13.3035,10.4526) rectangle (13.3433,10.5584);
\draw [color=c, fill=c] (13.3433,10.4526) rectangle (13.3831,10.5584);
\draw [color=c, fill=c] (13.3831,10.4526) rectangle (13.4229,10.5584);
\draw [color=c, fill=c] (13.4229,10.4526) rectangle (13.4627,10.5584);
\draw [color=c, fill=c] (13.4627,10.4526) rectangle (13.5025,10.5584);
\draw [color=c, fill=c] (13.5025,10.4526) rectangle (13.5423,10.5584);
\draw [color=c, fill=c] (13.5423,10.4526) rectangle (13.5821,10.5584);
\draw [color=c, fill=c] (13.5821,10.4526) rectangle (13.6219,10.5584);
\draw [color=c, fill=c] (13.6219,10.4526) rectangle (13.6617,10.5584);
\draw [color=c, fill=c] (13.6617,10.4526) rectangle (13.7015,10.5584);
\draw [color=c, fill=c] (13.7015,10.4526) rectangle (13.7413,10.5584);
\draw [color=c, fill=c] (13.7413,10.4526) rectangle (13.7811,10.5584);
\draw [color=c, fill=c] (13.7811,10.4526) rectangle (13.8209,10.5584);
\draw [color=c, fill=c] (13.8209,10.4526) rectangle (13.8607,10.5584);
\draw [color=c, fill=c] (13.8607,10.4526) rectangle (13.9005,10.5584);
\draw [color=c, fill=c] (13.9005,10.4526) rectangle (13.9403,10.5584);
\draw [color=c, fill=c] (13.9403,10.4526) rectangle (13.9801,10.5584);
\draw [color=c, fill=c] (13.9801,10.4526) rectangle (14.0199,10.5584);
\draw [color=c, fill=c] (14.0199,10.4526) rectangle (14.0597,10.5584);
\draw [color=c, fill=c] (14.0597,10.4526) rectangle (14.0995,10.5584);
\definecolor{c}{rgb}{0,0.733333,1};
\draw [color=c, fill=c] (14.0995,10.4526) rectangle (14.1393,10.5584);
\draw [color=c, fill=c] (14.1393,10.4526) rectangle (14.1791,10.5584);
\draw [color=c, fill=c] (14.1791,10.4526) rectangle (14.2189,10.5584);
\draw [color=c, fill=c] (14.2189,10.4526) rectangle (14.2587,10.5584);
\draw [color=c, fill=c] (14.2587,10.4526) rectangle (14.2985,10.5584);
\draw [color=c, fill=c] (14.2985,10.4526) rectangle (14.3383,10.5584);
\draw [color=c, fill=c] (14.3383,10.4526) rectangle (14.3781,10.5584);
\draw [color=c, fill=c] (14.3781,10.4526) rectangle (14.4179,10.5584);
\draw [color=c, fill=c] (14.4179,10.4526) rectangle (14.4577,10.5584);
\draw [color=c, fill=c] (14.4577,10.4526) rectangle (14.4975,10.5584);
\draw [color=c, fill=c] (14.4975,10.4526) rectangle (14.5373,10.5584);
\draw [color=c, fill=c] (14.5373,10.4526) rectangle (14.5771,10.5584);
\draw [color=c, fill=c] (14.5771,10.4526) rectangle (14.6169,10.5584);
\draw [color=c, fill=c] (14.6169,10.4526) rectangle (14.6567,10.5584);
\draw [color=c, fill=c] (14.6567,10.4526) rectangle (14.6965,10.5584);
\draw [color=c, fill=c] (14.6965,10.4526) rectangle (14.7363,10.5584);
\draw [color=c, fill=c] (14.7363,10.4526) rectangle (14.7761,10.5584);
\draw [color=c, fill=c] (14.7761,10.4526) rectangle (14.8159,10.5584);
\draw [color=c, fill=c] (14.8159,10.4526) rectangle (14.8557,10.5584);
\draw [color=c, fill=c] (14.8557,10.4526) rectangle (14.8955,10.5584);
\draw [color=c, fill=c] (14.8955,10.4526) rectangle (14.9353,10.5584);
\draw [color=c, fill=c] (14.9353,10.4526) rectangle (14.9751,10.5584);
\draw [color=c, fill=c] (14.9751,10.4526) rectangle (15.0149,10.5584);
\draw [color=c, fill=c] (15.0149,10.4526) rectangle (15.0547,10.5584);
\draw [color=c, fill=c] (15.0547,10.4526) rectangle (15.0945,10.5584);
\draw [color=c, fill=c] (15.0945,10.4526) rectangle (15.1343,10.5584);
\draw [color=c, fill=c] (15.1343,10.4526) rectangle (15.1741,10.5584);
\draw [color=c, fill=c] (15.1741,10.4526) rectangle (15.2139,10.5584);
\draw [color=c, fill=c] (15.2139,10.4526) rectangle (15.2537,10.5584);
\draw [color=c, fill=c] (15.2537,10.4526) rectangle (15.2935,10.5584);
\draw [color=c, fill=c] (15.2935,10.4526) rectangle (15.3333,10.5584);
\draw [color=c, fill=c] (15.3333,10.4526) rectangle (15.3731,10.5584);
\draw [color=c, fill=c] (15.3731,10.4526) rectangle (15.4129,10.5584);
\draw [color=c, fill=c] (15.4129,10.4526) rectangle (15.4527,10.5584);
\draw [color=c, fill=c] (15.4527,10.4526) rectangle (15.4925,10.5584);
\draw [color=c, fill=c] (15.4925,10.4526) rectangle (15.5323,10.5584);
\draw [color=c, fill=c] (15.5323,10.4526) rectangle (15.5721,10.5584);
\draw [color=c, fill=c] (15.5721,10.4526) rectangle (15.6119,10.5584);
\draw [color=c, fill=c] (15.6119,10.4526) rectangle (15.6517,10.5584);
\draw [color=c, fill=c] (15.6517,10.4526) rectangle (15.6915,10.5584);
\draw [color=c, fill=c] (15.6915,10.4526) rectangle (15.7313,10.5584);
\draw [color=c, fill=c] (15.7313,10.4526) rectangle (15.7711,10.5584);
\draw [color=c, fill=c] (15.7711,10.4526) rectangle (15.8109,10.5584);
\draw [color=c, fill=c] (15.8109,10.4526) rectangle (15.8507,10.5584);
\draw [color=c, fill=c] (15.8507,10.4526) rectangle (15.8905,10.5584);
\draw [color=c, fill=c] (15.8905,10.4526) rectangle (15.9303,10.5584);
\draw [color=c, fill=c] (15.9303,10.4526) rectangle (15.9701,10.5584);
\draw [color=c, fill=c] (15.9701,10.4526) rectangle (16.01,10.5584);
\draw [color=c, fill=c] (16.01,10.4526) rectangle (16.0498,10.5584);
\draw [color=c, fill=c] (16.0498,10.4526) rectangle (16.0896,10.5584);
\draw [color=c, fill=c] (16.0896,10.4526) rectangle (16.1294,10.5584);
\draw [color=c, fill=c] (16.1294,10.4526) rectangle (16.1692,10.5584);
\draw [color=c, fill=c] (16.1692,10.4526) rectangle (16.209,10.5584);
\draw [color=c, fill=c] (16.209,10.4526) rectangle (16.2488,10.5584);
\draw [color=c, fill=c] (16.2488,10.4526) rectangle (16.2886,10.5584);
\draw [color=c, fill=c] (16.2886,10.4526) rectangle (16.3284,10.5584);
\draw [color=c, fill=c] (16.3284,10.4526) rectangle (16.3682,10.5584);
\draw [color=c, fill=c] (16.3682,10.4526) rectangle (16.408,10.5584);
\draw [color=c, fill=c] (16.408,10.4526) rectangle (16.4478,10.5584);
\draw [color=c, fill=c] (16.4478,10.4526) rectangle (16.4876,10.5584);
\draw [color=c, fill=c] (16.4876,10.4526) rectangle (16.5274,10.5584);
\draw [color=c, fill=c] (16.5274,10.4526) rectangle (16.5672,10.5584);
\draw [color=c, fill=c] (16.5672,10.4526) rectangle (16.607,10.5584);
\draw [color=c, fill=c] (16.607,10.4526) rectangle (16.6468,10.5584);
\draw [color=c, fill=c] (16.6468,10.4526) rectangle (16.6866,10.5584);
\draw [color=c, fill=c] (16.6866,10.4526) rectangle (16.7264,10.5584);
\draw [color=c, fill=c] (16.7264,10.4526) rectangle (16.7662,10.5584);
\draw [color=c, fill=c] (16.7662,10.4526) rectangle (16.806,10.5584);
\draw [color=c, fill=c] (16.806,10.4526) rectangle (16.8458,10.5584);
\draw [color=c, fill=c] (16.8458,10.4526) rectangle (16.8856,10.5584);
\draw [color=c, fill=c] (16.8856,10.4526) rectangle (16.9254,10.5584);
\draw [color=c, fill=c] (16.9254,10.4526) rectangle (16.9652,10.5584);
\draw [color=c, fill=c] (16.9652,10.4526) rectangle (17.005,10.5584);
\draw [color=c, fill=c] (17.005,10.4526) rectangle (17.0448,10.5584);
\draw [color=c, fill=c] (17.0448,10.4526) rectangle (17.0846,10.5584);
\draw [color=c, fill=c] (17.0846,10.4526) rectangle (17.1244,10.5584);
\draw [color=c, fill=c] (17.1244,10.4526) rectangle (17.1642,10.5584);
\draw [color=c, fill=c] (17.1642,10.4526) rectangle (17.204,10.5584);
\draw [color=c, fill=c] (17.204,10.4526) rectangle (17.2438,10.5584);
\draw [color=c, fill=c] (17.2438,10.4526) rectangle (17.2836,10.5584);
\draw [color=c, fill=c] (17.2836,10.4526) rectangle (17.3234,10.5584);
\draw [color=c, fill=c] (17.3234,10.4526) rectangle (17.3632,10.5584);
\draw [color=c, fill=c] (17.3632,10.4526) rectangle (17.403,10.5584);
\draw [color=c, fill=c] (17.403,10.4526) rectangle (17.4428,10.5584);
\draw [color=c, fill=c] (17.4428,10.4526) rectangle (17.4826,10.5584);
\draw [color=c, fill=c] (17.4826,10.4526) rectangle (17.5224,10.5584);
\draw [color=c, fill=c] (17.5224,10.4526) rectangle (17.5622,10.5584);
\draw [color=c, fill=c] (17.5622,10.4526) rectangle (17.602,10.5584);
\draw [color=c, fill=c] (17.602,10.4526) rectangle (17.6418,10.5584);
\draw [color=c, fill=c] (17.6418,10.4526) rectangle (17.6816,10.5584);
\draw [color=c, fill=c] (17.6816,10.4526) rectangle (17.7214,10.5584);
\draw [color=c, fill=c] (17.7214,10.4526) rectangle (17.7612,10.5584);
\draw [color=c, fill=c] (17.7612,10.4526) rectangle (17.801,10.5584);
\draw [color=c, fill=c] (17.801,10.4526) rectangle (17.8408,10.5584);
\draw [color=c, fill=c] (17.8408,10.4526) rectangle (17.8806,10.5584);
\draw [color=c, fill=c] (17.8806,10.4526) rectangle (17.9204,10.5584);
\draw [color=c, fill=c] (17.9204,10.4526) rectangle (17.9602,10.5584);
\draw [color=c, fill=c] (17.9602,10.4526) rectangle (18,10.5584);
\definecolor{c}{rgb}{0.2,0,1};
\draw [color=c, fill=c] (2,10.5584) rectangle (2.0398,10.6643);
\draw [color=c, fill=c] (2.0398,10.5584) rectangle (2.0796,10.6643);
\draw [color=c, fill=c] (2.0796,10.5584) rectangle (2.1194,10.6643);
\draw [color=c, fill=c] (2.1194,10.5584) rectangle (2.1592,10.6643);
\draw [color=c, fill=c] (2.1592,10.5584) rectangle (2.19901,10.6643);
\draw [color=c, fill=c] (2.19901,10.5584) rectangle (2.23881,10.6643);
\draw [color=c, fill=c] (2.23881,10.5584) rectangle (2.27861,10.6643);
\draw [color=c, fill=c] (2.27861,10.5584) rectangle (2.31841,10.6643);
\draw [color=c, fill=c] (2.31841,10.5584) rectangle (2.35821,10.6643);
\draw [color=c, fill=c] (2.35821,10.5584) rectangle (2.39801,10.6643);
\draw [color=c, fill=c] (2.39801,10.5584) rectangle (2.43781,10.6643);
\draw [color=c, fill=c] (2.43781,10.5584) rectangle (2.47761,10.6643);
\draw [color=c, fill=c] (2.47761,10.5584) rectangle (2.51741,10.6643);
\draw [color=c, fill=c] (2.51741,10.5584) rectangle (2.55721,10.6643);
\draw [color=c, fill=c] (2.55721,10.5584) rectangle (2.59702,10.6643);
\draw [color=c, fill=c] (2.59702,10.5584) rectangle (2.63682,10.6643);
\draw [color=c, fill=c] (2.63682,10.5584) rectangle (2.67662,10.6643);
\draw [color=c, fill=c] (2.67662,10.5584) rectangle (2.71642,10.6643);
\draw [color=c, fill=c] (2.71642,10.5584) rectangle (2.75622,10.6643);
\draw [color=c, fill=c] (2.75622,10.5584) rectangle (2.79602,10.6643);
\draw [color=c, fill=c] (2.79602,10.5584) rectangle (2.83582,10.6643);
\draw [color=c, fill=c] (2.83582,10.5584) rectangle (2.87562,10.6643);
\draw [color=c, fill=c] (2.87562,10.5584) rectangle (2.91542,10.6643);
\draw [color=c, fill=c] (2.91542,10.5584) rectangle (2.95522,10.6643);
\draw [color=c, fill=c] (2.95522,10.5584) rectangle (2.99502,10.6643);
\draw [color=c, fill=c] (2.99502,10.5584) rectangle (3.03483,10.6643);
\draw [color=c, fill=c] (3.03483,10.5584) rectangle (3.07463,10.6643);
\draw [color=c, fill=c] (3.07463,10.5584) rectangle (3.11443,10.6643);
\draw [color=c, fill=c] (3.11443,10.5584) rectangle (3.15423,10.6643);
\draw [color=c, fill=c] (3.15423,10.5584) rectangle (3.19403,10.6643);
\draw [color=c, fill=c] (3.19403,10.5584) rectangle (3.23383,10.6643);
\draw [color=c, fill=c] (3.23383,10.5584) rectangle (3.27363,10.6643);
\draw [color=c, fill=c] (3.27363,10.5584) rectangle (3.31343,10.6643);
\draw [color=c, fill=c] (3.31343,10.5584) rectangle (3.35323,10.6643);
\draw [color=c, fill=c] (3.35323,10.5584) rectangle (3.39303,10.6643);
\draw [color=c, fill=c] (3.39303,10.5584) rectangle (3.43284,10.6643);
\draw [color=c, fill=c] (3.43284,10.5584) rectangle (3.47264,10.6643);
\draw [color=c, fill=c] (3.47264,10.5584) rectangle (3.51244,10.6643);
\draw [color=c, fill=c] (3.51244,10.5584) rectangle (3.55224,10.6643);
\draw [color=c, fill=c] (3.55224,10.5584) rectangle (3.59204,10.6643);
\draw [color=c, fill=c] (3.59204,10.5584) rectangle (3.63184,10.6643);
\draw [color=c, fill=c] (3.63184,10.5584) rectangle (3.67164,10.6643);
\draw [color=c, fill=c] (3.67164,10.5584) rectangle (3.71144,10.6643);
\draw [color=c, fill=c] (3.71144,10.5584) rectangle (3.75124,10.6643);
\draw [color=c, fill=c] (3.75124,10.5584) rectangle (3.79104,10.6643);
\draw [color=c, fill=c] (3.79104,10.5584) rectangle (3.83085,10.6643);
\draw [color=c, fill=c] (3.83085,10.5584) rectangle (3.87065,10.6643);
\draw [color=c, fill=c] (3.87065,10.5584) rectangle (3.91045,10.6643);
\draw [color=c, fill=c] (3.91045,10.5584) rectangle (3.95025,10.6643);
\draw [color=c, fill=c] (3.95025,10.5584) rectangle (3.99005,10.6643);
\draw [color=c, fill=c] (3.99005,10.5584) rectangle (4.02985,10.6643);
\draw [color=c, fill=c] (4.02985,10.5584) rectangle (4.06965,10.6643);
\draw [color=c, fill=c] (4.06965,10.5584) rectangle (4.10945,10.6643);
\draw [color=c, fill=c] (4.10945,10.5584) rectangle (4.14925,10.6643);
\draw [color=c, fill=c] (4.14925,10.5584) rectangle (4.18905,10.6643);
\draw [color=c, fill=c] (4.18905,10.5584) rectangle (4.22886,10.6643);
\draw [color=c, fill=c] (4.22886,10.5584) rectangle (4.26866,10.6643);
\draw [color=c, fill=c] (4.26866,10.5584) rectangle (4.30846,10.6643);
\draw [color=c, fill=c] (4.30846,10.5584) rectangle (4.34826,10.6643);
\draw [color=c, fill=c] (4.34826,10.5584) rectangle (4.38806,10.6643);
\draw [color=c, fill=c] (4.38806,10.5584) rectangle (4.42786,10.6643);
\draw [color=c, fill=c] (4.42786,10.5584) rectangle (4.46766,10.6643);
\draw [color=c, fill=c] (4.46766,10.5584) rectangle (4.50746,10.6643);
\draw [color=c, fill=c] (4.50746,10.5584) rectangle (4.54726,10.6643);
\draw [color=c, fill=c] (4.54726,10.5584) rectangle (4.58706,10.6643);
\draw [color=c, fill=c] (4.58706,10.5584) rectangle (4.62687,10.6643);
\draw [color=c, fill=c] (4.62687,10.5584) rectangle (4.66667,10.6643);
\draw [color=c, fill=c] (4.66667,10.5584) rectangle (4.70647,10.6643);
\draw [color=c, fill=c] (4.70647,10.5584) rectangle (4.74627,10.6643);
\draw [color=c, fill=c] (4.74627,10.5584) rectangle (4.78607,10.6643);
\draw [color=c, fill=c] (4.78607,10.5584) rectangle (4.82587,10.6643);
\draw [color=c, fill=c] (4.82587,10.5584) rectangle (4.86567,10.6643);
\draw [color=c, fill=c] (4.86567,10.5584) rectangle (4.90547,10.6643);
\draw [color=c, fill=c] (4.90547,10.5584) rectangle (4.94527,10.6643);
\draw [color=c, fill=c] (4.94527,10.5584) rectangle (4.98507,10.6643);
\draw [color=c, fill=c] (4.98507,10.5584) rectangle (5.02488,10.6643);
\draw [color=c, fill=c] (5.02488,10.5584) rectangle (5.06468,10.6643);
\draw [color=c, fill=c] (5.06468,10.5584) rectangle (5.10448,10.6643);
\draw [color=c, fill=c] (5.10448,10.5584) rectangle (5.14428,10.6643);
\draw [color=c, fill=c] (5.14428,10.5584) rectangle (5.18408,10.6643);
\draw [color=c, fill=c] (5.18408,10.5584) rectangle (5.22388,10.6643);
\draw [color=c, fill=c] (5.22388,10.5584) rectangle (5.26368,10.6643);
\draw [color=c, fill=c] (5.26368,10.5584) rectangle (5.30348,10.6643);
\draw [color=c, fill=c] (5.30348,10.5584) rectangle (5.34328,10.6643);
\draw [color=c, fill=c] (5.34328,10.5584) rectangle (5.38308,10.6643);
\draw [color=c, fill=c] (5.38308,10.5584) rectangle (5.42289,10.6643);
\draw [color=c, fill=c] (5.42289,10.5584) rectangle (5.46269,10.6643);
\draw [color=c, fill=c] (5.46269,10.5584) rectangle (5.50249,10.6643);
\draw [color=c, fill=c] (5.50249,10.5584) rectangle (5.54229,10.6643);
\draw [color=c, fill=c] (5.54229,10.5584) rectangle (5.58209,10.6643);
\draw [color=c, fill=c] (5.58209,10.5584) rectangle (5.62189,10.6643);
\draw [color=c, fill=c] (5.62189,10.5584) rectangle (5.66169,10.6643);
\draw [color=c, fill=c] (5.66169,10.5584) rectangle (5.70149,10.6643);
\draw [color=c, fill=c] (5.70149,10.5584) rectangle (5.74129,10.6643);
\draw [color=c, fill=c] (5.74129,10.5584) rectangle (5.78109,10.6643);
\draw [color=c, fill=c] (5.78109,10.5584) rectangle (5.8209,10.6643);
\draw [color=c, fill=c] (5.8209,10.5584) rectangle (5.8607,10.6643);
\draw [color=c, fill=c] (5.8607,10.5584) rectangle (5.9005,10.6643);
\draw [color=c, fill=c] (5.9005,10.5584) rectangle (5.9403,10.6643);
\draw [color=c, fill=c] (5.9403,10.5584) rectangle (5.9801,10.6643);
\draw [color=c, fill=c] (5.9801,10.5584) rectangle (6.0199,10.6643);
\draw [color=c, fill=c] (6.0199,10.5584) rectangle (6.0597,10.6643);
\draw [color=c, fill=c] (6.0597,10.5584) rectangle (6.0995,10.6643);
\draw [color=c, fill=c] (6.0995,10.5584) rectangle (6.1393,10.6643);
\draw [color=c, fill=c] (6.1393,10.5584) rectangle (6.1791,10.6643);
\draw [color=c, fill=c] (6.1791,10.5584) rectangle (6.21891,10.6643);
\draw [color=c, fill=c] (6.21891,10.5584) rectangle (6.25871,10.6643);
\draw [color=c, fill=c] (6.25871,10.5584) rectangle (6.29851,10.6643);
\draw [color=c, fill=c] (6.29851,10.5584) rectangle (6.33831,10.6643);
\draw [color=c, fill=c] (6.33831,10.5584) rectangle (6.37811,10.6643);
\draw [color=c, fill=c] (6.37811,10.5584) rectangle (6.41791,10.6643);
\draw [color=c, fill=c] (6.41791,10.5584) rectangle (6.45771,10.6643);
\draw [color=c, fill=c] (6.45771,10.5584) rectangle (6.49751,10.6643);
\draw [color=c, fill=c] (6.49751,10.5584) rectangle (6.53731,10.6643);
\draw [color=c, fill=c] (6.53731,10.5584) rectangle (6.57711,10.6643);
\draw [color=c, fill=c] (6.57711,10.5584) rectangle (6.61692,10.6643);
\draw [color=c, fill=c] (6.61692,10.5584) rectangle (6.65672,10.6643);
\draw [color=c, fill=c] (6.65672,10.5584) rectangle (6.69652,10.6643);
\draw [color=c, fill=c] (6.69652,10.5584) rectangle (6.73632,10.6643);
\draw [color=c, fill=c] (6.73632,10.5584) rectangle (6.77612,10.6643);
\draw [color=c, fill=c] (6.77612,10.5584) rectangle (6.81592,10.6643);
\draw [color=c, fill=c] (6.81592,10.5584) rectangle (6.85572,10.6643);
\draw [color=c, fill=c] (6.85572,10.5584) rectangle (6.89552,10.6643);
\draw [color=c, fill=c] (6.89552,10.5584) rectangle (6.93532,10.6643);
\draw [color=c, fill=c] (6.93532,10.5584) rectangle (6.97512,10.6643);
\draw [color=c, fill=c] (6.97512,10.5584) rectangle (7.01493,10.6643);
\draw [color=c, fill=c] (7.01493,10.5584) rectangle (7.05473,10.6643);
\draw [color=c, fill=c] (7.05473,10.5584) rectangle (7.09453,10.6643);
\draw [color=c, fill=c] (7.09453,10.5584) rectangle (7.13433,10.6643);
\draw [color=c, fill=c] (7.13433,10.5584) rectangle (7.17413,10.6643);
\draw [color=c, fill=c] (7.17413,10.5584) rectangle (7.21393,10.6643);
\draw [color=c, fill=c] (7.21393,10.5584) rectangle (7.25373,10.6643);
\draw [color=c, fill=c] (7.25373,10.5584) rectangle (7.29353,10.6643);
\draw [color=c, fill=c] (7.29353,10.5584) rectangle (7.33333,10.6643);
\draw [color=c, fill=c] (7.33333,10.5584) rectangle (7.37313,10.6643);
\draw [color=c, fill=c] (7.37313,10.5584) rectangle (7.41294,10.6643);
\draw [color=c, fill=c] (7.41294,10.5584) rectangle (7.45274,10.6643);
\draw [color=c, fill=c] (7.45274,10.5584) rectangle (7.49254,10.6643);
\draw [color=c, fill=c] (7.49254,10.5584) rectangle (7.53234,10.6643);
\draw [color=c, fill=c] (7.53234,10.5584) rectangle (7.57214,10.6643);
\draw [color=c, fill=c] (7.57214,10.5584) rectangle (7.61194,10.6643);
\draw [color=c, fill=c] (7.61194,10.5584) rectangle (7.65174,10.6643);
\draw [color=c, fill=c] (7.65174,10.5584) rectangle (7.69154,10.6643);
\draw [color=c, fill=c] (7.69154,10.5584) rectangle (7.73134,10.6643);
\draw [color=c, fill=c] (7.73134,10.5584) rectangle (7.77114,10.6643);
\draw [color=c, fill=c] (7.77114,10.5584) rectangle (7.81095,10.6643);
\draw [color=c, fill=c] (7.81095,10.5584) rectangle (7.85075,10.6643);
\definecolor{c}{rgb}{0,0.0800001,1};
\draw [color=c, fill=c] (7.85075,10.5584) rectangle (7.89055,10.6643);
\draw [color=c, fill=c] (7.89055,10.5584) rectangle (7.93035,10.6643);
\draw [color=c, fill=c] (7.93035,10.5584) rectangle (7.97015,10.6643);
\draw [color=c, fill=c] (7.97015,10.5584) rectangle (8.00995,10.6643);
\draw [color=c, fill=c] (8.00995,10.5584) rectangle (8.04975,10.6643);
\draw [color=c, fill=c] (8.04975,10.5584) rectangle (8.08955,10.6643);
\draw [color=c, fill=c] (8.08955,10.5584) rectangle (8.12935,10.6643);
\draw [color=c, fill=c] (8.12935,10.5584) rectangle (8.16915,10.6643);
\draw [color=c, fill=c] (8.16915,10.5584) rectangle (8.20895,10.6643);
\draw [color=c, fill=c] (8.20895,10.5584) rectangle (8.24876,10.6643);
\draw [color=c, fill=c] (8.24876,10.5584) rectangle (8.28856,10.6643);
\draw [color=c, fill=c] (8.28856,10.5584) rectangle (8.32836,10.6643);
\draw [color=c, fill=c] (8.32836,10.5584) rectangle (8.36816,10.6643);
\draw [color=c, fill=c] (8.36816,10.5584) rectangle (8.40796,10.6643);
\draw [color=c, fill=c] (8.40796,10.5584) rectangle (8.44776,10.6643);
\draw [color=c, fill=c] (8.44776,10.5584) rectangle (8.48756,10.6643);
\draw [color=c, fill=c] (8.48756,10.5584) rectangle (8.52736,10.6643);
\draw [color=c, fill=c] (8.52736,10.5584) rectangle (8.56716,10.6643);
\draw [color=c, fill=c] (8.56716,10.5584) rectangle (8.60697,10.6643);
\draw [color=c, fill=c] (8.60697,10.5584) rectangle (8.64677,10.6643);
\draw [color=c, fill=c] (8.64677,10.5584) rectangle (8.68657,10.6643);
\draw [color=c, fill=c] (8.68657,10.5584) rectangle (8.72637,10.6643);
\draw [color=c, fill=c] (8.72637,10.5584) rectangle (8.76617,10.6643);
\draw [color=c, fill=c] (8.76617,10.5584) rectangle (8.80597,10.6643);
\draw [color=c, fill=c] (8.80597,10.5584) rectangle (8.84577,10.6643);
\draw [color=c, fill=c] (8.84577,10.5584) rectangle (8.88557,10.6643);
\draw [color=c, fill=c] (8.88557,10.5584) rectangle (8.92537,10.6643);
\draw [color=c, fill=c] (8.92537,10.5584) rectangle (8.96517,10.6643);
\draw [color=c, fill=c] (8.96517,10.5584) rectangle (9.00498,10.6643);
\draw [color=c, fill=c] (9.00498,10.5584) rectangle (9.04478,10.6643);
\draw [color=c, fill=c] (9.04478,10.5584) rectangle (9.08458,10.6643);
\draw [color=c, fill=c] (9.08458,10.5584) rectangle (9.12438,10.6643);
\draw [color=c, fill=c] (9.12438,10.5584) rectangle (9.16418,10.6643);
\draw [color=c, fill=c] (9.16418,10.5584) rectangle (9.20398,10.6643);
\draw [color=c, fill=c] (9.20398,10.5584) rectangle (9.24378,10.6643);
\draw [color=c, fill=c] (9.24378,10.5584) rectangle (9.28358,10.6643);
\draw [color=c, fill=c] (9.28358,10.5584) rectangle (9.32338,10.6643);
\draw [color=c, fill=c] (9.32338,10.5584) rectangle (9.36318,10.6643);
\draw [color=c, fill=c] (9.36318,10.5584) rectangle (9.40298,10.6643);
\draw [color=c, fill=c] (9.40298,10.5584) rectangle (9.44279,10.6643);
\draw [color=c, fill=c] (9.44279,10.5584) rectangle (9.48259,10.6643);
\draw [color=c, fill=c] (9.48259,10.5584) rectangle (9.52239,10.6643);
\draw [color=c, fill=c] (9.52239,10.5584) rectangle (9.56219,10.6643);
\definecolor{c}{rgb}{0,0.266667,1};
\draw [color=c, fill=c] (9.56219,10.5584) rectangle (9.60199,10.6643);
\draw [color=c, fill=c] (9.60199,10.5584) rectangle (9.64179,10.6643);
\draw [color=c, fill=c] (9.64179,10.5584) rectangle (9.68159,10.6643);
\draw [color=c, fill=c] (9.68159,10.5584) rectangle (9.72139,10.6643);
\draw [color=c, fill=c] (9.72139,10.5584) rectangle (9.76119,10.6643);
\draw [color=c, fill=c] (9.76119,10.5584) rectangle (9.80099,10.6643);
\draw [color=c, fill=c] (9.80099,10.5584) rectangle (9.8408,10.6643);
\draw [color=c, fill=c] (9.8408,10.5584) rectangle (9.8806,10.6643);
\draw [color=c, fill=c] (9.8806,10.5584) rectangle (9.9204,10.6643);
\draw [color=c, fill=c] (9.9204,10.5584) rectangle (9.9602,10.6643);
\draw [color=c, fill=c] (9.9602,10.5584) rectangle (10,10.6643);
\draw [color=c, fill=c] (10,10.5584) rectangle (10.0398,10.6643);
\draw [color=c, fill=c] (10.0398,10.5584) rectangle (10.0796,10.6643);
\draw [color=c, fill=c] (10.0796,10.5584) rectangle (10.1194,10.6643);
\draw [color=c, fill=c] (10.1194,10.5584) rectangle (10.1592,10.6643);
\draw [color=c, fill=c] (10.1592,10.5584) rectangle (10.199,10.6643);
\draw [color=c, fill=c] (10.199,10.5584) rectangle (10.2388,10.6643);
\draw [color=c, fill=c] (10.2388,10.5584) rectangle (10.2786,10.6643);
\draw [color=c, fill=c] (10.2786,10.5584) rectangle (10.3184,10.6643);
\draw [color=c, fill=c] (10.3184,10.5584) rectangle (10.3582,10.6643);
\draw [color=c, fill=c] (10.3582,10.5584) rectangle (10.398,10.6643);
\draw [color=c, fill=c] (10.398,10.5584) rectangle (10.4378,10.6643);
\draw [color=c, fill=c] (10.4378,10.5584) rectangle (10.4776,10.6643);
\draw [color=c, fill=c] (10.4776,10.5584) rectangle (10.5174,10.6643);
\draw [color=c, fill=c] (10.5174,10.5584) rectangle (10.5572,10.6643);
\draw [color=c, fill=c] (10.5572,10.5584) rectangle (10.597,10.6643);
\draw [color=c, fill=c] (10.597,10.5584) rectangle (10.6368,10.6643);
\draw [color=c, fill=c] (10.6368,10.5584) rectangle (10.6766,10.6643);
\draw [color=c, fill=c] (10.6766,10.5584) rectangle (10.7164,10.6643);
\draw [color=c, fill=c] (10.7164,10.5584) rectangle (10.7562,10.6643);
\draw [color=c, fill=c] (10.7562,10.5584) rectangle (10.796,10.6643);
\draw [color=c, fill=c] (10.796,10.5584) rectangle (10.8358,10.6643);
\draw [color=c, fill=c] (10.8358,10.5584) rectangle (10.8756,10.6643);
\definecolor{c}{rgb}{0,0.546666,1};
\draw [color=c, fill=c] (10.8756,10.5584) rectangle (10.9154,10.6643);
\draw [color=c, fill=c] (10.9154,10.5584) rectangle (10.9552,10.6643);
\draw [color=c, fill=c] (10.9552,10.5584) rectangle (10.995,10.6643);
\draw [color=c, fill=c] (10.995,10.5584) rectangle (11.0348,10.6643);
\draw [color=c, fill=c] (11.0348,10.5584) rectangle (11.0746,10.6643);
\draw [color=c, fill=c] (11.0746,10.5584) rectangle (11.1144,10.6643);
\draw [color=c, fill=c] (11.1144,10.5584) rectangle (11.1542,10.6643);
\draw [color=c, fill=c] (11.1542,10.5584) rectangle (11.194,10.6643);
\draw [color=c, fill=c] (11.194,10.5584) rectangle (11.2338,10.6643);
\draw [color=c, fill=c] (11.2338,10.5584) rectangle (11.2736,10.6643);
\draw [color=c, fill=c] (11.2736,10.5584) rectangle (11.3134,10.6643);
\draw [color=c, fill=c] (11.3134,10.5584) rectangle (11.3532,10.6643);
\draw [color=c, fill=c] (11.3532,10.5584) rectangle (11.393,10.6643);
\draw [color=c, fill=c] (11.393,10.5584) rectangle (11.4328,10.6643);
\draw [color=c, fill=c] (11.4328,10.5584) rectangle (11.4726,10.6643);
\draw [color=c, fill=c] (11.4726,10.5584) rectangle (11.5124,10.6643);
\draw [color=c, fill=c] (11.5124,10.5584) rectangle (11.5522,10.6643);
\draw [color=c, fill=c] (11.5522,10.5584) rectangle (11.592,10.6643);
\draw [color=c, fill=c] (11.592,10.5584) rectangle (11.6318,10.6643);
\draw [color=c, fill=c] (11.6318,10.5584) rectangle (11.6716,10.6643);
\draw [color=c, fill=c] (11.6716,10.5584) rectangle (11.7114,10.6643);
\draw [color=c, fill=c] (11.7114,10.5584) rectangle (11.7512,10.6643);
\draw [color=c, fill=c] (11.7512,10.5584) rectangle (11.791,10.6643);
\draw [color=c, fill=c] (11.791,10.5584) rectangle (11.8308,10.6643);
\draw [color=c, fill=c] (11.8308,10.5584) rectangle (11.8706,10.6643);
\draw [color=c, fill=c] (11.8706,10.5584) rectangle (11.9104,10.6643);
\draw [color=c, fill=c] (11.9104,10.5584) rectangle (11.9502,10.6643);
\draw [color=c, fill=c] (11.9502,10.5584) rectangle (11.99,10.6643);
\draw [color=c, fill=c] (11.99,10.5584) rectangle (12.0299,10.6643);
\draw [color=c, fill=c] (12.0299,10.5584) rectangle (12.0697,10.6643);
\draw [color=c, fill=c] (12.0697,10.5584) rectangle (12.1095,10.6643);
\draw [color=c, fill=c] (12.1095,10.5584) rectangle (12.1493,10.6643);
\draw [color=c, fill=c] (12.1493,10.5584) rectangle (12.1891,10.6643);
\draw [color=c, fill=c] (12.1891,10.5584) rectangle (12.2289,10.6643);
\draw [color=c, fill=c] (12.2289,10.5584) rectangle (12.2687,10.6643);
\draw [color=c, fill=c] (12.2687,10.5584) rectangle (12.3085,10.6643);
\draw [color=c, fill=c] (12.3085,10.5584) rectangle (12.3483,10.6643);
\draw [color=c, fill=c] (12.3483,10.5584) rectangle (12.3881,10.6643);
\draw [color=c, fill=c] (12.3881,10.5584) rectangle (12.4279,10.6643);
\draw [color=c, fill=c] (12.4279,10.5584) rectangle (12.4677,10.6643);
\draw [color=c, fill=c] (12.4677,10.5584) rectangle (12.5075,10.6643);
\draw [color=c, fill=c] (12.5075,10.5584) rectangle (12.5473,10.6643);
\draw [color=c, fill=c] (12.5473,10.5584) rectangle (12.5871,10.6643);
\draw [color=c, fill=c] (12.5871,10.5584) rectangle (12.6269,10.6643);
\draw [color=c, fill=c] (12.6269,10.5584) rectangle (12.6667,10.6643);
\draw [color=c, fill=c] (12.6667,10.5584) rectangle (12.7065,10.6643);
\draw [color=c, fill=c] (12.7065,10.5584) rectangle (12.7463,10.6643);
\draw [color=c, fill=c] (12.7463,10.5584) rectangle (12.7861,10.6643);
\draw [color=c, fill=c] (12.7861,10.5584) rectangle (12.8259,10.6643);
\draw [color=c, fill=c] (12.8259,10.5584) rectangle (12.8657,10.6643);
\draw [color=c, fill=c] (12.8657,10.5584) rectangle (12.9055,10.6643);
\draw [color=c, fill=c] (12.9055,10.5584) rectangle (12.9453,10.6643);
\draw [color=c, fill=c] (12.9453,10.5584) rectangle (12.9851,10.6643);
\draw [color=c, fill=c] (12.9851,10.5584) rectangle (13.0249,10.6643);
\draw [color=c, fill=c] (13.0249,10.5584) rectangle (13.0647,10.6643);
\draw [color=c, fill=c] (13.0647,10.5584) rectangle (13.1045,10.6643);
\draw [color=c, fill=c] (13.1045,10.5584) rectangle (13.1443,10.6643);
\draw [color=c, fill=c] (13.1443,10.5584) rectangle (13.1841,10.6643);
\draw [color=c, fill=c] (13.1841,10.5584) rectangle (13.2239,10.6643);
\draw [color=c, fill=c] (13.2239,10.5584) rectangle (13.2637,10.6643);
\draw [color=c, fill=c] (13.2637,10.5584) rectangle (13.3035,10.6643);
\draw [color=c, fill=c] (13.3035,10.5584) rectangle (13.3433,10.6643);
\draw [color=c, fill=c] (13.3433,10.5584) rectangle (13.3831,10.6643);
\draw [color=c, fill=c] (13.3831,10.5584) rectangle (13.4229,10.6643);
\draw [color=c, fill=c] (13.4229,10.5584) rectangle (13.4627,10.6643);
\draw [color=c, fill=c] (13.4627,10.5584) rectangle (13.5025,10.6643);
\draw [color=c, fill=c] (13.5025,10.5584) rectangle (13.5423,10.6643);
\draw [color=c, fill=c] (13.5423,10.5584) rectangle (13.5821,10.6643);
\draw [color=c, fill=c] (13.5821,10.5584) rectangle (13.6219,10.6643);
\draw [color=c, fill=c] (13.6219,10.5584) rectangle (13.6617,10.6643);
\draw [color=c, fill=c] (13.6617,10.5584) rectangle (13.7015,10.6643);
\draw [color=c, fill=c] (13.7015,10.5584) rectangle (13.7413,10.6643);
\draw [color=c, fill=c] (13.7413,10.5584) rectangle (13.7811,10.6643);
\draw [color=c, fill=c] (13.7811,10.5584) rectangle (13.8209,10.6643);
\draw [color=c, fill=c] (13.8209,10.5584) rectangle (13.8607,10.6643);
\draw [color=c, fill=c] (13.8607,10.5584) rectangle (13.9005,10.6643);
\draw [color=c, fill=c] (13.9005,10.5584) rectangle (13.9403,10.6643);
\draw [color=c, fill=c] (13.9403,10.5584) rectangle (13.9801,10.6643);
\draw [color=c, fill=c] (13.9801,10.5584) rectangle (14.0199,10.6643);
\draw [color=c, fill=c] (14.0199,10.5584) rectangle (14.0597,10.6643);
\draw [color=c, fill=c] (14.0597,10.5584) rectangle (14.0995,10.6643);
\draw [color=c, fill=c] (14.0995,10.5584) rectangle (14.1393,10.6643);
\definecolor{c}{rgb}{0,0.733333,1};
\draw [color=c, fill=c] (14.1393,10.5584) rectangle (14.1791,10.6643);
\draw [color=c, fill=c] (14.1791,10.5584) rectangle (14.2189,10.6643);
\draw [color=c, fill=c] (14.2189,10.5584) rectangle (14.2587,10.6643);
\draw [color=c, fill=c] (14.2587,10.5584) rectangle (14.2985,10.6643);
\draw [color=c, fill=c] (14.2985,10.5584) rectangle (14.3383,10.6643);
\draw [color=c, fill=c] (14.3383,10.5584) rectangle (14.3781,10.6643);
\draw [color=c, fill=c] (14.3781,10.5584) rectangle (14.4179,10.6643);
\draw [color=c, fill=c] (14.4179,10.5584) rectangle (14.4577,10.6643);
\draw [color=c, fill=c] (14.4577,10.5584) rectangle (14.4975,10.6643);
\draw [color=c, fill=c] (14.4975,10.5584) rectangle (14.5373,10.6643);
\draw [color=c, fill=c] (14.5373,10.5584) rectangle (14.5771,10.6643);
\draw [color=c, fill=c] (14.5771,10.5584) rectangle (14.6169,10.6643);
\draw [color=c, fill=c] (14.6169,10.5584) rectangle (14.6567,10.6643);
\draw [color=c, fill=c] (14.6567,10.5584) rectangle (14.6965,10.6643);
\draw [color=c, fill=c] (14.6965,10.5584) rectangle (14.7363,10.6643);
\draw [color=c, fill=c] (14.7363,10.5584) rectangle (14.7761,10.6643);
\draw [color=c, fill=c] (14.7761,10.5584) rectangle (14.8159,10.6643);
\draw [color=c, fill=c] (14.8159,10.5584) rectangle (14.8557,10.6643);
\draw [color=c, fill=c] (14.8557,10.5584) rectangle (14.8955,10.6643);
\draw [color=c, fill=c] (14.8955,10.5584) rectangle (14.9353,10.6643);
\draw [color=c, fill=c] (14.9353,10.5584) rectangle (14.9751,10.6643);
\draw [color=c, fill=c] (14.9751,10.5584) rectangle (15.0149,10.6643);
\draw [color=c, fill=c] (15.0149,10.5584) rectangle (15.0547,10.6643);
\draw [color=c, fill=c] (15.0547,10.5584) rectangle (15.0945,10.6643);
\draw [color=c, fill=c] (15.0945,10.5584) rectangle (15.1343,10.6643);
\draw [color=c, fill=c] (15.1343,10.5584) rectangle (15.1741,10.6643);
\draw [color=c, fill=c] (15.1741,10.5584) rectangle (15.2139,10.6643);
\draw [color=c, fill=c] (15.2139,10.5584) rectangle (15.2537,10.6643);
\draw [color=c, fill=c] (15.2537,10.5584) rectangle (15.2935,10.6643);
\draw [color=c, fill=c] (15.2935,10.5584) rectangle (15.3333,10.6643);
\draw [color=c, fill=c] (15.3333,10.5584) rectangle (15.3731,10.6643);
\draw [color=c, fill=c] (15.3731,10.5584) rectangle (15.4129,10.6643);
\draw [color=c, fill=c] (15.4129,10.5584) rectangle (15.4527,10.6643);
\draw [color=c, fill=c] (15.4527,10.5584) rectangle (15.4925,10.6643);
\draw [color=c, fill=c] (15.4925,10.5584) rectangle (15.5323,10.6643);
\draw [color=c, fill=c] (15.5323,10.5584) rectangle (15.5721,10.6643);
\draw [color=c, fill=c] (15.5721,10.5584) rectangle (15.6119,10.6643);
\draw [color=c, fill=c] (15.6119,10.5584) rectangle (15.6517,10.6643);
\draw [color=c, fill=c] (15.6517,10.5584) rectangle (15.6915,10.6643);
\draw [color=c, fill=c] (15.6915,10.5584) rectangle (15.7313,10.6643);
\draw [color=c, fill=c] (15.7313,10.5584) rectangle (15.7711,10.6643);
\draw [color=c, fill=c] (15.7711,10.5584) rectangle (15.8109,10.6643);
\draw [color=c, fill=c] (15.8109,10.5584) rectangle (15.8507,10.6643);
\draw [color=c, fill=c] (15.8507,10.5584) rectangle (15.8905,10.6643);
\draw [color=c, fill=c] (15.8905,10.5584) rectangle (15.9303,10.6643);
\draw [color=c, fill=c] (15.9303,10.5584) rectangle (15.9701,10.6643);
\draw [color=c, fill=c] (15.9701,10.5584) rectangle (16.01,10.6643);
\draw [color=c, fill=c] (16.01,10.5584) rectangle (16.0498,10.6643);
\draw [color=c, fill=c] (16.0498,10.5584) rectangle (16.0896,10.6643);
\draw [color=c, fill=c] (16.0896,10.5584) rectangle (16.1294,10.6643);
\draw [color=c, fill=c] (16.1294,10.5584) rectangle (16.1692,10.6643);
\draw [color=c, fill=c] (16.1692,10.5584) rectangle (16.209,10.6643);
\draw [color=c, fill=c] (16.209,10.5584) rectangle (16.2488,10.6643);
\draw [color=c, fill=c] (16.2488,10.5584) rectangle (16.2886,10.6643);
\draw [color=c, fill=c] (16.2886,10.5584) rectangle (16.3284,10.6643);
\draw [color=c, fill=c] (16.3284,10.5584) rectangle (16.3682,10.6643);
\draw [color=c, fill=c] (16.3682,10.5584) rectangle (16.408,10.6643);
\draw [color=c, fill=c] (16.408,10.5584) rectangle (16.4478,10.6643);
\draw [color=c, fill=c] (16.4478,10.5584) rectangle (16.4876,10.6643);
\draw [color=c, fill=c] (16.4876,10.5584) rectangle (16.5274,10.6643);
\draw [color=c, fill=c] (16.5274,10.5584) rectangle (16.5672,10.6643);
\draw [color=c, fill=c] (16.5672,10.5584) rectangle (16.607,10.6643);
\draw [color=c, fill=c] (16.607,10.5584) rectangle (16.6468,10.6643);
\draw [color=c, fill=c] (16.6468,10.5584) rectangle (16.6866,10.6643);
\draw [color=c, fill=c] (16.6866,10.5584) rectangle (16.7264,10.6643);
\draw [color=c, fill=c] (16.7264,10.5584) rectangle (16.7662,10.6643);
\draw [color=c, fill=c] (16.7662,10.5584) rectangle (16.806,10.6643);
\draw [color=c, fill=c] (16.806,10.5584) rectangle (16.8458,10.6643);
\draw [color=c, fill=c] (16.8458,10.5584) rectangle (16.8856,10.6643);
\draw [color=c, fill=c] (16.8856,10.5584) rectangle (16.9254,10.6643);
\draw [color=c, fill=c] (16.9254,10.5584) rectangle (16.9652,10.6643);
\draw [color=c, fill=c] (16.9652,10.5584) rectangle (17.005,10.6643);
\draw [color=c, fill=c] (17.005,10.5584) rectangle (17.0448,10.6643);
\draw [color=c, fill=c] (17.0448,10.5584) rectangle (17.0846,10.6643);
\draw [color=c, fill=c] (17.0846,10.5584) rectangle (17.1244,10.6643);
\draw [color=c, fill=c] (17.1244,10.5584) rectangle (17.1642,10.6643);
\draw [color=c, fill=c] (17.1642,10.5584) rectangle (17.204,10.6643);
\draw [color=c, fill=c] (17.204,10.5584) rectangle (17.2438,10.6643);
\draw [color=c, fill=c] (17.2438,10.5584) rectangle (17.2836,10.6643);
\draw [color=c, fill=c] (17.2836,10.5584) rectangle (17.3234,10.6643);
\draw [color=c, fill=c] (17.3234,10.5584) rectangle (17.3632,10.6643);
\draw [color=c, fill=c] (17.3632,10.5584) rectangle (17.403,10.6643);
\draw [color=c, fill=c] (17.403,10.5584) rectangle (17.4428,10.6643);
\draw [color=c, fill=c] (17.4428,10.5584) rectangle (17.4826,10.6643);
\draw [color=c, fill=c] (17.4826,10.5584) rectangle (17.5224,10.6643);
\draw [color=c, fill=c] (17.5224,10.5584) rectangle (17.5622,10.6643);
\draw [color=c, fill=c] (17.5622,10.5584) rectangle (17.602,10.6643);
\draw [color=c, fill=c] (17.602,10.5584) rectangle (17.6418,10.6643);
\draw [color=c, fill=c] (17.6418,10.5584) rectangle (17.6816,10.6643);
\draw [color=c, fill=c] (17.6816,10.5584) rectangle (17.7214,10.6643);
\draw [color=c, fill=c] (17.7214,10.5584) rectangle (17.7612,10.6643);
\draw [color=c, fill=c] (17.7612,10.5584) rectangle (17.801,10.6643);
\draw [color=c, fill=c] (17.801,10.5584) rectangle (17.8408,10.6643);
\draw [color=c, fill=c] (17.8408,10.5584) rectangle (17.8806,10.6643);
\draw [color=c, fill=c] (17.8806,10.5584) rectangle (17.9204,10.6643);
\draw [color=c, fill=c] (17.9204,10.5584) rectangle (17.9602,10.6643);
\draw [color=c, fill=c] (17.9602,10.5584) rectangle (18,10.6643);
\definecolor{c}{rgb}{0.2,0,1};
\draw [color=c, fill=c] (2,10.6643) rectangle (2.0398,10.7701);
\draw [color=c, fill=c] (2.0398,10.6643) rectangle (2.0796,10.7701);
\draw [color=c, fill=c] (2.0796,10.6643) rectangle (2.1194,10.7701);
\draw [color=c, fill=c] (2.1194,10.6643) rectangle (2.1592,10.7701);
\draw [color=c, fill=c] (2.1592,10.6643) rectangle (2.19901,10.7701);
\draw [color=c, fill=c] (2.19901,10.6643) rectangle (2.23881,10.7701);
\draw [color=c, fill=c] (2.23881,10.6643) rectangle (2.27861,10.7701);
\draw [color=c, fill=c] (2.27861,10.6643) rectangle (2.31841,10.7701);
\draw [color=c, fill=c] (2.31841,10.6643) rectangle (2.35821,10.7701);
\draw [color=c, fill=c] (2.35821,10.6643) rectangle (2.39801,10.7701);
\draw [color=c, fill=c] (2.39801,10.6643) rectangle (2.43781,10.7701);
\draw [color=c, fill=c] (2.43781,10.6643) rectangle (2.47761,10.7701);
\draw [color=c, fill=c] (2.47761,10.6643) rectangle (2.51741,10.7701);
\draw [color=c, fill=c] (2.51741,10.6643) rectangle (2.55721,10.7701);
\draw [color=c, fill=c] (2.55721,10.6643) rectangle (2.59702,10.7701);
\draw [color=c, fill=c] (2.59702,10.6643) rectangle (2.63682,10.7701);
\draw [color=c, fill=c] (2.63682,10.6643) rectangle (2.67662,10.7701);
\draw [color=c, fill=c] (2.67662,10.6643) rectangle (2.71642,10.7701);
\draw [color=c, fill=c] (2.71642,10.6643) rectangle (2.75622,10.7701);
\draw [color=c, fill=c] (2.75622,10.6643) rectangle (2.79602,10.7701);
\draw [color=c, fill=c] (2.79602,10.6643) rectangle (2.83582,10.7701);
\draw [color=c, fill=c] (2.83582,10.6643) rectangle (2.87562,10.7701);
\draw [color=c, fill=c] (2.87562,10.6643) rectangle (2.91542,10.7701);
\draw [color=c, fill=c] (2.91542,10.6643) rectangle (2.95522,10.7701);
\draw [color=c, fill=c] (2.95522,10.6643) rectangle (2.99502,10.7701);
\draw [color=c, fill=c] (2.99502,10.6643) rectangle (3.03483,10.7701);
\draw [color=c, fill=c] (3.03483,10.6643) rectangle (3.07463,10.7701);
\draw [color=c, fill=c] (3.07463,10.6643) rectangle (3.11443,10.7701);
\draw [color=c, fill=c] (3.11443,10.6643) rectangle (3.15423,10.7701);
\draw [color=c, fill=c] (3.15423,10.6643) rectangle (3.19403,10.7701);
\draw [color=c, fill=c] (3.19403,10.6643) rectangle (3.23383,10.7701);
\draw [color=c, fill=c] (3.23383,10.6643) rectangle (3.27363,10.7701);
\draw [color=c, fill=c] (3.27363,10.6643) rectangle (3.31343,10.7701);
\draw [color=c, fill=c] (3.31343,10.6643) rectangle (3.35323,10.7701);
\draw [color=c, fill=c] (3.35323,10.6643) rectangle (3.39303,10.7701);
\draw [color=c, fill=c] (3.39303,10.6643) rectangle (3.43284,10.7701);
\draw [color=c, fill=c] (3.43284,10.6643) rectangle (3.47264,10.7701);
\draw [color=c, fill=c] (3.47264,10.6643) rectangle (3.51244,10.7701);
\draw [color=c, fill=c] (3.51244,10.6643) rectangle (3.55224,10.7701);
\draw [color=c, fill=c] (3.55224,10.6643) rectangle (3.59204,10.7701);
\draw [color=c, fill=c] (3.59204,10.6643) rectangle (3.63184,10.7701);
\draw [color=c, fill=c] (3.63184,10.6643) rectangle (3.67164,10.7701);
\draw [color=c, fill=c] (3.67164,10.6643) rectangle (3.71144,10.7701);
\draw [color=c, fill=c] (3.71144,10.6643) rectangle (3.75124,10.7701);
\draw [color=c, fill=c] (3.75124,10.6643) rectangle (3.79104,10.7701);
\draw [color=c, fill=c] (3.79104,10.6643) rectangle (3.83085,10.7701);
\draw [color=c, fill=c] (3.83085,10.6643) rectangle (3.87065,10.7701);
\draw [color=c, fill=c] (3.87065,10.6643) rectangle (3.91045,10.7701);
\draw [color=c, fill=c] (3.91045,10.6643) rectangle (3.95025,10.7701);
\draw [color=c, fill=c] (3.95025,10.6643) rectangle (3.99005,10.7701);
\draw [color=c, fill=c] (3.99005,10.6643) rectangle (4.02985,10.7701);
\draw [color=c, fill=c] (4.02985,10.6643) rectangle (4.06965,10.7701);
\draw [color=c, fill=c] (4.06965,10.6643) rectangle (4.10945,10.7701);
\draw [color=c, fill=c] (4.10945,10.6643) rectangle (4.14925,10.7701);
\draw [color=c, fill=c] (4.14925,10.6643) rectangle (4.18905,10.7701);
\draw [color=c, fill=c] (4.18905,10.6643) rectangle (4.22886,10.7701);
\draw [color=c, fill=c] (4.22886,10.6643) rectangle (4.26866,10.7701);
\draw [color=c, fill=c] (4.26866,10.6643) rectangle (4.30846,10.7701);
\draw [color=c, fill=c] (4.30846,10.6643) rectangle (4.34826,10.7701);
\draw [color=c, fill=c] (4.34826,10.6643) rectangle (4.38806,10.7701);
\draw [color=c, fill=c] (4.38806,10.6643) rectangle (4.42786,10.7701);
\draw [color=c, fill=c] (4.42786,10.6643) rectangle (4.46766,10.7701);
\draw [color=c, fill=c] (4.46766,10.6643) rectangle (4.50746,10.7701);
\draw [color=c, fill=c] (4.50746,10.6643) rectangle (4.54726,10.7701);
\draw [color=c, fill=c] (4.54726,10.6643) rectangle (4.58706,10.7701);
\draw [color=c, fill=c] (4.58706,10.6643) rectangle (4.62687,10.7701);
\draw [color=c, fill=c] (4.62687,10.6643) rectangle (4.66667,10.7701);
\draw [color=c, fill=c] (4.66667,10.6643) rectangle (4.70647,10.7701);
\draw [color=c, fill=c] (4.70647,10.6643) rectangle (4.74627,10.7701);
\draw [color=c, fill=c] (4.74627,10.6643) rectangle (4.78607,10.7701);
\draw [color=c, fill=c] (4.78607,10.6643) rectangle (4.82587,10.7701);
\draw [color=c, fill=c] (4.82587,10.6643) rectangle (4.86567,10.7701);
\draw [color=c, fill=c] (4.86567,10.6643) rectangle (4.90547,10.7701);
\draw [color=c, fill=c] (4.90547,10.6643) rectangle (4.94527,10.7701);
\draw [color=c, fill=c] (4.94527,10.6643) rectangle (4.98507,10.7701);
\draw [color=c, fill=c] (4.98507,10.6643) rectangle (5.02488,10.7701);
\draw [color=c, fill=c] (5.02488,10.6643) rectangle (5.06468,10.7701);
\draw [color=c, fill=c] (5.06468,10.6643) rectangle (5.10448,10.7701);
\draw [color=c, fill=c] (5.10448,10.6643) rectangle (5.14428,10.7701);
\draw [color=c, fill=c] (5.14428,10.6643) rectangle (5.18408,10.7701);
\draw [color=c, fill=c] (5.18408,10.6643) rectangle (5.22388,10.7701);
\draw [color=c, fill=c] (5.22388,10.6643) rectangle (5.26368,10.7701);
\draw [color=c, fill=c] (5.26368,10.6643) rectangle (5.30348,10.7701);
\draw [color=c, fill=c] (5.30348,10.6643) rectangle (5.34328,10.7701);
\draw [color=c, fill=c] (5.34328,10.6643) rectangle (5.38308,10.7701);
\draw [color=c, fill=c] (5.38308,10.6643) rectangle (5.42289,10.7701);
\draw [color=c, fill=c] (5.42289,10.6643) rectangle (5.46269,10.7701);
\draw [color=c, fill=c] (5.46269,10.6643) rectangle (5.50249,10.7701);
\draw [color=c, fill=c] (5.50249,10.6643) rectangle (5.54229,10.7701);
\draw [color=c, fill=c] (5.54229,10.6643) rectangle (5.58209,10.7701);
\draw [color=c, fill=c] (5.58209,10.6643) rectangle (5.62189,10.7701);
\draw [color=c, fill=c] (5.62189,10.6643) rectangle (5.66169,10.7701);
\draw [color=c, fill=c] (5.66169,10.6643) rectangle (5.70149,10.7701);
\draw [color=c, fill=c] (5.70149,10.6643) rectangle (5.74129,10.7701);
\draw [color=c, fill=c] (5.74129,10.6643) rectangle (5.78109,10.7701);
\draw [color=c, fill=c] (5.78109,10.6643) rectangle (5.8209,10.7701);
\draw [color=c, fill=c] (5.8209,10.6643) rectangle (5.8607,10.7701);
\draw [color=c, fill=c] (5.8607,10.6643) rectangle (5.9005,10.7701);
\draw [color=c, fill=c] (5.9005,10.6643) rectangle (5.9403,10.7701);
\draw [color=c, fill=c] (5.9403,10.6643) rectangle (5.9801,10.7701);
\draw [color=c, fill=c] (5.9801,10.6643) rectangle (6.0199,10.7701);
\draw [color=c, fill=c] (6.0199,10.6643) rectangle (6.0597,10.7701);
\draw [color=c, fill=c] (6.0597,10.6643) rectangle (6.0995,10.7701);
\draw [color=c, fill=c] (6.0995,10.6643) rectangle (6.1393,10.7701);
\draw [color=c, fill=c] (6.1393,10.6643) rectangle (6.1791,10.7701);
\draw [color=c, fill=c] (6.1791,10.6643) rectangle (6.21891,10.7701);
\draw [color=c, fill=c] (6.21891,10.6643) rectangle (6.25871,10.7701);
\draw [color=c, fill=c] (6.25871,10.6643) rectangle (6.29851,10.7701);
\draw [color=c, fill=c] (6.29851,10.6643) rectangle (6.33831,10.7701);
\draw [color=c, fill=c] (6.33831,10.6643) rectangle (6.37811,10.7701);
\draw [color=c, fill=c] (6.37811,10.6643) rectangle (6.41791,10.7701);
\draw [color=c, fill=c] (6.41791,10.6643) rectangle (6.45771,10.7701);
\draw [color=c, fill=c] (6.45771,10.6643) rectangle (6.49751,10.7701);
\draw [color=c, fill=c] (6.49751,10.6643) rectangle (6.53731,10.7701);
\draw [color=c, fill=c] (6.53731,10.6643) rectangle (6.57711,10.7701);
\draw [color=c, fill=c] (6.57711,10.6643) rectangle (6.61692,10.7701);
\draw [color=c, fill=c] (6.61692,10.6643) rectangle (6.65672,10.7701);
\draw [color=c, fill=c] (6.65672,10.6643) rectangle (6.69652,10.7701);
\draw [color=c, fill=c] (6.69652,10.6643) rectangle (6.73632,10.7701);
\draw [color=c, fill=c] (6.73632,10.6643) rectangle (6.77612,10.7701);
\draw [color=c, fill=c] (6.77612,10.6643) rectangle (6.81592,10.7701);
\draw [color=c, fill=c] (6.81592,10.6643) rectangle (6.85572,10.7701);
\draw [color=c, fill=c] (6.85572,10.6643) rectangle (6.89552,10.7701);
\draw [color=c, fill=c] (6.89552,10.6643) rectangle (6.93532,10.7701);
\draw [color=c, fill=c] (6.93532,10.6643) rectangle (6.97512,10.7701);
\draw [color=c, fill=c] (6.97512,10.6643) rectangle (7.01493,10.7701);
\draw [color=c, fill=c] (7.01493,10.6643) rectangle (7.05473,10.7701);
\draw [color=c, fill=c] (7.05473,10.6643) rectangle (7.09453,10.7701);
\draw [color=c, fill=c] (7.09453,10.6643) rectangle (7.13433,10.7701);
\draw [color=c, fill=c] (7.13433,10.6643) rectangle (7.17413,10.7701);
\draw [color=c, fill=c] (7.17413,10.6643) rectangle (7.21393,10.7701);
\draw [color=c, fill=c] (7.21393,10.6643) rectangle (7.25373,10.7701);
\draw [color=c, fill=c] (7.25373,10.6643) rectangle (7.29353,10.7701);
\draw [color=c, fill=c] (7.29353,10.6643) rectangle (7.33333,10.7701);
\draw [color=c, fill=c] (7.33333,10.6643) rectangle (7.37313,10.7701);
\draw [color=c, fill=c] (7.37313,10.6643) rectangle (7.41294,10.7701);
\draw [color=c, fill=c] (7.41294,10.6643) rectangle (7.45274,10.7701);
\draw [color=c, fill=c] (7.45274,10.6643) rectangle (7.49254,10.7701);
\draw [color=c, fill=c] (7.49254,10.6643) rectangle (7.53234,10.7701);
\draw [color=c, fill=c] (7.53234,10.6643) rectangle (7.57214,10.7701);
\draw [color=c, fill=c] (7.57214,10.6643) rectangle (7.61194,10.7701);
\draw [color=c, fill=c] (7.61194,10.6643) rectangle (7.65174,10.7701);
\draw [color=c, fill=c] (7.65174,10.6643) rectangle (7.69154,10.7701);
\draw [color=c, fill=c] (7.69154,10.6643) rectangle (7.73134,10.7701);
\draw [color=c, fill=c] (7.73134,10.6643) rectangle (7.77114,10.7701);
\draw [color=c, fill=c] (7.77114,10.6643) rectangle (7.81095,10.7701);
\draw [color=c, fill=c] (7.81095,10.6643) rectangle (7.85075,10.7701);
\definecolor{c}{rgb}{0,0.0800001,1};
\draw [color=c, fill=c] (7.85075,10.6643) rectangle (7.89055,10.7701);
\draw [color=c, fill=c] (7.89055,10.6643) rectangle (7.93035,10.7701);
\draw [color=c, fill=c] (7.93035,10.6643) rectangle (7.97015,10.7701);
\draw [color=c, fill=c] (7.97015,10.6643) rectangle (8.00995,10.7701);
\draw [color=c, fill=c] (8.00995,10.6643) rectangle (8.04975,10.7701);
\draw [color=c, fill=c] (8.04975,10.6643) rectangle (8.08955,10.7701);
\draw [color=c, fill=c] (8.08955,10.6643) rectangle (8.12935,10.7701);
\draw [color=c, fill=c] (8.12935,10.6643) rectangle (8.16915,10.7701);
\draw [color=c, fill=c] (8.16915,10.6643) rectangle (8.20895,10.7701);
\draw [color=c, fill=c] (8.20895,10.6643) rectangle (8.24876,10.7701);
\draw [color=c, fill=c] (8.24876,10.6643) rectangle (8.28856,10.7701);
\draw [color=c, fill=c] (8.28856,10.6643) rectangle (8.32836,10.7701);
\draw [color=c, fill=c] (8.32836,10.6643) rectangle (8.36816,10.7701);
\draw [color=c, fill=c] (8.36816,10.6643) rectangle (8.40796,10.7701);
\draw [color=c, fill=c] (8.40796,10.6643) rectangle (8.44776,10.7701);
\draw [color=c, fill=c] (8.44776,10.6643) rectangle (8.48756,10.7701);
\draw [color=c, fill=c] (8.48756,10.6643) rectangle (8.52736,10.7701);
\draw [color=c, fill=c] (8.52736,10.6643) rectangle (8.56716,10.7701);
\draw [color=c, fill=c] (8.56716,10.6643) rectangle (8.60697,10.7701);
\draw [color=c, fill=c] (8.60697,10.6643) rectangle (8.64677,10.7701);
\draw [color=c, fill=c] (8.64677,10.6643) rectangle (8.68657,10.7701);
\draw [color=c, fill=c] (8.68657,10.6643) rectangle (8.72637,10.7701);
\draw [color=c, fill=c] (8.72637,10.6643) rectangle (8.76617,10.7701);
\draw [color=c, fill=c] (8.76617,10.6643) rectangle (8.80597,10.7701);
\draw [color=c, fill=c] (8.80597,10.6643) rectangle (8.84577,10.7701);
\draw [color=c, fill=c] (8.84577,10.6643) rectangle (8.88557,10.7701);
\draw [color=c, fill=c] (8.88557,10.6643) rectangle (8.92537,10.7701);
\draw [color=c, fill=c] (8.92537,10.6643) rectangle (8.96517,10.7701);
\draw [color=c, fill=c] (8.96517,10.6643) rectangle (9.00498,10.7701);
\draw [color=c, fill=c] (9.00498,10.6643) rectangle (9.04478,10.7701);
\draw [color=c, fill=c] (9.04478,10.6643) rectangle (9.08458,10.7701);
\draw [color=c, fill=c] (9.08458,10.6643) rectangle (9.12438,10.7701);
\draw [color=c, fill=c] (9.12438,10.6643) rectangle (9.16418,10.7701);
\draw [color=c, fill=c] (9.16418,10.6643) rectangle (9.20398,10.7701);
\draw [color=c, fill=c] (9.20398,10.6643) rectangle (9.24378,10.7701);
\draw [color=c, fill=c] (9.24378,10.6643) rectangle (9.28358,10.7701);
\draw [color=c, fill=c] (9.28358,10.6643) rectangle (9.32338,10.7701);
\draw [color=c, fill=c] (9.32338,10.6643) rectangle (9.36318,10.7701);
\draw [color=c, fill=c] (9.36318,10.6643) rectangle (9.40298,10.7701);
\draw [color=c, fill=c] (9.40298,10.6643) rectangle (9.44279,10.7701);
\draw [color=c, fill=c] (9.44279,10.6643) rectangle (9.48259,10.7701);
\draw [color=c, fill=c] (9.48259,10.6643) rectangle (9.52239,10.7701);
\draw [color=c, fill=c] (9.52239,10.6643) rectangle (9.56219,10.7701);
\definecolor{c}{rgb}{0,0.266667,1};
\draw [color=c, fill=c] (9.56219,10.6643) rectangle (9.60199,10.7701);
\draw [color=c, fill=c] (9.60199,10.6643) rectangle (9.64179,10.7701);
\draw [color=c, fill=c] (9.64179,10.6643) rectangle (9.68159,10.7701);
\draw [color=c, fill=c] (9.68159,10.6643) rectangle (9.72139,10.7701);
\draw [color=c, fill=c] (9.72139,10.6643) rectangle (9.76119,10.7701);
\draw [color=c, fill=c] (9.76119,10.6643) rectangle (9.80099,10.7701);
\draw [color=c, fill=c] (9.80099,10.6643) rectangle (9.8408,10.7701);
\draw [color=c, fill=c] (9.8408,10.6643) rectangle (9.8806,10.7701);
\draw [color=c, fill=c] (9.8806,10.6643) rectangle (9.9204,10.7701);
\draw [color=c, fill=c] (9.9204,10.6643) rectangle (9.9602,10.7701);
\draw [color=c, fill=c] (9.9602,10.6643) rectangle (10,10.7701);
\draw [color=c, fill=c] (10,10.6643) rectangle (10.0398,10.7701);
\draw [color=c, fill=c] (10.0398,10.6643) rectangle (10.0796,10.7701);
\draw [color=c, fill=c] (10.0796,10.6643) rectangle (10.1194,10.7701);
\draw [color=c, fill=c] (10.1194,10.6643) rectangle (10.1592,10.7701);
\draw [color=c, fill=c] (10.1592,10.6643) rectangle (10.199,10.7701);
\draw [color=c, fill=c] (10.199,10.6643) rectangle (10.2388,10.7701);
\draw [color=c, fill=c] (10.2388,10.6643) rectangle (10.2786,10.7701);
\draw [color=c, fill=c] (10.2786,10.6643) rectangle (10.3184,10.7701);
\draw [color=c, fill=c] (10.3184,10.6643) rectangle (10.3582,10.7701);
\draw [color=c, fill=c] (10.3582,10.6643) rectangle (10.398,10.7701);
\draw [color=c, fill=c] (10.398,10.6643) rectangle (10.4378,10.7701);
\draw [color=c, fill=c] (10.4378,10.6643) rectangle (10.4776,10.7701);
\draw [color=c, fill=c] (10.4776,10.6643) rectangle (10.5174,10.7701);
\draw [color=c, fill=c] (10.5174,10.6643) rectangle (10.5572,10.7701);
\draw [color=c, fill=c] (10.5572,10.6643) rectangle (10.597,10.7701);
\draw [color=c, fill=c] (10.597,10.6643) rectangle (10.6368,10.7701);
\draw [color=c, fill=c] (10.6368,10.6643) rectangle (10.6766,10.7701);
\draw [color=c, fill=c] (10.6766,10.6643) rectangle (10.7164,10.7701);
\draw [color=c, fill=c] (10.7164,10.6643) rectangle (10.7562,10.7701);
\draw [color=c, fill=c] (10.7562,10.6643) rectangle (10.796,10.7701);
\draw [color=c, fill=c] (10.796,10.6643) rectangle (10.8358,10.7701);
\draw [color=c, fill=c] (10.8358,10.6643) rectangle (10.8756,10.7701);
\definecolor{c}{rgb}{0,0.546666,1};
\draw [color=c, fill=c] (10.8756,10.6643) rectangle (10.9154,10.7701);
\draw [color=c, fill=c] (10.9154,10.6643) rectangle (10.9552,10.7701);
\draw [color=c, fill=c] (10.9552,10.6643) rectangle (10.995,10.7701);
\draw [color=c, fill=c] (10.995,10.6643) rectangle (11.0348,10.7701);
\draw [color=c, fill=c] (11.0348,10.6643) rectangle (11.0746,10.7701);
\draw [color=c, fill=c] (11.0746,10.6643) rectangle (11.1144,10.7701);
\draw [color=c, fill=c] (11.1144,10.6643) rectangle (11.1542,10.7701);
\draw [color=c, fill=c] (11.1542,10.6643) rectangle (11.194,10.7701);
\draw [color=c, fill=c] (11.194,10.6643) rectangle (11.2338,10.7701);
\draw [color=c, fill=c] (11.2338,10.6643) rectangle (11.2736,10.7701);
\draw [color=c, fill=c] (11.2736,10.6643) rectangle (11.3134,10.7701);
\draw [color=c, fill=c] (11.3134,10.6643) rectangle (11.3532,10.7701);
\draw [color=c, fill=c] (11.3532,10.6643) rectangle (11.393,10.7701);
\draw [color=c, fill=c] (11.393,10.6643) rectangle (11.4328,10.7701);
\draw [color=c, fill=c] (11.4328,10.6643) rectangle (11.4726,10.7701);
\draw [color=c, fill=c] (11.4726,10.6643) rectangle (11.5124,10.7701);
\draw [color=c, fill=c] (11.5124,10.6643) rectangle (11.5522,10.7701);
\draw [color=c, fill=c] (11.5522,10.6643) rectangle (11.592,10.7701);
\draw [color=c, fill=c] (11.592,10.6643) rectangle (11.6318,10.7701);
\draw [color=c, fill=c] (11.6318,10.6643) rectangle (11.6716,10.7701);
\draw [color=c, fill=c] (11.6716,10.6643) rectangle (11.7114,10.7701);
\draw [color=c, fill=c] (11.7114,10.6643) rectangle (11.7512,10.7701);
\draw [color=c, fill=c] (11.7512,10.6643) rectangle (11.791,10.7701);
\draw [color=c, fill=c] (11.791,10.6643) rectangle (11.8308,10.7701);
\draw [color=c, fill=c] (11.8308,10.6643) rectangle (11.8706,10.7701);
\draw [color=c, fill=c] (11.8706,10.6643) rectangle (11.9104,10.7701);
\draw [color=c, fill=c] (11.9104,10.6643) rectangle (11.9502,10.7701);
\draw [color=c, fill=c] (11.9502,10.6643) rectangle (11.99,10.7701);
\draw [color=c, fill=c] (11.99,10.6643) rectangle (12.0299,10.7701);
\draw [color=c, fill=c] (12.0299,10.6643) rectangle (12.0697,10.7701);
\draw [color=c, fill=c] (12.0697,10.6643) rectangle (12.1095,10.7701);
\draw [color=c, fill=c] (12.1095,10.6643) rectangle (12.1493,10.7701);
\draw [color=c, fill=c] (12.1493,10.6643) rectangle (12.1891,10.7701);
\draw [color=c, fill=c] (12.1891,10.6643) rectangle (12.2289,10.7701);
\draw [color=c, fill=c] (12.2289,10.6643) rectangle (12.2687,10.7701);
\draw [color=c, fill=c] (12.2687,10.6643) rectangle (12.3085,10.7701);
\draw [color=c, fill=c] (12.3085,10.6643) rectangle (12.3483,10.7701);
\draw [color=c, fill=c] (12.3483,10.6643) rectangle (12.3881,10.7701);
\draw [color=c, fill=c] (12.3881,10.6643) rectangle (12.4279,10.7701);
\draw [color=c, fill=c] (12.4279,10.6643) rectangle (12.4677,10.7701);
\draw [color=c, fill=c] (12.4677,10.6643) rectangle (12.5075,10.7701);
\draw [color=c, fill=c] (12.5075,10.6643) rectangle (12.5473,10.7701);
\draw [color=c, fill=c] (12.5473,10.6643) rectangle (12.5871,10.7701);
\draw [color=c, fill=c] (12.5871,10.6643) rectangle (12.6269,10.7701);
\draw [color=c, fill=c] (12.6269,10.6643) rectangle (12.6667,10.7701);
\draw [color=c, fill=c] (12.6667,10.6643) rectangle (12.7065,10.7701);
\draw [color=c, fill=c] (12.7065,10.6643) rectangle (12.7463,10.7701);
\draw [color=c, fill=c] (12.7463,10.6643) rectangle (12.7861,10.7701);
\draw [color=c, fill=c] (12.7861,10.6643) rectangle (12.8259,10.7701);
\draw [color=c, fill=c] (12.8259,10.6643) rectangle (12.8657,10.7701);
\draw [color=c, fill=c] (12.8657,10.6643) rectangle (12.9055,10.7701);
\draw [color=c, fill=c] (12.9055,10.6643) rectangle (12.9453,10.7701);
\draw [color=c, fill=c] (12.9453,10.6643) rectangle (12.9851,10.7701);
\draw [color=c, fill=c] (12.9851,10.6643) rectangle (13.0249,10.7701);
\draw [color=c, fill=c] (13.0249,10.6643) rectangle (13.0647,10.7701);
\draw [color=c, fill=c] (13.0647,10.6643) rectangle (13.1045,10.7701);
\draw [color=c, fill=c] (13.1045,10.6643) rectangle (13.1443,10.7701);
\draw [color=c, fill=c] (13.1443,10.6643) rectangle (13.1841,10.7701);
\draw [color=c, fill=c] (13.1841,10.6643) rectangle (13.2239,10.7701);
\draw [color=c, fill=c] (13.2239,10.6643) rectangle (13.2637,10.7701);
\draw [color=c, fill=c] (13.2637,10.6643) rectangle (13.3035,10.7701);
\draw [color=c, fill=c] (13.3035,10.6643) rectangle (13.3433,10.7701);
\draw [color=c, fill=c] (13.3433,10.6643) rectangle (13.3831,10.7701);
\draw [color=c, fill=c] (13.3831,10.6643) rectangle (13.4229,10.7701);
\draw [color=c, fill=c] (13.4229,10.6643) rectangle (13.4627,10.7701);
\draw [color=c, fill=c] (13.4627,10.6643) rectangle (13.5025,10.7701);
\draw [color=c, fill=c] (13.5025,10.6643) rectangle (13.5423,10.7701);
\draw [color=c, fill=c] (13.5423,10.6643) rectangle (13.5821,10.7701);
\draw [color=c, fill=c] (13.5821,10.6643) rectangle (13.6219,10.7701);
\draw [color=c, fill=c] (13.6219,10.6643) rectangle (13.6617,10.7701);
\draw [color=c, fill=c] (13.6617,10.6643) rectangle (13.7015,10.7701);
\draw [color=c, fill=c] (13.7015,10.6643) rectangle (13.7413,10.7701);
\draw [color=c, fill=c] (13.7413,10.6643) rectangle (13.7811,10.7701);
\draw [color=c, fill=c] (13.7811,10.6643) rectangle (13.8209,10.7701);
\draw [color=c, fill=c] (13.8209,10.6643) rectangle (13.8607,10.7701);
\draw [color=c, fill=c] (13.8607,10.6643) rectangle (13.9005,10.7701);
\draw [color=c, fill=c] (13.9005,10.6643) rectangle (13.9403,10.7701);
\draw [color=c, fill=c] (13.9403,10.6643) rectangle (13.9801,10.7701);
\draw [color=c, fill=c] (13.9801,10.6643) rectangle (14.0199,10.7701);
\draw [color=c, fill=c] (14.0199,10.6643) rectangle (14.0597,10.7701);
\draw [color=c, fill=c] (14.0597,10.6643) rectangle (14.0995,10.7701);
\draw [color=c, fill=c] (14.0995,10.6643) rectangle (14.1393,10.7701);
\draw [color=c, fill=c] (14.1393,10.6643) rectangle (14.1791,10.7701);
\definecolor{c}{rgb}{0,0.733333,1};
\draw [color=c, fill=c] (14.1791,10.6643) rectangle (14.2189,10.7701);
\draw [color=c, fill=c] (14.2189,10.6643) rectangle (14.2587,10.7701);
\draw [color=c, fill=c] (14.2587,10.6643) rectangle (14.2985,10.7701);
\draw [color=c, fill=c] (14.2985,10.6643) rectangle (14.3383,10.7701);
\draw [color=c, fill=c] (14.3383,10.6643) rectangle (14.3781,10.7701);
\draw [color=c, fill=c] (14.3781,10.6643) rectangle (14.4179,10.7701);
\draw [color=c, fill=c] (14.4179,10.6643) rectangle (14.4577,10.7701);
\draw [color=c, fill=c] (14.4577,10.6643) rectangle (14.4975,10.7701);
\draw [color=c, fill=c] (14.4975,10.6643) rectangle (14.5373,10.7701);
\draw [color=c, fill=c] (14.5373,10.6643) rectangle (14.5771,10.7701);
\draw [color=c, fill=c] (14.5771,10.6643) rectangle (14.6169,10.7701);
\draw [color=c, fill=c] (14.6169,10.6643) rectangle (14.6567,10.7701);
\draw [color=c, fill=c] (14.6567,10.6643) rectangle (14.6965,10.7701);
\draw [color=c, fill=c] (14.6965,10.6643) rectangle (14.7363,10.7701);
\draw [color=c, fill=c] (14.7363,10.6643) rectangle (14.7761,10.7701);
\draw [color=c, fill=c] (14.7761,10.6643) rectangle (14.8159,10.7701);
\draw [color=c, fill=c] (14.8159,10.6643) rectangle (14.8557,10.7701);
\draw [color=c, fill=c] (14.8557,10.6643) rectangle (14.8955,10.7701);
\draw [color=c, fill=c] (14.8955,10.6643) rectangle (14.9353,10.7701);
\draw [color=c, fill=c] (14.9353,10.6643) rectangle (14.9751,10.7701);
\draw [color=c, fill=c] (14.9751,10.6643) rectangle (15.0149,10.7701);
\draw [color=c, fill=c] (15.0149,10.6643) rectangle (15.0547,10.7701);
\draw [color=c, fill=c] (15.0547,10.6643) rectangle (15.0945,10.7701);
\draw [color=c, fill=c] (15.0945,10.6643) rectangle (15.1343,10.7701);
\draw [color=c, fill=c] (15.1343,10.6643) rectangle (15.1741,10.7701);
\draw [color=c, fill=c] (15.1741,10.6643) rectangle (15.2139,10.7701);
\draw [color=c, fill=c] (15.2139,10.6643) rectangle (15.2537,10.7701);
\draw [color=c, fill=c] (15.2537,10.6643) rectangle (15.2935,10.7701);
\draw [color=c, fill=c] (15.2935,10.6643) rectangle (15.3333,10.7701);
\draw [color=c, fill=c] (15.3333,10.6643) rectangle (15.3731,10.7701);
\draw [color=c, fill=c] (15.3731,10.6643) rectangle (15.4129,10.7701);
\draw [color=c, fill=c] (15.4129,10.6643) rectangle (15.4527,10.7701);
\draw [color=c, fill=c] (15.4527,10.6643) rectangle (15.4925,10.7701);
\draw [color=c, fill=c] (15.4925,10.6643) rectangle (15.5323,10.7701);
\draw [color=c, fill=c] (15.5323,10.6643) rectangle (15.5721,10.7701);
\draw [color=c, fill=c] (15.5721,10.6643) rectangle (15.6119,10.7701);
\draw [color=c, fill=c] (15.6119,10.6643) rectangle (15.6517,10.7701);
\draw [color=c, fill=c] (15.6517,10.6643) rectangle (15.6915,10.7701);
\draw [color=c, fill=c] (15.6915,10.6643) rectangle (15.7313,10.7701);
\draw [color=c, fill=c] (15.7313,10.6643) rectangle (15.7711,10.7701);
\draw [color=c, fill=c] (15.7711,10.6643) rectangle (15.8109,10.7701);
\draw [color=c, fill=c] (15.8109,10.6643) rectangle (15.8507,10.7701);
\draw [color=c, fill=c] (15.8507,10.6643) rectangle (15.8905,10.7701);
\draw [color=c, fill=c] (15.8905,10.6643) rectangle (15.9303,10.7701);
\draw [color=c, fill=c] (15.9303,10.6643) rectangle (15.9701,10.7701);
\draw [color=c, fill=c] (15.9701,10.6643) rectangle (16.01,10.7701);
\draw [color=c, fill=c] (16.01,10.6643) rectangle (16.0498,10.7701);
\draw [color=c, fill=c] (16.0498,10.6643) rectangle (16.0896,10.7701);
\draw [color=c, fill=c] (16.0896,10.6643) rectangle (16.1294,10.7701);
\draw [color=c, fill=c] (16.1294,10.6643) rectangle (16.1692,10.7701);
\draw [color=c, fill=c] (16.1692,10.6643) rectangle (16.209,10.7701);
\draw [color=c, fill=c] (16.209,10.6643) rectangle (16.2488,10.7701);
\draw [color=c, fill=c] (16.2488,10.6643) rectangle (16.2886,10.7701);
\draw [color=c, fill=c] (16.2886,10.6643) rectangle (16.3284,10.7701);
\draw [color=c, fill=c] (16.3284,10.6643) rectangle (16.3682,10.7701);
\draw [color=c, fill=c] (16.3682,10.6643) rectangle (16.408,10.7701);
\draw [color=c, fill=c] (16.408,10.6643) rectangle (16.4478,10.7701);
\draw [color=c, fill=c] (16.4478,10.6643) rectangle (16.4876,10.7701);
\draw [color=c, fill=c] (16.4876,10.6643) rectangle (16.5274,10.7701);
\draw [color=c, fill=c] (16.5274,10.6643) rectangle (16.5672,10.7701);
\draw [color=c, fill=c] (16.5672,10.6643) rectangle (16.607,10.7701);
\draw [color=c, fill=c] (16.607,10.6643) rectangle (16.6468,10.7701);
\draw [color=c, fill=c] (16.6468,10.6643) rectangle (16.6866,10.7701);
\draw [color=c, fill=c] (16.6866,10.6643) rectangle (16.7264,10.7701);
\draw [color=c, fill=c] (16.7264,10.6643) rectangle (16.7662,10.7701);
\draw [color=c, fill=c] (16.7662,10.6643) rectangle (16.806,10.7701);
\draw [color=c, fill=c] (16.806,10.6643) rectangle (16.8458,10.7701);
\draw [color=c, fill=c] (16.8458,10.6643) rectangle (16.8856,10.7701);
\draw [color=c, fill=c] (16.8856,10.6643) rectangle (16.9254,10.7701);
\draw [color=c, fill=c] (16.9254,10.6643) rectangle (16.9652,10.7701);
\draw [color=c, fill=c] (16.9652,10.6643) rectangle (17.005,10.7701);
\draw [color=c, fill=c] (17.005,10.6643) rectangle (17.0448,10.7701);
\draw [color=c, fill=c] (17.0448,10.6643) rectangle (17.0846,10.7701);
\draw [color=c, fill=c] (17.0846,10.6643) rectangle (17.1244,10.7701);
\draw [color=c, fill=c] (17.1244,10.6643) rectangle (17.1642,10.7701);
\draw [color=c, fill=c] (17.1642,10.6643) rectangle (17.204,10.7701);
\draw [color=c, fill=c] (17.204,10.6643) rectangle (17.2438,10.7701);
\draw [color=c, fill=c] (17.2438,10.6643) rectangle (17.2836,10.7701);
\draw [color=c, fill=c] (17.2836,10.6643) rectangle (17.3234,10.7701);
\draw [color=c, fill=c] (17.3234,10.6643) rectangle (17.3632,10.7701);
\draw [color=c, fill=c] (17.3632,10.6643) rectangle (17.403,10.7701);
\draw [color=c, fill=c] (17.403,10.6643) rectangle (17.4428,10.7701);
\draw [color=c, fill=c] (17.4428,10.6643) rectangle (17.4826,10.7701);
\draw [color=c, fill=c] (17.4826,10.6643) rectangle (17.5224,10.7701);
\draw [color=c, fill=c] (17.5224,10.6643) rectangle (17.5622,10.7701);
\draw [color=c, fill=c] (17.5622,10.6643) rectangle (17.602,10.7701);
\draw [color=c, fill=c] (17.602,10.6643) rectangle (17.6418,10.7701);
\draw [color=c, fill=c] (17.6418,10.6643) rectangle (17.6816,10.7701);
\draw [color=c, fill=c] (17.6816,10.6643) rectangle (17.7214,10.7701);
\draw [color=c, fill=c] (17.7214,10.6643) rectangle (17.7612,10.7701);
\draw [color=c, fill=c] (17.7612,10.6643) rectangle (17.801,10.7701);
\draw [color=c, fill=c] (17.801,10.6643) rectangle (17.8408,10.7701);
\draw [color=c, fill=c] (17.8408,10.6643) rectangle (17.8806,10.7701);
\draw [color=c, fill=c] (17.8806,10.6643) rectangle (17.9204,10.7701);
\draw [color=c, fill=c] (17.9204,10.6643) rectangle (17.9602,10.7701);
\draw [color=c, fill=c] (17.9602,10.6643) rectangle (18,10.7701);
\definecolor{c}{rgb}{0.2,0,1};
\draw [color=c, fill=c] (2,10.7701) rectangle (2.0398,10.8759);
\draw [color=c, fill=c] (2.0398,10.7701) rectangle (2.0796,10.8759);
\draw [color=c, fill=c] (2.0796,10.7701) rectangle (2.1194,10.8759);
\draw [color=c, fill=c] (2.1194,10.7701) rectangle (2.1592,10.8759);
\draw [color=c, fill=c] (2.1592,10.7701) rectangle (2.19901,10.8759);
\draw [color=c, fill=c] (2.19901,10.7701) rectangle (2.23881,10.8759);
\draw [color=c, fill=c] (2.23881,10.7701) rectangle (2.27861,10.8759);
\draw [color=c, fill=c] (2.27861,10.7701) rectangle (2.31841,10.8759);
\draw [color=c, fill=c] (2.31841,10.7701) rectangle (2.35821,10.8759);
\draw [color=c, fill=c] (2.35821,10.7701) rectangle (2.39801,10.8759);
\draw [color=c, fill=c] (2.39801,10.7701) rectangle (2.43781,10.8759);
\draw [color=c, fill=c] (2.43781,10.7701) rectangle (2.47761,10.8759);
\draw [color=c, fill=c] (2.47761,10.7701) rectangle (2.51741,10.8759);
\draw [color=c, fill=c] (2.51741,10.7701) rectangle (2.55721,10.8759);
\draw [color=c, fill=c] (2.55721,10.7701) rectangle (2.59702,10.8759);
\draw [color=c, fill=c] (2.59702,10.7701) rectangle (2.63682,10.8759);
\draw [color=c, fill=c] (2.63682,10.7701) rectangle (2.67662,10.8759);
\draw [color=c, fill=c] (2.67662,10.7701) rectangle (2.71642,10.8759);
\draw [color=c, fill=c] (2.71642,10.7701) rectangle (2.75622,10.8759);
\draw [color=c, fill=c] (2.75622,10.7701) rectangle (2.79602,10.8759);
\draw [color=c, fill=c] (2.79602,10.7701) rectangle (2.83582,10.8759);
\draw [color=c, fill=c] (2.83582,10.7701) rectangle (2.87562,10.8759);
\draw [color=c, fill=c] (2.87562,10.7701) rectangle (2.91542,10.8759);
\draw [color=c, fill=c] (2.91542,10.7701) rectangle (2.95522,10.8759);
\draw [color=c, fill=c] (2.95522,10.7701) rectangle (2.99502,10.8759);
\draw [color=c, fill=c] (2.99502,10.7701) rectangle (3.03483,10.8759);
\draw [color=c, fill=c] (3.03483,10.7701) rectangle (3.07463,10.8759);
\draw [color=c, fill=c] (3.07463,10.7701) rectangle (3.11443,10.8759);
\draw [color=c, fill=c] (3.11443,10.7701) rectangle (3.15423,10.8759);
\draw [color=c, fill=c] (3.15423,10.7701) rectangle (3.19403,10.8759);
\draw [color=c, fill=c] (3.19403,10.7701) rectangle (3.23383,10.8759);
\draw [color=c, fill=c] (3.23383,10.7701) rectangle (3.27363,10.8759);
\draw [color=c, fill=c] (3.27363,10.7701) rectangle (3.31343,10.8759);
\draw [color=c, fill=c] (3.31343,10.7701) rectangle (3.35323,10.8759);
\draw [color=c, fill=c] (3.35323,10.7701) rectangle (3.39303,10.8759);
\draw [color=c, fill=c] (3.39303,10.7701) rectangle (3.43284,10.8759);
\draw [color=c, fill=c] (3.43284,10.7701) rectangle (3.47264,10.8759);
\draw [color=c, fill=c] (3.47264,10.7701) rectangle (3.51244,10.8759);
\draw [color=c, fill=c] (3.51244,10.7701) rectangle (3.55224,10.8759);
\draw [color=c, fill=c] (3.55224,10.7701) rectangle (3.59204,10.8759);
\draw [color=c, fill=c] (3.59204,10.7701) rectangle (3.63184,10.8759);
\draw [color=c, fill=c] (3.63184,10.7701) rectangle (3.67164,10.8759);
\draw [color=c, fill=c] (3.67164,10.7701) rectangle (3.71144,10.8759);
\draw [color=c, fill=c] (3.71144,10.7701) rectangle (3.75124,10.8759);
\draw [color=c, fill=c] (3.75124,10.7701) rectangle (3.79104,10.8759);
\draw [color=c, fill=c] (3.79104,10.7701) rectangle (3.83085,10.8759);
\draw [color=c, fill=c] (3.83085,10.7701) rectangle (3.87065,10.8759);
\draw [color=c, fill=c] (3.87065,10.7701) rectangle (3.91045,10.8759);
\draw [color=c, fill=c] (3.91045,10.7701) rectangle (3.95025,10.8759);
\draw [color=c, fill=c] (3.95025,10.7701) rectangle (3.99005,10.8759);
\draw [color=c, fill=c] (3.99005,10.7701) rectangle (4.02985,10.8759);
\draw [color=c, fill=c] (4.02985,10.7701) rectangle (4.06965,10.8759);
\draw [color=c, fill=c] (4.06965,10.7701) rectangle (4.10945,10.8759);
\draw [color=c, fill=c] (4.10945,10.7701) rectangle (4.14925,10.8759);
\draw [color=c, fill=c] (4.14925,10.7701) rectangle (4.18905,10.8759);
\draw [color=c, fill=c] (4.18905,10.7701) rectangle (4.22886,10.8759);
\draw [color=c, fill=c] (4.22886,10.7701) rectangle (4.26866,10.8759);
\draw [color=c, fill=c] (4.26866,10.7701) rectangle (4.30846,10.8759);
\draw [color=c, fill=c] (4.30846,10.7701) rectangle (4.34826,10.8759);
\draw [color=c, fill=c] (4.34826,10.7701) rectangle (4.38806,10.8759);
\draw [color=c, fill=c] (4.38806,10.7701) rectangle (4.42786,10.8759);
\draw [color=c, fill=c] (4.42786,10.7701) rectangle (4.46766,10.8759);
\draw [color=c, fill=c] (4.46766,10.7701) rectangle (4.50746,10.8759);
\draw [color=c, fill=c] (4.50746,10.7701) rectangle (4.54726,10.8759);
\draw [color=c, fill=c] (4.54726,10.7701) rectangle (4.58706,10.8759);
\draw [color=c, fill=c] (4.58706,10.7701) rectangle (4.62687,10.8759);
\draw [color=c, fill=c] (4.62687,10.7701) rectangle (4.66667,10.8759);
\draw [color=c, fill=c] (4.66667,10.7701) rectangle (4.70647,10.8759);
\draw [color=c, fill=c] (4.70647,10.7701) rectangle (4.74627,10.8759);
\draw [color=c, fill=c] (4.74627,10.7701) rectangle (4.78607,10.8759);
\draw [color=c, fill=c] (4.78607,10.7701) rectangle (4.82587,10.8759);
\draw [color=c, fill=c] (4.82587,10.7701) rectangle (4.86567,10.8759);
\draw [color=c, fill=c] (4.86567,10.7701) rectangle (4.90547,10.8759);
\draw [color=c, fill=c] (4.90547,10.7701) rectangle (4.94527,10.8759);
\draw [color=c, fill=c] (4.94527,10.7701) rectangle (4.98507,10.8759);
\draw [color=c, fill=c] (4.98507,10.7701) rectangle (5.02488,10.8759);
\draw [color=c, fill=c] (5.02488,10.7701) rectangle (5.06468,10.8759);
\draw [color=c, fill=c] (5.06468,10.7701) rectangle (5.10448,10.8759);
\draw [color=c, fill=c] (5.10448,10.7701) rectangle (5.14428,10.8759);
\draw [color=c, fill=c] (5.14428,10.7701) rectangle (5.18408,10.8759);
\draw [color=c, fill=c] (5.18408,10.7701) rectangle (5.22388,10.8759);
\draw [color=c, fill=c] (5.22388,10.7701) rectangle (5.26368,10.8759);
\draw [color=c, fill=c] (5.26368,10.7701) rectangle (5.30348,10.8759);
\draw [color=c, fill=c] (5.30348,10.7701) rectangle (5.34328,10.8759);
\draw [color=c, fill=c] (5.34328,10.7701) rectangle (5.38308,10.8759);
\draw [color=c, fill=c] (5.38308,10.7701) rectangle (5.42289,10.8759);
\draw [color=c, fill=c] (5.42289,10.7701) rectangle (5.46269,10.8759);
\draw [color=c, fill=c] (5.46269,10.7701) rectangle (5.50249,10.8759);
\draw [color=c, fill=c] (5.50249,10.7701) rectangle (5.54229,10.8759);
\draw [color=c, fill=c] (5.54229,10.7701) rectangle (5.58209,10.8759);
\draw [color=c, fill=c] (5.58209,10.7701) rectangle (5.62189,10.8759);
\draw [color=c, fill=c] (5.62189,10.7701) rectangle (5.66169,10.8759);
\draw [color=c, fill=c] (5.66169,10.7701) rectangle (5.70149,10.8759);
\draw [color=c, fill=c] (5.70149,10.7701) rectangle (5.74129,10.8759);
\draw [color=c, fill=c] (5.74129,10.7701) rectangle (5.78109,10.8759);
\draw [color=c, fill=c] (5.78109,10.7701) rectangle (5.8209,10.8759);
\draw [color=c, fill=c] (5.8209,10.7701) rectangle (5.8607,10.8759);
\draw [color=c, fill=c] (5.8607,10.7701) rectangle (5.9005,10.8759);
\draw [color=c, fill=c] (5.9005,10.7701) rectangle (5.9403,10.8759);
\draw [color=c, fill=c] (5.9403,10.7701) rectangle (5.9801,10.8759);
\draw [color=c, fill=c] (5.9801,10.7701) rectangle (6.0199,10.8759);
\draw [color=c, fill=c] (6.0199,10.7701) rectangle (6.0597,10.8759);
\draw [color=c, fill=c] (6.0597,10.7701) rectangle (6.0995,10.8759);
\draw [color=c, fill=c] (6.0995,10.7701) rectangle (6.1393,10.8759);
\draw [color=c, fill=c] (6.1393,10.7701) rectangle (6.1791,10.8759);
\draw [color=c, fill=c] (6.1791,10.7701) rectangle (6.21891,10.8759);
\draw [color=c, fill=c] (6.21891,10.7701) rectangle (6.25871,10.8759);
\draw [color=c, fill=c] (6.25871,10.7701) rectangle (6.29851,10.8759);
\draw [color=c, fill=c] (6.29851,10.7701) rectangle (6.33831,10.8759);
\draw [color=c, fill=c] (6.33831,10.7701) rectangle (6.37811,10.8759);
\draw [color=c, fill=c] (6.37811,10.7701) rectangle (6.41791,10.8759);
\draw [color=c, fill=c] (6.41791,10.7701) rectangle (6.45771,10.8759);
\draw [color=c, fill=c] (6.45771,10.7701) rectangle (6.49751,10.8759);
\draw [color=c, fill=c] (6.49751,10.7701) rectangle (6.53731,10.8759);
\draw [color=c, fill=c] (6.53731,10.7701) rectangle (6.57711,10.8759);
\draw [color=c, fill=c] (6.57711,10.7701) rectangle (6.61692,10.8759);
\draw [color=c, fill=c] (6.61692,10.7701) rectangle (6.65672,10.8759);
\draw [color=c, fill=c] (6.65672,10.7701) rectangle (6.69652,10.8759);
\draw [color=c, fill=c] (6.69652,10.7701) rectangle (6.73632,10.8759);
\draw [color=c, fill=c] (6.73632,10.7701) rectangle (6.77612,10.8759);
\draw [color=c, fill=c] (6.77612,10.7701) rectangle (6.81592,10.8759);
\draw [color=c, fill=c] (6.81592,10.7701) rectangle (6.85572,10.8759);
\draw [color=c, fill=c] (6.85572,10.7701) rectangle (6.89552,10.8759);
\draw [color=c, fill=c] (6.89552,10.7701) rectangle (6.93532,10.8759);
\draw [color=c, fill=c] (6.93532,10.7701) rectangle (6.97512,10.8759);
\draw [color=c, fill=c] (6.97512,10.7701) rectangle (7.01493,10.8759);
\draw [color=c, fill=c] (7.01493,10.7701) rectangle (7.05473,10.8759);
\draw [color=c, fill=c] (7.05473,10.7701) rectangle (7.09453,10.8759);
\draw [color=c, fill=c] (7.09453,10.7701) rectangle (7.13433,10.8759);
\draw [color=c, fill=c] (7.13433,10.7701) rectangle (7.17413,10.8759);
\draw [color=c, fill=c] (7.17413,10.7701) rectangle (7.21393,10.8759);
\draw [color=c, fill=c] (7.21393,10.7701) rectangle (7.25373,10.8759);
\draw [color=c, fill=c] (7.25373,10.7701) rectangle (7.29353,10.8759);
\draw [color=c, fill=c] (7.29353,10.7701) rectangle (7.33333,10.8759);
\draw [color=c, fill=c] (7.33333,10.7701) rectangle (7.37313,10.8759);
\draw [color=c, fill=c] (7.37313,10.7701) rectangle (7.41294,10.8759);
\draw [color=c, fill=c] (7.41294,10.7701) rectangle (7.45274,10.8759);
\draw [color=c, fill=c] (7.45274,10.7701) rectangle (7.49254,10.8759);
\draw [color=c, fill=c] (7.49254,10.7701) rectangle (7.53234,10.8759);
\draw [color=c, fill=c] (7.53234,10.7701) rectangle (7.57214,10.8759);
\draw [color=c, fill=c] (7.57214,10.7701) rectangle (7.61194,10.8759);
\draw [color=c, fill=c] (7.61194,10.7701) rectangle (7.65174,10.8759);
\draw [color=c, fill=c] (7.65174,10.7701) rectangle (7.69154,10.8759);
\draw [color=c, fill=c] (7.69154,10.7701) rectangle (7.73134,10.8759);
\draw [color=c, fill=c] (7.73134,10.7701) rectangle (7.77114,10.8759);
\draw [color=c, fill=c] (7.77114,10.7701) rectangle (7.81095,10.8759);
\draw [color=c, fill=c] (7.81095,10.7701) rectangle (7.85075,10.8759);
\draw [color=c, fill=c] (7.85075,10.7701) rectangle (7.89055,10.8759);
\definecolor{c}{rgb}{0,0.0800001,1};
\draw [color=c, fill=c] (7.89055,10.7701) rectangle (7.93035,10.8759);
\draw [color=c, fill=c] (7.93035,10.7701) rectangle (7.97015,10.8759);
\draw [color=c, fill=c] (7.97015,10.7701) rectangle (8.00995,10.8759);
\draw [color=c, fill=c] (8.00995,10.7701) rectangle (8.04975,10.8759);
\draw [color=c, fill=c] (8.04975,10.7701) rectangle (8.08955,10.8759);
\draw [color=c, fill=c] (8.08955,10.7701) rectangle (8.12935,10.8759);
\draw [color=c, fill=c] (8.12935,10.7701) rectangle (8.16915,10.8759);
\draw [color=c, fill=c] (8.16915,10.7701) rectangle (8.20895,10.8759);
\draw [color=c, fill=c] (8.20895,10.7701) rectangle (8.24876,10.8759);
\draw [color=c, fill=c] (8.24876,10.7701) rectangle (8.28856,10.8759);
\draw [color=c, fill=c] (8.28856,10.7701) rectangle (8.32836,10.8759);
\draw [color=c, fill=c] (8.32836,10.7701) rectangle (8.36816,10.8759);
\draw [color=c, fill=c] (8.36816,10.7701) rectangle (8.40796,10.8759);
\draw [color=c, fill=c] (8.40796,10.7701) rectangle (8.44776,10.8759);
\draw [color=c, fill=c] (8.44776,10.7701) rectangle (8.48756,10.8759);
\draw [color=c, fill=c] (8.48756,10.7701) rectangle (8.52736,10.8759);
\draw [color=c, fill=c] (8.52736,10.7701) rectangle (8.56716,10.8759);
\draw [color=c, fill=c] (8.56716,10.7701) rectangle (8.60697,10.8759);
\draw [color=c, fill=c] (8.60697,10.7701) rectangle (8.64677,10.8759);
\draw [color=c, fill=c] (8.64677,10.7701) rectangle (8.68657,10.8759);
\draw [color=c, fill=c] (8.68657,10.7701) rectangle (8.72637,10.8759);
\draw [color=c, fill=c] (8.72637,10.7701) rectangle (8.76617,10.8759);
\draw [color=c, fill=c] (8.76617,10.7701) rectangle (8.80597,10.8759);
\draw [color=c, fill=c] (8.80597,10.7701) rectangle (8.84577,10.8759);
\draw [color=c, fill=c] (8.84577,10.7701) rectangle (8.88557,10.8759);
\draw [color=c, fill=c] (8.88557,10.7701) rectangle (8.92537,10.8759);
\draw [color=c, fill=c] (8.92537,10.7701) rectangle (8.96517,10.8759);
\draw [color=c, fill=c] (8.96517,10.7701) rectangle (9.00498,10.8759);
\draw [color=c, fill=c] (9.00498,10.7701) rectangle (9.04478,10.8759);
\draw [color=c, fill=c] (9.04478,10.7701) rectangle (9.08458,10.8759);
\draw [color=c, fill=c] (9.08458,10.7701) rectangle (9.12438,10.8759);
\draw [color=c, fill=c] (9.12438,10.7701) rectangle (9.16418,10.8759);
\draw [color=c, fill=c] (9.16418,10.7701) rectangle (9.20398,10.8759);
\draw [color=c, fill=c] (9.20398,10.7701) rectangle (9.24378,10.8759);
\draw [color=c, fill=c] (9.24378,10.7701) rectangle (9.28358,10.8759);
\draw [color=c, fill=c] (9.28358,10.7701) rectangle (9.32338,10.8759);
\draw [color=c, fill=c] (9.32338,10.7701) rectangle (9.36318,10.8759);
\draw [color=c, fill=c] (9.36318,10.7701) rectangle (9.40298,10.8759);
\draw [color=c, fill=c] (9.40298,10.7701) rectangle (9.44279,10.8759);
\draw [color=c, fill=c] (9.44279,10.7701) rectangle (9.48259,10.8759);
\draw [color=c, fill=c] (9.48259,10.7701) rectangle (9.52239,10.8759);
\draw [color=c, fill=c] (9.52239,10.7701) rectangle (9.56219,10.8759);
\draw [color=c, fill=c] (9.56219,10.7701) rectangle (9.60199,10.8759);
\definecolor{c}{rgb}{0,0.266667,1};
\draw [color=c, fill=c] (9.60199,10.7701) rectangle (9.64179,10.8759);
\draw [color=c, fill=c] (9.64179,10.7701) rectangle (9.68159,10.8759);
\draw [color=c, fill=c] (9.68159,10.7701) rectangle (9.72139,10.8759);
\draw [color=c, fill=c] (9.72139,10.7701) rectangle (9.76119,10.8759);
\draw [color=c, fill=c] (9.76119,10.7701) rectangle (9.80099,10.8759);
\draw [color=c, fill=c] (9.80099,10.7701) rectangle (9.8408,10.8759);
\draw [color=c, fill=c] (9.8408,10.7701) rectangle (9.8806,10.8759);
\draw [color=c, fill=c] (9.8806,10.7701) rectangle (9.9204,10.8759);
\draw [color=c, fill=c] (9.9204,10.7701) rectangle (9.9602,10.8759);
\draw [color=c, fill=c] (9.9602,10.7701) rectangle (10,10.8759);
\draw [color=c, fill=c] (10,10.7701) rectangle (10.0398,10.8759);
\draw [color=c, fill=c] (10.0398,10.7701) rectangle (10.0796,10.8759);
\draw [color=c, fill=c] (10.0796,10.7701) rectangle (10.1194,10.8759);
\draw [color=c, fill=c] (10.1194,10.7701) rectangle (10.1592,10.8759);
\draw [color=c, fill=c] (10.1592,10.7701) rectangle (10.199,10.8759);
\draw [color=c, fill=c] (10.199,10.7701) rectangle (10.2388,10.8759);
\draw [color=c, fill=c] (10.2388,10.7701) rectangle (10.2786,10.8759);
\draw [color=c, fill=c] (10.2786,10.7701) rectangle (10.3184,10.8759);
\draw [color=c, fill=c] (10.3184,10.7701) rectangle (10.3582,10.8759);
\draw [color=c, fill=c] (10.3582,10.7701) rectangle (10.398,10.8759);
\draw [color=c, fill=c] (10.398,10.7701) rectangle (10.4378,10.8759);
\draw [color=c, fill=c] (10.4378,10.7701) rectangle (10.4776,10.8759);
\draw [color=c, fill=c] (10.4776,10.7701) rectangle (10.5174,10.8759);
\draw [color=c, fill=c] (10.5174,10.7701) rectangle (10.5572,10.8759);
\draw [color=c, fill=c] (10.5572,10.7701) rectangle (10.597,10.8759);
\draw [color=c, fill=c] (10.597,10.7701) rectangle (10.6368,10.8759);
\draw [color=c, fill=c] (10.6368,10.7701) rectangle (10.6766,10.8759);
\draw [color=c, fill=c] (10.6766,10.7701) rectangle (10.7164,10.8759);
\draw [color=c, fill=c] (10.7164,10.7701) rectangle (10.7562,10.8759);
\draw [color=c, fill=c] (10.7562,10.7701) rectangle (10.796,10.8759);
\draw [color=c, fill=c] (10.796,10.7701) rectangle (10.8358,10.8759);
\draw [color=c, fill=c] (10.8358,10.7701) rectangle (10.8756,10.8759);
\draw [color=c, fill=c] (10.8756,10.7701) rectangle (10.9154,10.8759);
\definecolor{c}{rgb}{0,0.546666,1};
\draw [color=c, fill=c] (10.9154,10.7701) rectangle (10.9552,10.8759);
\draw [color=c, fill=c] (10.9552,10.7701) rectangle (10.995,10.8759);
\draw [color=c, fill=c] (10.995,10.7701) rectangle (11.0348,10.8759);
\draw [color=c, fill=c] (11.0348,10.7701) rectangle (11.0746,10.8759);
\draw [color=c, fill=c] (11.0746,10.7701) rectangle (11.1144,10.8759);
\draw [color=c, fill=c] (11.1144,10.7701) rectangle (11.1542,10.8759);
\draw [color=c, fill=c] (11.1542,10.7701) rectangle (11.194,10.8759);
\draw [color=c, fill=c] (11.194,10.7701) rectangle (11.2338,10.8759);
\draw [color=c, fill=c] (11.2338,10.7701) rectangle (11.2736,10.8759);
\draw [color=c, fill=c] (11.2736,10.7701) rectangle (11.3134,10.8759);
\draw [color=c, fill=c] (11.3134,10.7701) rectangle (11.3532,10.8759);
\draw [color=c, fill=c] (11.3532,10.7701) rectangle (11.393,10.8759);
\draw [color=c, fill=c] (11.393,10.7701) rectangle (11.4328,10.8759);
\draw [color=c, fill=c] (11.4328,10.7701) rectangle (11.4726,10.8759);
\draw [color=c, fill=c] (11.4726,10.7701) rectangle (11.5124,10.8759);
\draw [color=c, fill=c] (11.5124,10.7701) rectangle (11.5522,10.8759);
\draw [color=c, fill=c] (11.5522,10.7701) rectangle (11.592,10.8759);
\draw [color=c, fill=c] (11.592,10.7701) rectangle (11.6318,10.8759);
\draw [color=c, fill=c] (11.6318,10.7701) rectangle (11.6716,10.8759);
\draw [color=c, fill=c] (11.6716,10.7701) rectangle (11.7114,10.8759);
\draw [color=c, fill=c] (11.7114,10.7701) rectangle (11.7512,10.8759);
\draw [color=c, fill=c] (11.7512,10.7701) rectangle (11.791,10.8759);
\draw [color=c, fill=c] (11.791,10.7701) rectangle (11.8308,10.8759);
\draw [color=c, fill=c] (11.8308,10.7701) rectangle (11.8706,10.8759);
\draw [color=c, fill=c] (11.8706,10.7701) rectangle (11.9104,10.8759);
\draw [color=c, fill=c] (11.9104,10.7701) rectangle (11.9502,10.8759);
\draw [color=c, fill=c] (11.9502,10.7701) rectangle (11.99,10.8759);
\draw [color=c, fill=c] (11.99,10.7701) rectangle (12.0299,10.8759);
\draw [color=c, fill=c] (12.0299,10.7701) rectangle (12.0697,10.8759);
\draw [color=c, fill=c] (12.0697,10.7701) rectangle (12.1095,10.8759);
\draw [color=c, fill=c] (12.1095,10.7701) rectangle (12.1493,10.8759);
\draw [color=c, fill=c] (12.1493,10.7701) rectangle (12.1891,10.8759);
\draw [color=c, fill=c] (12.1891,10.7701) rectangle (12.2289,10.8759);
\draw [color=c, fill=c] (12.2289,10.7701) rectangle (12.2687,10.8759);
\draw [color=c, fill=c] (12.2687,10.7701) rectangle (12.3085,10.8759);
\draw [color=c, fill=c] (12.3085,10.7701) rectangle (12.3483,10.8759);
\draw [color=c, fill=c] (12.3483,10.7701) rectangle (12.3881,10.8759);
\draw [color=c, fill=c] (12.3881,10.7701) rectangle (12.4279,10.8759);
\draw [color=c, fill=c] (12.4279,10.7701) rectangle (12.4677,10.8759);
\draw [color=c, fill=c] (12.4677,10.7701) rectangle (12.5075,10.8759);
\draw [color=c, fill=c] (12.5075,10.7701) rectangle (12.5473,10.8759);
\draw [color=c, fill=c] (12.5473,10.7701) rectangle (12.5871,10.8759);
\draw [color=c, fill=c] (12.5871,10.7701) rectangle (12.6269,10.8759);
\draw [color=c, fill=c] (12.6269,10.7701) rectangle (12.6667,10.8759);
\draw [color=c, fill=c] (12.6667,10.7701) rectangle (12.7065,10.8759);
\draw [color=c, fill=c] (12.7065,10.7701) rectangle (12.7463,10.8759);
\draw [color=c, fill=c] (12.7463,10.7701) rectangle (12.7861,10.8759);
\draw [color=c, fill=c] (12.7861,10.7701) rectangle (12.8259,10.8759);
\draw [color=c, fill=c] (12.8259,10.7701) rectangle (12.8657,10.8759);
\draw [color=c, fill=c] (12.8657,10.7701) rectangle (12.9055,10.8759);
\draw [color=c, fill=c] (12.9055,10.7701) rectangle (12.9453,10.8759);
\draw [color=c, fill=c] (12.9453,10.7701) rectangle (12.9851,10.8759);
\draw [color=c, fill=c] (12.9851,10.7701) rectangle (13.0249,10.8759);
\draw [color=c, fill=c] (13.0249,10.7701) rectangle (13.0647,10.8759);
\draw [color=c, fill=c] (13.0647,10.7701) rectangle (13.1045,10.8759);
\draw [color=c, fill=c] (13.1045,10.7701) rectangle (13.1443,10.8759);
\draw [color=c, fill=c] (13.1443,10.7701) rectangle (13.1841,10.8759);
\draw [color=c, fill=c] (13.1841,10.7701) rectangle (13.2239,10.8759);
\draw [color=c, fill=c] (13.2239,10.7701) rectangle (13.2637,10.8759);
\draw [color=c, fill=c] (13.2637,10.7701) rectangle (13.3035,10.8759);
\draw [color=c, fill=c] (13.3035,10.7701) rectangle (13.3433,10.8759);
\draw [color=c, fill=c] (13.3433,10.7701) rectangle (13.3831,10.8759);
\draw [color=c, fill=c] (13.3831,10.7701) rectangle (13.4229,10.8759);
\draw [color=c, fill=c] (13.4229,10.7701) rectangle (13.4627,10.8759);
\draw [color=c, fill=c] (13.4627,10.7701) rectangle (13.5025,10.8759);
\draw [color=c, fill=c] (13.5025,10.7701) rectangle (13.5423,10.8759);
\draw [color=c, fill=c] (13.5423,10.7701) rectangle (13.5821,10.8759);
\draw [color=c, fill=c] (13.5821,10.7701) rectangle (13.6219,10.8759);
\draw [color=c, fill=c] (13.6219,10.7701) rectangle (13.6617,10.8759);
\draw [color=c, fill=c] (13.6617,10.7701) rectangle (13.7015,10.8759);
\draw [color=c, fill=c] (13.7015,10.7701) rectangle (13.7413,10.8759);
\draw [color=c, fill=c] (13.7413,10.7701) rectangle (13.7811,10.8759);
\draw [color=c, fill=c] (13.7811,10.7701) rectangle (13.8209,10.8759);
\draw [color=c, fill=c] (13.8209,10.7701) rectangle (13.8607,10.8759);
\draw [color=c, fill=c] (13.8607,10.7701) rectangle (13.9005,10.8759);
\draw [color=c, fill=c] (13.9005,10.7701) rectangle (13.9403,10.8759);
\draw [color=c, fill=c] (13.9403,10.7701) rectangle (13.9801,10.8759);
\draw [color=c, fill=c] (13.9801,10.7701) rectangle (14.0199,10.8759);
\draw [color=c, fill=c] (14.0199,10.7701) rectangle (14.0597,10.8759);
\draw [color=c, fill=c] (14.0597,10.7701) rectangle (14.0995,10.8759);
\draw [color=c, fill=c] (14.0995,10.7701) rectangle (14.1393,10.8759);
\draw [color=c, fill=c] (14.1393,10.7701) rectangle (14.1791,10.8759);
\draw [color=c, fill=c] (14.1791,10.7701) rectangle (14.2189,10.8759);
\definecolor{c}{rgb}{0,0.733333,1};
\draw [color=c, fill=c] (14.2189,10.7701) rectangle (14.2587,10.8759);
\draw [color=c, fill=c] (14.2587,10.7701) rectangle (14.2985,10.8759);
\draw [color=c, fill=c] (14.2985,10.7701) rectangle (14.3383,10.8759);
\draw [color=c, fill=c] (14.3383,10.7701) rectangle (14.3781,10.8759);
\draw [color=c, fill=c] (14.3781,10.7701) rectangle (14.4179,10.8759);
\draw [color=c, fill=c] (14.4179,10.7701) rectangle (14.4577,10.8759);
\draw [color=c, fill=c] (14.4577,10.7701) rectangle (14.4975,10.8759);
\draw [color=c, fill=c] (14.4975,10.7701) rectangle (14.5373,10.8759);
\draw [color=c, fill=c] (14.5373,10.7701) rectangle (14.5771,10.8759);
\draw [color=c, fill=c] (14.5771,10.7701) rectangle (14.6169,10.8759);
\draw [color=c, fill=c] (14.6169,10.7701) rectangle (14.6567,10.8759);
\draw [color=c, fill=c] (14.6567,10.7701) rectangle (14.6965,10.8759);
\draw [color=c, fill=c] (14.6965,10.7701) rectangle (14.7363,10.8759);
\draw [color=c, fill=c] (14.7363,10.7701) rectangle (14.7761,10.8759);
\draw [color=c, fill=c] (14.7761,10.7701) rectangle (14.8159,10.8759);
\draw [color=c, fill=c] (14.8159,10.7701) rectangle (14.8557,10.8759);
\draw [color=c, fill=c] (14.8557,10.7701) rectangle (14.8955,10.8759);
\draw [color=c, fill=c] (14.8955,10.7701) rectangle (14.9353,10.8759);
\draw [color=c, fill=c] (14.9353,10.7701) rectangle (14.9751,10.8759);
\draw [color=c, fill=c] (14.9751,10.7701) rectangle (15.0149,10.8759);
\draw [color=c, fill=c] (15.0149,10.7701) rectangle (15.0547,10.8759);
\draw [color=c, fill=c] (15.0547,10.7701) rectangle (15.0945,10.8759);
\draw [color=c, fill=c] (15.0945,10.7701) rectangle (15.1343,10.8759);
\draw [color=c, fill=c] (15.1343,10.7701) rectangle (15.1741,10.8759);
\draw [color=c, fill=c] (15.1741,10.7701) rectangle (15.2139,10.8759);
\draw [color=c, fill=c] (15.2139,10.7701) rectangle (15.2537,10.8759);
\draw [color=c, fill=c] (15.2537,10.7701) rectangle (15.2935,10.8759);
\draw [color=c, fill=c] (15.2935,10.7701) rectangle (15.3333,10.8759);
\draw [color=c, fill=c] (15.3333,10.7701) rectangle (15.3731,10.8759);
\draw [color=c, fill=c] (15.3731,10.7701) rectangle (15.4129,10.8759);
\draw [color=c, fill=c] (15.4129,10.7701) rectangle (15.4527,10.8759);
\draw [color=c, fill=c] (15.4527,10.7701) rectangle (15.4925,10.8759);
\draw [color=c, fill=c] (15.4925,10.7701) rectangle (15.5323,10.8759);
\draw [color=c, fill=c] (15.5323,10.7701) rectangle (15.5721,10.8759);
\draw [color=c, fill=c] (15.5721,10.7701) rectangle (15.6119,10.8759);
\draw [color=c, fill=c] (15.6119,10.7701) rectangle (15.6517,10.8759);
\draw [color=c, fill=c] (15.6517,10.7701) rectangle (15.6915,10.8759);
\draw [color=c, fill=c] (15.6915,10.7701) rectangle (15.7313,10.8759);
\draw [color=c, fill=c] (15.7313,10.7701) rectangle (15.7711,10.8759);
\draw [color=c, fill=c] (15.7711,10.7701) rectangle (15.8109,10.8759);
\draw [color=c, fill=c] (15.8109,10.7701) rectangle (15.8507,10.8759);
\draw [color=c, fill=c] (15.8507,10.7701) rectangle (15.8905,10.8759);
\draw [color=c, fill=c] (15.8905,10.7701) rectangle (15.9303,10.8759);
\draw [color=c, fill=c] (15.9303,10.7701) rectangle (15.9701,10.8759);
\draw [color=c, fill=c] (15.9701,10.7701) rectangle (16.01,10.8759);
\draw [color=c, fill=c] (16.01,10.7701) rectangle (16.0498,10.8759);
\draw [color=c, fill=c] (16.0498,10.7701) rectangle (16.0896,10.8759);
\draw [color=c, fill=c] (16.0896,10.7701) rectangle (16.1294,10.8759);
\draw [color=c, fill=c] (16.1294,10.7701) rectangle (16.1692,10.8759);
\draw [color=c, fill=c] (16.1692,10.7701) rectangle (16.209,10.8759);
\draw [color=c, fill=c] (16.209,10.7701) rectangle (16.2488,10.8759);
\draw [color=c, fill=c] (16.2488,10.7701) rectangle (16.2886,10.8759);
\draw [color=c, fill=c] (16.2886,10.7701) rectangle (16.3284,10.8759);
\draw [color=c, fill=c] (16.3284,10.7701) rectangle (16.3682,10.8759);
\draw [color=c, fill=c] (16.3682,10.7701) rectangle (16.408,10.8759);
\draw [color=c, fill=c] (16.408,10.7701) rectangle (16.4478,10.8759);
\draw [color=c, fill=c] (16.4478,10.7701) rectangle (16.4876,10.8759);
\draw [color=c, fill=c] (16.4876,10.7701) rectangle (16.5274,10.8759);
\draw [color=c, fill=c] (16.5274,10.7701) rectangle (16.5672,10.8759);
\draw [color=c, fill=c] (16.5672,10.7701) rectangle (16.607,10.8759);
\draw [color=c, fill=c] (16.607,10.7701) rectangle (16.6468,10.8759);
\draw [color=c, fill=c] (16.6468,10.7701) rectangle (16.6866,10.8759);
\draw [color=c, fill=c] (16.6866,10.7701) rectangle (16.7264,10.8759);
\draw [color=c, fill=c] (16.7264,10.7701) rectangle (16.7662,10.8759);
\draw [color=c, fill=c] (16.7662,10.7701) rectangle (16.806,10.8759);
\draw [color=c, fill=c] (16.806,10.7701) rectangle (16.8458,10.8759);
\draw [color=c, fill=c] (16.8458,10.7701) rectangle (16.8856,10.8759);
\draw [color=c, fill=c] (16.8856,10.7701) rectangle (16.9254,10.8759);
\draw [color=c, fill=c] (16.9254,10.7701) rectangle (16.9652,10.8759);
\draw [color=c, fill=c] (16.9652,10.7701) rectangle (17.005,10.8759);
\draw [color=c, fill=c] (17.005,10.7701) rectangle (17.0448,10.8759);
\draw [color=c, fill=c] (17.0448,10.7701) rectangle (17.0846,10.8759);
\draw [color=c, fill=c] (17.0846,10.7701) rectangle (17.1244,10.8759);
\draw [color=c, fill=c] (17.1244,10.7701) rectangle (17.1642,10.8759);
\draw [color=c, fill=c] (17.1642,10.7701) rectangle (17.204,10.8759);
\draw [color=c, fill=c] (17.204,10.7701) rectangle (17.2438,10.8759);
\draw [color=c, fill=c] (17.2438,10.7701) rectangle (17.2836,10.8759);
\draw [color=c, fill=c] (17.2836,10.7701) rectangle (17.3234,10.8759);
\draw [color=c, fill=c] (17.3234,10.7701) rectangle (17.3632,10.8759);
\draw [color=c, fill=c] (17.3632,10.7701) rectangle (17.403,10.8759);
\draw [color=c, fill=c] (17.403,10.7701) rectangle (17.4428,10.8759);
\draw [color=c, fill=c] (17.4428,10.7701) rectangle (17.4826,10.8759);
\draw [color=c, fill=c] (17.4826,10.7701) rectangle (17.5224,10.8759);
\draw [color=c, fill=c] (17.5224,10.7701) rectangle (17.5622,10.8759);
\draw [color=c, fill=c] (17.5622,10.7701) rectangle (17.602,10.8759);
\draw [color=c, fill=c] (17.602,10.7701) rectangle (17.6418,10.8759);
\draw [color=c, fill=c] (17.6418,10.7701) rectangle (17.6816,10.8759);
\draw [color=c, fill=c] (17.6816,10.7701) rectangle (17.7214,10.8759);
\draw [color=c, fill=c] (17.7214,10.7701) rectangle (17.7612,10.8759);
\draw [color=c, fill=c] (17.7612,10.7701) rectangle (17.801,10.8759);
\draw [color=c, fill=c] (17.801,10.7701) rectangle (17.8408,10.8759);
\draw [color=c, fill=c] (17.8408,10.7701) rectangle (17.8806,10.8759);
\draw [color=c, fill=c] (17.8806,10.7701) rectangle (17.9204,10.8759);
\draw [color=c, fill=c] (17.9204,10.7701) rectangle (17.9602,10.8759);
\draw [color=c, fill=c] (17.9602,10.7701) rectangle (18,10.8759);
\definecolor{c}{rgb}{0.2,0,1};
\draw [color=c, fill=c] (2,10.8759) rectangle (2.0398,10.9818);
\draw [color=c, fill=c] (2.0398,10.8759) rectangle (2.0796,10.9818);
\draw [color=c, fill=c] (2.0796,10.8759) rectangle (2.1194,10.9818);
\draw [color=c, fill=c] (2.1194,10.8759) rectangle (2.1592,10.9818);
\draw [color=c, fill=c] (2.1592,10.8759) rectangle (2.19901,10.9818);
\draw [color=c, fill=c] (2.19901,10.8759) rectangle (2.23881,10.9818);
\draw [color=c, fill=c] (2.23881,10.8759) rectangle (2.27861,10.9818);
\draw [color=c, fill=c] (2.27861,10.8759) rectangle (2.31841,10.9818);
\draw [color=c, fill=c] (2.31841,10.8759) rectangle (2.35821,10.9818);
\draw [color=c, fill=c] (2.35821,10.8759) rectangle (2.39801,10.9818);
\draw [color=c, fill=c] (2.39801,10.8759) rectangle (2.43781,10.9818);
\draw [color=c, fill=c] (2.43781,10.8759) rectangle (2.47761,10.9818);
\draw [color=c, fill=c] (2.47761,10.8759) rectangle (2.51741,10.9818);
\draw [color=c, fill=c] (2.51741,10.8759) rectangle (2.55721,10.9818);
\draw [color=c, fill=c] (2.55721,10.8759) rectangle (2.59702,10.9818);
\draw [color=c, fill=c] (2.59702,10.8759) rectangle (2.63682,10.9818);
\draw [color=c, fill=c] (2.63682,10.8759) rectangle (2.67662,10.9818);
\draw [color=c, fill=c] (2.67662,10.8759) rectangle (2.71642,10.9818);
\draw [color=c, fill=c] (2.71642,10.8759) rectangle (2.75622,10.9818);
\draw [color=c, fill=c] (2.75622,10.8759) rectangle (2.79602,10.9818);
\draw [color=c, fill=c] (2.79602,10.8759) rectangle (2.83582,10.9818);
\draw [color=c, fill=c] (2.83582,10.8759) rectangle (2.87562,10.9818);
\draw [color=c, fill=c] (2.87562,10.8759) rectangle (2.91542,10.9818);
\draw [color=c, fill=c] (2.91542,10.8759) rectangle (2.95522,10.9818);
\draw [color=c, fill=c] (2.95522,10.8759) rectangle (2.99502,10.9818);
\draw [color=c, fill=c] (2.99502,10.8759) rectangle (3.03483,10.9818);
\draw [color=c, fill=c] (3.03483,10.8759) rectangle (3.07463,10.9818);
\draw [color=c, fill=c] (3.07463,10.8759) rectangle (3.11443,10.9818);
\draw [color=c, fill=c] (3.11443,10.8759) rectangle (3.15423,10.9818);
\draw [color=c, fill=c] (3.15423,10.8759) rectangle (3.19403,10.9818);
\draw [color=c, fill=c] (3.19403,10.8759) rectangle (3.23383,10.9818);
\draw [color=c, fill=c] (3.23383,10.8759) rectangle (3.27363,10.9818);
\draw [color=c, fill=c] (3.27363,10.8759) rectangle (3.31343,10.9818);
\draw [color=c, fill=c] (3.31343,10.8759) rectangle (3.35323,10.9818);
\draw [color=c, fill=c] (3.35323,10.8759) rectangle (3.39303,10.9818);
\draw [color=c, fill=c] (3.39303,10.8759) rectangle (3.43284,10.9818);
\draw [color=c, fill=c] (3.43284,10.8759) rectangle (3.47264,10.9818);
\draw [color=c, fill=c] (3.47264,10.8759) rectangle (3.51244,10.9818);
\draw [color=c, fill=c] (3.51244,10.8759) rectangle (3.55224,10.9818);
\draw [color=c, fill=c] (3.55224,10.8759) rectangle (3.59204,10.9818);
\draw [color=c, fill=c] (3.59204,10.8759) rectangle (3.63184,10.9818);
\draw [color=c, fill=c] (3.63184,10.8759) rectangle (3.67164,10.9818);
\draw [color=c, fill=c] (3.67164,10.8759) rectangle (3.71144,10.9818);
\draw [color=c, fill=c] (3.71144,10.8759) rectangle (3.75124,10.9818);
\draw [color=c, fill=c] (3.75124,10.8759) rectangle (3.79104,10.9818);
\draw [color=c, fill=c] (3.79104,10.8759) rectangle (3.83085,10.9818);
\draw [color=c, fill=c] (3.83085,10.8759) rectangle (3.87065,10.9818);
\draw [color=c, fill=c] (3.87065,10.8759) rectangle (3.91045,10.9818);
\draw [color=c, fill=c] (3.91045,10.8759) rectangle (3.95025,10.9818);
\draw [color=c, fill=c] (3.95025,10.8759) rectangle (3.99005,10.9818);
\draw [color=c, fill=c] (3.99005,10.8759) rectangle (4.02985,10.9818);
\draw [color=c, fill=c] (4.02985,10.8759) rectangle (4.06965,10.9818);
\draw [color=c, fill=c] (4.06965,10.8759) rectangle (4.10945,10.9818);
\draw [color=c, fill=c] (4.10945,10.8759) rectangle (4.14925,10.9818);
\draw [color=c, fill=c] (4.14925,10.8759) rectangle (4.18905,10.9818);
\draw [color=c, fill=c] (4.18905,10.8759) rectangle (4.22886,10.9818);
\draw [color=c, fill=c] (4.22886,10.8759) rectangle (4.26866,10.9818);
\draw [color=c, fill=c] (4.26866,10.8759) rectangle (4.30846,10.9818);
\draw [color=c, fill=c] (4.30846,10.8759) rectangle (4.34826,10.9818);
\draw [color=c, fill=c] (4.34826,10.8759) rectangle (4.38806,10.9818);
\draw [color=c, fill=c] (4.38806,10.8759) rectangle (4.42786,10.9818);
\draw [color=c, fill=c] (4.42786,10.8759) rectangle (4.46766,10.9818);
\draw [color=c, fill=c] (4.46766,10.8759) rectangle (4.50746,10.9818);
\draw [color=c, fill=c] (4.50746,10.8759) rectangle (4.54726,10.9818);
\draw [color=c, fill=c] (4.54726,10.8759) rectangle (4.58706,10.9818);
\draw [color=c, fill=c] (4.58706,10.8759) rectangle (4.62687,10.9818);
\draw [color=c, fill=c] (4.62687,10.8759) rectangle (4.66667,10.9818);
\draw [color=c, fill=c] (4.66667,10.8759) rectangle (4.70647,10.9818);
\draw [color=c, fill=c] (4.70647,10.8759) rectangle (4.74627,10.9818);
\draw [color=c, fill=c] (4.74627,10.8759) rectangle (4.78607,10.9818);
\draw [color=c, fill=c] (4.78607,10.8759) rectangle (4.82587,10.9818);
\draw [color=c, fill=c] (4.82587,10.8759) rectangle (4.86567,10.9818);
\draw [color=c, fill=c] (4.86567,10.8759) rectangle (4.90547,10.9818);
\draw [color=c, fill=c] (4.90547,10.8759) rectangle (4.94527,10.9818);
\draw [color=c, fill=c] (4.94527,10.8759) rectangle (4.98507,10.9818);
\draw [color=c, fill=c] (4.98507,10.8759) rectangle (5.02488,10.9818);
\draw [color=c, fill=c] (5.02488,10.8759) rectangle (5.06468,10.9818);
\draw [color=c, fill=c] (5.06468,10.8759) rectangle (5.10448,10.9818);
\draw [color=c, fill=c] (5.10448,10.8759) rectangle (5.14428,10.9818);
\draw [color=c, fill=c] (5.14428,10.8759) rectangle (5.18408,10.9818);
\draw [color=c, fill=c] (5.18408,10.8759) rectangle (5.22388,10.9818);
\draw [color=c, fill=c] (5.22388,10.8759) rectangle (5.26368,10.9818);
\draw [color=c, fill=c] (5.26368,10.8759) rectangle (5.30348,10.9818);
\draw [color=c, fill=c] (5.30348,10.8759) rectangle (5.34328,10.9818);
\draw [color=c, fill=c] (5.34328,10.8759) rectangle (5.38308,10.9818);
\draw [color=c, fill=c] (5.38308,10.8759) rectangle (5.42289,10.9818);
\draw [color=c, fill=c] (5.42289,10.8759) rectangle (5.46269,10.9818);
\draw [color=c, fill=c] (5.46269,10.8759) rectangle (5.50249,10.9818);
\draw [color=c, fill=c] (5.50249,10.8759) rectangle (5.54229,10.9818);
\draw [color=c, fill=c] (5.54229,10.8759) rectangle (5.58209,10.9818);
\draw [color=c, fill=c] (5.58209,10.8759) rectangle (5.62189,10.9818);
\draw [color=c, fill=c] (5.62189,10.8759) rectangle (5.66169,10.9818);
\draw [color=c, fill=c] (5.66169,10.8759) rectangle (5.70149,10.9818);
\draw [color=c, fill=c] (5.70149,10.8759) rectangle (5.74129,10.9818);
\draw [color=c, fill=c] (5.74129,10.8759) rectangle (5.78109,10.9818);
\draw [color=c, fill=c] (5.78109,10.8759) rectangle (5.8209,10.9818);
\draw [color=c, fill=c] (5.8209,10.8759) rectangle (5.8607,10.9818);
\draw [color=c, fill=c] (5.8607,10.8759) rectangle (5.9005,10.9818);
\draw [color=c, fill=c] (5.9005,10.8759) rectangle (5.9403,10.9818);
\draw [color=c, fill=c] (5.9403,10.8759) rectangle (5.9801,10.9818);
\draw [color=c, fill=c] (5.9801,10.8759) rectangle (6.0199,10.9818);
\draw [color=c, fill=c] (6.0199,10.8759) rectangle (6.0597,10.9818);
\draw [color=c, fill=c] (6.0597,10.8759) rectangle (6.0995,10.9818);
\draw [color=c, fill=c] (6.0995,10.8759) rectangle (6.1393,10.9818);
\draw [color=c, fill=c] (6.1393,10.8759) rectangle (6.1791,10.9818);
\draw [color=c, fill=c] (6.1791,10.8759) rectangle (6.21891,10.9818);
\draw [color=c, fill=c] (6.21891,10.8759) rectangle (6.25871,10.9818);
\draw [color=c, fill=c] (6.25871,10.8759) rectangle (6.29851,10.9818);
\draw [color=c, fill=c] (6.29851,10.8759) rectangle (6.33831,10.9818);
\draw [color=c, fill=c] (6.33831,10.8759) rectangle (6.37811,10.9818);
\draw [color=c, fill=c] (6.37811,10.8759) rectangle (6.41791,10.9818);
\draw [color=c, fill=c] (6.41791,10.8759) rectangle (6.45771,10.9818);
\draw [color=c, fill=c] (6.45771,10.8759) rectangle (6.49751,10.9818);
\draw [color=c, fill=c] (6.49751,10.8759) rectangle (6.53731,10.9818);
\draw [color=c, fill=c] (6.53731,10.8759) rectangle (6.57711,10.9818);
\draw [color=c, fill=c] (6.57711,10.8759) rectangle (6.61692,10.9818);
\draw [color=c, fill=c] (6.61692,10.8759) rectangle (6.65672,10.9818);
\draw [color=c, fill=c] (6.65672,10.8759) rectangle (6.69652,10.9818);
\draw [color=c, fill=c] (6.69652,10.8759) rectangle (6.73632,10.9818);
\draw [color=c, fill=c] (6.73632,10.8759) rectangle (6.77612,10.9818);
\draw [color=c, fill=c] (6.77612,10.8759) rectangle (6.81592,10.9818);
\draw [color=c, fill=c] (6.81592,10.8759) rectangle (6.85572,10.9818);
\draw [color=c, fill=c] (6.85572,10.8759) rectangle (6.89552,10.9818);
\draw [color=c, fill=c] (6.89552,10.8759) rectangle (6.93532,10.9818);
\draw [color=c, fill=c] (6.93532,10.8759) rectangle (6.97512,10.9818);
\draw [color=c, fill=c] (6.97512,10.8759) rectangle (7.01493,10.9818);
\draw [color=c, fill=c] (7.01493,10.8759) rectangle (7.05473,10.9818);
\draw [color=c, fill=c] (7.05473,10.8759) rectangle (7.09453,10.9818);
\draw [color=c, fill=c] (7.09453,10.8759) rectangle (7.13433,10.9818);
\draw [color=c, fill=c] (7.13433,10.8759) rectangle (7.17413,10.9818);
\draw [color=c, fill=c] (7.17413,10.8759) rectangle (7.21393,10.9818);
\draw [color=c, fill=c] (7.21393,10.8759) rectangle (7.25373,10.9818);
\draw [color=c, fill=c] (7.25373,10.8759) rectangle (7.29353,10.9818);
\draw [color=c, fill=c] (7.29353,10.8759) rectangle (7.33333,10.9818);
\draw [color=c, fill=c] (7.33333,10.8759) rectangle (7.37313,10.9818);
\draw [color=c, fill=c] (7.37313,10.8759) rectangle (7.41294,10.9818);
\draw [color=c, fill=c] (7.41294,10.8759) rectangle (7.45274,10.9818);
\draw [color=c, fill=c] (7.45274,10.8759) rectangle (7.49254,10.9818);
\draw [color=c, fill=c] (7.49254,10.8759) rectangle (7.53234,10.9818);
\draw [color=c, fill=c] (7.53234,10.8759) rectangle (7.57214,10.9818);
\draw [color=c, fill=c] (7.57214,10.8759) rectangle (7.61194,10.9818);
\draw [color=c, fill=c] (7.61194,10.8759) rectangle (7.65174,10.9818);
\draw [color=c, fill=c] (7.65174,10.8759) rectangle (7.69154,10.9818);
\draw [color=c, fill=c] (7.69154,10.8759) rectangle (7.73134,10.9818);
\draw [color=c, fill=c] (7.73134,10.8759) rectangle (7.77114,10.9818);
\draw [color=c, fill=c] (7.77114,10.8759) rectangle (7.81095,10.9818);
\draw [color=c, fill=c] (7.81095,10.8759) rectangle (7.85075,10.9818);
\draw [color=c, fill=c] (7.85075,10.8759) rectangle (7.89055,10.9818);
\definecolor{c}{rgb}{0,0.0800001,1};
\draw [color=c, fill=c] (7.89055,10.8759) rectangle (7.93035,10.9818);
\draw [color=c, fill=c] (7.93035,10.8759) rectangle (7.97015,10.9818);
\draw [color=c, fill=c] (7.97015,10.8759) rectangle (8.00995,10.9818);
\draw [color=c, fill=c] (8.00995,10.8759) rectangle (8.04975,10.9818);
\draw [color=c, fill=c] (8.04975,10.8759) rectangle (8.08955,10.9818);
\draw [color=c, fill=c] (8.08955,10.8759) rectangle (8.12935,10.9818);
\draw [color=c, fill=c] (8.12935,10.8759) rectangle (8.16915,10.9818);
\draw [color=c, fill=c] (8.16915,10.8759) rectangle (8.20895,10.9818);
\draw [color=c, fill=c] (8.20895,10.8759) rectangle (8.24876,10.9818);
\draw [color=c, fill=c] (8.24876,10.8759) rectangle (8.28856,10.9818);
\draw [color=c, fill=c] (8.28856,10.8759) rectangle (8.32836,10.9818);
\draw [color=c, fill=c] (8.32836,10.8759) rectangle (8.36816,10.9818);
\draw [color=c, fill=c] (8.36816,10.8759) rectangle (8.40796,10.9818);
\draw [color=c, fill=c] (8.40796,10.8759) rectangle (8.44776,10.9818);
\draw [color=c, fill=c] (8.44776,10.8759) rectangle (8.48756,10.9818);
\draw [color=c, fill=c] (8.48756,10.8759) rectangle (8.52736,10.9818);
\draw [color=c, fill=c] (8.52736,10.8759) rectangle (8.56716,10.9818);
\draw [color=c, fill=c] (8.56716,10.8759) rectangle (8.60697,10.9818);
\draw [color=c, fill=c] (8.60697,10.8759) rectangle (8.64677,10.9818);
\draw [color=c, fill=c] (8.64677,10.8759) rectangle (8.68657,10.9818);
\draw [color=c, fill=c] (8.68657,10.8759) rectangle (8.72637,10.9818);
\draw [color=c, fill=c] (8.72637,10.8759) rectangle (8.76617,10.9818);
\draw [color=c, fill=c] (8.76617,10.8759) rectangle (8.80597,10.9818);
\draw [color=c, fill=c] (8.80597,10.8759) rectangle (8.84577,10.9818);
\draw [color=c, fill=c] (8.84577,10.8759) rectangle (8.88557,10.9818);
\draw [color=c, fill=c] (8.88557,10.8759) rectangle (8.92537,10.9818);
\draw [color=c, fill=c] (8.92537,10.8759) rectangle (8.96517,10.9818);
\draw [color=c, fill=c] (8.96517,10.8759) rectangle (9.00498,10.9818);
\draw [color=c, fill=c] (9.00498,10.8759) rectangle (9.04478,10.9818);
\draw [color=c, fill=c] (9.04478,10.8759) rectangle (9.08458,10.9818);
\draw [color=c, fill=c] (9.08458,10.8759) rectangle (9.12438,10.9818);
\draw [color=c, fill=c] (9.12438,10.8759) rectangle (9.16418,10.9818);
\draw [color=c, fill=c] (9.16418,10.8759) rectangle (9.20398,10.9818);
\draw [color=c, fill=c] (9.20398,10.8759) rectangle (9.24378,10.9818);
\draw [color=c, fill=c] (9.24378,10.8759) rectangle (9.28358,10.9818);
\draw [color=c, fill=c] (9.28358,10.8759) rectangle (9.32338,10.9818);
\draw [color=c, fill=c] (9.32338,10.8759) rectangle (9.36318,10.9818);
\draw [color=c, fill=c] (9.36318,10.8759) rectangle (9.40298,10.9818);
\draw [color=c, fill=c] (9.40298,10.8759) rectangle (9.44279,10.9818);
\draw [color=c, fill=c] (9.44279,10.8759) rectangle (9.48259,10.9818);
\draw [color=c, fill=c] (9.48259,10.8759) rectangle (9.52239,10.9818);
\draw [color=c, fill=c] (9.52239,10.8759) rectangle (9.56219,10.9818);
\draw [color=c, fill=c] (9.56219,10.8759) rectangle (9.60199,10.9818);
\definecolor{c}{rgb}{0,0.266667,1};
\draw [color=c, fill=c] (9.60199,10.8759) rectangle (9.64179,10.9818);
\draw [color=c, fill=c] (9.64179,10.8759) rectangle (9.68159,10.9818);
\draw [color=c, fill=c] (9.68159,10.8759) rectangle (9.72139,10.9818);
\draw [color=c, fill=c] (9.72139,10.8759) rectangle (9.76119,10.9818);
\draw [color=c, fill=c] (9.76119,10.8759) rectangle (9.80099,10.9818);
\draw [color=c, fill=c] (9.80099,10.8759) rectangle (9.8408,10.9818);
\draw [color=c, fill=c] (9.8408,10.8759) rectangle (9.8806,10.9818);
\draw [color=c, fill=c] (9.8806,10.8759) rectangle (9.9204,10.9818);
\draw [color=c, fill=c] (9.9204,10.8759) rectangle (9.9602,10.9818);
\draw [color=c, fill=c] (9.9602,10.8759) rectangle (10,10.9818);
\draw [color=c, fill=c] (10,10.8759) rectangle (10.0398,10.9818);
\draw [color=c, fill=c] (10.0398,10.8759) rectangle (10.0796,10.9818);
\draw [color=c, fill=c] (10.0796,10.8759) rectangle (10.1194,10.9818);
\draw [color=c, fill=c] (10.1194,10.8759) rectangle (10.1592,10.9818);
\draw [color=c, fill=c] (10.1592,10.8759) rectangle (10.199,10.9818);
\draw [color=c, fill=c] (10.199,10.8759) rectangle (10.2388,10.9818);
\draw [color=c, fill=c] (10.2388,10.8759) rectangle (10.2786,10.9818);
\draw [color=c, fill=c] (10.2786,10.8759) rectangle (10.3184,10.9818);
\draw [color=c, fill=c] (10.3184,10.8759) rectangle (10.3582,10.9818);
\draw [color=c, fill=c] (10.3582,10.8759) rectangle (10.398,10.9818);
\draw [color=c, fill=c] (10.398,10.8759) rectangle (10.4378,10.9818);
\draw [color=c, fill=c] (10.4378,10.8759) rectangle (10.4776,10.9818);
\draw [color=c, fill=c] (10.4776,10.8759) rectangle (10.5174,10.9818);
\draw [color=c, fill=c] (10.5174,10.8759) rectangle (10.5572,10.9818);
\draw [color=c, fill=c] (10.5572,10.8759) rectangle (10.597,10.9818);
\draw [color=c, fill=c] (10.597,10.8759) rectangle (10.6368,10.9818);
\draw [color=c, fill=c] (10.6368,10.8759) rectangle (10.6766,10.9818);
\draw [color=c, fill=c] (10.6766,10.8759) rectangle (10.7164,10.9818);
\draw [color=c, fill=c] (10.7164,10.8759) rectangle (10.7562,10.9818);
\draw [color=c, fill=c] (10.7562,10.8759) rectangle (10.796,10.9818);
\draw [color=c, fill=c] (10.796,10.8759) rectangle (10.8358,10.9818);
\draw [color=c, fill=c] (10.8358,10.8759) rectangle (10.8756,10.9818);
\draw [color=c, fill=c] (10.8756,10.8759) rectangle (10.9154,10.9818);
\definecolor{c}{rgb}{0,0.546666,1};
\draw [color=c, fill=c] (10.9154,10.8759) rectangle (10.9552,10.9818);
\draw [color=c, fill=c] (10.9552,10.8759) rectangle (10.995,10.9818);
\draw [color=c, fill=c] (10.995,10.8759) rectangle (11.0348,10.9818);
\draw [color=c, fill=c] (11.0348,10.8759) rectangle (11.0746,10.9818);
\draw [color=c, fill=c] (11.0746,10.8759) rectangle (11.1144,10.9818);
\draw [color=c, fill=c] (11.1144,10.8759) rectangle (11.1542,10.9818);
\draw [color=c, fill=c] (11.1542,10.8759) rectangle (11.194,10.9818);
\draw [color=c, fill=c] (11.194,10.8759) rectangle (11.2338,10.9818);
\draw [color=c, fill=c] (11.2338,10.8759) rectangle (11.2736,10.9818);
\draw [color=c, fill=c] (11.2736,10.8759) rectangle (11.3134,10.9818);
\draw [color=c, fill=c] (11.3134,10.8759) rectangle (11.3532,10.9818);
\draw [color=c, fill=c] (11.3532,10.8759) rectangle (11.393,10.9818);
\draw [color=c, fill=c] (11.393,10.8759) rectangle (11.4328,10.9818);
\draw [color=c, fill=c] (11.4328,10.8759) rectangle (11.4726,10.9818);
\draw [color=c, fill=c] (11.4726,10.8759) rectangle (11.5124,10.9818);
\draw [color=c, fill=c] (11.5124,10.8759) rectangle (11.5522,10.9818);
\draw [color=c, fill=c] (11.5522,10.8759) rectangle (11.592,10.9818);
\draw [color=c, fill=c] (11.592,10.8759) rectangle (11.6318,10.9818);
\draw [color=c, fill=c] (11.6318,10.8759) rectangle (11.6716,10.9818);
\draw [color=c, fill=c] (11.6716,10.8759) rectangle (11.7114,10.9818);
\draw [color=c, fill=c] (11.7114,10.8759) rectangle (11.7512,10.9818);
\draw [color=c, fill=c] (11.7512,10.8759) rectangle (11.791,10.9818);
\draw [color=c, fill=c] (11.791,10.8759) rectangle (11.8308,10.9818);
\draw [color=c, fill=c] (11.8308,10.8759) rectangle (11.8706,10.9818);
\draw [color=c, fill=c] (11.8706,10.8759) rectangle (11.9104,10.9818);
\draw [color=c, fill=c] (11.9104,10.8759) rectangle (11.9502,10.9818);
\draw [color=c, fill=c] (11.9502,10.8759) rectangle (11.99,10.9818);
\draw [color=c, fill=c] (11.99,10.8759) rectangle (12.0299,10.9818);
\draw [color=c, fill=c] (12.0299,10.8759) rectangle (12.0697,10.9818);
\draw [color=c, fill=c] (12.0697,10.8759) rectangle (12.1095,10.9818);
\draw [color=c, fill=c] (12.1095,10.8759) rectangle (12.1493,10.9818);
\draw [color=c, fill=c] (12.1493,10.8759) rectangle (12.1891,10.9818);
\draw [color=c, fill=c] (12.1891,10.8759) rectangle (12.2289,10.9818);
\draw [color=c, fill=c] (12.2289,10.8759) rectangle (12.2687,10.9818);
\draw [color=c, fill=c] (12.2687,10.8759) rectangle (12.3085,10.9818);
\draw [color=c, fill=c] (12.3085,10.8759) rectangle (12.3483,10.9818);
\draw [color=c, fill=c] (12.3483,10.8759) rectangle (12.3881,10.9818);
\draw [color=c, fill=c] (12.3881,10.8759) rectangle (12.4279,10.9818);
\draw [color=c, fill=c] (12.4279,10.8759) rectangle (12.4677,10.9818);
\draw [color=c, fill=c] (12.4677,10.8759) rectangle (12.5075,10.9818);
\draw [color=c, fill=c] (12.5075,10.8759) rectangle (12.5473,10.9818);
\draw [color=c, fill=c] (12.5473,10.8759) rectangle (12.5871,10.9818);
\draw [color=c, fill=c] (12.5871,10.8759) rectangle (12.6269,10.9818);
\draw [color=c, fill=c] (12.6269,10.8759) rectangle (12.6667,10.9818);
\draw [color=c, fill=c] (12.6667,10.8759) rectangle (12.7065,10.9818);
\draw [color=c, fill=c] (12.7065,10.8759) rectangle (12.7463,10.9818);
\draw [color=c, fill=c] (12.7463,10.8759) rectangle (12.7861,10.9818);
\draw [color=c, fill=c] (12.7861,10.8759) rectangle (12.8259,10.9818);
\draw [color=c, fill=c] (12.8259,10.8759) rectangle (12.8657,10.9818);
\draw [color=c, fill=c] (12.8657,10.8759) rectangle (12.9055,10.9818);
\draw [color=c, fill=c] (12.9055,10.8759) rectangle (12.9453,10.9818);
\draw [color=c, fill=c] (12.9453,10.8759) rectangle (12.9851,10.9818);
\draw [color=c, fill=c] (12.9851,10.8759) rectangle (13.0249,10.9818);
\draw [color=c, fill=c] (13.0249,10.8759) rectangle (13.0647,10.9818);
\draw [color=c, fill=c] (13.0647,10.8759) rectangle (13.1045,10.9818);
\draw [color=c, fill=c] (13.1045,10.8759) rectangle (13.1443,10.9818);
\draw [color=c, fill=c] (13.1443,10.8759) rectangle (13.1841,10.9818);
\draw [color=c, fill=c] (13.1841,10.8759) rectangle (13.2239,10.9818);
\draw [color=c, fill=c] (13.2239,10.8759) rectangle (13.2637,10.9818);
\draw [color=c, fill=c] (13.2637,10.8759) rectangle (13.3035,10.9818);
\draw [color=c, fill=c] (13.3035,10.8759) rectangle (13.3433,10.9818);
\draw [color=c, fill=c] (13.3433,10.8759) rectangle (13.3831,10.9818);
\draw [color=c, fill=c] (13.3831,10.8759) rectangle (13.4229,10.9818);
\draw [color=c, fill=c] (13.4229,10.8759) rectangle (13.4627,10.9818);
\draw [color=c, fill=c] (13.4627,10.8759) rectangle (13.5025,10.9818);
\draw [color=c, fill=c] (13.5025,10.8759) rectangle (13.5423,10.9818);
\draw [color=c, fill=c] (13.5423,10.8759) rectangle (13.5821,10.9818);
\draw [color=c, fill=c] (13.5821,10.8759) rectangle (13.6219,10.9818);
\draw [color=c, fill=c] (13.6219,10.8759) rectangle (13.6617,10.9818);
\draw [color=c, fill=c] (13.6617,10.8759) rectangle (13.7015,10.9818);
\draw [color=c, fill=c] (13.7015,10.8759) rectangle (13.7413,10.9818);
\draw [color=c, fill=c] (13.7413,10.8759) rectangle (13.7811,10.9818);
\draw [color=c, fill=c] (13.7811,10.8759) rectangle (13.8209,10.9818);
\draw [color=c, fill=c] (13.8209,10.8759) rectangle (13.8607,10.9818);
\draw [color=c, fill=c] (13.8607,10.8759) rectangle (13.9005,10.9818);
\draw [color=c, fill=c] (13.9005,10.8759) rectangle (13.9403,10.9818);
\draw [color=c, fill=c] (13.9403,10.8759) rectangle (13.9801,10.9818);
\draw [color=c, fill=c] (13.9801,10.8759) rectangle (14.0199,10.9818);
\draw [color=c, fill=c] (14.0199,10.8759) rectangle (14.0597,10.9818);
\draw [color=c, fill=c] (14.0597,10.8759) rectangle (14.0995,10.9818);
\draw [color=c, fill=c] (14.0995,10.8759) rectangle (14.1393,10.9818);
\draw [color=c, fill=c] (14.1393,10.8759) rectangle (14.1791,10.9818);
\draw [color=c, fill=c] (14.1791,10.8759) rectangle (14.2189,10.9818);
\draw [color=c, fill=c] (14.2189,10.8759) rectangle (14.2587,10.9818);
\definecolor{c}{rgb}{0,0.733333,1};
\draw [color=c, fill=c] (14.2587,10.8759) rectangle (14.2985,10.9818);
\draw [color=c, fill=c] (14.2985,10.8759) rectangle (14.3383,10.9818);
\draw [color=c, fill=c] (14.3383,10.8759) rectangle (14.3781,10.9818);
\draw [color=c, fill=c] (14.3781,10.8759) rectangle (14.4179,10.9818);
\draw [color=c, fill=c] (14.4179,10.8759) rectangle (14.4577,10.9818);
\draw [color=c, fill=c] (14.4577,10.8759) rectangle (14.4975,10.9818);
\draw [color=c, fill=c] (14.4975,10.8759) rectangle (14.5373,10.9818);
\draw [color=c, fill=c] (14.5373,10.8759) rectangle (14.5771,10.9818);
\draw [color=c, fill=c] (14.5771,10.8759) rectangle (14.6169,10.9818);
\draw [color=c, fill=c] (14.6169,10.8759) rectangle (14.6567,10.9818);
\draw [color=c, fill=c] (14.6567,10.8759) rectangle (14.6965,10.9818);
\draw [color=c, fill=c] (14.6965,10.8759) rectangle (14.7363,10.9818);
\draw [color=c, fill=c] (14.7363,10.8759) rectangle (14.7761,10.9818);
\draw [color=c, fill=c] (14.7761,10.8759) rectangle (14.8159,10.9818);
\draw [color=c, fill=c] (14.8159,10.8759) rectangle (14.8557,10.9818);
\draw [color=c, fill=c] (14.8557,10.8759) rectangle (14.8955,10.9818);
\draw [color=c, fill=c] (14.8955,10.8759) rectangle (14.9353,10.9818);
\draw [color=c, fill=c] (14.9353,10.8759) rectangle (14.9751,10.9818);
\draw [color=c, fill=c] (14.9751,10.8759) rectangle (15.0149,10.9818);
\draw [color=c, fill=c] (15.0149,10.8759) rectangle (15.0547,10.9818);
\draw [color=c, fill=c] (15.0547,10.8759) rectangle (15.0945,10.9818);
\draw [color=c, fill=c] (15.0945,10.8759) rectangle (15.1343,10.9818);
\draw [color=c, fill=c] (15.1343,10.8759) rectangle (15.1741,10.9818);
\draw [color=c, fill=c] (15.1741,10.8759) rectangle (15.2139,10.9818);
\draw [color=c, fill=c] (15.2139,10.8759) rectangle (15.2537,10.9818);
\draw [color=c, fill=c] (15.2537,10.8759) rectangle (15.2935,10.9818);
\draw [color=c, fill=c] (15.2935,10.8759) rectangle (15.3333,10.9818);
\draw [color=c, fill=c] (15.3333,10.8759) rectangle (15.3731,10.9818);
\draw [color=c, fill=c] (15.3731,10.8759) rectangle (15.4129,10.9818);
\draw [color=c, fill=c] (15.4129,10.8759) rectangle (15.4527,10.9818);
\draw [color=c, fill=c] (15.4527,10.8759) rectangle (15.4925,10.9818);
\draw [color=c, fill=c] (15.4925,10.8759) rectangle (15.5323,10.9818);
\draw [color=c, fill=c] (15.5323,10.8759) rectangle (15.5721,10.9818);
\draw [color=c, fill=c] (15.5721,10.8759) rectangle (15.6119,10.9818);
\draw [color=c, fill=c] (15.6119,10.8759) rectangle (15.6517,10.9818);
\draw [color=c, fill=c] (15.6517,10.8759) rectangle (15.6915,10.9818);
\draw [color=c, fill=c] (15.6915,10.8759) rectangle (15.7313,10.9818);
\draw [color=c, fill=c] (15.7313,10.8759) rectangle (15.7711,10.9818);
\draw [color=c, fill=c] (15.7711,10.8759) rectangle (15.8109,10.9818);
\draw [color=c, fill=c] (15.8109,10.8759) rectangle (15.8507,10.9818);
\draw [color=c, fill=c] (15.8507,10.8759) rectangle (15.8905,10.9818);
\draw [color=c, fill=c] (15.8905,10.8759) rectangle (15.9303,10.9818);
\draw [color=c, fill=c] (15.9303,10.8759) rectangle (15.9701,10.9818);
\draw [color=c, fill=c] (15.9701,10.8759) rectangle (16.01,10.9818);
\draw [color=c, fill=c] (16.01,10.8759) rectangle (16.0498,10.9818);
\draw [color=c, fill=c] (16.0498,10.8759) rectangle (16.0896,10.9818);
\draw [color=c, fill=c] (16.0896,10.8759) rectangle (16.1294,10.9818);
\draw [color=c, fill=c] (16.1294,10.8759) rectangle (16.1692,10.9818);
\draw [color=c, fill=c] (16.1692,10.8759) rectangle (16.209,10.9818);
\draw [color=c, fill=c] (16.209,10.8759) rectangle (16.2488,10.9818);
\draw [color=c, fill=c] (16.2488,10.8759) rectangle (16.2886,10.9818);
\draw [color=c, fill=c] (16.2886,10.8759) rectangle (16.3284,10.9818);
\draw [color=c, fill=c] (16.3284,10.8759) rectangle (16.3682,10.9818);
\draw [color=c, fill=c] (16.3682,10.8759) rectangle (16.408,10.9818);
\draw [color=c, fill=c] (16.408,10.8759) rectangle (16.4478,10.9818);
\draw [color=c, fill=c] (16.4478,10.8759) rectangle (16.4876,10.9818);
\draw [color=c, fill=c] (16.4876,10.8759) rectangle (16.5274,10.9818);
\draw [color=c, fill=c] (16.5274,10.8759) rectangle (16.5672,10.9818);
\draw [color=c, fill=c] (16.5672,10.8759) rectangle (16.607,10.9818);
\draw [color=c, fill=c] (16.607,10.8759) rectangle (16.6468,10.9818);
\draw [color=c, fill=c] (16.6468,10.8759) rectangle (16.6866,10.9818);
\draw [color=c, fill=c] (16.6866,10.8759) rectangle (16.7264,10.9818);
\draw [color=c, fill=c] (16.7264,10.8759) rectangle (16.7662,10.9818);
\draw [color=c, fill=c] (16.7662,10.8759) rectangle (16.806,10.9818);
\draw [color=c, fill=c] (16.806,10.8759) rectangle (16.8458,10.9818);
\draw [color=c, fill=c] (16.8458,10.8759) rectangle (16.8856,10.9818);
\draw [color=c, fill=c] (16.8856,10.8759) rectangle (16.9254,10.9818);
\draw [color=c, fill=c] (16.9254,10.8759) rectangle (16.9652,10.9818);
\draw [color=c, fill=c] (16.9652,10.8759) rectangle (17.005,10.9818);
\draw [color=c, fill=c] (17.005,10.8759) rectangle (17.0448,10.9818);
\draw [color=c, fill=c] (17.0448,10.8759) rectangle (17.0846,10.9818);
\draw [color=c, fill=c] (17.0846,10.8759) rectangle (17.1244,10.9818);
\draw [color=c, fill=c] (17.1244,10.8759) rectangle (17.1642,10.9818);
\draw [color=c, fill=c] (17.1642,10.8759) rectangle (17.204,10.9818);
\draw [color=c, fill=c] (17.204,10.8759) rectangle (17.2438,10.9818);
\draw [color=c, fill=c] (17.2438,10.8759) rectangle (17.2836,10.9818);
\draw [color=c, fill=c] (17.2836,10.8759) rectangle (17.3234,10.9818);
\draw [color=c, fill=c] (17.3234,10.8759) rectangle (17.3632,10.9818);
\draw [color=c, fill=c] (17.3632,10.8759) rectangle (17.403,10.9818);
\draw [color=c, fill=c] (17.403,10.8759) rectangle (17.4428,10.9818);
\draw [color=c, fill=c] (17.4428,10.8759) rectangle (17.4826,10.9818);
\draw [color=c, fill=c] (17.4826,10.8759) rectangle (17.5224,10.9818);
\draw [color=c, fill=c] (17.5224,10.8759) rectangle (17.5622,10.9818);
\draw [color=c, fill=c] (17.5622,10.8759) rectangle (17.602,10.9818);
\draw [color=c, fill=c] (17.602,10.8759) rectangle (17.6418,10.9818);
\draw [color=c, fill=c] (17.6418,10.8759) rectangle (17.6816,10.9818);
\draw [color=c, fill=c] (17.6816,10.8759) rectangle (17.7214,10.9818);
\draw [color=c, fill=c] (17.7214,10.8759) rectangle (17.7612,10.9818);
\draw [color=c, fill=c] (17.7612,10.8759) rectangle (17.801,10.9818);
\draw [color=c, fill=c] (17.801,10.8759) rectangle (17.8408,10.9818);
\draw [color=c, fill=c] (17.8408,10.8759) rectangle (17.8806,10.9818);
\draw [color=c, fill=c] (17.8806,10.8759) rectangle (17.9204,10.9818);
\draw [color=c, fill=c] (17.9204,10.8759) rectangle (17.9602,10.9818);
\draw [color=c, fill=c] (17.9602,10.8759) rectangle (18,10.9818);
\definecolor{c}{rgb}{0.2,0,1};
\draw [color=c, fill=c] (2,10.9818) rectangle (2.0398,11.0876);
\draw [color=c, fill=c] (2.0398,10.9818) rectangle (2.0796,11.0876);
\draw [color=c, fill=c] (2.0796,10.9818) rectangle (2.1194,11.0876);
\draw [color=c, fill=c] (2.1194,10.9818) rectangle (2.1592,11.0876);
\draw [color=c, fill=c] (2.1592,10.9818) rectangle (2.19901,11.0876);
\draw [color=c, fill=c] (2.19901,10.9818) rectangle (2.23881,11.0876);
\draw [color=c, fill=c] (2.23881,10.9818) rectangle (2.27861,11.0876);
\draw [color=c, fill=c] (2.27861,10.9818) rectangle (2.31841,11.0876);
\draw [color=c, fill=c] (2.31841,10.9818) rectangle (2.35821,11.0876);
\draw [color=c, fill=c] (2.35821,10.9818) rectangle (2.39801,11.0876);
\draw [color=c, fill=c] (2.39801,10.9818) rectangle (2.43781,11.0876);
\draw [color=c, fill=c] (2.43781,10.9818) rectangle (2.47761,11.0876);
\draw [color=c, fill=c] (2.47761,10.9818) rectangle (2.51741,11.0876);
\draw [color=c, fill=c] (2.51741,10.9818) rectangle (2.55721,11.0876);
\draw [color=c, fill=c] (2.55721,10.9818) rectangle (2.59702,11.0876);
\draw [color=c, fill=c] (2.59702,10.9818) rectangle (2.63682,11.0876);
\draw [color=c, fill=c] (2.63682,10.9818) rectangle (2.67662,11.0876);
\draw [color=c, fill=c] (2.67662,10.9818) rectangle (2.71642,11.0876);
\draw [color=c, fill=c] (2.71642,10.9818) rectangle (2.75622,11.0876);
\draw [color=c, fill=c] (2.75622,10.9818) rectangle (2.79602,11.0876);
\draw [color=c, fill=c] (2.79602,10.9818) rectangle (2.83582,11.0876);
\draw [color=c, fill=c] (2.83582,10.9818) rectangle (2.87562,11.0876);
\draw [color=c, fill=c] (2.87562,10.9818) rectangle (2.91542,11.0876);
\draw [color=c, fill=c] (2.91542,10.9818) rectangle (2.95522,11.0876);
\draw [color=c, fill=c] (2.95522,10.9818) rectangle (2.99502,11.0876);
\draw [color=c, fill=c] (2.99502,10.9818) rectangle (3.03483,11.0876);
\draw [color=c, fill=c] (3.03483,10.9818) rectangle (3.07463,11.0876);
\draw [color=c, fill=c] (3.07463,10.9818) rectangle (3.11443,11.0876);
\draw [color=c, fill=c] (3.11443,10.9818) rectangle (3.15423,11.0876);
\draw [color=c, fill=c] (3.15423,10.9818) rectangle (3.19403,11.0876);
\draw [color=c, fill=c] (3.19403,10.9818) rectangle (3.23383,11.0876);
\draw [color=c, fill=c] (3.23383,10.9818) rectangle (3.27363,11.0876);
\draw [color=c, fill=c] (3.27363,10.9818) rectangle (3.31343,11.0876);
\draw [color=c, fill=c] (3.31343,10.9818) rectangle (3.35323,11.0876);
\draw [color=c, fill=c] (3.35323,10.9818) rectangle (3.39303,11.0876);
\draw [color=c, fill=c] (3.39303,10.9818) rectangle (3.43284,11.0876);
\draw [color=c, fill=c] (3.43284,10.9818) rectangle (3.47264,11.0876);
\draw [color=c, fill=c] (3.47264,10.9818) rectangle (3.51244,11.0876);
\draw [color=c, fill=c] (3.51244,10.9818) rectangle (3.55224,11.0876);
\draw [color=c, fill=c] (3.55224,10.9818) rectangle (3.59204,11.0876);
\draw [color=c, fill=c] (3.59204,10.9818) rectangle (3.63184,11.0876);
\draw [color=c, fill=c] (3.63184,10.9818) rectangle (3.67164,11.0876);
\draw [color=c, fill=c] (3.67164,10.9818) rectangle (3.71144,11.0876);
\draw [color=c, fill=c] (3.71144,10.9818) rectangle (3.75124,11.0876);
\draw [color=c, fill=c] (3.75124,10.9818) rectangle (3.79104,11.0876);
\draw [color=c, fill=c] (3.79104,10.9818) rectangle (3.83085,11.0876);
\draw [color=c, fill=c] (3.83085,10.9818) rectangle (3.87065,11.0876);
\draw [color=c, fill=c] (3.87065,10.9818) rectangle (3.91045,11.0876);
\draw [color=c, fill=c] (3.91045,10.9818) rectangle (3.95025,11.0876);
\draw [color=c, fill=c] (3.95025,10.9818) rectangle (3.99005,11.0876);
\draw [color=c, fill=c] (3.99005,10.9818) rectangle (4.02985,11.0876);
\draw [color=c, fill=c] (4.02985,10.9818) rectangle (4.06965,11.0876);
\draw [color=c, fill=c] (4.06965,10.9818) rectangle (4.10945,11.0876);
\draw [color=c, fill=c] (4.10945,10.9818) rectangle (4.14925,11.0876);
\draw [color=c, fill=c] (4.14925,10.9818) rectangle (4.18905,11.0876);
\draw [color=c, fill=c] (4.18905,10.9818) rectangle (4.22886,11.0876);
\draw [color=c, fill=c] (4.22886,10.9818) rectangle (4.26866,11.0876);
\draw [color=c, fill=c] (4.26866,10.9818) rectangle (4.30846,11.0876);
\draw [color=c, fill=c] (4.30846,10.9818) rectangle (4.34826,11.0876);
\draw [color=c, fill=c] (4.34826,10.9818) rectangle (4.38806,11.0876);
\draw [color=c, fill=c] (4.38806,10.9818) rectangle (4.42786,11.0876);
\draw [color=c, fill=c] (4.42786,10.9818) rectangle (4.46766,11.0876);
\draw [color=c, fill=c] (4.46766,10.9818) rectangle (4.50746,11.0876);
\draw [color=c, fill=c] (4.50746,10.9818) rectangle (4.54726,11.0876);
\draw [color=c, fill=c] (4.54726,10.9818) rectangle (4.58706,11.0876);
\draw [color=c, fill=c] (4.58706,10.9818) rectangle (4.62687,11.0876);
\draw [color=c, fill=c] (4.62687,10.9818) rectangle (4.66667,11.0876);
\draw [color=c, fill=c] (4.66667,10.9818) rectangle (4.70647,11.0876);
\draw [color=c, fill=c] (4.70647,10.9818) rectangle (4.74627,11.0876);
\draw [color=c, fill=c] (4.74627,10.9818) rectangle (4.78607,11.0876);
\draw [color=c, fill=c] (4.78607,10.9818) rectangle (4.82587,11.0876);
\draw [color=c, fill=c] (4.82587,10.9818) rectangle (4.86567,11.0876);
\draw [color=c, fill=c] (4.86567,10.9818) rectangle (4.90547,11.0876);
\draw [color=c, fill=c] (4.90547,10.9818) rectangle (4.94527,11.0876);
\draw [color=c, fill=c] (4.94527,10.9818) rectangle (4.98507,11.0876);
\draw [color=c, fill=c] (4.98507,10.9818) rectangle (5.02488,11.0876);
\draw [color=c, fill=c] (5.02488,10.9818) rectangle (5.06468,11.0876);
\draw [color=c, fill=c] (5.06468,10.9818) rectangle (5.10448,11.0876);
\draw [color=c, fill=c] (5.10448,10.9818) rectangle (5.14428,11.0876);
\draw [color=c, fill=c] (5.14428,10.9818) rectangle (5.18408,11.0876);
\draw [color=c, fill=c] (5.18408,10.9818) rectangle (5.22388,11.0876);
\draw [color=c, fill=c] (5.22388,10.9818) rectangle (5.26368,11.0876);
\draw [color=c, fill=c] (5.26368,10.9818) rectangle (5.30348,11.0876);
\draw [color=c, fill=c] (5.30348,10.9818) rectangle (5.34328,11.0876);
\draw [color=c, fill=c] (5.34328,10.9818) rectangle (5.38308,11.0876);
\draw [color=c, fill=c] (5.38308,10.9818) rectangle (5.42289,11.0876);
\draw [color=c, fill=c] (5.42289,10.9818) rectangle (5.46269,11.0876);
\draw [color=c, fill=c] (5.46269,10.9818) rectangle (5.50249,11.0876);
\draw [color=c, fill=c] (5.50249,10.9818) rectangle (5.54229,11.0876);
\draw [color=c, fill=c] (5.54229,10.9818) rectangle (5.58209,11.0876);
\draw [color=c, fill=c] (5.58209,10.9818) rectangle (5.62189,11.0876);
\draw [color=c, fill=c] (5.62189,10.9818) rectangle (5.66169,11.0876);
\draw [color=c, fill=c] (5.66169,10.9818) rectangle (5.70149,11.0876);
\draw [color=c, fill=c] (5.70149,10.9818) rectangle (5.74129,11.0876);
\draw [color=c, fill=c] (5.74129,10.9818) rectangle (5.78109,11.0876);
\draw [color=c, fill=c] (5.78109,10.9818) rectangle (5.8209,11.0876);
\draw [color=c, fill=c] (5.8209,10.9818) rectangle (5.8607,11.0876);
\draw [color=c, fill=c] (5.8607,10.9818) rectangle (5.9005,11.0876);
\draw [color=c, fill=c] (5.9005,10.9818) rectangle (5.9403,11.0876);
\draw [color=c, fill=c] (5.9403,10.9818) rectangle (5.9801,11.0876);
\draw [color=c, fill=c] (5.9801,10.9818) rectangle (6.0199,11.0876);
\draw [color=c, fill=c] (6.0199,10.9818) rectangle (6.0597,11.0876);
\draw [color=c, fill=c] (6.0597,10.9818) rectangle (6.0995,11.0876);
\draw [color=c, fill=c] (6.0995,10.9818) rectangle (6.1393,11.0876);
\draw [color=c, fill=c] (6.1393,10.9818) rectangle (6.1791,11.0876);
\draw [color=c, fill=c] (6.1791,10.9818) rectangle (6.21891,11.0876);
\draw [color=c, fill=c] (6.21891,10.9818) rectangle (6.25871,11.0876);
\draw [color=c, fill=c] (6.25871,10.9818) rectangle (6.29851,11.0876);
\draw [color=c, fill=c] (6.29851,10.9818) rectangle (6.33831,11.0876);
\draw [color=c, fill=c] (6.33831,10.9818) rectangle (6.37811,11.0876);
\draw [color=c, fill=c] (6.37811,10.9818) rectangle (6.41791,11.0876);
\draw [color=c, fill=c] (6.41791,10.9818) rectangle (6.45771,11.0876);
\draw [color=c, fill=c] (6.45771,10.9818) rectangle (6.49751,11.0876);
\draw [color=c, fill=c] (6.49751,10.9818) rectangle (6.53731,11.0876);
\draw [color=c, fill=c] (6.53731,10.9818) rectangle (6.57711,11.0876);
\draw [color=c, fill=c] (6.57711,10.9818) rectangle (6.61692,11.0876);
\draw [color=c, fill=c] (6.61692,10.9818) rectangle (6.65672,11.0876);
\draw [color=c, fill=c] (6.65672,10.9818) rectangle (6.69652,11.0876);
\draw [color=c, fill=c] (6.69652,10.9818) rectangle (6.73632,11.0876);
\draw [color=c, fill=c] (6.73632,10.9818) rectangle (6.77612,11.0876);
\draw [color=c, fill=c] (6.77612,10.9818) rectangle (6.81592,11.0876);
\draw [color=c, fill=c] (6.81592,10.9818) rectangle (6.85572,11.0876);
\draw [color=c, fill=c] (6.85572,10.9818) rectangle (6.89552,11.0876);
\draw [color=c, fill=c] (6.89552,10.9818) rectangle (6.93532,11.0876);
\draw [color=c, fill=c] (6.93532,10.9818) rectangle (6.97512,11.0876);
\draw [color=c, fill=c] (6.97512,10.9818) rectangle (7.01493,11.0876);
\draw [color=c, fill=c] (7.01493,10.9818) rectangle (7.05473,11.0876);
\draw [color=c, fill=c] (7.05473,10.9818) rectangle (7.09453,11.0876);
\draw [color=c, fill=c] (7.09453,10.9818) rectangle (7.13433,11.0876);
\draw [color=c, fill=c] (7.13433,10.9818) rectangle (7.17413,11.0876);
\draw [color=c, fill=c] (7.17413,10.9818) rectangle (7.21393,11.0876);
\draw [color=c, fill=c] (7.21393,10.9818) rectangle (7.25373,11.0876);
\draw [color=c, fill=c] (7.25373,10.9818) rectangle (7.29353,11.0876);
\draw [color=c, fill=c] (7.29353,10.9818) rectangle (7.33333,11.0876);
\draw [color=c, fill=c] (7.33333,10.9818) rectangle (7.37313,11.0876);
\draw [color=c, fill=c] (7.37313,10.9818) rectangle (7.41294,11.0876);
\draw [color=c, fill=c] (7.41294,10.9818) rectangle (7.45274,11.0876);
\draw [color=c, fill=c] (7.45274,10.9818) rectangle (7.49254,11.0876);
\draw [color=c, fill=c] (7.49254,10.9818) rectangle (7.53234,11.0876);
\draw [color=c, fill=c] (7.53234,10.9818) rectangle (7.57214,11.0876);
\draw [color=c, fill=c] (7.57214,10.9818) rectangle (7.61194,11.0876);
\draw [color=c, fill=c] (7.61194,10.9818) rectangle (7.65174,11.0876);
\draw [color=c, fill=c] (7.65174,10.9818) rectangle (7.69154,11.0876);
\draw [color=c, fill=c] (7.69154,10.9818) rectangle (7.73134,11.0876);
\draw [color=c, fill=c] (7.73134,10.9818) rectangle (7.77114,11.0876);
\draw [color=c, fill=c] (7.77114,10.9818) rectangle (7.81095,11.0876);
\draw [color=c, fill=c] (7.81095,10.9818) rectangle (7.85075,11.0876);
\draw [color=c, fill=c] (7.85075,10.9818) rectangle (7.89055,11.0876);
\draw [color=c, fill=c] (7.89055,10.9818) rectangle (7.93035,11.0876);
\definecolor{c}{rgb}{0,0.0800001,1};
\draw [color=c, fill=c] (7.93035,10.9818) rectangle (7.97015,11.0876);
\draw [color=c, fill=c] (7.97015,10.9818) rectangle (8.00995,11.0876);
\draw [color=c, fill=c] (8.00995,10.9818) rectangle (8.04975,11.0876);
\draw [color=c, fill=c] (8.04975,10.9818) rectangle (8.08955,11.0876);
\draw [color=c, fill=c] (8.08955,10.9818) rectangle (8.12935,11.0876);
\draw [color=c, fill=c] (8.12935,10.9818) rectangle (8.16915,11.0876);
\draw [color=c, fill=c] (8.16915,10.9818) rectangle (8.20895,11.0876);
\draw [color=c, fill=c] (8.20895,10.9818) rectangle (8.24876,11.0876);
\draw [color=c, fill=c] (8.24876,10.9818) rectangle (8.28856,11.0876);
\draw [color=c, fill=c] (8.28856,10.9818) rectangle (8.32836,11.0876);
\draw [color=c, fill=c] (8.32836,10.9818) rectangle (8.36816,11.0876);
\draw [color=c, fill=c] (8.36816,10.9818) rectangle (8.40796,11.0876);
\draw [color=c, fill=c] (8.40796,10.9818) rectangle (8.44776,11.0876);
\draw [color=c, fill=c] (8.44776,10.9818) rectangle (8.48756,11.0876);
\draw [color=c, fill=c] (8.48756,10.9818) rectangle (8.52736,11.0876);
\draw [color=c, fill=c] (8.52736,10.9818) rectangle (8.56716,11.0876);
\draw [color=c, fill=c] (8.56716,10.9818) rectangle (8.60697,11.0876);
\draw [color=c, fill=c] (8.60697,10.9818) rectangle (8.64677,11.0876);
\draw [color=c, fill=c] (8.64677,10.9818) rectangle (8.68657,11.0876);
\draw [color=c, fill=c] (8.68657,10.9818) rectangle (8.72637,11.0876);
\draw [color=c, fill=c] (8.72637,10.9818) rectangle (8.76617,11.0876);
\draw [color=c, fill=c] (8.76617,10.9818) rectangle (8.80597,11.0876);
\draw [color=c, fill=c] (8.80597,10.9818) rectangle (8.84577,11.0876);
\draw [color=c, fill=c] (8.84577,10.9818) rectangle (8.88557,11.0876);
\draw [color=c, fill=c] (8.88557,10.9818) rectangle (8.92537,11.0876);
\draw [color=c, fill=c] (8.92537,10.9818) rectangle (8.96517,11.0876);
\draw [color=c, fill=c] (8.96517,10.9818) rectangle (9.00498,11.0876);
\draw [color=c, fill=c] (9.00498,10.9818) rectangle (9.04478,11.0876);
\draw [color=c, fill=c] (9.04478,10.9818) rectangle (9.08458,11.0876);
\draw [color=c, fill=c] (9.08458,10.9818) rectangle (9.12438,11.0876);
\draw [color=c, fill=c] (9.12438,10.9818) rectangle (9.16418,11.0876);
\draw [color=c, fill=c] (9.16418,10.9818) rectangle (9.20398,11.0876);
\draw [color=c, fill=c] (9.20398,10.9818) rectangle (9.24378,11.0876);
\draw [color=c, fill=c] (9.24378,10.9818) rectangle (9.28358,11.0876);
\draw [color=c, fill=c] (9.28358,10.9818) rectangle (9.32338,11.0876);
\draw [color=c, fill=c] (9.32338,10.9818) rectangle (9.36318,11.0876);
\draw [color=c, fill=c] (9.36318,10.9818) rectangle (9.40298,11.0876);
\draw [color=c, fill=c] (9.40298,10.9818) rectangle (9.44279,11.0876);
\draw [color=c, fill=c] (9.44279,10.9818) rectangle (9.48259,11.0876);
\draw [color=c, fill=c] (9.48259,10.9818) rectangle (9.52239,11.0876);
\draw [color=c, fill=c] (9.52239,10.9818) rectangle (9.56219,11.0876);
\draw [color=c, fill=c] (9.56219,10.9818) rectangle (9.60199,11.0876);
\definecolor{c}{rgb}{0,0.266667,1};
\draw [color=c, fill=c] (9.60199,10.9818) rectangle (9.64179,11.0876);
\draw [color=c, fill=c] (9.64179,10.9818) rectangle (9.68159,11.0876);
\draw [color=c, fill=c] (9.68159,10.9818) rectangle (9.72139,11.0876);
\draw [color=c, fill=c] (9.72139,10.9818) rectangle (9.76119,11.0876);
\draw [color=c, fill=c] (9.76119,10.9818) rectangle (9.80099,11.0876);
\draw [color=c, fill=c] (9.80099,10.9818) rectangle (9.8408,11.0876);
\draw [color=c, fill=c] (9.8408,10.9818) rectangle (9.8806,11.0876);
\draw [color=c, fill=c] (9.8806,10.9818) rectangle (9.9204,11.0876);
\draw [color=c, fill=c] (9.9204,10.9818) rectangle (9.9602,11.0876);
\draw [color=c, fill=c] (9.9602,10.9818) rectangle (10,11.0876);
\draw [color=c, fill=c] (10,10.9818) rectangle (10.0398,11.0876);
\draw [color=c, fill=c] (10.0398,10.9818) rectangle (10.0796,11.0876);
\draw [color=c, fill=c] (10.0796,10.9818) rectangle (10.1194,11.0876);
\draw [color=c, fill=c] (10.1194,10.9818) rectangle (10.1592,11.0876);
\draw [color=c, fill=c] (10.1592,10.9818) rectangle (10.199,11.0876);
\draw [color=c, fill=c] (10.199,10.9818) rectangle (10.2388,11.0876);
\draw [color=c, fill=c] (10.2388,10.9818) rectangle (10.2786,11.0876);
\draw [color=c, fill=c] (10.2786,10.9818) rectangle (10.3184,11.0876);
\draw [color=c, fill=c] (10.3184,10.9818) rectangle (10.3582,11.0876);
\draw [color=c, fill=c] (10.3582,10.9818) rectangle (10.398,11.0876);
\draw [color=c, fill=c] (10.398,10.9818) rectangle (10.4378,11.0876);
\draw [color=c, fill=c] (10.4378,10.9818) rectangle (10.4776,11.0876);
\draw [color=c, fill=c] (10.4776,10.9818) rectangle (10.5174,11.0876);
\draw [color=c, fill=c] (10.5174,10.9818) rectangle (10.5572,11.0876);
\draw [color=c, fill=c] (10.5572,10.9818) rectangle (10.597,11.0876);
\draw [color=c, fill=c] (10.597,10.9818) rectangle (10.6368,11.0876);
\draw [color=c, fill=c] (10.6368,10.9818) rectangle (10.6766,11.0876);
\draw [color=c, fill=c] (10.6766,10.9818) rectangle (10.7164,11.0876);
\draw [color=c, fill=c] (10.7164,10.9818) rectangle (10.7562,11.0876);
\draw [color=c, fill=c] (10.7562,10.9818) rectangle (10.796,11.0876);
\draw [color=c, fill=c] (10.796,10.9818) rectangle (10.8358,11.0876);
\draw [color=c, fill=c] (10.8358,10.9818) rectangle (10.8756,11.0876);
\draw [color=c, fill=c] (10.8756,10.9818) rectangle (10.9154,11.0876);
\draw [color=c, fill=c] (10.9154,10.9818) rectangle (10.9552,11.0876);
\definecolor{c}{rgb}{0,0.546666,1};
\draw [color=c, fill=c] (10.9552,10.9818) rectangle (10.995,11.0876);
\draw [color=c, fill=c] (10.995,10.9818) rectangle (11.0348,11.0876);
\draw [color=c, fill=c] (11.0348,10.9818) rectangle (11.0746,11.0876);
\draw [color=c, fill=c] (11.0746,10.9818) rectangle (11.1144,11.0876);
\draw [color=c, fill=c] (11.1144,10.9818) rectangle (11.1542,11.0876);
\draw [color=c, fill=c] (11.1542,10.9818) rectangle (11.194,11.0876);
\draw [color=c, fill=c] (11.194,10.9818) rectangle (11.2338,11.0876);
\draw [color=c, fill=c] (11.2338,10.9818) rectangle (11.2736,11.0876);
\draw [color=c, fill=c] (11.2736,10.9818) rectangle (11.3134,11.0876);
\draw [color=c, fill=c] (11.3134,10.9818) rectangle (11.3532,11.0876);
\draw [color=c, fill=c] (11.3532,10.9818) rectangle (11.393,11.0876);
\draw [color=c, fill=c] (11.393,10.9818) rectangle (11.4328,11.0876);
\draw [color=c, fill=c] (11.4328,10.9818) rectangle (11.4726,11.0876);
\draw [color=c, fill=c] (11.4726,10.9818) rectangle (11.5124,11.0876);
\draw [color=c, fill=c] (11.5124,10.9818) rectangle (11.5522,11.0876);
\draw [color=c, fill=c] (11.5522,10.9818) rectangle (11.592,11.0876);
\draw [color=c, fill=c] (11.592,10.9818) rectangle (11.6318,11.0876);
\draw [color=c, fill=c] (11.6318,10.9818) rectangle (11.6716,11.0876);
\draw [color=c, fill=c] (11.6716,10.9818) rectangle (11.7114,11.0876);
\draw [color=c, fill=c] (11.7114,10.9818) rectangle (11.7512,11.0876);
\draw [color=c, fill=c] (11.7512,10.9818) rectangle (11.791,11.0876);
\draw [color=c, fill=c] (11.791,10.9818) rectangle (11.8308,11.0876);
\draw [color=c, fill=c] (11.8308,10.9818) rectangle (11.8706,11.0876);
\draw [color=c, fill=c] (11.8706,10.9818) rectangle (11.9104,11.0876);
\draw [color=c, fill=c] (11.9104,10.9818) rectangle (11.9502,11.0876);
\draw [color=c, fill=c] (11.9502,10.9818) rectangle (11.99,11.0876);
\draw [color=c, fill=c] (11.99,10.9818) rectangle (12.0299,11.0876);
\draw [color=c, fill=c] (12.0299,10.9818) rectangle (12.0697,11.0876);
\draw [color=c, fill=c] (12.0697,10.9818) rectangle (12.1095,11.0876);
\draw [color=c, fill=c] (12.1095,10.9818) rectangle (12.1493,11.0876);
\draw [color=c, fill=c] (12.1493,10.9818) rectangle (12.1891,11.0876);
\draw [color=c, fill=c] (12.1891,10.9818) rectangle (12.2289,11.0876);
\draw [color=c, fill=c] (12.2289,10.9818) rectangle (12.2687,11.0876);
\draw [color=c, fill=c] (12.2687,10.9818) rectangle (12.3085,11.0876);
\draw [color=c, fill=c] (12.3085,10.9818) rectangle (12.3483,11.0876);
\draw [color=c, fill=c] (12.3483,10.9818) rectangle (12.3881,11.0876);
\draw [color=c, fill=c] (12.3881,10.9818) rectangle (12.4279,11.0876);
\draw [color=c, fill=c] (12.4279,10.9818) rectangle (12.4677,11.0876);
\draw [color=c, fill=c] (12.4677,10.9818) rectangle (12.5075,11.0876);
\draw [color=c, fill=c] (12.5075,10.9818) rectangle (12.5473,11.0876);
\draw [color=c, fill=c] (12.5473,10.9818) rectangle (12.5871,11.0876);
\draw [color=c, fill=c] (12.5871,10.9818) rectangle (12.6269,11.0876);
\draw [color=c, fill=c] (12.6269,10.9818) rectangle (12.6667,11.0876);
\draw [color=c, fill=c] (12.6667,10.9818) rectangle (12.7065,11.0876);
\draw [color=c, fill=c] (12.7065,10.9818) rectangle (12.7463,11.0876);
\draw [color=c, fill=c] (12.7463,10.9818) rectangle (12.7861,11.0876);
\draw [color=c, fill=c] (12.7861,10.9818) rectangle (12.8259,11.0876);
\draw [color=c, fill=c] (12.8259,10.9818) rectangle (12.8657,11.0876);
\draw [color=c, fill=c] (12.8657,10.9818) rectangle (12.9055,11.0876);
\draw [color=c, fill=c] (12.9055,10.9818) rectangle (12.9453,11.0876);
\draw [color=c, fill=c] (12.9453,10.9818) rectangle (12.9851,11.0876);
\draw [color=c, fill=c] (12.9851,10.9818) rectangle (13.0249,11.0876);
\draw [color=c, fill=c] (13.0249,10.9818) rectangle (13.0647,11.0876);
\draw [color=c, fill=c] (13.0647,10.9818) rectangle (13.1045,11.0876);
\draw [color=c, fill=c] (13.1045,10.9818) rectangle (13.1443,11.0876);
\draw [color=c, fill=c] (13.1443,10.9818) rectangle (13.1841,11.0876);
\draw [color=c, fill=c] (13.1841,10.9818) rectangle (13.2239,11.0876);
\draw [color=c, fill=c] (13.2239,10.9818) rectangle (13.2637,11.0876);
\draw [color=c, fill=c] (13.2637,10.9818) rectangle (13.3035,11.0876);
\draw [color=c, fill=c] (13.3035,10.9818) rectangle (13.3433,11.0876);
\draw [color=c, fill=c] (13.3433,10.9818) rectangle (13.3831,11.0876);
\draw [color=c, fill=c] (13.3831,10.9818) rectangle (13.4229,11.0876);
\draw [color=c, fill=c] (13.4229,10.9818) rectangle (13.4627,11.0876);
\draw [color=c, fill=c] (13.4627,10.9818) rectangle (13.5025,11.0876);
\draw [color=c, fill=c] (13.5025,10.9818) rectangle (13.5423,11.0876);
\draw [color=c, fill=c] (13.5423,10.9818) rectangle (13.5821,11.0876);
\draw [color=c, fill=c] (13.5821,10.9818) rectangle (13.6219,11.0876);
\draw [color=c, fill=c] (13.6219,10.9818) rectangle (13.6617,11.0876);
\draw [color=c, fill=c] (13.6617,10.9818) rectangle (13.7015,11.0876);
\draw [color=c, fill=c] (13.7015,10.9818) rectangle (13.7413,11.0876);
\draw [color=c, fill=c] (13.7413,10.9818) rectangle (13.7811,11.0876);
\draw [color=c, fill=c] (13.7811,10.9818) rectangle (13.8209,11.0876);
\draw [color=c, fill=c] (13.8209,10.9818) rectangle (13.8607,11.0876);
\draw [color=c, fill=c] (13.8607,10.9818) rectangle (13.9005,11.0876);
\draw [color=c, fill=c] (13.9005,10.9818) rectangle (13.9403,11.0876);
\draw [color=c, fill=c] (13.9403,10.9818) rectangle (13.9801,11.0876);
\draw [color=c, fill=c] (13.9801,10.9818) rectangle (14.0199,11.0876);
\draw [color=c, fill=c] (14.0199,10.9818) rectangle (14.0597,11.0876);
\draw [color=c, fill=c] (14.0597,10.9818) rectangle (14.0995,11.0876);
\draw [color=c, fill=c] (14.0995,10.9818) rectangle (14.1393,11.0876);
\draw [color=c, fill=c] (14.1393,10.9818) rectangle (14.1791,11.0876);
\draw [color=c, fill=c] (14.1791,10.9818) rectangle (14.2189,11.0876);
\draw [color=c, fill=c] (14.2189,10.9818) rectangle (14.2587,11.0876);
\draw [color=c, fill=c] (14.2587,10.9818) rectangle (14.2985,11.0876);
\definecolor{c}{rgb}{0,0.733333,1};
\draw [color=c, fill=c] (14.2985,10.9818) rectangle (14.3383,11.0876);
\draw [color=c, fill=c] (14.3383,10.9818) rectangle (14.3781,11.0876);
\draw [color=c, fill=c] (14.3781,10.9818) rectangle (14.4179,11.0876);
\draw [color=c, fill=c] (14.4179,10.9818) rectangle (14.4577,11.0876);
\draw [color=c, fill=c] (14.4577,10.9818) rectangle (14.4975,11.0876);
\draw [color=c, fill=c] (14.4975,10.9818) rectangle (14.5373,11.0876);
\draw [color=c, fill=c] (14.5373,10.9818) rectangle (14.5771,11.0876);
\draw [color=c, fill=c] (14.5771,10.9818) rectangle (14.6169,11.0876);
\draw [color=c, fill=c] (14.6169,10.9818) rectangle (14.6567,11.0876);
\draw [color=c, fill=c] (14.6567,10.9818) rectangle (14.6965,11.0876);
\draw [color=c, fill=c] (14.6965,10.9818) rectangle (14.7363,11.0876);
\draw [color=c, fill=c] (14.7363,10.9818) rectangle (14.7761,11.0876);
\draw [color=c, fill=c] (14.7761,10.9818) rectangle (14.8159,11.0876);
\draw [color=c, fill=c] (14.8159,10.9818) rectangle (14.8557,11.0876);
\draw [color=c, fill=c] (14.8557,10.9818) rectangle (14.8955,11.0876);
\draw [color=c, fill=c] (14.8955,10.9818) rectangle (14.9353,11.0876);
\draw [color=c, fill=c] (14.9353,10.9818) rectangle (14.9751,11.0876);
\draw [color=c, fill=c] (14.9751,10.9818) rectangle (15.0149,11.0876);
\draw [color=c, fill=c] (15.0149,10.9818) rectangle (15.0547,11.0876);
\draw [color=c, fill=c] (15.0547,10.9818) rectangle (15.0945,11.0876);
\draw [color=c, fill=c] (15.0945,10.9818) rectangle (15.1343,11.0876);
\draw [color=c, fill=c] (15.1343,10.9818) rectangle (15.1741,11.0876);
\draw [color=c, fill=c] (15.1741,10.9818) rectangle (15.2139,11.0876);
\draw [color=c, fill=c] (15.2139,10.9818) rectangle (15.2537,11.0876);
\draw [color=c, fill=c] (15.2537,10.9818) rectangle (15.2935,11.0876);
\draw [color=c, fill=c] (15.2935,10.9818) rectangle (15.3333,11.0876);
\draw [color=c, fill=c] (15.3333,10.9818) rectangle (15.3731,11.0876);
\draw [color=c, fill=c] (15.3731,10.9818) rectangle (15.4129,11.0876);
\draw [color=c, fill=c] (15.4129,10.9818) rectangle (15.4527,11.0876);
\draw [color=c, fill=c] (15.4527,10.9818) rectangle (15.4925,11.0876);
\draw [color=c, fill=c] (15.4925,10.9818) rectangle (15.5323,11.0876);
\draw [color=c, fill=c] (15.5323,10.9818) rectangle (15.5721,11.0876);
\draw [color=c, fill=c] (15.5721,10.9818) rectangle (15.6119,11.0876);
\draw [color=c, fill=c] (15.6119,10.9818) rectangle (15.6517,11.0876);
\draw [color=c, fill=c] (15.6517,10.9818) rectangle (15.6915,11.0876);
\draw [color=c, fill=c] (15.6915,10.9818) rectangle (15.7313,11.0876);
\draw [color=c, fill=c] (15.7313,10.9818) rectangle (15.7711,11.0876);
\draw [color=c, fill=c] (15.7711,10.9818) rectangle (15.8109,11.0876);
\draw [color=c, fill=c] (15.8109,10.9818) rectangle (15.8507,11.0876);
\draw [color=c, fill=c] (15.8507,10.9818) rectangle (15.8905,11.0876);
\draw [color=c, fill=c] (15.8905,10.9818) rectangle (15.9303,11.0876);
\draw [color=c, fill=c] (15.9303,10.9818) rectangle (15.9701,11.0876);
\draw [color=c, fill=c] (15.9701,10.9818) rectangle (16.01,11.0876);
\draw [color=c, fill=c] (16.01,10.9818) rectangle (16.0498,11.0876);
\draw [color=c, fill=c] (16.0498,10.9818) rectangle (16.0896,11.0876);
\draw [color=c, fill=c] (16.0896,10.9818) rectangle (16.1294,11.0876);
\draw [color=c, fill=c] (16.1294,10.9818) rectangle (16.1692,11.0876);
\draw [color=c, fill=c] (16.1692,10.9818) rectangle (16.209,11.0876);
\draw [color=c, fill=c] (16.209,10.9818) rectangle (16.2488,11.0876);
\draw [color=c, fill=c] (16.2488,10.9818) rectangle (16.2886,11.0876);
\draw [color=c, fill=c] (16.2886,10.9818) rectangle (16.3284,11.0876);
\draw [color=c, fill=c] (16.3284,10.9818) rectangle (16.3682,11.0876);
\draw [color=c, fill=c] (16.3682,10.9818) rectangle (16.408,11.0876);
\draw [color=c, fill=c] (16.408,10.9818) rectangle (16.4478,11.0876);
\draw [color=c, fill=c] (16.4478,10.9818) rectangle (16.4876,11.0876);
\draw [color=c, fill=c] (16.4876,10.9818) rectangle (16.5274,11.0876);
\draw [color=c, fill=c] (16.5274,10.9818) rectangle (16.5672,11.0876);
\draw [color=c, fill=c] (16.5672,10.9818) rectangle (16.607,11.0876);
\draw [color=c, fill=c] (16.607,10.9818) rectangle (16.6468,11.0876);
\draw [color=c, fill=c] (16.6468,10.9818) rectangle (16.6866,11.0876);
\draw [color=c, fill=c] (16.6866,10.9818) rectangle (16.7264,11.0876);
\draw [color=c, fill=c] (16.7264,10.9818) rectangle (16.7662,11.0876);
\draw [color=c, fill=c] (16.7662,10.9818) rectangle (16.806,11.0876);
\draw [color=c, fill=c] (16.806,10.9818) rectangle (16.8458,11.0876);
\draw [color=c, fill=c] (16.8458,10.9818) rectangle (16.8856,11.0876);
\draw [color=c, fill=c] (16.8856,10.9818) rectangle (16.9254,11.0876);
\draw [color=c, fill=c] (16.9254,10.9818) rectangle (16.9652,11.0876);
\draw [color=c, fill=c] (16.9652,10.9818) rectangle (17.005,11.0876);
\draw [color=c, fill=c] (17.005,10.9818) rectangle (17.0448,11.0876);
\draw [color=c, fill=c] (17.0448,10.9818) rectangle (17.0846,11.0876);
\draw [color=c, fill=c] (17.0846,10.9818) rectangle (17.1244,11.0876);
\draw [color=c, fill=c] (17.1244,10.9818) rectangle (17.1642,11.0876);
\draw [color=c, fill=c] (17.1642,10.9818) rectangle (17.204,11.0876);
\draw [color=c, fill=c] (17.204,10.9818) rectangle (17.2438,11.0876);
\draw [color=c, fill=c] (17.2438,10.9818) rectangle (17.2836,11.0876);
\draw [color=c, fill=c] (17.2836,10.9818) rectangle (17.3234,11.0876);
\draw [color=c, fill=c] (17.3234,10.9818) rectangle (17.3632,11.0876);
\draw [color=c, fill=c] (17.3632,10.9818) rectangle (17.403,11.0876);
\draw [color=c, fill=c] (17.403,10.9818) rectangle (17.4428,11.0876);
\draw [color=c, fill=c] (17.4428,10.9818) rectangle (17.4826,11.0876);
\draw [color=c, fill=c] (17.4826,10.9818) rectangle (17.5224,11.0876);
\draw [color=c, fill=c] (17.5224,10.9818) rectangle (17.5622,11.0876);
\draw [color=c, fill=c] (17.5622,10.9818) rectangle (17.602,11.0876);
\draw [color=c, fill=c] (17.602,10.9818) rectangle (17.6418,11.0876);
\draw [color=c, fill=c] (17.6418,10.9818) rectangle (17.6816,11.0876);
\draw [color=c, fill=c] (17.6816,10.9818) rectangle (17.7214,11.0876);
\draw [color=c, fill=c] (17.7214,10.9818) rectangle (17.7612,11.0876);
\draw [color=c, fill=c] (17.7612,10.9818) rectangle (17.801,11.0876);
\draw [color=c, fill=c] (17.801,10.9818) rectangle (17.8408,11.0876);
\draw [color=c, fill=c] (17.8408,10.9818) rectangle (17.8806,11.0876);
\draw [color=c, fill=c] (17.8806,10.9818) rectangle (17.9204,11.0876);
\draw [color=c, fill=c] (17.9204,10.9818) rectangle (17.9602,11.0876);
\draw [color=c, fill=c] (17.9602,10.9818) rectangle (18,11.0876);
\definecolor{c}{rgb}{0.2,0,1};
\draw [color=c, fill=c] (2,11.0876) rectangle (2.0398,11.1935);
\draw [color=c, fill=c] (2.0398,11.0876) rectangle (2.0796,11.1935);
\draw [color=c, fill=c] (2.0796,11.0876) rectangle (2.1194,11.1935);
\draw [color=c, fill=c] (2.1194,11.0876) rectangle (2.1592,11.1935);
\draw [color=c, fill=c] (2.1592,11.0876) rectangle (2.19901,11.1935);
\draw [color=c, fill=c] (2.19901,11.0876) rectangle (2.23881,11.1935);
\draw [color=c, fill=c] (2.23881,11.0876) rectangle (2.27861,11.1935);
\draw [color=c, fill=c] (2.27861,11.0876) rectangle (2.31841,11.1935);
\draw [color=c, fill=c] (2.31841,11.0876) rectangle (2.35821,11.1935);
\draw [color=c, fill=c] (2.35821,11.0876) rectangle (2.39801,11.1935);
\draw [color=c, fill=c] (2.39801,11.0876) rectangle (2.43781,11.1935);
\draw [color=c, fill=c] (2.43781,11.0876) rectangle (2.47761,11.1935);
\draw [color=c, fill=c] (2.47761,11.0876) rectangle (2.51741,11.1935);
\draw [color=c, fill=c] (2.51741,11.0876) rectangle (2.55721,11.1935);
\draw [color=c, fill=c] (2.55721,11.0876) rectangle (2.59702,11.1935);
\draw [color=c, fill=c] (2.59702,11.0876) rectangle (2.63682,11.1935);
\draw [color=c, fill=c] (2.63682,11.0876) rectangle (2.67662,11.1935);
\draw [color=c, fill=c] (2.67662,11.0876) rectangle (2.71642,11.1935);
\draw [color=c, fill=c] (2.71642,11.0876) rectangle (2.75622,11.1935);
\draw [color=c, fill=c] (2.75622,11.0876) rectangle (2.79602,11.1935);
\draw [color=c, fill=c] (2.79602,11.0876) rectangle (2.83582,11.1935);
\draw [color=c, fill=c] (2.83582,11.0876) rectangle (2.87562,11.1935);
\draw [color=c, fill=c] (2.87562,11.0876) rectangle (2.91542,11.1935);
\draw [color=c, fill=c] (2.91542,11.0876) rectangle (2.95522,11.1935);
\draw [color=c, fill=c] (2.95522,11.0876) rectangle (2.99502,11.1935);
\draw [color=c, fill=c] (2.99502,11.0876) rectangle (3.03483,11.1935);
\draw [color=c, fill=c] (3.03483,11.0876) rectangle (3.07463,11.1935);
\draw [color=c, fill=c] (3.07463,11.0876) rectangle (3.11443,11.1935);
\draw [color=c, fill=c] (3.11443,11.0876) rectangle (3.15423,11.1935);
\draw [color=c, fill=c] (3.15423,11.0876) rectangle (3.19403,11.1935);
\draw [color=c, fill=c] (3.19403,11.0876) rectangle (3.23383,11.1935);
\draw [color=c, fill=c] (3.23383,11.0876) rectangle (3.27363,11.1935);
\draw [color=c, fill=c] (3.27363,11.0876) rectangle (3.31343,11.1935);
\draw [color=c, fill=c] (3.31343,11.0876) rectangle (3.35323,11.1935);
\draw [color=c, fill=c] (3.35323,11.0876) rectangle (3.39303,11.1935);
\draw [color=c, fill=c] (3.39303,11.0876) rectangle (3.43284,11.1935);
\draw [color=c, fill=c] (3.43284,11.0876) rectangle (3.47264,11.1935);
\draw [color=c, fill=c] (3.47264,11.0876) rectangle (3.51244,11.1935);
\draw [color=c, fill=c] (3.51244,11.0876) rectangle (3.55224,11.1935);
\draw [color=c, fill=c] (3.55224,11.0876) rectangle (3.59204,11.1935);
\draw [color=c, fill=c] (3.59204,11.0876) rectangle (3.63184,11.1935);
\draw [color=c, fill=c] (3.63184,11.0876) rectangle (3.67164,11.1935);
\draw [color=c, fill=c] (3.67164,11.0876) rectangle (3.71144,11.1935);
\draw [color=c, fill=c] (3.71144,11.0876) rectangle (3.75124,11.1935);
\draw [color=c, fill=c] (3.75124,11.0876) rectangle (3.79104,11.1935);
\draw [color=c, fill=c] (3.79104,11.0876) rectangle (3.83085,11.1935);
\draw [color=c, fill=c] (3.83085,11.0876) rectangle (3.87065,11.1935);
\draw [color=c, fill=c] (3.87065,11.0876) rectangle (3.91045,11.1935);
\draw [color=c, fill=c] (3.91045,11.0876) rectangle (3.95025,11.1935);
\draw [color=c, fill=c] (3.95025,11.0876) rectangle (3.99005,11.1935);
\draw [color=c, fill=c] (3.99005,11.0876) rectangle (4.02985,11.1935);
\draw [color=c, fill=c] (4.02985,11.0876) rectangle (4.06965,11.1935);
\draw [color=c, fill=c] (4.06965,11.0876) rectangle (4.10945,11.1935);
\draw [color=c, fill=c] (4.10945,11.0876) rectangle (4.14925,11.1935);
\draw [color=c, fill=c] (4.14925,11.0876) rectangle (4.18905,11.1935);
\draw [color=c, fill=c] (4.18905,11.0876) rectangle (4.22886,11.1935);
\draw [color=c, fill=c] (4.22886,11.0876) rectangle (4.26866,11.1935);
\draw [color=c, fill=c] (4.26866,11.0876) rectangle (4.30846,11.1935);
\draw [color=c, fill=c] (4.30846,11.0876) rectangle (4.34826,11.1935);
\draw [color=c, fill=c] (4.34826,11.0876) rectangle (4.38806,11.1935);
\draw [color=c, fill=c] (4.38806,11.0876) rectangle (4.42786,11.1935);
\draw [color=c, fill=c] (4.42786,11.0876) rectangle (4.46766,11.1935);
\draw [color=c, fill=c] (4.46766,11.0876) rectangle (4.50746,11.1935);
\draw [color=c, fill=c] (4.50746,11.0876) rectangle (4.54726,11.1935);
\draw [color=c, fill=c] (4.54726,11.0876) rectangle (4.58706,11.1935);
\draw [color=c, fill=c] (4.58706,11.0876) rectangle (4.62687,11.1935);
\draw [color=c, fill=c] (4.62687,11.0876) rectangle (4.66667,11.1935);
\draw [color=c, fill=c] (4.66667,11.0876) rectangle (4.70647,11.1935);
\draw [color=c, fill=c] (4.70647,11.0876) rectangle (4.74627,11.1935);
\draw [color=c, fill=c] (4.74627,11.0876) rectangle (4.78607,11.1935);
\draw [color=c, fill=c] (4.78607,11.0876) rectangle (4.82587,11.1935);
\draw [color=c, fill=c] (4.82587,11.0876) rectangle (4.86567,11.1935);
\draw [color=c, fill=c] (4.86567,11.0876) rectangle (4.90547,11.1935);
\draw [color=c, fill=c] (4.90547,11.0876) rectangle (4.94527,11.1935);
\draw [color=c, fill=c] (4.94527,11.0876) rectangle (4.98507,11.1935);
\draw [color=c, fill=c] (4.98507,11.0876) rectangle (5.02488,11.1935);
\draw [color=c, fill=c] (5.02488,11.0876) rectangle (5.06468,11.1935);
\draw [color=c, fill=c] (5.06468,11.0876) rectangle (5.10448,11.1935);
\draw [color=c, fill=c] (5.10448,11.0876) rectangle (5.14428,11.1935);
\draw [color=c, fill=c] (5.14428,11.0876) rectangle (5.18408,11.1935);
\draw [color=c, fill=c] (5.18408,11.0876) rectangle (5.22388,11.1935);
\draw [color=c, fill=c] (5.22388,11.0876) rectangle (5.26368,11.1935);
\draw [color=c, fill=c] (5.26368,11.0876) rectangle (5.30348,11.1935);
\draw [color=c, fill=c] (5.30348,11.0876) rectangle (5.34328,11.1935);
\draw [color=c, fill=c] (5.34328,11.0876) rectangle (5.38308,11.1935);
\draw [color=c, fill=c] (5.38308,11.0876) rectangle (5.42289,11.1935);
\draw [color=c, fill=c] (5.42289,11.0876) rectangle (5.46269,11.1935);
\draw [color=c, fill=c] (5.46269,11.0876) rectangle (5.50249,11.1935);
\draw [color=c, fill=c] (5.50249,11.0876) rectangle (5.54229,11.1935);
\draw [color=c, fill=c] (5.54229,11.0876) rectangle (5.58209,11.1935);
\draw [color=c, fill=c] (5.58209,11.0876) rectangle (5.62189,11.1935);
\draw [color=c, fill=c] (5.62189,11.0876) rectangle (5.66169,11.1935);
\draw [color=c, fill=c] (5.66169,11.0876) rectangle (5.70149,11.1935);
\draw [color=c, fill=c] (5.70149,11.0876) rectangle (5.74129,11.1935);
\draw [color=c, fill=c] (5.74129,11.0876) rectangle (5.78109,11.1935);
\draw [color=c, fill=c] (5.78109,11.0876) rectangle (5.8209,11.1935);
\draw [color=c, fill=c] (5.8209,11.0876) rectangle (5.8607,11.1935);
\draw [color=c, fill=c] (5.8607,11.0876) rectangle (5.9005,11.1935);
\draw [color=c, fill=c] (5.9005,11.0876) rectangle (5.9403,11.1935);
\draw [color=c, fill=c] (5.9403,11.0876) rectangle (5.9801,11.1935);
\draw [color=c, fill=c] (5.9801,11.0876) rectangle (6.0199,11.1935);
\draw [color=c, fill=c] (6.0199,11.0876) rectangle (6.0597,11.1935);
\draw [color=c, fill=c] (6.0597,11.0876) rectangle (6.0995,11.1935);
\draw [color=c, fill=c] (6.0995,11.0876) rectangle (6.1393,11.1935);
\draw [color=c, fill=c] (6.1393,11.0876) rectangle (6.1791,11.1935);
\draw [color=c, fill=c] (6.1791,11.0876) rectangle (6.21891,11.1935);
\draw [color=c, fill=c] (6.21891,11.0876) rectangle (6.25871,11.1935);
\draw [color=c, fill=c] (6.25871,11.0876) rectangle (6.29851,11.1935);
\draw [color=c, fill=c] (6.29851,11.0876) rectangle (6.33831,11.1935);
\draw [color=c, fill=c] (6.33831,11.0876) rectangle (6.37811,11.1935);
\draw [color=c, fill=c] (6.37811,11.0876) rectangle (6.41791,11.1935);
\draw [color=c, fill=c] (6.41791,11.0876) rectangle (6.45771,11.1935);
\draw [color=c, fill=c] (6.45771,11.0876) rectangle (6.49751,11.1935);
\draw [color=c, fill=c] (6.49751,11.0876) rectangle (6.53731,11.1935);
\draw [color=c, fill=c] (6.53731,11.0876) rectangle (6.57711,11.1935);
\draw [color=c, fill=c] (6.57711,11.0876) rectangle (6.61692,11.1935);
\draw [color=c, fill=c] (6.61692,11.0876) rectangle (6.65672,11.1935);
\draw [color=c, fill=c] (6.65672,11.0876) rectangle (6.69652,11.1935);
\draw [color=c, fill=c] (6.69652,11.0876) rectangle (6.73632,11.1935);
\draw [color=c, fill=c] (6.73632,11.0876) rectangle (6.77612,11.1935);
\draw [color=c, fill=c] (6.77612,11.0876) rectangle (6.81592,11.1935);
\draw [color=c, fill=c] (6.81592,11.0876) rectangle (6.85572,11.1935);
\draw [color=c, fill=c] (6.85572,11.0876) rectangle (6.89552,11.1935);
\draw [color=c, fill=c] (6.89552,11.0876) rectangle (6.93532,11.1935);
\draw [color=c, fill=c] (6.93532,11.0876) rectangle (6.97512,11.1935);
\draw [color=c, fill=c] (6.97512,11.0876) rectangle (7.01493,11.1935);
\draw [color=c, fill=c] (7.01493,11.0876) rectangle (7.05473,11.1935);
\draw [color=c, fill=c] (7.05473,11.0876) rectangle (7.09453,11.1935);
\draw [color=c, fill=c] (7.09453,11.0876) rectangle (7.13433,11.1935);
\draw [color=c, fill=c] (7.13433,11.0876) rectangle (7.17413,11.1935);
\draw [color=c, fill=c] (7.17413,11.0876) rectangle (7.21393,11.1935);
\draw [color=c, fill=c] (7.21393,11.0876) rectangle (7.25373,11.1935);
\draw [color=c, fill=c] (7.25373,11.0876) rectangle (7.29353,11.1935);
\draw [color=c, fill=c] (7.29353,11.0876) rectangle (7.33333,11.1935);
\draw [color=c, fill=c] (7.33333,11.0876) rectangle (7.37313,11.1935);
\draw [color=c, fill=c] (7.37313,11.0876) rectangle (7.41294,11.1935);
\draw [color=c, fill=c] (7.41294,11.0876) rectangle (7.45274,11.1935);
\draw [color=c, fill=c] (7.45274,11.0876) rectangle (7.49254,11.1935);
\draw [color=c, fill=c] (7.49254,11.0876) rectangle (7.53234,11.1935);
\draw [color=c, fill=c] (7.53234,11.0876) rectangle (7.57214,11.1935);
\draw [color=c, fill=c] (7.57214,11.0876) rectangle (7.61194,11.1935);
\draw [color=c, fill=c] (7.61194,11.0876) rectangle (7.65174,11.1935);
\draw [color=c, fill=c] (7.65174,11.0876) rectangle (7.69154,11.1935);
\draw [color=c, fill=c] (7.69154,11.0876) rectangle (7.73134,11.1935);
\draw [color=c, fill=c] (7.73134,11.0876) rectangle (7.77114,11.1935);
\draw [color=c, fill=c] (7.77114,11.0876) rectangle (7.81095,11.1935);
\draw [color=c, fill=c] (7.81095,11.0876) rectangle (7.85075,11.1935);
\draw [color=c, fill=c] (7.85075,11.0876) rectangle (7.89055,11.1935);
\draw [color=c, fill=c] (7.89055,11.0876) rectangle (7.93035,11.1935);
\draw [color=c, fill=c] (7.93035,11.0876) rectangle (7.97015,11.1935);
\definecolor{c}{rgb}{0,0.0800001,1};
\draw [color=c, fill=c] (7.97015,11.0876) rectangle (8.00995,11.1935);
\draw [color=c, fill=c] (8.00995,11.0876) rectangle (8.04975,11.1935);
\draw [color=c, fill=c] (8.04975,11.0876) rectangle (8.08955,11.1935);
\draw [color=c, fill=c] (8.08955,11.0876) rectangle (8.12935,11.1935);
\draw [color=c, fill=c] (8.12935,11.0876) rectangle (8.16915,11.1935);
\draw [color=c, fill=c] (8.16915,11.0876) rectangle (8.20895,11.1935);
\draw [color=c, fill=c] (8.20895,11.0876) rectangle (8.24876,11.1935);
\draw [color=c, fill=c] (8.24876,11.0876) rectangle (8.28856,11.1935);
\draw [color=c, fill=c] (8.28856,11.0876) rectangle (8.32836,11.1935);
\draw [color=c, fill=c] (8.32836,11.0876) rectangle (8.36816,11.1935);
\draw [color=c, fill=c] (8.36816,11.0876) rectangle (8.40796,11.1935);
\draw [color=c, fill=c] (8.40796,11.0876) rectangle (8.44776,11.1935);
\draw [color=c, fill=c] (8.44776,11.0876) rectangle (8.48756,11.1935);
\draw [color=c, fill=c] (8.48756,11.0876) rectangle (8.52736,11.1935);
\draw [color=c, fill=c] (8.52736,11.0876) rectangle (8.56716,11.1935);
\draw [color=c, fill=c] (8.56716,11.0876) rectangle (8.60697,11.1935);
\draw [color=c, fill=c] (8.60697,11.0876) rectangle (8.64677,11.1935);
\draw [color=c, fill=c] (8.64677,11.0876) rectangle (8.68657,11.1935);
\draw [color=c, fill=c] (8.68657,11.0876) rectangle (8.72637,11.1935);
\draw [color=c, fill=c] (8.72637,11.0876) rectangle (8.76617,11.1935);
\draw [color=c, fill=c] (8.76617,11.0876) rectangle (8.80597,11.1935);
\draw [color=c, fill=c] (8.80597,11.0876) rectangle (8.84577,11.1935);
\draw [color=c, fill=c] (8.84577,11.0876) rectangle (8.88557,11.1935);
\draw [color=c, fill=c] (8.88557,11.0876) rectangle (8.92537,11.1935);
\draw [color=c, fill=c] (8.92537,11.0876) rectangle (8.96517,11.1935);
\draw [color=c, fill=c] (8.96517,11.0876) rectangle (9.00498,11.1935);
\draw [color=c, fill=c] (9.00498,11.0876) rectangle (9.04478,11.1935);
\draw [color=c, fill=c] (9.04478,11.0876) rectangle (9.08458,11.1935);
\draw [color=c, fill=c] (9.08458,11.0876) rectangle (9.12438,11.1935);
\draw [color=c, fill=c] (9.12438,11.0876) rectangle (9.16418,11.1935);
\draw [color=c, fill=c] (9.16418,11.0876) rectangle (9.20398,11.1935);
\draw [color=c, fill=c] (9.20398,11.0876) rectangle (9.24378,11.1935);
\draw [color=c, fill=c] (9.24378,11.0876) rectangle (9.28358,11.1935);
\draw [color=c, fill=c] (9.28358,11.0876) rectangle (9.32338,11.1935);
\draw [color=c, fill=c] (9.32338,11.0876) rectangle (9.36318,11.1935);
\draw [color=c, fill=c] (9.36318,11.0876) rectangle (9.40298,11.1935);
\draw [color=c, fill=c] (9.40298,11.0876) rectangle (9.44279,11.1935);
\draw [color=c, fill=c] (9.44279,11.0876) rectangle (9.48259,11.1935);
\draw [color=c, fill=c] (9.48259,11.0876) rectangle (9.52239,11.1935);
\draw [color=c, fill=c] (9.52239,11.0876) rectangle (9.56219,11.1935);
\draw [color=c, fill=c] (9.56219,11.0876) rectangle (9.60199,11.1935);
\definecolor{c}{rgb}{0,0.266667,1};
\draw [color=c, fill=c] (9.60199,11.0876) rectangle (9.64179,11.1935);
\draw [color=c, fill=c] (9.64179,11.0876) rectangle (9.68159,11.1935);
\draw [color=c, fill=c] (9.68159,11.0876) rectangle (9.72139,11.1935);
\draw [color=c, fill=c] (9.72139,11.0876) rectangle (9.76119,11.1935);
\draw [color=c, fill=c] (9.76119,11.0876) rectangle (9.80099,11.1935);
\draw [color=c, fill=c] (9.80099,11.0876) rectangle (9.8408,11.1935);
\draw [color=c, fill=c] (9.8408,11.0876) rectangle (9.8806,11.1935);
\draw [color=c, fill=c] (9.8806,11.0876) rectangle (9.9204,11.1935);
\draw [color=c, fill=c] (9.9204,11.0876) rectangle (9.9602,11.1935);
\draw [color=c, fill=c] (9.9602,11.0876) rectangle (10,11.1935);
\draw [color=c, fill=c] (10,11.0876) rectangle (10.0398,11.1935);
\draw [color=c, fill=c] (10.0398,11.0876) rectangle (10.0796,11.1935);
\draw [color=c, fill=c] (10.0796,11.0876) rectangle (10.1194,11.1935);
\draw [color=c, fill=c] (10.1194,11.0876) rectangle (10.1592,11.1935);
\draw [color=c, fill=c] (10.1592,11.0876) rectangle (10.199,11.1935);
\draw [color=c, fill=c] (10.199,11.0876) rectangle (10.2388,11.1935);
\draw [color=c, fill=c] (10.2388,11.0876) rectangle (10.2786,11.1935);
\draw [color=c, fill=c] (10.2786,11.0876) rectangle (10.3184,11.1935);
\draw [color=c, fill=c] (10.3184,11.0876) rectangle (10.3582,11.1935);
\draw [color=c, fill=c] (10.3582,11.0876) rectangle (10.398,11.1935);
\draw [color=c, fill=c] (10.398,11.0876) rectangle (10.4378,11.1935);
\draw [color=c, fill=c] (10.4378,11.0876) rectangle (10.4776,11.1935);
\draw [color=c, fill=c] (10.4776,11.0876) rectangle (10.5174,11.1935);
\draw [color=c, fill=c] (10.5174,11.0876) rectangle (10.5572,11.1935);
\draw [color=c, fill=c] (10.5572,11.0876) rectangle (10.597,11.1935);
\draw [color=c, fill=c] (10.597,11.0876) rectangle (10.6368,11.1935);
\draw [color=c, fill=c] (10.6368,11.0876) rectangle (10.6766,11.1935);
\draw [color=c, fill=c] (10.6766,11.0876) rectangle (10.7164,11.1935);
\draw [color=c, fill=c] (10.7164,11.0876) rectangle (10.7562,11.1935);
\draw [color=c, fill=c] (10.7562,11.0876) rectangle (10.796,11.1935);
\draw [color=c, fill=c] (10.796,11.0876) rectangle (10.8358,11.1935);
\draw [color=c, fill=c] (10.8358,11.0876) rectangle (10.8756,11.1935);
\draw [color=c, fill=c] (10.8756,11.0876) rectangle (10.9154,11.1935);
\draw [color=c, fill=c] (10.9154,11.0876) rectangle (10.9552,11.1935);
\definecolor{c}{rgb}{0,0.546666,1};
\draw [color=c, fill=c] (10.9552,11.0876) rectangle (10.995,11.1935);
\draw [color=c, fill=c] (10.995,11.0876) rectangle (11.0348,11.1935);
\draw [color=c, fill=c] (11.0348,11.0876) rectangle (11.0746,11.1935);
\draw [color=c, fill=c] (11.0746,11.0876) rectangle (11.1144,11.1935);
\draw [color=c, fill=c] (11.1144,11.0876) rectangle (11.1542,11.1935);
\draw [color=c, fill=c] (11.1542,11.0876) rectangle (11.194,11.1935);
\draw [color=c, fill=c] (11.194,11.0876) rectangle (11.2338,11.1935);
\draw [color=c, fill=c] (11.2338,11.0876) rectangle (11.2736,11.1935);
\draw [color=c, fill=c] (11.2736,11.0876) rectangle (11.3134,11.1935);
\draw [color=c, fill=c] (11.3134,11.0876) rectangle (11.3532,11.1935);
\draw [color=c, fill=c] (11.3532,11.0876) rectangle (11.393,11.1935);
\draw [color=c, fill=c] (11.393,11.0876) rectangle (11.4328,11.1935);
\draw [color=c, fill=c] (11.4328,11.0876) rectangle (11.4726,11.1935);
\draw [color=c, fill=c] (11.4726,11.0876) rectangle (11.5124,11.1935);
\draw [color=c, fill=c] (11.5124,11.0876) rectangle (11.5522,11.1935);
\draw [color=c, fill=c] (11.5522,11.0876) rectangle (11.592,11.1935);
\draw [color=c, fill=c] (11.592,11.0876) rectangle (11.6318,11.1935);
\draw [color=c, fill=c] (11.6318,11.0876) rectangle (11.6716,11.1935);
\draw [color=c, fill=c] (11.6716,11.0876) rectangle (11.7114,11.1935);
\draw [color=c, fill=c] (11.7114,11.0876) rectangle (11.7512,11.1935);
\draw [color=c, fill=c] (11.7512,11.0876) rectangle (11.791,11.1935);
\draw [color=c, fill=c] (11.791,11.0876) rectangle (11.8308,11.1935);
\draw [color=c, fill=c] (11.8308,11.0876) rectangle (11.8706,11.1935);
\draw [color=c, fill=c] (11.8706,11.0876) rectangle (11.9104,11.1935);
\draw [color=c, fill=c] (11.9104,11.0876) rectangle (11.9502,11.1935);
\draw [color=c, fill=c] (11.9502,11.0876) rectangle (11.99,11.1935);
\draw [color=c, fill=c] (11.99,11.0876) rectangle (12.0299,11.1935);
\draw [color=c, fill=c] (12.0299,11.0876) rectangle (12.0697,11.1935);
\draw [color=c, fill=c] (12.0697,11.0876) rectangle (12.1095,11.1935);
\draw [color=c, fill=c] (12.1095,11.0876) rectangle (12.1493,11.1935);
\draw [color=c, fill=c] (12.1493,11.0876) rectangle (12.1891,11.1935);
\draw [color=c, fill=c] (12.1891,11.0876) rectangle (12.2289,11.1935);
\draw [color=c, fill=c] (12.2289,11.0876) rectangle (12.2687,11.1935);
\draw [color=c, fill=c] (12.2687,11.0876) rectangle (12.3085,11.1935);
\draw [color=c, fill=c] (12.3085,11.0876) rectangle (12.3483,11.1935);
\draw [color=c, fill=c] (12.3483,11.0876) rectangle (12.3881,11.1935);
\draw [color=c, fill=c] (12.3881,11.0876) rectangle (12.4279,11.1935);
\draw [color=c, fill=c] (12.4279,11.0876) rectangle (12.4677,11.1935);
\draw [color=c, fill=c] (12.4677,11.0876) rectangle (12.5075,11.1935);
\draw [color=c, fill=c] (12.5075,11.0876) rectangle (12.5473,11.1935);
\draw [color=c, fill=c] (12.5473,11.0876) rectangle (12.5871,11.1935);
\draw [color=c, fill=c] (12.5871,11.0876) rectangle (12.6269,11.1935);
\draw [color=c, fill=c] (12.6269,11.0876) rectangle (12.6667,11.1935);
\draw [color=c, fill=c] (12.6667,11.0876) rectangle (12.7065,11.1935);
\draw [color=c, fill=c] (12.7065,11.0876) rectangle (12.7463,11.1935);
\draw [color=c, fill=c] (12.7463,11.0876) rectangle (12.7861,11.1935);
\draw [color=c, fill=c] (12.7861,11.0876) rectangle (12.8259,11.1935);
\draw [color=c, fill=c] (12.8259,11.0876) rectangle (12.8657,11.1935);
\draw [color=c, fill=c] (12.8657,11.0876) rectangle (12.9055,11.1935);
\draw [color=c, fill=c] (12.9055,11.0876) rectangle (12.9453,11.1935);
\draw [color=c, fill=c] (12.9453,11.0876) rectangle (12.9851,11.1935);
\draw [color=c, fill=c] (12.9851,11.0876) rectangle (13.0249,11.1935);
\draw [color=c, fill=c] (13.0249,11.0876) rectangle (13.0647,11.1935);
\draw [color=c, fill=c] (13.0647,11.0876) rectangle (13.1045,11.1935);
\draw [color=c, fill=c] (13.1045,11.0876) rectangle (13.1443,11.1935);
\draw [color=c, fill=c] (13.1443,11.0876) rectangle (13.1841,11.1935);
\draw [color=c, fill=c] (13.1841,11.0876) rectangle (13.2239,11.1935);
\draw [color=c, fill=c] (13.2239,11.0876) rectangle (13.2637,11.1935);
\draw [color=c, fill=c] (13.2637,11.0876) rectangle (13.3035,11.1935);
\draw [color=c, fill=c] (13.3035,11.0876) rectangle (13.3433,11.1935);
\draw [color=c, fill=c] (13.3433,11.0876) rectangle (13.3831,11.1935);
\draw [color=c, fill=c] (13.3831,11.0876) rectangle (13.4229,11.1935);
\draw [color=c, fill=c] (13.4229,11.0876) rectangle (13.4627,11.1935);
\draw [color=c, fill=c] (13.4627,11.0876) rectangle (13.5025,11.1935);
\draw [color=c, fill=c] (13.5025,11.0876) rectangle (13.5423,11.1935);
\draw [color=c, fill=c] (13.5423,11.0876) rectangle (13.5821,11.1935);
\draw [color=c, fill=c] (13.5821,11.0876) rectangle (13.6219,11.1935);
\draw [color=c, fill=c] (13.6219,11.0876) rectangle (13.6617,11.1935);
\draw [color=c, fill=c] (13.6617,11.0876) rectangle (13.7015,11.1935);
\draw [color=c, fill=c] (13.7015,11.0876) rectangle (13.7413,11.1935);
\draw [color=c, fill=c] (13.7413,11.0876) rectangle (13.7811,11.1935);
\draw [color=c, fill=c] (13.7811,11.0876) rectangle (13.8209,11.1935);
\draw [color=c, fill=c] (13.8209,11.0876) rectangle (13.8607,11.1935);
\draw [color=c, fill=c] (13.8607,11.0876) rectangle (13.9005,11.1935);
\draw [color=c, fill=c] (13.9005,11.0876) rectangle (13.9403,11.1935);
\draw [color=c, fill=c] (13.9403,11.0876) rectangle (13.9801,11.1935);
\draw [color=c, fill=c] (13.9801,11.0876) rectangle (14.0199,11.1935);
\draw [color=c, fill=c] (14.0199,11.0876) rectangle (14.0597,11.1935);
\draw [color=c, fill=c] (14.0597,11.0876) rectangle (14.0995,11.1935);
\draw [color=c, fill=c] (14.0995,11.0876) rectangle (14.1393,11.1935);
\draw [color=c, fill=c] (14.1393,11.0876) rectangle (14.1791,11.1935);
\draw [color=c, fill=c] (14.1791,11.0876) rectangle (14.2189,11.1935);
\draw [color=c, fill=c] (14.2189,11.0876) rectangle (14.2587,11.1935);
\draw [color=c, fill=c] (14.2587,11.0876) rectangle (14.2985,11.1935);
\draw [color=c, fill=c] (14.2985,11.0876) rectangle (14.3383,11.1935);
\definecolor{c}{rgb}{0,0.733333,1};
\draw [color=c, fill=c] (14.3383,11.0876) rectangle (14.3781,11.1935);
\draw [color=c, fill=c] (14.3781,11.0876) rectangle (14.4179,11.1935);
\draw [color=c, fill=c] (14.4179,11.0876) rectangle (14.4577,11.1935);
\draw [color=c, fill=c] (14.4577,11.0876) rectangle (14.4975,11.1935);
\draw [color=c, fill=c] (14.4975,11.0876) rectangle (14.5373,11.1935);
\draw [color=c, fill=c] (14.5373,11.0876) rectangle (14.5771,11.1935);
\draw [color=c, fill=c] (14.5771,11.0876) rectangle (14.6169,11.1935);
\draw [color=c, fill=c] (14.6169,11.0876) rectangle (14.6567,11.1935);
\draw [color=c, fill=c] (14.6567,11.0876) rectangle (14.6965,11.1935);
\draw [color=c, fill=c] (14.6965,11.0876) rectangle (14.7363,11.1935);
\draw [color=c, fill=c] (14.7363,11.0876) rectangle (14.7761,11.1935);
\draw [color=c, fill=c] (14.7761,11.0876) rectangle (14.8159,11.1935);
\draw [color=c, fill=c] (14.8159,11.0876) rectangle (14.8557,11.1935);
\draw [color=c, fill=c] (14.8557,11.0876) rectangle (14.8955,11.1935);
\draw [color=c, fill=c] (14.8955,11.0876) rectangle (14.9353,11.1935);
\draw [color=c, fill=c] (14.9353,11.0876) rectangle (14.9751,11.1935);
\draw [color=c, fill=c] (14.9751,11.0876) rectangle (15.0149,11.1935);
\draw [color=c, fill=c] (15.0149,11.0876) rectangle (15.0547,11.1935);
\draw [color=c, fill=c] (15.0547,11.0876) rectangle (15.0945,11.1935);
\draw [color=c, fill=c] (15.0945,11.0876) rectangle (15.1343,11.1935);
\draw [color=c, fill=c] (15.1343,11.0876) rectangle (15.1741,11.1935);
\draw [color=c, fill=c] (15.1741,11.0876) rectangle (15.2139,11.1935);
\draw [color=c, fill=c] (15.2139,11.0876) rectangle (15.2537,11.1935);
\draw [color=c, fill=c] (15.2537,11.0876) rectangle (15.2935,11.1935);
\draw [color=c, fill=c] (15.2935,11.0876) rectangle (15.3333,11.1935);
\draw [color=c, fill=c] (15.3333,11.0876) rectangle (15.3731,11.1935);
\draw [color=c, fill=c] (15.3731,11.0876) rectangle (15.4129,11.1935);
\draw [color=c, fill=c] (15.4129,11.0876) rectangle (15.4527,11.1935);
\draw [color=c, fill=c] (15.4527,11.0876) rectangle (15.4925,11.1935);
\draw [color=c, fill=c] (15.4925,11.0876) rectangle (15.5323,11.1935);
\draw [color=c, fill=c] (15.5323,11.0876) rectangle (15.5721,11.1935);
\draw [color=c, fill=c] (15.5721,11.0876) rectangle (15.6119,11.1935);
\draw [color=c, fill=c] (15.6119,11.0876) rectangle (15.6517,11.1935);
\draw [color=c, fill=c] (15.6517,11.0876) rectangle (15.6915,11.1935);
\draw [color=c, fill=c] (15.6915,11.0876) rectangle (15.7313,11.1935);
\draw [color=c, fill=c] (15.7313,11.0876) rectangle (15.7711,11.1935);
\draw [color=c, fill=c] (15.7711,11.0876) rectangle (15.8109,11.1935);
\draw [color=c, fill=c] (15.8109,11.0876) rectangle (15.8507,11.1935);
\draw [color=c, fill=c] (15.8507,11.0876) rectangle (15.8905,11.1935);
\draw [color=c, fill=c] (15.8905,11.0876) rectangle (15.9303,11.1935);
\draw [color=c, fill=c] (15.9303,11.0876) rectangle (15.9701,11.1935);
\draw [color=c, fill=c] (15.9701,11.0876) rectangle (16.01,11.1935);
\draw [color=c, fill=c] (16.01,11.0876) rectangle (16.0498,11.1935);
\draw [color=c, fill=c] (16.0498,11.0876) rectangle (16.0896,11.1935);
\draw [color=c, fill=c] (16.0896,11.0876) rectangle (16.1294,11.1935);
\draw [color=c, fill=c] (16.1294,11.0876) rectangle (16.1692,11.1935);
\draw [color=c, fill=c] (16.1692,11.0876) rectangle (16.209,11.1935);
\draw [color=c, fill=c] (16.209,11.0876) rectangle (16.2488,11.1935);
\draw [color=c, fill=c] (16.2488,11.0876) rectangle (16.2886,11.1935);
\draw [color=c, fill=c] (16.2886,11.0876) rectangle (16.3284,11.1935);
\draw [color=c, fill=c] (16.3284,11.0876) rectangle (16.3682,11.1935);
\draw [color=c, fill=c] (16.3682,11.0876) rectangle (16.408,11.1935);
\draw [color=c, fill=c] (16.408,11.0876) rectangle (16.4478,11.1935);
\draw [color=c, fill=c] (16.4478,11.0876) rectangle (16.4876,11.1935);
\draw [color=c, fill=c] (16.4876,11.0876) rectangle (16.5274,11.1935);
\draw [color=c, fill=c] (16.5274,11.0876) rectangle (16.5672,11.1935);
\draw [color=c, fill=c] (16.5672,11.0876) rectangle (16.607,11.1935);
\draw [color=c, fill=c] (16.607,11.0876) rectangle (16.6468,11.1935);
\draw [color=c, fill=c] (16.6468,11.0876) rectangle (16.6866,11.1935);
\draw [color=c, fill=c] (16.6866,11.0876) rectangle (16.7264,11.1935);
\draw [color=c, fill=c] (16.7264,11.0876) rectangle (16.7662,11.1935);
\draw [color=c, fill=c] (16.7662,11.0876) rectangle (16.806,11.1935);
\draw [color=c, fill=c] (16.806,11.0876) rectangle (16.8458,11.1935);
\draw [color=c, fill=c] (16.8458,11.0876) rectangle (16.8856,11.1935);
\draw [color=c, fill=c] (16.8856,11.0876) rectangle (16.9254,11.1935);
\draw [color=c, fill=c] (16.9254,11.0876) rectangle (16.9652,11.1935);
\draw [color=c, fill=c] (16.9652,11.0876) rectangle (17.005,11.1935);
\draw [color=c, fill=c] (17.005,11.0876) rectangle (17.0448,11.1935);
\draw [color=c, fill=c] (17.0448,11.0876) rectangle (17.0846,11.1935);
\draw [color=c, fill=c] (17.0846,11.0876) rectangle (17.1244,11.1935);
\draw [color=c, fill=c] (17.1244,11.0876) rectangle (17.1642,11.1935);
\draw [color=c, fill=c] (17.1642,11.0876) rectangle (17.204,11.1935);
\draw [color=c, fill=c] (17.204,11.0876) rectangle (17.2438,11.1935);
\draw [color=c, fill=c] (17.2438,11.0876) rectangle (17.2836,11.1935);
\draw [color=c, fill=c] (17.2836,11.0876) rectangle (17.3234,11.1935);
\draw [color=c, fill=c] (17.3234,11.0876) rectangle (17.3632,11.1935);
\draw [color=c, fill=c] (17.3632,11.0876) rectangle (17.403,11.1935);
\draw [color=c, fill=c] (17.403,11.0876) rectangle (17.4428,11.1935);
\draw [color=c, fill=c] (17.4428,11.0876) rectangle (17.4826,11.1935);
\draw [color=c, fill=c] (17.4826,11.0876) rectangle (17.5224,11.1935);
\draw [color=c, fill=c] (17.5224,11.0876) rectangle (17.5622,11.1935);
\draw [color=c, fill=c] (17.5622,11.0876) rectangle (17.602,11.1935);
\draw [color=c, fill=c] (17.602,11.0876) rectangle (17.6418,11.1935);
\draw [color=c, fill=c] (17.6418,11.0876) rectangle (17.6816,11.1935);
\draw [color=c, fill=c] (17.6816,11.0876) rectangle (17.7214,11.1935);
\draw [color=c, fill=c] (17.7214,11.0876) rectangle (17.7612,11.1935);
\draw [color=c, fill=c] (17.7612,11.0876) rectangle (17.801,11.1935);
\draw [color=c, fill=c] (17.801,11.0876) rectangle (17.8408,11.1935);
\draw [color=c, fill=c] (17.8408,11.0876) rectangle (17.8806,11.1935);
\draw [color=c, fill=c] (17.8806,11.0876) rectangle (17.9204,11.1935);
\draw [color=c, fill=c] (17.9204,11.0876) rectangle (17.9602,11.1935);
\draw [color=c, fill=c] (17.9602,11.0876) rectangle (18,11.1935);
\definecolor{c}{rgb}{0.2,0,1};
\draw [color=c, fill=c] (2,11.1935) rectangle (2.0398,11.2993);
\draw [color=c, fill=c] (2.0398,11.1935) rectangle (2.0796,11.2993);
\draw [color=c, fill=c] (2.0796,11.1935) rectangle (2.1194,11.2993);
\draw [color=c, fill=c] (2.1194,11.1935) rectangle (2.1592,11.2993);
\draw [color=c, fill=c] (2.1592,11.1935) rectangle (2.19901,11.2993);
\draw [color=c, fill=c] (2.19901,11.1935) rectangle (2.23881,11.2993);
\draw [color=c, fill=c] (2.23881,11.1935) rectangle (2.27861,11.2993);
\draw [color=c, fill=c] (2.27861,11.1935) rectangle (2.31841,11.2993);
\draw [color=c, fill=c] (2.31841,11.1935) rectangle (2.35821,11.2993);
\draw [color=c, fill=c] (2.35821,11.1935) rectangle (2.39801,11.2993);
\draw [color=c, fill=c] (2.39801,11.1935) rectangle (2.43781,11.2993);
\draw [color=c, fill=c] (2.43781,11.1935) rectangle (2.47761,11.2993);
\draw [color=c, fill=c] (2.47761,11.1935) rectangle (2.51741,11.2993);
\draw [color=c, fill=c] (2.51741,11.1935) rectangle (2.55721,11.2993);
\draw [color=c, fill=c] (2.55721,11.1935) rectangle (2.59702,11.2993);
\draw [color=c, fill=c] (2.59702,11.1935) rectangle (2.63682,11.2993);
\draw [color=c, fill=c] (2.63682,11.1935) rectangle (2.67662,11.2993);
\draw [color=c, fill=c] (2.67662,11.1935) rectangle (2.71642,11.2993);
\draw [color=c, fill=c] (2.71642,11.1935) rectangle (2.75622,11.2993);
\draw [color=c, fill=c] (2.75622,11.1935) rectangle (2.79602,11.2993);
\draw [color=c, fill=c] (2.79602,11.1935) rectangle (2.83582,11.2993);
\draw [color=c, fill=c] (2.83582,11.1935) rectangle (2.87562,11.2993);
\draw [color=c, fill=c] (2.87562,11.1935) rectangle (2.91542,11.2993);
\draw [color=c, fill=c] (2.91542,11.1935) rectangle (2.95522,11.2993);
\draw [color=c, fill=c] (2.95522,11.1935) rectangle (2.99502,11.2993);
\draw [color=c, fill=c] (2.99502,11.1935) rectangle (3.03483,11.2993);
\draw [color=c, fill=c] (3.03483,11.1935) rectangle (3.07463,11.2993);
\draw [color=c, fill=c] (3.07463,11.1935) rectangle (3.11443,11.2993);
\draw [color=c, fill=c] (3.11443,11.1935) rectangle (3.15423,11.2993);
\draw [color=c, fill=c] (3.15423,11.1935) rectangle (3.19403,11.2993);
\draw [color=c, fill=c] (3.19403,11.1935) rectangle (3.23383,11.2993);
\draw [color=c, fill=c] (3.23383,11.1935) rectangle (3.27363,11.2993);
\draw [color=c, fill=c] (3.27363,11.1935) rectangle (3.31343,11.2993);
\draw [color=c, fill=c] (3.31343,11.1935) rectangle (3.35323,11.2993);
\draw [color=c, fill=c] (3.35323,11.1935) rectangle (3.39303,11.2993);
\draw [color=c, fill=c] (3.39303,11.1935) rectangle (3.43284,11.2993);
\draw [color=c, fill=c] (3.43284,11.1935) rectangle (3.47264,11.2993);
\draw [color=c, fill=c] (3.47264,11.1935) rectangle (3.51244,11.2993);
\draw [color=c, fill=c] (3.51244,11.1935) rectangle (3.55224,11.2993);
\draw [color=c, fill=c] (3.55224,11.1935) rectangle (3.59204,11.2993);
\draw [color=c, fill=c] (3.59204,11.1935) rectangle (3.63184,11.2993);
\draw [color=c, fill=c] (3.63184,11.1935) rectangle (3.67164,11.2993);
\draw [color=c, fill=c] (3.67164,11.1935) rectangle (3.71144,11.2993);
\draw [color=c, fill=c] (3.71144,11.1935) rectangle (3.75124,11.2993);
\draw [color=c, fill=c] (3.75124,11.1935) rectangle (3.79104,11.2993);
\draw [color=c, fill=c] (3.79104,11.1935) rectangle (3.83085,11.2993);
\draw [color=c, fill=c] (3.83085,11.1935) rectangle (3.87065,11.2993);
\draw [color=c, fill=c] (3.87065,11.1935) rectangle (3.91045,11.2993);
\draw [color=c, fill=c] (3.91045,11.1935) rectangle (3.95025,11.2993);
\draw [color=c, fill=c] (3.95025,11.1935) rectangle (3.99005,11.2993);
\draw [color=c, fill=c] (3.99005,11.1935) rectangle (4.02985,11.2993);
\draw [color=c, fill=c] (4.02985,11.1935) rectangle (4.06965,11.2993);
\draw [color=c, fill=c] (4.06965,11.1935) rectangle (4.10945,11.2993);
\draw [color=c, fill=c] (4.10945,11.1935) rectangle (4.14925,11.2993);
\draw [color=c, fill=c] (4.14925,11.1935) rectangle (4.18905,11.2993);
\draw [color=c, fill=c] (4.18905,11.1935) rectangle (4.22886,11.2993);
\draw [color=c, fill=c] (4.22886,11.1935) rectangle (4.26866,11.2993);
\draw [color=c, fill=c] (4.26866,11.1935) rectangle (4.30846,11.2993);
\draw [color=c, fill=c] (4.30846,11.1935) rectangle (4.34826,11.2993);
\draw [color=c, fill=c] (4.34826,11.1935) rectangle (4.38806,11.2993);
\draw [color=c, fill=c] (4.38806,11.1935) rectangle (4.42786,11.2993);
\draw [color=c, fill=c] (4.42786,11.1935) rectangle (4.46766,11.2993);
\draw [color=c, fill=c] (4.46766,11.1935) rectangle (4.50746,11.2993);
\draw [color=c, fill=c] (4.50746,11.1935) rectangle (4.54726,11.2993);
\draw [color=c, fill=c] (4.54726,11.1935) rectangle (4.58706,11.2993);
\draw [color=c, fill=c] (4.58706,11.1935) rectangle (4.62687,11.2993);
\draw [color=c, fill=c] (4.62687,11.1935) rectangle (4.66667,11.2993);
\draw [color=c, fill=c] (4.66667,11.1935) rectangle (4.70647,11.2993);
\draw [color=c, fill=c] (4.70647,11.1935) rectangle (4.74627,11.2993);
\draw [color=c, fill=c] (4.74627,11.1935) rectangle (4.78607,11.2993);
\draw [color=c, fill=c] (4.78607,11.1935) rectangle (4.82587,11.2993);
\draw [color=c, fill=c] (4.82587,11.1935) rectangle (4.86567,11.2993);
\draw [color=c, fill=c] (4.86567,11.1935) rectangle (4.90547,11.2993);
\draw [color=c, fill=c] (4.90547,11.1935) rectangle (4.94527,11.2993);
\draw [color=c, fill=c] (4.94527,11.1935) rectangle (4.98507,11.2993);
\draw [color=c, fill=c] (4.98507,11.1935) rectangle (5.02488,11.2993);
\draw [color=c, fill=c] (5.02488,11.1935) rectangle (5.06468,11.2993);
\draw [color=c, fill=c] (5.06468,11.1935) rectangle (5.10448,11.2993);
\draw [color=c, fill=c] (5.10448,11.1935) rectangle (5.14428,11.2993);
\draw [color=c, fill=c] (5.14428,11.1935) rectangle (5.18408,11.2993);
\draw [color=c, fill=c] (5.18408,11.1935) rectangle (5.22388,11.2993);
\draw [color=c, fill=c] (5.22388,11.1935) rectangle (5.26368,11.2993);
\draw [color=c, fill=c] (5.26368,11.1935) rectangle (5.30348,11.2993);
\draw [color=c, fill=c] (5.30348,11.1935) rectangle (5.34328,11.2993);
\draw [color=c, fill=c] (5.34328,11.1935) rectangle (5.38308,11.2993);
\draw [color=c, fill=c] (5.38308,11.1935) rectangle (5.42289,11.2993);
\draw [color=c, fill=c] (5.42289,11.1935) rectangle (5.46269,11.2993);
\draw [color=c, fill=c] (5.46269,11.1935) rectangle (5.50249,11.2993);
\draw [color=c, fill=c] (5.50249,11.1935) rectangle (5.54229,11.2993);
\draw [color=c, fill=c] (5.54229,11.1935) rectangle (5.58209,11.2993);
\draw [color=c, fill=c] (5.58209,11.1935) rectangle (5.62189,11.2993);
\draw [color=c, fill=c] (5.62189,11.1935) rectangle (5.66169,11.2993);
\draw [color=c, fill=c] (5.66169,11.1935) rectangle (5.70149,11.2993);
\draw [color=c, fill=c] (5.70149,11.1935) rectangle (5.74129,11.2993);
\draw [color=c, fill=c] (5.74129,11.1935) rectangle (5.78109,11.2993);
\draw [color=c, fill=c] (5.78109,11.1935) rectangle (5.8209,11.2993);
\draw [color=c, fill=c] (5.8209,11.1935) rectangle (5.8607,11.2993);
\draw [color=c, fill=c] (5.8607,11.1935) rectangle (5.9005,11.2993);
\draw [color=c, fill=c] (5.9005,11.1935) rectangle (5.9403,11.2993);
\draw [color=c, fill=c] (5.9403,11.1935) rectangle (5.9801,11.2993);
\draw [color=c, fill=c] (5.9801,11.1935) rectangle (6.0199,11.2993);
\draw [color=c, fill=c] (6.0199,11.1935) rectangle (6.0597,11.2993);
\draw [color=c, fill=c] (6.0597,11.1935) rectangle (6.0995,11.2993);
\draw [color=c, fill=c] (6.0995,11.1935) rectangle (6.1393,11.2993);
\draw [color=c, fill=c] (6.1393,11.1935) rectangle (6.1791,11.2993);
\draw [color=c, fill=c] (6.1791,11.1935) rectangle (6.21891,11.2993);
\draw [color=c, fill=c] (6.21891,11.1935) rectangle (6.25871,11.2993);
\draw [color=c, fill=c] (6.25871,11.1935) rectangle (6.29851,11.2993);
\draw [color=c, fill=c] (6.29851,11.1935) rectangle (6.33831,11.2993);
\draw [color=c, fill=c] (6.33831,11.1935) rectangle (6.37811,11.2993);
\draw [color=c, fill=c] (6.37811,11.1935) rectangle (6.41791,11.2993);
\draw [color=c, fill=c] (6.41791,11.1935) rectangle (6.45771,11.2993);
\draw [color=c, fill=c] (6.45771,11.1935) rectangle (6.49751,11.2993);
\draw [color=c, fill=c] (6.49751,11.1935) rectangle (6.53731,11.2993);
\draw [color=c, fill=c] (6.53731,11.1935) rectangle (6.57711,11.2993);
\draw [color=c, fill=c] (6.57711,11.1935) rectangle (6.61692,11.2993);
\draw [color=c, fill=c] (6.61692,11.1935) rectangle (6.65672,11.2993);
\draw [color=c, fill=c] (6.65672,11.1935) rectangle (6.69652,11.2993);
\draw [color=c, fill=c] (6.69652,11.1935) rectangle (6.73632,11.2993);
\draw [color=c, fill=c] (6.73632,11.1935) rectangle (6.77612,11.2993);
\draw [color=c, fill=c] (6.77612,11.1935) rectangle (6.81592,11.2993);
\draw [color=c, fill=c] (6.81592,11.1935) rectangle (6.85572,11.2993);
\draw [color=c, fill=c] (6.85572,11.1935) rectangle (6.89552,11.2993);
\draw [color=c, fill=c] (6.89552,11.1935) rectangle (6.93532,11.2993);
\draw [color=c, fill=c] (6.93532,11.1935) rectangle (6.97512,11.2993);
\draw [color=c, fill=c] (6.97512,11.1935) rectangle (7.01493,11.2993);
\draw [color=c, fill=c] (7.01493,11.1935) rectangle (7.05473,11.2993);
\draw [color=c, fill=c] (7.05473,11.1935) rectangle (7.09453,11.2993);
\draw [color=c, fill=c] (7.09453,11.1935) rectangle (7.13433,11.2993);
\draw [color=c, fill=c] (7.13433,11.1935) rectangle (7.17413,11.2993);
\draw [color=c, fill=c] (7.17413,11.1935) rectangle (7.21393,11.2993);
\draw [color=c, fill=c] (7.21393,11.1935) rectangle (7.25373,11.2993);
\draw [color=c, fill=c] (7.25373,11.1935) rectangle (7.29353,11.2993);
\draw [color=c, fill=c] (7.29353,11.1935) rectangle (7.33333,11.2993);
\draw [color=c, fill=c] (7.33333,11.1935) rectangle (7.37313,11.2993);
\draw [color=c, fill=c] (7.37313,11.1935) rectangle (7.41294,11.2993);
\draw [color=c, fill=c] (7.41294,11.1935) rectangle (7.45274,11.2993);
\draw [color=c, fill=c] (7.45274,11.1935) rectangle (7.49254,11.2993);
\draw [color=c, fill=c] (7.49254,11.1935) rectangle (7.53234,11.2993);
\draw [color=c, fill=c] (7.53234,11.1935) rectangle (7.57214,11.2993);
\draw [color=c, fill=c] (7.57214,11.1935) rectangle (7.61194,11.2993);
\draw [color=c, fill=c] (7.61194,11.1935) rectangle (7.65174,11.2993);
\draw [color=c, fill=c] (7.65174,11.1935) rectangle (7.69154,11.2993);
\draw [color=c, fill=c] (7.69154,11.1935) rectangle (7.73134,11.2993);
\draw [color=c, fill=c] (7.73134,11.1935) rectangle (7.77114,11.2993);
\draw [color=c, fill=c] (7.77114,11.1935) rectangle (7.81095,11.2993);
\draw [color=c, fill=c] (7.81095,11.1935) rectangle (7.85075,11.2993);
\draw [color=c, fill=c] (7.85075,11.1935) rectangle (7.89055,11.2993);
\draw [color=c, fill=c] (7.89055,11.1935) rectangle (7.93035,11.2993);
\draw [color=c, fill=c] (7.93035,11.1935) rectangle (7.97015,11.2993);
\definecolor{c}{rgb}{0,0.0800001,1};
\draw [color=c, fill=c] (7.97015,11.1935) rectangle (8.00995,11.2993);
\draw [color=c, fill=c] (8.00995,11.1935) rectangle (8.04975,11.2993);
\draw [color=c, fill=c] (8.04975,11.1935) rectangle (8.08955,11.2993);
\draw [color=c, fill=c] (8.08955,11.1935) rectangle (8.12935,11.2993);
\draw [color=c, fill=c] (8.12935,11.1935) rectangle (8.16915,11.2993);
\draw [color=c, fill=c] (8.16915,11.1935) rectangle (8.20895,11.2993);
\draw [color=c, fill=c] (8.20895,11.1935) rectangle (8.24876,11.2993);
\draw [color=c, fill=c] (8.24876,11.1935) rectangle (8.28856,11.2993);
\draw [color=c, fill=c] (8.28856,11.1935) rectangle (8.32836,11.2993);
\draw [color=c, fill=c] (8.32836,11.1935) rectangle (8.36816,11.2993);
\draw [color=c, fill=c] (8.36816,11.1935) rectangle (8.40796,11.2993);
\draw [color=c, fill=c] (8.40796,11.1935) rectangle (8.44776,11.2993);
\draw [color=c, fill=c] (8.44776,11.1935) rectangle (8.48756,11.2993);
\draw [color=c, fill=c] (8.48756,11.1935) rectangle (8.52736,11.2993);
\draw [color=c, fill=c] (8.52736,11.1935) rectangle (8.56716,11.2993);
\draw [color=c, fill=c] (8.56716,11.1935) rectangle (8.60697,11.2993);
\draw [color=c, fill=c] (8.60697,11.1935) rectangle (8.64677,11.2993);
\draw [color=c, fill=c] (8.64677,11.1935) rectangle (8.68657,11.2993);
\draw [color=c, fill=c] (8.68657,11.1935) rectangle (8.72637,11.2993);
\draw [color=c, fill=c] (8.72637,11.1935) rectangle (8.76617,11.2993);
\draw [color=c, fill=c] (8.76617,11.1935) rectangle (8.80597,11.2993);
\draw [color=c, fill=c] (8.80597,11.1935) rectangle (8.84577,11.2993);
\draw [color=c, fill=c] (8.84577,11.1935) rectangle (8.88557,11.2993);
\draw [color=c, fill=c] (8.88557,11.1935) rectangle (8.92537,11.2993);
\draw [color=c, fill=c] (8.92537,11.1935) rectangle (8.96517,11.2993);
\draw [color=c, fill=c] (8.96517,11.1935) rectangle (9.00498,11.2993);
\draw [color=c, fill=c] (9.00498,11.1935) rectangle (9.04478,11.2993);
\draw [color=c, fill=c] (9.04478,11.1935) rectangle (9.08458,11.2993);
\draw [color=c, fill=c] (9.08458,11.1935) rectangle (9.12438,11.2993);
\draw [color=c, fill=c] (9.12438,11.1935) rectangle (9.16418,11.2993);
\draw [color=c, fill=c] (9.16418,11.1935) rectangle (9.20398,11.2993);
\draw [color=c, fill=c] (9.20398,11.1935) rectangle (9.24378,11.2993);
\draw [color=c, fill=c] (9.24378,11.1935) rectangle (9.28358,11.2993);
\draw [color=c, fill=c] (9.28358,11.1935) rectangle (9.32338,11.2993);
\draw [color=c, fill=c] (9.32338,11.1935) rectangle (9.36318,11.2993);
\draw [color=c, fill=c] (9.36318,11.1935) rectangle (9.40298,11.2993);
\draw [color=c, fill=c] (9.40298,11.1935) rectangle (9.44279,11.2993);
\draw [color=c, fill=c] (9.44279,11.1935) rectangle (9.48259,11.2993);
\draw [color=c, fill=c] (9.48259,11.1935) rectangle (9.52239,11.2993);
\draw [color=c, fill=c] (9.52239,11.1935) rectangle (9.56219,11.2993);
\draw [color=c, fill=c] (9.56219,11.1935) rectangle (9.60199,11.2993);
\draw [color=c, fill=c] (9.60199,11.1935) rectangle (9.64179,11.2993);
\definecolor{c}{rgb}{0,0.266667,1};
\draw [color=c, fill=c] (9.64179,11.1935) rectangle (9.68159,11.2993);
\draw [color=c, fill=c] (9.68159,11.1935) rectangle (9.72139,11.2993);
\draw [color=c, fill=c] (9.72139,11.1935) rectangle (9.76119,11.2993);
\draw [color=c, fill=c] (9.76119,11.1935) rectangle (9.80099,11.2993);
\draw [color=c, fill=c] (9.80099,11.1935) rectangle (9.8408,11.2993);
\draw [color=c, fill=c] (9.8408,11.1935) rectangle (9.8806,11.2993);
\draw [color=c, fill=c] (9.8806,11.1935) rectangle (9.9204,11.2993);
\draw [color=c, fill=c] (9.9204,11.1935) rectangle (9.9602,11.2993);
\draw [color=c, fill=c] (9.9602,11.1935) rectangle (10,11.2993);
\draw [color=c, fill=c] (10,11.1935) rectangle (10.0398,11.2993);
\draw [color=c, fill=c] (10.0398,11.1935) rectangle (10.0796,11.2993);
\draw [color=c, fill=c] (10.0796,11.1935) rectangle (10.1194,11.2993);
\draw [color=c, fill=c] (10.1194,11.1935) rectangle (10.1592,11.2993);
\draw [color=c, fill=c] (10.1592,11.1935) rectangle (10.199,11.2993);
\draw [color=c, fill=c] (10.199,11.1935) rectangle (10.2388,11.2993);
\draw [color=c, fill=c] (10.2388,11.1935) rectangle (10.2786,11.2993);
\draw [color=c, fill=c] (10.2786,11.1935) rectangle (10.3184,11.2993);
\draw [color=c, fill=c] (10.3184,11.1935) rectangle (10.3582,11.2993);
\draw [color=c, fill=c] (10.3582,11.1935) rectangle (10.398,11.2993);
\draw [color=c, fill=c] (10.398,11.1935) rectangle (10.4378,11.2993);
\draw [color=c, fill=c] (10.4378,11.1935) rectangle (10.4776,11.2993);
\draw [color=c, fill=c] (10.4776,11.1935) rectangle (10.5174,11.2993);
\draw [color=c, fill=c] (10.5174,11.1935) rectangle (10.5572,11.2993);
\draw [color=c, fill=c] (10.5572,11.1935) rectangle (10.597,11.2993);
\draw [color=c, fill=c] (10.597,11.1935) rectangle (10.6368,11.2993);
\draw [color=c, fill=c] (10.6368,11.1935) rectangle (10.6766,11.2993);
\draw [color=c, fill=c] (10.6766,11.1935) rectangle (10.7164,11.2993);
\draw [color=c, fill=c] (10.7164,11.1935) rectangle (10.7562,11.2993);
\draw [color=c, fill=c] (10.7562,11.1935) rectangle (10.796,11.2993);
\draw [color=c, fill=c] (10.796,11.1935) rectangle (10.8358,11.2993);
\draw [color=c, fill=c] (10.8358,11.1935) rectangle (10.8756,11.2993);
\draw [color=c, fill=c] (10.8756,11.1935) rectangle (10.9154,11.2993);
\draw [color=c, fill=c] (10.9154,11.1935) rectangle (10.9552,11.2993);
\definecolor{c}{rgb}{0,0.546666,1};
\draw [color=c, fill=c] (10.9552,11.1935) rectangle (10.995,11.2993);
\draw [color=c, fill=c] (10.995,11.1935) rectangle (11.0348,11.2993);
\draw [color=c, fill=c] (11.0348,11.1935) rectangle (11.0746,11.2993);
\draw [color=c, fill=c] (11.0746,11.1935) rectangle (11.1144,11.2993);
\draw [color=c, fill=c] (11.1144,11.1935) rectangle (11.1542,11.2993);
\draw [color=c, fill=c] (11.1542,11.1935) rectangle (11.194,11.2993);
\draw [color=c, fill=c] (11.194,11.1935) rectangle (11.2338,11.2993);
\draw [color=c, fill=c] (11.2338,11.1935) rectangle (11.2736,11.2993);
\draw [color=c, fill=c] (11.2736,11.1935) rectangle (11.3134,11.2993);
\draw [color=c, fill=c] (11.3134,11.1935) rectangle (11.3532,11.2993);
\draw [color=c, fill=c] (11.3532,11.1935) rectangle (11.393,11.2993);
\draw [color=c, fill=c] (11.393,11.1935) rectangle (11.4328,11.2993);
\draw [color=c, fill=c] (11.4328,11.1935) rectangle (11.4726,11.2993);
\draw [color=c, fill=c] (11.4726,11.1935) rectangle (11.5124,11.2993);
\draw [color=c, fill=c] (11.5124,11.1935) rectangle (11.5522,11.2993);
\draw [color=c, fill=c] (11.5522,11.1935) rectangle (11.592,11.2993);
\draw [color=c, fill=c] (11.592,11.1935) rectangle (11.6318,11.2993);
\draw [color=c, fill=c] (11.6318,11.1935) rectangle (11.6716,11.2993);
\draw [color=c, fill=c] (11.6716,11.1935) rectangle (11.7114,11.2993);
\draw [color=c, fill=c] (11.7114,11.1935) rectangle (11.7512,11.2993);
\draw [color=c, fill=c] (11.7512,11.1935) rectangle (11.791,11.2993);
\draw [color=c, fill=c] (11.791,11.1935) rectangle (11.8308,11.2993);
\draw [color=c, fill=c] (11.8308,11.1935) rectangle (11.8706,11.2993);
\draw [color=c, fill=c] (11.8706,11.1935) rectangle (11.9104,11.2993);
\draw [color=c, fill=c] (11.9104,11.1935) rectangle (11.9502,11.2993);
\draw [color=c, fill=c] (11.9502,11.1935) rectangle (11.99,11.2993);
\draw [color=c, fill=c] (11.99,11.1935) rectangle (12.0299,11.2993);
\draw [color=c, fill=c] (12.0299,11.1935) rectangle (12.0697,11.2993);
\draw [color=c, fill=c] (12.0697,11.1935) rectangle (12.1095,11.2993);
\draw [color=c, fill=c] (12.1095,11.1935) rectangle (12.1493,11.2993);
\draw [color=c, fill=c] (12.1493,11.1935) rectangle (12.1891,11.2993);
\draw [color=c, fill=c] (12.1891,11.1935) rectangle (12.2289,11.2993);
\draw [color=c, fill=c] (12.2289,11.1935) rectangle (12.2687,11.2993);
\draw [color=c, fill=c] (12.2687,11.1935) rectangle (12.3085,11.2993);
\draw [color=c, fill=c] (12.3085,11.1935) rectangle (12.3483,11.2993);
\draw [color=c, fill=c] (12.3483,11.1935) rectangle (12.3881,11.2993);
\draw [color=c, fill=c] (12.3881,11.1935) rectangle (12.4279,11.2993);
\draw [color=c, fill=c] (12.4279,11.1935) rectangle (12.4677,11.2993);
\draw [color=c, fill=c] (12.4677,11.1935) rectangle (12.5075,11.2993);
\draw [color=c, fill=c] (12.5075,11.1935) rectangle (12.5473,11.2993);
\draw [color=c, fill=c] (12.5473,11.1935) rectangle (12.5871,11.2993);
\draw [color=c, fill=c] (12.5871,11.1935) rectangle (12.6269,11.2993);
\draw [color=c, fill=c] (12.6269,11.1935) rectangle (12.6667,11.2993);
\draw [color=c, fill=c] (12.6667,11.1935) rectangle (12.7065,11.2993);
\draw [color=c, fill=c] (12.7065,11.1935) rectangle (12.7463,11.2993);
\draw [color=c, fill=c] (12.7463,11.1935) rectangle (12.7861,11.2993);
\draw [color=c, fill=c] (12.7861,11.1935) rectangle (12.8259,11.2993);
\draw [color=c, fill=c] (12.8259,11.1935) rectangle (12.8657,11.2993);
\draw [color=c, fill=c] (12.8657,11.1935) rectangle (12.9055,11.2993);
\draw [color=c, fill=c] (12.9055,11.1935) rectangle (12.9453,11.2993);
\draw [color=c, fill=c] (12.9453,11.1935) rectangle (12.9851,11.2993);
\draw [color=c, fill=c] (12.9851,11.1935) rectangle (13.0249,11.2993);
\draw [color=c, fill=c] (13.0249,11.1935) rectangle (13.0647,11.2993);
\draw [color=c, fill=c] (13.0647,11.1935) rectangle (13.1045,11.2993);
\draw [color=c, fill=c] (13.1045,11.1935) rectangle (13.1443,11.2993);
\draw [color=c, fill=c] (13.1443,11.1935) rectangle (13.1841,11.2993);
\draw [color=c, fill=c] (13.1841,11.1935) rectangle (13.2239,11.2993);
\draw [color=c, fill=c] (13.2239,11.1935) rectangle (13.2637,11.2993);
\draw [color=c, fill=c] (13.2637,11.1935) rectangle (13.3035,11.2993);
\draw [color=c, fill=c] (13.3035,11.1935) rectangle (13.3433,11.2993);
\draw [color=c, fill=c] (13.3433,11.1935) rectangle (13.3831,11.2993);
\draw [color=c, fill=c] (13.3831,11.1935) rectangle (13.4229,11.2993);
\draw [color=c, fill=c] (13.4229,11.1935) rectangle (13.4627,11.2993);
\draw [color=c, fill=c] (13.4627,11.1935) rectangle (13.5025,11.2993);
\draw [color=c, fill=c] (13.5025,11.1935) rectangle (13.5423,11.2993);
\draw [color=c, fill=c] (13.5423,11.1935) rectangle (13.5821,11.2993);
\draw [color=c, fill=c] (13.5821,11.1935) rectangle (13.6219,11.2993);
\draw [color=c, fill=c] (13.6219,11.1935) rectangle (13.6617,11.2993);
\draw [color=c, fill=c] (13.6617,11.1935) rectangle (13.7015,11.2993);
\draw [color=c, fill=c] (13.7015,11.1935) rectangle (13.7413,11.2993);
\draw [color=c, fill=c] (13.7413,11.1935) rectangle (13.7811,11.2993);
\draw [color=c, fill=c] (13.7811,11.1935) rectangle (13.8209,11.2993);
\draw [color=c, fill=c] (13.8209,11.1935) rectangle (13.8607,11.2993);
\draw [color=c, fill=c] (13.8607,11.1935) rectangle (13.9005,11.2993);
\draw [color=c, fill=c] (13.9005,11.1935) rectangle (13.9403,11.2993);
\draw [color=c, fill=c] (13.9403,11.1935) rectangle (13.9801,11.2993);
\draw [color=c, fill=c] (13.9801,11.1935) rectangle (14.0199,11.2993);
\draw [color=c, fill=c] (14.0199,11.1935) rectangle (14.0597,11.2993);
\draw [color=c, fill=c] (14.0597,11.1935) rectangle (14.0995,11.2993);
\draw [color=c, fill=c] (14.0995,11.1935) rectangle (14.1393,11.2993);
\draw [color=c, fill=c] (14.1393,11.1935) rectangle (14.1791,11.2993);
\draw [color=c, fill=c] (14.1791,11.1935) rectangle (14.2189,11.2993);
\draw [color=c, fill=c] (14.2189,11.1935) rectangle (14.2587,11.2993);
\draw [color=c, fill=c] (14.2587,11.1935) rectangle (14.2985,11.2993);
\draw [color=c, fill=c] (14.2985,11.1935) rectangle (14.3383,11.2993);
\draw [color=c, fill=c] (14.3383,11.1935) rectangle (14.3781,11.2993);
\definecolor{c}{rgb}{0,0.733333,1};
\draw [color=c, fill=c] (14.3781,11.1935) rectangle (14.4179,11.2993);
\draw [color=c, fill=c] (14.4179,11.1935) rectangle (14.4577,11.2993);
\draw [color=c, fill=c] (14.4577,11.1935) rectangle (14.4975,11.2993);
\draw [color=c, fill=c] (14.4975,11.1935) rectangle (14.5373,11.2993);
\draw [color=c, fill=c] (14.5373,11.1935) rectangle (14.5771,11.2993);
\draw [color=c, fill=c] (14.5771,11.1935) rectangle (14.6169,11.2993);
\draw [color=c, fill=c] (14.6169,11.1935) rectangle (14.6567,11.2993);
\draw [color=c, fill=c] (14.6567,11.1935) rectangle (14.6965,11.2993);
\draw [color=c, fill=c] (14.6965,11.1935) rectangle (14.7363,11.2993);
\draw [color=c, fill=c] (14.7363,11.1935) rectangle (14.7761,11.2993);
\draw [color=c, fill=c] (14.7761,11.1935) rectangle (14.8159,11.2993);
\draw [color=c, fill=c] (14.8159,11.1935) rectangle (14.8557,11.2993);
\draw [color=c, fill=c] (14.8557,11.1935) rectangle (14.8955,11.2993);
\draw [color=c, fill=c] (14.8955,11.1935) rectangle (14.9353,11.2993);
\draw [color=c, fill=c] (14.9353,11.1935) rectangle (14.9751,11.2993);
\draw [color=c, fill=c] (14.9751,11.1935) rectangle (15.0149,11.2993);
\draw [color=c, fill=c] (15.0149,11.1935) rectangle (15.0547,11.2993);
\draw [color=c, fill=c] (15.0547,11.1935) rectangle (15.0945,11.2993);
\draw [color=c, fill=c] (15.0945,11.1935) rectangle (15.1343,11.2993);
\draw [color=c, fill=c] (15.1343,11.1935) rectangle (15.1741,11.2993);
\draw [color=c, fill=c] (15.1741,11.1935) rectangle (15.2139,11.2993);
\draw [color=c, fill=c] (15.2139,11.1935) rectangle (15.2537,11.2993);
\draw [color=c, fill=c] (15.2537,11.1935) rectangle (15.2935,11.2993);
\draw [color=c, fill=c] (15.2935,11.1935) rectangle (15.3333,11.2993);
\draw [color=c, fill=c] (15.3333,11.1935) rectangle (15.3731,11.2993);
\draw [color=c, fill=c] (15.3731,11.1935) rectangle (15.4129,11.2993);
\draw [color=c, fill=c] (15.4129,11.1935) rectangle (15.4527,11.2993);
\draw [color=c, fill=c] (15.4527,11.1935) rectangle (15.4925,11.2993);
\draw [color=c, fill=c] (15.4925,11.1935) rectangle (15.5323,11.2993);
\draw [color=c, fill=c] (15.5323,11.1935) rectangle (15.5721,11.2993);
\draw [color=c, fill=c] (15.5721,11.1935) rectangle (15.6119,11.2993);
\draw [color=c, fill=c] (15.6119,11.1935) rectangle (15.6517,11.2993);
\draw [color=c, fill=c] (15.6517,11.1935) rectangle (15.6915,11.2993);
\draw [color=c, fill=c] (15.6915,11.1935) rectangle (15.7313,11.2993);
\draw [color=c, fill=c] (15.7313,11.1935) rectangle (15.7711,11.2993);
\draw [color=c, fill=c] (15.7711,11.1935) rectangle (15.8109,11.2993);
\draw [color=c, fill=c] (15.8109,11.1935) rectangle (15.8507,11.2993);
\draw [color=c, fill=c] (15.8507,11.1935) rectangle (15.8905,11.2993);
\draw [color=c, fill=c] (15.8905,11.1935) rectangle (15.9303,11.2993);
\draw [color=c, fill=c] (15.9303,11.1935) rectangle (15.9701,11.2993);
\draw [color=c, fill=c] (15.9701,11.1935) rectangle (16.01,11.2993);
\draw [color=c, fill=c] (16.01,11.1935) rectangle (16.0498,11.2993);
\draw [color=c, fill=c] (16.0498,11.1935) rectangle (16.0896,11.2993);
\draw [color=c, fill=c] (16.0896,11.1935) rectangle (16.1294,11.2993);
\draw [color=c, fill=c] (16.1294,11.1935) rectangle (16.1692,11.2993);
\draw [color=c, fill=c] (16.1692,11.1935) rectangle (16.209,11.2993);
\draw [color=c, fill=c] (16.209,11.1935) rectangle (16.2488,11.2993);
\draw [color=c, fill=c] (16.2488,11.1935) rectangle (16.2886,11.2993);
\draw [color=c, fill=c] (16.2886,11.1935) rectangle (16.3284,11.2993);
\draw [color=c, fill=c] (16.3284,11.1935) rectangle (16.3682,11.2993);
\draw [color=c, fill=c] (16.3682,11.1935) rectangle (16.408,11.2993);
\draw [color=c, fill=c] (16.408,11.1935) rectangle (16.4478,11.2993);
\draw [color=c, fill=c] (16.4478,11.1935) rectangle (16.4876,11.2993);
\draw [color=c, fill=c] (16.4876,11.1935) rectangle (16.5274,11.2993);
\draw [color=c, fill=c] (16.5274,11.1935) rectangle (16.5672,11.2993);
\draw [color=c, fill=c] (16.5672,11.1935) rectangle (16.607,11.2993);
\draw [color=c, fill=c] (16.607,11.1935) rectangle (16.6468,11.2993);
\draw [color=c, fill=c] (16.6468,11.1935) rectangle (16.6866,11.2993);
\draw [color=c, fill=c] (16.6866,11.1935) rectangle (16.7264,11.2993);
\draw [color=c, fill=c] (16.7264,11.1935) rectangle (16.7662,11.2993);
\draw [color=c, fill=c] (16.7662,11.1935) rectangle (16.806,11.2993);
\draw [color=c, fill=c] (16.806,11.1935) rectangle (16.8458,11.2993);
\draw [color=c, fill=c] (16.8458,11.1935) rectangle (16.8856,11.2993);
\draw [color=c, fill=c] (16.8856,11.1935) rectangle (16.9254,11.2993);
\draw [color=c, fill=c] (16.9254,11.1935) rectangle (16.9652,11.2993);
\draw [color=c, fill=c] (16.9652,11.1935) rectangle (17.005,11.2993);
\draw [color=c, fill=c] (17.005,11.1935) rectangle (17.0448,11.2993);
\draw [color=c, fill=c] (17.0448,11.1935) rectangle (17.0846,11.2993);
\draw [color=c, fill=c] (17.0846,11.1935) rectangle (17.1244,11.2993);
\draw [color=c, fill=c] (17.1244,11.1935) rectangle (17.1642,11.2993);
\draw [color=c, fill=c] (17.1642,11.1935) rectangle (17.204,11.2993);
\draw [color=c, fill=c] (17.204,11.1935) rectangle (17.2438,11.2993);
\draw [color=c, fill=c] (17.2438,11.1935) rectangle (17.2836,11.2993);
\draw [color=c, fill=c] (17.2836,11.1935) rectangle (17.3234,11.2993);
\draw [color=c, fill=c] (17.3234,11.1935) rectangle (17.3632,11.2993);
\draw [color=c, fill=c] (17.3632,11.1935) rectangle (17.403,11.2993);
\draw [color=c, fill=c] (17.403,11.1935) rectangle (17.4428,11.2993);
\draw [color=c, fill=c] (17.4428,11.1935) rectangle (17.4826,11.2993);
\draw [color=c, fill=c] (17.4826,11.1935) rectangle (17.5224,11.2993);
\draw [color=c, fill=c] (17.5224,11.1935) rectangle (17.5622,11.2993);
\draw [color=c, fill=c] (17.5622,11.1935) rectangle (17.602,11.2993);
\draw [color=c, fill=c] (17.602,11.1935) rectangle (17.6418,11.2993);
\draw [color=c, fill=c] (17.6418,11.1935) rectangle (17.6816,11.2993);
\draw [color=c, fill=c] (17.6816,11.1935) rectangle (17.7214,11.2993);
\draw [color=c, fill=c] (17.7214,11.1935) rectangle (17.7612,11.2993);
\draw [color=c, fill=c] (17.7612,11.1935) rectangle (17.801,11.2993);
\draw [color=c, fill=c] (17.801,11.1935) rectangle (17.8408,11.2993);
\draw [color=c, fill=c] (17.8408,11.1935) rectangle (17.8806,11.2993);
\draw [color=c, fill=c] (17.8806,11.1935) rectangle (17.9204,11.2993);
\draw [color=c, fill=c] (17.9204,11.1935) rectangle (17.9602,11.2993);
\draw [color=c, fill=c] (17.9602,11.1935) rectangle (18,11.2993);
\definecolor{c}{rgb}{0.2,0,1};
\draw [color=c, fill=c] (2,11.2993) rectangle (2.0398,11.4052);
\draw [color=c, fill=c] (2.0398,11.2993) rectangle (2.0796,11.4052);
\draw [color=c, fill=c] (2.0796,11.2993) rectangle (2.1194,11.4052);
\draw [color=c, fill=c] (2.1194,11.2993) rectangle (2.1592,11.4052);
\draw [color=c, fill=c] (2.1592,11.2993) rectangle (2.19901,11.4052);
\draw [color=c, fill=c] (2.19901,11.2993) rectangle (2.23881,11.4052);
\draw [color=c, fill=c] (2.23881,11.2993) rectangle (2.27861,11.4052);
\draw [color=c, fill=c] (2.27861,11.2993) rectangle (2.31841,11.4052);
\draw [color=c, fill=c] (2.31841,11.2993) rectangle (2.35821,11.4052);
\draw [color=c, fill=c] (2.35821,11.2993) rectangle (2.39801,11.4052);
\draw [color=c, fill=c] (2.39801,11.2993) rectangle (2.43781,11.4052);
\draw [color=c, fill=c] (2.43781,11.2993) rectangle (2.47761,11.4052);
\draw [color=c, fill=c] (2.47761,11.2993) rectangle (2.51741,11.4052);
\draw [color=c, fill=c] (2.51741,11.2993) rectangle (2.55721,11.4052);
\draw [color=c, fill=c] (2.55721,11.2993) rectangle (2.59702,11.4052);
\draw [color=c, fill=c] (2.59702,11.2993) rectangle (2.63682,11.4052);
\draw [color=c, fill=c] (2.63682,11.2993) rectangle (2.67662,11.4052);
\draw [color=c, fill=c] (2.67662,11.2993) rectangle (2.71642,11.4052);
\draw [color=c, fill=c] (2.71642,11.2993) rectangle (2.75622,11.4052);
\draw [color=c, fill=c] (2.75622,11.2993) rectangle (2.79602,11.4052);
\draw [color=c, fill=c] (2.79602,11.2993) rectangle (2.83582,11.4052);
\draw [color=c, fill=c] (2.83582,11.2993) rectangle (2.87562,11.4052);
\draw [color=c, fill=c] (2.87562,11.2993) rectangle (2.91542,11.4052);
\draw [color=c, fill=c] (2.91542,11.2993) rectangle (2.95522,11.4052);
\draw [color=c, fill=c] (2.95522,11.2993) rectangle (2.99502,11.4052);
\draw [color=c, fill=c] (2.99502,11.2993) rectangle (3.03483,11.4052);
\draw [color=c, fill=c] (3.03483,11.2993) rectangle (3.07463,11.4052);
\draw [color=c, fill=c] (3.07463,11.2993) rectangle (3.11443,11.4052);
\draw [color=c, fill=c] (3.11443,11.2993) rectangle (3.15423,11.4052);
\draw [color=c, fill=c] (3.15423,11.2993) rectangle (3.19403,11.4052);
\draw [color=c, fill=c] (3.19403,11.2993) rectangle (3.23383,11.4052);
\draw [color=c, fill=c] (3.23383,11.2993) rectangle (3.27363,11.4052);
\draw [color=c, fill=c] (3.27363,11.2993) rectangle (3.31343,11.4052);
\draw [color=c, fill=c] (3.31343,11.2993) rectangle (3.35323,11.4052);
\draw [color=c, fill=c] (3.35323,11.2993) rectangle (3.39303,11.4052);
\draw [color=c, fill=c] (3.39303,11.2993) rectangle (3.43284,11.4052);
\draw [color=c, fill=c] (3.43284,11.2993) rectangle (3.47264,11.4052);
\draw [color=c, fill=c] (3.47264,11.2993) rectangle (3.51244,11.4052);
\draw [color=c, fill=c] (3.51244,11.2993) rectangle (3.55224,11.4052);
\draw [color=c, fill=c] (3.55224,11.2993) rectangle (3.59204,11.4052);
\draw [color=c, fill=c] (3.59204,11.2993) rectangle (3.63184,11.4052);
\draw [color=c, fill=c] (3.63184,11.2993) rectangle (3.67164,11.4052);
\draw [color=c, fill=c] (3.67164,11.2993) rectangle (3.71144,11.4052);
\draw [color=c, fill=c] (3.71144,11.2993) rectangle (3.75124,11.4052);
\draw [color=c, fill=c] (3.75124,11.2993) rectangle (3.79104,11.4052);
\draw [color=c, fill=c] (3.79104,11.2993) rectangle (3.83085,11.4052);
\draw [color=c, fill=c] (3.83085,11.2993) rectangle (3.87065,11.4052);
\draw [color=c, fill=c] (3.87065,11.2993) rectangle (3.91045,11.4052);
\draw [color=c, fill=c] (3.91045,11.2993) rectangle (3.95025,11.4052);
\draw [color=c, fill=c] (3.95025,11.2993) rectangle (3.99005,11.4052);
\draw [color=c, fill=c] (3.99005,11.2993) rectangle (4.02985,11.4052);
\draw [color=c, fill=c] (4.02985,11.2993) rectangle (4.06965,11.4052);
\draw [color=c, fill=c] (4.06965,11.2993) rectangle (4.10945,11.4052);
\draw [color=c, fill=c] (4.10945,11.2993) rectangle (4.14925,11.4052);
\draw [color=c, fill=c] (4.14925,11.2993) rectangle (4.18905,11.4052);
\draw [color=c, fill=c] (4.18905,11.2993) rectangle (4.22886,11.4052);
\draw [color=c, fill=c] (4.22886,11.2993) rectangle (4.26866,11.4052);
\draw [color=c, fill=c] (4.26866,11.2993) rectangle (4.30846,11.4052);
\draw [color=c, fill=c] (4.30846,11.2993) rectangle (4.34826,11.4052);
\draw [color=c, fill=c] (4.34826,11.2993) rectangle (4.38806,11.4052);
\draw [color=c, fill=c] (4.38806,11.2993) rectangle (4.42786,11.4052);
\draw [color=c, fill=c] (4.42786,11.2993) rectangle (4.46766,11.4052);
\draw [color=c, fill=c] (4.46766,11.2993) rectangle (4.50746,11.4052);
\draw [color=c, fill=c] (4.50746,11.2993) rectangle (4.54726,11.4052);
\draw [color=c, fill=c] (4.54726,11.2993) rectangle (4.58706,11.4052);
\draw [color=c, fill=c] (4.58706,11.2993) rectangle (4.62687,11.4052);
\draw [color=c, fill=c] (4.62687,11.2993) rectangle (4.66667,11.4052);
\draw [color=c, fill=c] (4.66667,11.2993) rectangle (4.70647,11.4052);
\draw [color=c, fill=c] (4.70647,11.2993) rectangle (4.74627,11.4052);
\draw [color=c, fill=c] (4.74627,11.2993) rectangle (4.78607,11.4052);
\draw [color=c, fill=c] (4.78607,11.2993) rectangle (4.82587,11.4052);
\draw [color=c, fill=c] (4.82587,11.2993) rectangle (4.86567,11.4052);
\draw [color=c, fill=c] (4.86567,11.2993) rectangle (4.90547,11.4052);
\draw [color=c, fill=c] (4.90547,11.2993) rectangle (4.94527,11.4052);
\draw [color=c, fill=c] (4.94527,11.2993) rectangle (4.98507,11.4052);
\draw [color=c, fill=c] (4.98507,11.2993) rectangle (5.02488,11.4052);
\draw [color=c, fill=c] (5.02488,11.2993) rectangle (5.06468,11.4052);
\draw [color=c, fill=c] (5.06468,11.2993) rectangle (5.10448,11.4052);
\draw [color=c, fill=c] (5.10448,11.2993) rectangle (5.14428,11.4052);
\draw [color=c, fill=c] (5.14428,11.2993) rectangle (5.18408,11.4052);
\draw [color=c, fill=c] (5.18408,11.2993) rectangle (5.22388,11.4052);
\draw [color=c, fill=c] (5.22388,11.2993) rectangle (5.26368,11.4052);
\draw [color=c, fill=c] (5.26368,11.2993) rectangle (5.30348,11.4052);
\draw [color=c, fill=c] (5.30348,11.2993) rectangle (5.34328,11.4052);
\draw [color=c, fill=c] (5.34328,11.2993) rectangle (5.38308,11.4052);
\draw [color=c, fill=c] (5.38308,11.2993) rectangle (5.42289,11.4052);
\draw [color=c, fill=c] (5.42289,11.2993) rectangle (5.46269,11.4052);
\draw [color=c, fill=c] (5.46269,11.2993) rectangle (5.50249,11.4052);
\draw [color=c, fill=c] (5.50249,11.2993) rectangle (5.54229,11.4052);
\draw [color=c, fill=c] (5.54229,11.2993) rectangle (5.58209,11.4052);
\draw [color=c, fill=c] (5.58209,11.2993) rectangle (5.62189,11.4052);
\draw [color=c, fill=c] (5.62189,11.2993) rectangle (5.66169,11.4052);
\draw [color=c, fill=c] (5.66169,11.2993) rectangle (5.70149,11.4052);
\draw [color=c, fill=c] (5.70149,11.2993) rectangle (5.74129,11.4052);
\draw [color=c, fill=c] (5.74129,11.2993) rectangle (5.78109,11.4052);
\draw [color=c, fill=c] (5.78109,11.2993) rectangle (5.8209,11.4052);
\draw [color=c, fill=c] (5.8209,11.2993) rectangle (5.8607,11.4052);
\draw [color=c, fill=c] (5.8607,11.2993) rectangle (5.9005,11.4052);
\draw [color=c, fill=c] (5.9005,11.2993) rectangle (5.9403,11.4052);
\draw [color=c, fill=c] (5.9403,11.2993) rectangle (5.9801,11.4052);
\draw [color=c, fill=c] (5.9801,11.2993) rectangle (6.0199,11.4052);
\draw [color=c, fill=c] (6.0199,11.2993) rectangle (6.0597,11.4052);
\draw [color=c, fill=c] (6.0597,11.2993) rectangle (6.0995,11.4052);
\draw [color=c, fill=c] (6.0995,11.2993) rectangle (6.1393,11.4052);
\draw [color=c, fill=c] (6.1393,11.2993) rectangle (6.1791,11.4052);
\draw [color=c, fill=c] (6.1791,11.2993) rectangle (6.21891,11.4052);
\draw [color=c, fill=c] (6.21891,11.2993) rectangle (6.25871,11.4052);
\draw [color=c, fill=c] (6.25871,11.2993) rectangle (6.29851,11.4052);
\draw [color=c, fill=c] (6.29851,11.2993) rectangle (6.33831,11.4052);
\draw [color=c, fill=c] (6.33831,11.2993) rectangle (6.37811,11.4052);
\draw [color=c, fill=c] (6.37811,11.2993) rectangle (6.41791,11.4052);
\draw [color=c, fill=c] (6.41791,11.2993) rectangle (6.45771,11.4052);
\draw [color=c, fill=c] (6.45771,11.2993) rectangle (6.49751,11.4052);
\draw [color=c, fill=c] (6.49751,11.2993) rectangle (6.53731,11.4052);
\draw [color=c, fill=c] (6.53731,11.2993) rectangle (6.57711,11.4052);
\draw [color=c, fill=c] (6.57711,11.2993) rectangle (6.61692,11.4052);
\draw [color=c, fill=c] (6.61692,11.2993) rectangle (6.65672,11.4052);
\draw [color=c, fill=c] (6.65672,11.2993) rectangle (6.69652,11.4052);
\draw [color=c, fill=c] (6.69652,11.2993) rectangle (6.73632,11.4052);
\draw [color=c, fill=c] (6.73632,11.2993) rectangle (6.77612,11.4052);
\draw [color=c, fill=c] (6.77612,11.2993) rectangle (6.81592,11.4052);
\draw [color=c, fill=c] (6.81592,11.2993) rectangle (6.85572,11.4052);
\draw [color=c, fill=c] (6.85572,11.2993) rectangle (6.89552,11.4052);
\draw [color=c, fill=c] (6.89552,11.2993) rectangle (6.93532,11.4052);
\draw [color=c, fill=c] (6.93532,11.2993) rectangle (6.97512,11.4052);
\draw [color=c, fill=c] (6.97512,11.2993) rectangle (7.01493,11.4052);
\draw [color=c, fill=c] (7.01493,11.2993) rectangle (7.05473,11.4052);
\draw [color=c, fill=c] (7.05473,11.2993) rectangle (7.09453,11.4052);
\draw [color=c, fill=c] (7.09453,11.2993) rectangle (7.13433,11.4052);
\draw [color=c, fill=c] (7.13433,11.2993) rectangle (7.17413,11.4052);
\draw [color=c, fill=c] (7.17413,11.2993) rectangle (7.21393,11.4052);
\draw [color=c, fill=c] (7.21393,11.2993) rectangle (7.25373,11.4052);
\draw [color=c, fill=c] (7.25373,11.2993) rectangle (7.29353,11.4052);
\draw [color=c, fill=c] (7.29353,11.2993) rectangle (7.33333,11.4052);
\draw [color=c, fill=c] (7.33333,11.2993) rectangle (7.37313,11.4052);
\draw [color=c, fill=c] (7.37313,11.2993) rectangle (7.41294,11.4052);
\draw [color=c, fill=c] (7.41294,11.2993) rectangle (7.45274,11.4052);
\draw [color=c, fill=c] (7.45274,11.2993) rectangle (7.49254,11.4052);
\draw [color=c, fill=c] (7.49254,11.2993) rectangle (7.53234,11.4052);
\draw [color=c, fill=c] (7.53234,11.2993) rectangle (7.57214,11.4052);
\draw [color=c, fill=c] (7.57214,11.2993) rectangle (7.61194,11.4052);
\draw [color=c, fill=c] (7.61194,11.2993) rectangle (7.65174,11.4052);
\draw [color=c, fill=c] (7.65174,11.2993) rectangle (7.69154,11.4052);
\draw [color=c, fill=c] (7.69154,11.2993) rectangle (7.73134,11.4052);
\draw [color=c, fill=c] (7.73134,11.2993) rectangle (7.77114,11.4052);
\draw [color=c, fill=c] (7.77114,11.2993) rectangle (7.81095,11.4052);
\draw [color=c, fill=c] (7.81095,11.2993) rectangle (7.85075,11.4052);
\draw [color=c, fill=c] (7.85075,11.2993) rectangle (7.89055,11.4052);
\draw [color=c, fill=c] (7.89055,11.2993) rectangle (7.93035,11.4052);
\draw [color=c, fill=c] (7.93035,11.2993) rectangle (7.97015,11.4052);
\draw [color=c, fill=c] (7.97015,11.2993) rectangle (8.00995,11.4052);
\definecolor{c}{rgb}{0,0.0800001,1};
\draw [color=c, fill=c] (8.00995,11.2993) rectangle (8.04975,11.4052);
\draw [color=c, fill=c] (8.04975,11.2993) rectangle (8.08955,11.4052);
\draw [color=c, fill=c] (8.08955,11.2993) rectangle (8.12935,11.4052);
\draw [color=c, fill=c] (8.12935,11.2993) rectangle (8.16915,11.4052);
\draw [color=c, fill=c] (8.16915,11.2993) rectangle (8.20895,11.4052);
\draw [color=c, fill=c] (8.20895,11.2993) rectangle (8.24876,11.4052);
\draw [color=c, fill=c] (8.24876,11.2993) rectangle (8.28856,11.4052);
\draw [color=c, fill=c] (8.28856,11.2993) rectangle (8.32836,11.4052);
\draw [color=c, fill=c] (8.32836,11.2993) rectangle (8.36816,11.4052);
\draw [color=c, fill=c] (8.36816,11.2993) rectangle (8.40796,11.4052);
\draw [color=c, fill=c] (8.40796,11.2993) rectangle (8.44776,11.4052);
\draw [color=c, fill=c] (8.44776,11.2993) rectangle (8.48756,11.4052);
\draw [color=c, fill=c] (8.48756,11.2993) rectangle (8.52736,11.4052);
\draw [color=c, fill=c] (8.52736,11.2993) rectangle (8.56716,11.4052);
\draw [color=c, fill=c] (8.56716,11.2993) rectangle (8.60697,11.4052);
\draw [color=c, fill=c] (8.60697,11.2993) rectangle (8.64677,11.4052);
\draw [color=c, fill=c] (8.64677,11.2993) rectangle (8.68657,11.4052);
\draw [color=c, fill=c] (8.68657,11.2993) rectangle (8.72637,11.4052);
\draw [color=c, fill=c] (8.72637,11.2993) rectangle (8.76617,11.4052);
\draw [color=c, fill=c] (8.76617,11.2993) rectangle (8.80597,11.4052);
\draw [color=c, fill=c] (8.80597,11.2993) rectangle (8.84577,11.4052);
\draw [color=c, fill=c] (8.84577,11.2993) rectangle (8.88557,11.4052);
\draw [color=c, fill=c] (8.88557,11.2993) rectangle (8.92537,11.4052);
\draw [color=c, fill=c] (8.92537,11.2993) rectangle (8.96517,11.4052);
\draw [color=c, fill=c] (8.96517,11.2993) rectangle (9.00498,11.4052);
\draw [color=c, fill=c] (9.00498,11.2993) rectangle (9.04478,11.4052);
\draw [color=c, fill=c] (9.04478,11.2993) rectangle (9.08458,11.4052);
\draw [color=c, fill=c] (9.08458,11.2993) rectangle (9.12438,11.4052);
\draw [color=c, fill=c] (9.12438,11.2993) rectangle (9.16418,11.4052);
\draw [color=c, fill=c] (9.16418,11.2993) rectangle (9.20398,11.4052);
\draw [color=c, fill=c] (9.20398,11.2993) rectangle (9.24378,11.4052);
\draw [color=c, fill=c] (9.24378,11.2993) rectangle (9.28358,11.4052);
\draw [color=c, fill=c] (9.28358,11.2993) rectangle (9.32338,11.4052);
\draw [color=c, fill=c] (9.32338,11.2993) rectangle (9.36318,11.4052);
\draw [color=c, fill=c] (9.36318,11.2993) rectangle (9.40298,11.4052);
\draw [color=c, fill=c] (9.40298,11.2993) rectangle (9.44279,11.4052);
\draw [color=c, fill=c] (9.44279,11.2993) rectangle (9.48259,11.4052);
\draw [color=c, fill=c] (9.48259,11.2993) rectangle (9.52239,11.4052);
\draw [color=c, fill=c] (9.52239,11.2993) rectangle (9.56219,11.4052);
\draw [color=c, fill=c] (9.56219,11.2993) rectangle (9.60199,11.4052);
\draw [color=c, fill=c] (9.60199,11.2993) rectangle (9.64179,11.4052);
\definecolor{c}{rgb}{0,0.266667,1};
\draw [color=c, fill=c] (9.64179,11.2993) rectangle (9.68159,11.4052);
\draw [color=c, fill=c] (9.68159,11.2993) rectangle (9.72139,11.4052);
\draw [color=c, fill=c] (9.72139,11.2993) rectangle (9.76119,11.4052);
\draw [color=c, fill=c] (9.76119,11.2993) rectangle (9.80099,11.4052);
\draw [color=c, fill=c] (9.80099,11.2993) rectangle (9.8408,11.4052);
\draw [color=c, fill=c] (9.8408,11.2993) rectangle (9.8806,11.4052);
\draw [color=c, fill=c] (9.8806,11.2993) rectangle (9.9204,11.4052);
\draw [color=c, fill=c] (9.9204,11.2993) rectangle (9.9602,11.4052);
\draw [color=c, fill=c] (9.9602,11.2993) rectangle (10,11.4052);
\draw [color=c, fill=c] (10,11.2993) rectangle (10.0398,11.4052);
\draw [color=c, fill=c] (10.0398,11.2993) rectangle (10.0796,11.4052);
\draw [color=c, fill=c] (10.0796,11.2993) rectangle (10.1194,11.4052);
\draw [color=c, fill=c] (10.1194,11.2993) rectangle (10.1592,11.4052);
\draw [color=c, fill=c] (10.1592,11.2993) rectangle (10.199,11.4052);
\draw [color=c, fill=c] (10.199,11.2993) rectangle (10.2388,11.4052);
\draw [color=c, fill=c] (10.2388,11.2993) rectangle (10.2786,11.4052);
\draw [color=c, fill=c] (10.2786,11.2993) rectangle (10.3184,11.4052);
\draw [color=c, fill=c] (10.3184,11.2993) rectangle (10.3582,11.4052);
\draw [color=c, fill=c] (10.3582,11.2993) rectangle (10.398,11.4052);
\draw [color=c, fill=c] (10.398,11.2993) rectangle (10.4378,11.4052);
\draw [color=c, fill=c] (10.4378,11.2993) rectangle (10.4776,11.4052);
\draw [color=c, fill=c] (10.4776,11.2993) rectangle (10.5174,11.4052);
\draw [color=c, fill=c] (10.5174,11.2993) rectangle (10.5572,11.4052);
\draw [color=c, fill=c] (10.5572,11.2993) rectangle (10.597,11.4052);
\draw [color=c, fill=c] (10.597,11.2993) rectangle (10.6368,11.4052);
\draw [color=c, fill=c] (10.6368,11.2993) rectangle (10.6766,11.4052);
\draw [color=c, fill=c] (10.6766,11.2993) rectangle (10.7164,11.4052);
\draw [color=c, fill=c] (10.7164,11.2993) rectangle (10.7562,11.4052);
\draw [color=c, fill=c] (10.7562,11.2993) rectangle (10.796,11.4052);
\draw [color=c, fill=c] (10.796,11.2993) rectangle (10.8358,11.4052);
\draw [color=c, fill=c] (10.8358,11.2993) rectangle (10.8756,11.4052);
\draw [color=c, fill=c] (10.8756,11.2993) rectangle (10.9154,11.4052);
\draw [color=c, fill=c] (10.9154,11.2993) rectangle (10.9552,11.4052);
\draw [color=c, fill=c] (10.9552,11.2993) rectangle (10.995,11.4052);
\definecolor{c}{rgb}{0,0.546666,1};
\draw [color=c, fill=c] (10.995,11.2993) rectangle (11.0348,11.4052);
\draw [color=c, fill=c] (11.0348,11.2993) rectangle (11.0746,11.4052);
\draw [color=c, fill=c] (11.0746,11.2993) rectangle (11.1144,11.4052);
\draw [color=c, fill=c] (11.1144,11.2993) rectangle (11.1542,11.4052);
\draw [color=c, fill=c] (11.1542,11.2993) rectangle (11.194,11.4052);
\draw [color=c, fill=c] (11.194,11.2993) rectangle (11.2338,11.4052);
\draw [color=c, fill=c] (11.2338,11.2993) rectangle (11.2736,11.4052);
\draw [color=c, fill=c] (11.2736,11.2993) rectangle (11.3134,11.4052);
\draw [color=c, fill=c] (11.3134,11.2993) rectangle (11.3532,11.4052);
\draw [color=c, fill=c] (11.3532,11.2993) rectangle (11.393,11.4052);
\draw [color=c, fill=c] (11.393,11.2993) rectangle (11.4328,11.4052);
\draw [color=c, fill=c] (11.4328,11.2993) rectangle (11.4726,11.4052);
\draw [color=c, fill=c] (11.4726,11.2993) rectangle (11.5124,11.4052);
\draw [color=c, fill=c] (11.5124,11.2993) rectangle (11.5522,11.4052);
\draw [color=c, fill=c] (11.5522,11.2993) rectangle (11.592,11.4052);
\draw [color=c, fill=c] (11.592,11.2993) rectangle (11.6318,11.4052);
\draw [color=c, fill=c] (11.6318,11.2993) rectangle (11.6716,11.4052);
\draw [color=c, fill=c] (11.6716,11.2993) rectangle (11.7114,11.4052);
\draw [color=c, fill=c] (11.7114,11.2993) rectangle (11.7512,11.4052);
\draw [color=c, fill=c] (11.7512,11.2993) rectangle (11.791,11.4052);
\draw [color=c, fill=c] (11.791,11.2993) rectangle (11.8308,11.4052);
\draw [color=c, fill=c] (11.8308,11.2993) rectangle (11.8706,11.4052);
\draw [color=c, fill=c] (11.8706,11.2993) rectangle (11.9104,11.4052);
\draw [color=c, fill=c] (11.9104,11.2993) rectangle (11.9502,11.4052);
\draw [color=c, fill=c] (11.9502,11.2993) rectangle (11.99,11.4052);
\draw [color=c, fill=c] (11.99,11.2993) rectangle (12.0299,11.4052);
\draw [color=c, fill=c] (12.0299,11.2993) rectangle (12.0697,11.4052);
\draw [color=c, fill=c] (12.0697,11.2993) rectangle (12.1095,11.4052);
\draw [color=c, fill=c] (12.1095,11.2993) rectangle (12.1493,11.4052);
\draw [color=c, fill=c] (12.1493,11.2993) rectangle (12.1891,11.4052);
\draw [color=c, fill=c] (12.1891,11.2993) rectangle (12.2289,11.4052);
\draw [color=c, fill=c] (12.2289,11.2993) rectangle (12.2687,11.4052);
\draw [color=c, fill=c] (12.2687,11.2993) rectangle (12.3085,11.4052);
\draw [color=c, fill=c] (12.3085,11.2993) rectangle (12.3483,11.4052);
\draw [color=c, fill=c] (12.3483,11.2993) rectangle (12.3881,11.4052);
\draw [color=c, fill=c] (12.3881,11.2993) rectangle (12.4279,11.4052);
\draw [color=c, fill=c] (12.4279,11.2993) rectangle (12.4677,11.4052);
\draw [color=c, fill=c] (12.4677,11.2993) rectangle (12.5075,11.4052);
\draw [color=c, fill=c] (12.5075,11.2993) rectangle (12.5473,11.4052);
\draw [color=c, fill=c] (12.5473,11.2993) rectangle (12.5871,11.4052);
\draw [color=c, fill=c] (12.5871,11.2993) rectangle (12.6269,11.4052);
\draw [color=c, fill=c] (12.6269,11.2993) rectangle (12.6667,11.4052);
\draw [color=c, fill=c] (12.6667,11.2993) rectangle (12.7065,11.4052);
\draw [color=c, fill=c] (12.7065,11.2993) rectangle (12.7463,11.4052);
\draw [color=c, fill=c] (12.7463,11.2993) rectangle (12.7861,11.4052);
\draw [color=c, fill=c] (12.7861,11.2993) rectangle (12.8259,11.4052);
\draw [color=c, fill=c] (12.8259,11.2993) rectangle (12.8657,11.4052);
\draw [color=c, fill=c] (12.8657,11.2993) rectangle (12.9055,11.4052);
\draw [color=c, fill=c] (12.9055,11.2993) rectangle (12.9453,11.4052);
\draw [color=c, fill=c] (12.9453,11.2993) rectangle (12.9851,11.4052);
\draw [color=c, fill=c] (12.9851,11.2993) rectangle (13.0249,11.4052);
\draw [color=c, fill=c] (13.0249,11.2993) rectangle (13.0647,11.4052);
\draw [color=c, fill=c] (13.0647,11.2993) rectangle (13.1045,11.4052);
\draw [color=c, fill=c] (13.1045,11.2993) rectangle (13.1443,11.4052);
\draw [color=c, fill=c] (13.1443,11.2993) rectangle (13.1841,11.4052);
\draw [color=c, fill=c] (13.1841,11.2993) rectangle (13.2239,11.4052);
\draw [color=c, fill=c] (13.2239,11.2993) rectangle (13.2637,11.4052);
\draw [color=c, fill=c] (13.2637,11.2993) rectangle (13.3035,11.4052);
\draw [color=c, fill=c] (13.3035,11.2993) rectangle (13.3433,11.4052);
\draw [color=c, fill=c] (13.3433,11.2993) rectangle (13.3831,11.4052);
\draw [color=c, fill=c] (13.3831,11.2993) rectangle (13.4229,11.4052);
\draw [color=c, fill=c] (13.4229,11.2993) rectangle (13.4627,11.4052);
\draw [color=c, fill=c] (13.4627,11.2993) rectangle (13.5025,11.4052);
\draw [color=c, fill=c] (13.5025,11.2993) rectangle (13.5423,11.4052);
\draw [color=c, fill=c] (13.5423,11.2993) rectangle (13.5821,11.4052);
\draw [color=c, fill=c] (13.5821,11.2993) rectangle (13.6219,11.4052);
\draw [color=c, fill=c] (13.6219,11.2993) rectangle (13.6617,11.4052);
\draw [color=c, fill=c] (13.6617,11.2993) rectangle (13.7015,11.4052);
\draw [color=c, fill=c] (13.7015,11.2993) rectangle (13.7413,11.4052);
\draw [color=c, fill=c] (13.7413,11.2993) rectangle (13.7811,11.4052);
\draw [color=c, fill=c] (13.7811,11.2993) rectangle (13.8209,11.4052);
\draw [color=c, fill=c] (13.8209,11.2993) rectangle (13.8607,11.4052);
\draw [color=c, fill=c] (13.8607,11.2993) rectangle (13.9005,11.4052);
\draw [color=c, fill=c] (13.9005,11.2993) rectangle (13.9403,11.4052);
\draw [color=c, fill=c] (13.9403,11.2993) rectangle (13.9801,11.4052);
\draw [color=c, fill=c] (13.9801,11.2993) rectangle (14.0199,11.4052);
\draw [color=c, fill=c] (14.0199,11.2993) rectangle (14.0597,11.4052);
\draw [color=c, fill=c] (14.0597,11.2993) rectangle (14.0995,11.4052);
\draw [color=c, fill=c] (14.0995,11.2993) rectangle (14.1393,11.4052);
\draw [color=c, fill=c] (14.1393,11.2993) rectangle (14.1791,11.4052);
\draw [color=c, fill=c] (14.1791,11.2993) rectangle (14.2189,11.4052);
\draw [color=c, fill=c] (14.2189,11.2993) rectangle (14.2587,11.4052);
\draw [color=c, fill=c] (14.2587,11.2993) rectangle (14.2985,11.4052);
\draw [color=c, fill=c] (14.2985,11.2993) rectangle (14.3383,11.4052);
\draw [color=c, fill=c] (14.3383,11.2993) rectangle (14.3781,11.4052);
\draw [color=c, fill=c] (14.3781,11.2993) rectangle (14.4179,11.4052);
\definecolor{c}{rgb}{0,0.733333,1};
\draw [color=c, fill=c] (14.4179,11.2993) rectangle (14.4577,11.4052);
\draw [color=c, fill=c] (14.4577,11.2993) rectangle (14.4975,11.4052);
\draw [color=c, fill=c] (14.4975,11.2993) rectangle (14.5373,11.4052);
\draw [color=c, fill=c] (14.5373,11.2993) rectangle (14.5771,11.4052);
\draw [color=c, fill=c] (14.5771,11.2993) rectangle (14.6169,11.4052);
\draw [color=c, fill=c] (14.6169,11.2993) rectangle (14.6567,11.4052);
\draw [color=c, fill=c] (14.6567,11.2993) rectangle (14.6965,11.4052);
\draw [color=c, fill=c] (14.6965,11.2993) rectangle (14.7363,11.4052);
\draw [color=c, fill=c] (14.7363,11.2993) rectangle (14.7761,11.4052);
\draw [color=c, fill=c] (14.7761,11.2993) rectangle (14.8159,11.4052);
\draw [color=c, fill=c] (14.8159,11.2993) rectangle (14.8557,11.4052);
\draw [color=c, fill=c] (14.8557,11.2993) rectangle (14.8955,11.4052);
\draw [color=c, fill=c] (14.8955,11.2993) rectangle (14.9353,11.4052);
\draw [color=c, fill=c] (14.9353,11.2993) rectangle (14.9751,11.4052);
\draw [color=c, fill=c] (14.9751,11.2993) rectangle (15.0149,11.4052);
\draw [color=c, fill=c] (15.0149,11.2993) rectangle (15.0547,11.4052);
\draw [color=c, fill=c] (15.0547,11.2993) rectangle (15.0945,11.4052);
\draw [color=c, fill=c] (15.0945,11.2993) rectangle (15.1343,11.4052);
\draw [color=c, fill=c] (15.1343,11.2993) rectangle (15.1741,11.4052);
\draw [color=c, fill=c] (15.1741,11.2993) rectangle (15.2139,11.4052);
\draw [color=c, fill=c] (15.2139,11.2993) rectangle (15.2537,11.4052);
\draw [color=c, fill=c] (15.2537,11.2993) rectangle (15.2935,11.4052);
\draw [color=c, fill=c] (15.2935,11.2993) rectangle (15.3333,11.4052);
\draw [color=c, fill=c] (15.3333,11.2993) rectangle (15.3731,11.4052);
\draw [color=c, fill=c] (15.3731,11.2993) rectangle (15.4129,11.4052);
\draw [color=c, fill=c] (15.4129,11.2993) rectangle (15.4527,11.4052);
\draw [color=c, fill=c] (15.4527,11.2993) rectangle (15.4925,11.4052);
\draw [color=c, fill=c] (15.4925,11.2993) rectangle (15.5323,11.4052);
\draw [color=c, fill=c] (15.5323,11.2993) rectangle (15.5721,11.4052);
\draw [color=c, fill=c] (15.5721,11.2993) rectangle (15.6119,11.4052);
\draw [color=c, fill=c] (15.6119,11.2993) rectangle (15.6517,11.4052);
\draw [color=c, fill=c] (15.6517,11.2993) rectangle (15.6915,11.4052);
\draw [color=c, fill=c] (15.6915,11.2993) rectangle (15.7313,11.4052);
\draw [color=c, fill=c] (15.7313,11.2993) rectangle (15.7711,11.4052);
\draw [color=c, fill=c] (15.7711,11.2993) rectangle (15.8109,11.4052);
\draw [color=c, fill=c] (15.8109,11.2993) rectangle (15.8507,11.4052);
\draw [color=c, fill=c] (15.8507,11.2993) rectangle (15.8905,11.4052);
\draw [color=c, fill=c] (15.8905,11.2993) rectangle (15.9303,11.4052);
\draw [color=c, fill=c] (15.9303,11.2993) rectangle (15.9701,11.4052);
\draw [color=c, fill=c] (15.9701,11.2993) rectangle (16.01,11.4052);
\draw [color=c, fill=c] (16.01,11.2993) rectangle (16.0498,11.4052);
\draw [color=c, fill=c] (16.0498,11.2993) rectangle (16.0896,11.4052);
\draw [color=c, fill=c] (16.0896,11.2993) rectangle (16.1294,11.4052);
\draw [color=c, fill=c] (16.1294,11.2993) rectangle (16.1692,11.4052);
\draw [color=c, fill=c] (16.1692,11.2993) rectangle (16.209,11.4052);
\draw [color=c, fill=c] (16.209,11.2993) rectangle (16.2488,11.4052);
\draw [color=c, fill=c] (16.2488,11.2993) rectangle (16.2886,11.4052);
\draw [color=c, fill=c] (16.2886,11.2993) rectangle (16.3284,11.4052);
\draw [color=c, fill=c] (16.3284,11.2993) rectangle (16.3682,11.4052);
\draw [color=c, fill=c] (16.3682,11.2993) rectangle (16.408,11.4052);
\draw [color=c, fill=c] (16.408,11.2993) rectangle (16.4478,11.4052);
\draw [color=c, fill=c] (16.4478,11.2993) rectangle (16.4876,11.4052);
\draw [color=c, fill=c] (16.4876,11.2993) rectangle (16.5274,11.4052);
\draw [color=c, fill=c] (16.5274,11.2993) rectangle (16.5672,11.4052);
\draw [color=c, fill=c] (16.5672,11.2993) rectangle (16.607,11.4052);
\draw [color=c, fill=c] (16.607,11.2993) rectangle (16.6468,11.4052);
\draw [color=c, fill=c] (16.6468,11.2993) rectangle (16.6866,11.4052);
\draw [color=c, fill=c] (16.6866,11.2993) rectangle (16.7264,11.4052);
\draw [color=c, fill=c] (16.7264,11.2993) rectangle (16.7662,11.4052);
\draw [color=c, fill=c] (16.7662,11.2993) rectangle (16.806,11.4052);
\draw [color=c, fill=c] (16.806,11.2993) rectangle (16.8458,11.4052);
\draw [color=c, fill=c] (16.8458,11.2993) rectangle (16.8856,11.4052);
\draw [color=c, fill=c] (16.8856,11.2993) rectangle (16.9254,11.4052);
\draw [color=c, fill=c] (16.9254,11.2993) rectangle (16.9652,11.4052);
\draw [color=c, fill=c] (16.9652,11.2993) rectangle (17.005,11.4052);
\draw [color=c, fill=c] (17.005,11.2993) rectangle (17.0448,11.4052);
\draw [color=c, fill=c] (17.0448,11.2993) rectangle (17.0846,11.4052);
\draw [color=c, fill=c] (17.0846,11.2993) rectangle (17.1244,11.4052);
\draw [color=c, fill=c] (17.1244,11.2993) rectangle (17.1642,11.4052);
\draw [color=c, fill=c] (17.1642,11.2993) rectangle (17.204,11.4052);
\draw [color=c, fill=c] (17.204,11.2993) rectangle (17.2438,11.4052);
\draw [color=c, fill=c] (17.2438,11.2993) rectangle (17.2836,11.4052);
\draw [color=c, fill=c] (17.2836,11.2993) rectangle (17.3234,11.4052);
\draw [color=c, fill=c] (17.3234,11.2993) rectangle (17.3632,11.4052);
\draw [color=c, fill=c] (17.3632,11.2993) rectangle (17.403,11.4052);
\draw [color=c, fill=c] (17.403,11.2993) rectangle (17.4428,11.4052);
\draw [color=c, fill=c] (17.4428,11.2993) rectangle (17.4826,11.4052);
\draw [color=c, fill=c] (17.4826,11.2993) rectangle (17.5224,11.4052);
\draw [color=c, fill=c] (17.5224,11.2993) rectangle (17.5622,11.4052);
\draw [color=c, fill=c] (17.5622,11.2993) rectangle (17.602,11.4052);
\draw [color=c, fill=c] (17.602,11.2993) rectangle (17.6418,11.4052);
\draw [color=c, fill=c] (17.6418,11.2993) rectangle (17.6816,11.4052);
\draw [color=c, fill=c] (17.6816,11.2993) rectangle (17.7214,11.4052);
\draw [color=c, fill=c] (17.7214,11.2993) rectangle (17.7612,11.4052);
\draw [color=c, fill=c] (17.7612,11.2993) rectangle (17.801,11.4052);
\draw [color=c, fill=c] (17.801,11.2993) rectangle (17.8408,11.4052);
\draw [color=c, fill=c] (17.8408,11.2993) rectangle (17.8806,11.4052);
\draw [color=c, fill=c] (17.8806,11.2993) rectangle (17.9204,11.4052);
\draw [color=c, fill=c] (17.9204,11.2993) rectangle (17.9602,11.4052);
\draw [color=c, fill=c] (17.9602,11.2993) rectangle (18,11.4052);
\definecolor{c}{rgb}{0.2,0,1};
\draw [color=c, fill=c] (2,11.4052) rectangle (2.0398,11.511);
\draw [color=c, fill=c] (2.0398,11.4052) rectangle (2.0796,11.511);
\draw [color=c, fill=c] (2.0796,11.4052) rectangle (2.1194,11.511);
\draw [color=c, fill=c] (2.1194,11.4052) rectangle (2.1592,11.511);
\draw [color=c, fill=c] (2.1592,11.4052) rectangle (2.19901,11.511);
\draw [color=c, fill=c] (2.19901,11.4052) rectangle (2.23881,11.511);
\draw [color=c, fill=c] (2.23881,11.4052) rectangle (2.27861,11.511);
\draw [color=c, fill=c] (2.27861,11.4052) rectangle (2.31841,11.511);
\draw [color=c, fill=c] (2.31841,11.4052) rectangle (2.35821,11.511);
\draw [color=c, fill=c] (2.35821,11.4052) rectangle (2.39801,11.511);
\draw [color=c, fill=c] (2.39801,11.4052) rectangle (2.43781,11.511);
\draw [color=c, fill=c] (2.43781,11.4052) rectangle (2.47761,11.511);
\draw [color=c, fill=c] (2.47761,11.4052) rectangle (2.51741,11.511);
\draw [color=c, fill=c] (2.51741,11.4052) rectangle (2.55721,11.511);
\draw [color=c, fill=c] (2.55721,11.4052) rectangle (2.59702,11.511);
\draw [color=c, fill=c] (2.59702,11.4052) rectangle (2.63682,11.511);
\draw [color=c, fill=c] (2.63682,11.4052) rectangle (2.67662,11.511);
\draw [color=c, fill=c] (2.67662,11.4052) rectangle (2.71642,11.511);
\draw [color=c, fill=c] (2.71642,11.4052) rectangle (2.75622,11.511);
\draw [color=c, fill=c] (2.75622,11.4052) rectangle (2.79602,11.511);
\draw [color=c, fill=c] (2.79602,11.4052) rectangle (2.83582,11.511);
\draw [color=c, fill=c] (2.83582,11.4052) rectangle (2.87562,11.511);
\draw [color=c, fill=c] (2.87562,11.4052) rectangle (2.91542,11.511);
\draw [color=c, fill=c] (2.91542,11.4052) rectangle (2.95522,11.511);
\draw [color=c, fill=c] (2.95522,11.4052) rectangle (2.99502,11.511);
\draw [color=c, fill=c] (2.99502,11.4052) rectangle (3.03483,11.511);
\draw [color=c, fill=c] (3.03483,11.4052) rectangle (3.07463,11.511);
\draw [color=c, fill=c] (3.07463,11.4052) rectangle (3.11443,11.511);
\draw [color=c, fill=c] (3.11443,11.4052) rectangle (3.15423,11.511);
\draw [color=c, fill=c] (3.15423,11.4052) rectangle (3.19403,11.511);
\draw [color=c, fill=c] (3.19403,11.4052) rectangle (3.23383,11.511);
\draw [color=c, fill=c] (3.23383,11.4052) rectangle (3.27363,11.511);
\draw [color=c, fill=c] (3.27363,11.4052) rectangle (3.31343,11.511);
\draw [color=c, fill=c] (3.31343,11.4052) rectangle (3.35323,11.511);
\draw [color=c, fill=c] (3.35323,11.4052) rectangle (3.39303,11.511);
\draw [color=c, fill=c] (3.39303,11.4052) rectangle (3.43284,11.511);
\draw [color=c, fill=c] (3.43284,11.4052) rectangle (3.47264,11.511);
\draw [color=c, fill=c] (3.47264,11.4052) rectangle (3.51244,11.511);
\draw [color=c, fill=c] (3.51244,11.4052) rectangle (3.55224,11.511);
\draw [color=c, fill=c] (3.55224,11.4052) rectangle (3.59204,11.511);
\draw [color=c, fill=c] (3.59204,11.4052) rectangle (3.63184,11.511);
\draw [color=c, fill=c] (3.63184,11.4052) rectangle (3.67164,11.511);
\draw [color=c, fill=c] (3.67164,11.4052) rectangle (3.71144,11.511);
\draw [color=c, fill=c] (3.71144,11.4052) rectangle (3.75124,11.511);
\draw [color=c, fill=c] (3.75124,11.4052) rectangle (3.79104,11.511);
\draw [color=c, fill=c] (3.79104,11.4052) rectangle (3.83085,11.511);
\draw [color=c, fill=c] (3.83085,11.4052) rectangle (3.87065,11.511);
\draw [color=c, fill=c] (3.87065,11.4052) rectangle (3.91045,11.511);
\draw [color=c, fill=c] (3.91045,11.4052) rectangle (3.95025,11.511);
\draw [color=c, fill=c] (3.95025,11.4052) rectangle (3.99005,11.511);
\draw [color=c, fill=c] (3.99005,11.4052) rectangle (4.02985,11.511);
\draw [color=c, fill=c] (4.02985,11.4052) rectangle (4.06965,11.511);
\draw [color=c, fill=c] (4.06965,11.4052) rectangle (4.10945,11.511);
\draw [color=c, fill=c] (4.10945,11.4052) rectangle (4.14925,11.511);
\draw [color=c, fill=c] (4.14925,11.4052) rectangle (4.18905,11.511);
\draw [color=c, fill=c] (4.18905,11.4052) rectangle (4.22886,11.511);
\draw [color=c, fill=c] (4.22886,11.4052) rectangle (4.26866,11.511);
\draw [color=c, fill=c] (4.26866,11.4052) rectangle (4.30846,11.511);
\draw [color=c, fill=c] (4.30846,11.4052) rectangle (4.34826,11.511);
\draw [color=c, fill=c] (4.34826,11.4052) rectangle (4.38806,11.511);
\draw [color=c, fill=c] (4.38806,11.4052) rectangle (4.42786,11.511);
\draw [color=c, fill=c] (4.42786,11.4052) rectangle (4.46766,11.511);
\draw [color=c, fill=c] (4.46766,11.4052) rectangle (4.50746,11.511);
\draw [color=c, fill=c] (4.50746,11.4052) rectangle (4.54726,11.511);
\draw [color=c, fill=c] (4.54726,11.4052) rectangle (4.58706,11.511);
\draw [color=c, fill=c] (4.58706,11.4052) rectangle (4.62687,11.511);
\draw [color=c, fill=c] (4.62687,11.4052) rectangle (4.66667,11.511);
\draw [color=c, fill=c] (4.66667,11.4052) rectangle (4.70647,11.511);
\draw [color=c, fill=c] (4.70647,11.4052) rectangle (4.74627,11.511);
\draw [color=c, fill=c] (4.74627,11.4052) rectangle (4.78607,11.511);
\draw [color=c, fill=c] (4.78607,11.4052) rectangle (4.82587,11.511);
\draw [color=c, fill=c] (4.82587,11.4052) rectangle (4.86567,11.511);
\draw [color=c, fill=c] (4.86567,11.4052) rectangle (4.90547,11.511);
\draw [color=c, fill=c] (4.90547,11.4052) rectangle (4.94527,11.511);
\draw [color=c, fill=c] (4.94527,11.4052) rectangle (4.98507,11.511);
\draw [color=c, fill=c] (4.98507,11.4052) rectangle (5.02488,11.511);
\draw [color=c, fill=c] (5.02488,11.4052) rectangle (5.06468,11.511);
\draw [color=c, fill=c] (5.06468,11.4052) rectangle (5.10448,11.511);
\draw [color=c, fill=c] (5.10448,11.4052) rectangle (5.14428,11.511);
\draw [color=c, fill=c] (5.14428,11.4052) rectangle (5.18408,11.511);
\draw [color=c, fill=c] (5.18408,11.4052) rectangle (5.22388,11.511);
\draw [color=c, fill=c] (5.22388,11.4052) rectangle (5.26368,11.511);
\draw [color=c, fill=c] (5.26368,11.4052) rectangle (5.30348,11.511);
\draw [color=c, fill=c] (5.30348,11.4052) rectangle (5.34328,11.511);
\draw [color=c, fill=c] (5.34328,11.4052) rectangle (5.38308,11.511);
\draw [color=c, fill=c] (5.38308,11.4052) rectangle (5.42289,11.511);
\draw [color=c, fill=c] (5.42289,11.4052) rectangle (5.46269,11.511);
\draw [color=c, fill=c] (5.46269,11.4052) rectangle (5.50249,11.511);
\draw [color=c, fill=c] (5.50249,11.4052) rectangle (5.54229,11.511);
\draw [color=c, fill=c] (5.54229,11.4052) rectangle (5.58209,11.511);
\draw [color=c, fill=c] (5.58209,11.4052) rectangle (5.62189,11.511);
\draw [color=c, fill=c] (5.62189,11.4052) rectangle (5.66169,11.511);
\draw [color=c, fill=c] (5.66169,11.4052) rectangle (5.70149,11.511);
\draw [color=c, fill=c] (5.70149,11.4052) rectangle (5.74129,11.511);
\draw [color=c, fill=c] (5.74129,11.4052) rectangle (5.78109,11.511);
\draw [color=c, fill=c] (5.78109,11.4052) rectangle (5.8209,11.511);
\draw [color=c, fill=c] (5.8209,11.4052) rectangle (5.8607,11.511);
\draw [color=c, fill=c] (5.8607,11.4052) rectangle (5.9005,11.511);
\draw [color=c, fill=c] (5.9005,11.4052) rectangle (5.9403,11.511);
\draw [color=c, fill=c] (5.9403,11.4052) rectangle (5.9801,11.511);
\draw [color=c, fill=c] (5.9801,11.4052) rectangle (6.0199,11.511);
\draw [color=c, fill=c] (6.0199,11.4052) rectangle (6.0597,11.511);
\draw [color=c, fill=c] (6.0597,11.4052) rectangle (6.0995,11.511);
\draw [color=c, fill=c] (6.0995,11.4052) rectangle (6.1393,11.511);
\draw [color=c, fill=c] (6.1393,11.4052) rectangle (6.1791,11.511);
\draw [color=c, fill=c] (6.1791,11.4052) rectangle (6.21891,11.511);
\draw [color=c, fill=c] (6.21891,11.4052) rectangle (6.25871,11.511);
\draw [color=c, fill=c] (6.25871,11.4052) rectangle (6.29851,11.511);
\draw [color=c, fill=c] (6.29851,11.4052) rectangle (6.33831,11.511);
\draw [color=c, fill=c] (6.33831,11.4052) rectangle (6.37811,11.511);
\draw [color=c, fill=c] (6.37811,11.4052) rectangle (6.41791,11.511);
\draw [color=c, fill=c] (6.41791,11.4052) rectangle (6.45771,11.511);
\draw [color=c, fill=c] (6.45771,11.4052) rectangle (6.49751,11.511);
\draw [color=c, fill=c] (6.49751,11.4052) rectangle (6.53731,11.511);
\draw [color=c, fill=c] (6.53731,11.4052) rectangle (6.57711,11.511);
\draw [color=c, fill=c] (6.57711,11.4052) rectangle (6.61692,11.511);
\draw [color=c, fill=c] (6.61692,11.4052) rectangle (6.65672,11.511);
\draw [color=c, fill=c] (6.65672,11.4052) rectangle (6.69652,11.511);
\draw [color=c, fill=c] (6.69652,11.4052) rectangle (6.73632,11.511);
\draw [color=c, fill=c] (6.73632,11.4052) rectangle (6.77612,11.511);
\draw [color=c, fill=c] (6.77612,11.4052) rectangle (6.81592,11.511);
\draw [color=c, fill=c] (6.81592,11.4052) rectangle (6.85572,11.511);
\draw [color=c, fill=c] (6.85572,11.4052) rectangle (6.89552,11.511);
\draw [color=c, fill=c] (6.89552,11.4052) rectangle (6.93532,11.511);
\draw [color=c, fill=c] (6.93532,11.4052) rectangle (6.97512,11.511);
\draw [color=c, fill=c] (6.97512,11.4052) rectangle (7.01493,11.511);
\draw [color=c, fill=c] (7.01493,11.4052) rectangle (7.05473,11.511);
\draw [color=c, fill=c] (7.05473,11.4052) rectangle (7.09453,11.511);
\draw [color=c, fill=c] (7.09453,11.4052) rectangle (7.13433,11.511);
\draw [color=c, fill=c] (7.13433,11.4052) rectangle (7.17413,11.511);
\draw [color=c, fill=c] (7.17413,11.4052) rectangle (7.21393,11.511);
\draw [color=c, fill=c] (7.21393,11.4052) rectangle (7.25373,11.511);
\draw [color=c, fill=c] (7.25373,11.4052) rectangle (7.29353,11.511);
\draw [color=c, fill=c] (7.29353,11.4052) rectangle (7.33333,11.511);
\draw [color=c, fill=c] (7.33333,11.4052) rectangle (7.37313,11.511);
\draw [color=c, fill=c] (7.37313,11.4052) rectangle (7.41294,11.511);
\draw [color=c, fill=c] (7.41294,11.4052) rectangle (7.45274,11.511);
\draw [color=c, fill=c] (7.45274,11.4052) rectangle (7.49254,11.511);
\draw [color=c, fill=c] (7.49254,11.4052) rectangle (7.53234,11.511);
\draw [color=c, fill=c] (7.53234,11.4052) rectangle (7.57214,11.511);
\draw [color=c, fill=c] (7.57214,11.4052) rectangle (7.61194,11.511);
\draw [color=c, fill=c] (7.61194,11.4052) rectangle (7.65174,11.511);
\draw [color=c, fill=c] (7.65174,11.4052) rectangle (7.69154,11.511);
\draw [color=c, fill=c] (7.69154,11.4052) rectangle (7.73134,11.511);
\draw [color=c, fill=c] (7.73134,11.4052) rectangle (7.77114,11.511);
\draw [color=c, fill=c] (7.77114,11.4052) rectangle (7.81095,11.511);
\draw [color=c, fill=c] (7.81095,11.4052) rectangle (7.85075,11.511);
\draw [color=c, fill=c] (7.85075,11.4052) rectangle (7.89055,11.511);
\draw [color=c, fill=c] (7.89055,11.4052) rectangle (7.93035,11.511);
\draw [color=c, fill=c] (7.93035,11.4052) rectangle (7.97015,11.511);
\draw [color=c, fill=c] (7.97015,11.4052) rectangle (8.00995,11.511);
\draw [color=c, fill=c] (8.00995,11.4052) rectangle (8.04975,11.511);
\definecolor{c}{rgb}{0,0.0800001,1};
\draw [color=c, fill=c] (8.04975,11.4052) rectangle (8.08955,11.511);
\draw [color=c, fill=c] (8.08955,11.4052) rectangle (8.12935,11.511);
\draw [color=c, fill=c] (8.12935,11.4052) rectangle (8.16915,11.511);
\draw [color=c, fill=c] (8.16915,11.4052) rectangle (8.20895,11.511);
\draw [color=c, fill=c] (8.20895,11.4052) rectangle (8.24876,11.511);
\draw [color=c, fill=c] (8.24876,11.4052) rectangle (8.28856,11.511);
\draw [color=c, fill=c] (8.28856,11.4052) rectangle (8.32836,11.511);
\draw [color=c, fill=c] (8.32836,11.4052) rectangle (8.36816,11.511);
\draw [color=c, fill=c] (8.36816,11.4052) rectangle (8.40796,11.511);
\draw [color=c, fill=c] (8.40796,11.4052) rectangle (8.44776,11.511);
\draw [color=c, fill=c] (8.44776,11.4052) rectangle (8.48756,11.511);
\draw [color=c, fill=c] (8.48756,11.4052) rectangle (8.52736,11.511);
\draw [color=c, fill=c] (8.52736,11.4052) rectangle (8.56716,11.511);
\draw [color=c, fill=c] (8.56716,11.4052) rectangle (8.60697,11.511);
\draw [color=c, fill=c] (8.60697,11.4052) rectangle (8.64677,11.511);
\draw [color=c, fill=c] (8.64677,11.4052) rectangle (8.68657,11.511);
\draw [color=c, fill=c] (8.68657,11.4052) rectangle (8.72637,11.511);
\draw [color=c, fill=c] (8.72637,11.4052) rectangle (8.76617,11.511);
\draw [color=c, fill=c] (8.76617,11.4052) rectangle (8.80597,11.511);
\draw [color=c, fill=c] (8.80597,11.4052) rectangle (8.84577,11.511);
\draw [color=c, fill=c] (8.84577,11.4052) rectangle (8.88557,11.511);
\draw [color=c, fill=c] (8.88557,11.4052) rectangle (8.92537,11.511);
\draw [color=c, fill=c] (8.92537,11.4052) rectangle (8.96517,11.511);
\draw [color=c, fill=c] (8.96517,11.4052) rectangle (9.00498,11.511);
\draw [color=c, fill=c] (9.00498,11.4052) rectangle (9.04478,11.511);
\draw [color=c, fill=c] (9.04478,11.4052) rectangle (9.08458,11.511);
\draw [color=c, fill=c] (9.08458,11.4052) rectangle (9.12438,11.511);
\draw [color=c, fill=c] (9.12438,11.4052) rectangle (9.16418,11.511);
\draw [color=c, fill=c] (9.16418,11.4052) rectangle (9.20398,11.511);
\draw [color=c, fill=c] (9.20398,11.4052) rectangle (9.24378,11.511);
\draw [color=c, fill=c] (9.24378,11.4052) rectangle (9.28358,11.511);
\draw [color=c, fill=c] (9.28358,11.4052) rectangle (9.32338,11.511);
\draw [color=c, fill=c] (9.32338,11.4052) rectangle (9.36318,11.511);
\draw [color=c, fill=c] (9.36318,11.4052) rectangle (9.40298,11.511);
\draw [color=c, fill=c] (9.40298,11.4052) rectangle (9.44279,11.511);
\draw [color=c, fill=c] (9.44279,11.4052) rectangle (9.48259,11.511);
\draw [color=c, fill=c] (9.48259,11.4052) rectangle (9.52239,11.511);
\draw [color=c, fill=c] (9.52239,11.4052) rectangle (9.56219,11.511);
\draw [color=c, fill=c] (9.56219,11.4052) rectangle (9.60199,11.511);
\draw [color=c, fill=c] (9.60199,11.4052) rectangle (9.64179,11.511);
\definecolor{c}{rgb}{0,0.266667,1};
\draw [color=c, fill=c] (9.64179,11.4052) rectangle (9.68159,11.511);
\draw [color=c, fill=c] (9.68159,11.4052) rectangle (9.72139,11.511);
\draw [color=c, fill=c] (9.72139,11.4052) rectangle (9.76119,11.511);
\draw [color=c, fill=c] (9.76119,11.4052) rectangle (9.80099,11.511);
\draw [color=c, fill=c] (9.80099,11.4052) rectangle (9.8408,11.511);
\draw [color=c, fill=c] (9.8408,11.4052) rectangle (9.8806,11.511);
\draw [color=c, fill=c] (9.8806,11.4052) rectangle (9.9204,11.511);
\draw [color=c, fill=c] (9.9204,11.4052) rectangle (9.9602,11.511);
\draw [color=c, fill=c] (9.9602,11.4052) rectangle (10,11.511);
\draw [color=c, fill=c] (10,11.4052) rectangle (10.0398,11.511);
\draw [color=c, fill=c] (10.0398,11.4052) rectangle (10.0796,11.511);
\draw [color=c, fill=c] (10.0796,11.4052) rectangle (10.1194,11.511);
\draw [color=c, fill=c] (10.1194,11.4052) rectangle (10.1592,11.511);
\draw [color=c, fill=c] (10.1592,11.4052) rectangle (10.199,11.511);
\draw [color=c, fill=c] (10.199,11.4052) rectangle (10.2388,11.511);
\draw [color=c, fill=c] (10.2388,11.4052) rectangle (10.2786,11.511);
\draw [color=c, fill=c] (10.2786,11.4052) rectangle (10.3184,11.511);
\draw [color=c, fill=c] (10.3184,11.4052) rectangle (10.3582,11.511);
\draw [color=c, fill=c] (10.3582,11.4052) rectangle (10.398,11.511);
\draw [color=c, fill=c] (10.398,11.4052) rectangle (10.4378,11.511);
\draw [color=c, fill=c] (10.4378,11.4052) rectangle (10.4776,11.511);
\draw [color=c, fill=c] (10.4776,11.4052) rectangle (10.5174,11.511);
\draw [color=c, fill=c] (10.5174,11.4052) rectangle (10.5572,11.511);
\draw [color=c, fill=c] (10.5572,11.4052) rectangle (10.597,11.511);
\draw [color=c, fill=c] (10.597,11.4052) rectangle (10.6368,11.511);
\draw [color=c, fill=c] (10.6368,11.4052) rectangle (10.6766,11.511);
\draw [color=c, fill=c] (10.6766,11.4052) rectangle (10.7164,11.511);
\draw [color=c, fill=c] (10.7164,11.4052) rectangle (10.7562,11.511);
\draw [color=c, fill=c] (10.7562,11.4052) rectangle (10.796,11.511);
\draw [color=c, fill=c] (10.796,11.4052) rectangle (10.8358,11.511);
\draw [color=c, fill=c] (10.8358,11.4052) rectangle (10.8756,11.511);
\draw [color=c, fill=c] (10.8756,11.4052) rectangle (10.9154,11.511);
\draw [color=c, fill=c] (10.9154,11.4052) rectangle (10.9552,11.511);
\draw [color=c, fill=c] (10.9552,11.4052) rectangle (10.995,11.511);
\definecolor{c}{rgb}{0,0.546666,1};
\draw [color=c, fill=c] (10.995,11.4052) rectangle (11.0348,11.511);
\draw [color=c, fill=c] (11.0348,11.4052) rectangle (11.0746,11.511);
\draw [color=c, fill=c] (11.0746,11.4052) rectangle (11.1144,11.511);
\draw [color=c, fill=c] (11.1144,11.4052) rectangle (11.1542,11.511);
\draw [color=c, fill=c] (11.1542,11.4052) rectangle (11.194,11.511);
\draw [color=c, fill=c] (11.194,11.4052) rectangle (11.2338,11.511);
\draw [color=c, fill=c] (11.2338,11.4052) rectangle (11.2736,11.511);
\draw [color=c, fill=c] (11.2736,11.4052) rectangle (11.3134,11.511);
\draw [color=c, fill=c] (11.3134,11.4052) rectangle (11.3532,11.511);
\draw [color=c, fill=c] (11.3532,11.4052) rectangle (11.393,11.511);
\draw [color=c, fill=c] (11.393,11.4052) rectangle (11.4328,11.511);
\draw [color=c, fill=c] (11.4328,11.4052) rectangle (11.4726,11.511);
\draw [color=c, fill=c] (11.4726,11.4052) rectangle (11.5124,11.511);
\draw [color=c, fill=c] (11.5124,11.4052) rectangle (11.5522,11.511);
\draw [color=c, fill=c] (11.5522,11.4052) rectangle (11.592,11.511);
\draw [color=c, fill=c] (11.592,11.4052) rectangle (11.6318,11.511);
\draw [color=c, fill=c] (11.6318,11.4052) rectangle (11.6716,11.511);
\draw [color=c, fill=c] (11.6716,11.4052) rectangle (11.7114,11.511);
\draw [color=c, fill=c] (11.7114,11.4052) rectangle (11.7512,11.511);
\draw [color=c, fill=c] (11.7512,11.4052) rectangle (11.791,11.511);
\draw [color=c, fill=c] (11.791,11.4052) rectangle (11.8308,11.511);
\draw [color=c, fill=c] (11.8308,11.4052) rectangle (11.8706,11.511);
\draw [color=c, fill=c] (11.8706,11.4052) rectangle (11.9104,11.511);
\draw [color=c, fill=c] (11.9104,11.4052) rectangle (11.9502,11.511);
\draw [color=c, fill=c] (11.9502,11.4052) rectangle (11.99,11.511);
\draw [color=c, fill=c] (11.99,11.4052) rectangle (12.0299,11.511);
\draw [color=c, fill=c] (12.0299,11.4052) rectangle (12.0697,11.511);
\draw [color=c, fill=c] (12.0697,11.4052) rectangle (12.1095,11.511);
\draw [color=c, fill=c] (12.1095,11.4052) rectangle (12.1493,11.511);
\draw [color=c, fill=c] (12.1493,11.4052) rectangle (12.1891,11.511);
\draw [color=c, fill=c] (12.1891,11.4052) rectangle (12.2289,11.511);
\draw [color=c, fill=c] (12.2289,11.4052) rectangle (12.2687,11.511);
\draw [color=c, fill=c] (12.2687,11.4052) rectangle (12.3085,11.511);
\draw [color=c, fill=c] (12.3085,11.4052) rectangle (12.3483,11.511);
\draw [color=c, fill=c] (12.3483,11.4052) rectangle (12.3881,11.511);
\draw [color=c, fill=c] (12.3881,11.4052) rectangle (12.4279,11.511);
\draw [color=c, fill=c] (12.4279,11.4052) rectangle (12.4677,11.511);
\draw [color=c, fill=c] (12.4677,11.4052) rectangle (12.5075,11.511);
\draw [color=c, fill=c] (12.5075,11.4052) rectangle (12.5473,11.511);
\draw [color=c, fill=c] (12.5473,11.4052) rectangle (12.5871,11.511);
\draw [color=c, fill=c] (12.5871,11.4052) rectangle (12.6269,11.511);
\draw [color=c, fill=c] (12.6269,11.4052) rectangle (12.6667,11.511);
\draw [color=c, fill=c] (12.6667,11.4052) rectangle (12.7065,11.511);
\draw [color=c, fill=c] (12.7065,11.4052) rectangle (12.7463,11.511);
\draw [color=c, fill=c] (12.7463,11.4052) rectangle (12.7861,11.511);
\draw [color=c, fill=c] (12.7861,11.4052) rectangle (12.8259,11.511);
\draw [color=c, fill=c] (12.8259,11.4052) rectangle (12.8657,11.511);
\draw [color=c, fill=c] (12.8657,11.4052) rectangle (12.9055,11.511);
\draw [color=c, fill=c] (12.9055,11.4052) rectangle (12.9453,11.511);
\draw [color=c, fill=c] (12.9453,11.4052) rectangle (12.9851,11.511);
\draw [color=c, fill=c] (12.9851,11.4052) rectangle (13.0249,11.511);
\draw [color=c, fill=c] (13.0249,11.4052) rectangle (13.0647,11.511);
\draw [color=c, fill=c] (13.0647,11.4052) rectangle (13.1045,11.511);
\draw [color=c, fill=c] (13.1045,11.4052) rectangle (13.1443,11.511);
\draw [color=c, fill=c] (13.1443,11.4052) rectangle (13.1841,11.511);
\draw [color=c, fill=c] (13.1841,11.4052) rectangle (13.2239,11.511);
\draw [color=c, fill=c] (13.2239,11.4052) rectangle (13.2637,11.511);
\draw [color=c, fill=c] (13.2637,11.4052) rectangle (13.3035,11.511);
\draw [color=c, fill=c] (13.3035,11.4052) rectangle (13.3433,11.511);
\draw [color=c, fill=c] (13.3433,11.4052) rectangle (13.3831,11.511);
\draw [color=c, fill=c] (13.3831,11.4052) rectangle (13.4229,11.511);
\draw [color=c, fill=c] (13.4229,11.4052) rectangle (13.4627,11.511);
\draw [color=c, fill=c] (13.4627,11.4052) rectangle (13.5025,11.511);
\draw [color=c, fill=c] (13.5025,11.4052) rectangle (13.5423,11.511);
\draw [color=c, fill=c] (13.5423,11.4052) rectangle (13.5821,11.511);
\draw [color=c, fill=c] (13.5821,11.4052) rectangle (13.6219,11.511);
\draw [color=c, fill=c] (13.6219,11.4052) rectangle (13.6617,11.511);
\draw [color=c, fill=c] (13.6617,11.4052) rectangle (13.7015,11.511);
\draw [color=c, fill=c] (13.7015,11.4052) rectangle (13.7413,11.511);
\draw [color=c, fill=c] (13.7413,11.4052) rectangle (13.7811,11.511);
\draw [color=c, fill=c] (13.7811,11.4052) rectangle (13.8209,11.511);
\draw [color=c, fill=c] (13.8209,11.4052) rectangle (13.8607,11.511);
\draw [color=c, fill=c] (13.8607,11.4052) rectangle (13.9005,11.511);
\draw [color=c, fill=c] (13.9005,11.4052) rectangle (13.9403,11.511);
\draw [color=c, fill=c] (13.9403,11.4052) rectangle (13.9801,11.511);
\draw [color=c, fill=c] (13.9801,11.4052) rectangle (14.0199,11.511);
\draw [color=c, fill=c] (14.0199,11.4052) rectangle (14.0597,11.511);
\draw [color=c, fill=c] (14.0597,11.4052) rectangle (14.0995,11.511);
\draw [color=c, fill=c] (14.0995,11.4052) rectangle (14.1393,11.511);
\draw [color=c, fill=c] (14.1393,11.4052) rectangle (14.1791,11.511);
\draw [color=c, fill=c] (14.1791,11.4052) rectangle (14.2189,11.511);
\draw [color=c, fill=c] (14.2189,11.4052) rectangle (14.2587,11.511);
\draw [color=c, fill=c] (14.2587,11.4052) rectangle (14.2985,11.511);
\draw [color=c, fill=c] (14.2985,11.4052) rectangle (14.3383,11.511);
\draw [color=c, fill=c] (14.3383,11.4052) rectangle (14.3781,11.511);
\draw [color=c, fill=c] (14.3781,11.4052) rectangle (14.4179,11.511);
\draw [color=c, fill=c] (14.4179,11.4052) rectangle (14.4577,11.511);
\definecolor{c}{rgb}{0,0.733333,1};
\draw [color=c, fill=c] (14.4577,11.4052) rectangle (14.4975,11.511);
\draw [color=c, fill=c] (14.4975,11.4052) rectangle (14.5373,11.511);
\draw [color=c, fill=c] (14.5373,11.4052) rectangle (14.5771,11.511);
\draw [color=c, fill=c] (14.5771,11.4052) rectangle (14.6169,11.511);
\draw [color=c, fill=c] (14.6169,11.4052) rectangle (14.6567,11.511);
\draw [color=c, fill=c] (14.6567,11.4052) rectangle (14.6965,11.511);
\draw [color=c, fill=c] (14.6965,11.4052) rectangle (14.7363,11.511);
\draw [color=c, fill=c] (14.7363,11.4052) rectangle (14.7761,11.511);
\draw [color=c, fill=c] (14.7761,11.4052) rectangle (14.8159,11.511);
\draw [color=c, fill=c] (14.8159,11.4052) rectangle (14.8557,11.511);
\draw [color=c, fill=c] (14.8557,11.4052) rectangle (14.8955,11.511);
\draw [color=c, fill=c] (14.8955,11.4052) rectangle (14.9353,11.511);
\draw [color=c, fill=c] (14.9353,11.4052) rectangle (14.9751,11.511);
\draw [color=c, fill=c] (14.9751,11.4052) rectangle (15.0149,11.511);
\draw [color=c, fill=c] (15.0149,11.4052) rectangle (15.0547,11.511);
\draw [color=c, fill=c] (15.0547,11.4052) rectangle (15.0945,11.511);
\draw [color=c, fill=c] (15.0945,11.4052) rectangle (15.1343,11.511);
\draw [color=c, fill=c] (15.1343,11.4052) rectangle (15.1741,11.511);
\draw [color=c, fill=c] (15.1741,11.4052) rectangle (15.2139,11.511);
\draw [color=c, fill=c] (15.2139,11.4052) rectangle (15.2537,11.511);
\draw [color=c, fill=c] (15.2537,11.4052) rectangle (15.2935,11.511);
\draw [color=c, fill=c] (15.2935,11.4052) rectangle (15.3333,11.511);
\draw [color=c, fill=c] (15.3333,11.4052) rectangle (15.3731,11.511);
\draw [color=c, fill=c] (15.3731,11.4052) rectangle (15.4129,11.511);
\draw [color=c, fill=c] (15.4129,11.4052) rectangle (15.4527,11.511);
\draw [color=c, fill=c] (15.4527,11.4052) rectangle (15.4925,11.511);
\draw [color=c, fill=c] (15.4925,11.4052) rectangle (15.5323,11.511);
\draw [color=c, fill=c] (15.5323,11.4052) rectangle (15.5721,11.511);
\draw [color=c, fill=c] (15.5721,11.4052) rectangle (15.6119,11.511);
\draw [color=c, fill=c] (15.6119,11.4052) rectangle (15.6517,11.511);
\draw [color=c, fill=c] (15.6517,11.4052) rectangle (15.6915,11.511);
\draw [color=c, fill=c] (15.6915,11.4052) rectangle (15.7313,11.511);
\draw [color=c, fill=c] (15.7313,11.4052) rectangle (15.7711,11.511);
\draw [color=c, fill=c] (15.7711,11.4052) rectangle (15.8109,11.511);
\draw [color=c, fill=c] (15.8109,11.4052) rectangle (15.8507,11.511);
\draw [color=c, fill=c] (15.8507,11.4052) rectangle (15.8905,11.511);
\draw [color=c, fill=c] (15.8905,11.4052) rectangle (15.9303,11.511);
\draw [color=c, fill=c] (15.9303,11.4052) rectangle (15.9701,11.511);
\draw [color=c, fill=c] (15.9701,11.4052) rectangle (16.01,11.511);
\draw [color=c, fill=c] (16.01,11.4052) rectangle (16.0498,11.511);
\draw [color=c, fill=c] (16.0498,11.4052) rectangle (16.0896,11.511);
\draw [color=c, fill=c] (16.0896,11.4052) rectangle (16.1294,11.511);
\draw [color=c, fill=c] (16.1294,11.4052) rectangle (16.1692,11.511);
\draw [color=c, fill=c] (16.1692,11.4052) rectangle (16.209,11.511);
\draw [color=c, fill=c] (16.209,11.4052) rectangle (16.2488,11.511);
\draw [color=c, fill=c] (16.2488,11.4052) rectangle (16.2886,11.511);
\draw [color=c, fill=c] (16.2886,11.4052) rectangle (16.3284,11.511);
\draw [color=c, fill=c] (16.3284,11.4052) rectangle (16.3682,11.511);
\draw [color=c, fill=c] (16.3682,11.4052) rectangle (16.408,11.511);
\draw [color=c, fill=c] (16.408,11.4052) rectangle (16.4478,11.511);
\draw [color=c, fill=c] (16.4478,11.4052) rectangle (16.4876,11.511);
\draw [color=c, fill=c] (16.4876,11.4052) rectangle (16.5274,11.511);
\draw [color=c, fill=c] (16.5274,11.4052) rectangle (16.5672,11.511);
\draw [color=c, fill=c] (16.5672,11.4052) rectangle (16.607,11.511);
\draw [color=c, fill=c] (16.607,11.4052) rectangle (16.6468,11.511);
\draw [color=c, fill=c] (16.6468,11.4052) rectangle (16.6866,11.511);
\draw [color=c, fill=c] (16.6866,11.4052) rectangle (16.7264,11.511);
\draw [color=c, fill=c] (16.7264,11.4052) rectangle (16.7662,11.511);
\draw [color=c, fill=c] (16.7662,11.4052) rectangle (16.806,11.511);
\draw [color=c, fill=c] (16.806,11.4052) rectangle (16.8458,11.511);
\draw [color=c, fill=c] (16.8458,11.4052) rectangle (16.8856,11.511);
\draw [color=c, fill=c] (16.8856,11.4052) rectangle (16.9254,11.511);
\draw [color=c, fill=c] (16.9254,11.4052) rectangle (16.9652,11.511);
\draw [color=c, fill=c] (16.9652,11.4052) rectangle (17.005,11.511);
\draw [color=c, fill=c] (17.005,11.4052) rectangle (17.0448,11.511);
\draw [color=c, fill=c] (17.0448,11.4052) rectangle (17.0846,11.511);
\draw [color=c, fill=c] (17.0846,11.4052) rectangle (17.1244,11.511);
\draw [color=c, fill=c] (17.1244,11.4052) rectangle (17.1642,11.511);
\draw [color=c, fill=c] (17.1642,11.4052) rectangle (17.204,11.511);
\draw [color=c, fill=c] (17.204,11.4052) rectangle (17.2438,11.511);
\draw [color=c, fill=c] (17.2438,11.4052) rectangle (17.2836,11.511);
\draw [color=c, fill=c] (17.2836,11.4052) rectangle (17.3234,11.511);
\draw [color=c, fill=c] (17.3234,11.4052) rectangle (17.3632,11.511);
\draw [color=c, fill=c] (17.3632,11.4052) rectangle (17.403,11.511);
\draw [color=c, fill=c] (17.403,11.4052) rectangle (17.4428,11.511);
\draw [color=c, fill=c] (17.4428,11.4052) rectangle (17.4826,11.511);
\draw [color=c, fill=c] (17.4826,11.4052) rectangle (17.5224,11.511);
\draw [color=c, fill=c] (17.5224,11.4052) rectangle (17.5622,11.511);
\draw [color=c, fill=c] (17.5622,11.4052) rectangle (17.602,11.511);
\draw [color=c, fill=c] (17.602,11.4052) rectangle (17.6418,11.511);
\draw [color=c, fill=c] (17.6418,11.4052) rectangle (17.6816,11.511);
\draw [color=c, fill=c] (17.6816,11.4052) rectangle (17.7214,11.511);
\draw [color=c, fill=c] (17.7214,11.4052) rectangle (17.7612,11.511);
\draw [color=c, fill=c] (17.7612,11.4052) rectangle (17.801,11.511);
\draw [color=c, fill=c] (17.801,11.4052) rectangle (17.8408,11.511);
\draw [color=c, fill=c] (17.8408,11.4052) rectangle (17.8806,11.511);
\draw [color=c, fill=c] (17.8806,11.4052) rectangle (17.9204,11.511);
\draw [color=c, fill=c] (17.9204,11.4052) rectangle (17.9602,11.511);
\draw [color=c, fill=c] (17.9602,11.4052) rectangle (18,11.511);
\definecolor{c}{rgb}{0.2,0,1};
\draw [color=c, fill=c] (2,11.511) rectangle (2.0398,11.6169);
\draw [color=c, fill=c] (2.0398,11.511) rectangle (2.0796,11.6169);
\draw [color=c, fill=c] (2.0796,11.511) rectangle (2.1194,11.6169);
\draw [color=c, fill=c] (2.1194,11.511) rectangle (2.1592,11.6169);
\draw [color=c, fill=c] (2.1592,11.511) rectangle (2.19901,11.6169);
\draw [color=c, fill=c] (2.19901,11.511) rectangle (2.23881,11.6169);
\draw [color=c, fill=c] (2.23881,11.511) rectangle (2.27861,11.6169);
\draw [color=c, fill=c] (2.27861,11.511) rectangle (2.31841,11.6169);
\draw [color=c, fill=c] (2.31841,11.511) rectangle (2.35821,11.6169);
\draw [color=c, fill=c] (2.35821,11.511) rectangle (2.39801,11.6169);
\draw [color=c, fill=c] (2.39801,11.511) rectangle (2.43781,11.6169);
\draw [color=c, fill=c] (2.43781,11.511) rectangle (2.47761,11.6169);
\draw [color=c, fill=c] (2.47761,11.511) rectangle (2.51741,11.6169);
\draw [color=c, fill=c] (2.51741,11.511) rectangle (2.55721,11.6169);
\draw [color=c, fill=c] (2.55721,11.511) rectangle (2.59702,11.6169);
\draw [color=c, fill=c] (2.59702,11.511) rectangle (2.63682,11.6169);
\draw [color=c, fill=c] (2.63682,11.511) rectangle (2.67662,11.6169);
\draw [color=c, fill=c] (2.67662,11.511) rectangle (2.71642,11.6169);
\draw [color=c, fill=c] (2.71642,11.511) rectangle (2.75622,11.6169);
\draw [color=c, fill=c] (2.75622,11.511) rectangle (2.79602,11.6169);
\draw [color=c, fill=c] (2.79602,11.511) rectangle (2.83582,11.6169);
\draw [color=c, fill=c] (2.83582,11.511) rectangle (2.87562,11.6169);
\draw [color=c, fill=c] (2.87562,11.511) rectangle (2.91542,11.6169);
\draw [color=c, fill=c] (2.91542,11.511) rectangle (2.95522,11.6169);
\draw [color=c, fill=c] (2.95522,11.511) rectangle (2.99502,11.6169);
\draw [color=c, fill=c] (2.99502,11.511) rectangle (3.03483,11.6169);
\draw [color=c, fill=c] (3.03483,11.511) rectangle (3.07463,11.6169);
\draw [color=c, fill=c] (3.07463,11.511) rectangle (3.11443,11.6169);
\draw [color=c, fill=c] (3.11443,11.511) rectangle (3.15423,11.6169);
\draw [color=c, fill=c] (3.15423,11.511) rectangle (3.19403,11.6169);
\draw [color=c, fill=c] (3.19403,11.511) rectangle (3.23383,11.6169);
\draw [color=c, fill=c] (3.23383,11.511) rectangle (3.27363,11.6169);
\draw [color=c, fill=c] (3.27363,11.511) rectangle (3.31343,11.6169);
\draw [color=c, fill=c] (3.31343,11.511) rectangle (3.35323,11.6169);
\draw [color=c, fill=c] (3.35323,11.511) rectangle (3.39303,11.6169);
\draw [color=c, fill=c] (3.39303,11.511) rectangle (3.43284,11.6169);
\draw [color=c, fill=c] (3.43284,11.511) rectangle (3.47264,11.6169);
\draw [color=c, fill=c] (3.47264,11.511) rectangle (3.51244,11.6169);
\draw [color=c, fill=c] (3.51244,11.511) rectangle (3.55224,11.6169);
\draw [color=c, fill=c] (3.55224,11.511) rectangle (3.59204,11.6169);
\draw [color=c, fill=c] (3.59204,11.511) rectangle (3.63184,11.6169);
\draw [color=c, fill=c] (3.63184,11.511) rectangle (3.67164,11.6169);
\draw [color=c, fill=c] (3.67164,11.511) rectangle (3.71144,11.6169);
\draw [color=c, fill=c] (3.71144,11.511) rectangle (3.75124,11.6169);
\draw [color=c, fill=c] (3.75124,11.511) rectangle (3.79104,11.6169);
\draw [color=c, fill=c] (3.79104,11.511) rectangle (3.83085,11.6169);
\draw [color=c, fill=c] (3.83085,11.511) rectangle (3.87065,11.6169);
\draw [color=c, fill=c] (3.87065,11.511) rectangle (3.91045,11.6169);
\draw [color=c, fill=c] (3.91045,11.511) rectangle (3.95025,11.6169);
\draw [color=c, fill=c] (3.95025,11.511) rectangle (3.99005,11.6169);
\draw [color=c, fill=c] (3.99005,11.511) rectangle (4.02985,11.6169);
\draw [color=c, fill=c] (4.02985,11.511) rectangle (4.06965,11.6169);
\draw [color=c, fill=c] (4.06965,11.511) rectangle (4.10945,11.6169);
\draw [color=c, fill=c] (4.10945,11.511) rectangle (4.14925,11.6169);
\draw [color=c, fill=c] (4.14925,11.511) rectangle (4.18905,11.6169);
\draw [color=c, fill=c] (4.18905,11.511) rectangle (4.22886,11.6169);
\draw [color=c, fill=c] (4.22886,11.511) rectangle (4.26866,11.6169);
\draw [color=c, fill=c] (4.26866,11.511) rectangle (4.30846,11.6169);
\draw [color=c, fill=c] (4.30846,11.511) rectangle (4.34826,11.6169);
\draw [color=c, fill=c] (4.34826,11.511) rectangle (4.38806,11.6169);
\draw [color=c, fill=c] (4.38806,11.511) rectangle (4.42786,11.6169);
\draw [color=c, fill=c] (4.42786,11.511) rectangle (4.46766,11.6169);
\draw [color=c, fill=c] (4.46766,11.511) rectangle (4.50746,11.6169);
\draw [color=c, fill=c] (4.50746,11.511) rectangle (4.54726,11.6169);
\draw [color=c, fill=c] (4.54726,11.511) rectangle (4.58706,11.6169);
\draw [color=c, fill=c] (4.58706,11.511) rectangle (4.62687,11.6169);
\draw [color=c, fill=c] (4.62687,11.511) rectangle (4.66667,11.6169);
\draw [color=c, fill=c] (4.66667,11.511) rectangle (4.70647,11.6169);
\draw [color=c, fill=c] (4.70647,11.511) rectangle (4.74627,11.6169);
\draw [color=c, fill=c] (4.74627,11.511) rectangle (4.78607,11.6169);
\draw [color=c, fill=c] (4.78607,11.511) rectangle (4.82587,11.6169);
\draw [color=c, fill=c] (4.82587,11.511) rectangle (4.86567,11.6169);
\draw [color=c, fill=c] (4.86567,11.511) rectangle (4.90547,11.6169);
\draw [color=c, fill=c] (4.90547,11.511) rectangle (4.94527,11.6169);
\draw [color=c, fill=c] (4.94527,11.511) rectangle (4.98507,11.6169);
\draw [color=c, fill=c] (4.98507,11.511) rectangle (5.02488,11.6169);
\draw [color=c, fill=c] (5.02488,11.511) rectangle (5.06468,11.6169);
\draw [color=c, fill=c] (5.06468,11.511) rectangle (5.10448,11.6169);
\draw [color=c, fill=c] (5.10448,11.511) rectangle (5.14428,11.6169);
\draw [color=c, fill=c] (5.14428,11.511) rectangle (5.18408,11.6169);
\draw [color=c, fill=c] (5.18408,11.511) rectangle (5.22388,11.6169);
\draw [color=c, fill=c] (5.22388,11.511) rectangle (5.26368,11.6169);
\draw [color=c, fill=c] (5.26368,11.511) rectangle (5.30348,11.6169);
\draw [color=c, fill=c] (5.30348,11.511) rectangle (5.34328,11.6169);
\draw [color=c, fill=c] (5.34328,11.511) rectangle (5.38308,11.6169);
\draw [color=c, fill=c] (5.38308,11.511) rectangle (5.42289,11.6169);
\draw [color=c, fill=c] (5.42289,11.511) rectangle (5.46269,11.6169);
\draw [color=c, fill=c] (5.46269,11.511) rectangle (5.50249,11.6169);
\draw [color=c, fill=c] (5.50249,11.511) rectangle (5.54229,11.6169);
\draw [color=c, fill=c] (5.54229,11.511) rectangle (5.58209,11.6169);
\draw [color=c, fill=c] (5.58209,11.511) rectangle (5.62189,11.6169);
\draw [color=c, fill=c] (5.62189,11.511) rectangle (5.66169,11.6169);
\draw [color=c, fill=c] (5.66169,11.511) rectangle (5.70149,11.6169);
\draw [color=c, fill=c] (5.70149,11.511) rectangle (5.74129,11.6169);
\draw [color=c, fill=c] (5.74129,11.511) rectangle (5.78109,11.6169);
\draw [color=c, fill=c] (5.78109,11.511) rectangle (5.8209,11.6169);
\draw [color=c, fill=c] (5.8209,11.511) rectangle (5.8607,11.6169);
\draw [color=c, fill=c] (5.8607,11.511) rectangle (5.9005,11.6169);
\draw [color=c, fill=c] (5.9005,11.511) rectangle (5.9403,11.6169);
\draw [color=c, fill=c] (5.9403,11.511) rectangle (5.9801,11.6169);
\draw [color=c, fill=c] (5.9801,11.511) rectangle (6.0199,11.6169);
\draw [color=c, fill=c] (6.0199,11.511) rectangle (6.0597,11.6169);
\draw [color=c, fill=c] (6.0597,11.511) rectangle (6.0995,11.6169);
\draw [color=c, fill=c] (6.0995,11.511) rectangle (6.1393,11.6169);
\draw [color=c, fill=c] (6.1393,11.511) rectangle (6.1791,11.6169);
\draw [color=c, fill=c] (6.1791,11.511) rectangle (6.21891,11.6169);
\draw [color=c, fill=c] (6.21891,11.511) rectangle (6.25871,11.6169);
\draw [color=c, fill=c] (6.25871,11.511) rectangle (6.29851,11.6169);
\draw [color=c, fill=c] (6.29851,11.511) rectangle (6.33831,11.6169);
\draw [color=c, fill=c] (6.33831,11.511) rectangle (6.37811,11.6169);
\draw [color=c, fill=c] (6.37811,11.511) rectangle (6.41791,11.6169);
\draw [color=c, fill=c] (6.41791,11.511) rectangle (6.45771,11.6169);
\draw [color=c, fill=c] (6.45771,11.511) rectangle (6.49751,11.6169);
\draw [color=c, fill=c] (6.49751,11.511) rectangle (6.53731,11.6169);
\draw [color=c, fill=c] (6.53731,11.511) rectangle (6.57711,11.6169);
\draw [color=c, fill=c] (6.57711,11.511) rectangle (6.61692,11.6169);
\draw [color=c, fill=c] (6.61692,11.511) rectangle (6.65672,11.6169);
\draw [color=c, fill=c] (6.65672,11.511) rectangle (6.69652,11.6169);
\draw [color=c, fill=c] (6.69652,11.511) rectangle (6.73632,11.6169);
\draw [color=c, fill=c] (6.73632,11.511) rectangle (6.77612,11.6169);
\draw [color=c, fill=c] (6.77612,11.511) rectangle (6.81592,11.6169);
\draw [color=c, fill=c] (6.81592,11.511) rectangle (6.85572,11.6169);
\draw [color=c, fill=c] (6.85572,11.511) rectangle (6.89552,11.6169);
\draw [color=c, fill=c] (6.89552,11.511) rectangle (6.93532,11.6169);
\draw [color=c, fill=c] (6.93532,11.511) rectangle (6.97512,11.6169);
\draw [color=c, fill=c] (6.97512,11.511) rectangle (7.01493,11.6169);
\draw [color=c, fill=c] (7.01493,11.511) rectangle (7.05473,11.6169);
\draw [color=c, fill=c] (7.05473,11.511) rectangle (7.09453,11.6169);
\draw [color=c, fill=c] (7.09453,11.511) rectangle (7.13433,11.6169);
\draw [color=c, fill=c] (7.13433,11.511) rectangle (7.17413,11.6169);
\draw [color=c, fill=c] (7.17413,11.511) rectangle (7.21393,11.6169);
\draw [color=c, fill=c] (7.21393,11.511) rectangle (7.25373,11.6169);
\draw [color=c, fill=c] (7.25373,11.511) rectangle (7.29353,11.6169);
\draw [color=c, fill=c] (7.29353,11.511) rectangle (7.33333,11.6169);
\draw [color=c, fill=c] (7.33333,11.511) rectangle (7.37313,11.6169);
\draw [color=c, fill=c] (7.37313,11.511) rectangle (7.41294,11.6169);
\draw [color=c, fill=c] (7.41294,11.511) rectangle (7.45274,11.6169);
\draw [color=c, fill=c] (7.45274,11.511) rectangle (7.49254,11.6169);
\draw [color=c, fill=c] (7.49254,11.511) rectangle (7.53234,11.6169);
\draw [color=c, fill=c] (7.53234,11.511) rectangle (7.57214,11.6169);
\draw [color=c, fill=c] (7.57214,11.511) rectangle (7.61194,11.6169);
\draw [color=c, fill=c] (7.61194,11.511) rectangle (7.65174,11.6169);
\draw [color=c, fill=c] (7.65174,11.511) rectangle (7.69154,11.6169);
\draw [color=c, fill=c] (7.69154,11.511) rectangle (7.73134,11.6169);
\draw [color=c, fill=c] (7.73134,11.511) rectangle (7.77114,11.6169);
\draw [color=c, fill=c] (7.77114,11.511) rectangle (7.81095,11.6169);
\draw [color=c, fill=c] (7.81095,11.511) rectangle (7.85075,11.6169);
\draw [color=c, fill=c] (7.85075,11.511) rectangle (7.89055,11.6169);
\draw [color=c, fill=c] (7.89055,11.511) rectangle (7.93035,11.6169);
\draw [color=c, fill=c] (7.93035,11.511) rectangle (7.97015,11.6169);
\draw [color=c, fill=c] (7.97015,11.511) rectangle (8.00995,11.6169);
\draw [color=c, fill=c] (8.00995,11.511) rectangle (8.04975,11.6169);
\draw [color=c, fill=c] (8.04975,11.511) rectangle (8.08955,11.6169);
\definecolor{c}{rgb}{0,0.0800001,1};
\draw [color=c, fill=c] (8.08955,11.511) rectangle (8.12935,11.6169);
\draw [color=c, fill=c] (8.12935,11.511) rectangle (8.16915,11.6169);
\draw [color=c, fill=c] (8.16915,11.511) rectangle (8.20895,11.6169);
\draw [color=c, fill=c] (8.20895,11.511) rectangle (8.24876,11.6169);
\draw [color=c, fill=c] (8.24876,11.511) rectangle (8.28856,11.6169);
\draw [color=c, fill=c] (8.28856,11.511) rectangle (8.32836,11.6169);
\draw [color=c, fill=c] (8.32836,11.511) rectangle (8.36816,11.6169);
\draw [color=c, fill=c] (8.36816,11.511) rectangle (8.40796,11.6169);
\draw [color=c, fill=c] (8.40796,11.511) rectangle (8.44776,11.6169);
\draw [color=c, fill=c] (8.44776,11.511) rectangle (8.48756,11.6169);
\draw [color=c, fill=c] (8.48756,11.511) rectangle (8.52736,11.6169);
\draw [color=c, fill=c] (8.52736,11.511) rectangle (8.56716,11.6169);
\draw [color=c, fill=c] (8.56716,11.511) rectangle (8.60697,11.6169);
\draw [color=c, fill=c] (8.60697,11.511) rectangle (8.64677,11.6169);
\draw [color=c, fill=c] (8.64677,11.511) rectangle (8.68657,11.6169);
\draw [color=c, fill=c] (8.68657,11.511) rectangle (8.72637,11.6169);
\draw [color=c, fill=c] (8.72637,11.511) rectangle (8.76617,11.6169);
\draw [color=c, fill=c] (8.76617,11.511) rectangle (8.80597,11.6169);
\draw [color=c, fill=c] (8.80597,11.511) rectangle (8.84577,11.6169);
\draw [color=c, fill=c] (8.84577,11.511) rectangle (8.88557,11.6169);
\draw [color=c, fill=c] (8.88557,11.511) rectangle (8.92537,11.6169);
\draw [color=c, fill=c] (8.92537,11.511) rectangle (8.96517,11.6169);
\draw [color=c, fill=c] (8.96517,11.511) rectangle (9.00498,11.6169);
\draw [color=c, fill=c] (9.00498,11.511) rectangle (9.04478,11.6169);
\draw [color=c, fill=c] (9.04478,11.511) rectangle (9.08458,11.6169);
\draw [color=c, fill=c] (9.08458,11.511) rectangle (9.12438,11.6169);
\draw [color=c, fill=c] (9.12438,11.511) rectangle (9.16418,11.6169);
\draw [color=c, fill=c] (9.16418,11.511) rectangle (9.20398,11.6169);
\draw [color=c, fill=c] (9.20398,11.511) rectangle (9.24378,11.6169);
\draw [color=c, fill=c] (9.24378,11.511) rectangle (9.28358,11.6169);
\draw [color=c, fill=c] (9.28358,11.511) rectangle (9.32338,11.6169);
\draw [color=c, fill=c] (9.32338,11.511) rectangle (9.36318,11.6169);
\draw [color=c, fill=c] (9.36318,11.511) rectangle (9.40298,11.6169);
\draw [color=c, fill=c] (9.40298,11.511) rectangle (9.44279,11.6169);
\draw [color=c, fill=c] (9.44279,11.511) rectangle (9.48259,11.6169);
\draw [color=c, fill=c] (9.48259,11.511) rectangle (9.52239,11.6169);
\draw [color=c, fill=c] (9.52239,11.511) rectangle (9.56219,11.6169);
\draw [color=c, fill=c] (9.56219,11.511) rectangle (9.60199,11.6169);
\draw [color=c, fill=c] (9.60199,11.511) rectangle (9.64179,11.6169);
\draw [color=c, fill=c] (9.64179,11.511) rectangle (9.68159,11.6169);
\definecolor{c}{rgb}{0,0.266667,1};
\draw [color=c, fill=c] (9.68159,11.511) rectangle (9.72139,11.6169);
\draw [color=c, fill=c] (9.72139,11.511) rectangle (9.76119,11.6169);
\draw [color=c, fill=c] (9.76119,11.511) rectangle (9.80099,11.6169);
\draw [color=c, fill=c] (9.80099,11.511) rectangle (9.8408,11.6169);
\draw [color=c, fill=c] (9.8408,11.511) rectangle (9.8806,11.6169);
\draw [color=c, fill=c] (9.8806,11.511) rectangle (9.9204,11.6169);
\draw [color=c, fill=c] (9.9204,11.511) rectangle (9.9602,11.6169);
\draw [color=c, fill=c] (9.9602,11.511) rectangle (10,11.6169);
\draw [color=c, fill=c] (10,11.511) rectangle (10.0398,11.6169);
\draw [color=c, fill=c] (10.0398,11.511) rectangle (10.0796,11.6169);
\draw [color=c, fill=c] (10.0796,11.511) rectangle (10.1194,11.6169);
\draw [color=c, fill=c] (10.1194,11.511) rectangle (10.1592,11.6169);
\draw [color=c, fill=c] (10.1592,11.511) rectangle (10.199,11.6169);
\draw [color=c, fill=c] (10.199,11.511) rectangle (10.2388,11.6169);
\draw [color=c, fill=c] (10.2388,11.511) rectangle (10.2786,11.6169);
\draw [color=c, fill=c] (10.2786,11.511) rectangle (10.3184,11.6169);
\draw [color=c, fill=c] (10.3184,11.511) rectangle (10.3582,11.6169);
\draw [color=c, fill=c] (10.3582,11.511) rectangle (10.398,11.6169);
\draw [color=c, fill=c] (10.398,11.511) rectangle (10.4378,11.6169);
\draw [color=c, fill=c] (10.4378,11.511) rectangle (10.4776,11.6169);
\draw [color=c, fill=c] (10.4776,11.511) rectangle (10.5174,11.6169);
\draw [color=c, fill=c] (10.5174,11.511) rectangle (10.5572,11.6169);
\draw [color=c, fill=c] (10.5572,11.511) rectangle (10.597,11.6169);
\draw [color=c, fill=c] (10.597,11.511) rectangle (10.6368,11.6169);
\draw [color=c, fill=c] (10.6368,11.511) rectangle (10.6766,11.6169);
\draw [color=c, fill=c] (10.6766,11.511) rectangle (10.7164,11.6169);
\draw [color=c, fill=c] (10.7164,11.511) rectangle (10.7562,11.6169);
\draw [color=c, fill=c] (10.7562,11.511) rectangle (10.796,11.6169);
\draw [color=c, fill=c] (10.796,11.511) rectangle (10.8358,11.6169);
\draw [color=c, fill=c] (10.8358,11.511) rectangle (10.8756,11.6169);
\draw [color=c, fill=c] (10.8756,11.511) rectangle (10.9154,11.6169);
\draw [color=c, fill=c] (10.9154,11.511) rectangle (10.9552,11.6169);
\draw [color=c, fill=c] (10.9552,11.511) rectangle (10.995,11.6169);
\definecolor{c}{rgb}{0,0.546666,1};
\draw [color=c, fill=c] (10.995,11.511) rectangle (11.0348,11.6169);
\draw [color=c, fill=c] (11.0348,11.511) rectangle (11.0746,11.6169);
\draw [color=c, fill=c] (11.0746,11.511) rectangle (11.1144,11.6169);
\draw [color=c, fill=c] (11.1144,11.511) rectangle (11.1542,11.6169);
\draw [color=c, fill=c] (11.1542,11.511) rectangle (11.194,11.6169);
\draw [color=c, fill=c] (11.194,11.511) rectangle (11.2338,11.6169);
\draw [color=c, fill=c] (11.2338,11.511) rectangle (11.2736,11.6169);
\draw [color=c, fill=c] (11.2736,11.511) rectangle (11.3134,11.6169);
\draw [color=c, fill=c] (11.3134,11.511) rectangle (11.3532,11.6169);
\draw [color=c, fill=c] (11.3532,11.511) rectangle (11.393,11.6169);
\draw [color=c, fill=c] (11.393,11.511) rectangle (11.4328,11.6169);
\draw [color=c, fill=c] (11.4328,11.511) rectangle (11.4726,11.6169);
\draw [color=c, fill=c] (11.4726,11.511) rectangle (11.5124,11.6169);
\draw [color=c, fill=c] (11.5124,11.511) rectangle (11.5522,11.6169);
\draw [color=c, fill=c] (11.5522,11.511) rectangle (11.592,11.6169);
\draw [color=c, fill=c] (11.592,11.511) rectangle (11.6318,11.6169);
\draw [color=c, fill=c] (11.6318,11.511) rectangle (11.6716,11.6169);
\draw [color=c, fill=c] (11.6716,11.511) rectangle (11.7114,11.6169);
\draw [color=c, fill=c] (11.7114,11.511) rectangle (11.7512,11.6169);
\draw [color=c, fill=c] (11.7512,11.511) rectangle (11.791,11.6169);
\draw [color=c, fill=c] (11.791,11.511) rectangle (11.8308,11.6169);
\draw [color=c, fill=c] (11.8308,11.511) rectangle (11.8706,11.6169);
\draw [color=c, fill=c] (11.8706,11.511) rectangle (11.9104,11.6169);
\draw [color=c, fill=c] (11.9104,11.511) rectangle (11.9502,11.6169);
\draw [color=c, fill=c] (11.9502,11.511) rectangle (11.99,11.6169);
\draw [color=c, fill=c] (11.99,11.511) rectangle (12.0299,11.6169);
\draw [color=c, fill=c] (12.0299,11.511) rectangle (12.0697,11.6169);
\draw [color=c, fill=c] (12.0697,11.511) rectangle (12.1095,11.6169);
\draw [color=c, fill=c] (12.1095,11.511) rectangle (12.1493,11.6169);
\draw [color=c, fill=c] (12.1493,11.511) rectangle (12.1891,11.6169);
\draw [color=c, fill=c] (12.1891,11.511) rectangle (12.2289,11.6169);
\draw [color=c, fill=c] (12.2289,11.511) rectangle (12.2687,11.6169);
\draw [color=c, fill=c] (12.2687,11.511) rectangle (12.3085,11.6169);
\draw [color=c, fill=c] (12.3085,11.511) rectangle (12.3483,11.6169);
\draw [color=c, fill=c] (12.3483,11.511) rectangle (12.3881,11.6169);
\draw [color=c, fill=c] (12.3881,11.511) rectangle (12.4279,11.6169);
\draw [color=c, fill=c] (12.4279,11.511) rectangle (12.4677,11.6169);
\draw [color=c, fill=c] (12.4677,11.511) rectangle (12.5075,11.6169);
\draw [color=c, fill=c] (12.5075,11.511) rectangle (12.5473,11.6169);
\draw [color=c, fill=c] (12.5473,11.511) rectangle (12.5871,11.6169);
\draw [color=c, fill=c] (12.5871,11.511) rectangle (12.6269,11.6169);
\draw [color=c, fill=c] (12.6269,11.511) rectangle (12.6667,11.6169);
\draw [color=c, fill=c] (12.6667,11.511) rectangle (12.7065,11.6169);
\draw [color=c, fill=c] (12.7065,11.511) rectangle (12.7463,11.6169);
\draw [color=c, fill=c] (12.7463,11.511) rectangle (12.7861,11.6169);
\draw [color=c, fill=c] (12.7861,11.511) rectangle (12.8259,11.6169);
\draw [color=c, fill=c] (12.8259,11.511) rectangle (12.8657,11.6169);
\draw [color=c, fill=c] (12.8657,11.511) rectangle (12.9055,11.6169);
\draw [color=c, fill=c] (12.9055,11.511) rectangle (12.9453,11.6169);
\draw [color=c, fill=c] (12.9453,11.511) rectangle (12.9851,11.6169);
\draw [color=c, fill=c] (12.9851,11.511) rectangle (13.0249,11.6169);
\draw [color=c, fill=c] (13.0249,11.511) rectangle (13.0647,11.6169);
\draw [color=c, fill=c] (13.0647,11.511) rectangle (13.1045,11.6169);
\draw [color=c, fill=c] (13.1045,11.511) rectangle (13.1443,11.6169);
\draw [color=c, fill=c] (13.1443,11.511) rectangle (13.1841,11.6169);
\draw [color=c, fill=c] (13.1841,11.511) rectangle (13.2239,11.6169);
\draw [color=c, fill=c] (13.2239,11.511) rectangle (13.2637,11.6169);
\draw [color=c, fill=c] (13.2637,11.511) rectangle (13.3035,11.6169);
\draw [color=c, fill=c] (13.3035,11.511) rectangle (13.3433,11.6169);
\draw [color=c, fill=c] (13.3433,11.511) rectangle (13.3831,11.6169);
\draw [color=c, fill=c] (13.3831,11.511) rectangle (13.4229,11.6169);
\draw [color=c, fill=c] (13.4229,11.511) rectangle (13.4627,11.6169);
\draw [color=c, fill=c] (13.4627,11.511) rectangle (13.5025,11.6169);
\draw [color=c, fill=c] (13.5025,11.511) rectangle (13.5423,11.6169);
\draw [color=c, fill=c] (13.5423,11.511) rectangle (13.5821,11.6169);
\draw [color=c, fill=c] (13.5821,11.511) rectangle (13.6219,11.6169);
\draw [color=c, fill=c] (13.6219,11.511) rectangle (13.6617,11.6169);
\draw [color=c, fill=c] (13.6617,11.511) rectangle (13.7015,11.6169);
\draw [color=c, fill=c] (13.7015,11.511) rectangle (13.7413,11.6169);
\draw [color=c, fill=c] (13.7413,11.511) rectangle (13.7811,11.6169);
\draw [color=c, fill=c] (13.7811,11.511) rectangle (13.8209,11.6169);
\draw [color=c, fill=c] (13.8209,11.511) rectangle (13.8607,11.6169);
\draw [color=c, fill=c] (13.8607,11.511) rectangle (13.9005,11.6169);
\draw [color=c, fill=c] (13.9005,11.511) rectangle (13.9403,11.6169);
\draw [color=c, fill=c] (13.9403,11.511) rectangle (13.9801,11.6169);
\draw [color=c, fill=c] (13.9801,11.511) rectangle (14.0199,11.6169);
\draw [color=c, fill=c] (14.0199,11.511) rectangle (14.0597,11.6169);
\draw [color=c, fill=c] (14.0597,11.511) rectangle (14.0995,11.6169);
\draw [color=c, fill=c] (14.0995,11.511) rectangle (14.1393,11.6169);
\draw [color=c, fill=c] (14.1393,11.511) rectangle (14.1791,11.6169);
\draw [color=c, fill=c] (14.1791,11.511) rectangle (14.2189,11.6169);
\draw [color=c, fill=c] (14.2189,11.511) rectangle (14.2587,11.6169);
\draw [color=c, fill=c] (14.2587,11.511) rectangle (14.2985,11.6169);
\draw [color=c, fill=c] (14.2985,11.511) rectangle (14.3383,11.6169);
\draw [color=c, fill=c] (14.3383,11.511) rectangle (14.3781,11.6169);
\draw [color=c, fill=c] (14.3781,11.511) rectangle (14.4179,11.6169);
\draw [color=c, fill=c] (14.4179,11.511) rectangle (14.4577,11.6169);
\draw [color=c, fill=c] (14.4577,11.511) rectangle (14.4975,11.6169);
\definecolor{c}{rgb}{0,0.733333,1};
\draw [color=c, fill=c] (14.4975,11.511) rectangle (14.5373,11.6169);
\draw [color=c, fill=c] (14.5373,11.511) rectangle (14.5771,11.6169);
\draw [color=c, fill=c] (14.5771,11.511) rectangle (14.6169,11.6169);
\draw [color=c, fill=c] (14.6169,11.511) rectangle (14.6567,11.6169);
\draw [color=c, fill=c] (14.6567,11.511) rectangle (14.6965,11.6169);
\draw [color=c, fill=c] (14.6965,11.511) rectangle (14.7363,11.6169);
\draw [color=c, fill=c] (14.7363,11.511) rectangle (14.7761,11.6169);
\draw [color=c, fill=c] (14.7761,11.511) rectangle (14.8159,11.6169);
\draw [color=c, fill=c] (14.8159,11.511) rectangle (14.8557,11.6169);
\draw [color=c, fill=c] (14.8557,11.511) rectangle (14.8955,11.6169);
\draw [color=c, fill=c] (14.8955,11.511) rectangle (14.9353,11.6169);
\draw [color=c, fill=c] (14.9353,11.511) rectangle (14.9751,11.6169);
\draw [color=c, fill=c] (14.9751,11.511) rectangle (15.0149,11.6169);
\draw [color=c, fill=c] (15.0149,11.511) rectangle (15.0547,11.6169);
\draw [color=c, fill=c] (15.0547,11.511) rectangle (15.0945,11.6169);
\draw [color=c, fill=c] (15.0945,11.511) rectangle (15.1343,11.6169);
\draw [color=c, fill=c] (15.1343,11.511) rectangle (15.1741,11.6169);
\draw [color=c, fill=c] (15.1741,11.511) rectangle (15.2139,11.6169);
\draw [color=c, fill=c] (15.2139,11.511) rectangle (15.2537,11.6169);
\draw [color=c, fill=c] (15.2537,11.511) rectangle (15.2935,11.6169);
\draw [color=c, fill=c] (15.2935,11.511) rectangle (15.3333,11.6169);
\draw [color=c, fill=c] (15.3333,11.511) rectangle (15.3731,11.6169);
\draw [color=c, fill=c] (15.3731,11.511) rectangle (15.4129,11.6169);
\draw [color=c, fill=c] (15.4129,11.511) rectangle (15.4527,11.6169);
\draw [color=c, fill=c] (15.4527,11.511) rectangle (15.4925,11.6169);
\draw [color=c, fill=c] (15.4925,11.511) rectangle (15.5323,11.6169);
\draw [color=c, fill=c] (15.5323,11.511) rectangle (15.5721,11.6169);
\draw [color=c, fill=c] (15.5721,11.511) rectangle (15.6119,11.6169);
\draw [color=c, fill=c] (15.6119,11.511) rectangle (15.6517,11.6169);
\draw [color=c, fill=c] (15.6517,11.511) rectangle (15.6915,11.6169);
\draw [color=c, fill=c] (15.6915,11.511) rectangle (15.7313,11.6169);
\draw [color=c, fill=c] (15.7313,11.511) rectangle (15.7711,11.6169);
\draw [color=c, fill=c] (15.7711,11.511) rectangle (15.8109,11.6169);
\draw [color=c, fill=c] (15.8109,11.511) rectangle (15.8507,11.6169);
\draw [color=c, fill=c] (15.8507,11.511) rectangle (15.8905,11.6169);
\draw [color=c, fill=c] (15.8905,11.511) rectangle (15.9303,11.6169);
\draw [color=c, fill=c] (15.9303,11.511) rectangle (15.9701,11.6169);
\draw [color=c, fill=c] (15.9701,11.511) rectangle (16.01,11.6169);
\draw [color=c, fill=c] (16.01,11.511) rectangle (16.0498,11.6169);
\draw [color=c, fill=c] (16.0498,11.511) rectangle (16.0896,11.6169);
\draw [color=c, fill=c] (16.0896,11.511) rectangle (16.1294,11.6169);
\draw [color=c, fill=c] (16.1294,11.511) rectangle (16.1692,11.6169);
\draw [color=c, fill=c] (16.1692,11.511) rectangle (16.209,11.6169);
\draw [color=c, fill=c] (16.209,11.511) rectangle (16.2488,11.6169);
\draw [color=c, fill=c] (16.2488,11.511) rectangle (16.2886,11.6169);
\draw [color=c, fill=c] (16.2886,11.511) rectangle (16.3284,11.6169);
\draw [color=c, fill=c] (16.3284,11.511) rectangle (16.3682,11.6169);
\draw [color=c, fill=c] (16.3682,11.511) rectangle (16.408,11.6169);
\draw [color=c, fill=c] (16.408,11.511) rectangle (16.4478,11.6169);
\draw [color=c, fill=c] (16.4478,11.511) rectangle (16.4876,11.6169);
\draw [color=c, fill=c] (16.4876,11.511) rectangle (16.5274,11.6169);
\draw [color=c, fill=c] (16.5274,11.511) rectangle (16.5672,11.6169);
\draw [color=c, fill=c] (16.5672,11.511) rectangle (16.607,11.6169);
\draw [color=c, fill=c] (16.607,11.511) rectangle (16.6468,11.6169);
\draw [color=c, fill=c] (16.6468,11.511) rectangle (16.6866,11.6169);
\draw [color=c, fill=c] (16.6866,11.511) rectangle (16.7264,11.6169);
\draw [color=c, fill=c] (16.7264,11.511) rectangle (16.7662,11.6169);
\draw [color=c, fill=c] (16.7662,11.511) rectangle (16.806,11.6169);
\draw [color=c, fill=c] (16.806,11.511) rectangle (16.8458,11.6169);
\draw [color=c, fill=c] (16.8458,11.511) rectangle (16.8856,11.6169);
\draw [color=c, fill=c] (16.8856,11.511) rectangle (16.9254,11.6169);
\draw [color=c, fill=c] (16.9254,11.511) rectangle (16.9652,11.6169);
\draw [color=c, fill=c] (16.9652,11.511) rectangle (17.005,11.6169);
\draw [color=c, fill=c] (17.005,11.511) rectangle (17.0448,11.6169);
\draw [color=c, fill=c] (17.0448,11.511) rectangle (17.0846,11.6169);
\draw [color=c, fill=c] (17.0846,11.511) rectangle (17.1244,11.6169);
\draw [color=c, fill=c] (17.1244,11.511) rectangle (17.1642,11.6169);
\draw [color=c, fill=c] (17.1642,11.511) rectangle (17.204,11.6169);
\draw [color=c, fill=c] (17.204,11.511) rectangle (17.2438,11.6169);
\draw [color=c, fill=c] (17.2438,11.511) rectangle (17.2836,11.6169);
\draw [color=c, fill=c] (17.2836,11.511) rectangle (17.3234,11.6169);
\draw [color=c, fill=c] (17.3234,11.511) rectangle (17.3632,11.6169);
\draw [color=c, fill=c] (17.3632,11.511) rectangle (17.403,11.6169);
\draw [color=c, fill=c] (17.403,11.511) rectangle (17.4428,11.6169);
\draw [color=c, fill=c] (17.4428,11.511) rectangle (17.4826,11.6169);
\draw [color=c, fill=c] (17.4826,11.511) rectangle (17.5224,11.6169);
\draw [color=c, fill=c] (17.5224,11.511) rectangle (17.5622,11.6169);
\draw [color=c, fill=c] (17.5622,11.511) rectangle (17.602,11.6169);
\draw [color=c, fill=c] (17.602,11.511) rectangle (17.6418,11.6169);
\draw [color=c, fill=c] (17.6418,11.511) rectangle (17.6816,11.6169);
\draw [color=c, fill=c] (17.6816,11.511) rectangle (17.7214,11.6169);
\draw [color=c, fill=c] (17.7214,11.511) rectangle (17.7612,11.6169);
\draw [color=c, fill=c] (17.7612,11.511) rectangle (17.801,11.6169);
\draw [color=c, fill=c] (17.801,11.511) rectangle (17.8408,11.6169);
\draw [color=c, fill=c] (17.8408,11.511) rectangle (17.8806,11.6169);
\draw [color=c, fill=c] (17.8806,11.511) rectangle (17.9204,11.6169);
\draw [color=c, fill=c] (17.9204,11.511) rectangle (17.9602,11.6169);
\draw [color=c, fill=c] (17.9602,11.511) rectangle (18,11.6169);
\definecolor{c}{rgb}{0.2,0,1};
\draw [color=c, fill=c] (2,11.6169) rectangle (2.0398,11.7227);
\draw [color=c, fill=c] (2.0398,11.6169) rectangle (2.0796,11.7227);
\draw [color=c, fill=c] (2.0796,11.6169) rectangle (2.1194,11.7227);
\draw [color=c, fill=c] (2.1194,11.6169) rectangle (2.1592,11.7227);
\draw [color=c, fill=c] (2.1592,11.6169) rectangle (2.19901,11.7227);
\draw [color=c, fill=c] (2.19901,11.6169) rectangle (2.23881,11.7227);
\draw [color=c, fill=c] (2.23881,11.6169) rectangle (2.27861,11.7227);
\draw [color=c, fill=c] (2.27861,11.6169) rectangle (2.31841,11.7227);
\draw [color=c, fill=c] (2.31841,11.6169) rectangle (2.35821,11.7227);
\draw [color=c, fill=c] (2.35821,11.6169) rectangle (2.39801,11.7227);
\draw [color=c, fill=c] (2.39801,11.6169) rectangle (2.43781,11.7227);
\draw [color=c, fill=c] (2.43781,11.6169) rectangle (2.47761,11.7227);
\draw [color=c, fill=c] (2.47761,11.6169) rectangle (2.51741,11.7227);
\draw [color=c, fill=c] (2.51741,11.6169) rectangle (2.55721,11.7227);
\draw [color=c, fill=c] (2.55721,11.6169) rectangle (2.59702,11.7227);
\draw [color=c, fill=c] (2.59702,11.6169) rectangle (2.63682,11.7227);
\draw [color=c, fill=c] (2.63682,11.6169) rectangle (2.67662,11.7227);
\draw [color=c, fill=c] (2.67662,11.6169) rectangle (2.71642,11.7227);
\draw [color=c, fill=c] (2.71642,11.6169) rectangle (2.75622,11.7227);
\draw [color=c, fill=c] (2.75622,11.6169) rectangle (2.79602,11.7227);
\draw [color=c, fill=c] (2.79602,11.6169) rectangle (2.83582,11.7227);
\draw [color=c, fill=c] (2.83582,11.6169) rectangle (2.87562,11.7227);
\draw [color=c, fill=c] (2.87562,11.6169) rectangle (2.91542,11.7227);
\draw [color=c, fill=c] (2.91542,11.6169) rectangle (2.95522,11.7227);
\draw [color=c, fill=c] (2.95522,11.6169) rectangle (2.99502,11.7227);
\draw [color=c, fill=c] (2.99502,11.6169) rectangle (3.03483,11.7227);
\draw [color=c, fill=c] (3.03483,11.6169) rectangle (3.07463,11.7227);
\draw [color=c, fill=c] (3.07463,11.6169) rectangle (3.11443,11.7227);
\draw [color=c, fill=c] (3.11443,11.6169) rectangle (3.15423,11.7227);
\draw [color=c, fill=c] (3.15423,11.6169) rectangle (3.19403,11.7227);
\draw [color=c, fill=c] (3.19403,11.6169) rectangle (3.23383,11.7227);
\draw [color=c, fill=c] (3.23383,11.6169) rectangle (3.27363,11.7227);
\draw [color=c, fill=c] (3.27363,11.6169) rectangle (3.31343,11.7227);
\draw [color=c, fill=c] (3.31343,11.6169) rectangle (3.35323,11.7227);
\draw [color=c, fill=c] (3.35323,11.6169) rectangle (3.39303,11.7227);
\draw [color=c, fill=c] (3.39303,11.6169) rectangle (3.43284,11.7227);
\draw [color=c, fill=c] (3.43284,11.6169) rectangle (3.47264,11.7227);
\draw [color=c, fill=c] (3.47264,11.6169) rectangle (3.51244,11.7227);
\draw [color=c, fill=c] (3.51244,11.6169) rectangle (3.55224,11.7227);
\draw [color=c, fill=c] (3.55224,11.6169) rectangle (3.59204,11.7227);
\draw [color=c, fill=c] (3.59204,11.6169) rectangle (3.63184,11.7227);
\draw [color=c, fill=c] (3.63184,11.6169) rectangle (3.67164,11.7227);
\draw [color=c, fill=c] (3.67164,11.6169) rectangle (3.71144,11.7227);
\draw [color=c, fill=c] (3.71144,11.6169) rectangle (3.75124,11.7227);
\draw [color=c, fill=c] (3.75124,11.6169) rectangle (3.79104,11.7227);
\draw [color=c, fill=c] (3.79104,11.6169) rectangle (3.83085,11.7227);
\draw [color=c, fill=c] (3.83085,11.6169) rectangle (3.87065,11.7227);
\draw [color=c, fill=c] (3.87065,11.6169) rectangle (3.91045,11.7227);
\draw [color=c, fill=c] (3.91045,11.6169) rectangle (3.95025,11.7227);
\draw [color=c, fill=c] (3.95025,11.6169) rectangle (3.99005,11.7227);
\draw [color=c, fill=c] (3.99005,11.6169) rectangle (4.02985,11.7227);
\draw [color=c, fill=c] (4.02985,11.6169) rectangle (4.06965,11.7227);
\draw [color=c, fill=c] (4.06965,11.6169) rectangle (4.10945,11.7227);
\draw [color=c, fill=c] (4.10945,11.6169) rectangle (4.14925,11.7227);
\draw [color=c, fill=c] (4.14925,11.6169) rectangle (4.18905,11.7227);
\draw [color=c, fill=c] (4.18905,11.6169) rectangle (4.22886,11.7227);
\draw [color=c, fill=c] (4.22886,11.6169) rectangle (4.26866,11.7227);
\draw [color=c, fill=c] (4.26866,11.6169) rectangle (4.30846,11.7227);
\draw [color=c, fill=c] (4.30846,11.6169) rectangle (4.34826,11.7227);
\draw [color=c, fill=c] (4.34826,11.6169) rectangle (4.38806,11.7227);
\draw [color=c, fill=c] (4.38806,11.6169) rectangle (4.42786,11.7227);
\draw [color=c, fill=c] (4.42786,11.6169) rectangle (4.46766,11.7227);
\draw [color=c, fill=c] (4.46766,11.6169) rectangle (4.50746,11.7227);
\draw [color=c, fill=c] (4.50746,11.6169) rectangle (4.54726,11.7227);
\draw [color=c, fill=c] (4.54726,11.6169) rectangle (4.58706,11.7227);
\draw [color=c, fill=c] (4.58706,11.6169) rectangle (4.62687,11.7227);
\draw [color=c, fill=c] (4.62687,11.6169) rectangle (4.66667,11.7227);
\draw [color=c, fill=c] (4.66667,11.6169) rectangle (4.70647,11.7227);
\draw [color=c, fill=c] (4.70647,11.6169) rectangle (4.74627,11.7227);
\draw [color=c, fill=c] (4.74627,11.6169) rectangle (4.78607,11.7227);
\draw [color=c, fill=c] (4.78607,11.6169) rectangle (4.82587,11.7227);
\draw [color=c, fill=c] (4.82587,11.6169) rectangle (4.86567,11.7227);
\draw [color=c, fill=c] (4.86567,11.6169) rectangle (4.90547,11.7227);
\draw [color=c, fill=c] (4.90547,11.6169) rectangle (4.94527,11.7227);
\draw [color=c, fill=c] (4.94527,11.6169) rectangle (4.98507,11.7227);
\draw [color=c, fill=c] (4.98507,11.6169) rectangle (5.02488,11.7227);
\draw [color=c, fill=c] (5.02488,11.6169) rectangle (5.06468,11.7227);
\draw [color=c, fill=c] (5.06468,11.6169) rectangle (5.10448,11.7227);
\draw [color=c, fill=c] (5.10448,11.6169) rectangle (5.14428,11.7227);
\draw [color=c, fill=c] (5.14428,11.6169) rectangle (5.18408,11.7227);
\draw [color=c, fill=c] (5.18408,11.6169) rectangle (5.22388,11.7227);
\draw [color=c, fill=c] (5.22388,11.6169) rectangle (5.26368,11.7227);
\draw [color=c, fill=c] (5.26368,11.6169) rectangle (5.30348,11.7227);
\draw [color=c, fill=c] (5.30348,11.6169) rectangle (5.34328,11.7227);
\draw [color=c, fill=c] (5.34328,11.6169) rectangle (5.38308,11.7227);
\draw [color=c, fill=c] (5.38308,11.6169) rectangle (5.42289,11.7227);
\draw [color=c, fill=c] (5.42289,11.6169) rectangle (5.46269,11.7227);
\draw [color=c, fill=c] (5.46269,11.6169) rectangle (5.50249,11.7227);
\draw [color=c, fill=c] (5.50249,11.6169) rectangle (5.54229,11.7227);
\draw [color=c, fill=c] (5.54229,11.6169) rectangle (5.58209,11.7227);
\draw [color=c, fill=c] (5.58209,11.6169) rectangle (5.62189,11.7227);
\draw [color=c, fill=c] (5.62189,11.6169) rectangle (5.66169,11.7227);
\draw [color=c, fill=c] (5.66169,11.6169) rectangle (5.70149,11.7227);
\draw [color=c, fill=c] (5.70149,11.6169) rectangle (5.74129,11.7227);
\draw [color=c, fill=c] (5.74129,11.6169) rectangle (5.78109,11.7227);
\draw [color=c, fill=c] (5.78109,11.6169) rectangle (5.8209,11.7227);
\draw [color=c, fill=c] (5.8209,11.6169) rectangle (5.8607,11.7227);
\draw [color=c, fill=c] (5.8607,11.6169) rectangle (5.9005,11.7227);
\draw [color=c, fill=c] (5.9005,11.6169) rectangle (5.9403,11.7227);
\draw [color=c, fill=c] (5.9403,11.6169) rectangle (5.9801,11.7227);
\draw [color=c, fill=c] (5.9801,11.6169) rectangle (6.0199,11.7227);
\draw [color=c, fill=c] (6.0199,11.6169) rectangle (6.0597,11.7227);
\draw [color=c, fill=c] (6.0597,11.6169) rectangle (6.0995,11.7227);
\draw [color=c, fill=c] (6.0995,11.6169) rectangle (6.1393,11.7227);
\draw [color=c, fill=c] (6.1393,11.6169) rectangle (6.1791,11.7227);
\draw [color=c, fill=c] (6.1791,11.6169) rectangle (6.21891,11.7227);
\draw [color=c, fill=c] (6.21891,11.6169) rectangle (6.25871,11.7227);
\draw [color=c, fill=c] (6.25871,11.6169) rectangle (6.29851,11.7227);
\draw [color=c, fill=c] (6.29851,11.6169) rectangle (6.33831,11.7227);
\draw [color=c, fill=c] (6.33831,11.6169) rectangle (6.37811,11.7227);
\draw [color=c, fill=c] (6.37811,11.6169) rectangle (6.41791,11.7227);
\draw [color=c, fill=c] (6.41791,11.6169) rectangle (6.45771,11.7227);
\draw [color=c, fill=c] (6.45771,11.6169) rectangle (6.49751,11.7227);
\draw [color=c, fill=c] (6.49751,11.6169) rectangle (6.53731,11.7227);
\draw [color=c, fill=c] (6.53731,11.6169) rectangle (6.57711,11.7227);
\draw [color=c, fill=c] (6.57711,11.6169) rectangle (6.61692,11.7227);
\draw [color=c, fill=c] (6.61692,11.6169) rectangle (6.65672,11.7227);
\draw [color=c, fill=c] (6.65672,11.6169) rectangle (6.69652,11.7227);
\draw [color=c, fill=c] (6.69652,11.6169) rectangle (6.73632,11.7227);
\draw [color=c, fill=c] (6.73632,11.6169) rectangle (6.77612,11.7227);
\draw [color=c, fill=c] (6.77612,11.6169) rectangle (6.81592,11.7227);
\draw [color=c, fill=c] (6.81592,11.6169) rectangle (6.85572,11.7227);
\draw [color=c, fill=c] (6.85572,11.6169) rectangle (6.89552,11.7227);
\draw [color=c, fill=c] (6.89552,11.6169) rectangle (6.93532,11.7227);
\draw [color=c, fill=c] (6.93532,11.6169) rectangle (6.97512,11.7227);
\draw [color=c, fill=c] (6.97512,11.6169) rectangle (7.01493,11.7227);
\draw [color=c, fill=c] (7.01493,11.6169) rectangle (7.05473,11.7227);
\draw [color=c, fill=c] (7.05473,11.6169) rectangle (7.09453,11.7227);
\draw [color=c, fill=c] (7.09453,11.6169) rectangle (7.13433,11.7227);
\draw [color=c, fill=c] (7.13433,11.6169) rectangle (7.17413,11.7227);
\draw [color=c, fill=c] (7.17413,11.6169) rectangle (7.21393,11.7227);
\draw [color=c, fill=c] (7.21393,11.6169) rectangle (7.25373,11.7227);
\draw [color=c, fill=c] (7.25373,11.6169) rectangle (7.29353,11.7227);
\draw [color=c, fill=c] (7.29353,11.6169) rectangle (7.33333,11.7227);
\draw [color=c, fill=c] (7.33333,11.6169) rectangle (7.37313,11.7227);
\draw [color=c, fill=c] (7.37313,11.6169) rectangle (7.41294,11.7227);
\draw [color=c, fill=c] (7.41294,11.6169) rectangle (7.45274,11.7227);
\draw [color=c, fill=c] (7.45274,11.6169) rectangle (7.49254,11.7227);
\draw [color=c, fill=c] (7.49254,11.6169) rectangle (7.53234,11.7227);
\draw [color=c, fill=c] (7.53234,11.6169) rectangle (7.57214,11.7227);
\draw [color=c, fill=c] (7.57214,11.6169) rectangle (7.61194,11.7227);
\draw [color=c, fill=c] (7.61194,11.6169) rectangle (7.65174,11.7227);
\draw [color=c, fill=c] (7.65174,11.6169) rectangle (7.69154,11.7227);
\draw [color=c, fill=c] (7.69154,11.6169) rectangle (7.73134,11.7227);
\draw [color=c, fill=c] (7.73134,11.6169) rectangle (7.77114,11.7227);
\draw [color=c, fill=c] (7.77114,11.6169) rectangle (7.81095,11.7227);
\draw [color=c, fill=c] (7.81095,11.6169) rectangle (7.85075,11.7227);
\draw [color=c, fill=c] (7.85075,11.6169) rectangle (7.89055,11.7227);
\draw [color=c, fill=c] (7.89055,11.6169) rectangle (7.93035,11.7227);
\draw [color=c, fill=c] (7.93035,11.6169) rectangle (7.97015,11.7227);
\draw [color=c, fill=c] (7.97015,11.6169) rectangle (8.00995,11.7227);
\draw [color=c, fill=c] (8.00995,11.6169) rectangle (8.04975,11.7227);
\draw [color=c, fill=c] (8.04975,11.6169) rectangle (8.08955,11.7227);
\definecolor{c}{rgb}{0,0.0800001,1};
\draw [color=c, fill=c] (8.08955,11.6169) rectangle (8.12935,11.7227);
\draw [color=c, fill=c] (8.12935,11.6169) rectangle (8.16915,11.7227);
\draw [color=c, fill=c] (8.16915,11.6169) rectangle (8.20895,11.7227);
\draw [color=c, fill=c] (8.20895,11.6169) rectangle (8.24876,11.7227);
\draw [color=c, fill=c] (8.24876,11.6169) rectangle (8.28856,11.7227);
\draw [color=c, fill=c] (8.28856,11.6169) rectangle (8.32836,11.7227);
\draw [color=c, fill=c] (8.32836,11.6169) rectangle (8.36816,11.7227);
\draw [color=c, fill=c] (8.36816,11.6169) rectangle (8.40796,11.7227);
\draw [color=c, fill=c] (8.40796,11.6169) rectangle (8.44776,11.7227);
\draw [color=c, fill=c] (8.44776,11.6169) rectangle (8.48756,11.7227);
\draw [color=c, fill=c] (8.48756,11.6169) rectangle (8.52736,11.7227);
\draw [color=c, fill=c] (8.52736,11.6169) rectangle (8.56716,11.7227);
\draw [color=c, fill=c] (8.56716,11.6169) rectangle (8.60697,11.7227);
\draw [color=c, fill=c] (8.60697,11.6169) rectangle (8.64677,11.7227);
\draw [color=c, fill=c] (8.64677,11.6169) rectangle (8.68657,11.7227);
\draw [color=c, fill=c] (8.68657,11.6169) rectangle (8.72637,11.7227);
\draw [color=c, fill=c] (8.72637,11.6169) rectangle (8.76617,11.7227);
\draw [color=c, fill=c] (8.76617,11.6169) rectangle (8.80597,11.7227);
\draw [color=c, fill=c] (8.80597,11.6169) rectangle (8.84577,11.7227);
\draw [color=c, fill=c] (8.84577,11.6169) rectangle (8.88557,11.7227);
\draw [color=c, fill=c] (8.88557,11.6169) rectangle (8.92537,11.7227);
\draw [color=c, fill=c] (8.92537,11.6169) rectangle (8.96517,11.7227);
\draw [color=c, fill=c] (8.96517,11.6169) rectangle (9.00498,11.7227);
\draw [color=c, fill=c] (9.00498,11.6169) rectangle (9.04478,11.7227);
\draw [color=c, fill=c] (9.04478,11.6169) rectangle (9.08458,11.7227);
\draw [color=c, fill=c] (9.08458,11.6169) rectangle (9.12438,11.7227);
\draw [color=c, fill=c] (9.12438,11.6169) rectangle (9.16418,11.7227);
\draw [color=c, fill=c] (9.16418,11.6169) rectangle (9.20398,11.7227);
\draw [color=c, fill=c] (9.20398,11.6169) rectangle (9.24378,11.7227);
\draw [color=c, fill=c] (9.24378,11.6169) rectangle (9.28358,11.7227);
\draw [color=c, fill=c] (9.28358,11.6169) rectangle (9.32338,11.7227);
\draw [color=c, fill=c] (9.32338,11.6169) rectangle (9.36318,11.7227);
\draw [color=c, fill=c] (9.36318,11.6169) rectangle (9.40298,11.7227);
\draw [color=c, fill=c] (9.40298,11.6169) rectangle (9.44279,11.7227);
\draw [color=c, fill=c] (9.44279,11.6169) rectangle (9.48259,11.7227);
\draw [color=c, fill=c] (9.48259,11.6169) rectangle (9.52239,11.7227);
\draw [color=c, fill=c] (9.52239,11.6169) rectangle (9.56219,11.7227);
\draw [color=c, fill=c] (9.56219,11.6169) rectangle (9.60199,11.7227);
\draw [color=c, fill=c] (9.60199,11.6169) rectangle (9.64179,11.7227);
\draw [color=c, fill=c] (9.64179,11.6169) rectangle (9.68159,11.7227);
\definecolor{c}{rgb}{0,0.266667,1};
\draw [color=c, fill=c] (9.68159,11.6169) rectangle (9.72139,11.7227);
\draw [color=c, fill=c] (9.72139,11.6169) rectangle (9.76119,11.7227);
\draw [color=c, fill=c] (9.76119,11.6169) rectangle (9.80099,11.7227);
\draw [color=c, fill=c] (9.80099,11.6169) rectangle (9.8408,11.7227);
\draw [color=c, fill=c] (9.8408,11.6169) rectangle (9.8806,11.7227);
\draw [color=c, fill=c] (9.8806,11.6169) rectangle (9.9204,11.7227);
\draw [color=c, fill=c] (9.9204,11.6169) rectangle (9.9602,11.7227);
\draw [color=c, fill=c] (9.9602,11.6169) rectangle (10,11.7227);
\draw [color=c, fill=c] (10,11.6169) rectangle (10.0398,11.7227);
\draw [color=c, fill=c] (10.0398,11.6169) rectangle (10.0796,11.7227);
\draw [color=c, fill=c] (10.0796,11.6169) rectangle (10.1194,11.7227);
\draw [color=c, fill=c] (10.1194,11.6169) rectangle (10.1592,11.7227);
\draw [color=c, fill=c] (10.1592,11.6169) rectangle (10.199,11.7227);
\draw [color=c, fill=c] (10.199,11.6169) rectangle (10.2388,11.7227);
\draw [color=c, fill=c] (10.2388,11.6169) rectangle (10.2786,11.7227);
\draw [color=c, fill=c] (10.2786,11.6169) rectangle (10.3184,11.7227);
\draw [color=c, fill=c] (10.3184,11.6169) rectangle (10.3582,11.7227);
\draw [color=c, fill=c] (10.3582,11.6169) rectangle (10.398,11.7227);
\draw [color=c, fill=c] (10.398,11.6169) rectangle (10.4378,11.7227);
\draw [color=c, fill=c] (10.4378,11.6169) rectangle (10.4776,11.7227);
\draw [color=c, fill=c] (10.4776,11.6169) rectangle (10.5174,11.7227);
\draw [color=c, fill=c] (10.5174,11.6169) rectangle (10.5572,11.7227);
\draw [color=c, fill=c] (10.5572,11.6169) rectangle (10.597,11.7227);
\draw [color=c, fill=c] (10.597,11.6169) rectangle (10.6368,11.7227);
\draw [color=c, fill=c] (10.6368,11.6169) rectangle (10.6766,11.7227);
\draw [color=c, fill=c] (10.6766,11.6169) rectangle (10.7164,11.7227);
\draw [color=c, fill=c] (10.7164,11.6169) rectangle (10.7562,11.7227);
\draw [color=c, fill=c] (10.7562,11.6169) rectangle (10.796,11.7227);
\draw [color=c, fill=c] (10.796,11.6169) rectangle (10.8358,11.7227);
\draw [color=c, fill=c] (10.8358,11.6169) rectangle (10.8756,11.7227);
\draw [color=c, fill=c] (10.8756,11.6169) rectangle (10.9154,11.7227);
\draw [color=c, fill=c] (10.9154,11.6169) rectangle (10.9552,11.7227);
\draw [color=c, fill=c] (10.9552,11.6169) rectangle (10.995,11.7227);
\draw [color=c, fill=c] (10.995,11.6169) rectangle (11.0348,11.7227);
\definecolor{c}{rgb}{0,0.546666,1};
\draw [color=c, fill=c] (11.0348,11.6169) rectangle (11.0746,11.7227);
\draw [color=c, fill=c] (11.0746,11.6169) rectangle (11.1144,11.7227);
\draw [color=c, fill=c] (11.1144,11.6169) rectangle (11.1542,11.7227);
\draw [color=c, fill=c] (11.1542,11.6169) rectangle (11.194,11.7227);
\draw [color=c, fill=c] (11.194,11.6169) rectangle (11.2338,11.7227);
\draw [color=c, fill=c] (11.2338,11.6169) rectangle (11.2736,11.7227);
\draw [color=c, fill=c] (11.2736,11.6169) rectangle (11.3134,11.7227);
\draw [color=c, fill=c] (11.3134,11.6169) rectangle (11.3532,11.7227);
\draw [color=c, fill=c] (11.3532,11.6169) rectangle (11.393,11.7227);
\draw [color=c, fill=c] (11.393,11.6169) rectangle (11.4328,11.7227);
\draw [color=c, fill=c] (11.4328,11.6169) rectangle (11.4726,11.7227);
\draw [color=c, fill=c] (11.4726,11.6169) rectangle (11.5124,11.7227);
\draw [color=c, fill=c] (11.5124,11.6169) rectangle (11.5522,11.7227);
\draw [color=c, fill=c] (11.5522,11.6169) rectangle (11.592,11.7227);
\draw [color=c, fill=c] (11.592,11.6169) rectangle (11.6318,11.7227);
\draw [color=c, fill=c] (11.6318,11.6169) rectangle (11.6716,11.7227);
\draw [color=c, fill=c] (11.6716,11.6169) rectangle (11.7114,11.7227);
\draw [color=c, fill=c] (11.7114,11.6169) rectangle (11.7512,11.7227);
\draw [color=c, fill=c] (11.7512,11.6169) rectangle (11.791,11.7227);
\draw [color=c, fill=c] (11.791,11.6169) rectangle (11.8308,11.7227);
\draw [color=c, fill=c] (11.8308,11.6169) rectangle (11.8706,11.7227);
\draw [color=c, fill=c] (11.8706,11.6169) rectangle (11.9104,11.7227);
\draw [color=c, fill=c] (11.9104,11.6169) rectangle (11.9502,11.7227);
\draw [color=c, fill=c] (11.9502,11.6169) rectangle (11.99,11.7227);
\draw [color=c, fill=c] (11.99,11.6169) rectangle (12.0299,11.7227);
\draw [color=c, fill=c] (12.0299,11.6169) rectangle (12.0697,11.7227);
\draw [color=c, fill=c] (12.0697,11.6169) rectangle (12.1095,11.7227);
\draw [color=c, fill=c] (12.1095,11.6169) rectangle (12.1493,11.7227);
\draw [color=c, fill=c] (12.1493,11.6169) rectangle (12.1891,11.7227);
\draw [color=c, fill=c] (12.1891,11.6169) rectangle (12.2289,11.7227);
\draw [color=c, fill=c] (12.2289,11.6169) rectangle (12.2687,11.7227);
\draw [color=c, fill=c] (12.2687,11.6169) rectangle (12.3085,11.7227);
\draw [color=c, fill=c] (12.3085,11.6169) rectangle (12.3483,11.7227);
\draw [color=c, fill=c] (12.3483,11.6169) rectangle (12.3881,11.7227);
\draw [color=c, fill=c] (12.3881,11.6169) rectangle (12.4279,11.7227);
\draw [color=c, fill=c] (12.4279,11.6169) rectangle (12.4677,11.7227);
\draw [color=c, fill=c] (12.4677,11.6169) rectangle (12.5075,11.7227);
\draw [color=c, fill=c] (12.5075,11.6169) rectangle (12.5473,11.7227);
\draw [color=c, fill=c] (12.5473,11.6169) rectangle (12.5871,11.7227);
\draw [color=c, fill=c] (12.5871,11.6169) rectangle (12.6269,11.7227);
\draw [color=c, fill=c] (12.6269,11.6169) rectangle (12.6667,11.7227);
\draw [color=c, fill=c] (12.6667,11.6169) rectangle (12.7065,11.7227);
\draw [color=c, fill=c] (12.7065,11.6169) rectangle (12.7463,11.7227);
\draw [color=c, fill=c] (12.7463,11.6169) rectangle (12.7861,11.7227);
\draw [color=c, fill=c] (12.7861,11.6169) rectangle (12.8259,11.7227);
\draw [color=c, fill=c] (12.8259,11.6169) rectangle (12.8657,11.7227);
\draw [color=c, fill=c] (12.8657,11.6169) rectangle (12.9055,11.7227);
\draw [color=c, fill=c] (12.9055,11.6169) rectangle (12.9453,11.7227);
\draw [color=c, fill=c] (12.9453,11.6169) rectangle (12.9851,11.7227);
\draw [color=c, fill=c] (12.9851,11.6169) rectangle (13.0249,11.7227);
\draw [color=c, fill=c] (13.0249,11.6169) rectangle (13.0647,11.7227);
\draw [color=c, fill=c] (13.0647,11.6169) rectangle (13.1045,11.7227);
\draw [color=c, fill=c] (13.1045,11.6169) rectangle (13.1443,11.7227);
\draw [color=c, fill=c] (13.1443,11.6169) rectangle (13.1841,11.7227);
\draw [color=c, fill=c] (13.1841,11.6169) rectangle (13.2239,11.7227);
\draw [color=c, fill=c] (13.2239,11.6169) rectangle (13.2637,11.7227);
\draw [color=c, fill=c] (13.2637,11.6169) rectangle (13.3035,11.7227);
\draw [color=c, fill=c] (13.3035,11.6169) rectangle (13.3433,11.7227);
\draw [color=c, fill=c] (13.3433,11.6169) rectangle (13.3831,11.7227);
\draw [color=c, fill=c] (13.3831,11.6169) rectangle (13.4229,11.7227);
\draw [color=c, fill=c] (13.4229,11.6169) rectangle (13.4627,11.7227);
\draw [color=c, fill=c] (13.4627,11.6169) rectangle (13.5025,11.7227);
\draw [color=c, fill=c] (13.5025,11.6169) rectangle (13.5423,11.7227);
\draw [color=c, fill=c] (13.5423,11.6169) rectangle (13.5821,11.7227);
\draw [color=c, fill=c] (13.5821,11.6169) rectangle (13.6219,11.7227);
\draw [color=c, fill=c] (13.6219,11.6169) rectangle (13.6617,11.7227);
\draw [color=c, fill=c] (13.6617,11.6169) rectangle (13.7015,11.7227);
\draw [color=c, fill=c] (13.7015,11.6169) rectangle (13.7413,11.7227);
\draw [color=c, fill=c] (13.7413,11.6169) rectangle (13.7811,11.7227);
\draw [color=c, fill=c] (13.7811,11.6169) rectangle (13.8209,11.7227);
\draw [color=c, fill=c] (13.8209,11.6169) rectangle (13.8607,11.7227);
\draw [color=c, fill=c] (13.8607,11.6169) rectangle (13.9005,11.7227);
\draw [color=c, fill=c] (13.9005,11.6169) rectangle (13.9403,11.7227);
\draw [color=c, fill=c] (13.9403,11.6169) rectangle (13.9801,11.7227);
\draw [color=c, fill=c] (13.9801,11.6169) rectangle (14.0199,11.7227);
\draw [color=c, fill=c] (14.0199,11.6169) rectangle (14.0597,11.7227);
\draw [color=c, fill=c] (14.0597,11.6169) rectangle (14.0995,11.7227);
\draw [color=c, fill=c] (14.0995,11.6169) rectangle (14.1393,11.7227);
\draw [color=c, fill=c] (14.1393,11.6169) rectangle (14.1791,11.7227);
\draw [color=c, fill=c] (14.1791,11.6169) rectangle (14.2189,11.7227);
\draw [color=c, fill=c] (14.2189,11.6169) rectangle (14.2587,11.7227);
\draw [color=c, fill=c] (14.2587,11.6169) rectangle (14.2985,11.7227);
\draw [color=c, fill=c] (14.2985,11.6169) rectangle (14.3383,11.7227);
\draw [color=c, fill=c] (14.3383,11.6169) rectangle (14.3781,11.7227);
\draw [color=c, fill=c] (14.3781,11.6169) rectangle (14.4179,11.7227);
\draw [color=c, fill=c] (14.4179,11.6169) rectangle (14.4577,11.7227);
\draw [color=c, fill=c] (14.4577,11.6169) rectangle (14.4975,11.7227);
\draw [color=c, fill=c] (14.4975,11.6169) rectangle (14.5373,11.7227);
\definecolor{c}{rgb}{0,0.733333,1};
\draw [color=c, fill=c] (14.5373,11.6169) rectangle (14.5771,11.7227);
\draw [color=c, fill=c] (14.5771,11.6169) rectangle (14.6169,11.7227);
\draw [color=c, fill=c] (14.6169,11.6169) rectangle (14.6567,11.7227);
\draw [color=c, fill=c] (14.6567,11.6169) rectangle (14.6965,11.7227);
\draw [color=c, fill=c] (14.6965,11.6169) rectangle (14.7363,11.7227);
\draw [color=c, fill=c] (14.7363,11.6169) rectangle (14.7761,11.7227);
\draw [color=c, fill=c] (14.7761,11.6169) rectangle (14.8159,11.7227);
\draw [color=c, fill=c] (14.8159,11.6169) rectangle (14.8557,11.7227);
\draw [color=c, fill=c] (14.8557,11.6169) rectangle (14.8955,11.7227);
\draw [color=c, fill=c] (14.8955,11.6169) rectangle (14.9353,11.7227);
\draw [color=c, fill=c] (14.9353,11.6169) rectangle (14.9751,11.7227);
\draw [color=c, fill=c] (14.9751,11.6169) rectangle (15.0149,11.7227);
\draw [color=c, fill=c] (15.0149,11.6169) rectangle (15.0547,11.7227);
\draw [color=c, fill=c] (15.0547,11.6169) rectangle (15.0945,11.7227);
\draw [color=c, fill=c] (15.0945,11.6169) rectangle (15.1343,11.7227);
\draw [color=c, fill=c] (15.1343,11.6169) rectangle (15.1741,11.7227);
\draw [color=c, fill=c] (15.1741,11.6169) rectangle (15.2139,11.7227);
\draw [color=c, fill=c] (15.2139,11.6169) rectangle (15.2537,11.7227);
\draw [color=c, fill=c] (15.2537,11.6169) rectangle (15.2935,11.7227);
\draw [color=c, fill=c] (15.2935,11.6169) rectangle (15.3333,11.7227);
\draw [color=c, fill=c] (15.3333,11.6169) rectangle (15.3731,11.7227);
\draw [color=c, fill=c] (15.3731,11.6169) rectangle (15.4129,11.7227);
\draw [color=c, fill=c] (15.4129,11.6169) rectangle (15.4527,11.7227);
\draw [color=c, fill=c] (15.4527,11.6169) rectangle (15.4925,11.7227);
\draw [color=c, fill=c] (15.4925,11.6169) rectangle (15.5323,11.7227);
\draw [color=c, fill=c] (15.5323,11.6169) rectangle (15.5721,11.7227);
\draw [color=c, fill=c] (15.5721,11.6169) rectangle (15.6119,11.7227);
\draw [color=c, fill=c] (15.6119,11.6169) rectangle (15.6517,11.7227);
\draw [color=c, fill=c] (15.6517,11.6169) rectangle (15.6915,11.7227);
\draw [color=c, fill=c] (15.6915,11.6169) rectangle (15.7313,11.7227);
\draw [color=c, fill=c] (15.7313,11.6169) rectangle (15.7711,11.7227);
\draw [color=c, fill=c] (15.7711,11.6169) rectangle (15.8109,11.7227);
\draw [color=c, fill=c] (15.8109,11.6169) rectangle (15.8507,11.7227);
\draw [color=c, fill=c] (15.8507,11.6169) rectangle (15.8905,11.7227);
\draw [color=c, fill=c] (15.8905,11.6169) rectangle (15.9303,11.7227);
\draw [color=c, fill=c] (15.9303,11.6169) rectangle (15.9701,11.7227);
\draw [color=c, fill=c] (15.9701,11.6169) rectangle (16.01,11.7227);
\draw [color=c, fill=c] (16.01,11.6169) rectangle (16.0498,11.7227);
\draw [color=c, fill=c] (16.0498,11.6169) rectangle (16.0896,11.7227);
\draw [color=c, fill=c] (16.0896,11.6169) rectangle (16.1294,11.7227);
\draw [color=c, fill=c] (16.1294,11.6169) rectangle (16.1692,11.7227);
\draw [color=c, fill=c] (16.1692,11.6169) rectangle (16.209,11.7227);
\draw [color=c, fill=c] (16.209,11.6169) rectangle (16.2488,11.7227);
\draw [color=c, fill=c] (16.2488,11.6169) rectangle (16.2886,11.7227);
\draw [color=c, fill=c] (16.2886,11.6169) rectangle (16.3284,11.7227);
\draw [color=c, fill=c] (16.3284,11.6169) rectangle (16.3682,11.7227);
\draw [color=c, fill=c] (16.3682,11.6169) rectangle (16.408,11.7227);
\draw [color=c, fill=c] (16.408,11.6169) rectangle (16.4478,11.7227);
\draw [color=c, fill=c] (16.4478,11.6169) rectangle (16.4876,11.7227);
\draw [color=c, fill=c] (16.4876,11.6169) rectangle (16.5274,11.7227);
\draw [color=c, fill=c] (16.5274,11.6169) rectangle (16.5672,11.7227);
\draw [color=c, fill=c] (16.5672,11.6169) rectangle (16.607,11.7227);
\draw [color=c, fill=c] (16.607,11.6169) rectangle (16.6468,11.7227);
\draw [color=c, fill=c] (16.6468,11.6169) rectangle (16.6866,11.7227);
\draw [color=c, fill=c] (16.6866,11.6169) rectangle (16.7264,11.7227);
\draw [color=c, fill=c] (16.7264,11.6169) rectangle (16.7662,11.7227);
\draw [color=c, fill=c] (16.7662,11.6169) rectangle (16.806,11.7227);
\draw [color=c, fill=c] (16.806,11.6169) rectangle (16.8458,11.7227);
\draw [color=c, fill=c] (16.8458,11.6169) rectangle (16.8856,11.7227);
\draw [color=c, fill=c] (16.8856,11.6169) rectangle (16.9254,11.7227);
\draw [color=c, fill=c] (16.9254,11.6169) rectangle (16.9652,11.7227);
\draw [color=c, fill=c] (16.9652,11.6169) rectangle (17.005,11.7227);
\draw [color=c, fill=c] (17.005,11.6169) rectangle (17.0448,11.7227);
\draw [color=c, fill=c] (17.0448,11.6169) rectangle (17.0846,11.7227);
\draw [color=c, fill=c] (17.0846,11.6169) rectangle (17.1244,11.7227);
\draw [color=c, fill=c] (17.1244,11.6169) rectangle (17.1642,11.7227);
\draw [color=c, fill=c] (17.1642,11.6169) rectangle (17.204,11.7227);
\draw [color=c, fill=c] (17.204,11.6169) rectangle (17.2438,11.7227);
\draw [color=c, fill=c] (17.2438,11.6169) rectangle (17.2836,11.7227);
\draw [color=c, fill=c] (17.2836,11.6169) rectangle (17.3234,11.7227);
\draw [color=c, fill=c] (17.3234,11.6169) rectangle (17.3632,11.7227);
\draw [color=c, fill=c] (17.3632,11.6169) rectangle (17.403,11.7227);
\draw [color=c, fill=c] (17.403,11.6169) rectangle (17.4428,11.7227);
\draw [color=c, fill=c] (17.4428,11.6169) rectangle (17.4826,11.7227);
\draw [color=c, fill=c] (17.4826,11.6169) rectangle (17.5224,11.7227);
\draw [color=c, fill=c] (17.5224,11.6169) rectangle (17.5622,11.7227);
\draw [color=c, fill=c] (17.5622,11.6169) rectangle (17.602,11.7227);
\draw [color=c, fill=c] (17.602,11.6169) rectangle (17.6418,11.7227);
\draw [color=c, fill=c] (17.6418,11.6169) rectangle (17.6816,11.7227);
\draw [color=c, fill=c] (17.6816,11.6169) rectangle (17.7214,11.7227);
\draw [color=c, fill=c] (17.7214,11.6169) rectangle (17.7612,11.7227);
\draw [color=c, fill=c] (17.7612,11.6169) rectangle (17.801,11.7227);
\draw [color=c, fill=c] (17.801,11.6169) rectangle (17.8408,11.7227);
\draw [color=c, fill=c] (17.8408,11.6169) rectangle (17.8806,11.7227);
\draw [color=c, fill=c] (17.8806,11.6169) rectangle (17.9204,11.7227);
\draw [color=c, fill=c] (17.9204,11.6169) rectangle (17.9602,11.7227);
\draw [color=c, fill=c] (17.9602,11.6169) rectangle (18,11.7227);
\definecolor{c}{rgb}{0.2,0,1};
\draw [color=c, fill=c] (2,11.7227) rectangle (2.0398,11.8286);
\draw [color=c, fill=c] (2.0398,11.7227) rectangle (2.0796,11.8286);
\draw [color=c, fill=c] (2.0796,11.7227) rectangle (2.1194,11.8286);
\draw [color=c, fill=c] (2.1194,11.7227) rectangle (2.1592,11.8286);
\draw [color=c, fill=c] (2.1592,11.7227) rectangle (2.19901,11.8286);
\draw [color=c, fill=c] (2.19901,11.7227) rectangle (2.23881,11.8286);
\draw [color=c, fill=c] (2.23881,11.7227) rectangle (2.27861,11.8286);
\draw [color=c, fill=c] (2.27861,11.7227) rectangle (2.31841,11.8286);
\draw [color=c, fill=c] (2.31841,11.7227) rectangle (2.35821,11.8286);
\draw [color=c, fill=c] (2.35821,11.7227) rectangle (2.39801,11.8286);
\draw [color=c, fill=c] (2.39801,11.7227) rectangle (2.43781,11.8286);
\draw [color=c, fill=c] (2.43781,11.7227) rectangle (2.47761,11.8286);
\draw [color=c, fill=c] (2.47761,11.7227) rectangle (2.51741,11.8286);
\draw [color=c, fill=c] (2.51741,11.7227) rectangle (2.55721,11.8286);
\draw [color=c, fill=c] (2.55721,11.7227) rectangle (2.59702,11.8286);
\draw [color=c, fill=c] (2.59702,11.7227) rectangle (2.63682,11.8286);
\draw [color=c, fill=c] (2.63682,11.7227) rectangle (2.67662,11.8286);
\draw [color=c, fill=c] (2.67662,11.7227) rectangle (2.71642,11.8286);
\draw [color=c, fill=c] (2.71642,11.7227) rectangle (2.75622,11.8286);
\draw [color=c, fill=c] (2.75622,11.7227) rectangle (2.79602,11.8286);
\draw [color=c, fill=c] (2.79602,11.7227) rectangle (2.83582,11.8286);
\draw [color=c, fill=c] (2.83582,11.7227) rectangle (2.87562,11.8286);
\draw [color=c, fill=c] (2.87562,11.7227) rectangle (2.91542,11.8286);
\draw [color=c, fill=c] (2.91542,11.7227) rectangle (2.95522,11.8286);
\draw [color=c, fill=c] (2.95522,11.7227) rectangle (2.99502,11.8286);
\draw [color=c, fill=c] (2.99502,11.7227) rectangle (3.03483,11.8286);
\draw [color=c, fill=c] (3.03483,11.7227) rectangle (3.07463,11.8286);
\draw [color=c, fill=c] (3.07463,11.7227) rectangle (3.11443,11.8286);
\draw [color=c, fill=c] (3.11443,11.7227) rectangle (3.15423,11.8286);
\draw [color=c, fill=c] (3.15423,11.7227) rectangle (3.19403,11.8286);
\draw [color=c, fill=c] (3.19403,11.7227) rectangle (3.23383,11.8286);
\draw [color=c, fill=c] (3.23383,11.7227) rectangle (3.27363,11.8286);
\draw [color=c, fill=c] (3.27363,11.7227) rectangle (3.31343,11.8286);
\draw [color=c, fill=c] (3.31343,11.7227) rectangle (3.35323,11.8286);
\draw [color=c, fill=c] (3.35323,11.7227) rectangle (3.39303,11.8286);
\draw [color=c, fill=c] (3.39303,11.7227) rectangle (3.43284,11.8286);
\draw [color=c, fill=c] (3.43284,11.7227) rectangle (3.47264,11.8286);
\draw [color=c, fill=c] (3.47264,11.7227) rectangle (3.51244,11.8286);
\draw [color=c, fill=c] (3.51244,11.7227) rectangle (3.55224,11.8286);
\draw [color=c, fill=c] (3.55224,11.7227) rectangle (3.59204,11.8286);
\draw [color=c, fill=c] (3.59204,11.7227) rectangle (3.63184,11.8286);
\draw [color=c, fill=c] (3.63184,11.7227) rectangle (3.67164,11.8286);
\draw [color=c, fill=c] (3.67164,11.7227) rectangle (3.71144,11.8286);
\draw [color=c, fill=c] (3.71144,11.7227) rectangle (3.75124,11.8286);
\draw [color=c, fill=c] (3.75124,11.7227) rectangle (3.79104,11.8286);
\draw [color=c, fill=c] (3.79104,11.7227) rectangle (3.83085,11.8286);
\draw [color=c, fill=c] (3.83085,11.7227) rectangle (3.87065,11.8286);
\draw [color=c, fill=c] (3.87065,11.7227) rectangle (3.91045,11.8286);
\draw [color=c, fill=c] (3.91045,11.7227) rectangle (3.95025,11.8286);
\draw [color=c, fill=c] (3.95025,11.7227) rectangle (3.99005,11.8286);
\draw [color=c, fill=c] (3.99005,11.7227) rectangle (4.02985,11.8286);
\draw [color=c, fill=c] (4.02985,11.7227) rectangle (4.06965,11.8286);
\draw [color=c, fill=c] (4.06965,11.7227) rectangle (4.10945,11.8286);
\draw [color=c, fill=c] (4.10945,11.7227) rectangle (4.14925,11.8286);
\draw [color=c, fill=c] (4.14925,11.7227) rectangle (4.18905,11.8286);
\draw [color=c, fill=c] (4.18905,11.7227) rectangle (4.22886,11.8286);
\draw [color=c, fill=c] (4.22886,11.7227) rectangle (4.26866,11.8286);
\draw [color=c, fill=c] (4.26866,11.7227) rectangle (4.30846,11.8286);
\draw [color=c, fill=c] (4.30846,11.7227) rectangle (4.34826,11.8286);
\draw [color=c, fill=c] (4.34826,11.7227) rectangle (4.38806,11.8286);
\draw [color=c, fill=c] (4.38806,11.7227) rectangle (4.42786,11.8286);
\draw [color=c, fill=c] (4.42786,11.7227) rectangle (4.46766,11.8286);
\draw [color=c, fill=c] (4.46766,11.7227) rectangle (4.50746,11.8286);
\draw [color=c, fill=c] (4.50746,11.7227) rectangle (4.54726,11.8286);
\draw [color=c, fill=c] (4.54726,11.7227) rectangle (4.58706,11.8286);
\draw [color=c, fill=c] (4.58706,11.7227) rectangle (4.62687,11.8286);
\draw [color=c, fill=c] (4.62687,11.7227) rectangle (4.66667,11.8286);
\draw [color=c, fill=c] (4.66667,11.7227) rectangle (4.70647,11.8286);
\draw [color=c, fill=c] (4.70647,11.7227) rectangle (4.74627,11.8286);
\draw [color=c, fill=c] (4.74627,11.7227) rectangle (4.78607,11.8286);
\draw [color=c, fill=c] (4.78607,11.7227) rectangle (4.82587,11.8286);
\draw [color=c, fill=c] (4.82587,11.7227) rectangle (4.86567,11.8286);
\draw [color=c, fill=c] (4.86567,11.7227) rectangle (4.90547,11.8286);
\draw [color=c, fill=c] (4.90547,11.7227) rectangle (4.94527,11.8286);
\draw [color=c, fill=c] (4.94527,11.7227) rectangle (4.98507,11.8286);
\draw [color=c, fill=c] (4.98507,11.7227) rectangle (5.02488,11.8286);
\draw [color=c, fill=c] (5.02488,11.7227) rectangle (5.06468,11.8286);
\draw [color=c, fill=c] (5.06468,11.7227) rectangle (5.10448,11.8286);
\draw [color=c, fill=c] (5.10448,11.7227) rectangle (5.14428,11.8286);
\draw [color=c, fill=c] (5.14428,11.7227) rectangle (5.18408,11.8286);
\draw [color=c, fill=c] (5.18408,11.7227) rectangle (5.22388,11.8286);
\draw [color=c, fill=c] (5.22388,11.7227) rectangle (5.26368,11.8286);
\draw [color=c, fill=c] (5.26368,11.7227) rectangle (5.30348,11.8286);
\draw [color=c, fill=c] (5.30348,11.7227) rectangle (5.34328,11.8286);
\draw [color=c, fill=c] (5.34328,11.7227) rectangle (5.38308,11.8286);
\draw [color=c, fill=c] (5.38308,11.7227) rectangle (5.42289,11.8286);
\draw [color=c, fill=c] (5.42289,11.7227) rectangle (5.46269,11.8286);
\draw [color=c, fill=c] (5.46269,11.7227) rectangle (5.50249,11.8286);
\draw [color=c, fill=c] (5.50249,11.7227) rectangle (5.54229,11.8286);
\draw [color=c, fill=c] (5.54229,11.7227) rectangle (5.58209,11.8286);
\draw [color=c, fill=c] (5.58209,11.7227) rectangle (5.62189,11.8286);
\draw [color=c, fill=c] (5.62189,11.7227) rectangle (5.66169,11.8286);
\draw [color=c, fill=c] (5.66169,11.7227) rectangle (5.70149,11.8286);
\draw [color=c, fill=c] (5.70149,11.7227) rectangle (5.74129,11.8286);
\draw [color=c, fill=c] (5.74129,11.7227) rectangle (5.78109,11.8286);
\draw [color=c, fill=c] (5.78109,11.7227) rectangle (5.8209,11.8286);
\draw [color=c, fill=c] (5.8209,11.7227) rectangle (5.8607,11.8286);
\draw [color=c, fill=c] (5.8607,11.7227) rectangle (5.9005,11.8286);
\draw [color=c, fill=c] (5.9005,11.7227) rectangle (5.9403,11.8286);
\draw [color=c, fill=c] (5.9403,11.7227) rectangle (5.9801,11.8286);
\draw [color=c, fill=c] (5.9801,11.7227) rectangle (6.0199,11.8286);
\draw [color=c, fill=c] (6.0199,11.7227) rectangle (6.0597,11.8286);
\draw [color=c, fill=c] (6.0597,11.7227) rectangle (6.0995,11.8286);
\draw [color=c, fill=c] (6.0995,11.7227) rectangle (6.1393,11.8286);
\draw [color=c, fill=c] (6.1393,11.7227) rectangle (6.1791,11.8286);
\draw [color=c, fill=c] (6.1791,11.7227) rectangle (6.21891,11.8286);
\draw [color=c, fill=c] (6.21891,11.7227) rectangle (6.25871,11.8286);
\draw [color=c, fill=c] (6.25871,11.7227) rectangle (6.29851,11.8286);
\draw [color=c, fill=c] (6.29851,11.7227) rectangle (6.33831,11.8286);
\draw [color=c, fill=c] (6.33831,11.7227) rectangle (6.37811,11.8286);
\draw [color=c, fill=c] (6.37811,11.7227) rectangle (6.41791,11.8286);
\draw [color=c, fill=c] (6.41791,11.7227) rectangle (6.45771,11.8286);
\draw [color=c, fill=c] (6.45771,11.7227) rectangle (6.49751,11.8286);
\draw [color=c, fill=c] (6.49751,11.7227) rectangle (6.53731,11.8286);
\draw [color=c, fill=c] (6.53731,11.7227) rectangle (6.57711,11.8286);
\draw [color=c, fill=c] (6.57711,11.7227) rectangle (6.61692,11.8286);
\draw [color=c, fill=c] (6.61692,11.7227) rectangle (6.65672,11.8286);
\draw [color=c, fill=c] (6.65672,11.7227) rectangle (6.69652,11.8286);
\draw [color=c, fill=c] (6.69652,11.7227) rectangle (6.73632,11.8286);
\draw [color=c, fill=c] (6.73632,11.7227) rectangle (6.77612,11.8286);
\draw [color=c, fill=c] (6.77612,11.7227) rectangle (6.81592,11.8286);
\draw [color=c, fill=c] (6.81592,11.7227) rectangle (6.85572,11.8286);
\draw [color=c, fill=c] (6.85572,11.7227) rectangle (6.89552,11.8286);
\draw [color=c, fill=c] (6.89552,11.7227) rectangle (6.93532,11.8286);
\draw [color=c, fill=c] (6.93532,11.7227) rectangle (6.97512,11.8286);
\draw [color=c, fill=c] (6.97512,11.7227) rectangle (7.01493,11.8286);
\draw [color=c, fill=c] (7.01493,11.7227) rectangle (7.05473,11.8286);
\draw [color=c, fill=c] (7.05473,11.7227) rectangle (7.09453,11.8286);
\draw [color=c, fill=c] (7.09453,11.7227) rectangle (7.13433,11.8286);
\draw [color=c, fill=c] (7.13433,11.7227) rectangle (7.17413,11.8286);
\draw [color=c, fill=c] (7.17413,11.7227) rectangle (7.21393,11.8286);
\draw [color=c, fill=c] (7.21393,11.7227) rectangle (7.25373,11.8286);
\draw [color=c, fill=c] (7.25373,11.7227) rectangle (7.29353,11.8286);
\draw [color=c, fill=c] (7.29353,11.7227) rectangle (7.33333,11.8286);
\draw [color=c, fill=c] (7.33333,11.7227) rectangle (7.37313,11.8286);
\draw [color=c, fill=c] (7.37313,11.7227) rectangle (7.41294,11.8286);
\draw [color=c, fill=c] (7.41294,11.7227) rectangle (7.45274,11.8286);
\draw [color=c, fill=c] (7.45274,11.7227) rectangle (7.49254,11.8286);
\draw [color=c, fill=c] (7.49254,11.7227) rectangle (7.53234,11.8286);
\draw [color=c, fill=c] (7.53234,11.7227) rectangle (7.57214,11.8286);
\draw [color=c, fill=c] (7.57214,11.7227) rectangle (7.61194,11.8286);
\draw [color=c, fill=c] (7.61194,11.7227) rectangle (7.65174,11.8286);
\draw [color=c, fill=c] (7.65174,11.7227) rectangle (7.69154,11.8286);
\draw [color=c, fill=c] (7.69154,11.7227) rectangle (7.73134,11.8286);
\draw [color=c, fill=c] (7.73134,11.7227) rectangle (7.77114,11.8286);
\draw [color=c, fill=c] (7.77114,11.7227) rectangle (7.81095,11.8286);
\draw [color=c, fill=c] (7.81095,11.7227) rectangle (7.85075,11.8286);
\draw [color=c, fill=c] (7.85075,11.7227) rectangle (7.89055,11.8286);
\draw [color=c, fill=c] (7.89055,11.7227) rectangle (7.93035,11.8286);
\draw [color=c, fill=c] (7.93035,11.7227) rectangle (7.97015,11.8286);
\draw [color=c, fill=c] (7.97015,11.7227) rectangle (8.00995,11.8286);
\draw [color=c, fill=c] (8.00995,11.7227) rectangle (8.04975,11.8286);
\draw [color=c, fill=c] (8.04975,11.7227) rectangle (8.08955,11.8286);
\draw [color=c, fill=c] (8.08955,11.7227) rectangle (8.12935,11.8286);
\definecolor{c}{rgb}{0,0.0800001,1};
\draw [color=c, fill=c] (8.12935,11.7227) rectangle (8.16915,11.8286);
\draw [color=c, fill=c] (8.16915,11.7227) rectangle (8.20895,11.8286);
\draw [color=c, fill=c] (8.20895,11.7227) rectangle (8.24876,11.8286);
\draw [color=c, fill=c] (8.24876,11.7227) rectangle (8.28856,11.8286);
\draw [color=c, fill=c] (8.28856,11.7227) rectangle (8.32836,11.8286);
\draw [color=c, fill=c] (8.32836,11.7227) rectangle (8.36816,11.8286);
\draw [color=c, fill=c] (8.36816,11.7227) rectangle (8.40796,11.8286);
\draw [color=c, fill=c] (8.40796,11.7227) rectangle (8.44776,11.8286);
\draw [color=c, fill=c] (8.44776,11.7227) rectangle (8.48756,11.8286);
\draw [color=c, fill=c] (8.48756,11.7227) rectangle (8.52736,11.8286);
\draw [color=c, fill=c] (8.52736,11.7227) rectangle (8.56716,11.8286);
\draw [color=c, fill=c] (8.56716,11.7227) rectangle (8.60697,11.8286);
\draw [color=c, fill=c] (8.60697,11.7227) rectangle (8.64677,11.8286);
\draw [color=c, fill=c] (8.64677,11.7227) rectangle (8.68657,11.8286);
\draw [color=c, fill=c] (8.68657,11.7227) rectangle (8.72637,11.8286);
\draw [color=c, fill=c] (8.72637,11.7227) rectangle (8.76617,11.8286);
\draw [color=c, fill=c] (8.76617,11.7227) rectangle (8.80597,11.8286);
\draw [color=c, fill=c] (8.80597,11.7227) rectangle (8.84577,11.8286);
\draw [color=c, fill=c] (8.84577,11.7227) rectangle (8.88557,11.8286);
\draw [color=c, fill=c] (8.88557,11.7227) rectangle (8.92537,11.8286);
\draw [color=c, fill=c] (8.92537,11.7227) rectangle (8.96517,11.8286);
\draw [color=c, fill=c] (8.96517,11.7227) rectangle (9.00498,11.8286);
\draw [color=c, fill=c] (9.00498,11.7227) rectangle (9.04478,11.8286);
\draw [color=c, fill=c] (9.04478,11.7227) rectangle (9.08458,11.8286);
\draw [color=c, fill=c] (9.08458,11.7227) rectangle (9.12438,11.8286);
\draw [color=c, fill=c] (9.12438,11.7227) rectangle (9.16418,11.8286);
\draw [color=c, fill=c] (9.16418,11.7227) rectangle (9.20398,11.8286);
\draw [color=c, fill=c] (9.20398,11.7227) rectangle (9.24378,11.8286);
\draw [color=c, fill=c] (9.24378,11.7227) rectangle (9.28358,11.8286);
\draw [color=c, fill=c] (9.28358,11.7227) rectangle (9.32338,11.8286);
\draw [color=c, fill=c] (9.32338,11.7227) rectangle (9.36318,11.8286);
\draw [color=c, fill=c] (9.36318,11.7227) rectangle (9.40298,11.8286);
\draw [color=c, fill=c] (9.40298,11.7227) rectangle (9.44279,11.8286);
\draw [color=c, fill=c] (9.44279,11.7227) rectangle (9.48259,11.8286);
\draw [color=c, fill=c] (9.48259,11.7227) rectangle (9.52239,11.8286);
\draw [color=c, fill=c] (9.52239,11.7227) rectangle (9.56219,11.8286);
\draw [color=c, fill=c] (9.56219,11.7227) rectangle (9.60199,11.8286);
\draw [color=c, fill=c] (9.60199,11.7227) rectangle (9.64179,11.8286);
\draw [color=c, fill=c] (9.64179,11.7227) rectangle (9.68159,11.8286);
\definecolor{c}{rgb}{0,0.266667,1};
\draw [color=c, fill=c] (9.68159,11.7227) rectangle (9.72139,11.8286);
\draw [color=c, fill=c] (9.72139,11.7227) rectangle (9.76119,11.8286);
\draw [color=c, fill=c] (9.76119,11.7227) rectangle (9.80099,11.8286);
\draw [color=c, fill=c] (9.80099,11.7227) rectangle (9.8408,11.8286);
\draw [color=c, fill=c] (9.8408,11.7227) rectangle (9.8806,11.8286);
\draw [color=c, fill=c] (9.8806,11.7227) rectangle (9.9204,11.8286);
\draw [color=c, fill=c] (9.9204,11.7227) rectangle (9.9602,11.8286);
\draw [color=c, fill=c] (9.9602,11.7227) rectangle (10,11.8286);
\draw [color=c, fill=c] (10,11.7227) rectangle (10.0398,11.8286);
\draw [color=c, fill=c] (10.0398,11.7227) rectangle (10.0796,11.8286);
\draw [color=c, fill=c] (10.0796,11.7227) rectangle (10.1194,11.8286);
\draw [color=c, fill=c] (10.1194,11.7227) rectangle (10.1592,11.8286);
\draw [color=c, fill=c] (10.1592,11.7227) rectangle (10.199,11.8286);
\draw [color=c, fill=c] (10.199,11.7227) rectangle (10.2388,11.8286);
\draw [color=c, fill=c] (10.2388,11.7227) rectangle (10.2786,11.8286);
\draw [color=c, fill=c] (10.2786,11.7227) rectangle (10.3184,11.8286);
\draw [color=c, fill=c] (10.3184,11.7227) rectangle (10.3582,11.8286);
\draw [color=c, fill=c] (10.3582,11.7227) rectangle (10.398,11.8286);
\draw [color=c, fill=c] (10.398,11.7227) rectangle (10.4378,11.8286);
\draw [color=c, fill=c] (10.4378,11.7227) rectangle (10.4776,11.8286);
\draw [color=c, fill=c] (10.4776,11.7227) rectangle (10.5174,11.8286);
\draw [color=c, fill=c] (10.5174,11.7227) rectangle (10.5572,11.8286);
\draw [color=c, fill=c] (10.5572,11.7227) rectangle (10.597,11.8286);
\draw [color=c, fill=c] (10.597,11.7227) rectangle (10.6368,11.8286);
\draw [color=c, fill=c] (10.6368,11.7227) rectangle (10.6766,11.8286);
\draw [color=c, fill=c] (10.6766,11.7227) rectangle (10.7164,11.8286);
\draw [color=c, fill=c] (10.7164,11.7227) rectangle (10.7562,11.8286);
\draw [color=c, fill=c] (10.7562,11.7227) rectangle (10.796,11.8286);
\draw [color=c, fill=c] (10.796,11.7227) rectangle (10.8358,11.8286);
\draw [color=c, fill=c] (10.8358,11.7227) rectangle (10.8756,11.8286);
\draw [color=c, fill=c] (10.8756,11.7227) rectangle (10.9154,11.8286);
\draw [color=c, fill=c] (10.9154,11.7227) rectangle (10.9552,11.8286);
\draw [color=c, fill=c] (10.9552,11.7227) rectangle (10.995,11.8286);
\draw [color=c, fill=c] (10.995,11.7227) rectangle (11.0348,11.8286);
\definecolor{c}{rgb}{0,0.546666,1};
\draw [color=c, fill=c] (11.0348,11.7227) rectangle (11.0746,11.8286);
\draw [color=c, fill=c] (11.0746,11.7227) rectangle (11.1144,11.8286);
\draw [color=c, fill=c] (11.1144,11.7227) rectangle (11.1542,11.8286);
\draw [color=c, fill=c] (11.1542,11.7227) rectangle (11.194,11.8286);
\draw [color=c, fill=c] (11.194,11.7227) rectangle (11.2338,11.8286);
\draw [color=c, fill=c] (11.2338,11.7227) rectangle (11.2736,11.8286);
\draw [color=c, fill=c] (11.2736,11.7227) rectangle (11.3134,11.8286);
\draw [color=c, fill=c] (11.3134,11.7227) rectangle (11.3532,11.8286);
\draw [color=c, fill=c] (11.3532,11.7227) rectangle (11.393,11.8286);
\draw [color=c, fill=c] (11.393,11.7227) rectangle (11.4328,11.8286);
\draw [color=c, fill=c] (11.4328,11.7227) rectangle (11.4726,11.8286);
\draw [color=c, fill=c] (11.4726,11.7227) rectangle (11.5124,11.8286);
\draw [color=c, fill=c] (11.5124,11.7227) rectangle (11.5522,11.8286);
\draw [color=c, fill=c] (11.5522,11.7227) rectangle (11.592,11.8286);
\draw [color=c, fill=c] (11.592,11.7227) rectangle (11.6318,11.8286);
\draw [color=c, fill=c] (11.6318,11.7227) rectangle (11.6716,11.8286);
\draw [color=c, fill=c] (11.6716,11.7227) rectangle (11.7114,11.8286);
\draw [color=c, fill=c] (11.7114,11.7227) rectangle (11.7512,11.8286);
\draw [color=c, fill=c] (11.7512,11.7227) rectangle (11.791,11.8286);
\draw [color=c, fill=c] (11.791,11.7227) rectangle (11.8308,11.8286);
\draw [color=c, fill=c] (11.8308,11.7227) rectangle (11.8706,11.8286);
\draw [color=c, fill=c] (11.8706,11.7227) rectangle (11.9104,11.8286);
\draw [color=c, fill=c] (11.9104,11.7227) rectangle (11.9502,11.8286);
\draw [color=c, fill=c] (11.9502,11.7227) rectangle (11.99,11.8286);
\draw [color=c, fill=c] (11.99,11.7227) rectangle (12.0299,11.8286);
\draw [color=c, fill=c] (12.0299,11.7227) rectangle (12.0697,11.8286);
\draw [color=c, fill=c] (12.0697,11.7227) rectangle (12.1095,11.8286);
\draw [color=c, fill=c] (12.1095,11.7227) rectangle (12.1493,11.8286);
\draw [color=c, fill=c] (12.1493,11.7227) rectangle (12.1891,11.8286);
\draw [color=c, fill=c] (12.1891,11.7227) rectangle (12.2289,11.8286);
\draw [color=c, fill=c] (12.2289,11.7227) rectangle (12.2687,11.8286);
\draw [color=c, fill=c] (12.2687,11.7227) rectangle (12.3085,11.8286);
\draw [color=c, fill=c] (12.3085,11.7227) rectangle (12.3483,11.8286);
\draw [color=c, fill=c] (12.3483,11.7227) rectangle (12.3881,11.8286);
\draw [color=c, fill=c] (12.3881,11.7227) rectangle (12.4279,11.8286);
\draw [color=c, fill=c] (12.4279,11.7227) rectangle (12.4677,11.8286);
\draw [color=c, fill=c] (12.4677,11.7227) rectangle (12.5075,11.8286);
\draw [color=c, fill=c] (12.5075,11.7227) rectangle (12.5473,11.8286);
\draw [color=c, fill=c] (12.5473,11.7227) rectangle (12.5871,11.8286);
\draw [color=c, fill=c] (12.5871,11.7227) rectangle (12.6269,11.8286);
\draw [color=c, fill=c] (12.6269,11.7227) rectangle (12.6667,11.8286);
\draw [color=c, fill=c] (12.6667,11.7227) rectangle (12.7065,11.8286);
\draw [color=c, fill=c] (12.7065,11.7227) rectangle (12.7463,11.8286);
\draw [color=c, fill=c] (12.7463,11.7227) rectangle (12.7861,11.8286);
\draw [color=c, fill=c] (12.7861,11.7227) rectangle (12.8259,11.8286);
\draw [color=c, fill=c] (12.8259,11.7227) rectangle (12.8657,11.8286);
\draw [color=c, fill=c] (12.8657,11.7227) rectangle (12.9055,11.8286);
\draw [color=c, fill=c] (12.9055,11.7227) rectangle (12.9453,11.8286);
\draw [color=c, fill=c] (12.9453,11.7227) rectangle (12.9851,11.8286);
\draw [color=c, fill=c] (12.9851,11.7227) rectangle (13.0249,11.8286);
\draw [color=c, fill=c] (13.0249,11.7227) rectangle (13.0647,11.8286);
\draw [color=c, fill=c] (13.0647,11.7227) rectangle (13.1045,11.8286);
\draw [color=c, fill=c] (13.1045,11.7227) rectangle (13.1443,11.8286);
\draw [color=c, fill=c] (13.1443,11.7227) rectangle (13.1841,11.8286);
\draw [color=c, fill=c] (13.1841,11.7227) rectangle (13.2239,11.8286);
\draw [color=c, fill=c] (13.2239,11.7227) rectangle (13.2637,11.8286);
\draw [color=c, fill=c] (13.2637,11.7227) rectangle (13.3035,11.8286);
\draw [color=c, fill=c] (13.3035,11.7227) rectangle (13.3433,11.8286);
\draw [color=c, fill=c] (13.3433,11.7227) rectangle (13.3831,11.8286);
\draw [color=c, fill=c] (13.3831,11.7227) rectangle (13.4229,11.8286);
\draw [color=c, fill=c] (13.4229,11.7227) rectangle (13.4627,11.8286);
\draw [color=c, fill=c] (13.4627,11.7227) rectangle (13.5025,11.8286);
\draw [color=c, fill=c] (13.5025,11.7227) rectangle (13.5423,11.8286);
\draw [color=c, fill=c] (13.5423,11.7227) rectangle (13.5821,11.8286);
\draw [color=c, fill=c] (13.5821,11.7227) rectangle (13.6219,11.8286);
\draw [color=c, fill=c] (13.6219,11.7227) rectangle (13.6617,11.8286);
\draw [color=c, fill=c] (13.6617,11.7227) rectangle (13.7015,11.8286);
\draw [color=c, fill=c] (13.7015,11.7227) rectangle (13.7413,11.8286);
\draw [color=c, fill=c] (13.7413,11.7227) rectangle (13.7811,11.8286);
\draw [color=c, fill=c] (13.7811,11.7227) rectangle (13.8209,11.8286);
\draw [color=c, fill=c] (13.8209,11.7227) rectangle (13.8607,11.8286);
\draw [color=c, fill=c] (13.8607,11.7227) rectangle (13.9005,11.8286);
\draw [color=c, fill=c] (13.9005,11.7227) rectangle (13.9403,11.8286);
\draw [color=c, fill=c] (13.9403,11.7227) rectangle (13.9801,11.8286);
\draw [color=c, fill=c] (13.9801,11.7227) rectangle (14.0199,11.8286);
\draw [color=c, fill=c] (14.0199,11.7227) rectangle (14.0597,11.8286);
\draw [color=c, fill=c] (14.0597,11.7227) rectangle (14.0995,11.8286);
\draw [color=c, fill=c] (14.0995,11.7227) rectangle (14.1393,11.8286);
\draw [color=c, fill=c] (14.1393,11.7227) rectangle (14.1791,11.8286);
\draw [color=c, fill=c] (14.1791,11.7227) rectangle (14.2189,11.8286);
\draw [color=c, fill=c] (14.2189,11.7227) rectangle (14.2587,11.8286);
\draw [color=c, fill=c] (14.2587,11.7227) rectangle (14.2985,11.8286);
\draw [color=c, fill=c] (14.2985,11.7227) rectangle (14.3383,11.8286);
\draw [color=c, fill=c] (14.3383,11.7227) rectangle (14.3781,11.8286);
\draw [color=c, fill=c] (14.3781,11.7227) rectangle (14.4179,11.8286);
\draw [color=c, fill=c] (14.4179,11.7227) rectangle (14.4577,11.8286);
\draw [color=c, fill=c] (14.4577,11.7227) rectangle (14.4975,11.8286);
\draw [color=c, fill=c] (14.4975,11.7227) rectangle (14.5373,11.8286);
\definecolor{c}{rgb}{0,0.733333,1};
\draw [color=c, fill=c] (14.5373,11.7227) rectangle (14.5771,11.8286);
\draw [color=c, fill=c] (14.5771,11.7227) rectangle (14.6169,11.8286);
\draw [color=c, fill=c] (14.6169,11.7227) rectangle (14.6567,11.8286);
\draw [color=c, fill=c] (14.6567,11.7227) rectangle (14.6965,11.8286);
\draw [color=c, fill=c] (14.6965,11.7227) rectangle (14.7363,11.8286);
\draw [color=c, fill=c] (14.7363,11.7227) rectangle (14.7761,11.8286);
\draw [color=c, fill=c] (14.7761,11.7227) rectangle (14.8159,11.8286);
\draw [color=c, fill=c] (14.8159,11.7227) rectangle (14.8557,11.8286);
\draw [color=c, fill=c] (14.8557,11.7227) rectangle (14.8955,11.8286);
\draw [color=c, fill=c] (14.8955,11.7227) rectangle (14.9353,11.8286);
\draw [color=c, fill=c] (14.9353,11.7227) rectangle (14.9751,11.8286);
\draw [color=c, fill=c] (14.9751,11.7227) rectangle (15.0149,11.8286);
\draw [color=c, fill=c] (15.0149,11.7227) rectangle (15.0547,11.8286);
\draw [color=c, fill=c] (15.0547,11.7227) rectangle (15.0945,11.8286);
\draw [color=c, fill=c] (15.0945,11.7227) rectangle (15.1343,11.8286);
\draw [color=c, fill=c] (15.1343,11.7227) rectangle (15.1741,11.8286);
\draw [color=c, fill=c] (15.1741,11.7227) rectangle (15.2139,11.8286);
\draw [color=c, fill=c] (15.2139,11.7227) rectangle (15.2537,11.8286);
\draw [color=c, fill=c] (15.2537,11.7227) rectangle (15.2935,11.8286);
\draw [color=c, fill=c] (15.2935,11.7227) rectangle (15.3333,11.8286);
\draw [color=c, fill=c] (15.3333,11.7227) rectangle (15.3731,11.8286);
\draw [color=c, fill=c] (15.3731,11.7227) rectangle (15.4129,11.8286);
\draw [color=c, fill=c] (15.4129,11.7227) rectangle (15.4527,11.8286);
\draw [color=c, fill=c] (15.4527,11.7227) rectangle (15.4925,11.8286);
\draw [color=c, fill=c] (15.4925,11.7227) rectangle (15.5323,11.8286);
\draw [color=c, fill=c] (15.5323,11.7227) rectangle (15.5721,11.8286);
\draw [color=c, fill=c] (15.5721,11.7227) rectangle (15.6119,11.8286);
\draw [color=c, fill=c] (15.6119,11.7227) rectangle (15.6517,11.8286);
\draw [color=c, fill=c] (15.6517,11.7227) rectangle (15.6915,11.8286);
\draw [color=c, fill=c] (15.6915,11.7227) rectangle (15.7313,11.8286);
\draw [color=c, fill=c] (15.7313,11.7227) rectangle (15.7711,11.8286);
\draw [color=c, fill=c] (15.7711,11.7227) rectangle (15.8109,11.8286);
\draw [color=c, fill=c] (15.8109,11.7227) rectangle (15.8507,11.8286);
\draw [color=c, fill=c] (15.8507,11.7227) rectangle (15.8905,11.8286);
\draw [color=c, fill=c] (15.8905,11.7227) rectangle (15.9303,11.8286);
\draw [color=c, fill=c] (15.9303,11.7227) rectangle (15.9701,11.8286);
\draw [color=c, fill=c] (15.9701,11.7227) rectangle (16.01,11.8286);
\draw [color=c, fill=c] (16.01,11.7227) rectangle (16.0498,11.8286);
\draw [color=c, fill=c] (16.0498,11.7227) rectangle (16.0896,11.8286);
\draw [color=c, fill=c] (16.0896,11.7227) rectangle (16.1294,11.8286);
\draw [color=c, fill=c] (16.1294,11.7227) rectangle (16.1692,11.8286);
\draw [color=c, fill=c] (16.1692,11.7227) rectangle (16.209,11.8286);
\draw [color=c, fill=c] (16.209,11.7227) rectangle (16.2488,11.8286);
\draw [color=c, fill=c] (16.2488,11.7227) rectangle (16.2886,11.8286);
\draw [color=c, fill=c] (16.2886,11.7227) rectangle (16.3284,11.8286);
\draw [color=c, fill=c] (16.3284,11.7227) rectangle (16.3682,11.8286);
\draw [color=c, fill=c] (16.3682,11.7227) rectangle (16.408,11.8286);
\draw [color=c, fill=c] (16.408,11.7227) rectangle (16.4478,11.8286);
\draw [color=c, fill=c] (16.4478,11.7227) rectangle (16.4876,11.8286);
\draw [color=c, fill=c] (16.4876,11.7227) rectangle (16.5274,11.8286);
\draw [color=c, fill=c] (16.5274,11.7227) rectangle (16.5672,11.8286);
\draw [color=c, fill=c] (16.5672,11.7227) rectangle (16.607,11.8286);
\draw [color=c, fill=c] (16.607,11.7227) rectangle (16.6468,11.8286);
\draw [color=c, fill=c] (16.6468,11.7227) rectangle (16.6866,11.8286);
\draw [color=c, fill=c] (16.6866,11.7227) rectangle (16.7264,11.8286);
\draw [color=c, fill=c] (16.7264,11.7227) rectangle (16.7662,11.8286);
\draw [color=c, fill=c] (16.7662,11.7227) rectangle (16.806,11.8286);
\draw [color=c, fill=c] (16.806,11.7227) rectangle (16.8458,11.8286);
\draw [color=c, fill=c] (16.8458,11.7227) rectangle (16.8856,11.8286);
\draw [color=c, fill=c] (16.8856,11.7227) rectangle (16.9254,11.8286);
\draw [color=c, fill=c] (16.9254,11.7227) rectangle (16.9652,11.8286);
\draw [color=c, fill=c] (16.9652,11.7227) rectangle (17.005,11.8286);
\draw [color=c, fill=c] (17.005,11.7227) rectangle (17.0448,11.8286);
\draw [color=c, fill=c] (17.0448,11.7227) rectangle (17.0846,11.8286);
\draw [color=c, fill=c] (17.0846,11.7227) rectangle (17.1244,11.8286);
\draw [color=c, fill=c] (17.1244,11.7227) rectangle (17.1642,11.8286);
\draw [color=c, fill=c] (17.1642,11.7227) rectangle (17.204,11.8286);
\draw [color=c, fill=c] (17.204,11.7227) rectangle (17.2438,11.8286);
\draw [color=c, fill=c] (17.2438,11.7227) rectangle (17.2836,11.8286);
\draw [color=c, fill=c] (17.2836,11.7227) rectangle (17.3234,11.8286);
\draw [color=c, fill=c] (17.3234,11.7227) rectangle (17.3632,11.8286);
\draw [color=c, fill=c] (17.3632,11.7227) rectangle (17.403,11.8286);
\draw [color=c, fill=c] (17.403,11.7227) rectangle (17.4428,11.8286);
\draw [color=c, fill=c] (17.4428,11.7227) rectangle (17.4826,11.8286);
\draw [color=c, fill=c] (17.4826,11.7227) rectangle (17.5224,11.8286);
\draw [color=c, fill=c] (17.5224,11.7227) rectangle (17.5622,11.8286);
\draw [color=c, fill=c] (17.5622,11.7227) rectangle (17.602,11.8286);
\draw [color=c, fill=c] (17.602,11.7227) rectangle (17.6418,11.8286);
\draw [color=c, fill=c] (17.6418,11.7227) rectangle (17.6816,11.8286);
\draw [color=c, fill=c] (17.6816,11.7227) rectangle (17.7214,11.8286);
\draw [color=c, fill=c] (17.7214,11.7227) rectangle (17.7612,11.8286);
\draw [color=c, fill=c] (17.7612,11.7227) rectangle (17.801,11.8286);
\draw [color=c, fill=c] (17.801,11.7227) rectangle (17.8408,11.8286);
\draw [color=c, fill=c] (17.8408,11.7227) rectangle (17.8806,11.8286);
\draw [color=c, fill=c] (17.8806,11.7227) rectangle (17.9204,11.8286);
\draw [color=c, fill=c] (17.9204,11.7227) rectangle (17.9602,11.8286);
\draw [color=c, fill=c] (17.9602,11.7227) rectangle (18,11.8286);
\definecolor{c}{rgb}{0.2,0,1};
\draw [color=c, fill=c] (2,11.8286) rectangle (2.0398,11.9344);
\draw [color=c, fill=c] (2.0398,11.8286) rectangle (2.0796,11.9344);
\draw [color=c, fill=c] (2.0796,11.8286) rectangle (2.1194,11.9344);
\draw [color=c, fill=c] (2.1194,11.8286) rectangle (2.1592,11.9344);
\draw [color=c, fill=c] (2.1592,11.8286) rectangle (2.19901,11.9344);
\draw [color=c, fill=c] (2.19901,11.8286) rectangle (2.23881,11.9344);
\draw [color=c, fill=c] (2.23881,11.8286) rectangle (2.27861,11.9344);
\draw [color=c, fill=c] (2.27861,11.8286) rectangle (2.31841,11.9344);
\draw [color=c, fill=c] (2.31841,11.8286) rectangle (2.35821,11.9344);
\draw [color=c, fill=c] (2.35821,11.8286) rectangle (2.39801,11.9344);
\draw [color=c, fill=c] (2.39801,11.8286) rectangle (2.43781,11.9344);
\draw [color=c, fill=c] (2.43781,11.8286) rectangle (2.47761,11.9344);
\draw [color=c, fill=c] (2.47761,11.8286) rectangle (2.51741,11.9344);
\draw [color=c, fill=c] (2.51741,11.8286) rectangle (2.55721,11.9344);
\draw [color=c, fill=c] (2.55721,11.8286) rectangle (2.59702,11.9344);
\draw [color=c, fill=c] (2.59702,11.8286) rectangle (2.63682,11.9344);
\draw [color=c, fill=c] (2.63682,11.8286) rectangle (2.67662,11.9344);
\draw [color=c, fill=c] (2.67662,11.8286) rectangle (2.71642,11.9344);
\draw [color=c, fill=c] (2.71642,11.8286) rectangle (2.75622,11.9344);
\draw [color=c, fill=c] (2.75622,11.8286) rectangle (2.79602,11.9344);
\draw [color=c, fill=c] (2.79602,11.8286) rectangle (2.83582,11.9344);
\draw [color=c, fill=c] (2.83582,11.8286) rectangle (2.87562,11.9344);
\draw [color=c, fill=c] (2.87562,11.8286) rectangle (2.91542,11.9344);
\draw [color=c, fill=c] (2.91542,11.8286) rectangle (2.95522,11.9344);
\draw [color=c, fill=c] (2.95522,11.8286) rectangle (2.99502,11.9344);
\draw [color=c, fill=c] (2.99502,11.8286) rectangle (3.03483,11.9344);
\draw [color=c, fill=c] (3.03483,11.8286) rectangle (3.07463,11.9344);
\draw [color=c, fill=c] (3.07463,11.8286) rectangle (3.11443,11.9344);
\draw [color=c, fill=c] (3.11443,11.8286) rectangle (3.15423,11.9344);
\draw [color=c, fill=c] (3.15423,11.8286) rectangle (3.19403,11.9344);
\draw [color=c, fill=c] (3.19403,11.8286) rectangle (3.23383,11.9344);
\draw [color=c, fill=c] (3.23383,11.8286) rectangle (3.27363,11.9344);
\draw [color=c, fill=c] (3.27363,11.8286) rectangle (3.31343,11.9344);
\draw [color=c, fill=c] (3.31343,11.8286) rectangle (3.35323,11.9344);
\draw [color=c, fill=c] (3.35323,11.8286) rectangle (3.39303,11.9344);
\draw [color=c, fill=c] (3.39303,11.8286) rectangle (3.43284,11.9344);
\draw [color=c, fill=c] (3.43284,11.8286) rectangle (3.47264,11.9344);
\draw [color=c, fill=c] (3.47264,11.8286) rectangle (3.51244,11.9344);
\draw [color=c, fill=c] (3.51244,11.8286) rectangle (3.55224,11.9344);
\draw [color=c, fill=c] (3.55224,11.8286) rectangle (3.59204,11.9344);
\draw [color=c, fill=c] (3.59204,11.8286) rectangle (3.63184,11.9344);
\draw [color=c, fill=c] (3.63184,11.8286) rectangle (3.67164,11.9344);
\draw [color=c, fill=c] (3.67164,11.8286) rectangle (3.71144,11.9344);
\draw [color=c, fill=c] (3.71144,11.8286) rectangle (3.75124,11.9344);
\draw [color=c, fill=c] (3.75124,11.8286) rectangle (3.79104,11.9344);
\draw [color=c, fill=c] (3.79104,11.8286) rectangle (3.83085,11.9344);
\draw [color=c, fill=c] (3.83085,11.8286) rectangle (3.87065,11.9344);
\draw [color=c, fill=c] (3.87065,11.8286) rectangle (3.91045,11.9344);
\draw [color=c, fill=c] (3.91045,11.8286) rectangle (3.95025,11.9344);
\draw [color=c, fill=c] (3.95025,11.8286) rectangle (3.99005,11.9344);
\draw [color=c, fill=c] (3.99005,11.8286) rectangle (4.02985,11.9344);
\draw [color=c, fill=c] (4.02985,11.8286) rectangle (4.06965,11.9344);
\draw [color=c, fill=c] (4.06965,11.8286) rectangle (4.10945,11.9344);
\draw [color=c, fill=c] (4.10945,11.8286) rectangle (4.14925,11.9344);
\draw [color=c, fill=c] (4.14925,11.8286) rectangle (4.18905,11.9344);
\draw [color=c, fill=c] (4.18905,11.8286) rectangle (4.22886,11.9344);
\draw [color=c, fill=c] (4.22886,11.8286) rectangle (4.26866,11.9344);
\draw [color=c, fill=c] (4.26866,11.8286) rectangle (4.30846,11.9344);
\draw [color=c, fill=c] (4.30846,11.8286) rectangle (4.34826,11.9344);
\draw [color=c, fill=c] (4.34826,11.8286) rectangle (4.38806,11.9344);
\draw [color=c, fill=c] (4.38806,11.8286) rectangle (4.42786,11.9344);
\draw [color=c, fill=c] (4.42786,11.8286) rectangle (4.46766,11.9344);
\draw [color=c, fill=c] (4.46766,11.8286) rectangle (4.50746,11.9344);
\draw [color=c, fill=c] (4.50746,11.8286) rectangle (4.54726,11.9344);
\draw [color=c, fill=c] (4.54726,11.8286) rectangle (4.58706,11.9344);
\draw [color=c, fill=c] (4.58706,11.8286) rectangle (4.62687,11.9344);
\draw [color=c, fill=c] (4.62687,11.8286) rectangle (4.66667,11.9344);
\draw [color=c, fill=c] (4.66667,11.8286) rectangle (4.70647,11.9344);
\draw [color=c, fill=c] (4.70647,11.8286) rectangle (4.74627,11.9344);
\draw [color=c, fill=c] (4.74627,11.8286) rectangle (4.78607,11.9344);
\draw [color=c, fill=c] (4.78607,11.8286) rectangle (4.82587,11.9344);
\draw [color=c, fill=c] (4.82587,11.8286) rectangle (4.86567,11.9344);
\draw [color=c, fill=c] (4.86567,11.8286) rectangle (4.90547,11.9344);
\draw [color=c, fill=c] (4.90547,11.8286) rectangle (4.94527,11.9344);
\draw [color=c, fill=c] (4.94527,11.8286) rectangle (4.98507,11.9344);
\draw [color=c, fill=c] (4.98507,11.8286) rectangle (5.02488,11.9344);
\draw [color=c, fill=c] (5.02488,11.8286) rectangle (5.06468,11.9344);
\draw [color=c, fill=c] (5.06468,11.8286) rectangle (5.10448,11.9344);
\draw [color=c, fill=c] (5.10448,11.8286) rectangle (5.14428,11.9344);
\draw [color=c, fill=c] (5.14428,11.8286) rectangle (5.18408,11.9344);
\draw [color=c, fill=c] (5.18408,11.8286) rectangle (5.22388,11.9344);
\draw [color=c, fill=c] (5.22388,11.8286) rectangle (5.26368,11.9344);
\draw [color=c, fill=c] (5.26368,11.8286) rectangle (5.30348,11.9344);
\draw [color=c, fill=c] (5.30348,11.8286) rectangle (5.34328,11.9344);
\draw [color=c, fill=c] (5.34328,11.8286) rectangle (5.38308,11.9344);
\draw [color=c, fill=c] (5.38308,11.8286) rectangle (5.42289,11.9344);
\draw [color=c, fill=c] (5.42289,11.8286) rectangle (5.46269,11.9344);
\draw [color=c, fill=c] (5.46269,11.8286) rectangle (5.50249,11.9344);
\draw [color=c, fill=c] (5.50249,11.8286) rectangle (5.54229,11.9344);
\draw [color=c, fill=c] (5.54229,11.8286) rectangle (5.58209,11.9344);
\draw [color=c, fill=c] (5.58209,11.8286) rectangle (5.62189,11.9344);
\draw [color=c, fill=c] (5.62189,11.8286) rectangle (5.66169,11.9344);
\draw [color=c, fill=c] (5.66169,11.8286) rectangle (5.70149,11.9344);
\draw [color=c, fill=c] (5.70149,11.8286) rectangle (5.74129,11.9344);
\draw [color=c, fill=c] (5.74129,11.8286) rectangle (5.78109,11.9344);
\draw [color=c, fill=c] (5.78109,11.8286) rectangle (5.8209,11.9344);
\draw [color=c, fill=c] (5.8209,11.8286) rectangle (5.8607,11.9344);
\draw [color=c, fill=c] (5.8607,11.8286) rectangle (5.9005,11.9344);
\draw [color=c, fill=c] (5.9005,11.8286) rectangle (5.9403,11.9344);
\draw [color=c, fill=c] (5.9403,11.8286) rectangle (5.9801,11.9344);
\draw [color=c, fill=c] (5.9801,11.8286) rectangle (6.0199,11.9344);
\draw [color=c, fill=c] (6.0199,11.8286) rectangle (6.0597,11.9344);
\draw [color=c, fill=c] (6.0597,11.8286) rectangle (6.0995,11.9344);
\draw [color=c, fill=c] (6.0995,11.8286) rectangle (6.1393,11.9344);
\draw [color=c, fill=c] (6.1393,11.8286) rectangle (6.1791,11.9344);
\draw [color=c, fill=c] (6.1791,11.8286) rectangle (6.21891,11.9344);
\draw [color=c, fill=c] (6.21891,11.8286) rectangle (6.25871,11.9344);
\draw [color=c, fill=c] (6.25871,11.8286) rectangle (6.29851,11.9344);
\draw [color=c, fill=c] (6.29851,11.8286) rectangle (6.33831,11.9344);
\draw [color=c, fill=c] (6.33831,11.8286) rectangle (6.37811,11.9344);
\draw [color=c, fill=c] (6.37811,11.8286) rectangle (6.41791,11.9344);
\draw [color=c, fill=c] (6.41791,11.8286) rectangle (6.45771,11.9344);
\draw [color=c, fill=c] (6.45771,11.8286) rectangle (6.49751,11.9344);
\draw [color=c, fill=c] (6.49751,11.8286) rectangle (6.53731,11.9344);
\draw [color=c, fill=c] (6.53731,11.8286) rectangle (6.57711,11.9344);
\draw [color=c, fill=c] (6.57711,11.8286) rectangle (6.61692,11.9344);
\draw [color=c, fill=c] (6.61692,11.8286) rectangle (6.65672,11.9344);
\draw [color=c, fill=c] (6.65672,11.8286) rectangle (6.69652,11.9344);
\draw [color=c, fill=c] (6.69652,11.8286) rectangle (6.73632,11.9344);
\draw [color=c, fill=c] (6.73632,11.8286) rectangle (6.77612,11.9344);
\draw [color=c, fill=c] (6.77612,11.8286) rectangle (6.81592,11.9344);
\draw [color=c, fill=c] (6.81592,11.8286) rectangle (6.85572,11.9344);
\draw [color=c, fill=c] (6.85572,11.8286) rectangle (6.89552,11.9344);
\draw [color=c, fill=c] (6.89552,11.8286) rectangle (6.93532,11.9344);
\draw [color=c, fill=c] (6.93532,11.8286) rectangle (6.97512,11.9344);
\draw [color=c, fill=c] (6.97512,11.8286) rectangle (7.01493,11.9344);
\draw [color=c, fill=c] (7.01493,11.8286) rectangle (7.05473,11.9344);
\draw [color=c, fill=c] (7.05473,11.8286) rectangle (7.09453,11.9344);
\draw [color=c, fill=c] (7.09453,11.8286) rectangle (7.13433,11.9344);
\draw [color=c, fill=c] (7.13433,11.8286) rectangle (7.17413,11.9344);
\draw [color=c, fill=c] (7.17413,11.8286) rectangle (7.21393,11.9344);
\draw [color=c, fill=c] (7.21393,11.8286) rectangle (7.25373,11.9344);
\draw [color=c, fill=c] (7.25373,11.8286) rectangle (7.29353,11.9344);
\draw [color=c, fill=c] (7.29353,11.8286) rectangle (7.33333,11.9344);
\draw [color=c, fill=c] (7.33333,11.8286) rectangle (7.37313,11.9344);
\draw [color=c, fill=c] (7.37313,11.8286) rectangle (7.41294,11.9344);
\draw [color=c, fill=c] (7.41294,11.8286) rectangle (7.45274,11.9344);
\draw [color=c, fill=c] (7.45274,11.8286) rectangle (7.49254,11.9344);
\draw [color=c, fill=c] (7.49254,11.8286) rectangle (7.53234,11.9344);
\draw [color=c, fill=c] (7.53234,11.8286) rectangle (7.57214,11.9344);
\draw [color=c, fill=c] (7.57214,11.8286) rectangle (7.61194,11.9344);
\draw [color=c, fill=c] (7.61194,11.8286) rectangle (7.65174,11.9344);
\draw [color=c, fill=c] (7.65174,11.8286) rectangle (7.69154,11.9344);
\draw [color=c, fill=c] (7.69154,11.8286) rectangle (7.73134,11.9344);
\draw [color=c, fill=c] (7.73134,11.8286) rectangle (7.77114,11.9344);
\draw [color=c, fill=c] (7.77114,11.8286) rectangle (7.81095,11.9344);
\draw [color=c, fill=c] (7.81095,11.8286) rectangle (7.85075,11.9344);
\draw [color=c, fill=c] (7.85075,11.8286) rectangle (7.89055,11.9344);
\draw [color=c, fill=c] (7.89055,11.8286) rectangle (7.93035,11.9344);
\draw [color=c, fill=c] (7.93035,11.8286) rectangle (7.97015,11.9344);
\draw [color=c, fill=c] (7.97015,11.8286) rectangle (8.00995,11.9344);
\draw [color=c, fill=c] (8.00995,11.8286) rectangle (8.04975,11.9344);
\draw [color=c, fill=c] (8.04975,11.8286) rectangle (8.08955,11.9344);
\draw [color=c, fill=c] (8.08955,11.8286) rectangle (8.12935,11.9344);
\draw [color=c, fill=c] (8.12935,11.8286) rectangle (8.16915,11.9344);
\definecolor{c}{rgb}{0,0.0800001,1};
\draw [color=c, fill=c] (8.16915,11.8286) rectangle (8.20895,11.9344);
\draw [color=c, fill=c] (8.20895,11.8286) rectangle (8.24876,11.9344);
\draw [color=c, fill=c] (8.24876,11.8286) rectangle (8.28856,11.9344);
\draw [color=c, fill=c] (8.28856,11.8286) rectangle (8.32836,11.9344);
\draw [color=c, fill=c] (8.32836,11.8286) rectangle (8.36816,11.9344);
\draw [color=c, fill=c] (8.36816,11.8286) rectangle (8.40796,11.9344);
\draw [color=c, fill=c] (8.40796,11.8286) rectangle (8.44776,11.9344);
\draw [color=c, fill=c] (8.44776,11.8286) rectangle (8.48756,11.9344);
\draw [color=c, fill=c] (8.48756,11.8286) rectangle (8.52736,11.9344);
\draw [color=c, fill=c] (8.52736,11.8286) rectangle (8.56716,11.9344);
\draw [color=c, fill=c] (8.56716,11.8286) rectangle (8.60697,11.9344);
\draw [color=c, fill=c] (8.60697,11.8286) rectangle (8.64677,11.9344);
\draw [color=c, fill=c] (8.64677,11.8286) rectangle (8.68657,11.9344);
\draw [color=c, fill=c] (8.68657,11.8286) rectangle (8.72637,11.9344);
\draw [color=c, fill=c] (8.72637,11.8286) rectangle (8.76617,11.9344);
\draw [color=c, fill=c] (8.76617,11.8286) rectangle (8.80597,11.9344);
\draw [color=c, fill=c] (8.80597,11.8286) rectangle (8.84577,11.9344);
\draw [color=c, fill=c] (8.84577,11.8286) rectangle (8.88557,11.9344);
\draw [color=c, fill=c] (8.88557,11.8286) rectangle (8.92537,11.9344);
\draw [color=c, fill=c] (8.92537,11.8286) rectangle (8.96517,11.9344);
\draw [color=c, fill=c] (8.96517,11.8286) rectangle (9.00498,11.9344);
\draw [color=c, fill=c] (9.00498,11.8286) rectangle (9.04478,11.9344);
\draw [color=c, fill=c] (9.04478,11.8286) rectangle (9.08458,11.9344);
\draw [color=c, fill=c] (9.08458,11.8286) rectangle (9.12438,11.9344);
\draw [color=c, fill=c] (9.12438,11.8286) rectangle (9.16418,11.9344);
\draw [color=c, fill=c] (9.16418,11.8286) rectangle (9.20398,11.9344);
\draw [color=c, fill=c] (9.20398,11.8286) rectangle (9.24378,11.9344);
\draw [color=c, fill=c] (9.24378,11.8286) rectangle (9.28358,11.9344);
\draw [color=c, fill=c] (9.28358,11.8286) rectangle (9.32338,11.9344);
\draw [color=c, fill=c] (9.32338,11.8286) rectangle (9.36318,11.9344);
\draw [color=c, fill=c] (9.36318,11.8286) rectangle (9.40298,11.9344);
\draw [color=c, fill=c] (9.40298,11.8286) rectangle (9.44279,11.9344);
\draw [color=c, fill=c] (9.44279,11.8286) rectangle (9.48259,11.9344);
\draw [color=c, fill=c] (9.48259,11.8286) rectangle (9.52239,11.9344);
\draw [color=c, fill=c] (9.52239,11.8286) rectangle (9.56219,11.9344);
\draw [color=c, fill=c] (9.56219,11.8286) rectangle (9.60199,11.9344);
\draw [color=c, fill=c] (9.60199,11.8286) rectangle (9.64179,11.9344);
\draw [color=c, fill=c] (9.64179,11.8286) rectangle (9.68159,11.9344);
\draw [color=c, fill=c] (9.68159,11.8286) rectangle (9.72139,11.9344);
\definecolor{c}{rgb}{0,0.266667,1};
\draw [color=c, fill=c] (9.72139,11.8286) rectangle (9.76119,11.9344);
\draw [color=c, fill=c] (9.76119,11.8286) rectangle (9.80099,11.9344);
\draw [color=c, fill=c] (9.80099,11.8286) rectangle (9.8408,11.9344);
\draw [color=c, fill=c] (9.8408,11.8286) rectangle (9.8806,11.9344);
\draw [color=c, fill=c] (9.8806,11.8286) rectangle (9.9204,11.9344);
\draw [color=c, fill=c] (9.9204,11.8286) rectangle (9.9602,11.9344);
\draw [color=c, fill=c] (9.9602,11.8286) rectangle (10,11.9344);
\draw [color=c, fill=c] (10,11.8286) rectangle (10.0398,11.9344);
\draw [color=c, fill=c] (10.0398,11.8286) rectangle (10.0796,11.9344);
\draw [color=c, fill=c] (10.0796,11.8286) rectangle (10.1194,11.9344);
\draw [color=c, fill=c] (10.1194,11.8286) rectangle (10.1592,11.9344);
\draw [color=c, fill=c] (10.1592,11.8286) rectangle (10.199,11.9344);
\draw [color=c, fill=c] (10.199,11.8286) rectangle (10.2388,11.9344);
\draw [color=c, fill=c] (10.2388,11.8286) rectangle (10.2786,11.9344);
\draw [color=c, fill=c] (10.2786,11.8286) rectangle (10.3184,11.9344);
\draw [color=c, fill=c] (10.3184,11.8286) rectangle (10.3582,11.9344);
\draw [color=c, fill=c] (10.3582,11.8286) rectangle (10.398,11.9344);
\draw [color=c, fill=c] (10.398,11.8286) rectangle (10.4378,11.9344);
\draw [color=c, fill=c] (10.4378,11.8286) rectangle (10.4776,11.9344);
\draw [color=c, fill=c] (10.4776,11.8286) rectangle (10.5174,11.9344);
\draw [color=c, fill=c] (10.5174,11.8286) rectangle (10.5572,11.9344);
\draw [color=c, fill=c] (10.5572,11.8286) rectangle (10.597,11.9344);
\draw [color=c, fill=c] (10.597,11.8286) rectangle (10.6368,11.9344);
\draw [color=c, fill=c] (10.6368,11.8286) rectangle (10.6766,11.9344);
\draw [color=c, fill=c] (10.6766,11.8286) rectangle (10.7164,11.9344);
\draw [color=c, fill=c] (10.7164,11.8286) rectangle (10.7562,11.9344);
\draw [color=c, fill=c] (10.7562,11.8286) rectangle (10.796,11.9344);
\draw [color=c, fill=c] (10.796,11.8286) rectangle (10.8358,11.9344);
\draw [color=c, fill=c] (10.8358,11.8286) rectangle (10.8756,11.9344);
\draw [color=c, fill=c] (10.8756,11.8286) rectangle (10.9154,11.9344);
\draw [color=c, fill=c] (10.9154,11.8286) rectangle (10.9552,11.9344);
\draw [color=c, fill=c] (10.9552,11.8286) rectangle (10.995,11.9344);
\draw [color=c, fill=c] (10.995,11.8286) rectangle (11.0348,11.9344);
\definecolor{c}{rgb}{0,0.546666,1};
\draw [color=c, fill=c] (11.0348,11.8286) rectangle (11.0746,11.9344);
\draw [color=c, fill=c] (11.0746,11.8286) rectangle (11.1144,11.9344);
\draw [color=c, fill=c] (11.1144,11.8286) rectangle (11.1542,11.9344);
\draw [color=c, fill=c] (11.1542,11.8286) rectangle (11.194,11.9344);
\draw [color=c, fill=c] (11.194,11.8286) rectangle (11.2338,11.9344);
\draw [color=c, fill=c] (11.2338,11.8286) rectangle (11.2736,11.9344);
\draw [color=c, fill=c] (11.2736,11.8286) rectangle (11.3134,11.9344);
\draw [color=c, fill=c] (11.3134,11.8286) rectangle (11.3532,11.9344);
\draw [color=c, fill=c] (11.3532,11.8286) rectangle (11.393,11.9344);
\draw [color=c, fill=c] (11.393,11.8286) rectangle (11.4328,11.9344);
\draw [color=c, fill=c] (11.4328,11.8286) rectangle (11.4726,11.9344);
\draw [color=c, fill=c] (11.4726,11.8286) rectangle (11.5124,11.9344);
\draw [color=c, fill=c] (11.5124,11.8286) rectangle (11.5522,11.9344);
\draw [color=c, fill=c] (11.5522,11.8286) rectangle (11.592,11.9344);
\draw [color=c, fill=c] (11.592,11.8286) rectangle (11.6318,11.9344);
\draw [color=c, fill=c] (11.6318,11.8286) rectangle (11.6716,11.9344);
\draw [color=c, fill=c] (11.6716,11.8286) rectangle (11.7114,11.9344);
\draw [color=c, fill=c] (11.7114,11.8286) rectangle (11.7512,11.9344);
\draw [color=c, fill=c] (11.7512,11.8286) rectangle (11.791,11.9344);
\draw [color=c, fill=c] (11.791,11.8286) rectangle (11.8308,11.9344);
\draw [color=c, fill=c] (11.8308,11.8286) rectangle (11.8706,11.9344);
\draw [color=c, fill=c] (11.8706,11.8286) rectangle (11.9104,11.9344);
\draw [color=c, fill=c] (11.9104,11.8286) rectangle (11.9502,11.9344);
\draw [color=c, fill=c] (11.9502,11.8286) rectangle (11.99,11.9344);
\draw [color=c, fill=c] (11.99,11.8286) rectangle (12.0299,11.9344);
\draw [color=c, fill=c] (12.0299,11.8286) rectangle (12.0697,11.9344);
\draw [color=c, fill=c] (12.0697,11.8286) rectangle (12.1095,11.9344);
\draw [color=c, fill=c] (12.1095,11.8286) rectangle (12.1493,11.9344);
\draw [color=c, fill=c] (12.1493,11.8286) rectangle (12.1891,11.9344);
\draw [color=c, fill=c] (12.1891,11.8286) rectangle (12.2289,11.9344);
\draw [color=c, fill=c] (12.2289,11.8286) rectangle (12.2687,11.9344);
\draw [color=c, fill=c] (12.2687,11.8286) rectangle (12.3085,11.9344);
\draw [color=c, fill=c] (12.3085,11.8286) rectangle (12.3483,11.9344);
\draw [color=c, fill=c] (12.3483,11.8286) rectangle (12.3881,11.9344);
\draw [color=c, fill=c] (12.3881,11.8286) rectangle (12.4279,11.9344);
\draw [color=c, fill=c] (12.4279,11.8286) rectangle (12.4677,11.9344);
\draw [color=c, fill=c] (12.4677,11.8286) rectangle (12.5075,11.9344);
\draw [color=c, fill=c] (12.5075,11.8286) rectangle (12.5473,11.9344);
\draw [color=c, fill=c] (12.5473,11.8286) rectangle (12.5871,11.9344);
\draw [color=c, fill=c] (12.5871,11.8286) rectangle (12.6269,11.9344);
\draw [color=c, fill=c] (12.6269,11.8286) rectangle (12.6667,11.9344);
\draw [color=c, fill=c] (12.6667,11.8286) rectangle (12.7065,11.9344);
\draw [color=c, fill=c] (12.7065,11.8286) rectangle (12.7463,11.9344);
\draw [color=c, fill=c] (12.7463,11.8286) rectangle (12.7861,11.9344);
\draw [color=c, fill=c] (12.7861,11.8286) rectangle (12.8259,11.9344);
\draw [color=c, fill=c] (12.8259,11.8286) rectangle (12.8657,11.9344);
\draw [color=c, fill=c] (12.8657,11.8286) rectangle (12.9055,11.9344);
\draw [color=c, fill=c] (12.9055,11.8286) rectangle (12.9453,11.9344);
\draw [color=c, fill=c] (12.9453,11.8286) rectangle (12.9851,11.9344);
\draw [color=c, fill=c] (12.9851,11.8286) rectangle (13.0249,11.9344);
\draw [color=c, fill=c] (13.0249,11.8286) rectangle (13.0647,11.9344);
\draw [color=c, fill=c] (13.0647,11.8286) rectangle (13.1045,11.9344);
\draw [color=c, fill=c] (13.1045,11.8286) rectangle (13.1443,11.9344);
\draw [color=c, fill=c] (13.1443,11.8286) rectangle (13.1841,11.9344);
\draw [color=c, fill=c] (13.1841,11.8286) rectangle (13.2239,11.9344);
\draw [color=c, fill=c] (13.2239,11.8286) rectangle (13.2637,11.9344);
\draw [color=c, fill=c] (13.2637,11.8286) rectangle (13.3035,11.9344);
\draw [color=c, fill=c] (13.3035,11.8286) rectangle (13.3433,11.9344);
\draw [color=c, fill=c] (13.3433,11.8286) rectangle (13.3831,11.9344);
\draw [color=c, fill=c] (13.3831,11.8286) rectangle (13.4229,11.9344);
\draw [color=c, fill=c] (13.4229,11.8286) rectangle (13.4627,11.9344);
\draw [color=c, fill=c] (13.4627,11.8286) rectangle (13.5025,11.9344);
\draw [color=c, fill=c] (13.5025,11.8286) rectangle (13.5423,11.9344);
\draw [color=c, fill=c] (13.5423,11.8286) rectangle (13.5821,11.9344);
\draw [color=c, fill=c] (13.5821,11.8286) rectangle (13.6219,11.9344);
\draw [color=c, fill=c] (13.6219,11.8286) rectangle (13.6617,11.9344);
\draw [color=c, fill=c] (13.6617,11.8286) rectangle (13.7015,11.9344);
\draw [color=c, fill=c] (13.7015,11.8286) rectangle (13.7413,11.9344);
\draw [color=c, fill=c] (13.7413,11.8286) rectangle (13.7811,11.9344);
\draw [color=c, fill=c] (13.7811,11.8286) rectangle (13.8209,11.9344);
\draw [color=c, fill=c] (13.8209,11.8286) rectangle (13.8607,11.9344);
\draw [color=c, fill=c] (13.8607,11.8286) rectangle (13.9005,11.9344);
\draw [color=c, fill=c] (13.9005,11.8286) rectangle (13.9403,11.9344);
\draw [color=c, fill=c] (13.9403,11.8286) rectangle (13.9801,11.9344);
\draw [color=c, fill=c] (13.9801,11.8286) rectangle (14.0199,11.9344);
\draw [color=c, fill=c] (14.0199,11.8286) rectangle (14.0597,11.9344);
\draw [color=c, fill=c] (14.0597,11.8286) rectangle (14.0995,11.9344);
\draw [color=c, fill=c] (14.0995,11.8286) rectangle (14.1393,11.9344);
\draw [color=c, fill=c] (14.1393,11.8286) rectangle (14.1791,11.9344);
\draw [color=c, fill=c] (14.1791,11.8286) rectangle (14.2189,11.9344);
\draw [color=c, fill=c] (14.2189,11.8286) rectangle (14.2587,11.9344);
\draw [color=c, fill=c] (14.2587,11.8286) rectangle (14.2985,11.9344);
\draw [color=c, fill=c] (14.2985,11.8286) rectangle (14.3383,11.9344);
\draw [color=c, fill=c] (14.3383,11.8286) rectangle (14.3781,11.9344);
\draw [color=c, fill=c] (14.3781,11.8286) rectangle (14.4179,11.9344);
\draw [color=c, fill=c] (14.4179,11.8286) rectangle (14.4577,11.9344);
\draw [color=c, fill=c] (14.4577,11.8286) rectangle (14.4975,11.9344);
\draw [color=c, fill=c] (14.4975,11.8286) rectangle (14.5373,11.9344);
\draw [color=c, fill=c] (14.5373,11.8286) rectangle (14.5771,11.9344);
\definecolor{c}{rgb}{0,0.733333,1};
\draw [color=c, fill=c] (14.5771,11.8286) rectangle (14.6169,11.9344);
\draw [color=c, fill=c] (14.6169,11.8286) rectangle (14.6567,11.9344);
\draw [color=c, fill=c] (14.6567,11.8286) rectangle (14.6965,11.9344);
\draw [color=c, fill=c] (14.6965,11.8286) rectangle (14.7363,11.9344);
\draw [color=c, fill=c] (14.7363,11.8286) rectangle (14.7761,11.9344);
\draw [color=c, fill=c] (14.7761,11.8286) rectangle (14.8159,11.9344);
\draw [color=c, fill=c] (14.8159,11.8286) rectangle (14.8557,11.9344);
\draw [color=c, fill=c] (14.8557,11.8286) rectangle (14.8955,11.9344);
\draw [color=c, fill=c] (14.8955,11.8286) rectangle (14.9353,11.9344);
\draw [color=c, fill=c] (14.9353,11.8286) rectangle (14.9751,11.9344);
\draw [color=c, fill=c] (14.9751,11.8286) rectangle (15.0149,11.9344);
\draw [color=c, fill=c] (15.0149,11.8286) rectangle (15.0547,11.9344);
\draw [color=c, fill=c] (15.0547,11.8286) rectangle (15.0945,11.9344);
\draw [color=c, fill=c] (15.0945,11.8286) rectangle (15.1343,11.9344);
\draw [color=c, fill=c] (15.1343,11.8286) rectangle (15.1741,11.9344);
\draw [color=c, fill=c] (15.1741,11.8286) rectangle (15.2139,11.9344);
\draw [color=c, fill=c] (15.2139,11.8286) rectangle (15.2537,11.9344);
\draw [color=c, fill=c] (15.2537,11.8286) rectangle (15.2935,11.9344);
\draw [color=c, fill=c] (15.2935,11.8286) rectangle (15.3333,11.9344);
\draw [color=c, fill=c] (15.3333,11.8286) rectangle (15.3731,11.9344);
\draw [color=c, fill=c] (15.3731,11.8286) rectangle (15.4129,11.9344);
\draw [color=c, fill=c] (15.4129,11.8286) rectangle (15.4527,11.9344);
\draw [color=c, fill=c] (15.4527,11.8286) rectangle (15.4925,11.9344);
\draw [color=c, fill=c] (15.4925,11.8286) rectangle (15.5323,11.9344);
\draw [color=c, fill=c] (15.5323,11.8286) rectangle (15.5721,11.9344);
\draw [color=c, fill=c] (15.5721,11.8286) rectangle (15.6119,11.9344);
\draw [color=c, fill=c] (15.6119,11.8286) rectangle (15.6517,11.9344);
\draw [color=c, fill=c] (15.6517,11.8286) rectangle (15.6915,11.9344);
\draw [color=c, fill=c] (15.6915,11.8286) rectangle (15.7313,11.9344);
\draw [color=c, fill=c] (15.7313,11.8286) rectangle (15.7711,11.9344);
\draw [color=c, fill=c] (15.7711,11.8286) rectangle (15.8109,11.9344);
\draw [color=c, fill=c] (15.8109,11.8286) rectangle (15.8507,11.9344);
\draw [color=c, fill=c] (15.8507,11.8286) rectangle (15.8905,11.9344);
\draw [color=c, fill=c] (15.8905,11.8286) rectangle (15.9303,11.9344);
\draw [color=c, fill=c] (15.9303,11.8286) rectangle (15.9701,11.9344);
\draw [color=c, fill=c] (15.9701,11.8286) rectangle (16.01,11.9344);
\draw [color=c, fill=c] (16.01,11.8286) rectangle (16.0498,11.9344);
\draw [color=c, fill=c] (16.0498,11.8286) rectangle (16.0896,11.9344);
\draw [color=c, fill=c] (16.0896,11.8286) rectangle (16.1294,11.9344);
\draw [color=c, fill=c] (16.1294,11.8286) rectangle (16.1692,11.9344);
\draw [color=c, fill=c] (16.1692,11.8286) rectangle (16.209,11.9344);
\draw [color=c, fill=c] (16.209,11.8286) rectangle (16.2488,11.9344);
\draw [color=c, fill=c] (16.2488,11.8286) rectangle (16.2886,11.9344);
\draw [color=c, fill=c] (16.2886,11.8286) rectangle (16.3284,11.9344);
\draw [color=c, fill=c] (16.3284,11.8286) rectangle (16.3682,11.9344);
\draw [color=c, fill=c] (16.3682,11.8286) rectangle (16.408,11.9344);
\draw [color=c, fill=c] (16.408,11.8286) rectangle (16.4478,11.9344);
\draw [color=c, fill=c] (16.4478,11.8286) rectangle (16.4876,11.9344);
\draw [color=c, fill=c] (16.4876,11.8286) rectangle (16.5274,11.9344);
\draw [color=c, fill=c] (16.5274,11.8286) rectangle (16.5672,11.9344);
\draw [color=c, fill=c] (16.5672,11.8286) rectangle (16.607,11.9344);
\draw [color=c, fill=c] (16.607,11.8286) rectangle (16.6468,11.9344);
\draw [color=c, fill=c] (16.6468,11.8286) rectangle (16.6866,11.9344);
\draw [color=c, fill=c] (16.6866,11.8286) rectangle (16.7264,11.9344);
\draw [color=c, fill=c] (16.7264,11.8286) rectangle (16.7662,11.9344);
\draw [color=c, fill=c] (16.7662,11.8286) rectangle (16.806,11.9344);
\draw [color=c, fill=c] (16.806,11.8286) rectangle (16.8458,11.9344);
\draw [color=c, fill=c] (16.8458,11.8286) rectangle (16.8856,11.9344);
\draw [color=c, fill=c] (16.8856,11.8286) rectangle (16.9254,11.9344);
\draw [color=c, fill=c] (16.9254,11.8286) rectangle (16.9652,11.9344);
\draw [color=c, fill=c] (16.9652,11.8286) rectangle (17.005,11.9344);
\draw [color=c, fill=c] (17.005,11.8286) rectangle (17.0448,11.9344);
\draw [color=c, fill=c] (17.0448,11.8286) rectangle (17.0846,11.9344);
\draw [color=c, fill=c] (17.0846,11.8286) rectangle (17.1244,11.9344);
\draw [color=c, fill=c] (17.1244,11.8286) rectangle (17.1642,11.9344);
\draw [color=c, fill=c] (17.1642,11.8286) rectangle (17.204,11.9344);
\draw [color=c, fill=c] (17.204,11.8286) rectangle (17.2438,11.9344);
\draw [color=c, fill=c] (17.2438,11.8286) rectangle (17.2836,11.9344);
\draw [color=c, fill=c] (17.2836,11.8286) rectangle (17.3234,11.9344);
\draw [color=c, fill=c] (17.3234,11.8286) rectangle (17.3632,11.9344);
\draw [color=c, fill=c] (17.3632,11.8286) rectangle (17.403,11.9344);
\draw [color=c, fill=c] (17.403,11.8286) rectangle (17.4428,11.9344);
\draw [color=c, fill=c] (17.4428,11.8286) rectangle (17.4826,11.9344);
\draw [color=c, fill=c] (17.4826,11.8286) rectangle (17.5224,11.9344);
\draw [color=c, fill=c] (17.5224,11.8286) rectangle (17.5622,11.9344);
\draw [color=c, fill=c] (17.5622,11.8286) rectangle (17.602,11.9344);
\draw [color=c, fill=c] (17.602,11.8286) rectangle (17.6418,11.9344);
\draw [color=c, fill=c] (17.6418,11.8286) rectangle (17.6816,11.9344);
\draw [color=c, fill=c] (17.6816,11.8286) rectangle (17.7214,11.9344);
\draw [color=c, fill=c] (17.7214,11.8286) rectangle (17.7612,11.9344);
\draw [color=c, fill=c] (17.7612,11.8286) rectangle (17.801,11.9344);
\draw [color=c, fill=c] (17.801,11.8286) rectangle (17.8408,11.9344);
\draw [color=c, fill=c] (17.8408,11.8286) rectangle (17.8806,11.9344);
\draw [color=c, fill=c] (17.8806,11.8286) rectangle (17.9204,11.9344);
\draw [color=c, fill=c] (17.9204,11.8286) rectangle (17.9602,11.9344);
\draw [color=c, fill=c] (17.9602,11.8286) rectangle (18,11.9344);
\definecolor{c}{rgb}{0.2,0,1};
\draw [color=c, fill=c] (2,11.9344) rectangle (2.0398,12.0403);
\draw [color=c, fill=c] (2.0398,11.9344) rectangle (2.0796,12.0403);
\draw [color=c, fill=c] (2.0796,11.9344) rectangle (2.1194,12.0403);
\draw [color=c, fill=c] (2.1194,11.9344) rectangle (2.1592,12.0403);
\draw [color=c, fill=c] (2.1592,11.9344) rectangle (2.19901,12.0403);
\draw [color=c, fill=c] (2.19901,11.9344) rectangle (2.23881,12.0403);
\draw [color=c, fill=c] (2.23881,11.9344) rectangle (2.27861,12.0403);
\draw [color=c, fill=c] (2.27861,11.9344) rectangle (2.31841,12.0403);
\draw [color=c, fill=c] (2.31841,11.9344) rectangle (2.35821,12.0403);
\draw [color=c, fill=c] (2.35821,11.9344) rectangle (2.39801,12.0403);
\draw [color=c, fill=c] (2.39801,11.9344) rectangle (2.43781,12.0403);
\draw [color=c, fill=c] (2.43781,11.9344) rectangle (2.47761,12.0403);
\draw [color=c, fill=c] (2.47761,11.9344) rectangle (2.51741,12.0403);
\draw [color=c, fill=c] (2.51741,11.9344) rectangle (2.55721,12.0403);
\draw [color=c, fill=c] (2.55721,11.9344) rectangle (2.59702,12.0403);
\draw [color=c, fill=c] (2.59702,11.9344) rectangle (2.63682,12.0403);
\draw [color=c, fill=c] (2.63682,11.9344) rectangle (2.67662,12.0403);
\draw [color=c, fill=c] (2.67662,11.9344) rectangle (2.71642,12.0403);
\draw [color=c, fill=c] (2.71642,11.9344) rectangle (2.75622,12.0403);
\draw [color=c, fill=c] (2.75622,11.9344) rectangle (2.79602,12.0403);
\draw [color=c, fill=c] (2.79602,11.9344) rectangle (2.83582,12.0403);
\draw [color=c, fill=c] (2.83582,11.9344) rectangle (2.87562,12.0403);
\draw [color=c, fill=c] (2.87562,11.9344) rectangle (2.91542,12.0403);
\draw [color=c, fill=c] (2.91542,11.9344) rectangle (2.95522,12.0403);
\draw [color=c, fill=c] (2.95522,11.9344) rectangle (2.99502,12.0403);
\draw [color=c, fill=c] (2.99502,11.9344) rectangle (3.03483,12.0403);
\draw [color=c, fill=c] (3.03483,11.9344) rectangle (3.07463,12.0403);
\draw [color=c, fill=c] (3.07463,11.9344) rectangle (3.11443,12.0403);
\draw [color=c, fill=c] (3.11443,11.9344) rectangle (3.15423,12.0403);
\draw [color=c, fill=c] (3.15423,11.9344) rectangle (3.19403,12.0403);
\draw [color=c, fill=c] (3.19403,11.9344) rectangle (3.23383,12.0403);
\draw [color=c, fill=c] (3.23383,11.9344) rectangle (3.27363,12.0403);
\draw [color=c, fill=c] (3.27363,11.9344) rectangle (3.31343,12.0403);
\draw [color=c, fill=c] (3.31343,11.9344) rectangle (3.35323,12.0403);
\draw [color=c, fill=c] (3.35323,11.9344) rectangle (3.39303,12.0403);
\draw [color=c, fill=c] (3.39303,11.9344) rectangle (3.43284,12.0403);
\draw [color=c, fill=c] (3.43284,11.9344) rectangle (3.47264,12.0403);
\draw [color=c, fill=c] (3.47264,11.9344) rectangle (3.51244,12.0403);
\draw [color=c, fill=c] (3.51244,11.9344) rectangle (3.55224,12.0403);
\draw [color=c, fill=c] (3.55224,11.9344) rectangle (3.59204,12.0403);
\draw [color=c, fill=c] (3.59204,11.9344) rectangle (3.63184,12.0403);
\draw [color=c, fill=c] (3.63184,11.9344) rectangle (3.67164,12.0403);
\draw [color=c, fill=c] (3.67164,11.9344) rectangle (3.71144,12.0403);
\draw [color=c, fill=c] (3.71144,11.9344) rectangle (3.75124,12.0403);
\draw [color=c, fill=c] (3.75124,11.9344) rectangle (3.79104,12.0403);
\draw [color=c, fill=c] (3.79104,11.9344) rectangle (3.83085,12.0403);
\draw [color=c, fill=c] (3.83085,11.9344) rectangle (3.87065,12.0403);
\draw [color=c, fill=c] (3.87065,11.9344) rectangle (3.91045,12.0403);
\draw [color=c, fill=c] (3.91045,11.9344) rectangle (3.95025,12.0403);
\draw [color=c, fill=c] (3.95025,11.9344) rectangle (3.99005,12.0403);
\draw [color=c, fill=c] (3.99005,11.9344) rectangle (4.02985,12.0403);
\draw [color=c, fill=c] (4.02985,11.9344) rectangle (4.06965,12.0403);
\draw [color=c, fill=c] (4.06965,11.9344) rectangle (4.10945,12.0403);
\draw [color=c, fill=c] (4.10945,11.9344) rectangle (4.14925,12.0403);
\draw [color=c, fill=c] (4.14925,11.9344) rectangle (4.18905,12.0403);
\draw [color=c, fill=c] (4.18905,11.9344) rectangle (4.22886,12.0403);
\draw [color=c, fill=c] (4.22886,11.9344) rectangle (4.26866,12.0403);
\draw [color=c, fill=c] (4.26866,11.9344) rectangle (4.30846,12.0403);
\draw [color=c, fill=c] (4.30846,11.9344) rectangle (4.34826,12.0403);
\draw [color=c, fill=c] (4.34826,11.9344) rectangle (4.38806,12.0403);
\draw [color=c, fill=c] (4.38806,11.9344) rectangle (4.42786,12.0403);
\draw [color=c, fill=c] (4.42786,11.9344) rectangle (4.46766,12.0403);
\draw [color=c, fill=c] (4.46766,11.9344) rectangle (4.50746,12.0403);
\draw [color=c, fill=c] (4.50746,11.9344) rectangle (4.54726,12.0403);
\draw [color=c, fill=c] (4.54726,11.9344) rectangle (4.58706,12.0403);
\draw [color=c, fill=c] (4.58706,11.9344) rectangle (4.62687,12.0403);
\draw [color=c, fill=c] (4.62687,11.9344) rectangle (4.66667,12.0403);
\draw [color=c, fill=c] (4.66667,11.9344) rectangle (4.70647,12.0403);
\draw [color=c, fill=c] (4.70647,11.9344) rectangle (4.74627,12.0403);
\draw [color=c, fill=c] (4.74627,11.9344) rectangle (4.78607,12.0403);
\draw [color=c, fill=c] (4.78607,11.9344) rectangle (4.82587,12.0403);
\draw [color=c, fill=c] (4.82587,11.9344) rectangle (4.86567,12.0403);
\draw [color=c, fill=c] (4.86567,11.9344) rectangle (4.90547,12.0403);
\draw [color=c, fill=c] (4.90547,11.9344) rectangle (4.94527,12.0403);
\draw [color=c, fill=c] (4.94527,11.9344) rectangle (4.98507,12.0403);
\draw [color=c, fill=c] (4.98507,11.9344) rectangle (5.02488,12.0403);
\draw [color=c, fill=c] (5.02488,11.9344) rectangle (5.06468,12.0403);
\draw [color=c, fill=c] (5.06468,11.9344) rectangle (5.10448,12.0403);
\draw [color=c, fill=c] (5.10448,11.9344) rectangle (5.14428,12.0403);
\draw [color=c, fill=c] (5.14428,11.9344) rectangle (5.18408,12.0403);
\draw [color=c, fill=c] (5.18408,11.9344) rectangle (5.22388,12.0403);
\draw [color=c, fill=c] (5.22388,11.9344) rectangle (5.26368,12.0403);
\draw [color=c, fill=c] (5.26368,11.9344) rectangle (5.30348,12.0403);
\draw [color=c, fill=c] (5.30348,11.9344) rectangle (5.34328,12.0403);
\draw [color=c, fill=c] (5.34328,11.9344) rectangle (5.38308,12.0403);
\draw [color=c, fill=c] (5.38308,11.9344) rectangle (5.42289,12.0403);
\draw [color=c, fill=c] (5.42289,11.9344) rectangle (5.46269,12.0403);
\draw [color=c, fill=c] (5.46269,11.9344) rectangle (5.50249,12.0403);
\draw [color=c, fill=c] (5.50249,11.9344) rectangle (5.54229,12.0403);
\draw [color=c, fill=c] (5.54229,11.9344) rectangle (5.58209,12.0403);
\draw [color=c, fill=c] (5.58209,11.9344) rectangle (5.62189,12.0403);
\draw [color=c, fill=c] (5.62189,11.9344) rectangle (5.66169,12.0403);
\draw [color=c, fill=c] (5.66169,11.9344) rectangle (5.70149,12.0403);
\draw [color=c, fill=c] (5.70149,11.9344) rectangle (5.74129,12.0403);
\draw [color=c, fill=c] (5.74129,11.9344) rectangle (5.78109,12.0403);
\draw [color=c, fill=c] (5.78109,11.9344) rectangle (5.8209,12.0403);
\draw [color=c, fill=c] (5.8209,11.9344) rectangle (5.8607,12.0403);
\draw [color=c, fill=c] (5.8607,11.9344) rectangle (5.9005,12.0403);
\draw [color=c, fill=c] (5.9005,11.9344) rectangle (5.9403,12.0403);
\draw [color=c, fill=c] (5.9403,11.9344) rectangle (5.9801,12.0403);
\draw [color=c, fill=c] (5.9801,11.9344) rectangle (6.0199,12.0403);
\draw [color=c, fill=c] (6.0199,11.9344) rectangle (6.0597,12.0403);
\draw [color=c, fill=c] (6.0597,11.9344) rectangle (6.0995,12.0403);
\draw [color=c, fill=c] (6.0995,11.9344) rectangle (6.1393,12.0403);
\draw [color=c, fill=c] (6.1393,11.9344) rectangle (6.1791,12.0403);
\draw [color=c, fill=c] (6.1791,11.9344) rectangle (6.21891,12.0403);
\draw [color=c, fill=c] (6.21891,11.9344) rectangle (6.25871,12.0403);
\draw [color=c, fill=c] (6.25871,11.9344) rectangle (6.29851,12.0403);
\draw [color=c, fill=c] (6.29851,11.9344) rectangle (6.33831,12.0403);
\draw [color=c, fill=c] (6.33831,11.9344) rectangle (6.37811,12.0403);
\draw [color=c, fill=c] (6.37811,11.9344) rectangle (6.41791,12.0403);
\draw [color=c, fill=c] (6.41791,11.9344) rectangle (6.45771,12.0403);
\draw [color=c, fill=c] (6.45771,11.9344) rectangle (6.49751,12.0403);
\draw [color=c, fill=c] (6.49751,11.9344) rectangle (6.53731,12.0403);
\draw [color=c, fill=c] (6.53731,11.9344) rectangle (6.57711,12.0403);
\draw [color=c, fill=c] (6.57711,11.9344) rectangle (6.61692,12.0403);
\draw [color=c, fill=c] (6.61692,11.9344) rectangle (6.65672,12.0403);
\draw [color=c, fill=c] (6.65672,11.9344) rectangle (6.69652,12.0403);
\draw [color=c, fill=c] (6.69652,11.9344) rectangle (6.73632,12.0403);
\draw [color=c, fill=c] (6.73632,11.9344) rectangle (6.77612,12.0403);
\draw [color=c, fill=c] (6.77612,11.9344) rectangle (6.81592,12.0403);
\draw [color=c, fill=c] (6.81592,11.9344) rectangle (6.85572,12.0403);
\draw [color=c, fill=c] (6.85572,11.9344) rectangle (6.89552,12.0403);
\draw [color=c, fill=c] (6.89552,11.9344) rectangle (6.93532,12.0403);
\draw [color=c, fill=c] (6.93532,11.9344) rectangle (6.97512,12.0403);
\draw [color=c, fill=c] (6.97512,11.9344) rectangle (7.01493,12.0403);
\draw [color=c, fill=c] (7.01493,11.9344) rectangle (7.05473,12.0403);
\draw [color=c, fill=c] (7.05473,11.9344) rectangle (7.09453,12.0403);
\draw [color=c, fill=c] (7.09453,11.9344) rectangle (7.13433,12.0403);
\draw [color=c, fill=c] (7.13433,11.9344) rectangle (7.17413,12.0403);
\draw [color=c, fill=c] (7.17413,11.9344) rectangle (7.21393,12.0403);
\draw [color=c, fill=c] (7.21393,11.9344) rectangle (7.25373,12.0403);
\draw [color=c, fill=c] (7.25373,11.9344) rectangle (7.29353,12.0403);
\draw [color=c, fill=c] (7.29353,11.9344) rectangle (7.33333,12.0403);
\draw [color=c, fill=c] (7.33333,11.9344) rectangle (7.37313,12.0403);
\draw [color=c, fill=c] (7.37313,11.9344) rectangle (7.41294,12.0403);
\draw [color=c, fill=c] (7.41294,11.9344) rectangle (7.45274,12.0403);
\draw [color=c, fill=c] (7.45274,11.9344) rectangle (7.49254,12.0403);
\draw [color=c, fill=c] (7.49254,11.9344) rectangle (7.53234,12.0403);
\draw [color=c, fill=c] (7.53234,11.9344) rectangle (7.57214,12.0403);
\draw [color=c, fill=c] (7.57214,11.9344) rectangle (7.61194,12.0403);
\draw [color=c, fill=c] (7.61194,11.9344) rectangle (7.65174,12.0403);
\draw [color=c, fill=c] (7.65174,11.9344) rectangle (7.69154,12.0403);
\draw [color=c, fill=c] (7.69154,11.9344) rectangle (7.73134,12.0403);
\draw [color=c, fill=c] (7.73134,11.9344) rectangle (7.77114,12.0403);
\draw [color=c, fill=c] (7.77114,11.9344) rectangle (7.81095,12.0403);
\draw [color=c, fill=c] (7.81095,11.9344) rectangle (7.85075,12.0403);
\draw [color=c, fill=c] (7.85075,11.9344) rectangle (7.89055,12.0403);
\draw [color=c, fill=c] (7.89055,11.9344) rectangle (7.93035,12.0403);
\draw [color=c, fill=c] (7.93035,11.9344) rectangle (7.97015,12.0403);
\draw [color=c, fill=c] (7.97015,11.9344) rectangle (8.00995,12.0403);
\draw [color=c, fill=c] (8.00995,11.9344) rectangle (8.04975,12.0403);
\draw [color=c, fill=c] (8.04975,11.9344) rectangle (8.08955,12.0403);
\draw [color=c, fill=c] (8.08955,11.9344) rectangle (8.12935,12.0403);
\draw [color=c, fill=c] (8.12935,11.9344) rectangle (8.16915,12.0403);
\draw [color=c, fill=c] (8.16915,11.9344) rectangle (8.20895,12.0403);
\definecolor{c}{rgb}{0,0.0800001,1};
\draw [color=c, fill=c] (8.20895,11.9344) rectangle (8.24876,12.0403);
\draw [color=c, fill=c] (8.24876,11.9344) rectangle (8.28856,12.0403);
\draw [color=c, fill=c] (8.28856,11.9344) rectangle (8.32836,12.0403);
\draw [color=c, fill=c] (8.32836,11.9344) rectangle (8.36816,12.0403);
\draw [color=c, fill=c] (8.36816,11.9344) rectangle (8.40796,12.0403);
\draw [color=c, fill=c] (8.40796,11.9344) rectangle (8.44776,12.0403);
\draw [color=c, fill=c] (8.44776,11.9344) rectangle (8.48756,12.0403);
\draw [color=c, fill=c] (8.48756,11.9344) rectangle (8.52736,12.0403);
\draw [color=c, fill=c] (8.52736,11.9344) rectangle (8.56716,12.0403);
\draw [color=c, fill=c] (8.56716,11.9344) rectangle (8.60697,12.0403);
\draw [color=c, fill=c] (8.60697,11.9344) rectangle (8.64677,12.0403);
\draw [color=c, fill=c] (8.64677,11.9344) rectangle (8.68657,12.0403);
\draw [color=c, fill=c] (8.68657,11.9344) rectangle (8.72637,12.0403);
\draw [color=c, fill=c] (8.72637,11.9344) rectangle (8.76617,12.0403);
\draw [color=c, fill=c] (8.76617,11.9344) rectangle (8.80597,12.0403);
\draw [color=c, fill=c] (8.80597,11.9344) rectangle (8.84577,12.0403);
\draw [color=c, fill=c] (8.84577,11.9344) rectangle (8.88557,12.0403);
\draw [color=c, fill=c] (8.88557,11.9344) rectangle (8.92537,12.0403);
\draw [color=c, fill=c] (8.92537,11.9344) rectangle (8.96517,12.0403);
\draw [color=c, fill=c] (8.96517,11.9344) rectangle (9.00498,12.0403);
\draw [color=c, fill=c] (9.00498,11.9344) rectangle (9.04478,12.0403);
\draw [color=c, fill=c] (9.04478,11.9344) rectangle (9.08458,12.0403);
\draw [color=c, fill=c] (9.08458,11.9344) rectangle (9.12438,12.0403);
\draw [color=c, fill=c] (9.12438,11.9344) rectangle (9.16418,12.0403);
\draw [color=c, fill=c] (9.16418,11.9344) rectangle (9.20398,12.0403);
\draw [color=c, fill=c] (9.20398,11.9344) rectangle (9.24378,12.0403);
\draw [color=c, fill=c] (9.24378,11.9344) rectangle (9.28358,12.0403);
\draw [color=c, fill=c] (9.28358,11.9344) rectangle (9.32338,12.0403);
\draw [color=c, fill=c] (9.32338,11.9344) rectangle (9.36318,12.0403);
\draw [color=c, fill=c] (9.36318,11.9344) rectangle (9.40298,12.0403);
\draw [color=c, fill=c] (9.40298,11.9344) rectangle (9.44279,12.0403);
\draw [color=c, fill=c] (9.44279,11.9344) rectangle (9.48259,12.0403);
\draw [color=c, fill=c] (9.48259,11.9344) rectangle (9.52239,12.0403);
\draw [color=c, fill=c] (9.52239,11.9344) rectangle (9.56219,12.0403);
\draw [color=c, fill=c] (9.56219,11.9344) rectangle (9.60199,12.0403);
\draw [color=c, fill=c] (9.60199,11.9344) rectangle (9.64179,12.0403);
\draw [color=c, fill=c] (9.64179,11.9344) rectangle (9.68159,12.0403);
\draw [color=c, fill=c] (9.68159,11.9344) rectangle (9.72139,12.0403);
\definecolor{c}{rgb}{0,0.266667,1};
\draw [color=c, fill=c] (9.72139,11.9344) rectangle (9.76119,12.0403);
\draw [color=c, fill=c] (9.76119,11.9344) rectangle (9.80099,12.0403);
\draw [color=c, fill=c] (9.80099,11.9344) rectangle (9.8408,12.0403);
\draw [color=c, fill=c] (9.8408,11.9344) rectangle (9.8806,12.0403);
\draw [color=c, fill=c] (9.8806,11.9344) rectangle (9.9204,12.0403);
\draw [color=c, fill=c] (9.9204,11.9344) rectangle (9.9602,12.0403);
\draw [color=c, fill=c] (9.9602,11.9344) rectangle (10,12.0403);
\draw [color=c, fill=c] (10,11.9344) rectangle (10.0398,12.0403);
\draw [color=c, fill=c] (10.0398,11.9344) rectangle (10.0796,12.0403);
\draw [color=c, fill=c] (10.0796,11.9344) rectangle (10.1194,12.0403);
\draw [color=c, fill=c] (10.1194,11.9344) rectangle (10.1592,12.0403);
\draw [color=c, fill=c] (10.1592,11.9344) rectangle (10.199,12.0403);
\draw [color=c, fill=c] (10.199,11.9344) rectangle (10.2388,12.0403);
\draw [color=c, fill=c] (10.2388,11.9344) rectangle (10.2786,12.0403);
\draw [color=c, fill=c] (10.2786,11.9344) rectangle (10.3184,12.0403);
\draw [color=c, fill=c] (10.3184,11.9344) rectangle (10.3582,12.0403);
\draw [color=c, fill=c] (10.3582,11.9344) rectangle (10.398,12.0403);
\draw [color=c, fill=c] (10.398,11.9344) rectangle (10.4378,12.0403);
\draw [color=c, fill=c] (10.4378,11.9344) rectangle (10.4776,12.0403);
\draw [color=c, fill=c] (10.4776,11.9344) rectangle (10.5174,12.0403);
\draw [color=c, fill=c] (10.5174,11.9344) rectangle (10.5572,12.0403);
\draw [color=c, fill=c] (10.5572,11.9344) rectangle (10.597,12.0403);
\draw [color=c, fill=c] (10.597,11.9344) rectangle (10.6368,12.0403);
\draw [color=c, fill=c] (10.6368,11.9344) rectangle (10.6766,12.0403);
\draw [color=c, fill=c] (10.6766,11.9344) rectangle (10.7164,12.0403);
\draw [color=c, fill=c] (10.7164,11.9344) rectangle (10.7562,12.0403);
\draw [color=c, fill=c] (10.7562,11.9344) rectangle (10.796,12.0403);
\draw [color=c, fill=c] (10.796,11.9344) rectangle (10.8358,12.0403);
\draw [color=c, fill=c] (10.8358,11.9344) rectangle (10.8756,12.0403);
\draw [color=c, fill=c] (10.8756,11.9344) rectangle (10.9154,12.0403);
\draw [color=c, fill=c] (10.9154,11.9344) rectangle (10.9552,12.0403);
\draw [color=c, fill=c] (10.9552,11.9344) rectangle (10.995,12.0403);
\draw [color=c, fill=c] (10.995,11.9344) rectangle (11.0348,12.0403);
\draw [color=c, fill=c] (11.0348,11.9344) rectangle (11.0746,12.0403);
\definecolor{c}{rgb}{0,0.546666,1};
\draw [color=c, fill=c] (11.0746,11.9344) rectangle (11.1144,12.0403);
\draw [color=c, fill=c] (11.1144,11.9344) rectangle (11.1542,12.0403);
\draw [color=c, fill=c] (11.1542,11.9344) rectangle (11.194,12.0403);
\draw [color=c, fill=c] (11.194,11.9344) rectangle (11.2338,12.0403);
\draw [color=c, fill=c] (11.2338,11.9344) rectangle (11.2736,12.0403);
\draw [color=c, fill=c] (11.2736,11.9344) rectangle (11.3134,12.0403);
\draw [color=c, fill=c] (11.3134,11.9344) rectangle (11.3532,12.0403);
\draw [color=c, fill=c] (11.3532,11.9344) rectangle (11.393,12.0403);
\draw [color=c, fill=c] (11.393,11.9344) rectangle (11.4328,12.0403);
\draw [color=c, fill=c] (11.4328,11.9344) rectangle (11.4726,12.0403);
\draw [color=c, fill=c] (11.4726,11.9344) rectangle (11.5124,12.0403);
\draw [color=c, fill=c] (11.5124,11.9344) rectangle (11.5522,12.0403);
\draw [color=c, fill=c] (11.5522,11.9344) rectangle (11.592,12.0403);
\draw [color=c, fill=c] (11.592,11.9344) rectangle (11.6318,12.0403);
\draw [color=c, fill=c] (11.6318,11.9344) rectangle (11.6716,12.0403);
\draw [color=c, fill=c] (11.6716,11.9344) rectangle (11.7114,12.0403);
\draw [color=c, fill=c] (11.7114,11.9344) rectangle (11.7512,12.0403);
\draw [color=c, fill=c] (11.7512,11.9344) rectangle (11.791,12.0403);
\draw [color=c, fill=c] (11.791,11.9344) rectangle (11.8308,12.0403);
\draw [color=c, fill=c] (11.8308,11.9344) rectangle (11.8706,12.0403);
\draw [color=c, fill=c] (11.8706,11.9344) rectangle (11.9104,12.0403);
\draw [color=c, fill=c] (11.9104,11.9344) rectangle (11.9502,12.0403);
\draw [color=c, fill=c] (11.9502,11.9344) rectangle (11.99,12.0403);
\draw [color=c, fill=c] (11.99,11.9344) rectangle (12.0299,12.0403);
\draw [color=c, fill=c] (12.0299,11.9344) rectangle (12.0697,12.0403);
\draw [color=c, fill=c] (12.0697,11.9344) rectangle (12.1095,12.0403);
\draw [color=c, fill=c] (12.1095,11.9344) rectangle (12.1493,12.0403);
\draw [color=c, fill=c] (12.1493,11.9344) rectangle (12.1891,12.0403);
\draw [color=c, fill=c] (12.1891,11.9344) rectangle (12.2289,12.0403);
\draw [color=c, fill=c] (12.2289,11.9344) rectangle (12.2687,12.0403);
\draw [color=c, fill=c] (12.2687,11.9344) rectangle (12.3085,12.0403);
\draw [color=c, fill=c] (12.3085,11.9344) rectangle (12.3483,12.0403);
\draw [color=c, fill=c] (12.3483,11.9344) rectangle (12.3881,12.0403);
\draw [color=c, fill=c] (12.3881,11.9344) rectangle (12.4279,12.0403);
\draw [color=c, fill=c] (12.4279,11.9344) rectangle (12.4677,12.0403);
\draw [color=c, fill=c] (12.4677,11.9344) rectangle (12.5075,12.0403);
\draw [color=c, fill=c] (12.5075,11.9344) rectangle (12.5473,12.0403);
\draw [color=c, fill=c] (12.5473,11.9344) rectangle (12.5871,12.0403);
\draw [color=c, fill=c] (12.5871,11.9344) rectangle (12.6269,12.0403);
\draw [color=c, fill=c] (12.6269,11.9344) rectangle (12.6667,12.0403);
\draw [color=c, fill=c] (12.6667,11.9344) rectangle (12.7065,12.0403);
\draw [color=c, fill=c] (12.7065,11.9344) rectangle (12.7463,12.0403);
\draw [color=c, fill=c] (12.7463,11.9344) rectangle (12.7861,12.0403);
\draw [color=c, fill=c] (12.7861,11.9344) rectangle (12.8259,12.0403);
\draw [color=c, fill=c] (12.8259,11.9344) rectangle (12.8657,12.0403);
\draw [color=c, fill=c] (12.8657,11.9344) rectangle (12.9055,12.0403);
\draw [color=c, fill=c] (12.9055,11.9344) rectangle (12.9453,12.0403);
\draw [color=c, fill=c] (12.9453,11.9344) rectangle (12.9851,12.0403);
\draw [color=c, fill=c] (12.9851,11.9344) rectangle (13.0249,12.0403);
\draw [color=c, fill=c] (13.0249,11.9344) rectangle (13.0647,12.0403);
\draw [color=c, fill=c] (13.0647,11.9344) rectangle (13.1045,12.0403);
\draw [color=c, fill=c] (13.1045,11.9344) rectangle (13.1443,12.0403);
\draw [color=c, fill=c] (13.1443,11.9344) rectangle (13.1841,12.0403);
\draw [color=c, fill=c] (13.1841,11.9344) rectangle (13.2239,12.0403);
\draw [color=c, fill=c] (13.2239,11.9344) rectangle (13.2637,12.0403);
\draw [color=c, fill=c] (13.2637,11.9344) rectangle (13.3035,12.0403);
\draw [color=c, fill=c] (13.3035,11.9344) rectangle (13.3433,12.0403);
\draw [color=c, fill=c] (13.3433,11.9344) rectangle (13.3831,12.0403);
\draw [color=c, fill=c] (13.3831,11.9344) rectangle (13.4229,12.0403);
\draw [color=c, fill=c] (13.4229,11.9344) rectangle (13.4627,12.0403);
\draw [color=c, fill=c] (13.4627,11.9344) rectangle (13.5025,12.0403);
\draw [color=c, fill=c] (13.5025,11.9344) rectangle (13.5423,12.0403);
\draw [color=c, fill=c] (13.5423,11.9344) rectangle (13.5821,12.0403);
\draw [color=c, fill=c] (13.5821,11.9344) rectangle (13.6219,12.0403);
\draw [color=c, fill=c] (13.6219,11.9344) rectangle (13.6617,12.0403);
\draw [color=c, fill=c] (13.6617,11.9344) rectangle (13.7015,12.0403);
\draw [color=c, fill=c] (13.7015,11.9344) rectangle (13.7413,12.0403);
\draw [color=c, fill=c] (13.7413,11.9344) rectangle (13.7811,12.0403);
\draw [color=c, fill=c] (13.7811,11.9344) rectangle (13.8209,12.0403);
\draw [color=c, fill=c] (13.8209,11.9344) rectangle (13.8607,12.0403);
\draw [color=c, fill=c] (13.8607,11.9344) rectangle (13.9005,12.0403);
\draw [color=c, fill=c] (13.9005,11.9344) rectangle (13.9403,12.0403);
\draw [color=c, fill=c] (13.9403,11.9344) rectangle (13.9801,12.0403);
\draw [color=c, fill=c] (13.9801,11.9344) rectangle (14.0199,12.0403);
\draw [color=c, fill=c] (14.0199,11.9344) rectangle (14.0597,12.0403);
\draw [color=c, fill=c] (14.0597,11.9344) rectangle (14.0995,12.0403);
\draw [color=c, fill=c] (14.0995,11.9344) rectangle (14.1393,12.0403);
\draw [color=c, fill=c] (14.1393,11.9344) rectangle (14.1791,12.0403);
\draw [color=c, fill=c] (14.1791,11.9344) rectangle (14.2189,12.0403);
\draw [color=c, fill=c] (14.2189,11.9344) rectangle (14.2587,12.0403);
\draw [color=c, fill=c] (14.2587,11.9344) rectangle (14.2985,12.0403);
\draw [color=c, fill=c] (14.2985,11.9344) rectangle (14.3383,12.0403);
\draw [color=c, fill=c] (14.3383,11.9344) rectangle (14.3781,12.0403);
\draw [color=c, fill=c] (14.3781,11.9344) rectangle (14.4179,12.0403);
\draw [color=c, fill=c] (14.4179,11.9344) rectangle (14.4577,12.0403);
\draw [color=c, fill=c] (14.4577,11.9344) rectangle (14.4975,12.0403);
\draw [color=c, fill=c] (14.4975,11.9344) rectangle (14.5373,12.0403);
\draw [color=c, fill=c] (14.5373,11.9344) rectangle (14.5771,12.0403);
\draw [color=c, fill=c] (14.5771,11.9344) rectangle (14.6169,12.0403);
\definecolor{c}{rgb}{0,0.733333,1};
\draw [color=c, fill=c] (14.6169,11.9344) rectangle (14.6567,12.0403);
\draw [color=c, fill=c] (14.6567,11.9344) rectangle (14.6965,12.0403);
\draw [color=c, fill=c] (14.6965,11.9344) rectangle (14.7363,12.0403);
\draw [color=c, fill=c] (14.7363,11.9344) rectangle (14.7761,12.0403);
\draw [color=c, fill=c] (14.7761,11.9344) rectangle (14.8159,12.0403);
\draw [color=c, fill=c] (14.8159,11.9344) rectangle (14.8557,12.0403);
\draw [color=c, fill=c] (14.8557,11.9344) rectangle (14.8955,12.0403);
\draw [color=c, fill=c] (14.8955,11.9344) rectangle (14.9353,12.0403);
\draw [color=c, fill=c] (14.9353,11.9344) rectangle (14.9751,12.0403);
\draw [color=c, fill=c] (14.9751,11.9344) rectangle (15.0149,12.0403);
\draw [color=c, fill=c] (15.0149,11.9344) rectangle (15.0547,12.0403);
\draw [color=c, fill=c] (15.0547,11.9344) rectangle (15.0945,12.0403);
\draw [color=c, fill=c] (15.0945,11.9344) rectangle (15.1343,12.0403);
\draw [color=c, fill=c] (15.1343,11.9344) rectangle (15.1741,12.0403);
\draw [color=c, fill=c] (15.1741,11.9344) rectangle (15.2139,12.0403);
\draw [color=c, fill=c] (15.2139,11.9344) rectangle (15.2537,12.0403);
\draw [color=c, fill=c] (15.2537,11.9344) rectangle (15.2935,12.0403);
\draw [color=c, fill=c] (15.2935,11.9344) rectangle (15.3333,12.0403);
\draw [color=c, fill=c] (15.3333,11.9344) rectangle (15.3731,12.0403);
\draw [color=c, fill=c] (15.3731,11.9344) rectangle (15.4129,12.0403);
\draw [color=c, fill=c] (15.4129,11.9344) rectangle (15.4527,12.0403);
\draw [color=c, fill=c] (15.4527,11.9344) rectangle (15.4925,12.0403);
\draw [color=c, fill=c] (15.4925,11.9344) rectangle (15.5323,12.0403);
\draw [color=c, fill=c] (15.5323,11.9344) rectangle (15.5721,12.0403);
\draw [color=c, fill=c] (15.5721,11.9344) rectangle (15.6119,12.0403);
\draw [color=c, fill=c] (15.6119,11.9344) rectangle (15.6517,12.0403);
\draw [color=c, fill=c] (15.6517,11.9344) rectangle (15.6915,12.0403);
\draw [color=c, fill=c] (15.6915,11.9344) rectangle (15.7313,12.0403);
\draw [color=c, fill=c] (15.7313,11.9344) rectangle (15.7711,12.0403);
\draw [color=c, fill=c] (15.7711,11.9344) rectangle (15.8109,12.0403);
\draw [color=c, fill=c] (15.8109,11.9344) rectangle (15.8507,12.0403);
\draw [color=c, fill=c] (15.8507,11.9344) rectangle (15.8905,12.0403);
\draw [color=c, fill=c] (15.8905,11.9344) rectangle (15.9303,12.0403);
\draw [color=c, fill=c] (15.9303,11.9344) rectangle (15.9701,12.0403);
\draw [color=c, fill=c] (15.9701,11.9344) rectangle (16.01,12.0403);
\draw [color=c, fill=c] (16.01,11.9344) rectangle (16.0498,12.0403);
\draw [color=c, fill=c] (16.0498,11.9344) rectangle (16.0896,12.0403);
\draw [color=c, fill=c] (16.0896,11.9344) rectangle (16.1294,12.0403);
\draw [color=c, fill=c] (16.1294,11.9344) rectangle (16.1692,12.0403);
\draw [color=c, fill=c] (16.1692,11.9344) rectangle (16.209,12.0403);
\draw [color=c, fill=c] (16.209,11.9344) rectangle (16.2488,12.0403);
\draw [color=c, fill=c] (16.2488,11.9344) rectangle (16.2886,12.0403);
\draw [color=c, fill=c] (16.2886,11.9344) rectangle (16.3284,12.0403);
\draw [color=c, fill=c] (16.3284,11.9344) rectangle (16.3682,12.0403);
\draw [color=c, fill=c] (16.3682,11.9344) rectangle (16.408,12.0403);
\draw [color=c, fill=c] (16.408,11.9344) rectangle (16.4478,12.0403);
\draw [color=c, fill=c] (16.4478,11.9344) rectangle (16.4876,12.0403);
\draw [color=c, fill=c] (16.4876,11.9344) rectangle (16.5274,12.0403);
\draw [color=c, fill=c] (16.5274,11.9344) rectangle (16.5672,12.0403);
\draw [color=c, fill=c] (16.5672,11.9344) rectangle (16.607,12.0403);
\draw [color=c, fill=c] (16.607,11.9344) rectangle (16.6468,12.0403);
\draw [color=c, fill=c] (16.6468,11.9344) rectangle (16.6866,12.0403);
\draw [color=c, fill=c] (16.6866,11.9344) rectangle (16.7264,12.0403);
\draw [color=c, fill=c] (16.7264,11.9344) rectangle (16.7662,12.0403);
\draw [color=c, fill=c] (16.7662,11.9344) rectangle (16.806,12.0403);
\draw [color=c, fill=c] (16.806,11.9344) rectangle (16.8458,12.0403);
\draw [color=c, fill=c] (16.8458,11.9344) rectangle (16.8856,12.0403);
\draw [color=c, fill=c] (16.8856,11.9344) rectangle (16.9254,12.0403);
\draw [color=c, fill=c] (16.9254,11.9344) rectangle (16.9652,12.0403);
\draw [color=c, fill=c] (16.9652,11.9344) rectangle (17.005,12.0403);
\draw [color=c, fill=c] (17.005,11.9344) rectangle (17.0448,12.0403);
\draw [color=c, fill=c] (17.0448,11.9344) rectangle (17.0846,12.0403);
\draw [color=c, fill=c] (17.0846,11.9344) rectangle (17.1244,12.0403);
\draw [color=c, fill=c] (17.1244,11.9344) rectangle (17.1642,12.0403);
\draw [color=c, fill=c] (17.1642,11.9344) rectangle (17.204,12.0403);
\draw [color=c, fill=c] (17.204,11.9344) rectangle (17.2438,12.0403);
\draw [color=c, fill=c] (17.2438,11.9344) rectangle (17.2836,12.0403);
\draw [color=c, fill=c] (17.2836,11.9344) rectangle (17.3234,12.0403);
\draw [color=c, fill=c] (17.3234,11.9344) rectangle (17.3632,12.0403);
\draw [color=c, fill=c] (17.3632,11.9344) rectangle (17.403,12.0403);
\draw [color=c, fill=c] (17.403,11.9344) rectangle (17.4428,12.0403);
\draw [color=c, fill=c] (17.4428,11.9344) rectangle (17.4826,12.0403);
\draw [color=c, fill=c] (17.4826,11.9344) rectangle (17.5224,12.0403);
\draw [color=c, fill=c] (17.5224,11.9344) rectangle (17.5622,12.0403);
\draw [color=c, fill=c] (17.5622,11.9344) rectangle (17.602,12.0403);
\draw [color=c, fill=c] (17.602,11.9344) rectangle (17.6418,12.0403);
\draw [color=c, fill=c] (17.6418,11.9344) rectangle (17.6816,12.0403);
\draw [color=c, fill=c] (17.6816,11.9344) rectangle (17.7214,12.0403);
\draw [color=c, fill=c] (17.7214,11.9344) rectangle (17.7612,12.0403);
\draw [color=c, fill=c] (17.7612,11.9344) rectangle (17.801,12.0403);
\draw [color=c, fill=c] (17.801,11.9344) rectangle (17.8408,12.0403);
\draw [color=c, fill=c] (17.8408,11.9344) rectangle (17.8806,12.0403);
\draw [color=c, fill=c] (17.8806,11.9344) rectangle (17.9204,12.0403);
\draw [color=c, fill=c] (17.9204,11.9344) rectangle (17.9602,12.0403);
\draw [color=c, fill=c] (17.9602,11.9344) rectangle (18,12.0403);
\draw [color=c, fill=c] (2,12.0403) rectangle (2.0398,12.1461);
\draw [color=c, fill=c] (2.0398,12.0403) rectangle (2.0796,12.1461);
\draw [color=c, fill=c] (2.0796,12.0403) rectangle (2.1194,12.1461);
\draw [color=c, fill=c] (2.1194,12.0403) rectangle (2.1592,12.1461);
\draw [color=c, fill=c] (2.1592,12.0403) rectangle (2.19901,12.1461);
\draw [color=c, fill=c] (2.19901,12.0403) rectangle (2.23881,12.1461);
\draw [color=c, fill=c] (2.23881,12.0403) rectangle (2.27861,12.1461);
\draw [color=c, fill=c] (2.27861,12.0403) rectangle (2.31841,12.1461);
\draw [color=c, fill=c] (2.31841,12.0403) rectangle (2.35821,12.1461);
\draw [color=c, fill=c] (2.35821,12.0403) rectangle (2.39801,12.1461);
\draw [color=c, fill=c] (2.39801,12.0403) rectangle (2.43781,12.1461);
\draw [color=c, fill=c] (2.43781,12.0403) rectangle (2.47761,12.1461);
\draw [color=c, fill=c] (2.47761,12.0403) rectangle (2.51741,12.1461);
\draw [color=c, fill=c] (2.51741,12.0403) rectangle (2.55721,12.1461);
\draw [color=c, fill=c] (2.55721,12.0403) rectangle (2.59702,12.1461);
\draw [color=c, fill=c] (2.59702,12.0403) rectangle (2.63682,12.1461);
\draw [color=c, fill=c] (2.63682,12.0403) rectangle (2.67662,12.1461);
\draw [color=c, fill=c] (2.67662,12.0403) rectangle (2.71642,12.1461);
\draw [color=c, fill=c] (2.71642,12.0403) rectangle (2.75622,12.1461);
\draw [color=c, fill=c] (2.75622,12.0403) rectangle (2.79602,12.1461);
\draw [color=c, fill=c] (2.79602,12.0403) rectangle (2.83582,12.1461);
\draw [color=c, fill=c] (2.83582,12.0403) rectangle (2.87562,12.1461);
\draw [color=c, fill=c] (2.87562,12.0403) rectangle (2.91542,12.1461);
\draw [color=c, fill=c] (2.91542,12.0403) rectangle (2.95522,12.1461);
\draw [color=c, fill=c] (2.95522,12.0403) rectangle (2.99502,12.1461);
\draw [color=c, fill=c] (2.99502,12.0403) rectangle (3.03483,12.1461);
\draw [color=c, fill=c] (3.03483,12.0403) rectangle (3.07463,12.1461);
\draw [color=c, fill=c] (3.07463,12.0403) rectangle (3.11443,12.1461);
\draw [color=c, fill=c] (3.11443,12.0403) rectangle (3.15423,12.1461);
\draw [color=c, fill=c] (3.15423,12.0403) rectangle (3.19403,12.1461);
\draw [color=c, fill=c] (3.19403,12.0403) rectangle (3.23383,12.1461);
\draw [color=c, fill=c] (3.23383,12.0403) rectangle (3.27363,12.1461);
\draw [color=c, fill=c] (3.27363,12.0403) rectangle (3.31343,12.1461);
\draw [color=c, fill=c] (3.31343,12.0403) rectangle (3.35323,12.1461);
\draw [color=c, fill=c] (3.35323,12.0403) rectangle (3.39303,12.1461);
\draw [color=c, fill=c] (3.39303,12.0403) rectangle (3.43284,12.1461);
\draw [color=c, fill=c] (3.43284,12.0403) rectangle (3.47264,12.1461);
\draw [color=c, fill=c] (3.47264,12.0403) rectangle (3.51244,12.1461);
\draw [color=c, fill=c] (3.51244,12.0403) rectangle (3.55224,12.1461);
\draw [color=c, fill=c] (3.55224,12.0403) rectangle (3.59204,12.1461);
\draw [color=c, fill=c] (3.59204,12.0403) rectangle (3.63184,12.1461);
\draw [color=c, fill=c] (3.63184,12.0403) rectangle (3.67164,12.1461);
\draw [color=c, fill=c] (3.67164,12.0403) rectangle (3.71144,12.1461);
\draw [color=c, fill=c] (3.71144,12.0403) rectangle (3.75124,12.1461);
\draw [color=c, fill=c] (3.75124,12.0403) rectangle (3.79104,12.1461);
\draw [color=c, fill=c] (3.79104,12.0403) rectangle (3.83085,12.1461);
\draw [color=c, fill=c] (3.83085,12.0403) rectangle (3.87065,12.1461);
\draw [color=c, fill=c] (3.87065,12.0403) rectangle (3.91045,12.1461);
\draw [color=c, fill=c] (3.91045,12.0403) rectangle (3.95025,12.1461);
\draw [color=c, fill=c] (3.95025,12.0403) rectangle (3.99005,12.1461);
\draw [color=c, fill=c] (3.99005,12.0403) rectangle (4.02985,12.1461);
\draw [color=c, fill=c] (4.02985,12.0403) rectangle (4.06965,12.1461);
\draw [color=c, fill=c] (4.06965,12.0403) rectangle (4.10945,12.1461);
\draw [color=c, fill=c] (4.10945,12.0403) rectangle (4.14925,12.1461);
\draw [color=c, fill=c] (4.14925,12.0403) rectangle (4.18905,12.1461);
\draw [color=c, fill=c] (4.18905,12.0403) rectangle (4.22886,12.1461);
\draw [color=c, fill=c] (4.22886,12.0403) rectangle (4.26866,12.1461);
\draw [color=c, fill=c] (4.26866,12.0403) rectangle (4.30846,12.1461);
\draw [color=c, fill=c] (4.30846,12.0403) rectangle (4.34826,12.1461);
\draw [color=c, fill=c] (4.34826,12.0403) rectangle (4.38806,12.1461);
\draw [color=c, fill=c] (4.38806,12.0403) rectangle (4.42786,12.1461);
\draw [color=c, fill=c] (4.42786,12.0403) rectangle (4.46766,12.1461);
\draw [color=c, fill=c] (4.46766,12.0403) rectangle (4.50746,12.1461);
\draw [color=c, fill=c] (4.50746,12.0403) rectangle (4.54726,12.1461);
\draw [color=c, fill=c] (4.54726,12.0403) rectangle (4.58706,12.1461);
\draw [color=c, fill=c] (4.58706,12.0403) rectangle (4.62687,12.1461);
\draw [color=c, fill=c] (4.62687,12.0403) rectangle (4.66667,12.1461);
\draw [color=c, fill=c] (4.66667,12.0403) rectangle (4.70647,12.1461);
\draw [color=c, fill=c] (4.70647,12.0403) rectangle (4.74627,12.1461);
\draw [color=c, fill=c] (4.74627,12.0403) rectangle (4.78607,12.1461);
\draw [color=c, fill=c] (4.78607,12.0403) rectangle (4.82587,12.1461);
\draw [color=c, fill=c] (4.82587,12.0403) rectangle (4.86567,12.1461);
\draw [color=c, fill=c] (4.86567,12.0403) rectangle (4.90547,12.1461);
\draw [color=c, fill=c] (4.90547,12.0403) rectangle (4.94527,12.1461);
\draw [color=c, fill=c] (4.94527,12.0403) rectangle (4.98507,12.1461);
\draw [color=c, fill=c] (4.98507,12.0403) rectangle (5.02488,12.1461);
\draw [color=c, fill=c] (5.02488,12.0403) rectangle (5.06468,12.1461);
\draw [color=c, fill=c] (5.06468,12.0403) rectangle (5.10448,12.1461);
\draw [color=c, fill=c] (5.10448,12.0403) rectangle (5.14428,12.1461);
\draw [color=c, fill=c] (5.14428,12.0403) rectangle (5.18408,12.1461);
\draw [color=c, fill=c] (5.18408,12.0403) rectangle (5.22388,12.1461);
\draw [color=c, fill=c] (5.22388,12.0403) rectangle (5.26368,12.1461);
\draw [color=c, fill=c] (5.26368,12.0403) rectangle (5.30348,12.1461);
\draw [color=c, fill=c] (5.30348,12.0403) rectangle (5.34328,12.1461);
\draw [color=c, fill=c] (5.34328,12.0403) rectangle (5.38308,12.1461);
\draw [color=c, fill=c] (5.38308,12.0403) rectangle (5.42289,12.1461);
\draw [color=c, fill=c] (5.42289,12.0403) rectangle (5.46269,12.1461);
\draw [color=c, fill=c] (5.46269,12.0403) rectangle (5.50249,12.1461);
\draw [color=c, fill=c] (5.50249,12.0403) rectangle (5.54229,12.1461);
\draw [color=c, fill=c] (5.54229,12.0403) rectangle (5.58209,12.1461);
\draw [color=c, fill=c] (5.58209,12.0403) rectangle (5.62189,12.1461);
\draw [color=c, fill=c] (5.62189,12.0403) rectangle (5.66169,12.1461);
\draw [color=c, fill=c] (5.66169,12.0403) rectangle (5.70149,12.1461);
\draw [color=c, fill=c] (5.70149,12.0403) rectangle (5.74129,12.1461);
\draw [color=c, fill=c] (5.74129,12.0403) rectangle (5.78109,12.1461);
\draw [color=c, fill=c] (5.78109,12.0403) rectangle (5.8209,12.1461);
\draw [color=c, fill=c] (5.8209,12.0403) rectangle (5.8607,12.1461);
\draw [color=c, fill=c] (5.8607,12.0403) rectangle (5.9005,12.1461);
\draw [color=c, fill=c] (5.9005,12.0403) rectangle (5.9403,12.1461);
\draw [color=c, fill=c] (5.9403,12.0403) rectangle (5.9801,12.1461);
\draw [color=c, fill=c] (5.9801,12.0403) rectangle (6.0199,12.1461);
\draw [color=c, fill=c] (6.0199,12.0403) rectangle (6.0597,12.1461);
\draw [color=c, fill=c] (6.0597,12.0403) rectangle (6.0995,12.1461);
\draw [color=c, fill=c] (6.0995,12.0403) rectangle (6.1393,12.1461);
\draw [color=c, fill=c] (6.1393,12.0403) rectangle (6.1791,12.1461);
\draw [color=c, fill=c] (6.1791,12.0403) rectangle (6.21891,12.1461);
\draw [color=c, fill=c] (6.21891,12.0403) rectangle (6.25871,12.1461);
\draw [color=c, fill=c] (6.25871,12.0403) rectangle (6.29851,12.1461);
\draw [color=c, fill=c] (6.29851,12.0403) rectangle (6.33831,12.1461);
\draw [color=c, fill=c] (6.33831,12.0403) rectangle (6.37811,12.1461);
\draw [color=c, fill=c] (6.37811,12.0403) rectangle (6.41791,12.1461);
\draw [color=c, fill=c] (6.41791,12.0403) rectangle (6.45771,12.1461);
\draw [color=c, fill=c] (6.45771,12.0403) rectangle (6.49751,12.1461);
\draw [color=c, fill=c] (6.49751,12.0403) rectangle (6.53731,12.1461);
\draw [color=c, fill=c] (6.53731,12.0403) rectangle (6.57711,12.1461);
\draw [color=c, fill=c] (6.57711,12.0403) rectangle (6.61692,12.1461);
\draw [color=c, fill=c] (6.61692,12.0403) rectangle (6.65672,12.1461);
\draw [color=c, fill=c] (6.65672,12.0403) rectangle (6.69652,12.1461);
\draw [color=c, fill=c] (6.69652,12.0403) rectangle (6.73632,12.1461);
\draw [color=c, fill=c] (6.73632,12.0403) rectangle (6.77612,12.1461);
\draw [color=c, fill=c] (6.77612,12.0403) rectangle (6.81592,12.1461);
\draw [color=c, fill=c] (6.81592,12.0403) rectangle (6.85572,12.1461);
\draw [color=c, fill=c] (6.85572,12.0403) rectangle (6.89552,12.1461);
\draw [color=c, fill=c] (6.89552,12.0403) rectangle (6.93532,12.1461);
\draw [color=c, fill=c] (6.93532,12.0403) rectangle (6.97512,12.1461);
\draw [color=c, fill=c] (6.97512,12.0403) rectangle (7.01493,12.1461);
\draw [color=c, fill=c] (7.01493,12.0403) rectangle (7.05473,12.1461);
\draw [color=c, fill=c] (7.05473,12.0403) rectangle (7.09453,12.1461);
\draw [color=c, fill=c] (7.09453,12.0403) rectangle (7.13433,12.1461);
\draw [color=c, fill=c] (7.13433,12.0403) rectangle (7.17413,12.1461);
\draw [color=c, fill=c] (7.17413,12.0403) rectangle (7.21393,12.1461);
\draw [color=c, fill=c] (7.21393,12.0403) rectangle (7.25373,12.1461);
\draw [color=c, fill=c] (7.25373,12.0403) rectangle (7.29353,12.1461);
\draw [color=c, fill=c] (7.29353,12.0403) rectangle (7.33333,12.1461);
\draw [color=c, fill=c] (7.33333,12.0403) rectangle (7.37313,12.1461);
\draw [color=c, fill=c] (7.37313,12.0403) rectangle (7.41294,12.1461);
\draw [color=c, fill=c] (7.41294,12.0403) rectangle (7.45274,12.1461);
\draw [color=c, fill=c] (7.45274,12.0403) rectangle (7.49254,12.1461);
\draw [color=c, fill=c] (7.49254,12.0403) rectangle (7.53234,12.1461);
\draw [color=c, fill=c] (7.53234,12.0403) rectangle (7.57214,12.1461);
\draw [color=c, fill=c] (7.57214,12.0403) rectangle (7.61194,12.1461);
\draw [color=c, fill=c] (7.61194,12.0403) rectangle (7.65174,12.1461);
\draw [color=c, fill=c] (7.65174,12.0403) rectangle (7.69154,12.1461);
\draw [color=c, fill=c] (7.69154,12.0403) rectangle (7.73134,12.1461);
\draw [color=c, fill=c] (7.73134,12.0403) rectangle (7.77114,12.1461);
\draw [color=c, fill=c] (7.77114,12.0403) rectangle (7.81095,12.1461);
\draw [color=c, fill=c] (7.81095,12.0403) rectangle (7.85075,12.1461);
\draw [color=c, fill=c] (7.85075,12.0403) rectangle (7.89055,12.1461);
\draw [color=c, fill=c] (7.89055,12.0403) rectangle (7.93035,12.1461);
\draw [color=c, fill=c] (7.93035,12.0403) rectangle (7.97015,12.1461);
\draw [color=c, fill=c] (7.97015,12.0403) rectangle (8.00995,12.1461);
\draw [color=c, fill=c] (8.00995,12.0403) rectangle (8.04975,12.1461);
\draw [color=c, fill=c] (8.04975,12.0403) rectangle (8.08955,12.1461);
\draw [color=c, fill=c] (8.08955,12.0403) rectangle (8.12935,12.1461);
\draw [color=c, fill=c] (8.12935,12.0403) rectangle (8.16915,12.1461);
\draw [color=c, fill=c] (8.16915,12.0403) rectangle (8.20895,12.1461);
\draw [color=c, fill=c] (8.20895,12.0403) rectangle (8.24876,12.1461);
\draw [color=c, fill=c] (8.24876,12.0403) rectangle (8.28856,12.1461);
\draw [color=c, fill=c] (8.28856,12.0403) rectangle (8.32836,12.1461);
\draw [color=c, fill=c] (8.32836,12.0403) rectangle (8.36816,12.1461);
\draw [color=c, fill=c] (8.36816,12.0403) rectangle (8.40796,12.1461);
\draw [color=c, fill=c] (8.40796,12.0403) rectangle (8.44776,12.1461);
\draw [color=c, fill=c] (8.44776,12.0403) rectangle (8.48756,12.1461);
\draw [color=c, fill=c] (8.48756,12.0403) rectangle (8.52736,12.1461);
\draw [color=c, fill=c] (8.52736,12.0403) rectangle (8.56716,12.1461);
\draw [color=c, fill=c] (8.56716,12.0403) rectangle (8.60697,12.1461);
\draw [color=c, fill=c] (8.60697,12.0403) rectangle (8.64677,12.1461);
\draw [color=c, fill=c] (8.64677,12.0403) rectangle (8.68657,12.1461);
\draw [color=c, fill=c] (8.68657,12.0403) rectangle (8.72637,12.1461);
\draw [color=c, fill=c] (8.72637,12.0403) rectangle (8.76617,12.1461);
\draw [color=c, fill=c] (8.76617,12.0403) rectangle (8.80597,12.1461);
\draw [color=c, fill=c] (8.80597,12.0403) rectangle (8.84577,12.1461);
\draw [color=c, fill=c] (8.84577,12.0403) rectangle (8.88557,12.1461);
\draw [color=c, fill=c] (8.88557,12.0403) rectangle (8.92537,12.1461);
\draw [color=c, fill=c] (8.92537,12.0403) rectangle (8.96517,12.1461);
\draw [color=c, fill=c] (8.96517,12.0403) rectangle (9.00498,12.1461);
\draw [color=c, fill=c] (9.00498,12.0403) rectangle (9.04478,12.1461);
\draw [color=c, fill=c] (9.04478,12.0403) rectangle (9.08458,12.1461);
\draw [color=c, fill=c] (9.08458,12.0403) rectangle (9.12438,12.1461);
\draw [color=c, fill=c] (9.12438,12.0403) rectangle (9.16418,12.1461);
\draw [color=c, fill=c] (9.16418,12.0403) rectangle (9.20398,12.1461);
\draw [color=c, fill=c] (9.20398,12.0403) rectangle (9.24378,12.1461);
\draw [color=c, fill=c] (9.24378,12.0403) rectangle (9.28358,12.1461);
\draw [color=c, fill=c] (9.28358,12.0403) rectangle (9.32338,12.1461);
\draw [color=c, fill=c] (9.32338,12.0403) rectangle (9.36318,12.1461);
\draw [color=c, fill=c] (9.36318,12.0403) rectangle (9.40298,12.1461);
\draw [color=c, fill=c] (9.40298,12.0403) rectangle (9.44279,12.1461);
\draw [color=c, fill=c] (9.44279,12.0403) rectangle (9.48259,12.1461);
\draw [color=c, fill=c] (9.48259,12.0403) rectangle (9.52239,12.1461);
\draw [color=c, fill=c] (9.52239,12.0403) rectangle (9.56219,12.1461);
\draw [color=c, fill=c] (9.56219,12.0403) rectangle (9.60199,12.1461);
\draw [color=c, fill=c] (9.60199,12.0403) rectangle (9.64179,12.1461);
\draw [color=c, fill=c] (9.64179,12.0403) rectangle (9.68159,12.1461);
\draw [color=c, fill=c] (9.68159,12.0403) rectangle (9.72139,12.1461);
\draw [color=c, fill=c] (9.72139,12.0403) rectangle (9.76119,12.1461);
\draw [color=c, fill=c] (9.76119,12.0403) rectangle (9.80099,12.1461);
\draw [color=c, fill=c] (9.80099,12.0403) rectangle (9.8408,12.1461);
\draw [color=c, fill=c] (9.8408,12.0403) rectangle (9.8806,12.1461);
\draw [color=c, fill=c] (9.8806,12.0403) rectangle (9.9204,12.1461);
\draw [color=c, fill=c] (9.9204,12.0403) rectangle (9.9602,12.1461);
\draw [color=c, fill=c] (9.9602,12.0403) rectangle (10,12.1461);
\draw [color=c, fill=c] (10,12.0403) rectangle (10.0398,12.1461);
\draw [color=c, fill=c] (10.0398,12.0403) rectangle (10.0796,12.1461);
\draw [color=c, fill=c] (10.0796,12.0403) rectangle (10.1194,12.1461);
\draw [color=c, fill=c] (10.1194,12.0403) rectangle (10.1592,12.1461);
\draw [color=c, fill=c] (10.1592,12.0403) rectangle (10.199,12.1461);
\draw [color=c, fill=c] (10.199,12.0403) rectangle (10.2388,12.1461);
\draw [color=c, fill=c] (10.2388,12.0403) rectangle (10.2786,12.1461);
\draw [color=c, fill=c] (10.2786,12.0403) rectangle (10.3184,12.1461);
\draw [color=c, fill=c] (10.3184,12.0403) rectangle (10.3582,12.1461);
\draw [color=c, fill=c] (10.3582,12.0403) rectangle (10.398,12.1461);
\draw [color=c, fill=c] (10.398,12.0403) rectangle (10.4378,12.1461);
\draw [color=c, fill=c] (10.4378,12.0403) rectangle (10.4776,12.1461);
\draw [color=c, fill=c] (10.4776,12.0403) rectangle (10.5174,12.1461);
\draw [color=c, fill=c] (10.5174,12.0403) rectangle (10.5572,12.1461);
\draw [color=c, fill=c] (10.5572,12.0403) rectangle (10.597,12.1461);
\draw [color=c, fill=c] (10.597,12.0403) rectangle (10.6368,12.1461);
\draw [color=c, fill=c] (10.6368,12.0403) rectangle (10.6766,12.1461);
\draw [color=c, fill=c] (10.6766,12.0403) rectangle (10.7164,12.1461);
\draw [color=c, fill=c] (10.7164,12.0403) rectangle (10.7562,12.1461);
\draw [color=c, fill=c] (10.7562,12.0403) rectangle (10.796,12.1461);
\draw [color=c, fill=c] (10.796,12.0403) rectangle (10.8358,12.1461);
\draw [color=c, fill=c] (10.8358,12.0403) rectangle (10.8756,12.1461);
\draw [color=c, fill=c] (10.8756,12.0403) rectangle (10.9154,12.1461);
\draw [color=c, fill=c] (10.9154,12.0403) rectangle (10.9552,12.1461);
\draw [color=c, fill=c] (10.9552,12.0403) rectangle (10.995,12.1461);
\draw [color=c, fill=c] (10.995,12.0403) rectangle (11.0348,12.1461);
\draw [color=c, fill=c] (11.0348,12.0403) rectangle (11.0746,12.1461);
\draw [color=c, fill=c] (11.0746,12.0403) rectangle (11.1144,12.1461);
\draw [color=c, fill=c] (11.1144,12.0403) rectangle (11.1542,12.1461);
\draw [color=c, fill=c] (11.1542,12.0403) rectangle (11.194,12.1461);
\draw [color=c, fill=c] (11.194,12.0403) rectangle (11.2338,12.1461);
\draw [color=c, fill=c] (11.2338,12.0403) rectangle (11.2736,12.1461);
\draw [color=c, fill=c] (11.2736,12.0403) rectangle (11.3134,12.1461);
\draw [color=c, fill=c] (11.3134,12.0403) rectangle (11.3532,12.1461);
\draw [color=c, fill=c] (11.3532,12.0403) rectangle (11.393,12.1461);
\draw [color=c, fill=c] (11.393,12.0403) rectangle (11.4328,12.1461);
\draw [color=c, fill=c] (11.4328,12.0403) rectangle (11.4726,12.1461);
\draw [color=c, fill=c] (11.4726,12.0403) rectangle (11.5124,12.1461);
\draw [color=c, fill=c] (11.5124,12.0403) rectangle (11.5522,12.1461);
\draw [color=c, fill=c] (11.5522,12.0403) rectangle (11.592,12.1461);
\draw [color=c, fill=c] (11.592,12.0403) rectangle (11.6318,12.1461);
\draw [color=c, fill=c] (11.6318,12.0403) rectangle (11.6716,12.1461);
\draw [color=c, fill=c] (11.6716,12.0403) rectangle (11.7114,12.1461);
\draw [color=c, fill=c] (11.7114,12.0403) rectangle (11.7512,12.1461);
\draw [color=c, fill=c] (11.7512,12.0403) rectangle (11.791,12.1461);
\draw [color=c, fill=c] (11.791,12.0403) rectangle (11.8308,12.1461);
\draw [color=c, fill=c] (11.8308,12.0403) rectangle (11.8706,12.1461);
\draw [color=c, fill=c] (11.8706,12.0403) rectangle (11.9104,12.1461);
\draw [color=c, fill=c] (11.9104,12.0403) rectangle (11.9502,12.1461);
\draw [color=c, fill=c] (11.9502,12.0403) rectangle (11.99,12.1461);
\draw [color=c, fill=c] (11.99,12.0403) rectangle (12.0299,12.1461);
\draw [color=c, fill=c] (12.0299,12.0403) rectangle (12.0697,12.1461);
\draw [color=c, fill=c] (12.0697,12.0403) rectangle (12.1095,12.1461);
\draw [color=c, fill=c] (12.1095,12.0403) rectangle (12.1493,12.1461);
\draw [color=c, fill=c] (12.1493,12.0403) rectangle (12.1891,12.1461);
\draw [color=c, fill=c] (12.1891,12.0403) rectangle (12.2289,12.1461);
\draw [color=c, fill=c] (12.2289,12.0403) rectangle (12.2687,12.1461);
\draw [color=c, fill=c] (12.2687,12.0403) rectangle (12.3085,12.1461);
\draw [color=c, fill=c] (12.3085,12.0403) rectangle (12.3483,12.1461);
\draw [color=c, fill=c] (12.3483,12.0403) rectangle (12.3881,12.1461);
\draw [color=c, fill=c] (12.3881,12.0403) rectangle (12.4279,12.1461);
\draw [color=c, fill=c] (12.4279,12.0403) rectangle (12.4677,12.1461);
\draw [color=c, fill=c] (12.4677,12.0403) rectangle (12.5075,12.1461);
\draw [color=c, fill=c] (12.5075,12.0403) rectangle (12.5473,12.1461);
\draw [color=c, fill=c] (12.5473,12.0403) rectangle (12.5871,12.1461);
\draw [color=c, fill=c] (12.5871,12.0403) rectangle (12.6269,12.1461);
\draw [color=c, fill=c] (12.6269,12.0403) rectangle (12.6667,12.1461);
\draw [color=c, fill=c] (12.6667,12.0403) rectangle (12.7065,12.1461);
\draw [color=c, fill=c] (12.7065,12.0403) rectangle (12.7463,12.1461);
\draw [color=c, fill=c] (12.7463,12.0403) rectangle (12.7861,12.1461);
\draw [color=c, fill=c] (12.7861,12.0403) rectangle (12.8259,12.1461);
\draw [color=c, fill=c] (12.8259,12.0403) rectangle (12.8657,12.1461);
\draw [color=c, fill=c] (12.8657,12.0403) rectangle (12.9055,12.1461);
\draw [color=c, fill=c] (12.9055,12.0403) rectangle (12.9453,12.1461);
\draw [color=c, fill=c] (12.9453,12.0403) rectangle (12.9851,12.1461);
\draw [color=c, fill=c] (12.9851,12.0403) rectangle (13.0249,12.1461);
\draw [color=c, fill=c] (13.0249,12.0403) rectangle (13.0647,12.1461);
\draw [color=c, fill=c] (13.0647,12.0403) rectangle (13.1045,12.1461);
\draw [color=c, fill=c] (13.1045,12.0403) rectangle (13.1443,12.1461);
\draw [color=c, fill=c] (13.1443,12.0403) rectangle (13.1841,12.1461);
\draw [color=c, fill=c] (13.1841,12.0403) rectangle (13.2239,12.1461);
\draw [color=c, fill=c] (13.2239,12.0403) rectangle (13.2637,12.1461);
\draw [color=c, fill=c] (13.2637,12.0403) rectangle (13.3035,12.1461);
\draw [color=c, fill=c] (13.3035,12.0403) rectangle (13.3433,12.1461);
\draw [color=c, fill=c] (13.3433,12.0403) rectangle (13.3831,12.1461);
\draw [color=c, fill=c] (13.3831,12.0403) rectangle (13.4229,12.1461);
\draw [color=c, fill=c] (13.4229,12.0403) rectangle (13.4627,12.1461);
\draw [color=c, fill=c] (13.4627,12.0403) rectangle (13.5025,12.1461);
\draw [color=c, fill=c] (13.5025,12.0403) rectangle (13.5423,12.1461);
\draw [color=c, fill=c] (13.5423,12.0403) rectangle (13.5821,12.1461);
\draw [color=c, fill=c] (13.5821,12.0403) rectangle (13.6219,12.1461);
\draw [color=c, fill=c] (13.6219,12.0403) rectangle (13.6617,12.1461);
\draw [color=c, fill=c] (13.6617,12.0403) rectangle (13.7015,12.1461);
\draw [color=c, fill=c] (13.7015,12.0403) rectangle (13.7413,12.1461);
\draw [color=c, fill=c] (13.7413,12.0403) rectangle (13.7811,12.1461);
\draw [color=c, fill=c] (13.7811,12.0403) rectangle (13.8209,12.1461);
\draw [color=c, fill=c] (13.8209,12.0403) rectangle (13.8607,12.1461);
\draw [color=c, fill=c] (13.8607,12.0403) rectangle (13.9005,12.1461);
\draw [color=c, fill=c] (13.9005,12.0403) rectangle (13.9403,12.1461);
\draw [color=c, fill=c] (13.9403,12.0403) rectangle (13.9801,12.1461);
\draw [color=c, fill=c] (13.9801,12.0403) rectangle (14.0199,12.1461);
\draw [color=c, fill=c] (14.0199,12.0403) rectangle (14.0597,12.1461);
\draw [color=c, fill=c] (14.0597,12.0403) rectangle (14.0995,12.1461);
\draw [color=c, fill=c] (14.0995,12.0403) rectangle (14.1393,12.1461);
\draw [color=c, fill=c] (14.1393,12.0403) rectangle (14.1791,12.1461);
\draw [color=c, fill=c] (14.1791,12.0403) rectangle (14.2189,12.1461);
\draw [color=c, fill=c] (14.2189,12.0403) rectangle (14.2587,12.1461);
\draw [color=c, fill=c] (14.2587,12.0403) rectangle (14.2985,12.1461);
\draw [color=c, fill=c] (14.2985,12.0403) rectangle (14.3383,12.1461);
\draw [color=c, fill=c] (14.3383,12.0403) rectangle (14.3781,12.1461);
\draw [color=c, fill=c] (14.3781,12.0403) rectangle (14.4179,12.1461);
\draw [color=c, fill=c] (14.4179,12.0403) rectangle (14.4577,12.1461);
\draw [color=c, fill=c] (14.4577,12.0403) rectangle (14.4975,12.1461);
\draw [color=c, fill=c] (14.4975,12.0403) rectangle (14.5373,12.1461);
\draw [color=c, fill=c] (14.5373,12.0403) rectangle (14.5771,12.1461);
\draw [color=c, fill=c] (14.5771,12.0403) rectangle (14.6169,12.1461);
\draw [color=c, fill=c] (14.6169,12.0403) rectangle (14.6567,12.1461);
\draw [color=c, fill=c] (14.6567,12.0403) rectangle (14.6965,12.1461);
\draw [color=c, fill=c] (14.6965,12.0403) rectangle (14.7363,12.1461);
\draw [color=c, fill=c] (14.7363,12.0403) rectangle (14.7761,12.1461);
\draw [color=c, fill=c] (14.7761,12.0403) rectangle (14.8159,12.1461);
\draw [color=c, fill=c] (14.8159,12.0403) rectangle (14.8557,12.1461);
\draw [color=c, fill=c] (14.8557,12.0403) rectangle (14.8955,12.1461);
\draw [color=c, fill=c] (14.8955,12.0403) rectangle (14.9353,12.1461);
\draw [color=c, fill=c] (14.9353,12.0403) rectangle (14.9751,12.1461);
\draw [color=c, fill=c] (14.9751,12.0403) rectangle (15.0149,12.1461);
\draw [color=c, fill=c] (15.0149,12.0403) rectangle (15.0547,12.1461);
\draw [color=c, fill=c] (15.0547,12.0403) rectangle (15.0945,12.1461);
\draw [color=c, fill=c] (15.0945,12.0403) rectangle (15.1343,12.1461);
\draw [color=c, fill=c] (15.1343,12.0403) rectangle (15.1741,12.1461);
\draw [color=c, fill=c] (15.1741,12.0403) rectangle (15.2139,12.1461);
\draw [color=c, fill=c] (15.2139,12.0403) rectangle (15.2537,12.1461);
\draw [color=c, fill=c] (15.2537,12.0403) rectangle (15.2935,12.1461);
\draw [color=c, fill=c] (15.2935,12.0403) rectangle (15.3333,12.1461);
\draw [color=c, fill=c] (15.3333,12.0403) rectangle (15.3731,12.1461);
\draw [color=c, fill=c] (15.3731,12.0403) rectangle (15.4129,12.1461);
\draw [color=c, fill=c] (15.4129,12.0403) rectangle (15.4527,12.1461);
\draw [color=c, fill=c] (15.4527,12.0403) rectangle (15.4925,12.1461);
\draw [color=c, fill=c] (15.4925,12.0403) rectangle (15.5323,12.1461);
\draw [color=c, fill=c] (15.5323,12.0403) rectangle (15.5721,12.1461);
\draw [color=c, fill=c] (15.5721,12.0403) rectangle (15.6119,12.1461);
\draw [color=c, fill=c] (15.6119,12.0403) rectangle (15.6517,12.1461);
\draw [color=c, fill=c] (15.6517,12.0403) rectangle (15.6915,12.1461);
\draw [color=c, fill=c] (15.6915,12.0403) rectangle (15.7313,12.1461);
\draw [color=c, fill=c] (15.7313,12.0403) rectangle (15.7711,12.1461);
\draw [color=c, fill=c] (15.7711,12.0403) rectangle (15.8109,12.1461);
\draw [color=c, fill=c] (15.8109,12.0403) rectangle (15.8507,12.1461);
\draw [color=c, fill=c] (15.8507,12.0403) rectangle (15.8905,12.1461);
\draw [color=c, fill=c] (15.8905,12.0403) rectangle (15.9303,12.1461);
\draw [color=c, fill=c] (15.9303,12.0403) rectangle (15.9701,12.1461);
\draw [color=c, fill=c] (15.9701,12.0403) rectangle (16.01,12.1461);
\draw [color=c, fill=c] (16.01,12.0403) rectangle (16.0498,12.1461);
\draw [color=c, fill=c] (16.0498,12.0403) rectangle (16.0896,12.1461);
\draw [color=c, fill=c] (16.0896,12.0403) rectangle (16.1294,12.1461);
\draw [color=c, fill=c] (16.1294,12.0403) rectangle (16.1692,12.1461);
\draw [color=c, fill=c] (16.1692,12.0403) rectangle (16.209,12.1461);
\draw [color=c, fill=c] (16.209,12.0403) rectangle (16.2488,12.1461);
\draw [color=c, fill=c] (16.2488,12.0403) rectangle (16.2886,12.1461);
\draw [color=c, fill=c] (16.2886,12.0403) rectangle (16.3284,12.1461);
\draw [color=c, fill=c] (16.3284,12.0403) rectangle (16.3682,12.1461);
\draw [color=c, fill=c] (16.3682,12.0403) rectangle (16.408,12.1461);
\draw [color=c, fill=c] (16.408,12.0403) rectangle (16.4478,12.1461);
\draw [color=c, fill=c] (16.4478,12.0403) rectangle (16.4876,12.1461);
\draw [color=c, fill=c] (16.4876,12.0403) rectangle (16.5274,12.1461);
\draw [color=c, fill=c] (16.5274,12.0403) rectangle (16.5672,12.1461);
\draw [color=c, fill=c] (16.5672,12.0403) rectangle (16.607,12.1461);
\draw [color=c, fill=c] (16.607,12.0403) rectangle (16.6468,12.1461);
\draw [color=c, fill=c] (16.6468,12.0403) rectangle (16.6866,12.1461);
\draw [color=c, fill=c] (16.6866,12.0403) rectangle (16.7264,12.1461);
\draw [color=c, fill=c] (16.7264,12.0403) rectangle (16.7662,12.1461);
\draw [color=c, fill=c] (16.7662,12.0403) rectangle (16.806,12.1461);
\draw [color=c, fill=c] (16.806,12.0403) rectangle (16.8458,12.1461);
\draw [color=c, fill=c] (16.8458,12.0403) rectangle (16.8856,12.1461);
\draw [color=c, fill=c] (16.8856,12.0403) rectangle (16.9254,12.1461);
\draw [color=c, fill=c] (16.9254,12.0403) rectangle (16.9652,12.1461);
\draw [color=c, fill=c] (16.9652,12.0403) rectangle (17.005,12.1461);
\draw [color=c, fill=c] (17.005,12.0403) rectangle (17.0448,12.1461);
\draw [color=c, fill=c] (17.0448,12.0403) rectangle (17.0846,12.1461);
\draw [color=c, fill=c] (17.0846,12.0403) rectangle (17.1244,12.1461);
\draw [color=c, fill=c] (17.1244,12.0403) rectangle (17.1642,12.1461);
\draw [color=c, fill=c] (17.1642,12.0403) rectangle (17.204,12.1461);
\draw [color=c, fill=c] (17.204,12.0403) rectangle (17.2438,12.1461);
\draw [color=c, fill=c] (17.2438,12.0403) rectangle (17.2836,12.1461);
\draw [color=c, fill=c] (17.2836,12.0403) rectangle (17.3234,12.1461);
\draw [color=c, fill=c] (17.3234,12.0403) rectangle (17.3632,12.1461);
\draw [color=c, fill=c] (17.3632,12.0403) rectangle (17.403,12.1461);
\draw [color=c, fill=c] (17.403,12.0403) rectangle (17.4428,12.1461);
\draw [color=c, fill=c] (17.4428,12.0403) rectangle (17.4826,12.1461);
\draw [color=c, fill=c] (17.4826,12.0403) rectangle (17.5224,12.1461);
\draw [color=c, fill=c] (17.5224,12.0403) rectangle (17.5622,12.1461);
\draw [color=c, fill=c] (17.5622,12.0403) rectangle (17.602,12.1461);
\draw [color=c, fill=c] (17.602,12.0403) rectangle (17.6418,12.1461);
\draw [color=c, fill=c] (17.6418,12.0403) rectangle (17.6816,12.1461);
\draw [color=c, fill=c] (17.6816,12.0403) rectangle (17.7214,12.1461);
\draw [color=c, fill=c] (17.7214,12.0403) rectangle (17.7612,12.1461);
\draw [color=c, fill=c] (17.7612,12.0403) rectangle (17.801,12.1461);
\draw [color=c, fill=c] (17.801,12.0403) rectangle (17.8408,12.1461);
\draw [color=c, fill=c] (17.8408,12.0403) rectangle (17.8806,12.1461);
\draw [color=c, fill=c] (17.8806,12.0403) rectangle (17.9204,12.1461);
\draw [color=c, fill=c] (17.9204,12.0403) rectangle (17.9602,12.1461);
\draw [color=c, fill=c] (17.9602,12.0403) rectangle (18,12.1461);
\definecolor{c}{rgb}{0,0,0};
\draw [c,line width=0.9] (2,1.34957) -- (18,1.34957);
\draw [c,line width=0.9] (2,1.67347) -- (2,1.34957);
\draw [c,line width=0.9] (2.39801,1.51152) -- (2.39801,1.34957);
\draw [c,line width=0.9] (2.79602,1.51152) -- (2.79602,1.34957);
\draw [c,line width=0.9] (3.19403,1.51152) -- (3.19403,1.34957);
\draw [c,line width=0.9] (3.59204,1.51152) -- (3.59204,1.34957);
\draw [c,line width=0.9] (3.99005,1.67347) -- (3.99005,1.34957);
\draw [c,line width=0.9] (4.38806,1.51152) -- (4.38806,1.34957);
\draw [c,line width=0.9] (4.78607,1.51152) -- (4.78607,1.34957);
\draw [c,line width=0.9] (5.18408,1.51152) -- (5.18408,1.34957);
\draw [c,line width=0.9] (5.58209,1.51152) -- (5.58209,1.34957);
\draw [c,line width=0.9] (5.9801,1.67347) -- (5.9801,1.34957);
\draw [c,line width=0.9] (6.37811,1.51152) -- (6.37811,1.34957);
\draw [c,line width=0.9] (6.77612,1.51152) -- (6.77612,1.34957);
\draw [c,line width=0.9] (7.17413,1.51152) -- (7.17413,1.34957);
\draw [c,line width=0.9] (7.57214,1.51152) -- (7.57214,1.34957);
\draw [c,line width=0.9] (7.97015,1.67347) -- (7.97015,1.34957);
\draw [c,line width=0.9] (8.36816,1.51152) -- (8.36816,1.34957);
\draw [c,line width=0.9] (8.76617,1.51152) -- (8.76617,1.34957);
\draw [c,line width=0.9] (9.16418,1.51152) -- (9.16418,1.34957);
\draw [c,line width=0.9] (9.56219,1.51152) -- (9.56219,1.34957);
\draw [c,line width=0.9] (9.9602,1.67347) -- (9.9602,1.34957);
\draw [c,line width=0.9] (10.3582,1.51152) -- (10.3582,1.34957);
\draw [c,line width=0.9] (10.7562,1.51152) -- (10.7562,1.34957);
\draw [c,line width=0.9] (11.1542,1.51152) -- (11.1542,1.34957);
\draw [c,line width=0.9] (11.5522,1.51152) -- (11.5522,1.34957);
\draw [c,line width=0.9] (11.9502,1.67347) -- (11.9502,1.34957);
\draw [c,line width=0.9] (12.3483,1.51152) -- (12.3483,1.34957);
\draw [c,line width=0.9] (12.7463,1.51152) -- (12.7463,1.34957);
\draw [c,line width=0.9] (13.1443,1.51152) -- (13.1443,1.34957);
\draw [c,line width=0.9] (13.5423,1.51152) -- (13.5423,1.34957);
\draw [c,line width=0.9] (13.9403,1.67347) -- (13.9403,1.34957);
\draw [c,line width=0.9] (14.3383,1.51152) -- (14.3383,1.34957);
\draw [c,line width=0.9] (14.7363,1.51152) -- (14.7363,1.34957);
\draw [c,line width=0.9] (15.1343,1.51152) -- (15.1343,1.34957);
\draw [c,line width=0.9] (15.5323,1.51152) -- (15.5323,1.34957);
\draw [c,line width=0.9] (15.9303,1.67347) -- (15.9303,1.34957);
\draw [c,line width=0.9] (16.3284,1.51152) -- (16.3284,1.34957);
\draw [c,line width=0.9] (16.7264,1.51152) -- (16.7264,1.34957);
\draw [c,line width=0.9] (17.1244,1.51152) -- (17.1244,1.34957);
\draw [c,line width=0.9] (17.5224,1.51152) -- (17.5224,1.34957);
\draw [c,line width=0.9] (17.9204,1.67347) -- (17.9204,1.34957);
\draw [c,line width=0.9] (17.9204,1.67347) -- (17.9204,1.34957);
\draw [anchor=base] (2,0.904212) node[scale=1.01821, color=c, rotate=0]{0};
\draw [anchor=base] (3.99005,0.904212) node[scale=1.01821, color=c, rotate=0]{50};
\draw [anchor=base] (5.9801,0.904212) node[scale=1.01821, color=c, rotate=0]{100};
\draw [anchor=base] (7.97015,0.904212) node[scale=1.01821, color=c, rotate=0]{150};
\draw [anchor=base] (9.9602,0.904212) node[scale=1.01821, color=c, rotate=0]{200};
\draw [anchor=base] (11.9502,0.904212) node[scale=1.01821, color=c, rotate=0]{250};
\draw [anchor=base] (13.9403,0.904212) node[scale=1.01821, color=c, rotate=0]{300};
\draw [anchor=base] (15.9303,0.904212) node[scale=1.01821, color=c, rotate=0]{350};
\draw [anchor=base] (17.9204,0.904212) node[scale=1.01821, color=c, rotate=0]{400};
\draw [c,line width=0.9] (2,1.34957) -- (2,12.1461);
\draw [c,line width=0.9] (2.48,1.34957) -- (2,1.34957);
\draw [c,line width=0.9] (2.24,1.87881) -- (2,1.87881);
\draw [c,line width=0.9] (2.24,2.40806) -- (2,2.40806);
\draw [c,line width=0.9] (2.24,2.9373) -- (2,2.9373);
\draw [c,line width=0.9] (2.48,3.46654) -- (2,3.46654);
\draw [c,line width=0.9] (2.24,3.99579) -- (2,3.99579);
\draw [c,line width=0.9] (2.24,4.52503) -- (2,4.52503);
\draw [c,line width=0.9] (2.24,5.05427) -- (2,5.05427);
\draw [c,line width=0.9] (2.48,5.58352) -- (2,5.58352);
\draw [c,line width=0.9] (2.24,6.11276) -- (2,6.11276);
\draw [c,line width=0.9] (2.24,6.642) -- (2,6.642);
\draw [c,line width=0.9] (2.24,7.17125) -- (2,7.17125);
\draw [c,line width=0.9] (2.48,7.70049) -- (2,7.70049);
\draw [c,line width=0.9] (2.24,8.22973) -- (2,8.22973);
\draw [c,line width=0.9] (2.24,8.75898) -- (2,8.75898);
\draw [c,line width=0.9] (2.24,9.28822) -- (2,9.28822);
\draw [c,line width=0.9] (2.48,9.81746) -- (2,9.81746);
\draw [c,line width=0.9] (2.24,10.3467) -- (2,10.3467);
\draw [c,line width=0.9] (2.24,10.8759) -- (2,10.8759);
\draw [c,line width=0.9] (2.24,11.4052) -- (2,11.4052);
\draw [c,line width=0.9] (2.48,11.9344) -- (2,11.9344);
\draw [c,line width=0.9] (2.48,11.9344) -- (2,11.9344);
\draw [anchor= east] (1.9,1.34957) node[scale=1.01821, color=c, rotate=0]{0};
\draw [anchor= east] (1.9,3.46654) node[scale=1.01821, color=c, rotate=0]{20};
\draw [anchor= east] (1.9,5.58352) node[scale=1.01821, color=c, rotate=0]{40};
\draw [anchor= east] (1.9,7.70049) node[scale=1.01821, color=c, rotate=0]{60};
\draw [anchor= east] (1.9,9.81746) node[scale=1.01821, color=c, rotate=0]{80};
\draw [anchor= east] (1.9,11.9344) node[scale=1.01821, color=c, rotate=0]{100};
\definecolor{c}{rgb}{1,1,1};
\draw [color=c, fill=c] (13.7249,8.65329) rectangle (17.7364,11.8911);
\definecolor{c}{rgb}{0,0,0};
\draw [c,line width=0.9] (13.7249,8.65329) -- (17.7364,8.65329);
\draw [c,line width=0.9] (17.7364,8.65329) -- (17.7364,11.8911);
\draw [c,line width=0.9] (17.7364,11.8911) -- (13.7249,11.8911);
\draw [c,line width=0.9] (13.7249,11.8911) -- (13.7249,8.65329);
\draw (15.7307,11.6213) node[scale=1.08185, color=c, rotate=0]{hxy};
\draw [c,line width=0.9] (13.7249,11.3515) -- (17.7364,11.3515);
\draw [anchor= west] (13.9255,11.0817) node[scale=1.01821, color=c, rotate=0]{Entries };
\draw [anchor= east] (17.5358,11.0817) node[scale=1.01821, color=c, rotate=0]{ 40501};
\draw [anchor= west] (13.9255,10.542) node[scale=1.01821, color=c, rotate=0]{Mean x };
\draw [anchor= east] (17.5358,10.542) node[scale=1.01821, color=c, rotate=0]{  115.1};
\draw [anchor= west] (13.9255,10.0024) node[scale=1.01821, color=c, rotate=0]{Mean y };
\draw [anchor= east] (17.5358,10.0024) node[scale=1.01821, color=c, rotate=0]{ -115.5};
\draw [anchor= west] (13.9255,9.46275) node[scale=1.01821, color=c, rotate=0]{Std Dev x };
\draw [anchor= east] (17.5358,9.46275) node[scale=1.01821, color=c, rotate=0]{  77.65};
\draw [anchor= west] (13.9255,8.92311) node[scale=1.01821, color=c, rotate=0]{Std Dev y };
\draw [anchor= east] (17.5358,8.92311) node[scale=1.01821, color=c, rotate=0]{  154.7};
\definecolor{c}{rgb}{0.386667,0,1};
\draw [color=c, fill=c] (18.1,1.34957) rectangle (19,1.8894);
\definecolor{c}{rgb}{0.2,0,1};
\draw [color=c, fill=c] (18.1,1.8894) rectangle (19,2.42923);
\definecolor{c}{rgb}{0,0.0800001,1};
\draw [color=c, fill=c] (18.1,2.42923) rectangle (19,2.96905);
\definecolor{c}{rgb}{0,0.266667,1};
\draw [color=c, fill=c] (18.1,2.96905) rectangle (19,3.50888);
\definecolor{c}{rgb}{0,0.546666,1};
\draw [color=c, fill=c] (18.1,3.50888) rectangle (19,4.04871);
\definecolor{c}{rgb}{0,0.733333,1};
\draw [color=c, fill=c] (18.1,4.04871) rectangle (19,4.58854);
\definecolor{c}{rgb}{0,1,0.986667};
\draw [color=c, fill=c] (18.1,4.58854) rectangle (19,5.12837);
\definecolor{c}{rgb}{0,1,0.8};
\draw [color=c, fill=c] (18.1,5.12837) rectangle (19,5.66819);
\definecolor{c}{rgb}{0,1,0.52};
\draw [color=c, fill=c] (18.1,5.66819) rectangle (19,6.20802);
\definecolor{c}{rgb}{0,1,0.333333};
\draw [color=c, fill=c] (18.1,6.20802) rectangle (19,6.74785);
\definecolor{c}{rgb}{0,1,0.0533333};
\draw [color=c, fill=c] (18.1,6.74785) rectangle (19,7.28768);
\definecolor{c}{rgb}{0.133333,1,0};
\draw [color=c, fill=c] (18.1,7.28768) rectangle (19,7.82751);
\definecolor{c}{rgb}{0.413333,1,0};
\draw [color=c, fill=c] (18.1,7.82751) rectangle (19,8.36734);
\definecolor{c}{rgb}{0.6,1,0};
\draw [color=c, fill=c] (18.1,8.36734) rectangle (19,8.90716);
\definecolor{c}{rgb}{0.88,1,0};
\draw [color=c, fill=c] (18.1,8.90716) rectangle (19,9.44699);
\definecolor{c}{rgb}{1,0.933333,0};
\draw [color=c, fill=c] (18.1,9.44699) rectangle (19,9.98682);
\definecolor{c}{rgb}{1,0.653333,0};
\draw [color=c, fill=c] (18.1,9.98682) rectangle (19,10.5266);
\definecolor{c}{rgb}{1,0.466667,0};
\draw [color=c, fill=c] (18.1,10.5266) rectangle (19,11.0665);
\definecolor{c}{rgb}{1,0.186667,0};
\draw [color=c, fill=c] (18.1,11.0665) rectangle (19,11.6063);
\definecolor{c}{rgb}{1,0,0};
\draw [color=c, fill=c] (18.1,11.6063) rectangle (19,12.1461);
\definecolor{c}{rgb}{0,0,0};
\draw [c,line width=0.9] (19,1.34957) -- (19,12.1461);
\draw [c,line width=0.9] (18.52,2.48642) -- (19,2.48642);
\draw [c,line width=0.9] (18.52,4.08651) -- (19,4.08651);
\draw [c,line width=0.9] (18.52,5.6866) -- (19,5.6866);
\draw [c,line width=0.9] (18.52,7.28669) -- (19,7.28669);
\draw [c,line width=0.9] (18.52,8.88678) -- (19,8.88678);
\draw [c,line width=0.9] (18.52,10.4869) -- (19,10.4869);
\draw [c,line width=0.9] (18.52,12.087) -- (19,12.087);
\draw [c,line width=0.9] (18.52,2.48642) -- (19,2.48642);
\draw [c,line width=0.9] (18.52,12.087) -- (19,12.087);
\draw [anchor= west] (19.1,2.48642) node[scale=1.01821, color=c, rotate=0]{-0.0005};
\draw [anchor= west] (19.1,4.08651) node[scale=1.01821, color=c, rotate=0]{0};
\draw [anchor= west] (19.1,5.6866) node[scale=1.01821, color=c, rotate=0]{0.0005};
\draw [anchor= west] (19.1,7.28669) node[scale=1.01821, color=c, rotate=0]{0.001};
\draw [anchor= west] (19.1,8.88678) node[scale=1.01821, color=c, rotate=0]{0.0015};
\draw [anchor= west] (19.1,10.4869) node[scale=1.01821, color=c, rotate=0]{0.002};
\draw [anchor= west] (19.1,12.087) node[scale=1.01821, color=c, rotate=0]{0.0025};
\definecolor{c}{rgb}{1,1,1};
\draw [color=c, fill=c] (13.7249,8.65329) rectangle (17.7364,11.8911);
\definecolor{c}{rgb}{0,0,0};
\draw [c,line width=0.9] (13.7249,8.65329) -- (17.7364,8.65329);
\draw [c,line width=0.9] (17.7364,8.65329) -- (17.7364,11.8911);
\draw [c,line width=0.9] (17.7364,11.8911) -- (13.7249,11.8911);
\draw [c,line width=0.9] (13.7249,11.8911) -- (13.7249,8.65329);
\draw (15.7307,11.6213) node[scale=1.08185, color=c, rotate=0]{hxy};
\draw [c,line width=0.9] (13.7249,11.3515) -- (17.7364,11.3515);
\draw [anchor= west] (13.9255,11.0817) node[scale=1.01821, color=c, rotate=0]{Entries };
\draw [anchor= east] (17.5358,11.0817) node[scale=1.01821, color=c, rotate=0]{ 40501};
\draw [anchor= west] (13.9255,10.542) node[scale=1.01821, color=c, rotate=0]{Mean x };
\draw [anchor= east] (17.5358,10.542) node[scale=1.01821, color=c, rotate=0]{  115.1};
\draw [anchor= west] (13.9255,10.0024) node[scale=1.01821, color=c, rotate=0]{Mean y };
\draw [anchor= east] (17.5358,10.0024) node[scale=1.01821, color=c, rotate=0]{ -115.5};
\draw [anchor= west] (13.9255,9.46275) node[scale=1.01821, color=c, rotate=0]{Std Dev x };
\draw [anchor= east] (17.5358,9.46275) node[scale=1.01821, color=c, rotate=0]{  77.65};
\draw [anchor= west] (13.9255,8.92311) node[scale=1.01821, color=c, rotate=0]{Std Dev y };
\draw [anchor= east] (17.5358,8.92311) node[scale=1.01821, color=c, rotate=0]{  154.7};
\draw (10,13.0156) node[scale=1.52731, color=c, rotate=0]{MagneticField x in y = 0};
\end{tikzpicture}


\subsection{Decadimento del muone cosmico}
Se il muone cosmico viene fermato all'interno dell'assorbitore, esso decadrà dopo un tempo che dipende dal tipo di muone che si ferma (muone o antimuone), che è distribuito
come un esponenziale dal tempo caratteristico che viene assunto come noto dalla letteratura. Perciò si considera il muone, fermo, trascorra un tempo che in media è il tempo
di vita di tale muone, e poi decada emettendo sostanzialmente un elettrone (non sono rilevabili gli altri prodotti del decadimento). \'E utile introdurre inoltre lo spin.
!!!!!ENRICO PARLA DI COME HAI TRATTATO LO SPIN!!!!!!!

\subsection{Implementazione dell'elettrone}
Una volta che il programma di simulazione genera un elettrone con la sua posizione e la sua direzione, taale elettrone viene fatto evolvere \" all'indietro \" rispetto a
come è stato fatto evolvere il muone inizialmente, e si considera se esso viene riassorbito all'interno dell'assorbitore e in quali scintillatori lascia segnali, e quanto
intensi sono tali segnali.

\subsection{Output della simulazione}
Gli output utili della simulazione descritta ai punti precedenti sono numerosi e interessanti:
\begin{intemize}
\item Efficienza: considerando solamente l'interazione tra i raggi cosmici e gli scintillatori è possibile stimare quanti muoni non risultano in coincidenza a causa
di condizioni geometriche non favorevoli, per esempio quelli che lasciano segnale attraversando ai bordi i due rivelatori superiori ma non entrano nemmeno nel terzo
rivelatore
\item Spettro temporale: data tutta la simulazione fatta, è possibile fare un plot del tempo che intercorre tra il passaggio del muone e quello dell'elettrone, riottenendo
l'esponenziale del tempo di decadimento
\item !!!!!!!Cos'era quel grafico figo?!!!!!!!!!!!!!!
\end{itemize}
