Dato che per completare l'esperimento si ha bisogno di anccora un semestre di lavoro, non si hanno al momento a disposizione dati relativi al reale obiettivo dell'esperimento
(per esempio alla misura del fattore di Landè dei muoni o al tempo di decadimento dei muoni), ma si hanno dati sulla caratterizzazione di parte del sistema di acquisizione,
in particolare sulla caratterizzazione dei fotodiodi e sulla misura di efficienza degli scintillatori.
\subsection{Caratterizzazione dei fotodiodi}
I fotodiodi sono stati caratterizzati in due modi: si è prima studiata la caratteristica tensione corrente (solo per alcuni fotodiodi visto che questa misura ha un fine
didattico più che utilità per l'esperimento stesso), e poi si è studiata la variazione del parametro di guadagno per ogni fotone al variare del voltaggio di bias per ogni
fotodiodo utilizzato.

\subsubsection{Studio della caratteristica tensione corrente}
Per studiare tale caratteristica si è utilizzato un picoamperometro collegato al diodo: Esso permette, in maniera simile a quanto viene fatto dai multimetri commerciali
impostati come ohmetri, di fornire una ben definita tensione e di misurare la corrente che attraversa l'oggetto generata da questa tensione. Quindi,
non si è fatto altro che mettere la PCB al buio (in modo da non rilevare una quantità troppo elevata di fotoni esterni quando si dà una tensione di bias al diodo),
collegare il picoamperometro al diodo e studiare come varia la corrente al variare della tensione fornita, ottenendo la curva caratteristica del diodo nelle sue tre diverse
sezioni: quellla del voltaggio diretto, dello spegnimento e del breakdown; quest'ultima risulta particolarmente interessante in quanto è in questa regione che funzioneranno
i diodi unaa volta collegato tutto l'esperimento. Le curve di caratterizzazione si possono vedere nel Grafici \ref{} e \ref{} per due dei diodi utilizzati
%!!!!!!!!!INSERIRE GRAFICI CARATTERIZZAZIONE IV DEI DIODI E POI METTERE IL NOME SUL REF ALLA RIGA SOPRA!!!!!!!!!!!!

Partendo da questi grafici si possono studiare diverse caratteristiche del fotodiodo, per esempio andando a considerare solamente i punti raccolti per voltaggi oltre
il voltaggio di breakdown si può andare a interpolare tali daatti con una retta ottenendo così la conduttanza equivalente del circuito quando il diodo è in breakdown.
%!!!!!!!!!INSERIRE GRAFICI PER PARTE IN BREAKDOWN E VALORE CONDUTTANZA PER I GRAFICI CHE SI HANNO!!!!!!!!!!!!!!!


\subsubsection{Studio dell'amplificazione dei fotodiodi}
Molto importtante per la regolazione del voltaggio di bias per i singoli fotodiodi è sapere esattamente il voltaggio di rbeakdown di tali diodi e a quale variazione di
voltaggio sia associato l'assorbimento di un fotone da parte di un fotodiodo. Per fare questo si è alimentato l'operazionale nella scheda contenente il fotodiodo, tale
scheda è stata messa al buio, e si è collegato il bias del fotodiodo al generatore di tensione, e l'output all'oscilloscopio. Quindi, si è fatta variare la tensione
di bias del fotodiodo e si sono raccolti un numero fisso di dati, come quelli che si possono vedere nel Grafico \ref{gr:}, che è uno dei tanti grafici che sono stati
%!!!!!!!!!!!!!!!!INSERIRE GRAFICO CON ESEMPIO SERIE GAUSSIANE; SISTEMARE REF E DATI POSO SOPRA!!!!!!!!!!!!!
trovati per studiare l'amplificazione dei fotodiodi. Da questo grafico è evidente come ci sono diverse gaussiane ben distinte, ad indicare che si vede il voltaggio
generato da un numero crescente di fotoni (infatti la gaussiana a voltaggio più basso sarà quella legata a un fotone, quella più a sinistra due fotoni eccetera).
Quelli che però sono visualizzati non sono reali fotoni ma pseudofotoni: infatti il diodo non è collegato al rivelatore e, oltretutto, è al buio, e quelli che si
vedono sono \"fotoni termici\", cioè eccitaazioni casuali nel semiconduttore che forma il diodo che vengono lette dal sistema come se fosse stato assorbito un fotone
da tale diodo. Grafici di questo tipo sono stati interpolati al variare della tensione di bias per ogni diodo con una funzione del tipo:
\begin{equation}
%!!!!!!!!!!!!!METTERE EQUAZIONE INTERPOLAZIONE DELLE TANTE GAUSSIANE!!!!!!!!!!!!
\end{equation}
Dove i coefficienti indicano !!!!!!!!!!!!!! e si può usare questa equazione perché !!!!!!!!!!!!!!!.\\

Mettendo assieme tutti i grafici per ogni diodo si ottengono delle rette che descrivono il variare dell'amplificazione (cioè in pratica del voltaggio per fotone) al variare
del voltaggio di bias. Questi grafici si possono vedere in questa sezione, e nella Tabella \ref{tab:} si possono vedere riassunti i risultati per ogni diodo.
%!!!!!!!!!!!!!METTERE GRAFICI RETTTA AL VARIARE DEL VOLTAGGIO E TABELLA CON RISULTATI PER TUTTI I DIODI STUDIATI!!!!!!!!!!!!!

\subsection{Stima dell'efficienza dell'apparato}
\'E stata fatta una seconda serie di misure per poter discutere dell'efficienza del sistema di acquisizione.
