La scheda di acquisizione ``Arietta'' è una scheda dal funzionamento relativamente semplice: essa è in grado di misurare la distanza temporale tra due segnali sufficientemente grandi da superare la sua soglia.\\
Essa va collegata ad un computer tramite presa ethernet e configurata da esso, dopodiché è necessario collegare tramite un cavo \textit{lemo} l'uscita del sistema di acquisizione a tale schedina con processore.\\
Arietta salva le differenze temporali tra i due segnali solamente quando tali differenze sono in un certo range di funzionamento (che va da qualche nanosecondo a circa una decina di microsecondi, per dei valori più precisi si fa riferimento alla calibrazione fatta poco più avanti), come \textit{tempi di clock} del processore interno, il che permette di misurare effettivamente le differenze temporali, è però necessario eseguire una calibrazione per capire quali tempi corrispondono ai tempi di clock. Inoltre, Arietta salva i dati su un buffer interno che va poi letto e svuotato manualmente utilizzando il computer che comanda la schedina stessa, e il buffer stesso ha una capienza di 4095 dati: questo non è stato un problema dal punto di vista sperimentale in quanto il rate di muoni (che si è studiato più avanti) è sufficientemente basso da permettere una facile lettura di Arietta senza perdere dati a causa dei tempi morti legati ai software di lettura stessi.\\

\subsubsection{Calibrazione}
Per effettuare la calibrazione si è utilizzato l'oscilloscopio come generatore di funzioni, e si è generata una funzione periodica data da due onde quadre relativamente vicine seguite da un tempo in cui la tensione era zero. Tali onde sono state impostate in maniera tale che esse fossero di lunghezza simile al segnale logico che poi è stato utilizzato come reale input di Arietta (proveniente dal discriminatore, di una lunghezza di circa 200~ns e di un'altezza di circa 2.0~V), come si può vedere nell'Immagine~\ref{img:calibration}.\\
\inputimg{calibration}

Una volta ottenuta l'onda voluta, si è collegata l'uscita del generatore di funzioni direttamente all'oscilloscopio in modo da fare una misura della distanza reale tra le due onde in termini di tempo, dopodiché si è cambiata la distanza tra i due impulsi di forma quadrata per poter andare a studiare il numero di tempi di clock associato a diversi tempi.\\
Una prima parte della calibrazione è stata fatta andando ad indagare a tempi distribuiti in tutta la zona in cui Arietta riesce a registrare: sono state prese 4095 misure per ogni tempo e si è fatta una media dei clock cycles ottenuti in modo da poter trovare la funzione di calibrazione
$$y = mx +q$$
con x clock cycles e y tempi in ns, in modo che poi fosse possibile fare una calibrazione del grafico finale ottenuto con i muoni.\\

Particolare attenzione si è fatta con la linearità di Arietta: si è voluto studiare quanto Arietta fosse lineare, soprattutto a bassi tempi e, quindi, per pochi cicli di clock. Per fare ciò come prima cosa si sono raccolti dei dati in sede sperimentale con la distanza tra i due segnali molto piccola, poi si è passati alla vera e propria analisi dati.\\

In pratica sono stati raccolti due campioni differenti: un campione esteso per tempi diversi che vanno da $1.8 \mu\text{s}$ a $10.7 \mu\text{s}$, che chiameremo $\mathcal{C}_1$, e un campione esteso solamente a tempi molto bassi tra 215~ns e $1.95\mu\text{s}$, che chiameremo $\mathcal{C}_2$, poi il campione $\mathcal{C}_3$ è dato dall'unione dei due campioni. Un ulteriore campione $\mathcal{C}_f$ è presente in tabella, e la sua descrizione verrà data più avanti. Non si è indagato al di sotto dei 215~ns in quanto allo step successivo che è stato possibile fare con il generatore di funzioni (che consisteva in 212~ns) Arietta non ha raccolto alcun dato. Le analisi sui tre campioni sono state fatte allo stesso modo, in modo da identificare eventuali problemi di Arietta: come prima cosa si è fatta un'interpolazione lineare dei dati per ottenere i coefficienti, poi si è fatta un'interpolazione quadratica per ottenere il coefficiente di ordine superiore e verificare quanto esso sia compatibile con zero, infine si è fatto un F-test per andare a studiare la necessità sull'introduzione di tale coefficiente del secondo ordine.\\

Le grandezze calcolate si possono leggere nella Tabella~\ref{tab:linearity} dove sono state fatte le interpolazioni per i tre campioni e se ne presentano coefficienti per l'interpolazione lineare ($m$ e $q$) e quelli per l'interpolazione quadratica ($m'$, $q'$, $\varepsilon$).\\
\inputtab{linearity}

Nella Tabella~\ref{tab:Ftest} si riportano invece gli esiti dell'F-test, in particolare il numero di gradi di libertà di numeratore e denominatore ($d_N$ e $d_D$), l'esito dell'F-test ($F$) e la probabilità corrispondete ($P_F$). La probabilità serve nel senso che, per esempio, una $P_F = 0.99$ significa che che si può rigettare con una confidenza del 99\% l'ipotesi secondo cui i modelli sono equivalenti, e quindi c'è necessità di utilizzare il modello più complesso (cioè con un parametro aggiuntivo).\\
\inputtab{Ftest}

D queste tabelle si evince come, effettivamente, ci sia un problema di non linearità a bassi tempi. Infatti, dalla stima dei coefficienti di secondo ordine, si vede come il $\mathcal{C}1$, che si ferma a $1.79 \mu\text{s}$ abbia un coefficiente molto compatibile con lo zero mentre gli altri due campioni, che arrivano fino a 215~ns hanno un coefficiente scarsamente compatibile con zero.\\

Guardando il risultato dell'F-test si vede che nel caso del primo campione non è possibile rigettare l'ipotesi di linearità nemmeno di $\sigma$, mentre il secondo campione in particolare rigetta la linearità di Arietta con un livello di confidenza pari al 99.4\%, decisamente alto.\\


Per scorrevolezza di lettura non si trovano qui tutte le interpolazioni fatte, che si possono trovare nelle appendici, ma si riportano esclusivamente le interpolazioni fatte su $\mathcal{C}2$  che rivelano la non linearità di Arietta a bassi tempi, che si possono vedere nell'Immagine~\ref{gr:tempi_bassi}.\\
\inputgraph{tempi_bassi}

Per ovviare al problema della non linearità si è capito (guardando i grafici) che le non linearità sorgono nel momento in cui i tempi diventano inferiori ai 335~ns, quindi si è deciso di ripetere le interpolazioni fatte per i campioni 1, 2 e 3 utilizzando solamente i dati sopra tale soglia, che formano il campione $\mathcal{C}_f$ le cui proprietà si possono vedere nella Tabella~\ref{tab:linearity}. Nell'immagine \ref{gr:final_calibration} si possono vedere le interpolazioni effettuate su tale campione, che effettivamente ha passato bene l'F-test.\\
\inputgraph{final_calibration}

Dato questo problema che si è trovato con la linearità di Arietta, in tutte le misure effettuate utilizzando tale scheda di acquisizione sono state tagliate le misura per tempi inferiori a 335~ns in quanto affette da un errore sistematico che non si è in grado di correggere.\\

Le misure finali della calibrazione indicano che i coefficienti  di calibrazione sono:
\begin{equation}
  m = (14.994 \pm 0.007) \text{ns/cycles} \hspace{2cm} q = (6 \pm 2) \text{ns}
\end{equation}


\subsubsection{Efficienza}
I dati che hanno permesso di studiare la linearità di Arietta sono stati utilizzati anche per stimarne l'efficienza: infatti i 4095 dati raccolti a differenze temporali ben definite provenivano da  4095 coppie di segnali e non di più: se la scheda di acquisizione avesse misurato meno dati di quelli inviati ad una certa differenza temporale allora ci sarebbe stato un problema di efficienza. Invece, per quanto riguarda Arietta, l'inefficienza sopra i 212~ns è ampiamente trascurabile, in quanto sono sempre stati letti tutti i segnali inviati. In particolare, dato il  numero di  dati utilizzati, l'inefficienza risulta al più dell'ordine del percento.
