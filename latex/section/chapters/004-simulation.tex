\section{Simulazioni}
Per poter lavorare al meglio con l'apparato strumentale dato e comprendere al meglio il meccanismo di funzionamento si è deciso di scrivere un programma in grado di simulare
il processo che avviene durante le vere e proprie sessioni di misura. In questa sezione si vuole descrivere brevemente il funzionamento di tale programma di simulazione e
l'output che tale programma è in grado di fornire.

\subsection{Generazione dei raggi cosmici}
Come prima cosa è necessario che il programma simuli dei raggi cosmici che siano realistici. Per fare questo si è utilizzata la nota distribuzione dei raggi cosmici:
\begin{equation}
	P\left(\theta, \varphi\right) = \cos^2{\theta}
	\label{eq:distr_cosmici}
\end{equation}
dove $\theta$ e $\varphi$ sono la direzione del raggio cosmico in coordinate sferiche.

Simulare l'energia \`e stato invece pi\`u complicato: raramente infatti le pubblicazioni trattano l'energia dei raggi cosmici a energie sotto \SI{1}{\GeV}. Purtroppo vedremo in seguito che l'assorbitore \`e in grado di fermare i muoni solo fino a energie di \SI{\sim 60}{\MeV}. Seguendo le indicazioni del PDG \cite{bib:Patrignani:2016xqp} (distribuzione piatta fino a \SI{\sim 1}{\GeV}, decrescente fino a \SI{\sim 10}{\GeV} e proporzionale a E$^{-2.7}$ dopo) e il Grafico \ref{gr:desc@VerticalMuonFluxSeaLevelPortella-Oliveira-etc} abbiamo ipotizzato una distribuzione di probabilit\`a proporzionale a 
\begin{equation}
	P\left(E\right) \propto \frac{1}{50+E^{2.7}}
	\label{eq:distr_cosmici_en}
\end{equation}
che soddisfa le informazioni che abbiamo. L'arbitrariet\`a di questa distribuzione per\`o non ci consente uno studio della rate che vada oltre l'ordine di grandezza.
Data la presenza del cemento sovrastante \`e necessario aggiustare la distribuzione dell'energia dei muoni prima di farli interagire con l'apparato. 
Si \`e perci\`o calcolata l'energia minima necessaria a passare lo spessore di cemento considerato (\SI{\sim 60}{\cm}), pi\`u i due rivelatori posti al di sopra dell'assorbitore, ed \`e risultata essere di \SI{0.31}{\GeV}. 
Una semplice traslazione non \`e per\`o sufficiente, in quanto la relazione range/energia non \`e lineare. 
Per ottenere una distribuzione sensata \`e stata fatta un'ulteriore simulazione, contenente un assorbitore in cemento e uno scintillatore di poliviniltoluene (la base polimerica degli scintillatori usati). 
La simulazione ha mostrato un minimo di probabilit\`a a energia 0, di cui \`e stato fatto un fit che poi \`e stato sovrapposto alla distribuzione traslata per ottenere la distribuzione finale.

Non vengono considerate correlazioni tra la direzione del muone e la sua energia.

Per ogni evento un raggio cosmico viene generato in un punto casuale del piano lungo \SI{700}{\mm} lungo $x$ e \SI{350}{\mm} lungo $y$, con $z$ pari alla superficie superiore del primo rivelatore e la direzione di tale raggio cosmico è data dall'Equazione \ref{eq:distr_cosmici}: questo \`e perch\'e un raggio cosmico pi\`u lontano dal bordo del rivelatore dovrebbe essere molto inclinato per poter interagire con l'apparato e a questo punto non riuscirebbe ad interagire con i rivelatori pi\`u bassi.

La carica del muone viene generata a partire dalle informazioni presenti nel PDG (\cite{bib:Patrignani:2016xqp}), che indicano come il rapporto tra le due cariche sia di circa 1.25/1.3 a favore dei $\mu^+$.

La polazzazione viene generata analogamente al \SI{20}{\percent}.

%Per non rendere la simulazione troppo pesante si \`e quindi tagliata la distribuzione della posizione del passaggio del cosmico, a priori uniforme, ad una distanza giusta per non compromettere il risultato.
I raggi cosmici vengono fatti evolvere nel limite ultrarelativistico: data la loro alta velocità si può considerare che essi seguano una traiettoria rettilinea nonostante
le forze esterne (in particolare rilevante in aria è l'effetto del campo magnetico che tenderebbe a deviare la traiettoria), e si muovano a velocità infinita (non disponiamo comunque della risoluzione temporale necessaria a rilevare differenze di tempo dell'ordine della frazione di nanosecondo come servirebbe in questo caso).


\subsection{Interazione con i rivelatori}
Quando viene fatta evolvere la traccia essa potrebbe entrare all'interno dei rivelatori, che possono rivelare tale passaggio. 

Nel modellizzare tale evento si considera l'interazione come un fenomeno esattamente deterministico e non stocastico come realmente è. 
Per poter comprendere cosa succede quando una traccia attraversa un rivelatore si utilizzano i dati sperimentali, ottenuti come descritto alla Sezione \ref{sec:efficiency}: data la moda della distribuzione sperimentale dei fotoni, essa si interpreta come il numero di fotoni generati dal passaggio di un muone cosmico quando attraversa lo spessore (noto) del rivelatore in direzione perpendicolare alla faccia del rivelatore stesso (ovvero la moda della distribuzione dell'inclinazione). 
In questo modo si può andare a stimare effettivamente quanti fotoni vengono generati per ogni mm di scintillatore attraversato dal muone (si noti che si stanno trascurando parecchi fattori, come per esempio il diverso assorbimento in diversi punti dell'assorbitore al variare della distanza dalla fibra ottica all'interno dello scintillatore, o la perdita di energia del muone a causa della scintillazione). 
Così, usando delle identità trigonometriche, è stato possibile trovare lo spazio percorso dal muone all'interno del rivelatore e, noto quest'ultimo, è stato possibile trovare il numero di fotoni che ci si aspetta arrivino ai canali di acquisizione. 
L'effetto del passaggio degli elettroni \`e stato considerato perfettamente equivalente a quello dei muoni, in assenza di dati.

Il segnale generato \`e usato come confronto con un arbitraria "soglia", a simulare il modulo di coincidenza (la soglia \`e stata impostata a 3.5 fotoni, $\sim$ 50~mV, come impostato nel reale modulo di coincidenza).

Per i muoni si \`e considerata una probabilit\`a trascurabile di fermarsi all'interno del rivelatore. Questo non rispecchia perfettamente la realt\`a, ma \`e stato considerato che questa probabilit\`a \`e bassa e  non porterebbe a eventi rilevati da Arietta, in quanto gli elettroni difficilmente produrrebbero il segnale adatto. \`E per\`o stato tenuto conto dell'attenuazione in energia durante il passaggio dei muoni.

Diversa \`e invece la situazione degli elettroni: si \`e infatti visto sperimentalmente che gli elettroni emessi dal decadimento del muone hanno una probabilit\`a non trascurabile di decadere nella plastica dello scintillatore. Per queste particelle si \`e quindi simulato un range, in funzione dell'energia di uscita dall'assorbitore, basandosi su un grafico dal datasheet del rivelatore (\cite{bib:SiPM}), che mostra il range degli elettroni circa proporzionale alla loro energia, con formula

\begin{equation}
	R(E) \approx 40 \frac{mm}{MeV}\cdot E
	\label{eq:range_elec}
\end{equation}

Da prove con la simulazione si vede che la dipendenza del rate calcolato da questa formula \`e bassa, minore di altre incertezze intrinseche della simulazione (come la distribuzione dell'energia dei muoni).


\subsection{Interazione con l'assorbitore}
L'interazione con l'assorbitore viene modellizzata completamente grazie ai dati tabulati di ESTAR del NIST (elettroni) e della sezione AtomicNuclearProperties del PDG (muoni). In queste tabelle infatti sono presenti sia i dati di range che i dati dello stopping power, entrambi in funzione dell'energia della particella.

L'interazione di una particella nell'assorbitore funziona perci\`o cos\`i: per prima cosa viene calcolato il range basandosi sull'energia di entrata. 
Se viene rilevato che la particella ha abbastanza energia per uscire \`e necessario calcolare l'energia di uscita\footnote{Nota: data l'assunzione mostrata al punto precendente secondo quale i muoni non possono fermarsi negli scintillatori, l'energia di uscita dei muoni non \`e in realt\`a utile e non viene quindi calcolata}. 
Viene quindi applicato l'algoritmo di Eulero per avere una decente stima dell'energia finale.

Data la non linearit\`a del range dei muoni in funzione dell'energia , la maggior parte dei muoni tenderebbe a fermarsi nei primi millimetri di assorbitore. Questo per\`o \`e contrastato dal minimo di energia dovuto al passagio nel cemento, che quindi rende la distribuzione piatta. Questo \`e da aspettarsi, in quanto non ha senso fisico che i muoni abbiano un picco di decadimento all'inizio dell'oggetto. Questa caratteristica \`e stata presa come vincolo, fittando la distribuzione del decadimento dei muoni nel cemento in modo che la distribuzione nell'assorbitore fosse pressoch\'e uniforme.


\subsection{Implementazione del campo magnetico}
Il campo magnetico dell'esperimento è generato da un solenoide finito a sezione rettangolare. 
Si è voluto andare ad utilizzare un campo che non fosse uniforme all'interno del solenoide, e per farlo si è risolto numericamente tale problema. 
Si è introdotta la corrente come pareti di corrente uniforme e costante, si è discretizzato lo spazio e si è utilizzato l'algoritmo di Jacobi per ottenere il potenziale vettore data la densità di corrente introdotta. 
Poi si è calcolato numericamente il rotore per andare a trovare effettivamente il campo magnetico. 
L'algoritmo è stato fatto girare su uno spazio più grande (circa un fattore 5 in y e z, mentre circa 2 in x) del solenoide, in modo che siano fisicamente sensate le condizioni al contorno assorbenti ai bordi del sistema e su solo un ottante dello spazio, impostando condizioni al contorno riflettenti o antiriflettenti (per conservare la simmetria del sistema) nei piani che separano i settori. Da questo calcolo si è trovato come effettivamente il campo magnetico non sia costante all'interno del solenoide ma abbia una dipendenza dalla posizione, come si può vedere nei grafici di Figure \ref{gr:campo_b@xy} e \ref{gr:campo_b@zy} dove si possono vedere le proiezioni sul piano y = 0 delle componenti x e z del campo magnetico. 
Questo processo di risoluzione numerica dell'equazione differenziale ha permesso di avere dei valori per il campo magnetico più realistici che dipendano dalla posizione presa in considerazione: in particolare si \`e stimato il modulo quadro del campo magnetico in corrispondenza del centro del solenoide e del centro della faccia dello scintillatore perpendicolare all'asse del soleoide stesso (il punto di massima differenza, come si pu\`o notare dai grafici), notando una variazione di circa il \SI{3.2}{\percent}. 

Questa simulazione non \`e per\`o stata utile in quanto nell'esperimento vero e proprio il campo magnetico non \`e stato disponibile.

\inputgraph{campo_b}

\subsection{Decadimento del muone cosmico}
Se il muone cosmico viene fermato all'interno dell'assorbitore, esso decadrà dopo un tempo che dipende dal tipo di muone che si ferma (muone o antimuone), che è distribuito
come un esponenziale dal tempo caratteristico che viene assunto come noto dalla letteratura. Perciò si considera il muone, fermo, trascorra un tempo che in media è il tempo
di vita di tale muone, e poi decada emettendo sostanzialmente un elettrone (non sono rilevabili gli altri prodotti del decadimento). La direzione dell'elettrone \`e simulabile a partire dalle caratteristiche del muone. Infatti, come detto in precedenza, la direzione dell'elettrone \`e correlata alla direzione dello spin del muone al momento del decadimento. Per ogni interazione perci\`o \`e stata considerata l'elicit\`a del muone: forward o backward. In entrambi i casi si \`e fatto ruotare lo spin attorno al campo magnetico nel punto di decadimento, per un tempo pari al tempo di decadimento simulato. La direzione dell'elettrone viene poi generata a partire dalla direzione dello spin, usando la formula 
\begin{equation}
	P(\theta, x) \propto \left(3 - 2 x - \cos{\theta}\left(1 - 2 x\right)\right)
\end{equation}
dove x \`e la frazione di semimassa del muone disponibile per l'elettrone, e $\theta$ \`e l'angolo tra lo spin del muone e la direzione dell'elettrone (\cite{bib:mudecay}). Questo ovviamente coincide con l'usare una descrizione classica e non quantistica dello spin, in cui a ruotare dovrebbero essere solo i valori medi delle tre osservabili quantistiche producendo un analogo risultato.

\subsection{Output della simulazione}
Gli output utili della simulazione descritta ai punti precedenti sono numerosi e interessanti:
\begin{itemize}
\item Efficienza: considerando solamente l'interazione tra i raggi cosmici e gli scintillatori è possibile stimare quanti muoni non risultano in coincidenza a causa
di condizioni geometriche non favorevoli, per esempio quelli che lasciano segnale attraversando ai bordi i due rivelatori superiori ma non entrano nemmeno nel terzo
rivelatore.
\item Informazioni varie: dalla simulazione, facendo istogrammi di quantit\`a intermedie, \`e possibile ricavare informazioni sul comportamento delle particelle. Un esempio di ci\`o \`e la distribuzione della morte degli elettroni.
\item Spettro temporale: data tutta la simulazione fatta, è possibile fare un plot del tempo che intercorre tra il passaggio del muone e quello dell'elettrone, riottenendo
l'esponenziale del tempo di decadimento, corretto con la rotazione dello spin, ovvero il grafico da cui si potrebbe calcolare il fattore di Land\'e. Questo ci permette di calcolare l'effetto di un campo magnetico non uniforme sulla forma del segnale.
\item Confronto del rate con quella sperimentale: \`e possibile confrontare il rate calcolata dalla simulazione con quella ottenuta durante la misura della vita media dei muoni. Questo ci permette di verficare che non ci siano inefficienze evidenti che non abbiamo considerato che potrebbero distorcere il risultato.
\item Stima delle coincidenze casuali attese durante l'esperimento: data la struttura dell'esperimento \`e possibile che due muoni arrivino abbastanza vicini e triggerino l'apparato in rapida successione. Questo porterebbe a un segnale falso, in quanto non abbiamo modo di distinguere il segnale di un elettrone da quello di un $\mu$.
\end{itemize}

\inputgraph{ball_1e9}
\subsubsection{Spettro temporale in campo magnetico}
Il grafico in Figura \ref{gr:ball_1e9} mostra il risultato di una simulazione con \SI{1e9} eventi nel caso in cui B sia considerato uniformemente uguale a \SI{55}{\gauss} e nel caso in cui il campo magnetico viene simulato. 
Si pu\`o vedere dal grafico come il campo magnetico influisca poco sui $\mu^-$, mentre produce l'effetto desiderato sui $\mu^+$. 
Da un fit sull'istogramma con la funzione

\begin{equation}
	\frac{\dd n_{e^+}}{\dd t} = \frac{N}{\tau_+}e^{\frac{x}{\tau_+}}\left[1 + \alpha\cdot\cos(\omega_L x + \varphi)\right]
	\label{eq:sim_fit}
\end{equation}

si ottiene un valore per $\alpha$ di \SI{9.1 \pm 0.3}{\percent}, mentre si ottiene \SI{3.10 \pm 0.08}{\radian} per $\varphi$. Quest'ultimo valore \`e da aspettarsi, in quanto gli elettroni hanno probabilit\`a massima si essere emessi in direzione opposta a quella dello spin, rendendo il valore atteso $\pi$.

Dal grafico si pu\`o anche notare come le due distribuzioni siano pressoch\'e identiche fino ad alti t, ed anche l\`a la discrepanza pu\`o essere assimilata ad errori statistici. Non \`e chiaro se questa uguaglianza \`e reale o dovuta a una statistica insufficiente.

\subsubsection{Informazioni varie}
Chiedendo durante la simulazione di stampare la posizione della morte degli elettroni \`e possibile costruire l'istogramma di grafico in Figura \ref{gr:edeath}.

Facendo l'integrale degli elettroni morti prima di poter dare un segnale valido (ovvero nel primo scintillatore di coincidenza e il \SI{43}{\percent} del secondo), su una simulazione da \num{1e7} eventi, si vede che \SI{0.022}{\percent} degli eventi simulati ha prodotto un elettrone che \`e morto dentro gli scintillatori.
Considerando la percentuale di eventi validi pari \SI{0.055}{\percent} si vede che il \SI{\sim 28}{\percent} degli eventi potenzialmente validi viene fermato dai rivelatori prima di dare un segnale.

Appare evidente che una consistente percentuale di elettroni muoia dentro gli scintillatori e che quindi l'aggiunta di ulteriori rivelatori in coincidenza, oltre al voluto effetto di ridurre il rumore, tende ad abbassare drammaticamente il rate di coincidenze valide (anche di ordini di grandezza, come evidenziato in una prima presa dati in cui questo effetto era stato trascurato). Questo ha portato ad una rimozione di rivelatori dall'esperimento, ottendendo infine la configurazione definitiva con soli due rivelatori di coincidenza e uno di veto.

\inputgraph{edeath}

\subsubsection{Stima della correzione geometrica alla stima dell'efficienza}
\`E interessante anche analizzare quanto il fattore geometrico contribuisce all'efficienza diversa dal \SI{100}{\percent} calcolata nella Sezione \ref{sec:eff_corr}. Ignorando gli elettroni infatti si pu\`o calcolare l'efficienza con una configurazione del tutto simile a quella utilizzata nella reale misura, ma impostando questa volta l'efficienza intrinseca a 1.

\subsubsection{Stima del rate}
In una sezione precedente abbiamo calcolato il rate di eventi validi a partire dalle nostre misure della vita media dei muoni. 
\`E possibile fare una stima basandosi sui risultati della simulazione: data infatti la frazione di eventi simulati che vengono rilevati come validi, uniti ai dati sperimentali sui muoni cosmici presi dal PDG (\cite{bib:Patrignani:2016xqp}), si pu\`o associare un valore per il rate di eventi.

L'output della simulazione usata \`e mostrato nel grafico in Figura \ref{gr:sim_30G}, dove sono stati simulati \SI{3e10}{muoni}.
\inputgraph{sim_30G}

La frequenza media di muoni integrata su tutti gli angoli \`e \SI{130}{muoni\per\square\metre\second}. La superficie che noi consideriamo \`e di \SI{1.40 x 0.70}{\m}. 
La frazione di muoni che passa la schermatura in cemento dell'edificio viene presa dalla simulazione come \SI{0.92}{\percent}.

Data un frazione di eventi validi di \num{5.47e-4}, si arriva a una rate stimata di \SI{0.0162}{\Hz}. 
Questo numero va confrontato con il rate sperimentale di \SI{0.0214 +- 0.0004}{\Hz}.

I due numeri non sono compatibili ma bisogna ricordare il grande numero di assunzioni imprecise fatte per questa simulazione. 
\`E possibile notare per\`o che i numeri corrispondono almeno come ordine di grandezza segno che non ci sono grandi inefficienze che non abbiamo considerato. 
Ad esempio, se non avessimo considerato la morte degli elettroni dentro gli scintillatori, come avevamo inizialmente fatto, gli eventi sarebbero decisamente di pi\`u, soprattutto nella configurazione a 6 scintillatori usata nel semestre scorso.

\subsubsection{Coincidenze casuali}
Guardando dalla simunlazione del punto precedente la percentuale di muoni che non ha prodotto segnale nonstante il trigger \`e possibile stimare il rate attesa di coincidenze casuali dovute a due $\mu$ in rapida successione.
Questo rate \`e infatti dato da

\begin{equation}
	R_{fake} = \left(R_\mu\cdot S\cdot\sigma_{cemento}\cdot\varepsilon_{apparato}\right)^2\cdot\Delta t
	\label{eq:fakes}
\end{equation}

dove $R_\mu$ \`e il rate di muoni cosmici, S \`e la superficie di produzione della simulazione, $\sigma_{cemento}$ \`e la frazione di muoni che passa il cemento, $\varepsilon_{apparato}$ \`e la frazione di muoni che si ferma nell'apparato ma non da segnale con l'elettrone e $\Delta t$ \`e la finestra temporale su cui Arietta lavora.

Sostituendo i valori ottenuti si ottiene una rate pari a \SI{4.16e-4}{\Hz}, trascurabile rispetto all rate di \SI{2.14 +- 0.04e-2}{\Hz} dei dati.
