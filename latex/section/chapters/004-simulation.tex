\section{Simulazioni}
Per poter lavorare al meglio con l'apparato strumentale dato e comprendere al meglio il meccanismo di funzionamento si è deciso di scrivere un programma in grado di simulare
il processo che avviene durante le vere e proprie sessioni di misura. In questa sezione si vuole descrivere brevemente il funzionamento di tale programma di simulazione e
l'output che tale programma è in grado di fornire.
\subsection{Generazione dei raggi cosmici}
Come prima cosa è necessario che il programma simuli dei raggi cosmici che siano realistici. Per fare questo si è utilizzata la nota distribuzione dei raggi cosmici:
\begin{equation}
	P\left(\theta, \varphi\right) = \cos^2{\theta}
	\label{eq:distr_cosmici}
\end{equation}
dove $\theta$ e $\varphi$ sono la direzione del raggio cosmico in coordinate sferiche.
Per ogni evento un raggio cosmico viene generato in un punto casuale del piano lungo 700 mm lungo $x$ e 350 mm lungo $y$, con $z$ pari alla superficie superiore del primo rivelatore e la direzione di tale raggio cosmico è data dall'Equazione \ref{eq:distar_cosmici}: questo \`e perché\'e un raggio cosmico pi\`u lontano dal bordo del rivelatore dovrebbe essere molto inclinato per poter interagire con l'apparato e a questo punto non riuscirebbe ad interagire con i rivelatori pi\`u bassi. Per non rendere la simulazione troppo pesante si \`e quindi tagliata la distribuzione della posizione del passaggio del cosmico, a priori uniforme, ad una distanza giusta per non compromettere il risultato.
Tali raggi cosmici vengono fatti evolvere nel limite ultrarelativistico: data la loro alta velocità si può considerare che essi seguano una traiettoria rettilinea nonostante
le forze esterne (in particolare rilevante in aria è l'effetto del campo magnetico che tenderebbe a deviare la traiettoria), e si muovano a velocità infinita (non disponiamo comunque della risoluzione temporale necessaria a rilevare differenze di tempo dell'ordine della frazione di nanosecondo come servirebbe in questo caso).

\subsection{Interazione con i rivelatori}
Quando viene fatto evolvere un muone esso potrebbe entrare all'interno dei rivelatori, che possono rivelare tale passaggio. Per modellizzare tale evento, come prima cosa,
si considera che il rivelatore non influenza il raggio cosmico (il muone non può quindi essere assorbito all'interno del rivelatore), inoltre
si modellizza l'interazione come un fenomeno esattamente deterministico e non stocastico come realmente è. Per poter comprendere cosa succede quando un muone attraversa
un rivelatore si utilizzano i dati sperimentali, ottenuti come descritto alla Sezione \ref{sec:efficiency}: data la moda della distribuzione
sperimentale dei fotoni, essa si interpreta come il numero di fotoni generati dal passaggio di un muone cosmico quando attraversa lo spessore (noto) del rivelatore
in direzione perpendicolare alla faccia del rivelatore stesso (ovvero la moda della distribuzione dell'inclinazione). In questo modo si può andare a stimare effettivamente quanti fotoni vengono generati per ogni mm di scintillatore attraversato dal muone (si noti che si stanno trascurando parecchi fattori, come per esempio il diverso assorbimento in diversi punti dell'assorbitore
al variare della distanza dalla fibra ottica all'interno dello scintillatore, o la perdita di energia del muone a causa della scintillazione). Così, usando delle identità trigonometriche, è stato possibile trovare lo spazio percorso dal muone all'interno del rivelatore e, noto quest'ultimo, è stato possibile trovare il numero di fotoni che ci si aspetta arrivino ai canali di acquisizione. Il segnale generato \`e usato come confronto con un arbitraria "soglia", a simulare il modulo di coincidenza (la soglia \`e stata impostata a 3.5 fotoni, $\sim$ 50~mV, come impostato nel reale modulo di coincidenza).

\subsection{Interazione con l'assorbitore}
Per quanto riguarda l'interazione con l'assorbitore, diversamente a quanto fatto per l'interazione con i rivelatori, si considera il processo come stocastico. Un muone
ha una probabilità di interagire con il materiale che dipende dal materiale stesso e dall'energia del muone. Tale relazione è stata semplificata, e si è considerato che il
muone si ferma in una distanza generata uniformemente in una regione spaziale ben definita\footnote{tale regione è stata impostata computazionalmente in modo che non si generino troppi
dati inutili, quindi a meno di un coefficiente moltiplicativo stimabile}, pari a 30~mm: questo nell'approssimazione che la perdita di energia sia definita esattamente dalla Bethe-Bloch e che la distribuzione dell'energia dei muoni sia uniforme da 0 ad un energia massima (si sta tagliando la distribuzione dell'energia dei muoni, ma da \cite{bib:Patrignani:2016xqp} si vede che la distribuzione \`e effettivamente uniforme fino a $\sim $800~MeV, e i muoni con energia sopra i 115 MeV difficilmente si fermerebbero). La posizione di interazione \`e importante in quanto da essa viene generato l'elettrone del decadimento, che deve comunque interagire con il sistema.

\subsection{Implementazione del campo magnetico}
Il campo magnetico dell'esperimento è generato da un solenoide finito a sezione rettangolare. Si è voluto andare ad utilizzare un campo che non fosse uniforme all'interno
del solenoide, e per farlo si è risolto numericamente tale problema. Si è introdotta la corrente come pareti di corrente uniforme e costante, si è discretizzato lo spazio
e si è utilizzato l'algoritmo di Jacobi per ottenere il potenziale vettore data la densità di corrente introdotta. Poi si è calcolato numericamente il rotore per
andare a trovare effettivamente il campo magnetico. L'algoritmo è stato fatto girare su uno spazio più grande (circa un fattore 5 in y e z, mentre circa 2 in x) del solenoide, in modo che siano
fisicamente sensate le condizioni al contorno assorbenti ai bordi del sistema e su solo un ottante dello spazio, impostando condizioni al contorno riflettenti o antiriflettenti (per conservare la simmetria del sistema) nei piani che separano i settori. Da questo calcolo si è trovato come effettivamente il campo magnetico non sia costante
all'interno del solenoide ma abbia una dipendenza dalla posizione, come si può vedere nelle Immagini \ref{gr:campo_b@xy} e \ref{gr:campo_b@zy} dove si possono vedere le proiezioni sul piano y = 0 delle componenti x e z del campo magnetico. Questo processo di risoluzione numerica dell'equazione differenziale ha permesso di avere dei valori per il campo magnetico più realistici che dipendano dalla posizione presa in considerazione: in particolare si \`e stimato il modulo quadro del campo magnetico in corrispondenza del centro del solenoide e del centro della faccia dello scintillatore perpendicolare all'asse del soleoide stesso (il punto di massima differenza, come si pu\`o notare dai grafici), notando una variazione di circa il 3.2\%. 
\inputgraph{campo_b}

\subsection{Decadimento del muone cosmico}
Se il muone cosmico viene fermato all'interno dell'assorbitore, esso decadrà dopo un tempo che dipende dal tipo di muone che si ferma (muone o antimuone), che è distribuito
come un esponenziale dal tempo caratteristico che viene assunto come noto dalla letteratura. Perciò si considera il muone, fermo, trascorra un tempo che in media è il tempo
di vita di tale muone, e poi decada emettendo sostanzialmente un elettrone (non sono rilevabili gli altri prodotti del decadimento). La direzione dell'elettrone \`e simulabile a partire dalle caratteristiche del muone. Infatti, come detto in precedenza, la direzione dell'elettrone \`e correlata alla direzione dello spin del muone al momento del decadimento. Per ogni interazione perci\`o \`e stata considerata l'elicit\`a del muone, forward, backward.In entrambi i casi si \`e fatto ruotare lo spin attorno al campo magnetico nel punto di decadimento, per un tempo pari al tempo di decadimento simulato. La direzione dell'elettrone viene poi generata a partire dalla direzione dello spin, usando la formula 
\begin{equation}
	P(\theta) = 1+a\cdot\cos{\theta}
\end{equation}
dove a \`e un coefficiente sperimentalmente determinato intorno a 1/3 (preso da \cite{bib:AJP-Amsler}). Questo ovviamente coincide con l'usare una descrizione classica e non quantistica dello spin, in cui a ruotare dovrebbero essere solo i valori medi delle tre osservabili quantistiche producendo un analogo risultato.

\subsection{Implementazione dell'elettrone}
Una volta che il programma di simulazione genera un elettrone con la sua posizione e la sua direzione, tale elettrone viene fatto evolvere allo stesso modo del muone, e si considera se esso viene riassorbito all'interno dell'assorbitore e in quali scintillatori lascia segnali, e quanto intensi sono tali segnali. In fase preliminare si \`e usato per l'interazione degli elettroni con l'assorbitore lo stesso range massimo usato per i muoni. Il range massimo effettivo sarebbe in realt\`a leggermente minore, sebbene dello stesso ordine di grandezza, a giustificare la scelta fatta. Tale approssimazione sar\`a rifinita in futuro durante il prossimo semestre.

\subsection{Output della simulazione}
Gli output utili della simulazione descritta ai punti precedenti sono numerosi e interessanti:
\begin{itemize}
\item Efficienza: considerando solamente l'interazione tra i raggi cosmici e gli scintillatori è possibile stimare quanti muoni non risultano in coincidenza a causa
di condizioni geometriche non favorevoli, per esempio quelli che lasciano segnale attraversando ai bordi i due rivelatori superiori ma non entrano nemmeno nel terzo
rivelatore.
\item Spettro temporale: data tutta la simulazione fatta, è possibile fare un plot del tempo che intercorre tra il passaggio del muone e quello dell'elettrone, riottenendo
l'esponenziale del tempo di decadimento, corretto con la rotazione dello spin, ovvero il grafico da cui si potrebbe calcolare il fattore di Land\'e. Questo ci permette di calcolare l'effetto di un campo magnetico non uniforme sulla forma del segnale.
\item Stima del tempo necessario per l'esperimento: sapendo il numero di eventi generati, il numero di segnali che hanno portato ad un trigger valido (in cui il muone ha interagito e l'elettrone \`e stato rilevato) e la frequenza media di arrivo di un muone cosmico si può dare una stima del tempo necessario per arrivare ad una certa statistica. Questo conto era gi\`a stato fatto l'anno precedente per la scelta del materiale e spessore dell'assorbitore, ma \`e stato comunque ripetuto per confronto.
\end{itemize}

\inputgraph{sim_fake_b}

\inputgraph{sim_sim_b}

\inputgraph{sim_both}

L'Immagine \ref{gr:sim_fake_b} mostra il risultato di una simulazione con $5\cdot 10^8$ eventi nel caso in cui B sia considerato uniformemente uguale a 55~G. Nel grafico si possono notare due serie di picchi lungo l'esponenziale, alternanti, una più alta (ad esempio in t = 1300~ns) e una più bassa (ad esempio in t = 2000~ns). Ci\`o \`e dovuto al fatto che i $\mu^+$ sono polarizzati all'indietro a causa del decadimento del pione, mentre i $\mu^-$ in avanti. Un muone con elicit\`a negativa avra probabilit\`a massima di produrre un elettrone rilevabile quando la fase della rotazione sar\`a un multiplo di $2\pi$, ovvero nella serie di picchi pari, mentre uno con elicit\`a positiva l'avr\`a in quelli dispari, quando la fase \`e uguale a $\pi+2k\pi$. Dato il maggior numero di muoni positivi nella regione temporale pi\`u significativa (dopo i 1000~ns) e la polarizzazione prevalentemente negativa di questi ultimi, si avr\`a l'ampiezza massima nella serie pari. Il fatto che si vedano due serie alternanti sono anche dovute alla non perfetta sinusoide descritta dall'istogramma: due sinusoidi in opposizione di fase, come ci si aspetterebbe dalla teoria, infatti andrebbero a sommarsi perfettamente in una sinusoide di ampiezza minore. Nel nostro sistema per\`o la parte sensibile al passaggio degli elettroni non occupa l'intero semispazio superiore, bens\`i un rettangolo che sottende un angolo solido $<2\pi$ sr. A causa di ci\`o l'intervallo di fase per cui gli elettroni vengono rilevati \`e minore di $\pi$, e la parte positiva dell'oscillazione ha una larghezza minore, impedendo che le due curve si sovrappongano esattamente.\\
Questo, se verr\`a notato anche nell'esperimento reale, significa che dovranno essere prese precauzioni particolari durante il fit, per evitare di introdurre errori sistematici usando una funzione che non descrive esattamente il modello. 

Nel grafico successivo, Immagine \ref{gr:sim_sim_b}, vediamo la simulazione principale, quella in cui B \`e stato preso dalla simulazione per un solenoide finito e rettangolare. I due grafici risultano molto simili, in quanto l'assorbitore \`e comunque posto al centro del solenoide, dove il campo magnetico \`e il pi\`u uniforme possibile (dalla simulazione si pu\`o notare come il campo magnetico sia pressoch\'e uniforme fino ai $\sim$350~mm, 150 grid points nelle immagini \ref{gr:campo_b@xy} e \ref{gr:campo_b@zy}). Si pu\`o vedere per\`o nell'Immagine \ref{gr:sim_both} come in realt\`a la non uniformit\`a del campo causi una leggera diminuzione dell'ampiezza dell'oscillazione.

\subsubsection{Stima della correzione geometrica alla stima dell'efficienza}
\`E interessante anche analizzare quanto il fattore geometrico contribuisce all'efficienza diversa dal 100\% calcolata nella Sezione \ref{sec:eff_corr}. Ignorando gli elettroni infatti si pu\`o calcolare l'efficienza con una configurazione del tutto simile a quella utilizzata nella reale misura, ma impostando questa volta l'efficienza intrinseca a 1. L'efficienza calcolata risulta essere 97.1 $\pm$~0.1\%.
