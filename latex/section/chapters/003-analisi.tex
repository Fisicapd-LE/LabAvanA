\section{Analisi dati}
Dati i numerosi problemi tecnici che si sono incontrati effettuando l'esperimento, non è stato possibile procedere alla creazione e caratterizzazione del campo magnetico, e naturalmente nemmeno alla misura del fattore di Landé, obiettivo finale dell'esperienza. Ciò che si è fatto è una caratterizzazione completa del sistema di acquisizione (scintillatori e schede di acquisizione) accompagnata da una misura sul tempo di vita media e rate medio dei muoni. Inoltre si è sviluppata una simulazione per permettere la comprensione e le correzioni da effettuare sulle misure fatte.

\subsection{Caratterizzazione dei diodi}
I SiPM sono stati caratterizzati in due modi. Si è prima studiata la caratteristica tensione corrente (solo per il primo modello della scheda, in quanto nelle versioni successive l'ingresso del bias, su cui \`e necessario agire per la misura, non \`e pi\`u in un cavo indipendente, ma unito agli altri e quindi di difficile accesso). Trovando la curva, \`e possibile vedere a che tensione si innesca il meccanismo del breakdown, ovvero a che tensione il diodo diventa operativo come rivelatore.
In seguito si è studiata la variazione del parametro di guadagno per ogni fotone al variare del voltaggio di bias per ogni diodo utilizzato. Il guadagno infatti, grazie alla resistenza di quenching, si prevede essere proporzionale alla sovratensione rispetto al breakdown. Estrapolando i dati fino a guadagno 0, si pu\`o ottenere un'altra stima della tensione di breakdown, che non ha bisogno stavolta dell'accesso all'ingresso di bias. Questo metodo è perciò stato usato per tutti i rivelatori caratterizzati.

\subsubsection{Studio della caratteristica tensione corrente}
L'unica scheda di acquisizione che si è potuta utilizzare per lo studio della caratteristica tensione corrente è quella che poi è andata nello scintillatore che nelle prossime sezioni verrà chiamato D. Per studiare tale caratteristica si è utilizzato un picoamperometro collegato al diodo: esso permette, in maniera simile a quanto viene fatto dai multimetri commerciali
impostati come ohmetri, di fornire una ben definita tensione e di misurare la corrente che attraversa l'oggetto generata da questa tensione. Quindi,
non si è fatto altro che mettere la PCB al buio (in modo da non rilevare una quantità troppo elevata di fotoni esterni quando si dà una tensione di bias al diodo),
collegare il picoamperometro al diodo e studiare come varia la corrente al variare della tensione fornita, ottenendo la curva caratteristica del diodo nelle sue due
sezioni più interessanti: quella dello spegnimento e quella del breakdown; quest'ultima risulta particolarmente interessante in quanto è in questa regione che funzioneranno
i diodi una volta collegato tutto l'esperimento. Le curve di caratterizzazione si possono vedere nei grafici in Figura~\ref{gr:picoamp} per i due  diodi nella schedina utilizzata (è stata utilizzata la schedina dello scintillatore D).
\inputgraph{picoamp}

In tale grafico si può vedere la corrente al variare della tensione di bias in scala logaritmica: infatti si prevede che nella fase  di breakdown si abbia un aumento esponenziale della corrente al variare della tensione, mentre nella fase di spegnimento in teoria la corrente non varia in funzione della tensione; si \`e perci\`o fittata la parte di breakdown con una retta. La parte di spegnimento \`e stata sempre fittata con una retta in quanto a noi interessa solo l'incrocio tra le due curve e i dati sembravano avere quella forma, probabilmente a causa di latenze dei condensatori elettrolitici presenti nel sistema di lettura e amplificazione del segnale.
 Si noti dal grafico che per fittare la sezione del breakdown si è fatto un fit solo con i primi dati dopo il breakdown: questo perché poi iniziano a essere rilevanti altri fenomeni che vanno a piegare la curva tensione-corrente, e perciò a rovinare la stima del voltaggio di breakdown.
Dall'intersezione delle rette di fit si trova il voltaggio di breakdown dei due diodi, che si possono vedere nella Tabella \ref{tab:breakdown_picoamp}.
\inputtab{breakdown_picoamp}
Si noti che questa misura è principalmente didattica, e non vuole dare un'effettiva utile stima del voltaggio di breakdown per i due diodi ma solamente un'ordine di grandezza
(infatti uno studio più accurato verrà fatto per questi e i successivi diodi nelle sezioni successive), perciò a questa misura non si associa incertezza.

\subsubsection{Studio dell'amplificazione dei diodi}
Molto importante per la regolazione del voltaggio di bias per i singoli diodi è sapere esattamente il voltaggio di breakdown di tali diodi e a quale variazione di
voltaggio sia associato l'assorbimento di un fotone da parte di un diodo. Per fare questo si è alimentato l'operazionale nella scheda contenente il diodo, tale
scheda è stata messa al buio, e si è collegato il bias del diodo al generatore di tensione, e l'output all'oscilloscopio. Quindi, si è fatta variare la tensione
di bias del diodo e si sono raccolti un numero fisso di dati. Per raccogliere tali dati è stato necessario impostare un trigger in modo che non si salvasse il rumore elettronico legato al circuito e all'oscilloscopio, ma solo le valanghe all'interno del SiPM: tale trigger è stato impostato a \SI{5}{\mV}.
Il grafico in Figura \ref{gr:6lemo_d1_gain_305} è uno dei tanti grafici che sono stati
\inputgraph{6lemo_d1_gain_305}
trovati per studiare l'amplificazione dei diodi (gli altri non si presentano per leggibilità della relazione). Da questo grafico è evidente come ci sono diverse gaussiane ben distinte, ad indicare che si vede il voltaggio
generato da un numero crescente di fotoni (infatti la gaussiana a voltaggio più basso sarà quella legata a un fotone, quella alla sua sinistra due fotoni eccetera).
I segnali rivelati sono dovuti sia ''fotoni termici``, cioè eccitazioni casuali nel semiconduttore che forma il diodo che vengono lette dal sistema come se fosse stato assorbito un fotone
da tale diodo, sia fotoni residui che sono riusciti a passare attraverso la schermatura. Grafici di questo tipo sono stati interpolati al variare della tensione di bias per ogni diodo con una funzione del tipo:

\begin{equation}
	{\cal N}\cdot\sum_{i=0}^{n}f_{\mathrm{poisson}}\left(i;\alpha\right)\cdot f_{\mathrm{gauss}}\left(x; d + G\cdot i, \sigma_i\right)
	\label{eq:segnale_buio}
\end{equation}

Dove "${\cal N}$" indica un coefficiente di normalizzazione, "n" \`e il numero di picchi visibili nel grafico, "i" \`e un indice che scorre sul numero di picchi, "$\alpha$" \`e il parametro della poissoniana, "d" \`e la media della prima gaussiana, le "$\sigma_i$" sono le sigma dei picchi e "G" e' il guadagno, il parametro che ci interessa in questo fit.

Questa equazione deriva dal fatto che il numero di fotoni rilevati, veri o termici che siano, obbedisce alla probabilit\`a poissoniana, in quanto essi hanno una probabilit\`a costante di essere visti, mentre il segnale generato da un singolo fotone \`e gaussiano, a causa della risoluzione finita del sistema scintillatore SiPM. L'utilizzo della poissoniana in realtà è solamente un'approssimazione: si sta considerando che l'unico evento che porta ad un segnale multiplo sia la generazione di due fotoni indipendenti: questa naturalmente è un'approssimazione che viene, però, giustificata dal fatto che la funzione utilizzata fitta abbastanza bene nella maggior parte dei casi. Questo vuol dire in pratica che il fit è dato da una somma di gaussiane (una per numero di picchi visibili nel grafico) riscalate con un'ampiezza data dal calcolo di una funzione poissoniana, tale per cui all'aumentare dei fotoni la probabilità di avere un conteggio
diminuisce. Facendo il fit in questo modo si tiene conto non solo del fatto che si hanno diverse gaussiane, ma si utilizza anche la distanza tra le gaussiane e  l'ampiezza
relativa tra le gaussiane, ottenendo una stima per il guadagno migliore rispetto a quella che si otterrebbe, per esempio, misurando semplicemente la distanza tra i picchi delle gaussiane.\\

Mettendo assieme tutti i grafici per ogni diodo si ottengono delle rette che descrivono il variare dell'amplificazione (cioè in pratica del voltaggio per fotone) al variare
della tensione di bias. Un esempio di questi grafici si può vedere in questa sezione (non si riportano tutti per fluidità di lettura, possono essere visti nelle appendici), e nella Tabella \ref{tab:6lemo_gain} si possono vedere riassunti i risultati per i due diodi dello scintillatore D (gli altri si trovano nelle appendici). Si noti che non si hanno dei valori per ogni tensione
di bias: questo avviene perché ad 	alcune tensioni il diodo non era ancora in breakdown e quindi non c'è stata la cascata che porta alla nascita del segnale in output anche dopo diversi minuti di presa dati; oppure gi\`a il secondo picco causa un segnale così ampio da saturare l'oscilloscopio (che è stato settato in modo che questo evento succeda raramente), impedendoci di misurare la distanza tra due picchi.
\inputgraph{6lemo_gain}
\inputtab{6lemo_gain}

Si riportano per completezza nella Tabella \ref{tab:breakdown_gain} i dati relativi a tutti i diodi studiati.
\inputtab{breakdown_gain}


\subsection{Stima dell'efficienza dell'apparato}
\label{sec:efficiency}
\subsubsection{Efficienza sperimentale}
\`E stata fatta una seconda serie di misure per poter discutere dell'efficienza del sistema di acquisizione. In queste misure il rivelatore D \`e stato posto all'interno del solenoide, insieme a quelli gi\`a analizzati dal gruppo dell'anno precedente. Collegando i tre rivelatori precedenti al generatore di coincidenze, si sono fatte misure del segnale rilevato dallo scintillatore in esame in corrispondenza del passaggio di un muone reale, indicato dalla presenza del segnale in tutti e tre gli altri (si è impostata la soglia del trigger a 3.5 fotoni equivalenti, contati a partire dalla calibrazione del guadagno nella sezione precedente, in modo da escludere parte dei "fotoni termici"). Dato l'elevato numero di rivelatori in coincidenza (4: ogni rivelatore quando l'esperimento sarà terminato avrà due diodi, nel setup che è stato fatto l'anno scorso ci sono due rivelatori con un solo diodo funzionante e un rivelatore con entrambi, posto tra i due precedenti), ci si aspetta che il numero di coincidenze casuali sia molto piccolo\footnote{nella sezione successiva si vedr\`a come triggerando su 4 diodi il rate di trigger non dipenda dalla lunghezza della finestra di coincidenza, il che indica che pochi trigger sono effettivamente casuali}.

Nei grafici di Figura \ref{gr:eff_simple} si possono vedere i conteggi effettuati dallo scintillatore sotto studio ogni volta che il sistema degli altri 3 scintillatori ha triggerato.

\inputgraph{eff_simple}
%!!!!!!!!!!!!GRAFICO LANDAU!!!!!!!!!!!!!!!!!!!!!!!!!!!

In questo grafico \`e stato anche fatto un fit con una funzione di Landau, in modo da ottenere i parametri del segnale lasciato da una MIP (minimum ionization particle), che con la nostra configurazione di rivelatori lascia con probabilit\`a massima un segnale di \SI{101.6 +- 0.8}{\mV} (è il parametro fittato della Landau, corrispondente circa a 7 fotoni equivalenti). \`E stata utilizzata la funzione di Landau in quanto essa è la funzione che descrive i processi di ionizzazione come quello che ci permette di rivelare la particella nel caso essa non si fermi all'interno del rivelatore:

\begin{equation}
	\rho\left(x\right) = \frac{1}{\pi}\int_0^\infty e^{-t\log(t)-xt}\sin(\pi t) \dd t
	\label{eq:landau}
\end{equation}

\inputtab{landau_mpv}

Come si pu\`o vedere dalla Tabella~\ref{tab:landau_mpv}, lo scintillatore C ha evidenti problemi di generazione in quanto la probabilit\`a massima di produzione di fotoni \`e la met\`a degli altri scintillatori.

Per confronto, su uno dei due diodi si sono fatte anche misure togliendo uno alla volta i rivelatori in coincidenza. Studiando il diodo due dello scintillatore 1 e rimuovendo la richiesta che triggeri lo scintillatore subito sotto quello studiato la differenza tra le due efficienze non è molta, perché è molto piccola la quantità di falsi eventi causati da fotoni termici. Il grafico dei conteggi risulta della stessa forma rispetto a quello presentato in precedenza. Andando invece a togliere il trigger dallo scintillatore più lontano da quello studiato l'efficienza diminuisce in quanto cambia il fattore geometrico (il fattore che tiene conto degli eventi che non vengono visti non per inefficienza del rivelatore stesso ma perché la traiettoria dei muoni è tale da farli entrare in tutti i rivelatori di trigger ma non in quello studiato). Togliendo i due diodi del rivelatore centrale si vede un crollo dell'efficienza a \SI{\sim 60}{\percent}. Questo \`e perch\'e con soli due rivelatori in coincidenza c'\`e una probabilit\`a non trascurabile che due segnali termici siano avvenuti contemporaneamente, simulando un muone che non \`e mai passato per il rivelatore in esame (questi eventi non sono trascurabili in quanto, tra l'altro, permettevano di visualizzare le altre gaussiane nello studio dell'amplificazione trattato nella sezione precedente). Togliendo questo trigger, cambia anche la forma dei conteggi: infatti oltre alla Landau si può vedere anche una distribuzione poissoniana legata alla generazione casuale termica, questo grafico si può vedere nel grafico in Figura \ref{gr:eff_no@2}. Andando invece a rimuovere il trigger su due diversi scintillatori si vede un calo dell'efficienza comunque considerevole e si può sempre riconoscere la poissoniana nel grafico dei conteggi come si vede nel grafico in Figura \ref{gr:eff_no@13}.\\
\inputgraph{eff_no}
\inputtab{efficiency}

Nella Tabella \ref{tab:efficiency} si possono vedere riassunte tutte le efficienze. Si noti che l'efficienza utilizzando due diodi nello stesso scintillatore è più alta rispetto a quella che si ottiene utilizzando due diodi in due scintillatori diversi; questa è conseguenza del fatto che in quest'ultimo caso si ha un falso trigger in più che consiste nel passaggio di un muone in uno degli scintillatori e un fotone termico nell'altro scintillatore.

\subsubsection{Correzione dell'efficienza considerando le coincidenze casuali}
\label{sec:eff_corr}
Per analizzare quanto è importante la generazione di fotoni termici si sono fatte ulteriori misure cambiando la lunghezza temporale della finestra nella quale dei segnali sono considerati in coincidenza, prima con due, poi con quattro rivelatori.
Interessante è lo studio dell'efficienza richiedendo che il segnale ci sia stato su due scintillatori. 
Si tratti il problema. Sia $n_{13}$ il rate di segnali che hanno triggerato sui due rivelatori 1 e 3, $n_\mu$ il rate di muoni che attraversano sia gli scintillatori di trigger che quello studiato, chiamato s, $n_\mu^\text{OUT}$ il rate di muoni che attraversano gli scintillatori di trigger ma non quello che si sta studiando e $n_\text{rand}$ il rate di coincidenze casuali legate alla comparsa di fotoni termici. Questo ultimo è dato dalla somma di tre coefficienti: i trigger completamente termici, e i due trigger termici per metà (un muone lascia un segnale in uno dei due scintillatori e nell'altro si genera casualmente un fotone termico). 
Siano le $n$ associate ai rate e le $N$ associate al reale numero, in modo tale che $n=N/t$ con $t$ tempo di acquisizione. Si avrà la relazione:
\begin{equation}
  n_{13} = n_\mu + n_\mu^\text{OUT} + n_\text{rand}
\end{equation}
Nella formula non compare l'efficienza dei rivelatori di trigger in quanto, una volta annullato il termine casuale gli altri termini sono direttamente proporzionali ad essa, consentendoci di semplificarla. Eventuale presenza di questo termine non pu\`o essere rivelata se non come leggera diminuzione del rate di muoni rivelati.

Si rappresenti ora su un grafico $n_{13}$ al variare della finestra di acquisizione $\tau$. 
Durante la fase di presa dati, a causa di un errore nel programma per l'impostazione della coincidenza, non si è modificata la finestra di trigger per entrambi i rivelatori, ma solamente quella legata al primo rivelatore, mentre il terzo rivelatore ha mantenuto una finestra di trigger costante fissata a \SI{215}{\ns}. Per raccogliere i dati si è semplicemente acquisito impostando l'oscilloscopio in modo che si fermasse dopo 3000 eventi, e poi si è calcolato $n_{13}$ facendo $3000/t$.
 %Questo grafico si può vedere nel grafico in Figura \ref{gr:eff_width_2}, dove si nota che i dati formano una retta: per comprendere il motivo di tale forma conviene considerare che nell'equazione sopra scritta l'unico termine che dipende da $\tau$ è il termine randomico.
%Tale termine avrà un'andamento lineare in quanto una volta che in un diodo si genera un fotone termico la probabilità che il segnale diventi un segnale di trigger è uguale al prodotto tra il rate di generazione di fotoni termici (che si considera costante) per l'ampiezza dell'intervallo, perciò. 
Considerando che la finestra di trigger per uno dei due rivelatori è costante e che la probabilità che un diodo generi un fotone termico sia costante e uguale a $\rho$ si ha che la probabilità di un falso evento (causato da una coppia di fotoni termici) è:
\begin{equation}
  P(\tau)= \rho \cdot \rho \tau + \rho \cdot \rho \cdot \SI{215}{\nano\s} = \rho^2 (\tau + \SI{215}{\nano\s}) 
\end{equation}
Analogamente anche i falsi eventi causati da un fotone termico e un segnale muonico in un rivelatore hanno un andamento simile come $\tau +215 ns$,
Perciò, sommando al tempo di trigger impostato manualmente anche il tempo di trigger fisso del rivelatore $3$ si trova il grafico che si può vedere nel grafico in Figura \ref{gr:rate_full_2}, dove si nota che i dati formano una retta: per comprendere il motivo di tale forma conviene considerare che nell'equazione che descrive i conteggi possibili l'unico termine che dipende da $\tau$ è il termine randomico.
\inputgraph{rate_full_2}
Tale termine avrà un'andamento lineare, come si \`e appena descritto. 
Ovviamente, se l'ampiezza di accettazione del trigger si cambiasse su più di uno scintillatore, questo termine andrebbe come il prodotto delle ampiezze modificate (se se ne tiene uno fisso), e dal polinomio di primo grado si salirebbe verso i polinomi di grado superiore, mentre se si fosse modificata l'ampiezza di trigger per entrambi i rivelatori si sarebbe fatto a meno del termine \SI{215}{\ns}, ottenendo un termine 2.
La retta che si può vedere nel grafico in Figura \ref{gr:rate_full_2} si può interpolare, e si trovano i parametri:
\begin{equation}
  m = \SI{0.03534 \pm 0.00001}{\Hz\per\nano\s} \hspace{2cm} q = \SI{11.051 \pm 0.002}{\Hz}
\end{equation}
Particolarmente importante è il valore del coefficiente $q$: infatti, nel limite per $ 215 ns + \tau \to 0$ il termine stocastico associato alla finestra variabile come già detto scompare, e quindi si ha la somma tra il termine reale di muoni e il termine geometrico. Per quanto riguarda il termine geometrico, una stima è stata fatta utilizzando la simulazione discussa nella prossima sezione, e ha rivelato che nel caso di questo trigger il numero di eventi che entrano in tutti i rivelatori è $k = \SI{97.1 \pm 0.1}{\percent}$, perciò il numero di eventi che sono inefficienti per motivi geometrici sono il \SI{2.9 \pm 0.1}{\percent}.\\

Si studi ora il rate di segnali registrati dall'oscilloscopio: anche questo rate si potrà scrivere come somma di segnali reali (cioè segnali in cui il muone ha attraversato i rivelatori di trigger e quello in esame) e segnali falsi, cioè segnali in cui almeno uno dei fotoni di trigger è un fotone termico. Anche in questo caso si avrà quindi un rate di segnali lineare nel parametro $\tau$, che si può vedere nel grafico in Figura \ref{gr:rate_full_2_seen}, dove è stata fatta un'altra interpolazione lineare che ha trovato i parametri:
\begin{equation}
  m' = \SI{0.0050 \pm 0.0008}{\Hz\per\ns} \hspace{2cm} q' = \SI{10.9 \pm 0.4}{\Hz}
\end{equation}
Si noti che si sta trascurando la generazione di fotoni termici da parte dello scintillatore analizzato (cioè i falsi positivi); per risolvere questo problema si è deciso di mettere una soglia di accettazione di 40 mV (cioè superiore a due fotoni) in modo da tagliare gran parte della poissoniana legata ai fotoni termici. Ovviamente questa operazione va ad introdurre un'incertezza sistematica difficilmente stimabile per l'efficienza del rivelatore.\\
\inputgraph{rate_full_2_seen}

A questo punto si può andare a trovare il rate di muoni che attraversano il sistema di acquisizione facendo
\begin{equation}
  n_\mu = q \cdot k = \SI{10.73 \pm 0.01}{\Hz}
\end{equation}

A questo punto si potrà definire l'efficienza come il limite per $\tau + \SI{215}{\nano\s} \to 0$ del rapporto
\begin{equation}
	\varepsilon = \frac{q'}{n_\mu}
	\label{eq:eff}
\end{equation}
Che permetterà di trovare l'effettiva efficienza del singolo scintillatore studiato. Andando a sostituire tutti i parametri e propagando gli errori si trova, con questa tecnica, un'efficienza di:
\begin{equation}
  \varepsilon = \SI{101 \pm 4}{\percent}
\end{equation}
che è superiore al \SI{100}{\percent} ma comunque compatibile con un'efficienza molto alta, a dimostrare che effettivamente se un muone attraversa per sufficiente spazio lo scintillatore, l'oscilloscopio registra un segnale abbastanza grande nella maggior parte dei casi. Si ricorda che il fatto che l'efficienza superi il \SI{100}{\percent} oltre all'essere dovuto all'incertezza statistica \`e legato anche all'incertezza sistematica prima discussa (e non stimata).\\

Una misura simile è stata fatta andando a triggerare sui 3 scintillatori disponibili dall'anno scorso (quindi sui 3 diodi funzionanti), ottenendo i grafici nel grafico in Figura \ref{gr:rate_4}, che rivelano come, in coincidenza di 4 diodi diversi, già il trigger sia praticamente solo reale e quasi mai termico (non si hanno abbastanza dati e la curva piega di così poco che risulta impossibile andare a fare una costruzione come quella precedente con interpolazione di polinomio, si preferisce fare una media pesata). Questa interpolazione, corretta per fattore geometrico, permette di trovare un rate di:
\begin{equation}
  n_\mu = q \cdot k = \SI{8.62 \pm 0.05}{\Hz}
\end{equation}
\inputgraph{rate_4}
Questo rate è visibilmente più basso del precedente: ciò avviene a causa di errori sistematici: probabilmente quando si taglia la poissoniana si continuano a mantenere eventi "termici" nel caso a trigger su due scintillatori, oppure lo scintillatore numero 2 è leggermente spostato rispetto all'1 e al 3, provocando una diminuzione del rate di eventi.
Inoltre si evince come l'efficienza è di:
\begin{equation}
  \varepsilon = \SI{100.2 \pm 0.6}{\percent}
\end{equation}
perfettamente in linea con quella trovata con il metodo precedente (il valore risulta superiore a \SI{100}{\percent} perché non si è tagliata la poissoniana in quanto decisamente molto più bassa rispetto al caso precedente, ma comunque non nulla, si \`e ottenuto un grafico molto simile a quelli di grafico in Figura \ref{gr:eff_simple}).

Dati i problemi tecnici citati in precedenza (tra cui ulteriormente la dissaldatura e riparazione del contatto del cavo di output del SiPM 2 dello scintillatore D, lo scintillatore C con un lemo del diodo C1 tenuto insieme dal nastro isolante, viti spanate, a cui si \`e sommata la necessit\`a di usare in fase di misura al posto dello scintillatore D, ampiamente caratterizzato, lo scintillatore C ereditato dall'anno precedente con due diodi funzionanti) si \`e deciso di procedere a una fase di ricalibrazione dei guadagni, efficienze e soglie di trigger imposte.

Questo \`e stato reso necessario dall'essersi accorti che i due diodi dello scintilaltore C presentavano un rate termico molto elevato. Si \`e pertanto deciso di diminuire il voltaggio di polarizzazione applicato a tale scintillatore per diminuire il rate dei falsi eventi.

La procedura seguita \`e stata quella esposta in precedenza per tutti i passaggi, tranne che nel calcolo dell'efficienza in quanto dai dati acquisiti in passato si era  potuto stabilire come la configurazione a "sandwich" (consistente nel posizionare lo scintillatore in esame in mezzo a quelli di trigger) fosse gi\`a in grado di abbattere efficacemente il problema legato alla geometria dell'apparato, rendendo superflua la simulazione del fattore geometrico.

\inputtab{eff_new}

Nella tabella \ref{tab:eff_new} si vedono le efficienze di acquisizione degli scintillatori. Tali efficienze sono nettamente inferiore a quelle mostrate in precedenza in quanto queste si riferiscono ad entrambi i diodi dello scintillatore in AND ed inoltre il taglio \`e stato eseguito a 3.5 fotoni, analogamente a quanto fatto nelle misure vere e proprie.

\inputtab{diode_guide}

\subsection{Studio di Arietta}
La scheda di acquisizione ``Arietta'' è una scheda dal funzionamento relativamente semplice: essa è in grado di misurare la distanza temporale tra due segnali sufficientemente grandi da superare la sua soglia.

Essa va collegata ad un computer tramite presa ethernet e configurata da esso, dopodiché è necessario collegare tramite un cavo \textit{lemo} l'uscita del sistema di acquisizione a tale schedina con processore.

Arietta salva le differenze temporali tra i due segnali solamente quando tali differenze sono in un certo range di funzionamento (che va da qualche nanosecondo a circa una decina di microsecondi, per dei valori più precisi si fa riferimento alla calibrazione fatta poco più avanti), come \textit{tempi di clock} del processore interno, il che permette di misurare effettivamente le differenze temporali. 
\`E però necessario eseguire una calibrazione per capire quali tempi corrispondono ai tempi di clock. 
Inoltre, Arietta salva i dati su un buffer interno che va poi letto e svuotato manualmente utilizzando il computer che comanda la schedina stessa, e il buffer stesso ha una capienza di 4095 dati: questo non è stato un problema dal punto di vista sperimentale in quanto il rate di muoni (che si è studiato più avanti) è sufficientemente basso da permettere una facile lettura di Arietta senza perdere dati a causa dei tempi morti legati ai software di lettura stessi.\\

\subsubsection{Calibrazione}
Per effettuare la calibrazione si è utilizzato l'oscilloscopio come generatore di funzioni, e si è generata una funzione periodica data da due onde quadre relativamente vicine seguite da un tempo in cui la tensione era zero. 
Tali onde sono state impostate in maniera tale che esse fossero di lunghezza simile al segnale logico che poi è stato utilizzato come reale input di Arietta (proveniente dal discriminatore, di una lunghezza di circa \SI{200}{\ns} e di un'altezza di circa \SI{2.0}{\V}), come si può vedere nell'Immagine~\ref{img:calibration}.\\
\inputimg{calibration}

Una volta ottenuta l'onda voluta, si è collegata l'uscita del generatore di funzioni direttamente all'oscilloscopio in modo da fare una misura della distanza reale tra le due onde in termini di tempo, dopodiché si è cambiata la distanza tra i due impulsi di forma quadrata per poter andare a studiare il numero di tempi di clock associato a diversi tempi.

Una prima parte della calibrazione è stata fatta andando ad indagare a tempi distribuiti in tutta la zona in cui Arietta riesce a registrare: sono state prese 4095 misure per ogni tempo e si è fatta una media dei clock cycles ottenuti in modo da poter trovare la funzione di calibrazione
$$y = mx +q$$
con x clock cycles e y tempi in ns, in modo che poi fosse possibile fare una calibrazione del grafico finale ottenuto con i muoni.\\

Particolare attenzione si è fatta con la linearità di Arietta: si è voluto studiare quanto Arietta fosse lineare, soprattutto a bassi tempi e, quindi, per pochi cicli di clock. Per fare ciò come prima cosa si sono raccolti dei dati in sede sperimentale con la distanza tra i due segnali molto piccola, poi si è passati alla vera e propria analisi dati.\\

In pratica sono stati raccolti due campioni differenti: un campione esteso per tempi diversi che vanno da \SI{1.8}{\micro\s} a \SI{10.7}{\micro\s}, che chiameremo $\mathcal{C}_1$, e un campione esteso solamente a tempi molto bassi tra \SI{215}{\nano\s} e \SI{1.95}{\micro\s}, che chiameremo $\mathcal{C}_2$, poi il campione $\mathcal{C}_3$ è dato dall'unione dei due campioni. 
Un ulteriore campione $\mathcal{C}_f$ è presente in tabella, e la sua descrizione verrà data più avanti. 
Non si è indagato al di sotto dei \SI{215}{\micro\s} in quanto allo step successivo che è stato possibile fare con il generatore di funzioni (che consisteva in \SI{212}{\ns}) Arietta non ha raccolto alcun dato. 
Le analisi sui tre campioni sono state fatte allo stesso modo, in modo da identificare eventuali problemi di Arietta: come prima cosa si è fatta un'interpolazione lineare dei dati per ottenere i coefficienti, poi si è fatta un'interpolazione quadratica per ottenere il coefficiente di ordine superiore e verificare quanto esso sia compatibile con zero, infine si è fatto un F-test per andare a studiare la necessità sull'introduzione di tale coefficiente del secondo ordine.\\

Le grandezze calcolate si possono leggere nella Tabella~\ref{tab:linearity} dove sono state fatte le interpolazioni per i tre campioni e se ne presentano coefficienti per l'interpolazione lineare ($m$ e $q$) e quelli per l'interpolazione quadratica ($m'$, $q'$, $\varepsilon$).\\
\inputtab{linearity}

Nella Tabella~\ref{tab:Ftest} si riportano invece gli esiti dell'F-test, in particolare il numero di gradi di libertà di numeratore e denominatore ($N_N$ e $N_D$), l'esito dell'F-test ($F$) ed il paramero della distribuzione di Fischer tale per cui la probabilità sia uguale al \SI{95}{\percent} ($\bar{F}$). 
Questo  significa che se il valore di $F$ è superiore al valore di $\bar{F}$ allora si potrà rigettare con una confidence level del \SI{95}{\percent} l'ipotesi che i due modelli siano equivalenti, e  sarà quindi necessaria l'introduzione del termine di secondo grado per descrivere il campione.\\
\inputtab{Ftest}

Da queste tabelle si evince come, effettivamente, ci sia un problema di non linearità a bassi tempi. 
Infatti, dalla stima dei coefficienti di secondo ordine, si vede come il $\mathcal{C}_1$, che si ferma a \SI{1.79}{\micro\s} abbia un coefficiente molto compatibile con lo zero mentre gli altri due campioni, che arrivano fino a \SI{215}{\micro\s} hanno un coefficiente scarsamente compatibile con zero.\\

Guardando il risultato dell'F-test si vede che nel caso del primo campione non è assolutamente possibile rigettare l'ipotesi di linearità, mentre il secondo campione in particolare rigetterebbe la linearità di Arietta anche con un livello di confidenza pari al \SI{99.4}{\percent}, decisamente alto.\\


Per scorrevolezza di lettura non si trovano qui tutte le interpolazioni fatte, che si possono trovare nelle appendici, ma si riportano esclusivamente le interpolazioni fatte su $\mathcal{C}_2$  che rivelano la non linearità di Arietta a bassi tempi, che si possono vedere nell'Immagine~\ref{gr:tempi_bassi}.\\
\inputgraph{tempi_bassi}

Per ovviare al problema della non linearità si è capito (guardando i grafici) che le non linearità sorgono nel momento in cui i tempi diventano inferiori ai \SI{335}{\ns}, quindi si è deciso di ripetere le interpolazioni fatte per i campioni 1, 2 e 3 utilizzando solamente i dati sopra tale soglia, che formano il campione $\mathcal{C}_f$ le cui proprietà si possono vedere nella Tabella~\ref{tab:linearity}. 
Nell'immagine \ref{gr:final_calibration} si possono vedere le interpolazioni effettuate su tale campione, che effettivamente ha passato bene l'F-test.\\
\inputgraph{final_calibration}

Dato questo problema che si è trovato con la linearità di Arietta, in tutte le misure effettuate utilizzando tale scheda di acquisizione sono state tagliate le misura per tempi inferiori a \SI{335}{\ns} in quanto affette da un errore sistematico che non si è in grado di correggere.\\

Le misure finali della calibrazione indicano che i coefficienti di calibrazione sono:
\begin{equation}
  m = \SI{14.994 \pm 0.007}{\nano\s\per cycles} \hspace{2cm} q = \SI{6 \pm 2}{\nano\s}
\end{equation}


\subsubsection{Efficienza}
I dati che hanno permesso di studiare la linearità di Arietta sono stati utilizzati anche per stimarne l'efficienza: infatti i 4095 dati raccolti a differenze temporali ben definite provenivano da  4095 coppie di segnali e non di più: se la scheda di acquisizione avesse misurato meno dati di quelli inviati ad una certa differenza temporale allora ci sarebbe stato un problema di efficienza. 
Invece, per quanto riguarda Arietta, l'inefficienza sopra i \SI{212}{\ns} è ampiamente trascurabile, in quanto sono sempre stati letti tutti i segnali inviati. In particolare, dato il  numero di  dati utilizzati, l'inefficienza risulta al più dell'ordine del percento.


\subsection{Studio dei muoni}
Una stima sulla vita media dei muoni è stata effettuata lasciando per più tempo possibile in sistema in acquisizione. 
Si è settato il trigger in maniera tale che venisse inviato un segnale alla scheda di accquisizione di tempi Arietta quando davano segnali sopra la soglia  impostata (di tre fotoni) tutti e quattro i diodi posti sui due scintillatori sopra l'assorbitore, e come veto si è impostato in OR il segnale dei due diodi presenti nello scintillatore posto sotto l'assorbitore, con soglia messa sempre a tre fotoni. 
La presa dati si è svolta in 17 giorni non continuativi: prima dall'11 al 19 gennaio e successivamente dal 24 al 31 gennaio. 
Il motivo dell'interruzione è stato di natura tecnica, in quanto è stata staccata l'alimentazione a nostra insaputa nei laboratori e quindi è stato necessario riavviare il sistema di acquisizione. Si sono raccolti in totale 14634 segnali.

\subsubsection{Vita media}
Prima di poter calcolare la vita  media, è opportuno ricordare i problemi che si possono incontrare in tale studio. 
Come prima cosa, i dati che si sono raccolti sono dati provenienti da fondo (sempre presente),  muoni positivi e muoni negativi. 
I muoni positivi si fermano all'interno dell'assorbitore, decadono con una costante caratteristica di decadimento, ed emettono un elettrone che (quando c'è stato un  trigger del sistema) attraversa gli scintillatori. 
I muoni negativi, invece, quando si fermano all'interno dell'assorbitore, tendono ad essere catturati dal materiale stesso, emettendo un elettrone (che scalzano dall'atomo da cui sono stati catturati) con un tempo caratteristico di cattura. 
Il tempo caratteristico di cattura è molto inferiore al tempo caratteristico di decadimento, quindi ci si aspetta di vedere nel grafico una coppia di esponenziali: uno legato al decadimento ed uno legato alla cattura. 
I tempi di attraversamento dell'apparato da parte dei leptoni sono considerati trascurabili.\\

Però è importante tenere conto delle inefficienze del sistema di acquisizione, che si possono trovare a diversi livelli. 
Ci si concentri in particolare sulle inefficienze che possono andare ad inficiare le misure di vita media. 
Una prima inefficienza è legata all'inefficienza della coppia discriminatore-Arietta: infatti dato che il discriminatore ha come uscita segnali logici dall'ampiezza di \SI{200}{\ns}, Arietta non è in grado di misurare tempi inferiori ai \SI{215}{\ns}, come è stato verificato e discusso nella fase di descrizione della scheda di acquisizione stessa. 
Questo si traduce come un'assenza di dati sperimentali al di sotto di tale tempo, che è stata confermata dall'acquisizione dati effettuata. 
Un altro problema è nella non linearità di Arietta: non si può utilizzare la calibrazione stimata per tempi inferiori ai \SI{335}{\ns}, quindi è necessario tagliare i dati al di sotto di tale tempo, una volta effettuata la calibrazione. 
Un'altra inefficienza sperimentale si può avere nella catena scintillatore-discriminatore: infatti all'interno di tali strumenti sono presenti parecchie componenti elettroniche che hanno un tempo caratteristico di funzionamento e potrebbero non funzionare se il segnale è più rapido di una certa soglia. 
Per poter tener conto di questa evenienza è necessario andare a guardare l'andamento vero e proprio dei dati raccolti, in  modo da poterli discutere e interpretare.\\
\inputgraph{all_data_log}

I dati raccolti, e tagliati secondo il limite imposto dalla non linearità di Arietta, si possono vedere nel Grafico~\ref{gr:all_data}, dove sono stati semplicemente presentati i dati una volta effettuata la calibrazione del sistema di acquisizione in scala logaritmica. 
Effettivamente da tale grafico si evince  come ci sia un problema legato all'efficienza dell'elettronica: a tempi bassi ci si aspetta di vedere per lo meno la coda dell'esponenziale di cattura dei muoni negativi, che ha un tempo caratteristico molto inferiore rispetto a quello di decadimento dei muoni positivi, ed invece si vede un andamento tale per cui i dati sperimentali al posto che aumentare diminuiscono, a rappresentare un basso numero di eventi a bassi tempi che non coincide con l'evento fisico che ci si aspetta di vedere. 
Si interpreta tale fenomeno come una conseguenza di inefficienze di tutta la catena di acquisizione precedente alla scheda Arietta.\\

Per risolvere questo problema si è semplicemente tagliato il grafico a \SI{1000}{\ns}, oltre il quale non si vede più inefficienza, non avendo a disposizione metodi migliori per tenere conto di questo problema sistematico.\\

Come consegueza di questo taglio non è stato possibile andare ad interpolare la costante di cattura dei muoni negativi, in quanto completamente inghiottita dalle inefficienze del sistema di acquisizione. 
Quindi, si è deciso di interpolare i dati sperimentali con un semplice esponenziale sovrapposto ad un fondo costante. La funzione di interpolazione che si è utilizzata ha la forma:
\begin{equation}
  y = \frac{N_\text{bkg} \cdot bw}{t_\textit{max}-t_\textit{min}}+\frac{N_{\mu+}bw}{\tau \left(e^{-\frac{t_\textit{min}}{\tau}} - e^{-\frac{t_\textit{max}}{\tau}}\right)}
  e^{-\frac{t}{\tau}} 
  \label{eq:fit_mean_life}
\end{equation}
I parametri di fit di questa funzione sono i tre parametri $N_\text{bkg}$, $N_{\mu+}$ e $\tau$. 
Inoltre $bw$ indica la \textit{bin width}, cioè la larghezza dei bin utilizzati per creare l'istogramma, misurata naturalmente in ns, ed il termine davanti all'esponenziale vero e proprio tiene conto della normalizzazione. 
Questo termine è necessario perché $N_\mu$ dia effettivamente una stima dei muoni positivi che hanno attraversato il sistema di acquisizione nell'intervallo di tempo scelto. Questa funzione è stata fittata sui dati sperimentali tagliati come precedentemente descritto, ottenendo il Grafico \ref{gr:all_data}.\\
\inputgraph{all_data}

Da quanto si vede dal grafico, il valore sul tempo di decadimento dei muoni è molto compatibile con il valore noto in  letteratura, infatti si ha un tempo di decadimento di:
\begin{equation}
  \tau = \SI{2.21+-0.03}{\micro\second}
\end{equation}
da confrontare con il valore in letteratura~\cite{bib:Patrignani:2016xqp} di \SI{2.1969811+-0.0000022}{\micro\second}.\\

A questo dato, a cui viene associato l'errore statistico, andrebbe associata anche una stima delle incertezze sistematiche. 
In particolare, si è analizzata l'incertezza sperimentale legata alla calibrazione che potrebbe non essere corretta (infatti naturalmente i parametri di calibrazione utilizzati sono associate ad un'incertezza). 
Per  stimare l'errore sistematico legato a tale fenomeno si è deciso di fare un'ulteriore interpolazione, simile a quella appena fatta, ma senza aver calibrato il  grafico. 
A questo punto i parametri di calibrazione sono stati utilizzati per trasformare la $\tau$ in \textit{clock cycles} così trovata in nanosecondi. 
L'operazione a posteriori ha permesso di propagare gli errori sull'operazione di cambio di unità di misura, e quindi avere una stima dell'incertezza legata a tale trasformazione.
Tenendo conto degli errori sui parametri di calibrazione interpolati e della loro correlazione, tale grandezza risulta di \SI{2}{\ns}, ampiamente trascurabile rispetto all'incertezza statistica dell'ordine della decina di nanosecondi.

Dal grafico inoltre si evince come il fondo sia molto piccolo (dell'ordine dell'\SI{1}{\percent}) rispetto ai dati sperimentali, e quindi non si effettuano ulteriori analisi per la sua riduzione. 
Non si rivelano inoltre, in tale grafico, fenomeni che potrebbero evidenziare errori sperimentali o inefficienze evidenti del sistema, come nel grafico precedente il taglio a \SI{1}{\us}.

\subsubsection{Rate di muoni}
I dati raccolti sono stati utilizzati anche per fare una stima del rate di dati che è stato acquisito, in modo che sia confrontabile con la simulazione effettuata e discussa più avanti, così da avere evidenza di eventuali inefficienze di cui non si è tenuto conto. Per fare una stima del rate non si sono potuti utilizzare tutti i dati analizzati precedentemente, in quanto a causa degli spegnimenti non previsti del sistema di acquisizione non si conosce il tempo in cui sono stati raccolti tali dati.\\

Quindi, si è ripetuta l'analisi precedentemente fatta solamente con i dati raccolti in giornate intere, cioè quelli raccolti in 13 giorni, ossia in \SI{1123200}{\s}. Si sono utilizzati questi dati perch\'e abbiamo assunto che se un giorno era preceduto e seguito da un altro di presa dati, l'acquisizione era continuata per l'intera giornata. 
In realtà, i dati venivano salvati ogni 5 minuti, perciò non si è in linea di principio sicuri sul tempo di acquisizione, ed andrebbe associato un errore. 
Però, considerando che le \SI{24}{\hour} di un giorno sono esattamente divisibili per i 5 minuti, nel caso la prima acquisizione del giorno non sia iniziata a mezzanotte ma qualche minuto prima, anche l'ultima del giorno sarebbe finita qualche minuto prima, perciò si è sicuri che l'incertezza sul tempo sia trascurabile rispetto alle altre incertezze in gioco.\\
\inputgraph{for_rate}

I dati solamente dei 13 giorni utilizzati si possono vedere nel grafico di Figura \ref{gr:for_rate}, dove l'interpolazione è stata fatta con la stessa funzione descritta precedentemente. Si è in particolare interessati al parametri $N_{\mu+}$, che rappresenta il numero di eventi registrati nell'intervallo di tempo analizzato. 
Data la legge di decadimento esponenziale, si può trovare il numero di eventi totali una volta noti gli eventi tra un tempo $t_\textit{min}$ e un tempo $t_\textit{max}$  tramite la formula:
\begin{equation}
  \bar{N}_{\mu+} = \frac{N_{\mu+}}{e^{\frac{t_\textit{min}}{\tau}}-e^{\frac{-t_\textit{max}}{\tau}}}
  \label{eq:tot_mup}
\end{equation}
L'utilizzo di questa formula permette di avere una stima del numero di muoni che si preveda si sarebbero visti se il sistema di acquisizione fosse perfetto e permettesse di vedere decadimenti a qualsiasi tempo. Con questa formula, si arriva ad una stima di:
\begin{equation}
  \bar{N}_{\mu+} = \num{1.63+-0.01e4}
\end{equation}
muoni positivi. 
Questo valore si vuole utilizzare per ottenere il rate di muoni incidenti sull'apparato, che si può calcolare considerando che quanto appena trovato è il numero di muoni positivi, ed il rapporto tra i muoni positivi e i muoni negativi (che non si possono vedere con l'apparato utilizzato per i problemi discussi sopra) è tra 1.25 e 1.30 a 1\footnote{Il dato, preso da~\cite{bib:Patrignani:2016xqp}, è proprio tra  1.25 e 1.30, quindi si è deciso di considerare come rapporto $1.275\pm0.025$}. 
Quindi, si ha che il numero totale di muoni che si sarebbero visti se si fosse stati in grado di vedere anche i muoni negativi, sarebbe stato:
\begin{equation}
  \bar{N}_{\mu} = \num{2.40 \pm 0.04e4}
\end{equation}
Conoscendo poi il tempo nel quale si sono raccolti questi dati è possibile andare a stimare il rate di eventi che abbiano caratteristiche (dal punto di vista dell'energia del muone, di quella dell'elettrone e delle loro direzioni) che il sistema di acquisizione sarebbe in grado di registrare se non avesse problemi dal punto di vista temporale:
\begin{equation}
  \nu_\mu = \SI{0.0214 \pm 0.0004}{\Hz}
\end{equation}

